\chapter{Internet and WWW Services}
\nodename{Internet and WWW}
\index{WWW}
\index{Internet}
\index{World-Wide Web}

The modules described in this chapter provide various services to
World-Wide Web (WWW) clients and/or services, and a few modules
related to news and email.  They are all implemented in Python.  Some
of these modules require the presence of the system-dependent module
\code{sockets}, which is currently only fully supported on Unix and
Windows NT.  Here is an overview:

\begin{description}

\item[cgi]
--- Common Gateway Interface, used to interpret forms in server-side
scripts.

\item[urllib]
--- Open an arbitrary object given by URL (requires sockets).

\item[httplib]
--- HTTP protocol client (requires sockets).

\item[ftplib]
--- FTP protocol client (requires sockets).

\item[gopherlib]
--- Gopher protocol client (requires sockets).

\item[nntplib]
--- NNTP protocol client (requires sockets).

\item[urlparse]
--- Parse a URL string into a tuple (addressing scheme identifier, network
location, path, parameters, query string, fragment identifier).

\item[sgmllib]
--- Only as much of an SGML parser as needed to parse HTML.

\item[htmllib]
--- A (slow) parser for HTML documents.

\item[formatter]
--- Generic output formatter and device interface.

\item[rfc822]
--- Parse RFC-822 style mail headers.

\item[mimetools]
--- Tools for parsing MIME style message bodies.

\item[binhex]
--- Encode and decode files in binhex4 format.

\item[uu]
--- Encode and decode files in uuencode format.

\item[binascii]
--- Tools for converting between binary and various ascii-encoded binary 
representation

\item[xdrlib]
--- The External Data Representation Standard as described in RFC 1014,
written by Sun Microsystems, Inc. June 1987.

\item[mailcap]
--- Mailcap file handling.  See RFC 1524.

\item[base64]
--- Encode/decode binary files using the MIME base64 encoding.

\item[quopri]
--- Encode/decode binary files using the MIME quoted-printable encoding.

\item[SocketServer]
--- A framework for network servers.

\item[mailbox]
--- Read various mailbox formats.

\end{description}
