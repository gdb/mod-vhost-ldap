\section{Standard Module \sectcode{rexec}}
\label{module-rexec}
\stmodindex{rexec}


This module contains the \class{RExec} class, which supports
\method{r_exec()}, \method{r_eval()}, \method{r_execfile()}, and
\method{r_import()} methods, which are restricted versions of the standard
Python functions \method{exec()}, \method{eval()}, \method{execfile()}, and
the \keyword{import} statement.
Code executed in this restricted environment will
only have access to modules and functions that are deemed safe; you
can subclass \class{RExec} to add or remove capabilities as desired.

\emph{Note:} The \class{RExec} class can prevent code from performing
unsafe operations like reading or writing disk files, or using TCP/IP
sockets.  However, it does not protect against code using extremely
large amounts of memory or CPU time.  

\begin{classdesc}{RExec}{\optional{hooks\optional{, verbose}}}
Returns an instance of the \class{RExec} class.  

\var{hooks} is an instance of the \class{RHooks} class or a subclass of it.
If it is omitted or \code{None}, the default \class{RHooks} class is
instantiated.
Whenever the \module{RExec} module searches for a module (even a
built-in one) or reads a module's code, it doesn't actually go out to
the file system itself.  Rather, it calls methods of an \class{RHooks}
instance that was passed to or created by its constructor.  (Actually,
the \class{RExec} object doesn't make these calls --- they are made by
a module loader object that's part of the \class{RExec} object.  This
allows another level of flexibility, e.g. using packages.)

By providing an alternate \class{RHooks} object, we can control the
file system accesses made to import a module, without changing the
actual algorithm that controls the order in which those accesses are
made.  For instance, we could substitute an \class{RHooks} object that
passes all filesystem requests to a file server elsewhere, via some
RPC mechanism such as ILU.  Grail's applet loader uses this to support
importing applets from a URL for a directory.

If \var{verbose} is true, additional debugging output may be sent to
standard output.
\end{classdesc}

The \class{RExec} class has the following class attributes, which are
used by the \method{__init__()} method.  Changing them on an existing
instance won't have any effect; instead, create a subclass of
\class{RExec} and assign them new values in the class definition.
Instances of the new class will then use those new values.  All these
attributes are tuples of strings.

\setindexsubitem{(RExec object attribute)}
\begin{datadesc}{nok_builtin_names}
Contains the names of built-in functions which will \emph{not} be
available to programs running in the restricted environment.  The
value for \class{RExec} is \code{('open',} \code{'reload',}
\code{'__import__')}.  (This gives the exceptions, because by far the
majority of built-in functions are harmless.  A subclass that wants to
override this variable should probably start with the value from the
base class and concatenate additional forbidden functions --- when new
dangerous built-in functions are added to Python, they will also be
added to this module.)
\end{datadesc}

\begin{datadesc}{ok_builtin_modules}
Contains the names of built-in modules which can be safely imported.
The value for \class{RExec} is \code{('audioop',} \code{'array',}
\code{'binascii',} \code{'cmath',} \code{'errno',} \code{'imageop',}
\code{'marshal',} \code{'math',} \code{'md5',} \code{'operator',}
\code{'parser',} \code{'regex',} \code{'rotor',} \code{'select',}
\code{'strop',} \code{'struct',} \code{'time')}.  A similar remark
about overriding this variable applies --- use the value from the base
class as a starting point.
\end{datadesc}

\begin{datadesc}{ok_path}
Contains the directories which will be searched when an \keyword{import}
is performed in the restricted environment.  
The value for \class{RExec} is the same as \code{sys.path} (at the time
the module is loaded) for unrestricted code.
\end{datadesc}

\begin{datadesc}{ok_posix_names}
% Should this be called ok_os_names?
Contains the names of the functions in the \module{os} module which will be
available to programs running in the restricted environment.  The
value for \class{RExec} is \code{('error',} \code{'fstat',}
\code{'listdir',} \code{'lstat',} \code{'readlink',} \code{'stat',}
\code{'times',} \code{'uname',} \code{'getpid',} \code{'getppid',}
\code{'getcwd',} \code{'getuid',} \code{'getgid',} \code{'geteuid',}
\code{'getegid')}.
\end{datadesc}

\begin{datadesc}{ok_sys_names}
Contains the names of the functions and variables in the \module{sys}
module which will be available to programs running in the restricted
environment.  The value for \class{RExec} is \code{('ps1',}
\code{'ps2',} \code{'copyright',} \code{'version',} \code{'platform',}
\code{'exit',} \code{'maxint')}.
\end{datadesc}

\class{RExec} instances support the following methods:
\setindexsubitem{(RExec object method)}

\begin{funcdesc}{r_eval}{code}
\var{code} must either be a string containing a Python expression, or
a compiled code object, which will be evaluated in the restricted
environment's \module{__main__} module.  The value of the expression or
code object will be returned.
\end{funcdesc}

\begin{funcdesc}{r_exec}{code}
\var{code} must either be a string containing one or more lines of
Python code, or a compiled code object, which will be executed in the
restricted environment's \module{__main__} module.
\end{funcdesc}

\begin{funcdesc}{r_execfile}{filename}
Execute the Python code contained in the file \var{filename} in the
restricted environment's \module{__main__} module.
\end{funcdesc}

Methods whose names begin with \samp{s_} are similar to the functions
beginning with \samp{r_}, but the code will be granted access to
restricted versions of the standard I/O streams \code{sys.stdin},
\code{sys.stderr}, and \code{sys.stdout}.

\begin{funcdesc}{s_eval}{code}
\var{code} must be a string containing a Python expression, which will
be evaluated in the restricted environment.  
\end{funcdesc}

\begin{funcdesc}{s_exec}{code}
\var{code} must be a string containing one or more lines of Python code,
which will be executed in the restricted environment.  
\end{funcdesc}

\begin{funcdesc}{s_execfile}{code}
Execute the Python code contained in the file \var{filename} in the
restricted environment.
\end{funcdesc}

\class{RExec} objects must also support various methods which will be
implicitly called by code executing in the restricted environment.
Overriding these methods in a subclass is used to change the policies
enforced by a restricted environment.

\begin{funcdesc}{r_import}{modulename\optional{, globals, locals, fromlist}}
Import the module \var{modulename}, raising an \exception{ImportError}
exception if the module is considered unsafe.
\end{funcdesc}

\begin{funcdesc}{r_open}{filename\optional{, mode\optional{, bufsize}}}
Method called when \function{open()} is called in the restricted
environment.  The arguments are identical to those of \function{open()},
and a file object (or a class instance compatible with file objects)
should be returned.  \class{RExec}'s default behaviour is allow opening
any file for reading, but forbidding any attempt to write a file.  See
the example below for an implementation of a less restrictive
\method{r_open()}.
\end{funcdesc}

\begin{funcdesc}{r_reload}{module}
Reload the module object \var{module}, re-parsing and re-initializing it.  
\end{funcdesc}

\begin{funcdesc}{r_unload}{module}
Unload the module object \var{module} (i.e., remove it from the
restricted environment's \code{sys.modules} dictionary).
\end{funcdesc}

And their equivalents with access to restricted standard I/O streams:

\begin{funcdesc}{s_import}{modulename\optional{, globals, locals, fromlist}}
Import the module \var{modulename}, raising an \exception{ImportError}
exception if the module is considered unsafe.
\end{funcdesc}

\begin{funcdesc}{s_reload}{module}
Reload the module object \var{module}, re-parsing and re-initializing it.  
\end{funcdesc}

\begin{funcdesc}{s_unload}{module}
Unload the module object \var{module}.   
% XXX what are the semantics of this?  
\end{funcdesc}

\subsection{An example}

Let us say that we want a slightly more relaxed policy than the
standard \class{RExec} class.  For example, if we're willing to allow
files in \file{/tmp} to be written, we can subclass the \class{RExec}
class:

\begin{verbatim}
class TmpWriterRExec(rexec.RExec):
    def r_open(self, file, mode='r', buf=-1):
        if mode in ('r', 'rb'):
            pass
        elif mode in ('w', 'wb', 'a', 'ab'):
            # check filename : must begin with /tmp/
            if file[:5]!='/tmp/': 
                raise IOError, "can't write outside /tmp"
            elif (string.find(file, '/../') >= 0 or
                 file[:3] == '../' or file[-3:] == '/..'):
                raise IOError, "'..' in filename forbidden"
        else: raise IOError, "Illegal open() mode"
        return open(file, mode, buf)
\end{verbatim}
%
Notice that the above code will occasionally forbid a perfectly valid
filename; for example, code in the restricted environment won't be
able to open a file called \file{/tmp/foo/../bar}.  To fix this, the
\method{r_open()} method would have to simplify the filename to
\file{/tmp/bar}, which would require splitting apart the filename and
performing various operations on it.  In cases where security is at
stake, it may be preferable to write simple code which is sometimes
overly restrictive, instead of more general code that is also more
complex and may harbor a subtle security hole.
