\section{Standard Module \sectcode{rexec}}
\stmodindex{rexec}
\renewcommand{\indexsubitem}{(in module rexec)}

This module contains the \code{RExec} class, which supports
\code{r_exec()}, \code{r_eval()}, \code{r_execfile()}, and
\code{r_import()} methods, which are restricted versions of the standard
Python functions \code{exec()}, \code{eval()}, \code{execfile()}, and
\code{import()}.  Code executed in this restricted environment will
only have access to modules and functions that are deemed safe; you
can subclass \code{RExec} to add or remove capabilities as desired.

\emph{Note:} The \code{RExec} class can prevent code from performing
unsafe operations like reading or writing disk files, or using TCP/IP
sockets.  However, it does not protect against code using extremely
large amounts of memory or CPU time.  
% XXX is there any protection against this?

\begin{funcdesc}{RExec}{\optional{hooks\, verbose} }
Returns an instance of the \code{RExec} class.  

% XXX is ihooks.py documented?  If yes, there should be a ref here

\var{hooks} is an instance of the \code{RHooks} class or a subclass of it.
Whenever the RExec module searches for a module (even a built-in one)
or reads a module's code, it doesn't actually go out to the file
system itself.  Rather, it calls methods of an RHooks instance that
was passed to or created by its constructor.  (Actually, the RExec
object doesn't make these calls---they are made by a module loader
object that's part of the RExec object.  This allows another level of
flexibility, e.g. using packages.)

By providing an alternate RHooks object, we can control the actual
file system accesses made to import a module, without changing the
actual algorithm that controls the order in which those accesses are
made.  For instance, we could substitute an RHooks object that passes
all filesystem requests to a file server elsewhere, via some RPC
mechanism such as ILU.  Grail's applet loader uses this to support
importing applets from a URL for a directory.

% XXX does verbose actually do anything at the moment?
If \var{verbose} is true, additional debugging output will be sent to
standard output.
\end{funcdesc}

RExec instances have the following attributes, which are used by the
\code{__init__} method.  Changing them on an existing instance won't
have any effect; instead, create a subclass of \code{RExec} and assign
them new values in the class definition.  Instances of the new class
will then use those new values.  All these attributes are tuples of
strings.

\renewcommand{\indexsubitem}{(RExec object attribute)}
\begin{datadesc}{nok_builtin_names}
Contains the names of built-in functions which will \emph{not} be
 available to programs running in the restricted environment.  The
 value for \code{RExec} is \code{('open',} \code{reload',}
 \code{__import__')}.
\end{datadesc}

\begin{datadesc}{ok_builtin_modules}
Contains the names of built-in modules which can be safely imported.
The value for \code{RExec} is \code{('array',} \code{'binascii',} \code{'audioop',}
\code{'imageop',} \code{'marshal',} \code{'math',} \code{'md5',} \code{'parser',} \code{'regex',} \code{'rotor',}
\code{'select',} \code{'strop',} \code{'struct',} \code{'time')}.
\end{datadesc}

\begin{datadesc}{ok_path}
Contains the directories which will be searched when an \code{import}
is performed in the restricted environment.  
The value for \code{RExec} is the same as \code{sys.path} for
unrestricted code.
\end{datadesc}

\begin{datadesc}{ok_posix_names}
% Should this be called ok_os_names?
Contains the names of the functions in the \code{os} module which will be
available to programs running in the restricted environment.  The
value for \code{RExec} is \code{('error',} \code{'fstat',}
\code{'listdir',} \code{'lstat',} \code{'readlink',} \code{'stat',}
\code{'times',} \code{'uname',} \code{'getpid',} \code{'getppid',}
\code{'getcwd',} \code{'getuid',} \code{'getgid',} \code{'geteuid',}
\code{'getegid')}.
\end{datadesc}

\begin{datadesc}{ok_sys_names}
Contains the names of the functions and variables in the \code{sys} module which will be
available to programs running in the restricted environment.  The
value for \code{RExec} is \code{('ps1',} \code{'ps2',}
\code{'copyright',} \code{'version',} \code{'platform',} \code{'exit',}
\code{'maxint')}.
\end{datadesc}

RExec instances support the following methods:
\renewcommand{\indexsubitem}{(RExec object method)}

\begin{funcdesc}{r_eval}{code}
\var{code} must either be a string containing a Python expression, or a compiled code object, which will
be evaluated in the restricted environment.  The value of the expression or code object will be returned.
\end{funcdesc}

\begin{funcdesc}{r_exec}{code}
\var{code} must either be a string containing one or more lines of Python code,  or a compiled code object,
which will be executed in the restricted environment.  
\end{funcdesc}

\begin{funcdesc}{r_execfile}{filename}
Execute the Python code contained in the file \var{filename} in the
restricted environment.
\end{funcdesc}

Methods whose names begin with \code{s_} are similar to the functions
beginning with \code{r_}, but the code will be granted access to
restricted versions of \code{sys.stdin}, \code{sys.stderr}, and
\code{sys.stdout}.  

\begin{funcdesc}{s_eval}{code}
\var{code} must be a string containing a Python expression, which will
be evaluated in the restricted environment.  
\end{funcdesc}

\begin{funcdesc}{s_exec}{code}
\var{code} must be a string containing one or more lines of Python code,
which will be executed in the restricted environment.  
\end{funcdesc}

\begin{funcdesc}{s_execfile}{code}
Execute the Python code contained in the file \var{filename} in the
restricted environment.
\end{funcdesc}

\code{RExec} objects must also support various methods which will be implicitly called 
by code executing in the restricted environment.  Overriding these
methods in a subclass is used to change the policies enforced by a restricted environment.

\begin{funcdesc}{r_import}{modulename\optional{\, globals, locals, fromlist}}
Import the module \var{modulename}, raising an \code{ImportError} exception
if the module is considered unsafe.  
\end{funcdesc}

\begin{funcdesc}{r_open}{filename\optional{\, mode\optional{\, bufsize}}}
Method called when \code{open()} is called in the restricted
environment.  The arguments are identical to those of \code{open()},
and a file object (or a class instance compatible with file objects)
should be returned.  \code{RExec}'s default behaviour is allow opening
any file for reading, but forbidding any attempt to write a file.  See
the example below for an implementation of a less restrictive \code{r_open()}.
\end{funcdesc}

\begin{funcdesc}{r_reload}{module}
Reload the module object \var{module}, re-parsing and re-initializing it.  
\end{funcdesc}

\begin{funcdesc}{r_unload}{module}
Unload the module object \var{module}.   
% XXX what are the semantics of this?  
\end{funcdesc}

\begin{funcdesc}{s_import}{modulename\optional{\, globals, locals, fromlist}}
Import the module \var{modulename}, raising an \code{ImportError} exception
if the module is considered unsafe.  
\end{funcdesc}

\begin{funcdesc}{s_reload}{module}
Reload the module object \var{module}, re-parsing and re-initializing it.  
\end{funcdesc}

\begin{funcdesc}{s_unload}{module}
Unload the module object \var{module}.   
% XXX what are the semantics of this?  
\end{funcdesc}

\subsection{An example}

Let us say that we want a slightly more relaxed policy than the
standard RExec class.  For example, if we're willing to allow files in
\file{/tmp} to be written, we can subclass the \code{RExec} class:

\bcode\begin{verbatim}
class TmpWriterRExec(rexec.RExec):
    def r_open(self, file, mode='r', buf=-1):
        if mode in ('r', 'rb'): pass 
	elif mode in ('w', 'wb'):
	    # check filename : must begin with /tmp/
	    if file[0:5]!='/tmp/': 
		raise IOError, "can't open files for writing outside of /tmp"
	    elif string.find(file, '/../')!=-1:
		raise IOError, "'..' in filename; open for writing forbidden"
        return open(file, mode, buf)
\end{verbatim}\ecode

Notice that the above code will occasionally forbid a perfectly valid
filename; for example, code in the restricted environment won't be
able to open a file called \file{/tmp/foo/../bar}.  To fix this, the
\code{r_open} method would have to simplify the filename to
\file{/tmp/bar}, which would require splitting apart the filename and
performing various operations on it.  In cases where security is at
stake, it may be preferable to write simple code which is sometimes
overly restrictive, instead of more general code that is also more
complex and may harbor a subtle security hole.
