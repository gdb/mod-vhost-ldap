% Format this file with latex.

\documentstyle[myformat]{report}

\title{\bf
	Python Reference Manual \\
	{\em Incomplete Draft}
}
	
\author{
	Guido van Rossum \\
	Dept. CST, CWI, Kruislaan 413 \\
	1098 SJ Amsterdam, The Netherlands \\
	E-mail: {\tt guido@cwi.nl}
}

\begin{document}

\pagenumbering{roman}

\maketitle

\begin{abstract}

\noindent
Python is a simple, yet powerful programming language that bridges the
gap between C and shell programming, and is thus ideally suited for
``throw-away programming''
and rapid prototyping.  Its syntax is put
together from constructs borrowed from a variety of other languages;
most prominent are influences from ABC, C, Modula-3 and Icon.

The Python interpreter is easily extended with new functions and data
types implemented in C.  Python is also suitable as an extension
language for highly customizable C applications such as editors or
window managers.

Python is available for various operating systems, amongst which
several flavors of {\UNIX}, Amoeba, the Apple Macintosh O.S.,
and MS-DOS.

This reference manual describes the syntax and ``core semantics'' of
the language.  It is terse, but exact and complete.  The semantics of
non-essential built-in object types and of the built-in functions and
modules are described in the {\em Python Library Reference}.  For an
informal introduction to the language, see the {\em Python Tutorial}.

\end{abstract}

\pagebreak

{
\parskip = 0mm
\tableofcontents
}

\pagebreak

\pagenumbering{arabic}

\chapter{Introduction}

This reference manual describes the Python programming language.
It is not intended as a tutorial.

While I am trying to be as precise as possible, I chose to use English
rather than formal specifications for everything except syntax and
lexical analysis.  This should make the document better understandable
to the average reader, but will leave room for ambiguities.
Consequently, if you were coming from Mars and tried to re-implement
Python from this document alone, you might have to guess things and in
fact you would be implementing quite a different language.
On the other hand, if you are using
Python and wonder what the precise rules about a particular area of
the language are, you should definitely be able to find it here.

It is dangerous to add too many implementation details to a language
reference document -- the implementation may change, and other
implementations of the same language may work differently.  On the
other hand, there is currently only one Python implementation, and
its particular quirks are sometimes worth being mentioned, especially
where the implementation imposes additional limitations.

Every Python implementation comes with a number of built-in and
standard modules.  These are not documented here, but in the separate
{\em Python Library Reference} document.  A few built-in modules are
mentioned when they interact in a significant way with the language
definition.

\section{Warning}

This version of the manual is incomplete.  Sections that still need to
be written or need considerable work are marked with ``XXX''.

\section{Notation}

The descriptions of lexical analysis and syntax use a modified BNF
grammar notation.  This uses the following style of definition:

\begin{verbatim}
name:           lcletter (lcletter | "_")*
lcletter:       "a"..."z"
\end{verbatim}

The first line says that a \verb\name\ is an \verb\lcletter\ followed by
a sequence of zero or more \verb\lcletter\s and underscores.  An
\verb\lcletter\ in turn is any of the single characters `a' through `z'.
(This rule is actually adhered to for the names defined in syntax and
grammar rules in this document.)

Each rule begins with a name (which is the name defined by the rule)
and a colon, and is wholly contained on one line.  A vertical bar
(\verb\|\) is used to separate alternatives; it is the least binding
operator in this notation.  A star (\verb\*\) means zero or more
repetitions of the preceding item; likewise, a plus (\verb\+\) means
one or more repetitions, and a question mark (\verb\?\) zero or one
(in other words, the preceding item is optional).  These three
operators bind as tightly as possible; parentheses are used for
grouping.  Literal strings are enclosed in double quotes.  White space
is only meaningful to separate tokens.

In lexical definitions (as the example above), two more conventions
are used: Two literal characters separated by three dots mean a choice
of any single character in the given (inclusive) range of ASCII
characters.  A phrase between angular brackets (\verb\<...>\) gives an
informal description of the symbol defined; e.g., this could be used
to describe the notion of `control character' if needed.

Even though the notation used is almost the same, there is a big
difference between the meaning of lexical and syntactic definitions:
a lexical definition operates on the individual characters of the
input source, while a syntax definition operates on the stream of
tokens generated by the lexical analysis.

\chapter{Lexical analysis}

A Python program is read by a {\em parser}.  Input to the parser is a
stream of {\em tokens}, generated by the {\em lexical analyzer}.  This
chapter describes how the lexical analyzer breaks a file into tokens.

\section{Line structure}

A Python program is divided in a number of logical lines.  The end of
a logical line is represented by the token NEWLINE.  Statements cannot
cross logical line boundaries except where NEWLINE is allowed by the
syntax (e.g., between statements in compound statements).

\subsection{Comments}

A comment starts with a hash character (\verb\#\) that is not part of
a string literal, and ends at the end of the physical line.  A comment
always signifies the end of the logical line.  Comments are ignored by
the syntax.

\subsection{Line joining}

Two or more physical lines may be joined into logical lines using
backslash characters (\verb/\/), as follows: when a physical line ends
in a backslash that is not part of a string literal or comment, it is
joined with the following forming a single logical line, deleting the
backslash and the following end-of-line character.  For example:
%
\begin{verbatim}
samplingrates =	(48000, AL.RATE_48000), \
                (44100, AL.RATE_44100), \
                (32000, AL.RATE_32000), \
                (22050, AL.RATE_22050), \
                (16000, AL.RATE_16000), \
                (11025, AL.RATE_11025), \
                ( 8000,  AL.RATE_8000)
\end{verbatim}

\subsection{Blank lines}

A logical line that contains only spaces, tabs, and possibly a
comment, is ignored (i.e., no NEWLINE token is generated), except that
during interactive input of statements, an entirely blank logical line
terminates a multi-line statement.

\subsection{Indentation}

Leading whitespace (spaces and tabs) at the beginning of a logical
line is used to compute the indentation level of the line, which in
turn is used to determine the grouping of statements.

First, tabs are replaced (from left to right) by one to eight spaces
such that the total number of characters up to there is a multiple of
eight (this is intended to be the same rule as used by UNIX).  The
total number of spaces preceding the first non-blank character then
determines the line's indentation.  Indentation cannot be split over
multiple physical lines using backslashes.

The indentation levels of consecutive lines are used to generate
INDENT and DEDENT tokens, using a stack, as follows.

Before the first line of the file is read, a single zero is pushed on
the stack; this will never be popped off again.  The numbers pushed on
the stack will always be strictly increasing from bottom to top.  At
the beginning of each logical line, the line's indentation level is
compared to the top of the stack.  If it is equal, nothing happens.
If it larger, it is pushed on the stack, and one INDENT token is
generated.  If it is smaller, it {\em must} be one of the numbers
occurring on the stack; all numbers on the stack that are larger are
popped off, and for each number popped off a DEDENT token is
generated.  At the end of the file, a DEDENT token is generated for
each number remaining on the stack that is larger than zero.

Here is an example of a correctly (though confusingly) indented piece
of Python code:

\begin{verbatim}
def perm(l):

        # Compute the list of all permutations of l

    if len(l) <= 1:
                  return [l]
    r = []
    for i in range(len(l)):
             s = l[:i] + l[i+1:]
             p = perm(s)
             for x in p:
              r.append(l[i:i+1] + x)
    return r
\end{verbatim}

The following example shows various indentation errors:

\begin{verbatim}
    def perm(l):                        # error: first line indented
    for i in range(len(l)):             # error: not indented
        s = l[:i] + l[i+1:]
            p = perm(l[:i] + l[i+1:])   # error: unexpected indent
            for x in p:
                    r.append(l[i:i+1] + x)
                return r                # error: inconsistent indent
\end{verbatim}

(Actually, the first three errors are detected by the parser; only the
last error is found by the lexical analyzer -- the indentation of
\verb\return r\ does not match a level popped off the stack.)

\section{Other tokens}

Besides NEWLINE, INDENT and DEDENT, the following categories of tokens
exist: identifiers, keywords, literals, operators, and delimiters.
Spaces and tabs are not tokens, but serve to delimit tokens.  Where
ambiguity exists, a token comprises the longest possible string that
forms a legal token, when read from left to right.

\section{Identifiers}

Identifiers are described by the following regular expressions:

\begin{verbatim}
identifier:     (letter|"_") (letter|digit|"_")*
letter:         lowercase | uppercase
lowercase:      "a"..."z"
uppercase:      "A"..."Z"
digit:          "0"..."9"
\end{verbatim}

Identifiers are unlimited in length.  Case is significant.

\subsection{Keywords}

The following identifiers are used as reserved words, or {\em
keywords} of the language, and cannot be used as ordinary
identifiers.  They must be spelled exactly as written here:

\begin{verbatim}
and        del        for        in         print
break      elif       from       is         raise
class      else       global     not        return
continue   except     if         or         try
def        finally    import     pass       while
\end{verbatim}

%	# This Python program sorts and formats the above table
%	import string
%	l = []
%	try:
%		while 1:
%			l = l + string.split(raw_input())
%	except EOFError:
%		pass
%	l.sort()
%	for i in range((len(l)+4)/5):
%		for j in range(i, len(l), 5):
%			print string.ljust(l[j], 10),
%		print

\section{Literals}

\subsection{String literals}

String literals are described by the following regular expressions:

\begin{verbatim}
stringliteral:  "'" stringitem* "'"
stringitem:     stringchar | escapeseq
stringchar:     <any ASCII character except newline or "\" or "'">
escapeseq:      "'" <any ASCII character except newline>
\end{verbatim}

String literals cannot span physical line boundaries.  Escape
sequences in strings are actually interpreted according to rules
simular to those used by Standard C.  The recognized escape sequences
are:

\begin{center}
\begin{tabular}{|l|l|}
\hline
\verb/\\/	& Backslash (\verb/\/) \\
\verb/\'/	& Single quote (\verb/'/) \\
\verb/\a/	& ASCII Bell (BEL) \\
\verb/\b/	& ASCII Backspace (BS) \\
%\verb/\E/	& ASCII Escape (ESC) \\
\verb/\f/	& ASCII Formfeed (FF) \\
\verb/\n/	& ASCII Linefeed (LF) \\
\verb/\r/	& ASCII Carriage Return (CR) \\
\verb/\t/	& ASCII Horizontal Tab (TAB) \\
\verb/\v/	& ASCII Vertical Tab (VT) \\
\verb/\/{\em ooo}	& ASCII character with octal value {\em ooo} \\
\verb/\x/{em xx...}	& ASCII character with hex value {\em xx...} \\
\hline
\end{tabular}
\end{center}

In strict compatibility with in Standard C, up to three octal digits are
accepted, but an unlimited number of hex digits is taken to be part of
the hex escape (and then the lower 8 bits of the resulting hex number
are used in all current implementations...).

All unrecognized escape sequences are left in the string unchanged,
i.e., {\em the backslash is left in the string.}  (This rule is
useful when debugging: if an escape sequence is mistyped, the
resulting output is more easily recognized as broken.  It also helps a
great deal for string literals used as regular expressions or
otherwise passed to other modules that do their own escape handling --
but you may end up quadrupling backslashes that must appear literally.)

\subsection{Numeric literals}

There are three types of numeric literals: plain integers, long
integers, and floating point numbers.

Integers and long integers are described by the following regular expressions:

\begin{verbatim}
longinteger:    integer ("l"|"L")
integer:        decimalinteger | octinteger | hexinteger
decimalinteger: nonzerodigit digit* | "0"
octinteger:     "0" octdigit+
hexinteger:     "0" ("x"|"X") hexdigit+

nonzerodigit:   "1"..."9"
octdigit:       "0"..."7"
hexdigit:        digit|"a"..."f"|"A"..."F"
\end{verbatim}

Although both lower case `l'and upper case `L' are allowed as suffix
for long integers, it is strongly recommended to always use `L', since
the letter `l' looks too much like the digit `1'.

(Plain) integer decimal literals must be at most $2^{31} - 1$ (i.e., the
largest positive integer, assuming 32-bit arithmetic); octal and
hexadecimal literals may be as large as $2^{32} - 1$.  There is no limit
for long integer literals.

Some examples of (plain and long) integer literals:

\begin{verbatim}
7     2147483647                        0177    0x80000000
3L    79228162514264337593543950336L    0377L   0100000000L
\end{verbatim}

Floating point numbers are described by the following regular expressions:

\begin{verbatim}
floatnumber:    pointfloat | exponentfloat
pointfloat:     [intpart] fraction | intpart "."
exponentfloat:  (intpart | pointfloat) exponent
intpart:        digit+
fraction:       "." digit+
exponent:       ("e"|"E") ["+"|"-"] digit+
\end{verbatim}

The range of floating point literals is implementation-dependent.

Some examples of floating point literals:

\begin{verbatim}
3.14    10.    .001    1e100    3.14e-10
\end{verbatim}

Note that numeric literals do not include a sign; a phrase like
\verb\-1\ is actually an expression composed of the operator
\verb\-\ and the literal \verb\1\.

\section{Operators}

The following tokens are operators:

\begin{verbatim}
+       -       *       /       %
<<      >>      &       |       ^       ~
<       ==      >       <=      <>      !=      >=
\end{verbatim}

The comparison operators \verb\<>\ and \verb\!=\ are alternate
spellings of the same operator.

\section{Delimiters}

The following tokens serve as delimiters or otherwise have a special
meaning:

\begin{verbatim}
(       )       [       ]       {       }
;       ,       :       .       `       =
\end{verbatim}

The following printing ASCII characters are not used in Python (except
in string literals and in comments).  Their occurrence is an
unconditional error:

\begin{verbatim}
!       @       $       "       ?
\end{verbatim}

They may be used by future versions of the language though!

\chapter{Execution model}

\section{Objects, values and types}

I won't try to define rigorously here what an object is, but I'll give
some properties of objects that are important to know about.

Every object has an identity, a type and a value.  An object's {\em
identity} never changes once it has been created; think of it as the
object's (permanent) address.  An object's {\em type} determines the
operations that an object supports (e.g., does it have a length?)  and
also defines the ``meaning'' of the object's value.  The type also
never changes.  The {\em value} of some objects can change; whether
this is possible is a property of its type.

Objects are never explicitly destroyed; however, when they become
unreachable they may be garbage-collected.  An implementation is
allowed to delay garbage collection or omit it altogether -- it is a
matter of implementation quality how garbage collection is
implemented, as long as no objects are collected that are still
reachable.  (Implementation note: the current implementation uses a
reference-counting scheme which collects most objects as soon as they
become onreachable, but never collects garbage containing circular
references.)

Note that the use of the implementation's tracing or debugging
facilities may keep objects alive that would normally be collectable.

(Some objects contain references to ``external'' resources such as
open files.  It is understood that these resources are freed when the
object is garbage-collected, but since garbage collection is not
guaranteed, such objects also provide an explicit way to release the
external resource (e.g., a \verb\close\ method).  Programs are strongly
recommended to use this.)

Some objects contain references to other objects.  These references
are part of the object's value; in most cases, when such a
``container'' object is compared to another (of the same type), the
comparison applies to the {\em values} of the referenced objects (not
their identities).

Types affect almost all aspects of objects.
Even object identity is affected in some sense: for immutable
types, operations that compute new values may actually return a
reference to any existing object with the same type and value, while
for mutable objects this is not allowed.  E.g., after

\begin{verbatim}
a = 1; b = 1; c = []; d = []
\end{verbatim}

\verb\a\ and \verb\b\ may or may not refer to the same object, but
\verb\c\ and \verb\d\ are guaranteed to refer to two different, unique,
newly created lists.

\section{The standard type hierarchy}

XXX None, sequences, numbers, mappings, ...

\section{Execution frames, name spaces, and scopes}

XXX code blocks, scopes, name spaces, name binding, exceptions

\chapter{Expressions and conditions}

From now on, extended BNF notation will be used to describe syntax,
not lexical analysis.

This chapter explains the meaning of the elements of expressions and
conditions.  Conditions are a superset of expressions, and a condition
may be used wherever an expression is required by enclosing it in
parentheses.  The only places where expressions are used in the syntax
instead of conditions is in expression statements and on the
right-hand side of assignments; this catches some nasty bugs like
accedentally writing \verb\x == 1\ instead of \verb\x = 1\.

The comma has several roles in Python's syntax.  It is usually an
operator with a lower precedence than all others, but occasionally
serves other purposes as well; e.g., it separates function arguments,
is used in list and dictionary constructors, and has special semantics
in \verb\print\ statements.

When (one alternative of) a syntax rule has the form

\begin{verbatim}
name:           othername
\end{verbatim}

and no semantics are given, the semantics of this form of \verb\name\
are the same as for \verb\othername\.

\section{Arithmetic conversions}

When a description of an arithmetic operator below uses the phrase
``the numeric arguments are converted to a common type'',
this both means that if either argument is not a number, a
{\tt TypeError} exception is raised, and that otherwise
the following conversions are applied:

\begin{itemize}
\item	First, if either argument is a floating point number,
	the other is converted to floating point;
\item	else, if either argument is a long integer,
	the other is converted to long integer;
\item	otherwise, both must be plain integers and no conversion
	is necessary.
\end{itemize}

(Note: ``plain integers'' in Python are at least 32 bits in size;
``long integers'' are arbitrary precision integers.)

\section{Atoms}

Atoms are the most basic elements of expressions.  Forms enclosed in
reverse quotes or in parentheses, brackets or braces are also
categorized syntactically as atoms.  The syntax for atoms is:

\begin{verbatim}
atom:           identifier | literal | enclosure
enclosure:      parenth_form | list_display | dict_display | string_conversion
\end{verbatim}

\subsection{Identifiers (Names)}

An identifier occurring as an atom is a reference to a local, global
or built-in name binding.  If a name can be assigned to anywhere in a
code block, and is not mentioned in a \verb\global\ statement in that
code block, it refers to a local name throughout that code block.
Otherwise, it refers to a global name if one exists, else to a
built-in name.

When the name is bound to an object, evaluation of the atom yields
that object.  When a name is not bound, an attempt to evaluate it
raises a {\tt NameError} exception.

\subsection{Literals}

Python knows string and numeric literals:

\begin{verbatim}
literal:        stringliteral | integer | longinteger | floatnumber
\end{verbatim}

Evaluation of a literal yields an object of the given type
(string, integer, long integer, floating point number)
with the given value.
The value may be approximated in the case of floating point literals.

All literals correspond to immutable data types, and hence the
object's identity is less important than its value.  Multiple
evaluations of literals with the same value (either the same
occurrence in the program text or a different occurrence) may obtain
the same object or a different object with the same value.

(In the original implementation, all literals in the same code block
with the same type and value yield the same object.)

\subsection{Parenthesized form}

A parenthesized form is an optional condition list enclosed in
parentheses:

\begin{verbatim}
parenth_form:      "(" [condition_list] ")"
\end{verbatim}

A parenthesized condition list yields whatever that condition list
yields.

An empty pair of parentheses yields an empty tuple object (since
tuples are immutable, the rules for literals apply here).

(Note that tuples are not formed by the parentheses, but rather by use
of the comma operator.  The exception is the empty tuple, for which
parentheses {\em are} required -- allowing unparenthesized ``nothing''
in expressions would causes ambiguities and allow common typos to
pass uncaught.)

\subsection{List displays}

A list display is a possibly empty series of conditions enclosed in
square brackets:

\begin{verbatim}
list_display:   "[" [condition_list] "]"
\end{verbatim}

A list display yields a new list object.

If it has no condition list, the list object has no items.
Otherwise, the elements of the condition list are evaluated
from left to right and inserted in the list object in that order.

\subsection{Dictionary displays}

A dictionary display is a possibly empty series of key/datum pairs
enclosed in curly braces:

\begin{verbatim}
dict_display:   "{" [key_datum_list] "}"
key_datum_list: [key_datum ("," key_datum)* [","]
key_datum:      condition ":" condition
\end{verbatim}

A dictionary display yields a new dictionary object.

The key/datum pairs are evaluated from left to right to define the
entries of the dictionary: each key object is used as a key into the
dictionary to store the corresponding datum.

Keys must be strings, otherwise a {\tt TypeError} exception is raised.%
\footnote{
This restriction may be lifted in a future version of the language.
}
Clashes between duplicate keys are not detected; the last datum
(textually rightmost in the display) stored for a given key value
prevails.

\subsection{String conversions}

A string conversion is a condition list enclosed in {\em reverse} (or
backward) quotes:

\begin{verbatim}
string_conversion: "`" condition_list "`"
\end{verbatim}

A string conversion evaluates the contained condition list and converts the
resulting object into a string according to rules specific to its type.

If the object is a string, a number, \verb\None\, or a tuple, list or
dictionary containing only objects whose type is one of these, the
resulting string is a valid Python expression which can be passed to
the built-in function \verb\eval()\ to yield an expression with the
same value (or an approximation, if floating point numbers are
involved).

(In particular, converting a string adds quotes around it and converts
``funny'' characters to escape sequences that are safe to print.)

It is illegal to attempt to convert recursive objects (e.g., lists or
dictionaries that contain a reference to themselves, directly or
indirectly.)

\section{Primaries}

Primaries represent the most tightly bound operations of the language.
Their syntax is:

\begin{verbatim}
primary:        atom | attributeref | subscription | slicing | call
\end{verbatim}

\subsection{Attribute references}

An attribute reference is a primary followed by a period and a name:

\begin{verbatim}
attributeref:   primary "." identifier
\end{verbatim}

The primary must evaluate to an object of a type that supports
attribute references, e.g., a module or a list.  This object is then
asked to produce the attribute whose name is the identifier.  If this
attribute is not available, the exception \verb\AttributeError\ is
raised.  Otherwise, the type and value of the object produced is
determined by the object.  Multiple evaluations of the same attribute
reference may yield different objects.

\subsection{Subscriptions}

A subscription selects an item of a sequence or mapping object:

\begin{verbatim}
subscription:   primary "[" condition "]"
\end{verbatim}

The primary must evaluate to an object of a sequence or mapping type.

If it is a mapping, the condition must evaluate to an object whose
value is one of the keys of the mapping, and the subscription selects
the value in the mapping that corresponds to that key.

If it is a sequence, the condition must evaluate to a nonnegative
plain integer smaller than the number of items in the sequence, and
the subscription selects the item whose index is that value (counting
from zero).

A string's  items are characters.  A character is not a separate data
type but a string of exactly one character.

\subsection{Slicings}

A slicing selects a range of items in a sequence object:

\begin{verbatim}
slicing:        primary "[" [condition] ":" [condition] "]"
\end{verbatim}

XXX

\subsection{Calls}

A call calls a function with a possibly empty series of arguments:

\begin{verbatim}
call:           primary "(" [condition_list] ")"
\end{verbatim}

The primary must evaluate to a callable object.  Callable objects are
user-defined functions, built-in functions, methods of built-in
objects (``built-in methods''), class objects, and methods of class
instances (``user-defined methods'').  If it is a class, the argument
list must be empty.

XXX explain what happens on function call

\section{Factors}

Factors represent the unary numeric operators.
Their syntax is:

\begin{verbatim}
factor:         primary | "-" factor | "+" factor | "~" factor
\end{verbatim}

The unary \verb\-\ operator yields the negative of its numeric argument.

The unary \verb\+\ operator yields its numeric argument unchanged.

The unary \verb\~\ operator yields the bit-wise negation of its
(plain or long) integral numerical argument, using 2's complement.

In all three cases, if the argument does not have the proper type,
a {\tt TypeError} exception is raised.

\section{Terms}

Terms represent the most tightly binding binary operators:

\begin{verbatim}
term:           factor | term "*" factor | term "/" factor | term "%" factor
\end{verbatim}

The \verb\*\ (multiplication) operator yields the product of its
arguments.  The arguments must either both be numbers, or one argument
must be a plain integer and the other must be a sequence.  In the
former case, the numbers are converted to a common type and then
multiplied together.  In the latter case, sequence repetition is
performed; a negative repetition factor yields the empty string.

The \verb|"/"| (division) operator yields the quotient of its
arguments.  The numeric arguments are first converted to a common
type.  (Plain or long) integer division yields an integer of the same
type; the result is that of mathematical division with the {\em floor}
operator applied to the result, to match the modulo operator.
Division by zero raises a {\tt RuntimeError} exception.

The \verb|"%"| (modulo) operator yields the remainder from the
division of the first argument by the second.  The numeric arguments
are first converted to a common type.  A zero right argument raises a
{\tt RuntimeError} exception.  The arguments may be floating point
numbers, e.g., $3.14 \% 0.7$ equals $0.34$.  The modulo operator
always yields a result with the same sign as its second operand (or
zero); the absolute value of the result is strictly smaller than the
second operand.

The integer division and modulo operators are connected by the
following identity: $x = (x/y)*y + (x\%y)$.

\section{Arithmetic expressions}

\begin{verbatim}
arith_expr:     term | arith_expr "+" term | arith_expr "-" term
\end{verbatim}

HIRO

The \verb|"+"| operator yields the sum of its arguments.  The
arguments must either both be numbers, or both sequences.  In the
former case, the numbers are converted to a common type and then added
together.  In the latter case, the sequences are concatenated
directly.

The \verb|"-"| operator yields the difference of its arguments.
The numeric arguments are first converted to a common type.

\section{Shift expressions}

\begin{verbatim}
shift_expr:     arith_expr | shift_expr "<<" arith_expr | shift_expr ">>" arith_expr
\end{verbatim}

These operators accept (plain) integers as arguments only.
They shift their left argument to the left or right by the number of bits
given by the right argument.  Shifts are ``logical"", e.g., bits shifted
out on one end are lost, and bits shifted in are zero;
negative numbers are shifted as if they were unsigned in C.
Negative shift counts and shift counts greater than {\em or equal to}
the word size yield undefined results.

\section{Bitwise AND expressions}

\begin{verbatim}
and_expr:       shift_expr | and_expr "&" shift_expr
\end{verbatim}

This operator yields the bitwise AND of its arguments,
which must be (plain) integers.

\section{Bitwise XOR expressions}

\begin{verbatim}
xor_expr:       and_expr | xor_expr "^" and_expr
\end{verbatim}

This operator yields the bitwise exclusive OR of its arguments,
which must be (plain) integers.

\section{Bitwise OR expressions}

\begin{verbatim}
or_expr:       xor_expr | or_expr "|" xor_expr
\end{verbatim}

This operator yields the bitwise OR of its arguments,
which must be (plain) integers.

\section{Expressions and expression lists}

\begin{verbatim}
expression:     or_expression
expr_list:      expression ("," expression)* [","]
\end{verbatim}

An expression list containing at least one comma yields a new tuple.
The length of the tuple is the number of expressions in the list.
The expressions are evaluated from left to right.

The trailing comma is required only to create a single tuple;
it is optional in all other cases (a single expression without
a trailing comma doesn't create a tuple, but rather yields the
value of that expression).

To create an empty tuple, use an empty pair of parentheses: \verb\()\.

\section{Comparisons}

\begin{verbatim}
comparison:     expression (comp_operator expression)*
comp_operator:  "<"|">"|"=="|">="|"<="|"<>"|"!="|"is" ["not"]|["not"] "in"
\end{verbatim}

Comparisons yield integer value: 1 for true, 0 for false.

Comparisons can be chained arbitrarily,
e.g., $x < y <= z$ is equivalent to
$x < y$ {\tt and} $y <= z$, except that $y$ is evaluated only once
(but in both cases $z$ is not evaluated at all when $x < y$ is
found to be false).

Formally, $e_0 op_1 e_1 op_2 e_2 ...e_{n-1} op_n e_n$ is equivalent to
$e_0 op_1 e_1$ {\tt and} $e_1 op_2 e_2$ {\tt and} ... {\tt and}
$e_{n-1} op_n e_n$, except that each expression is evaluated at most once.

Note that $e_0 op_1 e_1 op_2 e_2$ does not imply any kind of comparison
between $e_0$ and $e_2$, e.g., $x < y > z$ is perfectly legal.

The forms \verb\<>\ and \verb\!=\ are equivalent.

The operators {\tt "<", ">", "==", ">=", "<="}, and {\tt "<>"} compare
the values of two objects.  The objects needn't have the same type.
If both are numbers, they are compared to a common type.
Otherwise, objects of different types {\em always} compare unequal,
and are ordered consistently but arbitrarily, except that
the value \verb\None\ compares smaller than the values of any other type.

(This unusual
definition of comparison is done to simplify the definition of
operations like sorting and the \verb\in\ and \verb\not in\ operators.)

Comparison of objects of the same type depends on the type:

\begin{itemize}
\item	Numbers are compared arithmetically.
\item	Strings are compared lexicographically using the numeric
	equivalents (the result of the built-in function ord())
	of their characters.
\item	Tuples and lists are compared lexicographically
	using comparison of corresponding items.
\item	Dictionaries compare unequal unless they are the same object;
	the choice whether one dictionary object is considered smaller
	or larger than another one is made arbitrarily but
	consistently within one execution of a program.
\item	The latter rule is also used for most other built-in types.
\end{itemize}

The operators \verb\in\ and \verb\not in\ test for sequence membership:
if $y$ is a sequence, $x {\tt in} y$ is true if and only if there exists
an index $i$ such that $x = y_i$.
$x {\tt not in} y$ yields the inverse truth value.
The exception {\tt TypeError} is raised when $y$ is not a sequence,
or when $y$ is a string and $x$ is not a string of length one.

The operators \verb\is\ and \verb\is not\ compare object identity:
$x {\tt is} y$ is true if and only if $x$ and $y$ are the same object.
$x {\tt is not} y$ yields the inverse truth value.

\section{Boolean operators}

\begin{verbatim}
condition:      or_test
or_test:        and_test | or_test "or" and_test
and_test:       not_test | and_test "and" not_test
not_test:       comparison | "not" not_test
\end{verbatim}

In the context of Boolean operators, and also when conditions are
used by control flow statements, the following values are interpreted
as false: None, numeric zero of all types, empty sequences (strings,
tuples and lists), and empty mappings (dictionaries).
All other values are interpreted as true.

The operator \verb\not\ yields 1 if its argument is false, 0 otherwise.

The condition $x {\tt and} y$ first evaluates $x$; if $x$ is false,
$x$ is returned; otherwise, $y$ is evaluated and returned.

The condition $x {\tt or} y$ first evaluates $x$; if $x$ is true,
$x$ is returned; otherwise, $y$ is evaluated and returned.

(Note that \verb\and\ and \verb\or\ do not restrict the value and type
they return to 0 and 1, but rather return the last evaluated argument.
This is sometimes useful, e.g., if $s$ is a string, which should be
replaced by a default value if it is empty, $s {\tt or} 'foo'$
returns the desired value.  Because \verb\not\ has to invent a value
anyway, it does not bother to return a value of the same type as its
argument, so \verb\not 'foo'\ yields $0$, not $''$.)

\chapter{Simple statements}

Simple statements are comprised within a single logical line.
Several simple statements may occor on a single line separated
by semicolons.  The syntax for simple statements is:

\begin{verbatim}
stmt_list:      simple_stmt (";" simple_stmt)* [";"]
simple_stmt:    expression_stmt
              | assignment
              | pass_stmt
              | del_stmt
              | print_stmt
              | return_stmt
              | raise_stmt
              | break_stmt
              | continue_stmt
              | import_stmt
              | global_stmt
\end{verbatim}

\section{Expression statements}

\begin{verbatim}
expression_stmt: expression_list
\end{verbatim}

An expression statement evaluates the expression list (which may
be a single expression).
If the value is not \verb\None\, it is converted to a string
using the rules for string conversions, and the resulting string
is written to standard output on a line by itself.

(The exception for \verb\None\ is made so that procedure calls,
which are syntactically equivalent to expressions,
do not cause any output.)

\section{Assignments}

\begin{verbatim}
assignment:     target_list ("=" target_list)* "=" expression_list
target_list:    target ("," target)* [","]
target:         identifier | "(" target_list ")" | "[" target_list "]"
              | attributeref | subscription | slicing
\end{verbatim}

(See the section on primaries for the definition of the last
three symbols.)

An assignment evaluates the expression list (remember that this can
be a single expression or a comma-separated list,
the latter yielding a tuple)
and assigns the single resulting object to each of the target lists,
from left to right.

Assignment is defined recursively depending on the type of the
target.  Where assignment is to part of a mutable object
(through an attribute reference, subscription or slicing),
the mutable object must ultimately perform the
assignment and decide about its validity, raising an exception
if the assignment is unacceptable.  The rules observed by
various types and the exceptions raised are given with the
definition of the object types (some of which are defined
in the library reference).

Assignment of an object to a target list is recursively
defined as follows.

\begin{itemize}
\item
If the target list contains no commas (except in nested constructs):
the object is assigned to the single target contained in the list.

\item
If the target list contains commas (that are not in nested constructs):
the object must be a tuple with as many items
as the list contains targets, and the items are assigned, from left
to right, to the corresponding targets.

\end{itemize}

Assignment of an object to a (non-list)
target is recursively defined as follows.

\begin{itemize}

\item
If the target is an identifier (name):
the object is bound to that name
in the current local scope.  Any previous binding of the same name
is undone.

\item
If the target is a target list enclosed in parentheses:
the object is assigned to that target list.

\item
If the target is a target list enclosed in square brackets:
the object must be a list with as many items
as the target list contains targets,
and the list's items are assigned, from left to right,
to the corresponding targets.

\item
If the target is an attribute reference:
The primary expression in the reference is evaluated.
It should yield an object with assignable attributes;
if this is not the case, a {\tt TypeError} exception is raised.
That object is then asked to assign the assigned object
to the given attribute; if it cannot perform the assignment,
it raises an exception.

\item
If the target is a subscription:
The primary expression in the reference is evaluated.
It should yield either a mutable sequence object or a mapping
(dictionary) object.
Next, the subscript expression is evaluated.

If the primary is a sequence object, the subscript must yield a
nonnegative integer smaller than the sequence's length,
and the sequence is asked to assign the assigned object
to its item with that index.

If the primary is a mapping object, the subscript must have a
type compatible with the mapping's key type,
and the mapping is then asked to to create a key/datum pair
which maps the subscript to the assigned object.

Various exceptions can be raised.

\item
If the target is a slicing:
The primary expression in the reference is evaluated.
It should yield a mutable sequence object (currently only lists).
The assigned object should be a sequence object of the same type.
Next, the lower and upper bound expressions are evaluated,
insofar they are present; defaults are zero and the sequence's length.
The bounds should evaluate to (small) integers.
If either bound is negative, the sequence's length is added to it (once).
The resulting bounds are clipped to lie between zero
and the sequence's length, inclusive.
(XXX Shouldn't this description be with expressions?)
Finally, the sequence object is asked to replace the items
indicated by the slice with the items of the assigned sequence.
This may change the sequence's length, if it allows it.

\end{itemize}

(In the original implementation, the syntax for targets is taken
to be the same as for expressions, and invalid syntax is rejected
during the code generation phase, causing less detailed error
messages.)

\section{The {\tt pass} statement}

\begin{verbatim}
pass_stmt:      "pass"
\end{verbatim}

{\tt pass} is a null operation -- when it is executed,
nothing happens.

\section{The {\tt del} statement}

\begin{verbatim}
del_stmt:       "del" target_list
\end{verbatim}

Deletion is recursively defined similar to assignment.

(XXX Rather that spelling it out in full details,
here are some hints.)

Deletion of a target list recursively deletes each target,
from left to right.

Deletion of a name removes the binding of that name (which must exist)
from the local scope.

Deletion of attribute references, subscriptions and slicings
is passed to the primary object involved; deletion of a slicing
is in general equivalent to assignment of an empty slice of the
right type (but even this is determined by the sliced object).

\section{The {\tt print} statement}

\begin{verbatim}
print_stmt:     "print" [ condition ("," condition)* [","] ]
\end{verbatim}

{\tt print} evaluates each condition in turn and writes the resulting
object to standard output (see below).
If an object is not a string, it is first converted to
a string using the rules for string conversions.
The (resulting or original) string is then written.
A space is written before each object is (converted and) written,
unless the output system believes it is positioned at the beginning
of a line.  This is the case: (1) when no characters have been written
to standard output; or (2) when the last character written to
standard output is \verb/\n/;
or (3) when the last I/O operation
on standard output was not a \verb\print\ statement.

Finally,
a \verb/\n/ character is written at the end,
unless the \verb\print\ statement ends with a comma.
This is the only action if the statement contains just the keyword
\verb\print\.

Standard output is defined as the file object named \verb\stdout\
in the built-in module \verb\sys\.  If no such object exists,
or if it is not a writable file, a {\tt RuntimeError} exception is raised.
(The original implementation attempts to write to the system's original
standard output instead, but this is not safe, and should be fixed.)

\section{The {\tt return} statement}

\begin{verbatim}
return_stmt:    "return" [condition_list]
\end{verbatim}

\verb\return\ may only occur syntactically nested in a function
definition, not within a nested class definition.

If a condition list is present, it is evaluated, else \verb\None\
is substituted.

\verb\return\ leaves the current function call with the condition
list (or \verb\None\) as return value.

When \verb\return\ passes control out of a \verb\try\ statement
with a \verb\finally\ clause, that finally clause is executed
before really leaving the function.
(XXX This should be made more exact, a la Modula-3.)

\section{The {\tt raise} statement}

\begin{verbatim}
raise_stmt:     "raise" condition ["," condition]
\end{verbatim}

\verb\raise\ evaluates its first condition, which must yield
a string object.  If there is a second condition, this is evaluated,
else \verb\None\ is substituted.

It then raises the exception identified by the first object,
with the second one (or \verb\None\) as its parameter.

\section{The {\tt break} statement}

\begin{verbatim}
break_stmt:     "break"
\end{verbatim}

\verb\break\ may only occur syntactically nested in a \verb\for\
or \verb\while\ loop, not nested in a function or class definition.

It terminates the neares enclosing loop, skipping the optional
\verb\else\ clause if the loop has one.

If a \verb\for\ loop is terminated by \verb\break\, the loop control
target (list) keeps its current value.

When \verb\break\ passes control out of a \verb\try\ statement
with a \verb\finally\ clause, that finally clause is executed
before really leaving the loop.

\section{The {\tt continue} statement}

\begin{verbatim}
continue_stmt:  "continue"
\end{verbatim}

\verb\continue\ may only occur syntactically nested in a \verb\for\
or \verb\while\ loop, not nested in a function or class definition,
and {\em not nested in a \verb\try\ statement with a \verb\finally\
clause}.

It continues with the next cycle of the nearest enclosing loop.

\section{The {\tt import} statement}

\begin{verbatim}
import_stmt:    "import" identifier ("," identifier)*
              | "from" identifier "import" identifier ("," identifier)*
              | "from" identifier "import" "*"
\end{verbatim}

(XXX To be done.)

\section{The {\tt global} statement}

\begin{verbatim}
global_stmt:    "global" identifier ("," identifier)*
\end{verbatim}

(XXX To be done.)

\chapter{Compound statements}

(XXX The semantic definitions of this chapter are still to be done.)

\begin{verbatim}
statement:      stmt_list NEWLINE | compound_stmt
compound_stmt:  if_stmt | while_stmt | for_stmt | try_stmt | funcdef | classdef
suite:          statement | NEWLINE INDENT statement+ DEDENT
\end{verbatim}

\section{The {\tt if} statement}

\begin{verbatim}
if_stmt:        "if" condition ":" suite
               ("elif" condition ":" suite)*
               ["else" ":" suite]
\end{verbatim}

\section{The {\tt while} statement}

\begin{verbatim}
while_stmt:     "while" condition ":" suite ["else" ":" suite]
\end{verbatim}

\section{The {\tt for} statement}

\begin{verbatim}
for_stmt:       "for" target_list "in" condition_list ":" suite
               ["else" ":" suite]
\end{verbatim}

\section{The {\tt try} statement}

\begin{verbatim}
try_stmt:       "try" ":" suite
               ("except" condition ["," condition] ":" suite)*
               ["finally" ":" suite]
\end{verbatim}

\section{Function definitions}

\begin{verbatim}
funcdef:        "def" identifier "(" [parameter_list] ")" ":" suite
parameter_list: parameter ("," parameter)*
parameter:      identifier | "(" parameter_list ")"
\end{verbatim}

\section{Class definitions}

\begin{verbatim}
classdef:       "class" identifier [inheritance] ":" suite
inheritance:    "(" expression ("," expression)* ")"
\end{verbatim}

XXX Syntax for scripts, modules
XXX Syntax for interactive input, eval, exec, input
XXX New definition of expressions (as conditions)

\end{document}
