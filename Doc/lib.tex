% Format this file with latex.

\documentstyle[myformat]{report}		% To preview with xdvi

% Support to leave changebars in...
\newcommand{\cbstart}{}
\newcommand{\cbend}{}

% A command to force the text after an item to start on a new line
\newcommand{\itembreak}{
	\mbox{}\\*[0mm]
}

% A command to define a function item
\newcommand{\funcitem}[2]{\item[#1(#2)]}

% A command to define a data item
\newcommand{\dataitem}[1]{\item[#1]}

% A command to define an exception item
\newcommand{\excitem}[2]{
\item[#1 = {\tt '#2'}]
\itembreak
}

\title{\bf
	Python Library Reference
}

\author{
	Guido van Rossum \\
	Dept. CST, CWI, Kruislaan 413 \\
	1098 SJ Amsterdam, The Netherlands \\
	E-mail: {\tt guido@cwi.nl}
}

\begin{document}

\pagenumbering{roman}

\maketitle

\begin{abstract}

\noindent
This document describes the built-in types, exceptions and functions and
the standard modules that come with the Python system.
It assumes basic knowledge about the Python language.
For an informal introduction to the language, see the Tutorial document.
The Language Reference document (XXX not yet existing)
gives a more formal definition of the language.

\end{abstract}

\pagebreak

\tableofcontents

\pagebreak

\pagenumbering{arabic}

\input{lib1.tex}	% intro; built-in types, functions and exceptions
\input{lib2.tex}	% built-in modules
\input{lib3.tex}	% standard modules
\input{lib4.tex}	% OS-dependent appendixes
\input{lib5.tex}	% Graphics appendixes

\end{document}
