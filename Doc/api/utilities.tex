\chapter{Utilities \label{utilities}}

The functions in this chapter perform various utility tasks, ranging
from helping C code be more portable across platforms, using Python
modules from C, and parsing function arguments and constructing Python
values from C values.


\section{Operating System Utilities \label{os}}

\begin{cfuncdesc}{int}{Py_FdIsInteractive}{FILE *fp, char *filename}
  Return true (nonzero) if the standard I/O file \var{fp} with name
  \var{filename} is deemed interactive.  This is the case for files
  for which \samp{isatty(fileno(\var{fp}))} is true.  If the global
  flag \cdata{Py_InteractiveFlag} is true, this function also returns
  true if the \var{filename} pointer is \NULL{} or if the name is
  equal to one of the strings \code{'<stdin>'} or \code{'???'}.
\end{cfuncdesc}

\begin{cfuncdesc}{long}{PyOS_GetLastModificationTime}{char *filename}
  Return the time of last modification of the file \var{filename}.
  The result is encoded in the same way as the timestamp returned by
  the standard C library function \cfunction{time()}.
\end{cfuncdesc}

\begin{cfuncdesc}{void}{PyOS_AfterFork}{}
  Function to update some internal state after a process fork; this
  should be called in the new process if the Python interpreter will
  continue to be used.  If a new executable is loaded into the new
  process, this function does not need to be called.
\end{cfuncdesc}

\begin{cfuncdesc}{int}{PyOS_CheckStack}{}
  Return true when the interpreter runs out of stack space.  This is a
  reliable check, but is only available when \constant{USE_STACKCHECK}
  is defined (currently on Windows using the Microsoft Visual \Cpp{}
  compiler and on the Macintosh).  \constant{USE_CHECKSTACK} will be
  defined automatically; you should never change the definition in
  your own code.
\end{cfuncdesc}

\begin{cfuncdesc}{PyOS_sighandler_t}{PyOS_getsig}{int i}
  Return the current signal handler for signal \var{i}.  This is a
  thin wrapper around either \cfunction{sigaction()} or
  \cfunction{signal()}.  Do not call those functions directly!
  \ctype{PyOS_sighandler_t} is a typedef alias for \ctype{void
  (*)(int)}.
\end{cfuncdesc}

\begin{cfuncdesc}{PyOS_sighandler_t}{PyOS_setsig}{int i, PyOS_sighandler_t h}
  Set the signal handler for signal \var{i} to be \var{h}; return the
  old signal handler.  This is a thin wrapper around either
  \cfunction{sigaction()} or \cfunction{signal()}.  Do not call those
  functions directly!  \ctype{PyOS_sighandler_t} is a typedef alias
  for \ctype{void (*)(int)}.
\end{cfuncdesc}


\section{Process Control \label{processControl}}

\begin{cfuncdesc}{void}{Py_FatalError}{char *message}
  Print a fatal error message and kill the process.  No cleanup is
  performed.  This function should only be invoked when a condition is
  detected that would make it dangerous to continue using the Python
  interpreter; e.g., when the object administration appears to be
  corrupted.  On \UNIX, the standard C library function
  \cfunction{abort()}\ttindex{abort()} is called which will attempt to
  produce a \file{core} file.
\end{cfuncdesc}

\begin{cfuncdesc}{void}{Py_Exit}{int status}
  Exit the current process.  This calls
  \cfunction{Py_Finalize()}\ttindex{Py_Finalize()} and then calls the
  standard C library function
  \code{exit(\var{status})}\ttindex{exit()}.
\end{cfuncdesc}

\begin{cfuncdesc}{int}{Py_AtExit}{void (*func) ()}
  Register a cleanup function to be called by
  \cfunction{Py_Finalize()}\ttindex{Py_Finalize()}.  The cleanup
  function will be called with no arguments and should return no
  value.  At most 32 \index{cleanup functions}cleanup functions can be
  registered.  When the registration is successful,
  \cfunction{Py_AtExit()} returns \code{0}; on failure, it returns
  \code{-1}.  The cleanup function registered last is called first.
  Each cleanup function will be called at most once.  Since Python's
  internal finallization will have completed before the cleanup
  function, no Python APIs should be called by \var{func}.
\end{cfuncdesc}


\section{Importing Modules \label{importing}}

\begin{cfuncdesc}{PyObject*}{PyImport_ImportModule}{char *name}
  This is a simplified interface to
  \cfunction{PyImport_ImportModuleEx()} below, leaving the
  \var{globals} and \var{locals} arguments set to \NULL.  When the
  \var{name} argument contains a dot (when it specifies a submodule of
  a package), the \var{fromlist} argument is set to the list
  \code{['*']} so that the return value is the named module rather
  than the top-level package containing it as would otherwise be the
  case.  (Unfortunately, this has an additional side effect when
  \var{name} in fact specifies a subpackage instead of a submodule:
  the submodules specified in the package's \code{__all__} variable
  are \index{package variable!\code{__all__}}
  \withsubitem{(package variable)}{\ttindex{__all__}}loaded.)  Return
  a new reference to the imported module, or \NULL{} with an exception
  set on failure (the module may still be created in this case ---
  examine \code{sys.modules} to find out).
  \withsubitem{(in module sys)}{\ttindex{modules}}
\end{cfuncdesc}

\begin{cfuncdesc}{PyObject*}{PyImport_ImportModuleEx}{char *name,
                       PyObject *globals, PyObject *locals, PyObject *fromlist}
  Import a module.  This is best described by referring to the
  built-in Python function
  \function{__import__()}\bifuncindex{__import__}, as the standard
  \function{__import__()} function calls this function directly.

  The return value is a new reference to the imported module or
  top-level package, or \NULL{} with an exception set on failure (the
  module may still be created in this case).  Like for
  \function{__import__()}, the return value when a submodule of a
  package was requested is normally the top-level package, unless a
  non-empty \var{fromlist} was given.
\end{cfuncdesc}

\begin{cfuncdesc}{PyObject*}{PyImport_Import}{PyObject *name}
  This is a higher-level interface that calls the current ``import
  hook function''.  It invokes the \function{__import__()} function
  from the \code{__builtins__} of the current globals.  This means
  that the import is done using whatever import hooks are installed in
  the current environment, e.g. by \module{rexec}\refstmodindex{rexec}
  or \module{ihooks}\refstmodindex{ihooks}.
\end{cfuncdesc}

\begin{cfuncdesc}{PyObject*}{PyImport_ReloadModule}{PyObject *m}
  Reload a module.  This is best described by referring to the
  built-in Python function \function{reload()}\bifuncindex{reload}, as
  the standard \function{reload()} function calls this function
  directly.  Return a new reference to the reloaded module, or \NULL{}
  with an exception set on failure (the module still exists in this
  case).
\end{cfuncdesc}

\begin{cfuncdesc}{PyObject*}{PyImport_AddModule}{char *name}
  Return the module object corresponding to a module name.  The
  \var{name} argument may be of the form \code{package.module}).
  First check the modules dictionary if there's one there, and if not,
  create a new one and insert in in the modules dictionary.
  \note{This function does not load or import the module; if the
  module wasn't already loaded, you will get an empty module object.
  Use \cfunction{PyImport_ImportModule()} or one of its variants to
  import a module.  Return \NULL{} with an exception set on failure.}
\end{cfuncdesc}

\begin{cfuncdesc}{PyObject*}{PyImport_ExecCodeModule}{char *name, PyObject *co}
  Given a module name (possibly of the form \code{package.module}) and
  a code object read from a Python bytecode file or obtained from the
  built-in function \function{compile()}\bifuncindex{compile}, load
  the module.  Return a new reference to the module object, or \NULL{}
  with an exception set if an error occurred (the module may still be
  created in this case).  (This function would reload the module if it
  was already imported.)
\end{cfuncdesc}

\begin{cfuncdesc}{long}{PyImport_GetMagicNumber}{}
  Return the magic number for Python bytecode files
  (a.k.a. \file{.pyc} and \file{.pyo} files).  The magic number should
  be present in the first four bytes of the bytecode file, in
  little-endian byte order.
\end{cfuncdesc}

\begin{cfuncdesc}{PyObject*}{PyImport_GetModuleDict}{}
  Return the dictionary used for the module administration
  (a.k.a.\ \code{sys.modules}).  Note that this is a per-interpreter
  variable.
\end{cfuncdesc}

\begin{cfuncdesc}{void}{_PyImport_Init}{}
  Initialize the import mechanism.  For internal use only.
\end{cfuncdesc}

\begin{cfuncdesc}{void}{PyImport_Cleanup}{}
  Empty the module table.  For internal use only.
\end{cfuncdesc}

\begin{cfuncdesc}{void}{_PyImport_Fini}{}
  Finalize the import mechanism.  For internal use only.
\end{cfuncdesc}

\begin{cfuncdesc}{PyObject*}{_PyImport_FindExtension}{char *, char *}
  For internal use only.
\end{cfuncdesc}

\begin{cfuncdesc}{PyObject*}{_PyImport_FixupExtension}{char *, char *}
  For internal use only.
\end{cfuncdesc}

\begin{cfuncdesc}{int}{PyImport_ImportFrozenModule}{char *name}
  Load a frozen module named \var{name}.  Return \code{1} for success,
  \code{0} if the module is not found, and \code{-1} with an exception
  set if the initialization failed.  To access the imported module on
  a successful load, use \cfunction{PyImport_ImportModule()}.  (Note
  the misnomer --- this function would reload the module if it was
  already imported.)
\end{cfuncdesc}

\begin{ctypedesc}[_frozen]{struct _frozen}
  This is the structure type definition for frozen module descriptors,
  as generated by the \program{freeze}\index{freeze utility} utility
  (see \file{Tools/freeze/} in the Python source distribution).  Its
  definition, found in \file{Include/import.h}, is:

\begin{verbatim}
struct _frozen {
    char *name;
    unsigned char *code;
    int size;
};
\end{verbatim}
\end{ctypedesc}

\begin{cvardesc}{struct _frozen*}{PyImport_FrozenModules}
  This pointer is initialized to point to an array of \ctype{struct
  _frozen} records, terminated by one whose members are all \NULL{} or
  zero.  When a frozen module is imported, it is searched in this
  table.  Third-party code could play tricks with this to provide a
  dynamically created collection of frozen modules.
\end{cvardesc}

\begin{cfuncdesc}{int}{PyImport_AppendInittab}{char *name,
                                               void (*initfunc)(void)}
  Add a single module to the existing table of built-in modules.  This
  is a convenience wrapper around
  \cfunction{PyImport_ExtendInittab()}, returning \code{-1} if the
  table could not be extended.  The new module can be imported by the
  name \var{name}, and uses the function \var{initfunc} as the
  initialization function called on the first attempted import.  This
  should be called before \cfunction{Py_Initialize()}.
\end{cfuncdesc}

\begin{ctypedesc}[_inittab]{struct _inittab}
  Structure describing a single entry in the list of built-in
  modules.  Each of these structures gives the name and initialization
  function for a module built into the interpreter.  Programs which
  embed Python may use an array of these structures in conjunction
  with \cfunction{PyImport_ExtendInittab()} to provide additional
  built-in modules.  The structure is defined in
  \file{Include/import.h} as:

\begin{verbatim}
struct _inittab {
    char *name;
    void (*initfunc)(void);
};
\end{verbatim}
\end{ctypedesc}

\begin{cfuncdesc}{int}{PyImport_ExtendInittab}{struct _inittab *newtab}
  Add a collection of modules to the table of built-in modules.  The
  \var{newtab} array must end with a sentinel entry which contains
  \NULL{} for the \member{name} field; failure to provide the sentinel
  value can result in a memory fault.  Returns \code{0} on success or
  \code{-1} if insufficient memory could be allocated to extend the
  internal table.  In the event of failure, no modules are added to
  the internal table.  This should be called before
  \cfunction{Py_Initialize()}.
\end{cfuncdesc}


\section{Data marshalling support \label{marshalling-utils}}

These routines allow C code to work with serialized objects using the
same data format as the \module{marshal} module.  There are functions
to write data into the serialization format, and additional functions
that can be used to read the data back.  Files used to store marshalled
data must be opened in binary mode.

Numeric values are stored with the least significant byte first.

\begin{cfuncdesc}{void}{PyMarshal_WriteLongToFile}{long value, FILE *file}
  Marshal a \ctype{long} integer, \var{value}, to \var{file}.  This
  will only write the least-significant 32 bits of \var{value};
  regardless of the size of the native \ctype{long} type.
\end{cfuncdesc}

\begin{cfuncdesc}{void}{PyMarshal_WriteShortToFile}{short value, FILE *file}
  Marshal a \ctype{short} integer, \var{value}, to \var{file}.  This
  will only write the least-significant 16 bits of \var{value};
  regardless of the size of the native \ctype{short} type.
\end{cfuncdesc}

\begin{cfuncdesc}{void}{PyMarshal_WriteObjectToFile}{PyObject *value,
                                                     FILE *file}
  Marshal a Python object, \var{value}, to \var{file}.
\end{cfuncdesc}

\begin{cfuncdesc}{PyObject*}{PyMarshal_WriteObjectToString}{PyObject *value}
  Return a string object containing the marshalled representation of
  \var{value}.
\end{cfuncdesc}

The following functions allow marshalled values to be read back in.

XXX What about error detection?  It appears that reading past the end
of the file will always result in a negative numeric value (where
that's relevant), but it's not clear that negative values won't be
handled properly when there's no error.  What's the right way to tell?
Should only non-negative values be written using these routines?

\begin{cfuncdesc}{long}{PyMarshal_ReadLongFromFile}{FILE *file}
  Return a C \ctype{long} from the data stream in a \ctype{FILE*}
  opened for reading.  Only a 32-bit value can be read in using
  this function, regardless of the native size of \ctype{long}.
\end{cfuncdesc}

\begin{cfuncdesc}{int}{PyMarshal_ReadShortFromFile}{FILE *file}
  Return a C \ctype{short} from the data stream in a \ctype{FILE*}
  opened for reading.  Only a 16-bit value can be read in using
  this function, regardless of the native size of \ctype{short}.
\end{cfuncdesc}

\begin{cfuncdesc}{PyObject*}{PyMarshal_ReadObjectFromFile}{FILE *file}
  Return a Python object from the data stream in a \ctype{FILE*}
  opened for reading.  On error, sets the appropriate exception
  (\exception{EOFError} or \exception{TypeError}) and returns \NULL.
\end{cfuncdesc}

\begin{cfuncdesc}{PyObject*}{PyMarshal_ReadLastObjectFromFile}{FILE *file}
  Return a Python object from the data stream in a \ctype{FILE*}
  opened for reading.  Unlike
  \cfunction{PyMarshal_ReadObjectFromFile()}, this function assumes
  that no further objects will be read from the file, allowing it to
  aggressively load file data into memory so that the de-serialization
  can operate from data in memory rather than reading a byte at a time
  from the file.  Only use these variant if you are certain that you
  won't be reading anything else from the file.  On error, sets the
  appropriate exception (\exception{EOFError} or
  \exception{TypeError}) and returns \NULL.
\end{cfuncdesc}

\begin{cfuncdesc}{PyObject*}{PyMarshal_ReadObjectFromString}{char *string,
                                                             int len}
  Return a Python object from the data stream in a character buffer
  containing \var{len} bytes pointed to by \var{string}.  On error,
  sets the appropriate exception (\exception{EOFError} or
  \exception{TypeError}) and returns \NULL.
\end{cfuncdesc}


\section{Parsing arguments and building values
         \label{arg-parsing}}

These functions are useful when creating your own extensions functions
and methods.  Additional information and examples are available in
\citetitle[../ext/ext.html]{Extending and Embedding the Python
Interpreter}.

The first three of these functions described,
\cfunction{PyArg_ParseTuple()},
\cfunction{PyArg_ParseTupleAndKeywords()}, and
\cfunction{PyArg_Parse()}, all use \emph{format strings} which are
used to tell the function about the expected arguments.  The format
strings use the same syntax for each of these functions.

A format string consists of zero or more ``format units.''  A format
unit describes one Python object; it is usually a single character or
a parenthesized sequence of format units.  With a few exceptions, a
format unit that is not a parenthesized sequence normally corresponds
to a single address argument to these functions.  In the following
description, the quoted form is the format unit; the entry in (round)
parentheses is the Python object type that matches the format unit;
and the entry in [square] brackets is the type of the C variable(s)
whose address should be passed.

\begin{description}
  \item[\samp{s} (string or Unicode object) {[char *]}]
  Convert a Python string or Unicode object to a C pointer to a
  character string.  You must not provide storage for the string
  itself; a pointer to an existing string is stored into the character
  pointer variable whose address you pass.  The C string is
  NUL-terminated.  The Python string must not contain embedded NUL
  bytes; if it does, a \exception{TypeError} exception is raised.
  Unicode objects are converted to C strings using the default
  encoding.  If this conversion fails, a \exception{UnicodeError} is
  raised.

  \item[\samp{s\#} (string, Unicode or any read buffer compatible object)
  {[char *, int]}]
  This variant on \samp{s} stores into two C variables, the first one
  a pointer to a character string, the second one its length.  In this
  case the Python string may contain embedded null bytes.  Unicode
  objects pass back a pointer to the default encoded string version of
  the object if such a conversion is possible.  All other read-buffer
  compatible objects pass back a reference to the raw internal data
  representation.

  \item[\samp{z} (string or \code{None}) {[char *]}]
  Like \samp{s}, but the Python object may also be \code{None}, in
  which case the C pointer is set to \NULL.

  \item[\samp{z\#} (string or \code{None} or any read buffer
  compatible object) {[char *, int]}]
  This is to \samp{s\#} as \samp{z} is to \samp{s}.

  \item[\samp{u} (Unicode object) {[Py_UNICODE *]}]
  Convert a Python Unicode object to a C pointer to a NUL-terminated
  buffer of 16-bit Unicode (UTF-16) data.  As with \samp{s}, there is
  no need to provide storage for the Unicode data buffer; a pointer to
  the existing Unicode data is stored into the \ctype{Py_UNICODE}
  pointer variable whose address you pass.

  \item[\samp{u\#} (Unicode object) {[Py_UNICODE *, int]}]
  This variant on \samp{u} stores into two C variables, the first one
  a pointer to a Unicode data buffer, the second one its length.
  Non-Unicode objects are handled by interpreting their read-buffer
  pointer as pointer to a \ctype{Py_UNICODE} array.

  \item[\samp{es} (string, Unicode object or character buffer
  compatible object) {[const char *encoding, char **buffer]}]
  This variant on \samp{s} is used for encoding Unicode and objects
  convertible to Unicode into a character buffer. It only works for
  encoded data without embedded NUL bytes.

  This format requires two arguments.  The first is only used as
  input, and must be a \ctype{char*} which points to the name of an
  encoding as a NUL-terminated string, or \NULL, in which case the
  default encoding is used.  An exception is raised if the named
  encoding is not known to Python.  The second argument must be a
  \ctype{char**}; the value of the pointer it references will be set
  to a buffer with the contents of the argument text.  The text will
  be encoded in the encoding specified by the first argument.

  \cfunction{PyArg_ParseTuple()} will allocate a buffer of the needed
  size, copy the encoded data into this buffer and adjust
  \var{*buffer} to reference the newly allocated storage.  The caller
  is responsible for calling \cfunction{PyMem_Free()} to free the
  allocated buffer after use.

  \item[\samp{et} (string, Unicode object or character buffer
  compatible object) {[const char *encoding, char **buffer]}]
  Same as \samp{es} except that 8-bit string objects are passed
  through without recoding them.  Instead, the implementation assumes
  that the string object uses the encoding passed in as parameter.

  \item[\samp{es\#} (string, Unicode object or character buffer compatible
  object) {[const char *encoding, char **buffer, int *buffer_length]}]
  This variant on \samp{s\#} is used for encoding Unicode and objects
  convertible to Unicode into a character buffer.  Unlike the
  \samp{es} format, this variant allows input data which contains NUL
  characters.

  It requires three arguments.  The first is only used as input, and
  must be a \ctype{char*} which points to the name of an encoding as a
  NUL-terminated string, or \NULL, in which case the default encoding
  is used.  An exception is raised if the named encoding is not known
  to Python.  The second argument must be a \ctype{char**}; the value
  of the pointer it references will be set to a buffer with the
  contents of the argument text.  The text will be encoded in the
  encoding specified by the first argument.  The third argument must
  be a pointer to an integer; the referenced integer will be set to
  the number of bytes in the output buffer.

  There are two modes of operation:

  If \var{*buffer} points a \NULL{} pointer, the function will
  allocate a buffer of the needed size, copy the encoded data into
  this buffer and set \var{*buffer} to reference the newly allocated
  storage.  The caller is responsible for calling
  \cfunction{PyMem_Free()} to free the allocated buffer after usage.

  If \var{*buffer} points to a non-\NULL{} pointer (an already
  allocated buffer), \cfunction{PyArg_ParseTuple()} will use this
  location as the buffer and interpret the initial value of
  \var{*buffer_length} as the buffer size.  It will then copy the
  encoded data into the buffer and NUL-terminate it.  If the buffer
  is not large enough, a \exception{ValueError} will be set.

  In both cases, \var{*buffer_length} is set to the length of the
  encoded data without the trailing NUL byte.

  \item[\samp{et\#} (string, Unicode object or character buffer compatible
  object) {[const char *encoding, char **buffer]}]
  Same as \samp{es\#} except that string objects are passed through
  without recoding them. Instead, the implementation assumes that the
  string object uses the encoding passed in as parameter.

  \item[\samp{b} (integer) {[char]}]
  Convert a Python integer to a tiny int, stored in a C \ctype{char}.

  \item[\samp{h} (integer) {[short int]}]
  Convert a Python integer to a C \ctype{short int}.

  \item[\samp{i} (integer) {[int]}]
  Convert a Python integer to a plain C \ctype{int}.

  \item[\samp{l} (integer) {[long int]}]
  Convert a Python integer to a C \ctype{long int}.

  \item[\samp{L} (integer) {[LONG_LONG]}]
  Convert a Python integer to a C \ctype{long long}.  This format is
  only available on platforms that support \ctype{long long} (or
  \ctype{_int64} on Windows).

  \item[\samp{c} (string of length 1) {[char]}]
  Convert a Python character, represented as a string of length 1, to
  a C \ctype{char}.

  \item[\samp{f} (float) {[float]}]
  Convert a Python floating point number to a C \ctype{float}.

  \item[\samp{d} (float) {[double]}]
  Convert a Python floating point number to a C \ctype{double}.

  \item[\samp{D} (complex) {[Py_complex]}]
  Convert a Python complex number to a C \ctype{Py_complex} structure.

  \item[\samp{O} (object) {[PyObject *]}]
  Store a Python object (without any conversion) in a C object
  pointer.  The C program thus receives the actual object that was
  passed.  The object's reference count is not increased.  The pointer
  stored is not \NULL.

  \item[\samp{O!} (object) {[\var{typeobject}, PyObject *]}]
  Store a Python object in a C object pointer.  This is similar to
  \samp{O}, but takes two C arguments: the first is the address of a
  Python type object, the second is the address of the C variable (of
  type \ctype{PyObject*}) into which the object pointer is stored.  If
  the Python object does not have the required type,
  \exception{TypeError} is raised.

  \item[\samp{O\&} (object) {[\var{converter}, \var{anything}]}]
  Convert a Python object to a C variable through a \var{converter}
  function.  This takes two arguments: the first is a function, the
  second is the address of a C variable (of arbitrary type), converted
  to \ctype{void *}.  The \var{converter} function in turn is called
  as follows:

  \var{status}\code{ = }\var{converter}\code{(}\var{object},
  \var{address}\code{);}

  where \var{object} is the Python object to be converted and
  \var{address} is the \ctype{void*} argument that was passed to the
  \cfunction{PyArg_Parse*()} function.  The returned \var{status}
  should be \code{1} for a successful conversion and \code{0} if the
  conversion has failed.  When the conversion fails, the
  \var{converter} function should raise an exception.

  \item[\samp{S} (string) {[PyStringObject *]}]
  Like \samp{O} but requires that the Python object is a string
  object.  Raises \exception{TypeError} if the object is not a string
  object.  The C variable may also be declared as \ctype{PyObject*}.

  \item[\samp{U} (Unicode string) {[PyUnicodeObject *]}]
  Like \samp{O} but requires that the Python object is a Unicode
  object.  Raises \exception{TypeError} if the object is not a Unicode
  object.  The C variable may also be declared as \ctype{PyObject*}.

  \item[\samp{t\#} (read-only character buffer) {[char *, int]}]
  Like \samp{s\#}, but accepts any object which implements the
  read-only buffer interface.  The \ctype{char*} variable is set to
  point to the first byte of the buffer, and the \ctype{int} is set to
  the length of the buffer.  Only single-segment buffer objects are
  accepted; \exception{TypeError} is raised for all others.

  \item[\samp{w} (read-write character buffer) {[char *]}]
  Similar to \samp{s}, but accepts any object which implements the
  read-write buffer interface.  The caller must determine the length
  of the buffer by other means, or use \samp{w\#} instead.  Only
  single-segment buffer objects are accepted; \exception{TypeError} is
  raised for all others.

  \item[\samp{w\#} (read-write character buffer) {[char *, int]}]
  Like \samp{s\#}, but accepts any object which implements the
  read-write buffer interface.  The \ctype{char *} variable is set to
  point to the first byte of the buffer, and the \ctype{int} is set to
  the length of the buffer.  Only single-segment buffer objects are
  accepted; \exception{TypeError} is raised for all others.

  \item[\samp{(\var{items})} (tuple) {[\var{matching-items}]}]
  The object must be a Python sequence whose length is the number of
  format units in \var{items}.  The C arguments must correspond to the
  individual format units in \var{items}.  Format units for sequences
  may be nested.

  \note{Prior to Python version 1.5.2, this format specifier only
  accepted a tuple containing the individual parameters, not an
  arbitrary sequence.  Code which previously caused
  \exception{TypeError} to be raised here may now proceed without an
  exception.  This is not expected to be a problem for existing code.}
\end{description}

It is possible to pass Python long integers where integers are
requested; however no proper range checking is done --- the most
significant bits are silently truncated when the receiving field is
too small to receive the value (actually, the semantics are inherited
from downcasts in C --- your mileage may vary).

A few other characters have a meaning in a format string.  These may
not occur inside nested parentheses.  They are:

\begin{description}
  \item[\samp{|}]
  Indicates that the remaining arguments in the Python argument list
  are optional.  The C variables corresponding to optional arguments
  should be initialized to their default value --- when an optional
  argument is not specified, \cfunction{PyArg_ParseTuple()} does not
  touch the contents of the corresponding C variable(s).

  \item[\samp{:}]
  The list of format units ends here; the string after the colon is
  used as the function name in error messages (the ``associated
  value'' of the exception that \cfunction{PyArg_ParseTuple()}
  raises).

  \item[\samp{;}]
  The list of format units ends here; the string after the semicolon
  is used as the error message \emph{instead} of the default error
  message.  Clearly, \samp{:} and \samp{;} mutually exclude each
  other.
\end{description}

Note that any Python object references which are provided to the
caller are \emph{borrowed} references; do not decrement their
reference count!

Additional arguments passed to these functions must be addresses of
variables whose type is determined by the format string; these are
used to store values from the input tuple.  There are a few cases, as
described in the list of format units above, where these parameters
are used as input values; they should match what is specified for the
corresponding format unit in that case.

For the conversion to succeed, the \var{arg} object must match the
format and the format must be exhausted.  On success, the
\cfunction{PyArg_Parse*()} functions return true, otherwise they
return false and raise an appropriate exception.

\begin{cfuncdesc}{int}{PyArg_ParseTuple}{PyObject *args, char *format,
                                         \moreargs}
  Parse the parameters of a function that takes only positional
  parameters into local variables.  Returns true on success; on
  failure, it returns false and raises the appropriate exception.
\end{cfuncdesc}

\begin{cfuncdesc}{int}{PyArg_ParseTupleAndKeywords}{PyObject *args,
                       PyObject *kw, char *format, char *keywords[],
                       \moreargs}
  Parse the parameters of a function that takes both positional and
  keyword parameters into local variables.  Returns true on success;
  on failure, it returns false and raises the appropriate exception.
\end{cfuncdesc}

\begin{cfuncdesc}{int}{PyArg_Parse}{PyObject *args, char *format,
                                    \moreargs}
  Function used to deconstruct the argument lists of ``old-style''
  functions --- these are functions which use the
  \constant{METH_OLDARGS} parameter parsing method.  This is not
  recommended for use in parameter parsing in new code, and most code
  in the standard interpreter has been modified to no longer use this
  for that purpose.  It does remain a convenient way to decompose
  other tuples, however, and may continue to be used for that
  purpose.
\end{cfuncdesc}

\begin{cfuncdesc}{int}{PyArg_UnpackTuple}{PyObject *args, char *name,
                                          int min, int max, \moreargs}
  A simpler form of parameter retrieval which does not use a format
  string to specify the types of the arguments.  Functions which use
  this method to retrieve their parameters should be declared as
  \constant{METH_VARARGS} in function or method tables.  The tuple
  containing the actual parameters should be passed as \var{args}; it
  must actually be a tuple.  The length of the tuple must be at least
  \var{min} and no more than \var{max}; \var{min} and \var{max} may be
  equal.  Additional arguments must be passed to the function, each of
  which should be a pointer to a \ctype{PyObject*} variable; these
  will be filled in with the values from \var{args}; they will contain
  borrowed references.  The variables which correspond to optional
  parameters not given by \var{args} will not be filled in; these
  should be initialized by the caller.
  This function returns true on success and false if \var{args} is not
  a tuple or contains the wrong number of elements; an exception will
  be set if there was a failure.

  This is an example of the use of this function, taken from the
  sources for the \module{_weakref} helper module for weak references:

\begin{verbatim}
static PyObject *
weakref_ref(PyObject *self, PyObject *args)
{
    PyObject *object;
    PyObject *callback = NULL;
    PyObject *result = NULL;

    if (PyArg_UnpackTuple(args, "ref", 1, 2, &object, &callback)) {
        result = PyWeakref_NewRef(object, callback);
    }
    return result;
}
\end{verbatim}

  The call to \cfunction{PyArg_UnpackTuple()} in this example is
  entirely equivalent to this call to \cfunction{PyArg_ParseTuple()}:

\begin{verbatim}
PyArg_ParseTuple(args, "O|O:ref", &object, &callback)
\end{verbatim}

  \versionadded{2.2}
\end{cfuncdesc}

\begin{cfuncdesc}{PyObject*}{Py_BuildValue}{char *format,
                                            \moreargs}
  Create a new value based on a format string similar to those
  accepted by the \cfunction{PyArg_Parse*()} family of functions and a
  sequence of values.  Returns the value or \NULL{} in the case of an
  error; an exception will be raised if \NULL{} is returned.

  \cfunction{Py_BuildValue()} does not always build a tuple.  It
  builds a tuple only if its format string contains two or more format
  units.  If the format string is empty, it returns \code{None}; if it
  contains exactly one format unit, it returns whatever object is
  described by that format unit.  To force it to return a tuple of
  size 0 or one, parenthesize the format string.

  When memory buffers are passed as parameters to supply data to build
  objects, as for the \samp{s} and \samp{s\#} formats, the required
  data is copied.  Buffers provided by the caller are never referenced
  by the objects created by \cfunction{Py_BuildValue()}.  In other
  words, if your code invokes \cfunction{malloc()} and passes the
  allocated memory to \cfunction{Py_BuildValue()}, your code is
  responsible for calling \cfunction{free()} for that memory once
  \cfunction{Py_BuildValue()} returns.

  In the following description, the quoted form is the format unit;
  the entry in (round) parentheses is the Python object type that the
  format unit will return; and the entry in [square] brackets is the
  type of the C value(s) to be passed.

  The characters space, tab, colon and comma are ignored in format
  strings (but not within format units such as \samp{s\#}).  This can
  be used to make long format strings a tad more readable.

  \begin{description}
    \item[\samp{s} (string) {[char *]}]
    Convert a null-terminated C string to a Python object.  If the C
    string pointer is \NULL, \code{None} is used.

    \item[\samp{s\#} (string) {[char *, int]}]
    Convert a C string and its length to a Python object.  If the C
    string pointer is \NULL, the length is ignored and \code{None} is
    returned.

    \item[\samp{z} (string or \code{None}) {[char *]}]
    Same as \samp{s}.

    \item[\samp{z\#} (string or \code{None}) {[char *, int]}]
    Same as \samp{s\#}.

    \item[\samp{u} (Unicode string) {[Py_UNICODE *]}]
    Convert a null-terminated buffer of Unicode (UCS-2) data to a
    Python Unicode object.  If the Unicode buffer pointer is \NULL,
    \code{None} is returned.

    \item[\samp{u\#} (Unicode string) {[Py_UNICODE *, int]}]
    Convert a Unicode (UCS-2) data buffer and its length to a Python
    Unicode object.   If the Unicode buffer pointer is \NULL, the
    length is ignored and \code{None} is returned.

    \item[\samp{i} (integer) {[int]}]
    Convert a plain C \ctype{int} to a Python integer object.

    \item[\samp{b} (integer) {[char]}]
    Same as \samp{i}.

    \item[\samp{h} (integer) {[short int]}]
    Same as \samp{i}.

    \item[\samp{l} (integer) {[long int]}]
    Convert a C \ctype{long int} to a Python integer object.

    \item[\samp{c} (string of length 1) {[char]}]
    Convert a C \ctype{int} representing a character to a Python
    string of length 1.

    \item[\samp{d} (float) {[double]}]
    Convert a C \ctype{double} to a Python floating point number.

    \item[\samp{f} (float) {[float]}]
    Same as \samp{d}.

    \item[\samp{D} (complex) {[Py_complex *]}]
    Convert a C \ctype{Py_complex} structure to a Python complex
    number.

    \item[\samp{O} (object) {[PyObject *]}]
    Pass a Python object untouched (except for its reference count,
    which is incremented by one).  If the object passed in is a
    \NULL{} pointer, it is assumed that this was caused because the
    call producing the argument found an error and set an exception.
    Therefore, \cfunction{Py_BuildValue()} will return \NULL{} but
    won't raise an exception.  If no exception has been raised yet,
    \exception{SystemError} is set.

    \item[\samp{S} (object) {[PyObject *]}]
    Same as \samp{O}.

    \item[\samp{U} (object) {[PyObject *]}]
    Same as \samp{O}.

    \item[\samp{N} (object) {[PyObject *]}]
    Same as \samp{O}, except it doesn't increment the reference count
    on the object.  Useful when the object is created by a call to an
    object constructor in the argument list.

    \item[\samp{O\&} (object) {[\var{converter}, \var{anything}]}]
    Convert \var{anything} to a Python object through a
    \var{converter} function.  The function is called with
    \var{anything} (which should be compatible with \ctype{void *}) as
    its argument and should return a ``new'' Python object, or \NULL{}
    if an error occurred.

    \item[\samp{(\var{items})} (tuple) {[\var{matching-items}]}]
    Convert a sequence of C values to a Python tuple with the same
    number of items.

    \item[\samp{[\var{items}]} (list) {[\var{matching-items}]}]
    Convert a sequence of C values to a Python list with the same
    number of items.

    \item[\samp{\{\var{items}\}} (dictionary) {[\var{matching-items}]}]
    Convert a sequence of C values to a Python dictionary.  Each pair
    of consecutive C values adds one item to the dictionary, serving
    as key and value, respectively.

  \end{description}

  If there is an error in the format string, the
  \exception{SystemError} exception is set and \NULL{} returned.
\end{cfuncdesc}
