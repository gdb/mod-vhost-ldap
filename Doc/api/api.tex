\documentclass{manual}

\title{Python/C API Reference Manual}

\author{Guido van Rossum\\
	Fred L. Drake, Jr., editor}
\authoraddress{
	\strong{Python Software Foundation}\\
	Email: \email{docs@python.org}
}

\date{20 June, 2006}			% XXX update before final release!
\input{patchlevel}		% include Python version information


\makeindex			% tell \index to actually write the .idx file


\begin{document}

\maketitle

\ifhtml
\chapter*{Front Matter\label{front}}
\fi

\leftline{Copyright \copyright{} 2000, BeOpen.com.}
\leftline{Copyright \copyright{} 1995-2000, Corporation for National Research Initiatives.}
\leftline{Copyright \copyright{} 1990-1995, Stichting Mathematisch Centrum.}
\leftline{All rights reserved.}

Redistribution and use in source and binary forms, with or without
modification, are permitted provided that the following conditions are
met:

\begin{itemize}
\item
Redistributions of source code must retain the above copyright
notice, this list of conditions and the following disclaimer.

\item
Redistributions in binary form must reproduce the above copyright
notice, this list of conditions and the following disclaimer in the
documentation and/or other materials provided with the distribution.

\item
Neither names of the copyright holders nor the names of their
contributors may be used to endorse or promote products derived from
this software without specific prior written permission.
\end{itemize}

THIS SOFTWARE IS PROVIDED BY THE COPYRIGHT HOLDERS AND CONTRIBUTORS
``AS IS'' AND ANY EXPRESS OR IMPLIED WARRANTIES, INCLUDING, BUT NOT
LIMITED TO, THE IMPLIED WARRANTIES OF MERCHANTABILITY AND FITNESS FOR
A PARTICULAR PURPOSE ARE DISCLAIMED.  IN NO EVENT SHALL THE COPYRIGHT
HOLDERS OR CONTRIBUTORS BE LIABLE FOR ANY DIRECT, INDIRECT,
INCIDENTAL, SPECIAL, EXEMPLARY, OR CONSEQUENTIAL DAMAGES (INCLUDING,
BUT NOT LIMITED TO, PROCUREMENT OF SUBSTITUTE GOODS OR SERVICES; LOSS
OF USE, DATA, OR PROFITS; OR BUSINESS INTERRUPTION) HOWEVER CAUSED AND
ON ANY THEORY OF LIABILITY, WHETHER IN CONTRACT, STRICT LIABILITY, OR
TORT (INCLUDING NEGLIGENCE OR OTHERWISE) ARISING IN ANY WAY OUT OF THE
USE OF THIS SOFTWARE, EVEN IF ADVISED OF THE POSSIBILITY OF SUCH
DAMAGE.


\begin{abstract}

\noindent
This manual documents the API used by C and \Cpp{} programmers who
want to write extension modules or embed Python.  It is a companion to
\citetitle[../ext/ext.html]{Extending and Embedding the Python
Interpreter}, which describes the general principles of extension
writing but does not document the API functions in detail.

\strong{Warning:} The current version of this document is incomplete.
I hope that it is nevertheless useful.  I will continue to work on it,
and release new versions from time to time, independent from Python
source code releases.

\end{abstract}

\tableofcontents

% XXX Consider moving all this back to ext.tex and giving api.tex
% XXX a *really* short intro only.

\chapter{Introduction \label{intro}}

The Application Programmer's Interface to Python gives C and
\Cpp{} programmers access to the Python interpreter at a variety of
levels.  The API is equally usable from \Cpp{}, but for brevity it is
generally referred to as the Python/C API.  There are two
fundamentally different reasons for using the Python/C API.  The first
reason is to write \emph{extension modules} for specific purposes;
these are C modules that extend the Python interpreter.  This is
probably the most common use.  The second reason is to use Python as a
component in a larger application; this technique is generally
referred to as \dfn{embedding} Python in an application.

Writing an extension module is a relatively well-understood process, 
where a ``cookbook'' approach works well.  There are several tools 
that automate the process to some extent.  While people have embedded 
Python in other applications since its early existence, the process of 
embedding Python is less straightforward than writing an extension.  

Many API functions are useful independent of whether you're embedding 
or extending Python; moreover, most applications that embed Python 
will need to provide a custom extension as well, so it's probably a 
good idea to become familiar with writing an extension before 
attempting to embed Python in a real application.


\section{Include Files \label{includes}}

All function, type and macro definitions needed to use the Python/C
API are included in your code by the following line:

\begin{verbatim}
#include "Python.h"
\end{verbatim}

This implies inclusion of the following standard headers:
\code{<stdio.h>}, \code{<string.h>}, \code{<errno.h>},
\code{<limits.h>}, and \code{<stdlib.h>} (if available).
Since Python may define some pre-processor definitions which affect
the standard headers on some systems, you must include \file{Python.h}
before any standard headers are included.

All user visible names defined by Python.h (except those defined by
the included standard headers) have one of the prefixes \samp{Py} or
\samp{_Py}.  Names beginning with \samp{_Py} are for internal use by
the Python implementation and should not be used by extension writers.
Structure member names do not have a reserved prefix.

\strong{Important:} user code should never define names that begin
with \samp{Py} or \samp{_Py}.  This confuses the reader, and
jeopardizes the portability of the user code to future Python
versions, which may define additional names beginning with one of
these prefixes.

The header files are typically installed with Python.  On \UNIX, these 
are located in the directories
\file{\envvar{prefix}/include/python\var{version}/} and
\file{\envvar{exec_prefix}/include/python\var{version}/}, where
\envvar{prefix} and \envvar{exec_prefix} are defined by the
corresponding parameters to Python's \program{configure} script and
\var{version} is \code{sys.version[:3]}.  On Windows, the headers are
installed in \file{\envvar{prefix}/include}, where \envvar{prefix} is
the installation directory specified to the installer.

To include the headers, place both directories (if different) on your
compiler's search path for includes.  Do \emph{not} place the parent
directories on the search path and then use
\samp{\#include <python\shortversion/Python.h>}; this will break on
multi-platform builds since the platform independent headers under
\envvar{prefix} include the platform specific headers from
\envvar{exec_prefix}.

\Cpp{} users should note that though the API is defined entirely using
C, the header files do properly declare the entry points to be
\code{extern "C"}, so there is no need to do anything special to use
the API from \Cpp.


\section{Objects, Types and Reference Counts \label{objects}}

Most Python/C API functions have one or more arguments as well as a
return value of type \ctype{PyObject*}.  This type is a pointer
to an opaque data type representing an arbitrary Python
object.  Since all Python object types are treated the same way by the
Python language in most situations (e.g., assignments, scope rules,
and argument passing), it is only fitting that they should be
represented by a single C type.  Almost all Python objects live on the
heap: you never declare an automatic or static variable of type
\ctype{PyObject}, only pointer variables of type \ctype{PyObject*} can 
be declared.  The sole exception are the type objects\obindex{type};
since these must never be deallocated, they are typically static
\ctype{PyTypeObject} objects.

All Python objects (even Python integers) have a \dfn{type} and a
\dfn{reference count}.  An object's type determines what kind of object 
it is (e.g., an integer, a list, or a user-defined function; there are 
many more as explained in the \citetitle[../ref/ref.html]{Python
Reference Manual}).  For each of the well-known types there is a macro
to check whether an object is of that type; for instance,
\samp{PyList_Check(\var{a})} is true if (and only if) the object
pointed to by \var{a} is a Python list.


\subsection{Reference Counts \label{refcounts}}

The reference count is important because today's computers have a 
finite (and often severely limited) memory size; it counts how many 
different places there are that have a reference to an object.  Such a 
place could be another object, or a global (or static) C variable, or 
a local variable in some C function.  When an object's reference count 
becomes zero, the object is deallocated.  If it contains references to 
other objects, their reference count is decremented.  Those other 
objects may be deallocated in turn, if this decrement makes their 
reference count become zero, and so on.  (There's an obvious problem 
with objects that reference each other here; for now, the solution is 
``don't do that.'')

Reference counts are always manipulated explicitly.  The normal way is 
to use the macro \cfunction{Py_INCREF()}\ttindex{Py_INCREF()} to
increment an object's reference count by one, and
\cfunction{Py_DECREF()}\ttindex{Py_DECREF()} to decrement it by  
one.  The \cfunction{Py_DECREF()} macro is considerably more complex
than the incref one, since it must check whether the reference count
becomes zero and then cause the object's deallocator to be called.
The deallocator is a function pointer contained in the object's type
structure.  The type-specific deallocator takes care of decrementing
the reference counts for other objects contained in the object if this
is a compound object type, such as a list, as well as performing any
additional finalization that's needed.  There's no chance that the
reference count can overflow; at least as many bits are used to hold
the reference count as there are distinct memory locations in virtual
memory (assuming \code{sizeof(long) >= sizeof(char*)}).  Thus, the
reference count increment is a simple operation.

It is not necessary to increment an object's reference count for every 
local variable that contains a pointer to an object.  In theory, the 
object's reference count goes up by one when the variable is made to 
point to it and it goes down by one when the variable goes out of 
scope.  However, these two cancel each other out, so at the end the 
reference count hasn't changed.  The only real reason to use the 
reference count is to prevent the object from being deallocated as 
long as our variable is pointing to it.  If we know that there is at 
least one other reference to the object that lives at least as long as 
our variable, there is no need to increment the reference count 
temporarily.  An important situation where this arises is in objects 
that are passed as arguments to C functions in an extension module 
that are called from Python; the call mechanism guarantees to hold a 
reference to every argument for the duration of the call.

However, a common pitfall is to extract an object from a list and
hold on to it for a while without incrementing its reference count.
Some other operation might conceivably remove the object from the
list, decrementing its reference count and possible deallocating it.
The real danger is that innocent-looking operations may invoke
arbitrary Python code which could do this; there is a code path which
allows control to flow back to the user from a \cfunction{Py_DECREF()},
so almost any operation is potentially dangerous.

A safe approach is to always use the generic operations (functions 
whose name begins with \samp{PyObject_}, \samp{PyNumber_},
\samp{PySequence_} or \samp{PyMapping_}).  These operations always
increment the reference count of the object they return.  This leaves
the caller with the responsibility to call
\cfunction{Py_DECREF()} when they are done with the result; this soon
becomes second nature.


\subsubsection{Reference Count Details \label{refcountDetails}}

The reference count behavior of functions in the Python/C API is best 
explained in terms of \emph{ownership of references}.  Note that we 
talk of owning references, never of owning objects; objects are always 
shared!  When a function owns a reference, it has to dispose of it 
properly --- either by passing ownership on (usually to its caller) or 
by calling \cfunction{Py_DECREF()} or \cfunction{Py_XDECREF()}.  When
a function passes ownership of a reference on to its caller, the
caller is said to receive a \emph{new} reference.  When no ownership
is transferred, the caller is said to \emph{borrow} the reference.
Nothing needs to be done for a borrowed reference.

Conversely, when a calling function passes it a reference to an 
object, there are two possibilities: the function \emph{steals} a 
reference to the object, or it does not.  Few functions steal 
references; the two notable exceptions are
\cfunction{PyList_SetItem()}\ttindex{PyList_SetItem()} and
\cfunction{PyTuple_SetItem()}\ttindex{PyTuple_SetItem()}, which 
steal a reference to the item (but not to the tuple or list into which
the item is put!).  These functions were designed to steal a reference
because of a common idiom for populating a tuple or list with newly
created objects; for example, the code to create the tuple \code{(1,
2, "three")} could look like this (forgetting about error handling for
the moment; a better way to code this is shown below):

\begin{verbatim}
PyObject *t;

t = PyTuple_New(3);
PyTuple_SetItem(t, 0, PyInt_FromLong(1L));
PyTuple_SetItem(t, 1, PyInt_FromLong(2L));
PyTuple_SetItem(t, 2, PyString_FromString("three"));
\end{verbatim}

Incidentally, \cfunction{PyTuple_SetItem()} is the \emph{only} way to
set tuple items; \cfunction{PySequence_SetItem()} and
\cfunction{PyObject_SetItem()} refuse to do this since tuples are an
immutable data type.  You should only use
\cfunction{PyTuple_SetItem()} for tuples that you are creating
yourself.

Equivalent code for populating a list can be written using 
\cfunction{PyList_New()} and \cfunction{PyList_SetItem()}.  Such code
can also use \cfunction{PySequence_SetItem()}; this illustrates the
difference between the two (the extra \cfunction{Py_DECREF()} calls):

\begin{verbatim}
PyObject *l, *x;

l = PyList_New(3);
x = PyInt_FromLong(1L);
PySequence_SetItem(l, 0, x); Py_DECREF(x);
x = PyInt_FromLong(2L);
PySequence_SetItem(l, 1, x); Py_DECREF(x);
x = PyString_FromString("three");
PySequence_SetItem(l, 2, x); Py_DECREF(x);
\end{verbatim}

You might find it strange that the ``recommended'' approach takes more
code.  However, in practice, you will rarely use these ways of
creating and populating a tuple or list.  There's a generic function,
\cfunction{Py_BuildValue()}, that can create most common objects from
C values, directed by a \dfn{format string}.  For example, the
above two blocks of code could be replaced by the following (which
also takes care of the error checking):

\begin{verbatim}
PyObject *t, *l;

t = Py_BuildValue("(iis)", 1, 2, "three");
l = Py_BuildValue("[iis]", 1, 2, "three");
\end{verbatim}

It is much more common to use \cfunction{PyObject_SetItem()} and
friends with items whose references you are only borrowing, like
arguments that were passed in to the function you are writing.  In
that case, their behaviour regarding reference counts is much saner,
since you don't have to increment a reference count so you can give a
reference away (``have it be stolen'').  For example, this function
sets all items of a list (actually, any mutable sequence) to a given
item:

\begin{verbatim}
int set_all(PyObject *target, PyObject *item)
{
    int i, n;

    n = PyObject_Length(target);
    if (n < 0)
        return -1;
    for (i = 0; i < n; i++) {
        if (PyObject_SetItem(target, i, item) < 0)
            return -1;
    }
    return 0;
}
\end{verbatim}
\ttindex{set_all()}

The situation is slightly different for function return values.  
While passing a reference to most functions does not change your 
ownership responsibilities for that reference, many functions that 
return a referece to an object give you ownership of the reference.
The reason is simple: in many cases, the returned object is created 
on the fly, and the reference you get is the only reference to the 
object.  Therefore, the generic functions that return object 
references, like \cfunction{PyObject_GetItem()} and 
\cfunction{PySequence_GetItem()}, always return a new reference (the
caller becomes the owner of the reference).

It is important to realize that whether you own a reference returned 
by a function depends on which function you call only --- \emph{the
plumage} (the type of the type of the object passed as an
argument to the function) \emph{doesn't enter into it!}  Thus, if you 
extract an item from a list using \cfunction{PyList_GetItem()}, you
don't own the reference --- but if you obtain the same item from the
same list using \cfunction{PySequence_GetItem()} (which happens to
take exactly the same arguments), you do own a reference to the
returned object.

Here is an example of how you could write a function that computes the
sum of the items in a list of integers; once using 
\cfunction{PyList_GetItem()}\ttindex{PyList_GetItem()}, and once using
\cfunction{PySequence_GetItem()}\ttindex{PySequence_GetItem()}.

\begin{verbatim}
long sum_list(PyObject *list)
{
    int i, n;
    long total = 0;
    PyObject *item;

    n = PyList_Size(list);
    if (n < 0)
        return -1; /* Not a list */
    for (i = 0; i < n; i++) {
        item = PyList_GetItem(list, i); /* Can't fail */
        if (!PyInt_Check(item)) continue; /* Skip non-integers */
        total += PyInt_AsLong(item);
    }
    return total;
}
\end{verbatim}
\ttindex{sum_list()}

\begin{verbatim}
long sum_sequence(PyObject *sequence)
{
    int i, n;
    long total = 0;
    PyObject *item;
    n = PySequence_Length(sequence);
    if (n < 0)
        return -1; /* Has no length */
    for (i = 0; i < n; i++) {
        item = PySequence_GetItem(sequence, i);
        if (item == NULL)
            return -1; /* Not a sequence, or other failure */
        if (PyInt_Check(item))
            total += PyInt_AsLong(item);
        Py_DECREF(item); /* Discard reference ownership */
    }
    return total;
}
\end{verbatim}
\ttindex{sum_sequence()}


\subsection{Types \label{types}}

There are few other data types that play a significant role in 
the Python/C API; most are simple C types such as \ctype{int}, 
\ctype{long}, \ctype{double} and \ctype{char*}.  A few structure types 
are used to describe static tables used to list the functions exported 
by a module or the data attributes of a new object type, and another
is used to describe the value of a complex number.  These will 
be discussed together with the functions that use them.


\section{Exceptions \label{exceptions}}

The Python programmer only needs to deal with exceptions if specific 
error handling is required; unhandled exceptions are automatically 
propagated to the caller, then to the caller's caller, and so on, until
they reach the top-level interpreter, where they are reported to the 
user accompanied by a stack traceback.

For C programmers, however, error checking always has to be explicit.  
All functions in the Python/C API can raise exceptions, unless an 
explicit claim is made otherwise in a function's documentation.  In 
general, when a function encounters an error, it sets an exception, 
discards any object references that it owns, and returns an 
error indicator --- usually \NULL{} or \code{-1}.  A few functions 
return a Boolean true/false result, with false indicating an error.
Very few functions return no explicit error indicator or have an 
ambiguous return value, and require explicit testing for errors with 
\cfunction{PyErr_Occurred()}\ttindex{PyErr_Occurred()}.

Exception state is maintained in per-thread storage (this is 
equivalent to using global storage in an unthreaded application).  A 
thread can be in one of two states: an exception has occurred, or not.
The function \cfunction{PyErr_Occurred()} can be used to check for
this: it returns a borrowed reference to the exception type object
when an exception has occurred, and \NULL{} otherwise.  There are a
number of functions to set the exception state:
\cfunction{PyErr_SetString()}\ttindex{PyErr_SetString()} is the most
common (though not the most general) function to set the exception
state, and \cfunction{PyErr_Clear()}\ttindex{PyErr_Clear()} clears the
exception state.

The full exception state consists of three objects (all of which can 
be \NULL{}): the exception type, the corresponding exception 
value, and the traceback.  These have the same meanings as the Python
\withsubitem{(in module sys)}{
  \ttindex{exc_type}\ttindex{exc_value}\ttindex{exc_traceback}}
objects \code{sys.exc_type}, \code{sys.exc_value}, and
\code{sys.exc_traceback}; however, they are not the same: the Python
objects represent the last exception being handled by a Python 
\keyword{try} \ldots\ \keyword{except} statement, while the C level
exception state only exists while an exception is being passed on
between C functions until it reaches the Python bytecode interpreter's 
main loop, which takes care of transferring it to \code{sys.exc_type}
and friends.

Note that starting with Python 1.5, the preferred, thread-safe way to 
access the exception state from Python code is to call the function
\withsubitem{(in module sys)}{\ttindex{exc_info()}}
\function{sys.exc_info()}, which returns the per-thread exception state 
for Python code.  Also, the semantics of both ways to access the 
exception state have changed so that a function which catches an 
exception will save and restore its thread's exception state so as to 
preserve the exception state of its caller.  This prevents common bugs 
in exception handling code caused by an innocent-looking function 
overwriting the exception being handled; it also reduces the often 
unwanted lifetime extension for objects that are referenced by the 
stack frames in the traceback.

As a general principle, a function that calls another function to 
perform some task should check whether the called function raised an 
exception, and if so, pass the exception state on to its caller.  It 
should discard any object references that it owns, and return an 
error indicator, but it should \emph{not} set another exception ---
that would overwrite the exception that was just raised, and lose
important information about the exact cause of the error.

A simple example of detecting exceptions and passing them on is shown
in the \cfunction{sum_sequence()}\ttindex{sum_sequence()} example
above.  It so happens that that example doesn't need to clean up any
owned references when it detects an error.  The following example
function shows some error cleanup.  First, to remind you why you like
Python, we show the equivalent Python code:

\begin{verbatim}
def incr_item(dict, key):
    try:
        item = dict[key]
    except KeyError:
        item = 0
    dict[key] = item + 1
\end{verbatim}
\ttindex{incr_item()}

Here is the corresponding C code, in all its glory:

\begin{verbatim}
int incr_item(PyObject *dict, PyObject *key)
{
    /* Objects all initialized to NULL for Py_XDECREF */
    PyObject *item = NULL, *const_one = NULL, *incremented_item = NULL;
    int rv = -1; /* Return value initialized to -1 (failure) */

    item = PyObject_GetItem(dict, key);
    if (item == NULL) {
        /* Handle KeyError only: */
        if (!PyErr_ExceptionMatches(PyExc_KeyError))
            goto error;

        /* Clear the error and use zero: */
        PyErr_Clear();
        item = PyInt_FromLong(0L);
        if (item == NULL)
            goto error;
    }
    const_one = PyInt_FromLong(1L);
    if (const_one == NULL)
        goto error;

    incremented_item = PyNumber_Add(item, const_one);
    if (incremented_item == NULL)
        goto error;

    if (PyObject_SetItem(dict, key, incremented_item) < 0)
        goto error;
    rv = 0; /* Success */
    /* Continue with cleanup code */

 error:
    /* Cleanup code, shared by success and failure path */

    /* Use Py_XDECREF() to ignore NULL references */
    Py_XDECREF(item);
    Py_XDECREF(const_one);
    Py_XDECREF(incremented_item);

    return rv; /* -1 for error, 0 for success */
}
\end{verbatim}
\ttindex{incr_item()}

This example represents an endorsed use of the \keyword{goto} statement 
in C!  It illustrates the use of
\cfunction{PyErr_ExceptionMatches()}\ttindex{PyErr_ExceptionMatches()} and
\cfunction{PyErr_Clear()}\ttindex{PyErr_Clear()} to
handle specific exceptions, and the use of
\cfunction{Py_XDECREF()}\ttindex{Py_XDECREF()} to
dispose of owned references that may be \NULL{} (note the
\character{X} in the name; \cfunction{Py_DECREF()} would crash when
confronted with a \NULL{} reference).  It is important that the
variables used to hold owned references are initialized to \NULL{} for
this to work; likewise, the proposed return value is initialized to
\code{-1} (failure) and only set to success after the final call made
is successful.


\section{Embedding Python \label{embedding}}

The one important task that only embedders (as opposed to extension
writers) of the Python interpreter have to worry about is the
initialization, and possibly the finalization, of the Python
interpreter.  Most functionality of the interpreter can only be used
after the interpreter has been initialized.

The basic initialization function is
\cfunction{Py_Initialize()}\ttindex{Py_Initialize()}.
This initializes the table of loaded modules, and creates the
fundamental modules \module{__builtin__}\refbimodindex{__builtin__},
\module{__main__}\refbimodindex{__main__}, \module{sys}\refbimodindex{sys},
and \module{exceptions}.\refbimodindex{exceptions}  It also initializes
the module search path (\code{sys.path}).%
\indexiii{module}{search}{path}
\withsubitem{(in module sys)}{\ttindex{path}}

\cfunction{Py_Initialize()} does not set the ``script argument list'' 
(\code{sys.argv}).  If this variable is needed by Python code that 
will be executed later, it must be set explicitly with a call to 
\code{PySys_SetArgv(\var{argc},
\var{argv})}\ttindex{PySys_SetArgv()} subsequent to the call to
\cfunction{Py_Initialize()}.

On most systems (in particular, on \UNIX{} and Windows, although the
details are slightly different),
\cfunction{Py_Initialize()} calculates the module search path based
upon its best guess for the location of the standard Python
interpreter executable, assuming that the Python library is found in a
fixed location relative to the Python interpreter executable.  In
particular, it looks for a directory named
\file{lib/python\shortversion} relative to the parent directory where
the executable named \file{python} is found on the shell command
search path (the environment variable \envvar{PATH}).

For instance, if the Python executable is found in
\file{/usr/local/bin/python}, it will assume that the libraries are in
\file{/usr/local/lib/python\shortversion}.  (In fact, this particular path
is also the ``fallback'' location, used when no executable file named
\file{python} is found along \envvar{PATH}.)  The user can override
this behavior by setting the environment variable \envvar{PYTHONHOME},
or insert additional directories in front of the standard path by
setting \envvar{PYTHONPATH}.

The embedding application can steer the search by calling 
\code{Py_SetProgramName(\var{file})}\ttindex{Py_SetProgramName()} \emph{before} calling 
\cfunction{Py_Initialize()}.  Note that \envvar{PYTHONHOME} still
overrides this and \envvar{PYTHONPATH} is still inserted in front of
the standard path.  An application that requires total control has to
provide its own implementation of
\cfunction{Py_GetPath()}\ttindex{Py_GetPath()},
\cfunction{Py_GetPrefix()}\ttindex{Py_GetPrefix()},
\cfunction{Py_GetExecPrefix()}\ttindex{Py_GetExecPrefix()}, and
\cfunction{Py_GetProgramFullPath()}\ttindex{Py_GetProgramFullPath()} (all
defined in \file{Modules/getpath.c}).

Sometimes, it is desirable to ``uninitialize'' Python.  For instance, 
the application may want to start over (make another call to 
\cfunction{Py_Initialize()}) or the application is simply done with its 
use of Python and wants to free all memory allocated by Python.  This
can be accomplished by calling \cfunction{Py_Finalize()}.  The function
\cfunction{Py_IsInitialized()}\ttindex{Py_IsInitialized()} returns
true if Python is currently in the initialized state.  More
information about these functions is given in a later chapter.


\chapter{The Very High Level Layer \label{veryhigh}}

The functions in this chapter will let you execute Python source code
given in a file or a buffer, but they will not let you interact in a
more detailed way with the interpreter.

Several of these functions accept a start symbol from the grammar as a 
parameter.  The available start symbols are \constant{Py_eval_input},
\constant{Py_file_input}, and \constant{Py_single_input}.  These are
described following the functions which accept them as parameters.

Note also that several of these functions take \ctype{FILE*}
parameters.  On particular issue which needs to be handled carefully
is that the \ctype{FILE} structure for different C libraries can be
different and incompatible.  Under Windows (at least), it is possible
for dynamically linked extensions to actually use different libraries,
so care should be taken that \ctype{FILE*} parameters are only passed
to these functions if it is certain that they were created by the same
library that the Python runtime is using.

\begin{cfuncdesc}{int}{Py_Main}{int argc, char **argv}
  The main program for the standard interpreter.  This is made
  available for programs which embed Python.  The \var{argc} and
  \var{argv} parameters should be prepared exactly as those which are
  passed to a C program's \cfunction{main()} function.  It is
  important to note that the argument list may be modified (but the
  contents of the strings pointed to by the argument list are not).
  The return value will be the integer passed to the
  \function{sys.exit()} function, \code{1} if the interpreter exits
  due to an exception, or \code{2} if the parameter list does not
  represent a valid Python command line.
\end{cfuncdesc}

\begin{cfuncdesc}{int}{PyRun_AnyFile}{FILE *fp, char *filename}
  If \var{fp} refers to a file associated with an interactive device
  (console or terminal input or \UNIX{} pseudo-terminal), return the
  value of \cfunction{PyRun_InteractiveLoop()}, otherwise return the
  result of \cfunction{PyRun_SimpleFile()}.  If \var{filename} is
  \NULL{}, this function uses \code{"???"} as the filename.
\end{cfuncdesc}

\begin{cfuncdesc}{int}{PyRun_SimpleString}{char *command}
  Executes the Python source code from \var{command} in the
  \module{__main__} module.  If \module{__main__} does not already
  exist, it is created.  Returns \code{0} on success or \code{-1} if
  an exception was raised.  If there was an error, there is no way to
  get the exception information.
\end{cfuncdesc}

\begin{cfuncdesc}{int}{PyRun_SimpleFile}{FILE *fp, char *filename}
  Similar to \cfunction{PyRun_SimpleString()}, but the Python source
  code is read from \var{fp} instead of an in-memory string.
  \var{filename} should be the name of the file.
\end{cfuncdesc}

\begin{cfuncdesc}{int}{PyRun_InteractiveOne}{FILE *fp, char *filename}
  Read and execute a single statement from a file associated with an
  interactive device.  If \var{filename} is \NULL, \code{"???"} is
  used instead.  The user will be prompted using \code{sys.ps1} and
  \code{sys.ps2}.  Returns \code{0} when the input was executed
  successfully, \code{-1} if there was an exception, or an error code
  from the \file{errcode.h} include file distributed as part of Python
  in case of a parse error.  (Note that \file{errcode.h} is not
  included by \file{Python.h}, so must be included specifically if
  needed.)
\end{cfuncdesc}

\begin{cfuncdesc}{int}{PyRun_InteractiveLoop}{FILE *fp, char *filename}
  Read and execute statements from a file associated with an
  interactive device until \EOF{} is reached.  If \var{filename} is
  \NULL, \code{"???"} is used instead.  The user will be prompted
  using \code{sys.ps1} and \code{sys.ps2}.  Returns \code{0} at \EOF.
\end{cfuncdesc}

\begin{cfuncdesc}{struct _node*}{PyParser_SimpleParseString}{char *str,
                                                             int start}
  Parse Python source code from \var{str} using the start token
  \var{start}.  The result can be used to create a code object which
  can be evaluated efficiently.  This is useful if a code fragment
  must be evaluated many times.
\end{cfuncdesc}

\begin{cfuncdesc}{struct _node*}{PyParser_SimpleParseFile}{FILE *fp,
                                 char *filename, int start}
  Similar to \cfunction{PyParser_SimpleParseString()}, but the Python
  source code is read from \var{fp} instead of an in-memory string.
  \var{filename} should be the name of the file.
\end{cfuncdesc}

\begin{cfuncdesc}{PyObject*}{PyRun_String}{char *str, int start,
                                           PyObject *globals,
                                           PyObject *locals}
  Execute Python source code from \var{str} in the context specified
  by the dictionaries \var{globals} and \var{locals}.  The parameter
  \var{start} specifies the start token that should be used to parse
  the source code.

  Returns the result of executing the code as a Python object, or
  \NULL{} if an exception was raised.
\end{cfuncdesc}

\begin{cfuncdesc}{PyObject*}{PyRun_File}{FILE *fp, char *filename,
                                         int start, PyObject *globals,
                                         PyObject *locals}
  Similar to \cfunction{PyRun_String()}, but the Python source code is 
  read from \var{fp} instead of an in-memory string.
  \var{filename} should be the name of the file.
\end{cfuncdesc}

\begin{cfuncdesc}{PyObject*}{Py_CompileString}{char *str, char *filename,
                                               int start}
  Parse and compile the Python source code in \var{str}, returning the 
  resulting code object.  The start token is given by \var{start};
  this can be used to constrain the code which can be compiled and should
  be \constant{Py_eval_input}, \constant{Py_file_input}, or
  \constant{Py_single_input}.  The filename specified by
  \var{filename} is used to construct the code object and may appear
  in tracebacks or \exception{SyntaxError} exception messages.  This
  returns \NULL{} if the code cannot be parsed or compiled.
\end{cfuncdesc}

\begin{cvardesc}{int}{Py_eval_input}
  The start symbol from the Python grammar for isolated expressions;
  for use with \cfunction{Py_CompileString()}\ttindex{Py_CompileString()}.
\end{cvardesc}

\begin{cvardesc}{int}{Py_file_input}
  The start symbol from the Python grammar for sequences of statements
  as read from a file or other source; for use with
  \cfunction{Py_CompileString()}\ttindex{Py_CompileString()}.  This is
  the symbol to use when compiling arbitrarily long Python source code.
\end{cvardesc}

\begin{cvardesc}{int}{Py_single_input}
  The start symbol from the Python grammar for a single statement; for 
  use with \cfunction{Py_CompileString()}\ttindex{Py_CompileString()}.
  This is the symbol used for the interactive interpreter loop.
\end{cvardesc}


\chapter{Reference Counting \label{countingRefs}}

The macros in this section are used for managing reference counts
of Python objects.

\begin{cfuncdesc}{void}{Py_INCREF}{PyObject *o}
Increment the reference count for object \var{o}.  The object must
not be \NULL{}; if you aren't sure that it isn't \NULL{}, use
\cfunction{Py_XINCREF()}.
\end{cfuncdesc}

\begin{cfuncdesc}{void}{Py_XINCREF}{PyObject *o}
Increment the reference count for object \var{o}.  The object may be
\NULL{}, in which case the macro has no effect.
\end{cfuncdesc}

\begin{cfuncdesc}{void}{Py_DECREF}{PyObject *o}
Decrement the reference count for object \var{o}.  The object must
not be \NULL{}; if you aren't sure that it isn't \NULL{}, use
\cfunction{Py_XDECREF()}.  If the reference count reaches zero, the
object's type's deallocation function (which must not be \NULL{}) is
invoked.

\strong{Warning:} The deallocation function can cause arbitrary Python
code to be invoked (e.g. when a class instance with a
\method{__del__()} method is deallocated).  While exceptions in such
code are not propagated, the executed code has free access to all
Python global variables.  This means that any object that is reachable
from a global variable should be in a consistent state before
\cfunction{Py_DECREF()} is invoked.  For example, code to delete an
object from a list should copy a reference to the deleted object in a
temporary variable, update the list data structure, and then call
\cfunction{Py_DECREF()} for the temporary variable.
\end{cfuncdesc}

\begin{cfuncdesc}{void}{Py_XDECREF}{PyObject *o}
Decrement the reference count for object \var{o}.  The object may be
\NULL{}, in which case the macro has no effect; otherwise the effect
is the same as for \cfunction{Py_DECREF()}, and the same warning
applies.
\end{cfuncdesc}

The following functions or macros are only for use within the
interpreter core: \cfunction{_Py_Dealloc()},
\cfunction{_Py_ForgetReference()}, \cfunction{_Py_NewReference()}, as
well as the global variable \cdata{_Py_RefTotal}.


\chapter{Exception Handling \label{exceptionHandling}}

The functions described in this chapter will let you handle and raise Python
exceptions.  It is important to understand some of the basics of
Python exception handling.  It works somewhat like the
\UNIX{} \cdata{errno} variable: there is a global indicator (per
thread) of the last error that occurred.  Most functions don't clear
this on success, but will set it to indicate the cause of the error on
failure.  Most functions also return an error indicator, usually
\NULL{} if they are supposed to return a pointer, or \code{-1} if they
return an integer (exception: the \cfunction{PyArg_Parse*()} functions
return \code{1} for success and \code{0} for failure).  When a
function must fail because some function it called failed, it
generally doesn't set the error indicator; the function it called
already set it.

The error indicator consists of three Python objects corresponding to
\withsubitem{(in module sys)}{
  \ttindex{exc_type}\ttindex{exc_value}\ttindex{exc_traceback}}
the Python variables \code{sys.exc_type}, \code{sys.exc_value} and
\code{sys.exc_traceback}.  API functions exist to interact with the
error indicator in various ways.  There is a separate error indicator
for each thread.

% XXX Order of these should be more thoughtful.
% Either alphabetical or some kind of structure.

\begin{cfuncdesc}{void}{PyErr_Print}{}
Print a standard traceback to \code{sys.stderr} and clear the error
indicator.  Call this function only when the error indicator is set.
(Otherwise it will cause a fatal error!)
\end{cfuncdesc}

\begin{cfuncdesc}{PyObject*}{PyErr_Occurred}{}
Test whether the error indicator is set.  If set, return the exception
\emph{type} (the first argument to the last call to one of the
\cfunction{PyErr_Set*()} functions or to \cfunction{PyErr_Restore()}).  If
not set, return \NULL{}.  You do not own a reference to the return
value, so you do not need to \cfunction{Py_DECREF()} it.
\strong{Note:}  Do not compare the return value to a specific
exception; use \cfunction{PyErr_ExceptionMatches()} instead, shown
below.  (The comparison could easily fail since the exception may be
an instance instead of a class, in the case of a class exception, or
it may the a subclass of the expected exception.)
\end{cfuncdesc}

\begin{cfuncdesc}{int}{PyErr_ExceptionMatches}{PyObject *exc}
Equivalent to
\samp{PyErr_GivenExceptionMatches(PyErr_Occurred(), \var{exc})}.
This should only be called when an exception is actually set; a memory 
access violation will occur if no exception has been raised.
\end{cfuncdesc}

\begin{cfuncdesc}{int}{PyErr_GivenExceptionMatches}{PyObject *given, PyObject *exc}
Return true if the \var{given} exception matches the exception in
\var{exc}.  If \var{exc} is a class object, this also returns true
when \var{given} is an instance of a subclass.  If \var{exc} is a tuple, all
exceptions in the tuple (and recursively in subtuples) are searched
for a match.  If \var{given} is \NULL, a memory access violation will
occur.
\end{cfuncdesc}

\begin{cfuncdesc}{void}{PyErr_NormalizeException}{PyObject**exc, PyObject**val, PyObject**tb}
Under certain circumstances, the values returned by
\cfunction{PyErr_Fetch()} below can be ``unnormalized'', meaning that
\code{*\var{exc}} is a class object but \code{*\var{val}} is not an
instance of the  same class.  This function can be used to instantiate
the class in that case.  If the values are already normalized, nothing
happens.  The delayed normalization is implemented to improve
performance.
\end{cfuncdesc}

\begin{cfuncdesc}{void}{PyErr_Clear}{}
Clear the error indicator.  If the error indicator is not set, there
is no effect.
\end{cfuncdesc}

\begin{cfuncdesc}{void}{PyErr_Fetch}{PyObject **ptype, PyObject **pvalue,
                                     PyObject **ptraceback}
Retrieve the error indicator into three variables whose addresses are
passed.  If the error indicator is not set, set all three variables to
\NULL{}.  If it is set, it will be cleared and you own a reference to
each object retrieved.  The value and traceback object may be
\NULL{} even when the type object is not.  \strong{Note:}  This
function is normally only used by code that needs to handle exceptions
or by code that needs to save and restore the error indicator
temporarily.
\end{cfuncdesc}

\begin{cfuncdesc}{void}{PyErr_Restore}{PyObject *type, PyObject *value,
                                       PyObject *traceback}
Set  the error indicator from the three objects.  If the error
indicator is already set, it is cleared first.  If the objects are
\NULL{}, the error indicator is cleared.  Do not pass a \NULL{} type
and non-\NULL{} value or traceback.  The exception type should be a
string or class; if it is a class, the value should be an instance of
that class.  Do not pass an invalid exception type or value.
(Violating these rules will cause subtle problems later.)  This call
takes away a reference to each object: you must own a reference
to each object before the call and after the call you no longer own
these references.  (If you don't understand this, don't use this
function.  I warned you.)  \strong{Note:}  This function is normally
only used by code that needs to save and restore the error indicator
temporarily.
\end{cfuncdesc}

\begin{cfuncdesc}{void}{PyErr_SetString}{PyObject *type, char *message}
This is the most common way to set the error indicator.  The first
argument specifies the exception type; it is normally one of the
standard exceptions, e.g. \cdata{PyExc_RuntimeError}.  You need not
increment its reference count.  The second argument is an error
message; it is converted to a string object.
\end{cfuncdesc}

\begin{cfuncdesc}{void}{PyErr_SetObject}{PyObject *type, PyObject *value}
This function is similar to \cfunction{PyErr_SetString()} but lets you
specify an arbitrary Python object for the ``value'' of the exception.
You need not increment its reference count.
\end{cfuncdesc}

\begin{cfuncdesc}{PyObject*}{PyErr_Format}{PyObject *exception,
                                           const char *format, \moreargs}
This function sets the error indicator.  \var{exception} should be a
Python exception (string or class, not an instance).
\var{format} should be a string, containing format codes, similar to 
\cfunction{printf}. The \code{width.precision} before a format code
is parsed, but the width part is ignored.

\begin{tableii}{c|l}{character}{Character}{Meaning}
  \lineii{c}{Character, as an \ctype{int} parameter}
  \lineii{d}{Number in decimal, as an \ctype{int} parameter}
  \lineii{x}{Number in hexadecimal, as an \ctype{int} parameter}
  \lineii{x}{A string, as a \ctype{char *} parameter}
\end{tableii}

An unrecognized format character causes all the rest of
the format string to be copied as-is to the result string,
and any extra arguments discarded.

A new reference is returned, which is owned by the caller.
\end{cfuncdesc}

\begin{cfuncdesc}{void}{PyErr_SetNone}{PyObject *type}
This is a shorthand for \samp{PyErr_SetObject(\var{type}, Py_None)}.
\end{cfuncdesc}

\begin{cfuncdesc}{int}{PyErr_BadArgument}{}
This is a shorthand for \samp{PyErr_SetString(PyExc_TypeError,
\var{message})}, where \var{message} indicates that a built-in operation
was invoked with an illegal argument.  It is mostly for internal use.
\end{cfuncdesc}

\begin{cfuncdesc}{PyObject*}{PyErr_NoMemory}{}
This is a shorthand for \samp{PyErr_SetNone(PyExc_MemoryError)}; it
returns \NULL{} so an object allocation function can write
\samp{return PyErr_NoMemory();} when it runs out of memory.
\end{cfuncdesc}

\begin{cfuncdesc}{PyObject*}{PyErr_SetFromErrno}{PyObject *type}
This is a convenience function to raise an exception when a C library
function has returned an error and set the C variable \cdata{errno}.
It constructs a tuple object whose first item is the integer
\cdata{errno} value and whose second item is the corresponding error
message (gotten from \cfunction{strerror()}\ttindex{strerror()}), and
then calls
\samp{PyErr_SetObject(\var{type}, \var{object})}.  On \UNIX{}, when
the \cdata{errno} value is \constant{EINTR}, indicating an interrupted
system call, this calls \cfunction{PyErr_CheckSignals()}, and if that set
the error indicator, leaves it set to that.  The function always
returns \NULL{}, so a wrapper function around a system call can write 
\samp{return PyErr_SetFromErrno();} when  the system call returns an
error.
\end{cfuncdesc}

\begin{cfuncdesc}{PyObject*}{PyErr_SetFromErrnoWithFilename}{PyObject *type,
                                                             char *filename}
Similar to \cfunction{PyErr_SetFromErrno()}, with the additional
behavior that if \var{filename} is not \NULL, it is passed to the
constructor of \var{type} as a third parameter.  In the case of
exceptions such as \exception{IOError} and \exception{OSError}, this
is used to define the \member{filename} attribute of the exception
instance.
\end{cfuncdesc}

\begin{cfuncdesc}{void}{PyErr_BadInternalCall}{}
This is a shorthand for \samp{PyErr_SetString(PyExc_TypeError,
\var{message})}, where \var{message} indicates that an internal
operation (e.g. a Python/C API function) was invoked with an illegal
argument.  It is mostly for internal use.
\end{cfuncdesc}

\begin{cfuncdesc}{int}{PyErr_Warn}{PyObject *category, char *message}
Issue a warning message.  The \var{category} argument is a warning
category (see below) or \NULL; the \var{message} argument is a message
string.

This function normally prints a warning message to \var{sys.stderr};
however, it is also possible that the user has specified that warnings
are to be turned into errors, and in that case this will raise an
exception.  It is also possible that the function raises an exception
because of a problem with the warning machinery (the implementation
imports the \module{warnings} module to do the heavy lifting).  The
return value is \code{0} if no exception is raised, or \code{-1} if
an exception is raised.  (It is not possible to determine whether a
warning message is actually printed, nor what the reason is for the
exception; this is intentional.)  If an exception is raised, the
caller should do its normal exception handling
(e.g. \cfunction{Py_DECREF()} owned references and return an error
value).

Warning categories must be subclasses of \cdata{Warning}; the default
warning category is \cdata{RuntimeWarning}.  The standard Python
warning categories are available as global variables whose names are
\samp{PyExc_} followed by the Python exception name.  These have the
type \ctype{PyObject*}; they are all class objects.  Their names are
\cdata{PyExc_Warning}, \cdata{PyExc_UserWarning},
\cdata{PyExc_DeprecationWarning}, \cdata{PyExc_SyntaxWarning}, and
\cdata{PyExc_RuntimeWarning}.  \cdata{PyExc_Warning} is a subclass of
\cdata{PyExc_Exception}; the other warning categories are subclasses
of \cdata{PyExc_Warning}.

For information about warning control, see the documentation for the
\module{warnings} module and the \programopt{-W} option in the command
line documentation.  There is no C API for warning control.
\end{cfuncdesc}

\begin{cfuncdesc}{int}{PyErr_WarnExplicit}{PyObject *category, char *message,
char *filename, int lineno, char *module, PyObject *registry}
Issue a warning message with explicit control over all warning
attributes.  This is a straightforward wrapper around the Python
function \function{warnings.warn_explicit()}, see there for more
information.  The \var{module} and \var{registry} arguments may be
set to \code{NULL} to get the default effect described there.
\end{cfuncdesc}

\begin{cfuncdesc}{int}{PyErr_CheckSignals}{}
This function interacts with Python's signal handling.  It checks
whether a signal has been sent to the processes and if so, invokes the
corresponding signal handler.  If the
\module{signal}\refbimodindex{signal} module is supported, this can
invoke a signal handler written in Python.  In all cases, the default
effect for \constant{SIGINT}\ttindex{SIGINT} is to raise the
\withsubitem{(built-in exception)}{\ttindex{KeyboardInterrupt}}
\exception{KeyboardInterrupt} exception.  If an exception is raised the 
error indicator is set and the function returns \code{1}; otherwise
the function returns \code{0}.  The error indicator may or may not be
cleared if it was previously set.
\end{cfuncdesc}

\begin{cfuncdesc}{void}{PyErr_SetInterrupt}{}
This function is obsolete.  It simulates the effect of a
\constant{SIGINT}\ttindex{SIGINT} signal arriving --- the next time
\cfunction{PyErr_CheckSignals()} is called,
\withsubitem{(built-in exception)}{\ttindex{KeyboardInterrupt}}
\exception{KeyboardInterrupt} will be raised.
It may be called without holding the interpreter lock.
\end{cfuncdesc}

\begin{cfuncdesc}{PyObject*}{PyErr_NewException}{char *name,
                                                 PyObject *base,
                                                 PyObject *dict}
This utility function creates and returns a new exception object.  The
\var{name} argument must be the name of the new exception, a C string
of the form \code{module.class}.  The \var{base} and
\var{dict} arguments are normally \NULL{}.  This creates a
class object derived from the root for all exceptions, the built-in
name \exception{Exception} (accessible in C as
\cdata{PyExc_Exception}).  The \member{__module__} attribute of the
new class is set to the first part (up to the last dot) of the
\var{name} argument, and the class name is set to the last part (after
the last dot).  The \var{base} argument can be used to specify an
alternate base class.  The \var{dict} argument can be used to specify
a dictionary of class variables and methods.
\end{cfuncdesc}

\begin{cfuncdesc}{void}{PyErr_WriteUnraisable}{PyObject *obj}
This utility function prints a warning message to \var{sys.stderr}
when an exception has been set but it is impossible for the
interpreter to actually raise the exception.  It is used, for example,
when an exception occurs in an \member{__del__} method.

The function is called with a single argument \var{obj} that
identifies where the context in which the unraisable exception
occurred.  The repr of \var{obj} will be printed in the warning
message.
\end{cfuncdesc}

\section{Standard Exceptions \label{standardExceptions}}

All standard Python exceptions are available as global variables whose
names are \samp{PyExc_} followed by the Python exception name.  These
have the type \ctype{PyObject*}; they are all class objects.  For
completeness, here are all the variables:

\begin{tableiii}{l|l|c}{cdata}{C Name}{Python Name}{Notes}
  \lineiii{PyExc_Exception}{\exception{Exception}}{(1)}
  \lineiii{PyExc_StandardError}{\exception{StandardError}}{(1)}
  \lineiii{PyExc_ArithmeticError}{\exception{ArithmeticError}}{(1)}
  \lineiii{PyExc_LookupError}{\exception{LookupError}}{(1)}
  \lineiii{PyExc_AssertionError}{\exception{AssertionError}}{}
  \lineiii{PyExc_AttributeError}{\exception{AttributeError}}{}
  \lineiii{PyExc_EOFError}{\exception{EOFError}}{}
  \lineiii{PyExc_EnvironmentError}{\exception{EnvironmentError}}{(1)}
  \lineiii{PyExc_FloatingPointError}{\exception{FloatingPointError}}{}
  \lineiii{PyExc_IOError}{\exception{IOError}}{}
  \lineiii{PyExc_ImportError}{\exception{ImportError}}{}
  \lineiii{PyExc_IndexError}{\exception{IndexError}}{}
  \lineiii{PyExc_KeyError}{\exception{KeyError}}{}
  \lineiii{PyExc_KeyboardInterrupt}{\exception{KeyboardInterrupt}}{}
  \lineiii{PyExc_MemoryError}{\exception{MemoryError}}{}
  \lineiii{PyExc_NameError}{\exception{NameError}}{}
  \lineiii{PyExc_NotImplementedError}{\exception{NotImplementedError}}{}
  \lineiii{PyExc_OSError}{\exception{OSError}}{}
  \lineiii{PyExc_OverflowError}{\exception{OverflowError}}{}
  \lineiii{PyExc_RuntimeError}{\exception{RuntimeError}}{}
  \lineiii{PyExc_SyntaxError}{\exception{SyntaxError}}{}
  \lineiii{PyExc_SystemError}{\exception{SystemError}}{}
  \lineiii{PyExc_SystemExit}{\exception{SystemExit}}{}
  \lineiii{PyExc_TypeError}{\exception{TypeError}}{}
  \lineiii{PyExc_ValueError}{\exception{ValueError}}{}
  \lineiii{PyExc_WindowsError}{\exception{WindowsError}}{(2)}
  \lineiii{PyExc_ZeroDivisionError}{\exception{ZeroDivisionError}}{}
\end{tableiii}

\noindent
Notes:
\begin{description}
\item[(1)]
  This is a base class for other standard exceptions.

\item[(2)]
  Only defined on Windows; protect code that uses this by testing that
  the preprocessor macro \code{MS_WINDOWS} is defined.
\end{description}


\section{Deprecation of String Exceptions}

All exceptions built into Python or provided in the standard library
are derived from \exception{Exception}.
\withsubitem{(built-in exception)}{\ttindex{Exception}}

String exceptions are still supported in the interpreter to allow
existing code to run unmodified, but this will also change in a future 
release.


\chapter{Utilities \label{utilities}}

The functions in this chapter perform various utility tasks, ranging
from helping C code be more portable across platforms, using Python
modules from C, and parsing function arguments and constructing Python
values from C values.


\section{Operating System Utilities \label{os}}

\begin{cfuncdesc}{int}{Py_FdIsInteractive}{FILE *fp, char *filename}
Return true (nonzero) if the standard I/O file \var{fp} with name
\var{filename} is deemed interactive.  This is the case for files for
which \samp{isatty(fileno(\var{fp}))} is true.  If the global flag
\cdata{Py_InteractiveFlag} is true, this function also returns true if
the \var{filename} pointer is \NULL{} or if the name is equal to one of
the strings \code{'<stdin>'} or \code{'???'}.
\end{cfuncdesc}

\begin{cfuncdesc}{long}{PyOS_GetLastModificationTime}{char *filename}
Return the time of last modification of the file \var{filename}.
The result is encoded in the same way as the timestamp returned by
the standard C library function \cfunction{time()}.
\end{cfuncdesc}

\begin{cfuncdesc}{void}{PyOS_AfterFork}{}
Function to update some internal state after a process fork; this
should be called in the new process if the Python interpreter will
continue to be used.  If a new executable is loaded into the new
process, this function does not need to be called.
\end{cfuncdesc}

\begin{cfuncdesc}{int}{PyOS_CheckStack}{}
Return true when the interpreter runs out of stack space.  This is a
reliable check, but is only available when \code{USE_STACKCHECK} is
defined (currently on Windows using the Microsoft Visual C++ compiler
and on the Macintosh).  \code{USE_CHECKSTACK} will be defined
automatically; you should never change the definition in your own
code.
\end{cfuncdesc}

\begin{cfuncdesc}{PyOS_sighandler_t}{PyOS_getsig}{int i}
Return the current signal handler for signal \var{i}.
This is a thin wrapper around either \cfunction{sigaction} or
\cfunction{signal}.  Do not call those functions directly!
\ctype{PyOS_sighandler_t} is a typedef alias for \ctype{void (*)(int)}.
\end{cfuncdesc}

\begin{cfuncdesc}{PyOS_sighandler_t}{PyOS_setsig}{int i, PyOS_sighandler_t h}
Set the signal handler for signal \var{i} to be \var{h};
return the old signal handler.
This is a thin wrapper around either \cfunction{sigaction} or
\cfunction{signal}.  Do not call those functions directly!
\ctype{PyOS_sighandler_t} is a typedef alias for \ctype{void (*)(int)}.
\end{cfuncdesc}


\section{Process Control \label{processControl}}

\begin{cfuncdesc}{void}{Py_FatalError}{char *message}
Print a fatal error message and kill the process.  No cleanup is
performed.  This function should only be invoked when a condition is
detected that would make it dangerous to continue using the Python
interpreter; e.g., when the object administration appears to be
corrupted.  On \UNIX{}, the standard C library function
\cfunction{abort()}\ttindex{abort()} is called which will attempt to
produce a \file{core} file.
\end{cfuncdesc}

\begin{cfuncdesc}{void}{Py_Exit}{int status}
Exit the current process.  This calls
\cfunction{Py_Finalize()}\ttindex{Py_Finalize()} and
then calls the standard C library function
\code{exit(\var{status})}\ttindex{exit()}.
\end{cfuncdesc}

\begin{cfuncdesc}{int}{Py_AtExit}{void (*func) ()}
Register a cleanup function to be called by
\cfunction{Py_Finalize()}\ttindex{Py_Finalize()}.
The cleanup function will be called with no arguments and should
return no value.  At most 32 \index{cleanup functions}cleanup
functions can be registered.
When the registration is successful, \cfunction{Py_AtExit()} returns
\code{0}; on failure, it returns \code{-1}.  The cleanup function
registered last is called first.  Each cleanup function will be called
at most once.  Since Python's internal finallization will have
completed before the cleanup function, no Python APIs should be called
by \var{func}.
\end{cfuncdesc}


\section{Importing Modules \label{importing}}

\begin{cfuncdesc}{PyObject*}{PyImport_ImportModule}{char *name}
This is a simplified interface to
\cfunction{PyImport_ImportModuleEx()} below, leaving the
\var{globals} and \var{locals} arguments set to \NULL{}.  When the
\var{name} argument contains a dot (when it specifies a
submodule of a package), the \var{fromlist} argument is set to the
list \code{['*']} so that the return value is the named module rather
than the top-level package containing it as would otherwise be the
case.  (Unfortunately, this has an additional side effect when
\var{name} in fact specifies a subpackage instead of a submodule: the
submodules specified in the package's \code{__all__} variable are
\index{package variable!\code{__all__}}
\withsubitem{(package variable)}{\ttindex{__all__}}loaded.)  Return a
new reference to the imported module, or
\NULL{} with an exception set on failure (the module may still be
created in this case --- examine \code{sys.modules} to find out).
\withsubitem{(in module sys)}{\ttindex{modules}}
\end{cfuncdesc}

\begin{cfuncdesc}{PyObject*}{PyImport_ImportModuleEx}{char *name,
                       PyObject *globals, PyObject *locals, PyObject *fromlist}
Import a module.  This is best described by referring to the built-in
Python function \function{__import__()}\bifuncindex{__import__}, as
the standard \function{__import__()} function calls this function
directly.

The return value is a new reference to the imported module or
top-level package, or \NULL{} with an exception set on failure
(the module may still be created in this case).  Like for
\function{__import__()}, the return value when a submodule of a
package was requested is normally the top-level package, unless a
non-empty \var{fromlist} was given.
\end{cfuncdesc}

\begin{cfuncdesc}{PyObject*}{PyImport_Import}{PyObject *name}
This is a higher-level interface that calls the current ``import hook
function''.  It invokes the \function{__import__()} function from the
\code{__builtins__} of the current globals.  This means that the
import is done using whatever import hooks are installed in the
current environment, e.g. by \module{rexec}\refstmodindex{rexec} or
\module{ihooks}\refstmodindex{ihooks}.
\end{cfuncdesc}

\begin{cfuncdesc}{PyObject*}{PyImport_ReloadModule}{PyObject *m}
Reload a module.  This is best described by referring to the built-in
Python function \function{reload()}\bifuncindex{reload}, as the standard
\function{reload()} function calls this function directly.  Return a
new reference to the reloaded module, or \NULL{} with an exception set
on failure (the module still exists in this case).
\end{cfuncdesc}

\begin{cfuncdesc}{PyObject*}{PyImport_AddModule}{char *name}
Return the module object corresponding to a module name.  The
\var{name} argument may be of the form \code{package.module}).  First
check the modules dictionary if there's one there, and if not, create
a new one and insert in in the modules dictionary.
Warning: this function does not load or import the module; if the
module wasn't already loaded, you will get an empty module object.
Use \cfunction{PyImport_ImportModule()} or one of its variants to
import a module.
Return \NULL{} with an exception set on failure.
\end{cfuncdesc}

\begin{cfuncdesc}{PyObject*}{PyImport_ExecCodeModule}{char *name, PyObject *co}
Given a module name (possibly of the form \code{package.module}) and a
code object read from a Python bytecode file or obtained from the
built-in function \function{compile()}\bifuncindex{compile}, load the
module.  Return a new reference to the module object, or \NULL{} with
an exception set if an error occurred (the module may still be created
in this case).  (This function would reload the module if it was
already imported.)
\end{cfuncdesc}

\begin{cfuncdesc}{long}{PyImport_GetMagicNumber}{}
Return the magic number for Python bytecode files (a.k.a.
\file{.pyc} and \file{.pyo} files).  The magic number should be
present in the first four bytes of the bytecode file, in little-endian
byte order.
\end{cfuncdesc}

\begin{cfuncdesc}{PyObject*}{PyImport_GetModuleDict}{}
Return the dictionary used for the module administration
(a.k.a. \code{sys.modules}).  Note that this is a per-interpreter
variable.
\end{cfuncdesc}

\begin{cfuncdesc}{void}{_PyImport_Init}{}
Initialize the import mechanism.  For internal use only.
\end{cfuncdesc}

\begin{cfuncdesc}{void}{PyImport_Cleanup}{}
Empty the module table.  For internal use only.
\end{cfuncdesc}

\begin{cfuncdesc}{void}{_PyImport_Fini}{}
Finalize the import mechanism.  For internal use only.
\end{cfuncdesc}

\begin{cfuncdesc}{PyObject*}{_PyImport_FindExtension}{char *, char *}
For internal use only.
\end{cfuncdesc}

\begin{cfuncdesc}{PyObject*}{_PyImport_FixupExtension}{char *, char *}
For internal use only.
\end{cfuncdesc}

\begin{cfuncdesc}{int}{PyImport_ImportFrozenModule}{char *name}
Load a frozen module named \var{name}.  Return \code{1} for success,
\code{0} if the module is not found, and \code{-1} with an exception
set if the initialization failed.  To access the imported module on a
successful load, use \cfunction{PyImport_ImportModule()}.
(Note the misnomer --- this function would reload the module if it was
already imported.)
\end{cfuncdesc}

\begin{ctypedesc}[_frozen]{struct _frozen}
This is the structure type definition for frozen module descriptors,
as generated by the \program{freeze}\index{freeze utility} utility
(see \file{Tools/freeze/} in the Python source distribution).  Its
definition, found in \file{Include/import.h}, is:

\begin{verbatim}
struct _frozen {
    char *name;
    unsigned char *code;
    int size;
};
\end{verbatim}
\end{ctypedesc}

\begin{cvardesc}{struct _frozen*}{PyImport_FrozenModules}
This pointer is initialized to point to an array of \ctype{struct
_frozen} records, terminated by one whose members are all
\NULL{} or zero.  When a frozen module is imported, it is searched in
this table.  Third-party code could play tricks with this to provide a 
dynamically created collection of frozen modules.
\end{cvardesc}

\begin{cfuncdesc}{int}{PyImport_AppendInittab}{char *name,
                                               void (*initfunc)(void)}
Add a single module to the existing table of built-in modules.  This
is a convenience wrapper around \cfunction{PyImport_ExtendInittab()},
returning \code{-1} if the table could not be extended.  The new
module can be imported by the name \var{name}, and uses the function
\var{initfunc} as the initialization function called on the first
attempted import.  This should be called before
\cfunction{Py_Initialize()}.
\end{cfuncdesc}

\begin{ctypedesc}[_inittab]{struct _inittab}
Structure describing a single entry in the list of built-in modules.
Each of these structures gives the name and initialization function
for a module built into the interpreter.  Programs which embed Python
may use an array of these structures in conjunction with
\cfunction{PyImport_ExtendInittab()} to provide additional built-in
modules.  The structure is defined in \file{Include/import.h} as:

\begin{verbatim}
struct _inittab {
    char *name;
    void (*initfunc)(void);
};
\end{verbatim}
\end{ctypedesc}

\begin{cfuncdesc}{int}{PyImport_ExtendInittab}{struct _inittab *newtab}
Add a collection of modules to the table of built-in modules.  The
\var{newtab} array must end with a sentinel entry which contains
\NULL{} for the \member{name} field; failure to provide the sentinel
value can result in a memory fault.  Returns \code{0} on success or
\code{-1} if insufficient memory could be allocated to extend the
internal table.  In the event of failure, no modules are added to the
internal table.  This should be called before
\cfunction{Py_Initialize()}.
\end{cfuncdesc}


\section{Parsing arguments and building values
         \label{arg-parsing}}

These functions are useful when creating your own extensions functions
and methods.  Additional information and examples are available in
\citetitle[../ext/ext.html]{Extending and Embedding the Python
Interpreter}.

\begin{cfuncdesc}{int}{PyArg_ParseTuple}{PyObject *args, char *format,
                                         \moreargs}
  Parse the parameters of a function that takes only positional
  parameters into local variables.  Returns true on success; on
  failure, it returns false and raises the appropriate exception.  See
  \citetitle[../ext/parseTuple.html]{Extending and Embedding the
  Python Interpreter} for more information.
\end{cfuncdesc}

\begin{cfuncdesc}{int}{PyArg_ParseTupleAndKeywords}{PyObject *args,
                       PyObject *kw, char *format, char *keywords[],
                       \moreargs}
  Parse the parameters of a function that takes both positional and
  keyword parameters into local variables.  Returns true on success;
  on failure, it returns false and raises the appropriate exception.
  See \citetitle[../ext/parseTupleAndKeywords.html]{Extending and
  Embedding the Python Interpreter} for more information.
\end{cfuncdesc}

\begin{cfuncdesc}{int}{PyArg_Parse}{PyObject *args, char *format,
                                    \moreargs}
  Function used to deconstruct the argument lists of ``old-style''
  functions --- these are functions which use the
  \constant{METH_OLDARGS} parameter parsing method.  This is not
  recommended for use in parameter parsing in new code, and most code
  in the standard interpreter has been modified to no longer use this
  for that purpose.  It does remain a convenient way to decompose
  other tuples, however, and may continue to be used for that
  purpose.
\end{cfuncdesc}

\begin{cfuncdesc}{PyObject*}{Py_BuildValue}{char *format,
                                            \moreargs}
  Create a new value based on a format string similar to those
  accepted by the \cfunction{PyArg_Parse*()} family of functions and a
  sequence of values.  Returns the value or \NULL{} in the case of an
  error; an exception will be raised if \NULL{} is returned.  For more
  information on the format string and additional parameters, see
  \citetitle[../ext/buildValue.html]{Extending and Embedding the
  Python Interpreter}.
\end{cfuncdesc}



\chapter{Abstract Objects Layer \label{abstract}}

The functions in this chapter interact with Python objects regardless
of their type, or with wide classes of object types (e.g. all
numerical types, or all sequence types).  When used on object types
for which they do not apply, they will raise a Python exception.

\section{Object Protocol \label{object}}

\begin{cfuncdesc}{int}{PyObject_Print}{PyObject *o, FILE *fp, int flags}
Print an object \var{o}, on file \var{fp}.  Returns \code{-1} on error.
The flags argument is used to enable certain printing options.  The
only option currently supported is \constant{Py_PRINT_RAW}; if given,
the \function{str()} of the object is written instead of the
\function{repr()}.
\end{cfuncdesc}

\begin{cfuncdesc}{int}{PyObject_HasAttrString}{PyObject *o, char *attr_name}
Returns \code{1} if \var{o} has the attribute \var{attr_name}, and
\code{0} otherwise.  This is equivalent to the Python expression
\samp{hasattr(\var{o}, \var{attr_name})}.
This function always succeeds.
\end{cfuncdesc}

\begin{cfuncdesc}{PyObject*}{PyObject_GetAttrString}{PyObject *o,
                                                     char *attr_name}
Retrieve an attribute named \var{attr_name} from object \var{o}.
Returns the attribute value on success, or \NULL{} on failure.
This is the equivalent of the Python expression
\samp{\var{o}.\var{attr_name}}.
\end{cfuncdesc}


\begin{cfuncdesc}{int}{PyObject_HasAttr}{PyObject *o, PyObject *attr_name}
Returns \code{1} if \var{o} has the attribute \var{attr_name}, and
\code{0} otherwise.  This is equivalent to the Python expression
\samp{hasattr(\var{o}, \var{attr_name})}. 
This function always succeeds.
\end{cfuncdesc}


\begin{cfuncdesc}{PyObject*}{PyObject_GetAttr}{PyObject *o,
                                               PyObject *attr_name}
Retrieve an attribute named \var{attr_name} from object \var{o}.
Returns the attribute value on success, or \NULL{} on failure.
This is the equivalent of the Python expression
\samp{\var{o}.\var{attr_name}}.
\end{cfuncdesc}


\begin{cfuncdesc}{int}{PyObject_SetAttrString}{PyObject *o,
                                               char *attr_name, PyObject *v}
Set the value of the attribute named \var{attr_name}, for object
\var{o}, to the value \var{v}. Returns \code{-1} on failure.  This is
the equivalent of the Python statement \samp{\var{o}.\var{attr_name} =
\var{v}}.
\end{cfuncdesc}


\begin{cfuncdesc}{int}{PyObject_SetAttr}{PyObject *o,
                                         PyObject *attr_name, PyObject *v}
Set the value of the attribute named \var{attr_name}, for
object \var{o},
to the value \var{v}. Returns \code{-1} on failure.  This is
the equivalent of the Python statement \samp{\var{o}.\var{attr_name} =
\var{v}}.
\end{cfuncdesc}


\begin{cfuncdesc}{int}{PyObject_DelAttrString}{PyObject *o, char *attr_name}
Delete attribute named \var{attr_name}, for object \var{o}. Returns
\code{-1} on failure.  This is the equivalent of the Python
statement: \samp{del \var{o}.\var{attr_name}}.
\end{cfuncdesc}


\begin{cfuncdesc}{int}{PyObject_DelAttr}{PyObject *o, PyObject *attr_name}
Delete attribute named \var{attr_name}, for object \var{o}. Returns
\code{-1} on failure.  This is the equivalent of the Python
statement \samp{del \var{o}.\var{attr_name}}.
\end{cfuncdesc}


\begin{cfuncdesc}{int}{PyObject_Cmp}{PyObject *o1, PyObject *o2, int *result}
Compare the values of \var{o1} and \var{o2} using a routine provided
by \var{o1}, if one exists, otherwise with a routine provided by
\var{o2}.  The result of the comparison is returned in \var{result}.
Returns \code{-1} on failure.  This is the equivalent of the Python
statement\bifuncindex{cmp} \samp{\var{result} = cmp(\var{o1}, \var{o2})}.
\end{cfuncdesc}


\begin{cfuncdesc}{int}{PyObject_Compare}{PyObject *o1, PyObject *o2}
Compare the values of \var{o1} and \var{o2} using a routine provided
by \var{o1}, if one exists, otherwise with a routine provided by
\var{o2}.  Returns the result of the comparison on success.  On error,
the value returned is undefined; use \cfunction{PyErr_Occurred()} to
detect an error.  This is equivalent to the Python
expression\bifuncindex{cmp} \samp{cmp(\var{o1}, \var{o2})}.
\end{cfuncdesc}


\begin{cfuncdesc}{PyObject*}{PyObject_Repr}{PyObject *o}
Compute a string representation of object \var{o}.  Returns the
string representation on success, \NULL{} on failure.  This is
the equivalent of the Python expression \samp{repr(\var{o})}.
Called by the \function{repr()}\bifuncindex{repr} built-in function
and by reverse quotes.
\end{cfuncdesc}


\begin{cfuncdesc}{PyObject*}{PyObject_Str}{PyObject *o}
Compute a string representation of object \var{o}.  Returns the
string representation on success, \NULL{} on failure.  This is
the equivalent of the Python expression \samp{str(\var{o})}.
Called by the \function{str()}\bifuncindex{str} built-in function and
by the \keyword{print} statement.
\end{cfuncdesc}


\begin{cfuncdesc}{PyObject*}{PyObject_Unicode}{PyObject *o}
Compute a Unicode string representation of object \var{o}.  Returns the
Unicode string representation on success, \NULL{} on failure.  This is
the equivalent of the Python expression \samp{unistr(\var{o})}.
Called by the \function{unistr()}\bifuncindex{unistr} built-in function.
\end{cfuncdesc}

\begin{cfuncdesc}{int}{PyObject_IsInstance}{PyObject *inst, PyObject *cls}
Return \code{1} if \var{inst} is an instance of the class \var{cls} or
a subclass of \var{cls}.  If \var{cls} is a type object rather than a
class object, \cfunction{PyObject_IsInstance()} returns \code{1} if
\var{inst} is of type \var{cls}.  If \var{inst} is not a class
instance and \var{cls} is neither a type object or class object,
\var{inst} must have a \member{__class__} attribute --- the class
relationship of the value of that attribute with \var{cls} will be
used to determine the result of this function.
\versionadded{2.1}
\end{cfuncdesc}

Subclass determination is done in a fairly straightforward way, but
includes a wrinkle that implementors of extensions to the class system
may want to be aware of.  If \class{A} and \class{B} are class
objects, \class{B} is a subclass of \class{A} if it inherits from
\class{A} either directly or indirectly.  If either is not a class
object, a more general mechanism is used to determine the class
relationship of the two objects.  When testing if \var{B} is a
subclass of \var{A}, if \var{A} is \var{B},
\cfunction{PyObject_IsSubclass()} returns true.  If \var{A} and
\var{B} are different objects, \var{B}'s \member{__bases__} attribute
is searched in a depth-first fashion for \var{A} --- the presence of
the \member{__bases__} attribute is considered sufficient for this
determination.

\begin{cfuncdesc}{int}{PyObject_IsSubclass}{PyObject *derived,
                                            PyObject *cls}
Returns \code{1} if the class \var{derived} is identical to or derived
from the class \var{cls}, otherwise returns \code{0}.  In case of an
error, returns \code{-1}.  If either \var{derived} or \var{cls} is not
an actual class object, this function uses the generic algorithm
described above.
\versionadded{2.1}
\end{cfuncdesc}


\begin{cfuncdesc}{int}{PyCallable_Check}{PyObject *o}
Determine if the object \var{o} is callable.  Return \code{1} if the
object is callable and \code{0} otherwise.
This function always succeeds.
\end{cfuncdesc}


\begin{cfuncdesc}{PyObject*}{PyObject_CallObject}{PyObject *callable_object,
                                                  PyObject *args}
Call a callable Python object \var{callable_object}, with
arguments given by the tuple \var{args}.  If no arguments are
needed, then \var{args} may be \NULL{}.  Returns the result of the
call on success, or \NULL{} on failure.  This is the equivalent
of the Python expression \samp{apply(\var{callable_object},
\var{args})} or \samp{\var{callable_object}(*\var{args})}.
\bifuncindex{apply}
\end{cfuncdesc}

\begin{cfuncdesc}{PyObject*}{PyObject_CallFunction}{PyObject *callable_object,
                                                    char *format, ...}
Call a callable Python object \var{callable_object}, with a
variable number of C arguments. The C arguments are described
using a \cfunction{Py_BuildValue()} style format string. The format may
be \NULL{}, indicating that no arguments are provided.  Returns the
result of the call on success, or \NULL{} on failure.  This is
the equivalent of the Python expression
\samp{apply(\var{callable_object}\var{args})} or
\samp{\var{callable_object}(*\var{args})}.
\bifuncindex{apply}
\end{cfuncdesc}


\begin{cfuncdesc}{PyObject*}{PyObject_CallMethod}{PyObject *o,
                                           char *method, char *format, ...}
Call the method named \var{m} of object \var{o} with a variable number
of C arguments.  The C arguments are described by a
\cfunction{Py_BuildValue()} format string.  The format may be \NULL{},
indicating that no arguments are provided. Returns the result of the
call on success, or \NULL{} on failure.  This is the equivalent of the
Python expression \samp{\var{o}.\var{method}(\var{args})}.
Note that special method names, such as \method{__add__()},
\method{__getitem__()}, and so on are not supported.  The specific
abstract-object routines for these must be used.
\end{cfuncdesc}


\begin{cfuncdesc}{int}{PyObject_Hash}{PyObject *o}
Compute and return the hash value of an object \var{o}.  On
failure, return \code{-1}.  This is the equivalent of the Python
expression \samp{hash(\var{o})}.\bifuncindex{hash}
\end{cfuncdesc}


\begin{cfuncdesc}{int}{PyObject_IsTrue}{PyObject *o}
Returns \code{1} if the object \var{o} is considered to be true, and
\code{0} otherwise. This is equivalent to the Python expression
\samp{not not \var{o}}.
This function always succeeds.
\end{cfuncdesc}


\begin{cfuncdesc}{PyObject*}{PyObject_Type}{PyObject *o}
When \var{o} is non-\NULL, returns a type object corresponding to the
object type of object \var{o}. On failure, raises
\exception{SystemError} and returns \NULL.  This is equivalent to the
Python expression \code{type(\var{o})}.
\bifuncindex{type}
\end{cfuncdesc}

\begin{cfuncdesc}{int}{PyObject_TypeCheck}{PyObject *o, PyTypeObject *type}
Return true if the object \var{o} is of type \var{type} or a subtype
of \var{type}.  Both parameters must be non-\NULL.
\versionadded{2.2}
\end{cfuncdesc}

\begin{cfuncdesc}{int}{PyObject_Length}{PyObject *o}
Return the length of object \var{o}.  If the object \var{o} provides
both sequence and mapping protocols, the sequence length is
returned.  On error, \code{-1} is returned.  This is the equivalent
to the Python expression \samp{len(\var{o})}.\bifuncindex{len}
\end{cfuncdesc}


\begin{cfuncdesc}{PyObject*}{PyObject_GetItem}{PyObject *o, PyObject *key}
Return element of \var{o} corresponding to the object \var{key} or
\NULL{} on failure. This is the equivalent of the Python expression
\samp{\var{o}[\var{key}]}.
\end{cfuncdesc}


\begin{cfuncdesc}{int}{PyObject_SetItem}{PyObject *o,
                                         PyObject *key, PyObject *v}
Map the object \var{key} to the value \var{v}.
Returns \code{-1} on failure.  This is the equivalent
of the Python statement \samp{\var{o}[\var{key}] = \var{v}}.
\end{cfuncdesc}


\begin{cfuncdesc}{int}{PyObject_DelItem}{PyObject *o, PyObject *key}
Delete the mapping for \var{key} from \var{o}.  Returns \code{-1} on
failure. This is the equivalent of the Python statement \samp{del
\var{o}[\var{key}]}.
\end{cfuncdesc}

\begin{cfuncdesc}{int}{PyObject_AsFileDescriptor}{PyObject *o}
Derives a file-descriptor from a Python object.  If the object
is an integer or long integer, its value is returned.  If not, the
object's \method{fileno()} method is called if it exists; the method
must return an integer or long integer, which is returned as the file
descriptor value.  Returns \code{-1} on failure.
\end{cfuncdesc}

\begin{cfuncdesc}{PyObject*}{PyObject_Dir}{PyObject *o}
This is equivalent to the Python expression \samp{dir(\var{o})},
returning a (possibly empty) list of strings appropriate for the
object argument, or \NULL{} in case of error.
If the argument is \NULL{}, this is like the Python \samp{dir()},
returning the names of the current locals; in this case, if no
execution frame is active then \NULL{} is returned but
\cfunction{PyErr_Occurred()} will return false.
\end{cfuncdesc}


\section{Number Protocol \label{number}}

\begin{cfuncdesc}{int}{PyNumber_Check}{PyObject *o}
Returns \code{1} if the object \var{o} provides numeric protocols, and
false otherwise. 
This function always succeeds.
\end{cfuncdesc}


\begin{cfuncdesc}{PyObject*}{PyNumber_Add}{PyObject *o1, PyObject *o2}
Returns the result of adding \var{o1} and \var{o2}, or \NULL{} on
failure.  This is the equivalent of the Python expression
\samp{\var{o1} + \var{o2}}.
\end{cfuncdesc}


\begin{cfuncdesc}{PyObject*}{PyNumber_Subtract}{PyObject *o1, PyObject *o2}
Returns the result of subtracting \var{o2} from \var{o1}, or
\NULL{} on failure.  This is the equivalent of the Python expression
\samp{\var{o1} - \var{o2}}.
\end{cfuncdesc}


\begin{cfuncdesc}{PyObject*}{PyNumber_Multiply}{PyObject *o1, PyObject *o2}
Returns the result of multiplying \var{o1} and \var{o2}, or \NULL{} on
failure.  This is the equivalent of the Python expression
\samp{\var{o1} * \var{o2}}.
\end{cfuncdesc}


\begin{cfuncdesc}{PyObject*}{PyNumber_Divide}{PyObject *o1, PyObject *o2}
Returns the result of dividing \var{o1} by \var{o2}, or \NULL{} on
failure. 
This is the equivalent of the Python expression \samp{\var{o1} /
\var{o2}}.
\end{cfuncdesc}


\begin{cfuncdesc}{PyObject*}{PyNumber_FloorDivide}{PyObject *o1, PyObject *o2}
Return the floor of \var{o1} divided by \var{o2}, or \NULL{} on
failure.  This is equivalent to the ``classic'' division of integers.
\versionadded{2.2}
\end{cfuncdesc}


\begin{cfuncdesc}{PyObject*}{PyNumber_TrueDivide}{PyObject *o1, PyObject *o2}
Return a reasonable approximation for the mathematical value of
\var{o1} divided by \var{o2}, or \NULL{} on failure.  The return value
is ``approximate'' because binary floating point numbers are
approximate; it is not possible to represent all real numbers in base
two.  This function can return a floating point value when passed two
integers.
\versionadded{2.2}
\end{cfuncdesc}


\begin{cfuncdesc}{PyObject*}{PyNumber_Remainder}{PyObject *o1, PyObject *o2}
Returns the remainder of dividing \var{o1} by \var{o2}, or \NULL{} on
failure.  This is the equivalent of the Python expression
\samp{\var{o1} \%\ \var{o2}}.
\end{cfuncdesc}


\begin{cfuncdesc}{PyObject*}{PyNumber_Divmod}{PyObject *o1, PyObject *o2}
See the built-in function \function{divmod()}\bifuncindex{divmod}.
Returns \NULL{} on failure.  This is the equivalent of the Python
expression \samp{divmod(\var{o1}, \var{o2})}.
\end{cfuncdesc}


\begin{cfuncdesc}{PyObject*}{PyNumber_Power}{PyObject *o1,
                                             PyObject *o2, PyObject *o3}
See the built-in function \function{pow()}\bifuncindex{pow}.  Returns
\NULL{} on failure. This is the equivalent of the Python expression
\samp{pow(\var{o1}, \var{o2}, \var{o3})}, where \var{o3} is optional.
If \var{o3} is to be ignored, pass \cdata{Py_None} in its place
(passing \NULL{} for \var{o3} would cause an illegal memory access).
\end{cfuncdesc}


\begin{cfuncdesc}{PyObject*}{PyNumber_Negative}{PyObject *o}
Returns the negation of \var{o} on success, or \NULL{} on failure.
This is the equivalent of the Python expression \samp{-\var{o}}.
\end{cfuncdesc}


\begin{cfuncdesc}{PyObject*}{PyNumber_Positive}{PyObject *o}
Returns \var{o} on success, or \NULL{} on failure.
This is the equivalent of the Python expression \samp{+\var{o}}.
\end{cfuncdesc}


\begin{cfuncdesc}{PyObject*}{PyNumber_Absolute}{PyObject *o}
Returns the absolute value of \var{o}, or \NULL{} on failure.  This is
the equivalent of the Python expression \samp{abs(\var{o})}.
\bifuncindex{abs}
\end{cfuncdesc}


\begin{cfuncdesc}{PyObject*}{PyNumber_Invert}{PyObject *o}
Returns the bitwise negation of \var{o} on success, or \NULL{} on
failure.  This is the equivalent of the Python expression
\samp{\~\var{o}}.
\end{cfuncdesc}


\begin{cfuncdesc}{PyObject*}{PyNumber_Lshift}{PyObject *o1, PyObject *o2}
Returns the result of left shifting \var{o1} by \var{o2} on success,
or \NULL{} on failure.  This is the equivalent of the Python
expression \samp{\var{o1} <\code{<} \var{o2}}.
\end{cfuncdesc}


\begin{cfuncdesc}{PyObject*}{PyNumber_Rshift}{PyObject *o1, PyObject *o2}
Returns the result of right shifting \var{o1} by \var{o2} on success,
or \NULL{} on failure.  This is the equivalent of the Python
expression \samp{\var{o1} >\code{>} \var{o2}}.
\end{cfuncdesc}


\begin{cfuncdesc}{PyObject*}{PyNumber_And}{PyObject *o1, PyObject *o2}
Returns the ``bitwise and'' of \var{o2} and \var{o2} on success and
\NULL{} on failure. This is the equivalent of the Python expression
\samp{\var{o1} \&\ \var{o2}}.
\end{cfuncdesc}


\begin{cfuncdesc}{PyObject*}{PyNumber_Xor}{PyObject *o1, PyObject *o2}
Returns the ``bitwise exclusive or'' of \var{o1} by \var{o2} on success,
or \NULL{} on failure.  This is the equivalent of the Python
expression \samp{\var{o1} \textasciicircum{} \var{o2}}.
\end{cfuncdesc}

\begin{cfuncdesc}{PyObject*}{PyNumber_Or}{PyObject *o1, PyObject *o2}
Returns the ``bitwise or'' of \var{o1} and \var{o2} on success, or
\NULL{} on failure.  This is the equivalent of the Python expression
\samp{\var{o1} | \var{o2}}.
\end{cfuncdesc}


\begin{cfuncdesc}{PyObject*}{PyNumber_InPlaceAdd}{PyObject *o1, PyObject *o2}
Returns the result of adding \var{o1} and \var{o2}, or \NULL{} on
failure.  The operation is done \emph{in-place} when \var{o1} supports
it.  This is the equivalent of the Python statement \samp{\var{o1} +=
\var{o2}}.
\end{cfuncdesc}


\begin{cfuncdesc}{PyObject*}{PyNumber_InPlaceSubtract}{PyObject *o1,
                                                       PyObject *o2}
Returns the result of subtracting \var{o2} from \var{o1}, or
\NULL{} on failure.  The operation is done \emph{in-place} when
\var{o1} supports it.  This is the equivalent of the Python statement
\samp{\var{o1} -= \var{o2}}.
\end{cfuncdesc}


\begin{cfuncdesc}{PyObject*}{PyNumber_InPlaceMultiply}{PyObject *o1,
                                                       PyObject *o2}
Returns the result of multiplying \var{o1} and \var{o2}, or \NULL{} on
failure.  The operation is done \emph{in-place} when \var{o1} supports it. 
This is the equivalent of the Python statement \samp{\var{o1} *= \var{o2}}.
\end{cfuncdesc}


\begin{cfuncdesc}{PyObject*}{PyNumber_InPlaceDivide}{PyObject *o1,
                                                     PyObject *o2}
Returns the result of dividing \var{o1} by \var{o2}, or \NULL{} on
failure.  The operation is done \emph{in-place} when \var{o1} supports
it. This is the equivalent of the Python statement \samp{\var{o1} /=
\var{o2}}.
\end{cfuncdesc}


\begin{cfuncdesc}{PyObject*}{PyNumber_InPlaceFloorDivide}{PyObject *o1,
                                                          PyObject *o2}
Returns the mathematical of dividing \var{o1} by \var{o2}, or \NULL{}
on failure.  The operation is done \emph{in-place} when \var{o1}
supports it.  This is the equivalent of the Python statement
\samp{\var{o1} //= \var{o2}}.
\versionadded{2.2}
\end{cfuncdesc}


\begin{cfuncdesc}{PyObject*}{PyNumber_InPlaceTrueDivide}{PyObject *o1,
                                                         PyObject *o2}
Return a reasonable approximation for the mathematical value of
\var{o1} divided by \var{o2}, or \NULL{} on failure.  The return value
is ``approximate'' because binary floating point numbers are
approximate; it is not possible to represent all real numbers in base
two.  This function can return a floating point value when passed two
integers.  The operation is done \emph{in-place} when \var{o1}
supports it.
\versionadded{2.2}
\end{cfuncdesc}


\begin{cfuncdesc}{PyObject*}{PyNumber_InPlaceRemainder}{PyObject *o1,
                                                        PyObject *o2}
Returns the remainder of dividing \var{o1} by \var{o2}, or \NULL{} on
failure.  The operation is done \emph{in-place} when \var{o1} supports it. 
This is the equivalent of the Python statement \samp{\var{o1} \%= \var{o2}}.
\end{cfuncdesc}


\begin{cfuncdesc}{PyObject*}{PyNumber_InPlacePower}{PyObject *o1,
                                                    PyObject *o2, PyObject *o3}
See the built-in function \function{pow()}.\bifuncindex{pow}  Returns
\NULL{} on failure.  The operation is done \emph{in-place} when
\var{o1} supports it.  This is the equivalent of the Python statement
\samp{\var{o1} **= \var{o2}} when o3 is \cdata{Py_None}, or an
in-place variant of \samp{pow(\var{o1}, \var{o2}, \var{o3})}
otherwise. If \var{o3} is to be ignored, pass \cdata{Py_None} in its
place (passing \NULL{} for \var{o3} would cause an illegal memory
access).
\end{cfuncdesc}

\begin{cfuncdesc}{PyObject*}{PyNumber_InPlaceLshift}{PyObject *o1,
                                                     PyObject *o2}
Returns the result of left shifting \var{o1} by \var{o2} on success,
or \NULL{} on failure.  The operation is done \emph{in-place} when
\var{o1} supports it.  This is the equivalent of the Python statement
\samp{\var{o1} <\code{<=} \var{o2}}.
\end{cfuncdesc}


\begin{cfuncdesc}{PyObject*}{PyNumber_InPlaceRshift}{PyObject *o1,
                                                     PyObject *o2}
Returns the result of right shifting \var{o1} by \var{o2} on success,
or \NULL{} on failure.  The operation is done \emph{in-place} when
\var{o1} supports it.  This is the equivalent of the Python statement
\samp{\var{o1} >\code{>=} \var{o2}}.
\end{cfuncdesc}


\begin{cfuncdesc}{PyObject*}{PyNumber_InPlaceAnd}{PyObject *o1, PyObject *o2}
Returns the ``bitwise and'' of \var{o1} and \var{o2} on success
and \NULL{} on failure. The operation is done \emph{in-place} when
\var{o1} supports it.  This is the equivalent of the Python statement
\samp{\var{o1} \&= \var{o2}}.
\end{cfuncdesc}


\begin{cfuncdesc}{PyObject*}{PyNumber_InPlaceXor}{PyObject *o1, PyObject *o2}
Returns the ``bitwise exclusive or'' of \var{o1} by \var{o2} on
success, or \NULL{} on failure.  The operation is done \emph{in-place}
when \var{o1} supports it.  This is the equivalent of the Python
statement \samp{\var{o1} \textasciicircum= \var{o2}}.
\end{cfuncdesc}

\begin{cfuncdesc}{PyObject*}{PyNumber_InPlaceOr}{PyObject *o1, PyObject *o2}
Returns the ``bitwise or'' of \var{o1} and \var{o2} on success, or
\NULL{} on failure.  The operation is done \emph{in-place} when
\var{o1} supports it.  This is the equivalent of the Python statement
\samp{\var{o1} |= \var{o2}}.
\end{cfuncdesc}

\begin{cfuncdesc}{int}{PyNumber_Coerce}{PyObject **p1, PyObject **p2}
This function takes the addresses of two variables of type
\ctype{PyObject*}.  If the objects pointed to by \code{*\var{p1}} and
\code{*\var{p2}} have the same type, increment their reference count
and return \code{0} (success). If the objects can be converted to a
common numeric type, replace \code{*p1} and \code{*p2} by their
converted value (with 'new' reference counts), and return \code{0}.
If no conversion is possible, or if some other error occurs, return
\code{-1} (failure) and don't increment the reference counts.  The
call \code{PyNumber_Coerce(\&o1, \&o2)} is equivalent to the Python
statement \samp{\var{o1}, \var{o2} = coerce(\var{o1}, \var{o2})}.
\bifuncindex{coerce}
\end{cfuncdesc}

\begin{cfuncdesc}{PyObject*}{PyNumber_Int}{PyObject *o}
Returns the \var{o} converted to an integer object on success, or
\NULL{} on failure.  This is the equivalent of the Python
expression \samp{int(\var{o})}.\bifuncindex{int}
\end{cfuncdesc}

\begin{cfuncdesc}{PyObject*}{PyNumber_Long}{PyObject *o}
Returns the \var{o} converted to a long integer object on success,
or \NULL{} on failure.  This is the equivalent of the Python
expression \samp{long(\var{o})}.\bifuncindex{long}
\end{cfuncdesc}

\begin{cfuncdesc}{PyObject*}{PyNumber_Float}{PyObject *o}
Returns the \var{o} converted to a float object on success, or
\NULL{} on failure.  This is the equivalent of the Python expression
\samp{float(\var{o})}.\bifuncindex{float}
\end{cfuncdesc}


\section{Sequence Protocol \label{sequence}}

\begin{cfuncdesc}{int}{PySequence_Check}{PyObject *o}
Return \code{1} if the object provides sequence protocol, and
\code{0} otherwise.  This function always succeeds.
\end{cfuncdesc}

\begin{cfuncdesc}{int}{PySequence_Size}{PyObject *o}
Returns the number of objects in sequence \var{o} on success, and
\code{-1} on failure.  For objects that do not provide sequence
protocol, this is equivalent to the Python expression
\samp{len(\var{o})}.\bifuncindex{len}
\end{cfuncdesc}

\begin{cfuncdesc}{int}{PySequence_Length}{PyObject *o}
Alternate name for \cfunction{PySequence_Size()}.
\end{cfuncdesc}

\begin{cfuncdesc}{PyObject*}{PySequence_Concat}{PyObject *o1, PyObject *o2}
Return the concatenation of \var{o1} and \var{o2} on success, and \NULL{} on
failure.   This is the equivalent of the Python
expression \samp{\var{o1} + \var{o2}}.
\end{cfuncdesc}


\begin{cfuncdesc}{PyObject*}{PySequence_Repeat}{PyObject *o, int count}
Return the result of repeating sequence object
\var{o} \var{count} times, or \NULL{} on failure.  This is the
equivalent of the Python expression \samp{\var{o} * \var{count}}.
\end{cfuncdesc}

\begin{cfuncdesc}{PyObject*}{PySequence_InPlaceConcat}{PyObject *o1,
                                                       PyObject *o2}
Return the concatenation of \var{o1} and \var{o2} on success, and \NULL{} on
failure.  The operation is done \emph{in-place} when \var{o1} supports it. 
This is the equivalent of the Python expression \samp{\var{o1} += \var{o2}}.
\end{cfuncdesc}


\begin{cfuncdesc}{PyObject*}{PySequence_InPlaceRepeat}{PyObject *o, int count}
Return the result of repeating sequence object \var{o} \var{count} times, or
\NULL{} on failure.  The operation is done \emph{in-place} when \var{o}
supports it.  This is the equivalent of the Python expression \samp{\var{o}
*= \var{count}}.
\end{cfuncdesc}


\begin{cfuncdesc}{PyObject*}{PySequence_GetItem}{PyObject *o, int i}
Return the \var{i}th element of \var{o}, or \NULL{} on failure. This
is the equivalent of the Python expression \samp{\var{o}[\var{i}]}.
\end{cfuncdesc}


\begin{cfuncdesc}{PyObject*}{PySequence_GetSlice}{PyObject *o, int i1, int i2}
Return the slice of sequence object \var{o} between \var{i1} and
\var{i2}, or \NULL{} on failure. This is the equivalent of the Python
expression \samp{\var{o}[\var{i1}:\var{i2}]}.
\end{cfuncdesc}


\begin{cfuncdesc}{int}{PySequence_SetItem}{PyObject *o, int i, PyObject *v}
Assign object \var{v} to the \var{i}th element of \var{o}.
Returns \code{-1} on failure.  This is the equivalent of the Python
statement \samp{\var{o}[\var{i}] = \var{v}}.
\end{cfuncdesc}

\begin{cfuncdesc}{int}{PySequence_DelItem}{PyObject *o, int i}
Delete the \var{i}th element of object \var{o}.  Returns
\code{-1} on failure.  This is the equivalent of the Python
statement \samp{del \var{o}[\var{i}]}.
\end{cfuncdesc}

\begin{cfuncdesc}{int}{PySequence_SetSlice}{PyObject *o, int i1,
                                            int i2, PyObject *v}
Assign the sequence object \var{v} to the slice in sequence
object \var{o} from \var{i1} to \var{i2}.  This is the equivalent of
the Python statement \samp{\var{o}[\var{i1}:\var{i2}] = \var{v}}.
\end{cfuncdesc}

\begin{cfuncdesc}{int}{PySequence_DelSlice}{PyObject *o, int i1, int i2}
Delete the slice in sequence object \var{o} from \var{i1} to \var{i2}.
Returns \code{-1} on failure. This is the equivalent of the Python
statement \samp{del \var{o}[\var{i1}:\var{i2}]}.
\end{cfuncdesc}

\begin{cfuncdesc}{PyObject*}{PySequence_Tuple}{PyObject *o}
Returns the \var{o} as a tuple on success, and \NULL{} on failure.
This is equivalent to the Python expression \samp{tuple(\var{o})}.
\bifuncindex{tuple}
\end{cfuncdesc}

\begin{cfuncdesc}{int}{PySequence_Count}{PyObject *o, PyObject *value}
Return the number of occurrences of \var{value} in \var{o}, that is,
return the number of keys for which \code{\var{o}[\var{key}] ==
\var{value}}.  On failure, return \code{-1}.  This is equivalent to
the Python expression \samp{\var{o}.count(\var{value})}.
\end{cfuncdesc}

\begin{cfuncdesc}{int}{PySequence_Contains}{PyObject *o, PyObject *value}
Determine if \var{o} contains \var{value}.  If an item in \var{o} is
equal to \var{value}, return \code{1}, otherwise return \code{0}.  On
error, return \code{-1}.  This is equivalent to the Python expression
\samp{\var{value} in \var{o}}.
\end{cfuncdesc}

\begin{cfuncdesc}{int}{PySequence_Index}{PyObject *o, PyObject *value}
Return the first index \var{i} for which \code{\var{o}[\var{i}] ==
\var{value}}.  On error, return \code{-1}.    This is equivalent to
the Python expression \samp{\var{o}.index(\var{value})}.
\end{cfuncdesc}

\begin{cfuncdesc}{PyObject*}{PySequence_List}{PyObject *o}
Return a list object with the same contents as the arbitrary sequence
\var{o}.  The returned list is guaranteed to be new.
\end{cfuncdesc}

\begin{cfuncdesc}{PyObject*}{PySequence_Tuple}{PyObject *o}
Return a tuple object with the same contents as the arbitrary sequence
\var{o}.  If \var{o} is a tuple, a new reference will be returned,
otherwise a tuple will be constructed with the appropriate contents.
\end{cfuncdesc}


\begin{cfuncdesc}{PyObject*}{PySequence_Fast}{PyObject *o, const char *m}
Returns the sequence \var{o} as a tuple, unless it is already a
tuple or list, in which case \var{o} is returned.  Use
\cfunction{PySequence_Fast_GET_ITEM()} to access the members of the
result.  Returns \NULL{} on failure.  If the object is not a sequence,
raises \exception{TypeError} with \var{m} as the message text.
\end{cfuncdesc}

\begin{cfuncdesc}{PyObject*}{PySequence_Fast_GET_ITEM}{PyObject *o, int i}
Return the \var{i}th element of \var{o}, assuming that \var{o} was
returned by \cfunction{PySequence_Fast()}, and that \var{i} is within
bounds.  The caller is expected to get the length of the sequence by
calling \cfunction{PySequence_Size()} on \var{o}, since lists and tuples
are guaranteed to always return their true length.
\end{cfuncdesc}


\section{Mapping Protocol \label{mapping}}

\begin{cfuncdesc}{int}{PyMapping_Check}{PyObject *o}
Return \code{1} if the object provides mapping protocol, and
\code{0} otherwise.  This function always succeeds.
\end{cfuncdesc}


\begin{cfuncdesc}{int}{PyMapping_Length}{PyObject *o}
Returns the number of keys in object \var{o} on success, and
\code{-1} on failure.  For objects that do not provide mapping
protocol, this is equivalent to the Python expression
\samp{len(\var{o})}.\bifuncindex{len}
\end{cfuncdesc}


\begin{cfuncdesc}{int}{PyMapping_DelItemString}{PyObject *o, char *key}
Remove the mapping for object \var{key} from the object \var{o}.
Return \code{-1} on failure.  This is equivalent to
the Python statement \samp{del \var{o}[\var{key}]}.
\end{cfuncdesc}


\begin{cfuncdesc}{int}{PyMapping_DelItem}{PyObject *o, PyObject *key}
Remove the mapping for object \var{key} from the object \var{o}.
Return \code{-1} on failure.  This is equivalent to
the Python statement \samp{del \var{o}[\var{key}]}.
\end{cfuncdesc}


\begin{cfuncdesc}{int}{PyMapping_HasKeyString}{PyObject *o, char *key}
On success, return \code{1} if the mapping object has the key
\var{key} and \code{0} otherwise.  This is equivalent to the Python
expression \samp{\var{o}.has_key(\var{key})}. 
This function always succeeds.
\end{cfuncdesc}


\begin{cfuncdesc}{int}{PyMapping_HasKey}{PyObject *o, PyObject *key}
Return \code{1} if the mapping object has the key \var{key} and
\code{0} otherwise.  This is equivalent to the Python expression
\samp{\var{o}.has_key(\var{key})}. 
This function always succeeds.
\end{cfuncdesc}


\begin{cfuncdesc}{PyObject*}{PyMapping_Keys}{PyObject *o}
On success, return a list of the keys in object \var{o}.  On
failure, return \NULL{}. This is equivalent to the Python
expression \samp{\var{o}.keys()}.
\end{cfuncdesc}


\begin{cfuncdesc}{PyObject*}{PyMapping_Values}{PyObject *o}
On success, return a list of the values in object \var{o}.  On
failure, return \NULL{}. This is equivalent to the Python
expression \samp{\var{o}.values()}.
\end{cfuncdesc}


\begin{cfuncdesc}{PyObject*}{PyMapping_Items}{PyObject *o}
On success, return a list of the items in object \var{o}, where
each item is a tuple containing a key-value pair.  On
failure, return \NULL{}. This is equivalent to the Python
expression \samp{\var{o}.items()}.
\end{cfuncdesc}


\begin{cfuncdesc}{PyObject*}{PyMapping_GetItemString}{PyObject *o, char *key}
Return element of \var{o} corresponding to the object \var{key} or
\NULL{} on failure. This is the equivalent of the Python expression
\samp{\var{o}[\var{key}]}.
\end{cfuncdesc}

\begin{cfuncdesc}{int}{PyMapping_SetItemString}{PyObject *o, char *key,
                                                PyObject *v}
Map the object \var{key} to the value \var{v} in object \var{o}.
Returns \code{-1} on failure.  This is the equivalent of the Python
statement \samp{\var{o}[\var{key}] = \var{v}}.
\end{cfuncdesc}


\section{Iterator Protocol \label{iterator}}

\versionadded{2.2}

There are only a couple of functions specifically for working with
iterators.

\begin{cfuncdesc}{int}{PyIter_Check}{PyObject *o}
  Return true if the object \var{o} supports the iterator protocol.
\end{cfuncdesc}

\begin{cfuncdesc}{PyObject*}{PyIter_Next}{PyObject *o}
  Return the next value from the iteration \var{o}.  If the object is
  an iterator, this retrieves the next value from the iteration, and
  returns \NULL{} with no exception set if there are no remaining
  items.  If the object is not an iterator, \exception{TypeError} is
  raised, or if there is an error in retrieving the item, returns
  \NULL{} and passes along the exception.
\end{cfuncdesc}

To write a loop which iterates over an iterator, the C code should
look something like this:

\begin{verbatim}
PyObject *iterator = ...;
PyObject *item;

while (item = PyIter_Next(iter)) {
    /* do something with item */
}
if (PyErr_Occurred()) {
    /* propogate error */
}
else {
    /* continue doing useful work */
}
\end{verbatim}


\chapter{Concrete Objects Layer \label{concrete}}

The functions in this chapter are specific to certain Python object
types.  Passing them an object of the wrong type is not a good idea;
if you receive an object from a Python program and you are not sure
that it has the right type, you must perform a type check first;
for example, to check that an object is a dictionary, use
\cfunction{PyDict_Check()}.  The chapter is structured like the
``family tree'' of Python object types.

\strong{Warning:}
While the functions described in this chapter carefully check the type
of the objects which are passed in, many of them do not check for
\NULL{} being passed instead of a valid object.  Allowing \NULL{} to
be passed in can cause memory access violations and immediate
termination of the interpreter.


\section{Fundamental Objects \label{fundamental}}

This section describes Python type objects and the singleton object 
\code{None}.


\subsection{Type Objects \label{typeObjects}}

\obindex{type}
\begin{ctypedesc}{PyTypeObject}
The C structure of the objects used to describe built-in types.
\end{ctypedesc}

\begin{cvardesc}{PyObject*}{PyType_Type}
This is the type object for type objects; it is the same object as
\code{types.TypeType} in the Python layer.
\withsubitem{(in module types)}{\ttindex{TypeType}}
\end{cvardesc}

\begin{cfuncdesc}{int}{PyType_Check}{PyObject *o}
Returns true is the object \var{o} is a type object.
\end{cfuncdesc}

\begin{cfuncdesc}{int}{PyType_HasFeature}{PyObject *o, int feature}
Returns true if the type object \var{o} sets the feature
\var{feature}.  Type features are denoted by single bit flags.
\end{cfuncdesc}

\begin{cfuncdesc}{int}{PyType_IsSubtype}{PyTypeObject *a, PyTypeObject *b}
Returns true if \var{a} is a subtype of \var{b}.
\versionadded{2.2}
\end{cfuncdesc}

\begin{cfuncdesc}{PyObject*}{PyType_GenericAlloc}{PyTypeObject *type,
                                                  int nitems}
\versionadded{2.2}
\end{cfuncdesc}

\begin{cfuncdesc}{PyObject*}{PyType_GenericNew}{PyTypeObject *type,
                                            PyObject *args, PyObject *kwds}
\versionadded{2.2}
\end{cfuncdesc}

\begin{cfuncdesc}{int}{PyType_Ready}{PyTypeObject *type}
\versionadded{2.2}
\end{cfuncdesc}


\subsection{The None Object \label{noneObject}}

\obindex{None@\texttt{None}}
Note that the \ctype{PyTypeObject} for \code{None} is not directly
exposed in the Python/C API.  Since \code{None} is a singleton,
testing for object identity (using \samp{==} in C) is sufficient.
There is no \cfunction{PyNone_Check()} function for the same reason.

\begin{cvardesc}{PyObject*}{Py_None}
The Python \code{None} object, denoting lack of value.  This object has
no methods.
\end{cvardesc}


\section{Numeric Objects \label{numericObjects}}

\obindex{numeric}


\subsection{Plain Integer Objects \label{intObjects}}

\obindex{integer}
\begin{ctypedesc}{PyIntObject}
This subtype of \ctype{PyObject} represents a Python integer object.
\end{ctypedesc}

\begin{cvardesc}{PyTypeObject}{PyInt_Type}
This instance of \ctype{PyTypeObject} represents the Python plain 
integer type.  This is the same object as \code{types.IntType}.
\withsubitem{(in modules types)}{\ttindex{IntType}}
\end{cvardesc}

\begin{cfuncdesc}{int}{PyInt_Check}{PyObject* o}
Returns true if \var{o} is of type \cdata{PyInt_Type} or a subtype of
\cdata{PyInt_Type}.
\versionchanged[Allowed subtypes to be accepted]{2.2}
\end{cfuncdesc}

\begin{cfuncdesc}{int}{PyInt_CheckExact}{PyObject* o}
Returns true if \var{o} is of type \cdata{PyInt_Type}, but not a
subtype of \cdata{PyInt_Type}.
\versionadded{2.2}
\end{cfuncdesc}

\begin{cfuncdesc}{PyObject*}{PyInt_FromLong}{long ival}
Creates a new integer object with a value of \var{ival}.

The current implementation keeps an array of integer objects for all
integers between \code{-1} and \code{100}, when you create an int in
that range you actually just get back a reference to the existing
object. So it should be possible to change the value of \code{1}. I
suspect the behaviour of Python in this case is undefined. :-)
\end{cfuncdesc}

\begin{cfuncdesc}{long}{PyInt_AsLong}{PyObject *io}
Will first attempt to cast the object to a \ctype{PyIntObject}, if
it is not already one, and then return its value.
\end{cfuncdesc}

\begin{cfuncdesc}{long}{PyInt_AS_LONG}{PyObject *io}
Returns the value of the object \var{io}.  No error checking is
performed.
\end{cfuncdesc}

\begin{cfuncdesc}{long}{PyInt_GetMax}{}
Returns the system's idea of the largest integer it can handle
(\constant{LONG_MAX}\ttindex{LONG_MAX}, as defined in the system
header files).
\end{cfuncdesc}


\subsection{Long Integer Objects \label{longObjects}}

\obindex{long integer}
\begin{ctypedesc}{PyLongObject}
This subtype of \ctype{PyObject} represents a Python long integer
object.
\end{ctypedesc}

\begin{cvardesc}{PyTypeObject}{PyLong_Type}
This instance of \ctype{PyTypeObject} represents the Python long
integer type.  This is the same object as \code{types.LongType}.
\withsubitem{(in modules types)}{\ttindex{LongType}}
\end{cvardesc}

\begin{cfuncdesc}{int}{PyLong_Check}{PyObject *p}
Returns true if its argument is a \ctype{PyLongObject} or a subtype of
\ctype{PyLongObject}.
\versionchanged[Allowed subtypes to be accepted]{2.2}
\end{cfuncdesc}

\begin{cfuncdesc}{int}{PyLong_CheckExact}{PyObject *p}
Returns true if its argument is a \ctype{PyLongObject}, but not a
subtype of \ctype{PyLongObject}.
\versionadded{2.2}
\end{cfuncdesc}

\begin{cfuncdesc}{PyObject*}{PyLong_FromLong}{long v}
Returns a new \ctype{PyLongObject} object from \var{v}, or \NULL{} on
failure.
\end{cfuncdesc}

\begin{cfuncdesc}{PyObject*}{PyLong_FromUnsignedLong}{unsigned long v}
Returns a new \ctype{PyLongObject} object from a C \ctype{unsigned
long}, or \NULL{} on failure.
\end{cfuncdesc}

\begin{cfuncdesc}{PyObject*}{PyLong_FromLongLong}{long long v}
Returns a new \ctype{PyLongObject} object from a C \ctype{long long},
or \NULL{} on failure.
\end{cfuncdesc}

\begin{cfuncdesc}{PyObject*}{PyLong_FromUnsignedLongLong}{unsigned long long v}
Returns a new \ctype{PyLongObject} object from a C \ctype{unsigned
long long}, or \NULL{} on failure.
\end{cfuncdesc}

\begin{cfuncdesc}{PyObject*}{PyLong_FromDouble}{double v}
Returns a new \ctype{PyLongObject} object from the integer part of
\var{v}, or \NULL{} on failure.
\end{cfuncdesc}

\begin{cfuncdesc}{PyObject*}{PyLong_FromString}{char *str, char **pend,
                                                int base}
Return a new \ctype{PyLongObject} based on the string value in
\var{str}, which is interpreted according to the radix in \var{base}.
If \var{pend} is non-\NULL, \code{*\var{pend}} will point to the first 
character in \var{str} which follows the representation of the
number.  If \var{base} is \code{0}, the radix will be determined base
on the leading characters of \var{str}: if \var{str} starts with
\code{'0x'} or \code{'0X'}, radix 16 will be used; if \var{str} starts 
with \code{'0'}, radix 8 will be used; otherwise radix 10 will be
used.  If \var{base} is not \code{0}, it must be between \code{2} and
\code{36}, inclusive.  Leading spaces are ignored.  If there are no
digits, \exception{ValueError} will be raised.
\end{cfuncdesc}

\begin{cfuncdesc}{PyObject*}{PyLong_FromUnicode}{Py_UNICODE *u,
                                                 int length, int base}
Convert a sequence of Unicode digits to a Python long integer value.
The first parameter, \var{u}, points to the first character of the
Unicode string, \var{length} gives the number of characters, and
\var{base} is the radix for the conversion.  The radix must be in the
range [2, 36]; if it is out of range, \exception{ValueError} will be
raised.
\versionadded{1.6}
\end{cfuncdesc}

\begin{cfuncdesc}{PyObject*}{PyLong_FromVoidPtr}{void *p}
Create a Python integer or long integer from the pointer \var{p}.  The
pointer value can be retrieved from the resulting value using
\cfunction{PyLong_AsVoidPtr()}.
\versionadded{1.5.2}
\end{cfuncdesc}

\begin{cfuncdesc}{long}{PyLong_AsLong}{PyObject *pylong}
Returns a C \ctype{long} representation of the contents of
\var{pylong}.  If \var{pylong} is greater than
\constant{LONG_MAX}\ttindex{LONG_MAX}, an \exception{OverflowError} is
raised.\withsubitem{(built-in exception)}{\ttindex{OverflowError}}
\end{cfuncdesc}

\begin{cfuncdesc}{unsigned long}{PyLong_AsUnsignedLong}{PyObject *pylong}
Returns a C \ctype{unsigned long} representation of the contents of 
\var{pylong}.  If \var{pylong} is greater than
\constant{ULONG_MAX}\ttindex{ULONG_MAX}, an \exception{OverflowError}
is raised.\withsubitem{(built-in exception)}{\ttindex{OverflowError}}
\end{cfuncdesc}

\begin{cfuncdesc}{long long}{PyLong_AsLongLong}{PyObject *pylong}
Return a C \ctype{long long} from a Python long integer.  If
\var{pylong} cannot be represented as a \ctype{long long}, an
\exception{OverflowError} will be raised.
\versionadded{2.2}
\end{cfuncdesc}

\begin{cfuncdesc}{unsigned long long}{PyLong_AsUnsignedLongLong}{PyObject
                                                                 *pylong}
Return a C \ctype{unsigned long long} from a Python long integer.  If
\var{pylong} cannot be represented as an \ctype{unsigned long long},
an \exception{OverflowError} will be raised if the value is positive,
or a \exception{TypeError} will be raised if the value is negative.
\versionadded{2.2}
\end{cfuncdesc}

\begin{cfuncdesc}{double}{PyLong_AsDouble}{PyObject *pylong}
Returns a C \ctype{double} representation of the contents of
\var{pylong}.  If \var{pylong} cannot be approximately represented as
a \ctype{double}, an \exception{OverflowError} exception is raised and
\code{-1.0} will be returned.
\end{cfuncdesc}

\begin{cfuncdesc}{void*}{PyLong_AsVoidPtr}{PyObject *pylong}
Convert a Python integer or long integer \var{pylong} to a C
\ctype{void} pointer.  If \var{pylong} cannot be converted, an
\exception{OverflowError} will be raised.  This is only assured to
produce a usable \ctype{void} pointer for values created with
\cfunction{PyLong_FromVoidPtr()}.
\versionadded{1.5.2}
\end{cfuncdesc}


\subsection{Floating Point Objects \label{floatObjects}}

\obindex{floating point}
\begin{ctypedesc}{PyFloatObject}
This subtype of \ctype{PyObject} represents a Python floating point
object.
\end{ctypedesc}

\begin{cvardesc}{PyTypeObject}{PyFloat_Type}
This instance of \ctype{PyTypeObject} represents the Python floating
point type.  This is the same object as \code{types.FloatType}.
\withsubitem{(in modules types)}{\ttindex{FloatType}}
\end{cvardesc}

\begin{cfuncdesc}{int}{PyFloat_Check}{PyObject *p}
Returns true if its argument is a \ctype{PyFloatObject} or a subtype
of \ctype{PyFloatObject}.
\versionchanged[Allowed subtypes to be accepted]{2.2}
\end{cfuncdesc}

\begin{cfuncdesc}{int}{PyFloat_CheckExact}{PyObject *p}
Returns true if its argument is a \ctype{PyFloatObject}, but not a
subtype of \ctype{PyFloatObject}.
\versionadded{2.2}
\end{cfuncdesc}

\begin{cfuncdesc}{PyObject*}{PyFloat_FromDouble}{double v}
Creates a \ctype{PyFloatObject} object from \var{v}, or \NULL{} on
failure.
\end{cfuncdesc}

\begin{cfuncdesc}{double}{PyFloat_AsDouble}{PyObject *pyfloat}
Returns a C \ctype{double} representation of the contents of \var{pyfloat}.
\end{cfuncdesc}

\begin{cfuncdesc}{double}{PyFloat_AS_DOUBLE}{PyObject *pyfloat}
Returns a C \ctype{double} representation of the contents of
\var{pyfloat}, but without error checking.
\end{cfuncdesc}


\subsection{Complex Number Objects \label{complexObjects}}

\obindex{complex number}
Python's complex number objects are implemented as two distinct types
when viewed from the C API:  one is the Python object exposed to
Python programs, and the other is a C structure which represents the
actual complex number value.  The API provides functions for working
with both.

\subsubsection{Complex Numbers as C Structures}

Note that the functions which accept these structures as parameters
and return them as results do so \emph{by value} rather than
dereferencing them through pointers.  This is consistent throughout
the API.

\begin{ctypedesc}{Py_complex}
The C structure which corresponds to the value portion of a Python
complex number object.  Most of the functions for dealing with complex
number objects use structures of this type as input or output values,
as appropriate.  It is defined as:

\begin{verbatim}
typedef struct {
   double real;
   double imag;
} Py_complex;
\end{verbatim}
\end{ctypedesc}

\begin{cfuncdesc}{Py_complex}{_Py_c_sum}{Py_complex left, Py_complex right}
Return the sum of two complex numbers, using the C
\ctype{Py_complex} representation.
\end{cfuncdesc}

\begin{cfuncdesc}{Py_complex}{_Py_c_diff}{Py_complex left, Py_complex right}
Return the difference between two complex numbers, using the C
\ctype{Py_complex} representation.
\end{cfuncdesc}

\begin{cfuncdesc}{Py_complex}{_Py_c_neg}{Py_complex complex}
Return the negation of the complex number \var{complex}, using the C
\ctype{Py_complex} representation.
\end{cfuncdesc}

\begin{cfuncdesc}{Py_complex}{_Py_c_prod}{Py_complex left, Py_complex right}
Return the product of two complex numbers, using the C
\ctype{Py_complex} representation.
\end{cfuncdesc}

\begin{cfuncdesc}{Py_complex}{_Py_c_quot}{Py_complex dividend,
                                          Py_complex divisor}
Return the quotient of two complex numbers, using the C
\ctype{Py_complex} representation.
\end{cfuncdesc}

\begin{cfuncdesc}{Py_complex}{_Py_c_pow}{Py_complex num, Py_complex exp}
Return the exponentiation of \var{num} by \var{exp}, using the C
\ctype{Py_complex} representation.
\end{cfuncdesc}


\subsubsection{Complex Numbers as Python Objects}

\begin{ctypedesc}{PyComplexObject}
This subtype of \ctype{PyObject} represents a Python complex number object.
\end{ctypedesc}

\begin{cvardesc}{PyTypeObject}{PyComplex_Type}
This instance of \ctype{PyTypeObject} represents the Python complex 
number type.
\end{cvardesc}

\begin{cfuncdesc}{int}{PyComplex_Check}{PyObject *p}
Returns true if its argument is a \ctype{PyComplexObject} or a subtype
of \ctype{PyComplexObject}.
\versionchanged[Allowed subtypes to be accepted]{2.2}
\end{cfuncdesc}

\begin{cfuncdesc}{int}{PyComplex_CheckExact}{PyObject *p}
Returns true if its argument is a \ctype{PyComplexObject}, but not a
subtype of \ctype{PyComplexObject}.
\versionadded{2.2}
\end{cfuncdesc}

\begin{cfuncdesc}{PyObject*}{PyComplex_FromCComplex}{Py_complex v}
Create a new Python complex number object from a C
\ctype{Py_complex} value.
\end{cfuncdesc}

\begin{cfuncdesc}{PyObject*}{PyComplex_FromDoubles}{double real, double imag}
Returns a new \ctype{PyComplexObject} object from \var{real} and \var{imag}.
\end{cfuncdesc}

\begin{cfuncdesc}{double}{PyComplex_RealAsDouble}{PyObject *op}
Returns the real part of \var{op} as a C \ctype{double}.
\end{cfuncdesc}

\begin{cfuncdesc}{double}{PyComplex_ImagAsDouble}{PyObject *op}
Returns the imaginary part of \var{op} as a C \ctype{double}.
\end{cfuncdesc}

\begin{cfuncdesc}{Py_complex}{PyComplex_AsCComplex}{PyObject *op}
Returns the \ctype{Py_complex} value of the complex number \var{op}.
\end{cfuncdesc}



\section{Sequence Objects \label{sequenceObjects}}

\obindex{sequence}
Generic operations on sequence objects were discussed in the previous 
chapter; this section deals with the specific kinds of sequence 
objects that are intrinsic to the Python language.


\subsection{String Objects \label{stringObjects}}

These functions raise \exception{TypeError} when expecting a string
parameter and are called with a non-string parameter.

\obindex{string}
\begin{ctypedesc}{PyStringObject}
This subtype of \ctype{PyObject} represents a Python string object.
\end{ctypedesc}

\begin{cvardesc}{PyTypeObject}{PyString_Type}
This instance of \ctype{PyTypeObject} represents the Python string
type; it is the same object as \code{types.TypeType} in the Python
layer.\withsubitem{(in module types)}{\ttindex{StringType}}.
\end{cvardesc}

\begin{cfuncdesc}{int}{PyString_Check}{PyObject *o}
Returns true if the object \var{o} is a string object or an instance
of a subtype of the string type.
\versionchanged[Allowed subtypes to be accepted]{2.2}
\end{cfuncdesc}

\begin{cfuncdesc}{int}{PyString_CheckExact}{PyObject *o}
Returns true if the object \var{o} is a string object, but not an
instance of a subtype of the string type.
\versionadded{2.2}
\end{cfuncdesc}

\begin{cfuncdesc}{PyObject*}{PyString_FromString}{const char *v}
Returns a new string object with the value \var{v} on success, and
\NULL{} on failure.
\end{cfuncdesc}

\begin{cfuncdesc}{PyObject*}{PyString_FromStringAndSize}{const char *v,
                                                         int len}
Returns a new string object with the value \var{v} and length
\var{len} on success, and \NULL{} on failure.  If \var{v} is \NULL{},
the contents of the string are uninitialized.
\end{cfuncdesc}

\begin{cfuncdesc}{PyObject*}{PyString_FromFormat}{const char *format, ...}
Takes a C \code{printf}-style \var{format} string and a variable
number of arguments, calculates the size of the resulting Python
string and returns a string with the values formatted into it.  The
variable arguments must be C types and must correspond exactly to the
format characters in the \var{format} string.  The following format
characters are allowed:
\begin{tableiii}{l|l|l}{member}{Format Characters}{Type}{Comment}
    \lineiii{\%\%}{\emph{n/a}}{The literal \% character.}
    \lineiii{\%c}{int}{A single character, represented as an C int.}
    \lineiii{\%d}{int}{Exactly equivalent to \code{printf("\%d")}.}
    \lineiii{\%ld}{long}{Exactly equivalent to \code{printf("\%ld")}.}
    \lineiii{\%i}{int}{Exactly equivalent to \code{printf("\%i")}.}
    \lineiii{\%x}{int}{Exactly equivalent to \code{printf("\%x")}.}
    \lineiii{\%s}{char*}{A null-terminated C character array.}
    \lineiii{\%p}{void*}{The hex representation of a C pointer.
	Mostly equivalent to \code{printf("\%p")} except that it is
	guaranteed to start with the literal \code{0x} regardless of
	what the platform's \code{printf} yields.}
\end{tableiii}
\end{cfuncdesc}

\begin{cfuncdesc}{PyObject*}{PyString_FromFormatV}{const char *format,
                                                   va_list vargs}
Identical to \function{PyString_FromFormat()} except that it takes
exactly two arguments.
\end{cfuncdesc}

\begin{cfuncdesc}{int}{PyString_Size}{PyObject *string}
Returns the length of the string in string object \var{string}.
\end{cfuncdesc}

\begin{cfuncdesc}{int}{PyString_GET_SIZE}{PyObject *string}
Macro form of \cfunction{PyString_Size()} but without error
checking.
\end{cfuncdesc}

\begin{cfuncdesc}{char*}{PyString_AsString}{PyObject *string}
Returns a null-terminated representation of the contents of
\var{string}.  The pointer refers to the internal buffer of
\var{string}, not a copy.  The data must not be modified in any way,
unless the string was just created using
\code{PyString_FromStringAndSize(NULL, \var{size})}.
It must not be deallocated.
\end{cfuncdesc}

\begin{cfuncdesc}{char*}{PyString_AS_STRING}{PyObject *string}
Macro form of \cfunction{PyString_AsString()} but without error
checking.
\end{cfuncdesc}

\begin{cfuncdesc}{int}{PyString_AsStringAndSize}{PyObject *obj,
                                                 char **buffer,
                                                 int *length}
Returns a null-terminated representation of the contents of the object
\var{obj} through the output variables \var{buffer} and \var{length}.

The function accepts both string and Unicode objects as input. For
Unicode objects it returns the default encoded version of the object.
If \var{length} is set to \NULL{}, the resulting buffer may not contain
null characters; if it does, the function returns -1 and a
TypeError is raised.

The buffer refers to an internal string buffer of \var{obj}, not a
copy. The data must not be modified in any way, unless the string was
just created using \code{PyString_FromStringAndSize(NULL,
\var{size})}.  It must not be deallocated.
\end{cfuncdesc}

\begin{cfuncdesc}{void}{PyString_Concat}{PyObject **string,
                                         PyObject *newpart}
Creates a new string object in \var{*string} containing the
contents of \var{newpart} appended to \var{string}; the caller will
own the new reference.  The reference to the old value of \var{string}
will be stolen.  If the new string
cannot be created, the old reference to \var{string} will still be
discarded and the value of \var{*string} will be set to
\NULL{}; the appropriate exception will be set.
\end{cfuncdesc}

\begin{cfuncdesc}{void}{PyString_ConcatAndDel}{PyObject **string,
                                               PyObject *newpart}
Creates a new string object in \var{*string} containing the contents
of \var{newpart} appended to \var{string}.  This version decrements
the reference count of \var{newpart}.
\end{cfuncdesc}

\begin{cfuncdesc}{int}{_PyString_Resize}{PyObject **string, int newsize}
A way to resize a string object even though it is ``immutable''.  
Only use this to build up a brand new string object; don't use this if
the string may already be known in other parts of the code.
\end{cfuncdesc}

\begin{cfuncdesc}{PyObject*}{PyString_Format}{PyObject *format,
                                              PyObject *args}
Returns a new string object from \var{format} and \var{args}.  Analogous
to \code{\var{format} \%\ \var{args}}.  The \var{args} argument must be
a tuple.
\end{cfuncdesc}

\begin{cfuncdesc}{void}{PyString_InternInPlace}{PyObject **string}
Intern the argument \var{*string} in place.  The argument must be the
address of a pointer variable pointing to a Python string object.
If there is an existing interned string that is the same as
\var{*string}, it sets \var{*string} to it (decrementing the reference 
count of the old string object and incrementing the reference count of
the interned string object), otherwise it leaves \var{*string} alone
and interns it (incrementing its reference count).  (Clarification:
even though there is a lot of talk about reference counts, think of
this function as reference-count-neutral; you own the object after
the call if and only if you owned it before the call.)
\end{cfuncdesc}

\begin{cfuncdesc}{PyObject*}{PyString_InternFromString}{const char *v}
A combination of \cfunction{PyString_FromString()} and
\cfunction{PyString_InternInPlace()}, returning either a new string object
that has been interned, or a new (``owned'') reference to an earlier
interned string object with the same value.
\end{cfuncdesc}

\begin{cfuncdesc}{PyObject*}{PyString_Decode}{const char *s,
                                               int size,
                                               const char *encoding,
                                               const char *errors}
Creates an object by decoding \var{size} bytes of the encoded
buffer \var{s} using the codec registered
for \var{encoding}. \var{encoding} and \var{errors} have the same meaning
as the parameters of the same name in the unicode() builtin
function. The codec to be used is looked up using the Python codec
registry. Returns \NULL{} in case an exception was raised by the
codec.
\end{cfuncdesc}

\begin{cfuncdesc}{PyObject*}{PyString_AsDecodedObject}{PyObject *str,
                                               const char *encoding,
                                               const char *errors}
Decodes a string object by passing it to the codec registered
for \var{encoding} and returns the result as Python 
object. \var{encoding} and \var{errors} have the same meaning as the
parameters of the same name in the string .encode() method. The codec
to be used is looked up using the Python codec registry. Returns
\NULL{} in case an exception was raised by the codec.
\end{cfuncdesc}

\begin{cfuncdesc}{PyObject*}{PyString_Encode}{const char *s,
                                               int size,
                                               const char *encoding,
                                               const char *errors}
Encodes the \ctype{char} buffer of the given size by passing it to 
the codec registered for \var{encoding} and returns a Python object. 
\var{encoding} and \var{errors} have the same
meaning as the parameters of the same name in the string .encode()
method. The codec to be used is looked up using the Python codec
registry. Returns \NULL{} in case an exception was raised by the
codec.
\end{cfuncdesc}

\begin{cfuncdesc}{PyObject*}{PyString_AsEncodedObject}{PyObject *str,
                                               const char *encoding,
                                               const char *errors}
Encodes a string object using the codec registered
for \var{encoding} and returns the result as Python 
object. \var{encoding} and \var{errors} have the same meaning as the
parameters of the same name in the string .encode() method. The codec
to be used is looked up using the Python codec registry. Returns
\NULL{} in case an exception was raised by the codec.
\end{cfuncdesc}


\subsection{Unicode Objects \label{unicodeObjects}}
\sectionauthor{Marc-Andre Lemburg}{mal@lemburg.com}

%--- Unicode Type -------------------------------------------------------

These are the basic Unicode object types used for the Unicode
implementation in Python:

\begin{ctypedesc}{Py_UNICODE}
This type represents a 16-bit unsigned storage type which is used by
Python internally as basis for holding Unicode ordinals. On platforms
where \ctype{wchar_t} is available and also has 16-bits,
\ctype{Py_UNICODE} is a typedef alias for \ctype{wchar_t} to enhance
native platform compatibility. On all other platforms,
\ctype{Py_UNICODE} is a typedef alias for \ctype{unsigned short}.
\end{ctypedesc}

\begin{ctypedesc}{PyUnicodeObject}
This subtype of \ctype{PyObject} represents a Python Unicode object.
\end{ctypedesc}

\begin{cvardesc}{PyTypeObject}{PyUnicode_Type}
This instance of \ctype{PyTypeObject} represents the Python Unicode type.
\end{cvardesc}

%--- These are really C macros... is there a macrodesc TeX macro ?

The following APIs are really C macros and can be used to do fast
checks and to access internal read-only data of Unicode objects:

\begin{cfuncdesc}{int}{PyUnicode_Check}{PyObject *o}
Returns true if the object \var{o} is a Unicode object or an instance
of a Unicode subtype.
\versionchanged[Allowed subtypes to be accepted]{2.2}
\end{cfuncdesc}

\begin{cfuncdesc}{int}{PyUnicode_CheckExact}{PyObject *o}
Returns true if the object \var{o} is a Unicode object, but not an
instance of a subtype.
\versionadded{2.2}
\end{cfuncdesc}

\begin{cfuncdesc}{int}{PyUnicode_GET_SIZE}{PyObject *o}
Returns the size of the object.  o has to be a
PyUnicodeObject (not checked).
\end{cfuncdesc}

\begin{cfuncdesc}{int}{PyUnicode_GET_DATA_SIZE}{PyObject *o}
Returns the size of the object's internal buffer in bytes. o has to be
a PyUnicodeObject (not checked).
\end{cfuncdesc}

\begin{cfuncdesc}{Py_UNICODE*}{PyUnicode_AS_UNICODE}{PyObject *o}
Returns a pointer to the internal Py_UNICODE buffer of the object. o
has to be a PyUnicodeObject (not checked).
\end{cfuncdesc}

\begin{cfuncdesc}{const char*}{PyUnicode_AS_DATA}{PyObject *o}
Returns a (const char *) pointer to the internal buffer of the object.
o has to be a PyUnicodeObject (not checked).
\end{cfuncdesc}

% --- Unicode character properties ---------------------------------------

Unicode provides many different character properties. The most often
needed ones are available through these macros which are mapped to C
functions depending on the Python configuration.

\begin{cfuncdesc}{int}{Py_UNICODE_ISSPACE}{Py_UNICODE ch}
Returns 1/0 depending on whether \var{ch} is a whitespace character.
\end{cfuncdesc}

\begin{cfuncdesc}{int}{Py_UNICODE_ISLOWER}{Py_UNICODE ch}
Returns 1/0 depending on whether \var{ch} is a lowercase character.
\end{cfuncdesc}

\begin{cfuncdesc}{int}{Py_UNICODE_ISUPPER}{Py_UNICODE ch}
Returns 1/0 depending on whether \var{ch} is an uppercase character.
\end{cfuncdesc}

\begin{cfuncdesc}{int}{Py_UNICODE_ISTITLE}{Py_UNICODE ch}
Returns 1/0 depending on whether \var{ch} is a titlecase character.
\end{cfuncdesc}

\begin{cfuncdesc}{int}{Py_UNICODE_ISLINEBREAK}{Py_UNICODE ch}
Returns 1/0 depending on whether \var{ch} is a linebreak character.
\end{cfuncdesc}

\begin{cfuncdesc}{int}{Py_UNICODE_ISDECIMAL}{Py_UNICODE ch}
Returns 1/0 depending on whether \var{ch} is a decimal character.
\end{cfuncdesc}

\begin{cfuncdesc}{int}{Py_UNICODE_ISDIGIT}{Py_UNICODE ch}
Returns 1/0 depending on whether \var{ch} is a digit character.
\end{cfuncdesc}

\begin{cfuncdesc}{int}{Py_UNICODE_ISNUMERIC}{Py_UNICODE ch}
Returns 1/0 depending on whether \var{ch} is a numeric character.
\end{cfuncdesc}

\begin{cfuncdesc}{int}{Py_UNICODE_ISALPHA}{Py_UNICODE ch}
Returns 1/0 depending on whether \var{ch} is an alphabetic character.
\end{cfuncdesc}

\begin{cfuncdesc}{int}{Py_UNICODE_ISALNUM}{Py_UNICODE ch}
Returns 1/0 depending on whether \var{ch} is an alphanumeric character.
\end{cfuncdesc}

These APIs can be used for fast direct character conversions:

\begin{cfuncdesc}{Py_UNICODE}{Py_UNICODE_TOLOWER}{Py_UNICODE ch}
Returns the character \var{ch} converted to lower case.
\end{cfuncdesc}

\begin{cfuncdesc}{Py_UNICODE}{Py_UNICODE_TOUPPER}{Py_UNICODE ch}
Returns the character \var{ch} converted to upper case.
\end{cfuncdesc}

\begin{cfuncdesc}{Py_UNICODE}{Py_UNICODE_TOTITLE}{Py_UNICODE ch}
Returns the character \var{ch} converted to title case.
\end{cfuncdesc}

\begin{cfuncdesc}{int}{Py_UNICODE_TODECIMAL}{Py_UNICODE ch}
Returns the character \var{ch} converted to a decimal positive integer.
Returns -1 in case this is not possible. Does not raise exceptions.
\end{cfuncdesc}

\begin{cfuncdesc}{int}{Py_UNICODE_TODIGIT}{Py_UNICODE ch}
Returns the character \var{ch} converted to a single digit integer.
Returns -1 in case this is not possible. Does not raise exceptions.
\end{cfuncdesc}

\begin{cfuncdesc}{double}{Py_UNICODE_TONUMERIC}{Py_UNICODE ch}
Returns the character \var{ch} converted to a (positive) double.
Returns -1.0 in case this is not possible. Does not raise exceptions.
\end{cfuncdesc}

% --- Plain Py_UNICODE ---------------------------------------------------

To create Unicode objects and access their basic sequence properties,
use these APIs:

\begin{cfuncdesc}{PyObject*}{PyUnicode_FromUnicode}{const Py_UNICODE *u,
                                                    int size} 

Create a Unicode Object from the Py_UNICODE buffer \var{u} of the
given size. \var{u} may be \NULL{} which causes the contents to be
undefined. It is the user's responsibility to fill in the needed data.
The buffer is copied into the new object. If the buffer is not \NULL{},
the return value might be a shared object. Therefore, modification of
the resulting Unicode Object is only allowed when \var{u} is \NULL{}.
\end{cfuncdesc}

\begin{cfuncdesc}{Py_UNICODE*}{PyUnicode_AsUnicode}{PyObject *unicode}
Return a read-only pointer to the Unicode object's internal
\ctype{Py_UNICODE} buffer.
\end{cfuncdesc}

\begin{cfuncdesc}{int}{PyUnicode_GetSize}{PyObject *unicode}
Return the length of the Unicode object.
\end{cfuncdesc}

\begin{cfuncdesc}{PyObject*}{PyUnicode_FromEncodedObject}{PyObject *obj,
                                                      const char *encoding,
                                                      const char *errors}

Coerce an encoded object obj to an Unicode object and return a
reference with incremented refcount.

Coercion is done in the following way:
\begin{enumerate}
\item  Unicode objects are passed back as-is with incremented
      refcount. Note: these cannot be decoded; passing a non-NULL
      value for encoding will result in a TypeError.

\item String and other char buffer compatible objects are decoded
      according to the given encoding and using the error handling
      defined by errors. Both can be NULL to have the interface use
      the default values (see the next section for details).

\item All other objects cause an exception.
\end{enumerate}
The API returns NULL in case of an error. The caller is responsible
for decref'ing the returned objects.
\end{cfuncdesc}

\begin{cfuncdesc}{PyObject*}{PyUnicode_FromObject}{PyObject *obj}

Shortcut for PyUnicode_FromEncodedObject(obj, NULL, ``strict'')
which is used throughout the interpreter whenever coercion to
Unicode is needed.
\end{cfuncdesc}

% --- wchar_t support for platforms which support it ---------------------

If the platform supports \ctype{wchar_t} and provides a header file
wchar.h, Python can interface directly to this type using the
following functions. Support is optimized if Python's own
\ctype{Py_UNICODE} type is identical to the system's \ctype{wchar_t}.

\begin{cfuncdesc}{PyObject*}{PyUnicode_FromWideChar}{const wchar_t *w,
                                                     int size}
Create a Unicode Object from the \ctype{whcar_t} buffer \var{w} of the
given size. Returns \NULL{} on failure.
\end{cfuncdesc}

\begin{cfuncdesc}{int}{PyUnicode_AsWideChar}{PyUnicodeObject *unicode,
                                             wchar_t *w,
                                             int size}
Copies the Unicode Object contents into the \ctype{whcar_t} buffer
\var{w}.  At most \var{size} \ctype{whcar_t} characters are copied.
Returns the number of \ctype{whcar_t} characters copied or -1 in case
of an error.
\end{cfuncdesc}


\subsubsection{Builtin Codecs \label{builtinCodecs}}

Python provides a set of builtin codecs which are written in C
for speed. All of these codecs are directly usable via the
following functions.

Many of the following APIs take two arguments encoding and
errors. These parameters encoding and errors have the same semantics
as the ones of the builtin unicode() Unicode object constructor.

Setting encoding to NULL causes the default encoding to be used which
is \ASCII{}. The file system calls should use
\var{Py_FileSystemDefaultEncoding} as the encoding for file
names. This variable should be treated as read-only: On some systems,
it will be a pointer to a static string, on others, it will change at
run-time, e.g. when the application invokes setlocale.

Error handling is set by errors which may also be set to NULL meaning
to use the default handling defined for the codec. Default error
handling for all builtin codecs is ``strict'' (ValueErrors are raised).

The codecs all use a similar interface. Only deviation from the
following generic ones are documented for simplicity.

% --- Generic Codecs -----------------------------------------------------

These are the generic codec APIs:

\begin{cfuncdesc}{PyObject*}{PyUnicode_Decode}{const char *s,
                                               int size,
                                               const char *encoding,
                                               const char *errors}
Create a Unicode object by decoding \var{size} bytes of the encoded
string \var{s}. \var{encoding} and \var{errors} have the same meaning
as the parameters of the same name in the unicode() builtin
function. The codec to be used is looked up using the Python codec
registry. Returns \NULL{} in case an exception was raised by the
codec.
\end{cfuncdesc}

\begin{cfuncdesc}{PyObject*}{PyUnicode_Encode}{const Py_UNICODE *s,
                                               int size,
                                               const char *encoding,
                                               const char *errors}
Encodes the \ctype{Py_UNICODE} buffer of the given size and returns a
Python string object. \var{encoding} and \var{errors} have the same
meaning as the parameters of the same name in the Unicode .encode()
method. The codec to be used is looked up using the Python codec
registry. Returns \NULL{} in case an exception was raised by the
codec.
\end{cfuncdesc}

\begin{cfuncdesc}{PyObject*}{PyUnicode_AsEncodedString}{PyObject *unicode,
                                               const char *encoding,
                                               const char *errors}
Encodes a Unicode object and returns the result as Python string
object. \var{encoding} and \var{errors} have the same meaning as the
parameters of the same name in the Unicode .encode() method. The codec
to be used is looked up using the Python codec registry. Returns
\NULL{} in case an exception was raised by the codec.
\end{cfuncdesc}

% --- UTF-8 Codecs -------------------------------------------------------

These are the UTF-8 codec APIs:

\begin{cfuncdesc}{PyObject*}{PyUnicode_DecodeUTF8}{const char *s,
                                               int size,
                                               const char *errors}
Creates a Unicode object by decoding \var{size} bytes of the UTF-8
encoded string \var{s}. Returns \NULL{} in case an exception was
raised by the codec.
\end{cfuncdesc}

\begin{cfuncdesc}{PyObject*}{PyUnicode_EncodeUTF8}{const Py_UNICODE *s,
                                               int size,
                                               const char *errors}
Encodes the \ctype{Py_UNICODE} buffer of the given size using UTF-8
and returns a Python string object.  Returns \NULL{} in case an
exception was raised by the codec.
\end{cfuncdesc}

\begin{cfuncdesc}{PyObject*}{PyUnicode_AsUTF8String}{PyObject *unicode}
Encodes a Unicode objects using UTF-8 and returns the result as Python
string object. Error handling is ``strict''. Returns
\NULL{} in case an exception was raised by the codec.
\end{cfuncdesc}

% --- UTF-16 Codecs ------------------------------------------------------ */

These are the UTF-16 codec APIs:

\begin{cfuncdesc}{PyObject*}{PyUnicode_DecodeUTF16}{const char *s,
                                               int size,
                                               const char *errors,
                                               int *byteorder}
Decodes \var{length} bytes from a UTF-16 encoded buffer string and
returns the corresponding Unicode object.

\var{errors} (if non-NULL) defines the error handling. It defaults
to ``strict''.

If \var{byteorder} is non-\NULL{}, the decoder starts decoding using
the given byte order:

\begin{verbatim}
   *byteorder == -1: little endian
   *byteorder == 0:  native order
   *byteorder == 1:  big endian
\end{verbatim}

and then switches according to all byte order marks (BOM) it finds in
the input data. BOM marks are not copied into the resulting Unicode
string.  After completion, \var{*byteorder} is set to the current byte
order at the end of input data.

If \var{byteorder} is \NULL{}, the codec starts in native order mode.

Returns \NULL{} in case an exception was raised by the codec.
\end{cfuncdesc}

\begin{cfuncdesc}{PyObject*}{PyUnicode_EncodeUTF16}{const Py_UNICODE *s,
                                               int size,
                                               const char *errors,
                                               int byteorder}
Returns a Python string object holding the UTF-16 encoded value of the
Unicode data in \var{s}.

If \var{byteorder} is not \code{0}, output is written according to the
following byte order:

\begin{verbatim}
   byteorder == -1: little endian
   byteorder == 0:  native byte order (writes a BOM mark)
   byteorder == 1:  big endian
\end{verbatim}

If byteorder is \code{0}, the output string will always start with the
Unicode BOM mark (U+FEFF). In the other two modes, no BOM mark is
prepended.

Note that \ctype{Py_UNICODE} data is being interpreted as UTF-16
reduced to UCS-2. This trick makes it possible to add full UTF-16
capabilities at a later point without comprimising the APIs.

Returns \NULL{} in case an exception was raised by the codec.
\end{cfuncdesc}

\begin{cfuncdesc}{PyObject*}{PyUnicode_AsUTF16String}{PyObject *unicode}
Returns a Python string using the UTF-16 encoding in native byte
order. The string always starts with a BOM mark. Error handling is
``strict''. Returns \NULL{} in case an exception was raised by the
codec.
\end{cfuncdesc}

% --- Unicode-Escape Codecs ----------------------------------------------

These are the ``Unicode Esacpe'' codec APIs:

\begin{cfuncdesc}{PyObject*}{PyUnicode_DecodeUnicodeEscape}{const char *s,
                                               int size,
                                               const char *errors}
Creates a Unicode object by decoding \var{size} bytes of the Unicode-Esacpe
encoded string \var{s}. Returns \NULL{} in case an exception was
raised by the codec.
\end{cfuncdesc}

\begin{cfuncdesc}{PyObject*}{PyUnicode_EncodeUnicodeEscape}{const Py_UNICODE *s,
                                               int size,
                                               const char *errors}
Encodes the \ctype{Py_UNICODE} buffer of the given size using Unicode-Escape
and returns a Python string object.  Returns \NULL{} in case an
exception was raised by the codec.
\end{cfuncdesc}

\begin{cfuncdesc}{PyObject*}{PyUnicode_AsUnicodeEscapeString}{PyObject *unicode}
Encodes a Unicode objects using Unicode-Escape and returns the result
as Python string object. Error handling is ``strict''. Returns
\NULL{} in case an exception was raised by the codec.
\end{cfuncdesc}

% --- Raw-Unicode-Escape Codecs ------------------------------------------

These are the ``Raw Unicode Esacpe'' codec APIs:

\begin{cfuncdesc}{PyObject*}{PyUnicode_DecodeRawUnicodeEscape}{const char *s,
                                               int size,
                                               const char *errors}
Creates a Unicode object by decoding \var{size} bytes of the Raw-Unicode-Esacpe
encoded string \var{s}. Returns \NULL{} in case an exception was
raised by the codec.
\end{cfuncdesc}

\begin{cfuncdesc}{PyObject*}{PyUnicode_EncodeRawUnicodeEscape}{const Py_UNICODE *s,
                                               int size,
                                               const char *errors}
Encodes the \ctype{Py_UNICODE} buffer of the given size using Raw-Unicode-Escape
and returns a Python string object.  Returns \NULL{} in case an
exception was raised by the codec.
\end{cfuncdesc}

\begin{cfuncdesc}{PyObject*}{PyUnicode_AsRawUnicodeEscapeString}{PyObject *unicode}
Encodes a Unicode objects using Raw-Unicode-Escape and returns the result
as Python string object. Error handling is ``strict''. Returns
\NULL{} in case an exception was raised by the codec.
\end{cfuncdesc}

% --- Latin-1 Codecs ----------------------------------------------------- 

These are the Latin-1 codec APIs:

Latin-1 corresponds to the first 256 Unicode ordinals and only these
are accepted by the codecs during encoding.

\begin{cfuncdesc}{PyObject*}{PyUnicode_DecodeLatin1}{const char *s,
                                                     int size,
                                                     const char *errors}
Creates a Unicode object by decoding \var{size} bytes of the Latin-1
encoded string \var{s}. Returns \NULL{} in case an exception was
raised by the codec.
\end{cfuncdesc}

\begin{cfuncdesc}{PyObject*}{PyUnicode_EncodeLatin1}{const Py_UNICODE *s,
                                                     int size,
                                                     const char *errors}
Encodes the \ctype{Py_UNICODE} buffer of the given size using Latin-1
and returns a Python string object.  Returns \NULL{} in case an
exception was raised by the codec.
\end{cfuncdesc}

\begin{cfuncdesc}{PyObject*}{PyUnicode_AsLatin1String}{PyObject *unicode}
Encodes a Unicode objects using Latin-1 and returns the result as
Python string object. Error handling is ``strict''. Returns
\NULL{} in case an exception was raised by the codec.
\end{cfuncdesc}

% --- ASCII Codecs ------------------------------------------------------- 

These are the \ASCII{} codec APIs.  Only 7-bit \ASCII{} data is
accepted. All other codes generate errors.

\begin{cfuncdesc}{PyObject*}{PyUnicode_DecodeASCII}{const char *s,
                                                    int size,
                                                    const char *errors}
Creates a Unicode object by decoding \var{size} bytes of the
\ASCII{} encoded string \var{s}. Returns \NULL{} in case an exception
was raised by the codec.
\end{cfuncdesc}

\begin{cfuncdesc}{PyObject*}{PyUnicode_EncodeASCII}{const Py_UNICODE *s,
                                                    int size,
                                                    const char *errors}
Encodes the \ctype{Py_UNICODE} buffer of the given size using
\ASCII{} and returns a Python string object.  Returns \NULL{} in case
an exception was raised by the codec.
\end{cfuncdesc}

\begin{cfuncdesc}{PyObject*}{PyUnicode_AsASCIIString}{PyObject *unicode}
Encodes a Unicode objects using \ASCII{} and returns the result as Python
string object. Error handling is ``strict''. Returns
\NULL{} in case an exception was raised by the codec.
\end{cfuncdesc}

% --- Character Map Codecs ----------------------------------------------- 

These are the mapping codec APIs:

This codec is special in that it can be used to implement many
different codecs (and this is in fact what was done to obtain most of
the standard codecs included in the \module{encodings} package). The
codec uses mapping to encode and decode characters.

Decoding mappings must map single string characters to single Unicode
characters, integers (which are then interpreted as Unicode ordinals)
or None (meaning "undefined mapping" and causing an error). 

Encoding mappings must map single Unicode characters to single string
characters, integers (which are then interpreted as Latin-1 ordinals)
or None (meaning "undefined mapping" and causing an error).

The mapping objects provided must only support the __getitem__ mapping
interface.

If a character lookup fails with a LookupError, the character is
copied as-is meaning that its ordinal value will be interpreted as
Unicode or Latin-1 ordinal resp. Because of this, mappings only need
to contain those mappings which map characters to different code
points.

\begin{cfuncdesc}{PyObject*}{PyUnicode_DecodeCharmap}{const char *s,
                                               int size,
                                               PyObject *mapping,
                                               const char *errors}
Creates a Unicode object by decoding \var{size} bytes of the encoded
string \var{s} using the given \var{mapping} object.  Returns \NULL{}
in case an exception was raised by the codec.
\end{cfuncdesc}

\begin{cfuncdesc}{PyObject*}{PyUnicode_EncodeCharmap}{const Py_UNICODE *s,
                                               int size,
                                               PyObject *mapping,
                                               const char *errors}
Encodes the \ctype{Py_UNICODE} buffer of the given size using the
given \var{mapping} object and returns a Python string object.
Returns \NULL{} in case an exception was raised by the codec.
\end{cfuncdesc}

\begin{cfuncdesc}{PyObject*}{PyUnicode_AsCharmapString}{PyObject *unicode,
                                                        PyObject *mapping}
Encodes a Unicode objects using the given \var{mapping} object and
returns the result as Python string object. Error handling is
``strict''. Returns \NULL{} in case an exception was raised by the
codec.
\end{cfuncdesc}

The following codec API is special in that maps Unicode to Unicode.

\begin{cfuncdesc}{PyObject*}{PyUnicode_TranslateCharmap}{const Py_UNICODE *s,
                                               int size,
                                               PyObject *table,
                                               const char *errors}
Translates a \ctype{Py_UNICODE} buffer of the given length by applying
a character mapping \var{table} to it and returns the resulting
Unicode object.  Returns \NULL{} when an exception was raised by the
codec.

The \var{mapping} table must map Unicode ordinal integers to Unicode
ordinal integers or None (causing deletion of the character).

Mapping tables must only provide the __getitem__ interface,
e.g. dictionaries or sequences. Unmapped character ordinals (ones
which cause a LookupError) are left untouched and are copied as-is.
\end{cfuncdesc}

% --- MBCS codecs for Windows --------------------------------------------

These are the MBCS codec APIs. They are currently only available on
Windows and use the Win32 MBCS converters to implement the
conversions.  Note that MBCS (or DBCS) is a class of encodings, not
just one.  The target encoding is defined by the user settings on the
machine running the codec.

\begin{cfuncdesc}{PyObject*}{PyUnicode_DecodeMBCS}{const char *s,
                                               int size,
                                               const char *errors}
Creates a Unicode object by decoding \var{size} bytes of the MBCS
encoded string \var{s}.  Returns \NULL{} in case an exception was
raised by the codec.
\end{cfuncdesc}

\begin{cfuncdesc}{PyObject*}{PyUnicode_EncodeMBCS}{const Py_UNICODE *s,
                                               int size,
                                               const char *errors}
Encodes the \ctype{Py_UNICODE} buffer of the given size using MBCS
and returns a Python string object.  Returns \NULL{} in case an
exception was raised by the codec.
\end{cfuncdesc}

\begin{cfuncdesc}{PyObject*}{PyUnicode_AsMBCSString}{PyObject *unicode}
Encodes a Unicode objects using MBCS and returns the result as Python
string object.  Error handling is ``strict''.  Returns \NULL{} in case
an exception was raised by the codec.
\end{cfuncdesc}

% --- Methods & Slots ----------------------------------------------------

\subsubsection{Methods and Slot Functions \label{unicodeMethodsAndSlots}}

The following APIs are capable of handling Unicode objects and strings
on input (we refer to them as strings in the descriptions) and return
Unicode objects or integers as apporpriate.

They all return \NULL{} or -1 in case an exception occurrs.

\begin{cfuncdesc}{PyObject*}{PyUnicode_Concat}{PyObject *left,
                                               PyObject *right}
Concat two strings giving a new Unicode string.
\end{cfuncdesc}

\begin{cfuncdesc}{PyObject*}{PyUnicode_Split}{PyObject *s,
                                              PyObject *sep,
                                              int maxsplit}
Split a string giving a list of Unicode strings.

If sep is NULL, splitting will be done at all whitespace
substrings. Otherwise, splits occur at the given separator.

At most maxsplit splits will be done. If negative, no limit is set.

Separators are not included in the resulting list.
\end{cfuncdesc}

\begin{cfuncdesc}{PyObject*}{PyUnicode_Splitlines}{PyObject *s,
                                                   int maxsplit}
Split a Unicode string at line breaks, returning a list of Unicode
strings.  CRLF is considered to be one line break.  The Line break
characters are not included in the resulting strings.
\end{cfuncdesc}

\begin{cfuncdesc}{PyObject*}{PyUnicode_Translate}{PyObject *str,
                                                  PyObject *table,
                                                  const char *errors}
Translate a string by applying a character mapping table to it and
return the resulting Unicode object.

The mapping table must map Unicode ordinal integers to Unicode ordinal
integers or None (causing deletion of the character).

Mapping tables must only provide the __getitem__ interface,
e.g. dictionaries or sequences. Unmapped character ordinals (ones
which cause a LookupError) are left untouched and are copied as-is.

\var{errors} has the usual meaning for codecs. It may be \NULL{}
which indicates to use the default error handling.
\end{cfuncdesc}

\begin{cfuncdesc}{PyObject*}{PyUnicode_Join}{PyObject *separator,
                                             PyObject *seq}
Join a sequence of strings using the given separator and return
the resulting Unicode string.
\end{cfuncdesc}

\begin{cfuncdesc}{PyObject*}{PyUnicode_Tailmatch}{PyObject *str,
                                                  PyObject *substr,
                                                  int start,
                                                  int end,
                                                  int direction}
Return 1 if \var{substr} matches \var{str}[\var{start}:\var{end}] at
the given tail end (\var{direction} == -1 means to do a prefix match,
\var{direction} == 1 a suffix match), 0 otherwise.
\end{cfuncdesc}

\begin{cfuncdesc}{PyObject*}{PyUnicode_Find}{PyObject *str,
                                                  PyObject *substr,
                                                  int start,
                                                  int end,
                                                  int direction}
Return the first position of \var{substr} in
\var{str}[\var{start}:\var{end}] using the given \var{direction}
(\var{direction} == 1 means to do a forward search,
\var{direction} == -1 a backward search), 0 otherwise.
\end{cfuncdesc}

\begin{cfuncdesc}{PyObject*}{PyUnicode_Count}{PyObject *str,
                                                  PyObject *substr,
                                                  int start,
                                                  int end}
Count the number of occurrences of \var{substr} in
\var{str}[\var{start}:\var{end}]
\end{cfuncdesc}

\begin{cfuncdesc}{PyObject*}{PyUnicode_Replace}{PyObject *str,
                                                PyObject *substr,
                                                PyObject *replstr,
                                                int maxcount}
Replace at most \var{maxcount} occurrences of \var{substr} in
\var{str} with \var{replstr} and return the resulting Unicode object.
\var{maxcount} == -1 means: replace all occurrences.
\end{cfuncdesc}

\begin{cfuncdesc}{int}{PyUnicode_Compare}{PyObject *left, PyObject *right}
Compare two strings and return -1, 0, 1 for less than, equal,
greater than resp.
\end{cfuncdesc}

\begin{cfuncdesc}{PyObject*}{PyUnicode_Format}{PyObject *format,
                                              PyObject *args}
Returns a new string object from \var{format} and \var{args}; this is
analogous to \code{\var{format} \%\ \var{args}}.  The
\var{args} argument must be a tuple.
\end{cfuncdesc}

\begin{cfuncdesc}{int}{PyUnicode_Contains}{PyObject *container,
                                           PyObject *element}
Checks whether \var{element} is contained in \var{container} and
returns true or false accordingly.

\var{element} has to coerce to a one element Unicode string. \code{-1} is
returned in case of an error.
\end{cfuncdesc}


\subsection{Buffer Objects \label{bufferObjects}}
\sectionauthor{Greg Stein}{gstein@lyra.org}

\obindex{buffer}
Python objects implemented in C can export a group of functions called
the ``buffer\index{buffer interface} interface.''  These functions can
be used by an object to expose its data in a raw, byte-oriented
format. Clients of the object can use the buffer interface to access
the object data directly, without needing to copy it first.

Two examples of objects that support 
the buffer interface are strings and arrays. The string object exposes 
the character contents in the buffer interface's byte-oriented
form. An array can also expose its contents, but it should be noted
that array elements may be multi-byte values.

An example user of the buffer interface is the file object's
\method{write()} method. Any object that can export a series of bytes
through the buffer interface can be written to a file. There are a
number of format codes to \cfunction{PyArg_ParseTuple()} that operate 
against an object's buffer interface, returning data from the target
object.

More information on the buffer interface is provided in the section
``Buffer Object Structures'' (section \ref{buffer-structs}), under
the description for \ctype{PyBufferProcs}\ttindex{PyBufferProcs}.

A ``buffer object'' is defined in the \file{bufferobject.h} header
(included by \file{Python.h}). These objects look very similar to
string objects at the Python programming level: they support slicing,
indexing, concatenation, and some other standard string
operations. However, their data can come from one of two sources: from
a block of memory, or from another object which exports the buffer
interface.

Buffer objects are useful as a way to expose the data from another
object's buffer interface to the Python programmer. They can also be
used as a zero-copy slicing mechanism. Using their ability to
reference a block of memory, it is possible to expose any data to the
Python programmer quite easily. The memory could be a large, constant
array in a C extension, it could be a raw block of memory for
manipulation before passing to an operating system library, or it
could be used to pass around structured data in its native, in-memory
format.

\begin{ctypedesc}{PyBufferObject}
This subtype of \ctype{PyObject} represents a buffer object.
\end{ctypedesc}

\begin{cvardesc}{PyTypeObject}{PyBuffer_Type}
The instance of \ctype{PyTypeObject} which represents the Python
buffer type; it is the same object as \code{types.BufferType} in the
Python layer.\withsubitem{(in module types)}{\ttindex{BufferType}}.
\end{cvardesc}

\begin{cvardesc}{int}{Py_END_OF_BUFFER}
This constant may be passed as the \var{size} parameter to
\cfunction{PyBuffer_FromObject()} or
\cfunction{PyBuffer_FromReadWriteObject()}. It indicates that the new
\ctype{PyBufferObject} should refer to \var{base} object from the
specified \var{offset} to the end of its exported buffer. Using this
enables the caller to avoid querying the \var{base} object for its
length.
\end{cvardesc}

\begin{cfuncdesc}{int}{PyBuffer_Check}{PyObject *p}
Return true if the argument has type \cdata{PyBuffer_Type}.
\end{cfuncdesc}

\begin{cfuncdesc}{PyObject*}{PyBuffer_FromObject}{PyObject *base,
                                                  int offset, int size}
Return a new read-only buffer object.  This raises
\exception{TypeError} if \var{base} doesn't support the read-only
buffer protocol or doesn't provide exactly one buffer segment, or it
raises \exception{ValueError} if \var{offset} is less than zero. The
buffer will hold a reference to the \var{base} object, and the
buffer's contents will refer to the \var{base} object's buffer
interface, starting as position \var{offset} and extending for
\var{size} bytes. If \var{size} is \constant{Py_END_OF_BUFFER}, then
the new buffer's contents extend to the length of the
\var{base} object's exported buffer data.
\end{cfuncdesc}

\begin{cfuncdesc}{PyObject*}{PyBuffer_FromReadWriteObject}{PyObject *base,
                                                           int offset,
                                                           int size}
Return a new writable buffer object.  Parameters and exceptions are
similar to those for \cfunction{PyBuffer_FromObject()}.
If the \var{base} object does not export the writeable buffer
protocol, then \exception{TypeError} is raised.
\end{cfuncdesc}

\begin{cfuncdesc}{PyObject*}{PyBuffer_FromMemory}{void *ptr, int size}
Return a new read-only buffer object that reads from a specified
location in memory, with a specified size.
The caller is responsible for ensuring that the memory buffer, passed
in as \var{ptr}, is not deallocated while the returned buffer object
exists.  Raises \exception{ValueError} if \var{size} is less than
zero.  Note that \constant{Py_END_OF_BUFFER} may \emph{not} be passed
for the \var{size} parameter; \exception{ValueError} will be raised in 
that case.
\end{cfuncdesc}

\begin{cfuncdesc}{PyObject*}{PyBuffer_FromReadWriteMemory}{void *ptr, int size}
Similar to \cfunction{PyBuffer_FromMemory()}, but the returned buffer
is writable.
\end{cfuncdesc}

\begin{cfuncdesc}{PyObject*}{PyBuffer_New}{int size}
Returns a new writable buffer object that maintains its own memory
buffer of \var{size} bytes.  \exception{ValueError} is returned if
\var{size} is not zero or positive.
\end{cfuncdesc}


\subsection{Tuple Objects \label{tupleObjects}}

\obindex{tuple}
\begin{ctypedesc}{PyTupleObject}
This subtype of \ctype{PyObject} represents a Python tuple object.
\end{ctypedesc}

\begin{cvardesc}{PyTypeObject}{PyTuple_Type}
This instance of \ctype{PyTypeObject} represents the Python tuple
type; it is the same object as \code{types.TupleType} in the Python
layer.\withsubitem{(in module types)}{\ttindex{TupleType}}.
\end{cvardesc}

\begin{cfuncdesc}{int}{PyTuple_Check}{PyObject *p}
Return true if \var{p} is a tuple object or an instance of a subtype
of the tuple type.
\versionchanged[Allowed subtypes to be accepted]{2.2}
\end{cfuncdesc}

\begin{cfuncdesc}{int}{PyTuple_CheckExact}{PyObject *p}
Return true if \var{p} is a tuple object, but not an instance of
a subtype of the tuple type.
\versionadded{2.2}
\end{cfuncdesc}

\begin{cfuncdesc}{PyObject*}{PyTuple_New}{int len}
Return a new tuple object of size \var{len}, or \NULL{} on failure.
\end{cfuncdesc}

\begin{cfuncdesc}{int}{PyTuple_Size}{PyObject *p}
Takes a pointer to a tuple object, and returns the size
of that tuple.
\end{cfuncdesc}

\begin{cfuncdesc}{int}{PyTuple_GET_SIZE}{PyObject *p}
Return the size of the tuple \var{p}, which must be non-\NULL{} and
point to a tuple; no error checking is performed.
\end{cfuncdesc}

\begin{cfuncdesc}{PyObject*}{PyTuple_GetItem}{PyObject *p, int pos}
Returns the object at position \var{pos} in the tuple pointed
to by \var{p}.  If \var{pos} is out of bounds, returns \NULL{} and
sets an \exception{IndexError} exception.
\end{cfuncdesc}

\begin{cfuncdesc}{PyObject*}{PyTuple_GET_ITEM}{PyObject *p, int pos}
Like \cfunction{PyTuple_GetItem()}, but does no checking of its
arguments.
\end{cfuncdesc}

\begin{cfuncdesc}{PyObject*}{PyTuple_GetSlice}{PyObject *p,
                                               int low, int high}
Takes a slice of the tuple pointed to by \var{p} from
\var{low} to \var{high} and returns it as a new tuple.
\end{cfuncdesc}

\begin{cfuncdesc}{int}{PyTuple_SetItem}{PyObject *p,
                                        int pos, PyObject *o}
Inserts a reference to object \var{o} at position \var{pos} of
the tuple pointed to by \var{p}. It returns \code{0} on success.
\strong{Note:}  This function ``steals'' a reference to \var{o}.
\end{cfuncdesc}

\begin{cfuncdesc}{void}{PyTuple_SET_ITEM}{PyObject *p,
                                          int pos, PyObject *o}
Like \cfunction{PyTuple_SetItem()}, but does no error checking, and
should \emph{only} be used to fill in brand new tuples.
\strong{Note:}  This function ``steals'' a reference to \var{o}.
\end{cfuncdesc}

\begin{cfuncdesc}{int}{_PyTuple_Resize}{PyObject **p, int newsize}
Can be used to resize a tuple.  \var{newsize} will be the new length
of the tuple.  Because tuples are \emph{supposed} to be immutable,
this should only be used if there is only one reference to the object.
Do \emph{not} use this if the tuple may already be known to some other
part of the code.  The tuple will always grow or shrink at the end.
Think of this as destroying the old tuple and creating a new one, only
more efficiently.  Returns \code{0} on success.  Client code should
never assume that the resulting value of \code{*\var{p}} will be the
same as before calling this function.  If the object referenced by
\code{*\var{p}} is replaced, the original \code{*\var{p}} is
destroyed.  On failure, returns \code{-1} and sets \code{*\var{p}} to
\NULL, and raises \exception{MemoryError} or \exception{SystemError}.
\versionchanged[Removed unused third parameter, \var{last_is_sticky}]{2.2}
\end{cfuncdesc}


\subsection{List Objects \label{listObjects}}

\obindex{list}
\begin{ctypedesc}{PyListObject}
This subtype of \ctype{PyObject} represents a Python list object.
\end{ctypedesc}

\begin{cvardesc}{PyTypeObject}{PyList_Type}
This instance of \ctype{PyTypeObject} represents the Python list
type.  This is the same object as \code{types.ListType}.
\withsubitem{(in module types)}{\ttindex{ListType}}
\end{cvardesc}

\begin{cfuncdesc}{int}{PyList_Check}{PyObject *p}
Returns true if its argument is a \ctype{PyListObject}.
\end{cfuncdesc}

\begin{cfuncdesc}{PyObject*}{PyList_New}{int len}
Returns a new list of length \var{len} on success, or \NULL{} on
failure.
\end{cfuncdesc}

\begin{cfuncdesc}{int}{PyList_Size}{PyObject *list}
Returns the length of the list object in \var{list}; this is
equivalent to \samp{len(\var{list})} on a list object.
\bifuncindex{len}
\end{cfuncdesc}

\begin{cfuncdesc}{int}{PyList_GET_SIZE}{PyObject *list}
Macro form of \cfunction{PyList_Size()} without error checking.
\end{cfuncdesc}

\begin{cfuncdesc}{PyObject*}{PyList_GetItem}{PyObject *list, int index}
Returns the object at position \var{pos} in the list pointed
to by \var{p}.  If \var{pos} is out of bounds, returns \NULL{} and
sets an \exception{IndexError} exception.
\end{cfuncdesc}

\begin{cfuncdesc}{PyObject*}{PyList_GET_ITEM}{PyObject *list, int i}
Macro form of \cfunction{PyList_GetItem()} without error checking.
\end{cfuncdesc}

\begin{cfuncdesc}{int}{PyList_SetItem}{PyObject *list, int index,
                                       PyObject *item}
Sets the item at index \var{index} in list to \var{item}.
Returns \code{0} on success or \code{-1} on failure.
\strong{Note:}  This function ``steals'' a reference to \var{item} and
discards a reference to an item already in the list at the affected
position.
\end{cfuncdesc}

\begin{cfuncdesc}{void}{PyList_SET_ITEM}{PyObject *list, int i,
                                              PyObject *o}
Macro form of \cfunction{PyList_SetItem()} without error checking.
\strong{Note:}  This function ``steals'' a reference to \var{item},
and, unlike \cfunction{PyList_SetItem()}, does \emph{not} discard a
reference to any item that it being replaced; any reference in
\var{list} at position \var{i} will be leaked.  This is normally only
used to fill in new lists where there is no previous content.
\end{cfuncdesc}

\begin{cfuncdesc}{int}{PyList_Insert}{PyObject *list, int index,
                                      PyObject *item}
Inserts the item \var{item} into list \var{list} in front of index
\var{index}.  Returns \code{0} if successful; returns \code{-1} and
raises an exception if unsuccessful.  Analogous to
\code{\var{list}.insert(\var{index}, \var{item})}.
\end{cfuncdesc}

\begin{cfuncdesc}{int}{PyList_Append}{PyObject *list, PyObject *item}
Appends the object \var{item} at the end of list \var{list}.  Returns
\code{0} if successful; returns \code{-1} and sets an exception if
unsuccessful.  Analogous to \code{\var{list}.append(\var{item})}.
\end{cfuncdesc}

\begin{cfuncdesc}{PyObject*}{PyList_GetSlice}{PyObject *list,
                                              int low, int high}
Returns a list of the objects in \var{list} containing the objects 
\emph{between} \var{low} and \var{high}.  Returns NULL and sets an
exception if unsuccessful.
Analogous to \code{\var{list}[\var{low}:\var{high}]}.
\end{cfuncdesc}

\begin{cfuncdesc}{int}{PyList_SetSlice}{PyObject *list,
                                        int low, int high,
                                        PyObject *itemlist}
Sets the slice of \var{list} between \var{low} and \var{high} to the
contents of \var{itemlist}.  Analogous to
\code{\var{list}[\var{low}:\var{high}] = \var{itemlist}}.  Returns
\code{0} on success, \code{-1} on failure.
\end{cfuncdesc}

\begin{cfuncdesc}{int}{PyList_Sort}{PyObject *list}
Sorts the items of \var{list} in place.  Returns \code{0} on success,
\code{-1} on failure.  This is equivalent to
\samp{\var{list}.sort()}.
\end{cfuncdesc}

\begin{cfuncdesc}{int}{PyList_Reverse}{PyObject *list}
Reverses the items of \var{list} in place.  Returns \code{0} on
success, \code{-1} on failure.  This is the equivalent of
\samp{\var{list}.reverse()}.
\end{cfuncdesc}

\begin{cfuncdesc}{PyObject*}{PyList_AsTuple}{PyObject *list}
Returns a new tuple object containing the contents of \var{list};
equivalent to \samp{tuple(\var{list})}.\bifuncindex{tuple}
\end{cfuncdesc}


\section{Mapping Objects \label{mapObjects}}

\obindex{mapping}


\subsection{Dictionary Objects \label{dictObjects}}

\obindex{dictionary}
\begin{ctypedesc}{PyDictObject}
This subtype of \ctype{PyObject} represents a Python dictionary object.
\end{ctypedesc}

\begin{cvardesc}{PyTypeObject}{PyDict_Type}
This instance of \ctype{PyTypeObject} represents the Python dictionary 
type.  This is exposed to Python programs as \code{types.DictType} and 
\code{types.DictionaryType}.
\withsubitem{(in module types)}{\ttindex{DictType}\ttindex{DictionaryType}}
\end{cvardesc}

\begin{cfuncdesc}{int}{PyDict_Check}{PyObject *p}
Returns true if its argument is a \ctype{PyDictObject}.
\end{cfuncdesc}

\begin{cfuncdesc}{PyObject*}{PyDict_New}{}
Returns a new empty dictionary, or \NULL{} on failure.
\end{cfuncdesc}

\begin{cfuncdesc}{PyObject*}{PyDictProxy_New}{PyObject *dict}
Return a proxy object for a mapping which enforces read-only
behavior.  This is normally used to create a proxy to prevent
modification of the dictionary for non-dynamic class types.
\versionadded{2.2}
\end{cfuncdesc}

\begin{cfuncdesc}{void}{PyDict_Clear}{PyObject *p}
Empties an existing dictionary of all key-value pairs.
\end{cfuncdesc}

\begin{cfuncdesc}{PyObject*}{PyDict_Copy}{PyObject *p}
Returns a new dictionary that contains the same key-value pairs as
\var{p}.
\versionadded{1.6}
\end{cfuncdesc}

\begin{cfuncdesc}{int}{PyDict_SetItem}{PyObject *p, PyObject *key,
                                       PyObject *val}
Inserts \var{value} into the dictionary \var{p} with a key of \var{key}.
\var{key} must be hashable; if it isn't, \exception{TypeError} will be 
raised.
Returns \code{0} on success or \code{-1} on failure.
\end{cfuncdesc}

\begin{cfuncdesc}{int}{PyDict_SetItemString}{PyObject *p,
            char *key,
            PyObject *val}
Inserts \var{value} into the dictionary \var{p} using \var{key}
as a key. \var{key} should be a \ctype{char*}.  The key object is
created using \code{PyString_FromString(\var{key})}.
Returns \code{0} on success or \code{-1} on failure.
\ttindex{PyString_FromString()}
\end{cfuncdesc}

\begin{cfuncdesc}{int}{PyDict_DelItem}{PyObject *p, PyObject *key}
Removes the entry in dictionary \var{p} with key \var{key}.
\var{key} must be hashable; if it isn't, \exception{TypeError} is
raised.
\end{cfuncdesc}

\begin{cfuncdesc}{int}{PyDict_DelItemString}{PyObject *p, char *key}
Removes the entry in dictionary \var{p} which has a key
specified by the string \var{key}.
Returns \code{0} on success or \code{-1} on failure.
\end{cfuncdesc}

\begin{cfuncdesc}{PyObject*}{PyDict_GetItem}{PyObject *p, PyObject *key}
Returns the object from dictionary \var{p} which has a key
\var{key}.  Returns \NULL{} if the key \var{key} is not present, but
\emph{without} setting an exception.
\end{cfuncdesc}

\begin{cfuncdesc}{PyObject*}{PyDict_GetItemString}{PyObject *p, char *key}
This is the same as \cfunction{PyDict_GetItem()}, but \var{key} is
specified as a \ctype{char*}, rather than a \ctype{PyObject*}.
\end{cfuncdesc}

\begin{cfuncdesc}{PyObject*}{PyDict_Items}{PyObject *p}
Returns a \ctype{PyListObject} containing all the items 
from the dictionary, as in the dictinoary method \method{items()} (see
the \citetitle[../lib/lib.html]{Python Library Reference}).
\end{cfuncdesc}

\begin{cfuncdesc}{PyObject*}{PyDict_Keys}{PyObject *p}
Returns a \ctype{PyListObject} containing all the keys 
from the dictionary, as in the dictionary method \method{keys()} (see the
\citetitle[../lib/lib.html]{Python Library Reference}).
\end{cfuncdesc}

\begin{cfuncdesc}{PyObject*}{PyDict_Values}{PyObject *p}
Returns a \ctype{PyListObject} containing all the values 
from the dictionary \var{p}, as in the dictionary method
\method{values()} (see the \citetitle[../lib/lib.html]{Python Library
Reference}).
\end{cfuncdesc}

\begin{cfuncdesc}{int}{PyDict_Size}{PyObject *p}
Returns the number of items in the dictionary.  This is equivalent to
\samp{len(\var{p})} on a dictionary.\bifuncindex{len}
\end{cfuncdesc}

\begin{cfuncdesc}{int}{PyDict_Next}{PyObject *p, int *ppos,
                                    PyObject **pkey, PyObject **pvalue}
Iterate over all key-value pairs in the dictionary \var{p}.  The
\ctype{int} referred to by \var{ppos} must be initialized to \code{0}
prior to the first call to this function to start the iteration; the
function returns true for each pair in the dictionary, and false once
all pairs have been reported.  The parameters \var{pkey} and
\var{pvalue} should either point to \ctype{PyObject*} variables that
will be filled in with each key and value, respectively, or may be
\NULL.

For example:

\begin{verbatim}
PyObject *key, *value;
int pos = 0;

while (PyDict_Next(self->dict, &pos, &key, &value)) {
    /* do something interesting with the values... */
    ...
}
\end{verbatim}

The dictionary \var{p} should not be mutated during iteration.  It is
safe (since Python 2.1) to modify the values of the keys as you
iterate over the dictionary, but only so long as the set of keys does
not change.  For example:

\begin{verbatim}
PyObject *key, *value;
int pos = 0;

while (PyDict_Next(self->dict, &pos, &key, &value)) {
    int i = PyInt_AS_LONG(value) + 1;
    PyObject *o = PyInt_FromLong(i);
    if (o == NULL)
        return -1;
    if (PyDict_SetItem(self->dict, key, o) < 0) {
        Py_DECREF(o);
        return -1;
    }
    Py_DECREF(o);
}
\end{verbatim}
\end{cfuncdesc}

\begin{cfuncdesc}{int}{PyDict_Merge}{PyObject *a, PyObject *b, int override}
Iterate over dictionary \var{b} adding key-value pairs to dictionary
\var{a}.  If \var{override} is true, existing pairs in \var{a} will be
replaced if a matching key is found in \var{b}, otherwise pairs will
only be added if there is not a matching key in \var{a}.  Returns
\code{0} on success or \code{-1} if an exception was raised.
\versionadded{2.2}
\end{cfuncdesc}

\begin{cfuncdesc}{int}{PyDict_Update}{PyObject *a, PyObject *b}
This is the same as \code{PyDict_Merge(\var{a}, \var{b}, 1)} in C, or
\code{\var{a}.update(\var{b})} in Python.  Returns \code{0} on success
or \code{-1} if an exception was raised.
\versionadded{2.2}
\end{cfuncdesc}


\section{Other Objects \label{otherObjects}}

\subsection{File Objects \label{fileObjects}}

\obindex{file}
Python's built-in file objects are implemented entirely on the
\ctype{FILE*} support from the C standard library.  This is an
implementation detail and may change in future releases of Python.

\begin{ctypedesc}{PyFileObject}
This subtype of \ctype{PyObject} represents a Python file object.
\end{ctypedesc}

\begin{cvardesc}{PyTypeObject}{PyFile_Type}
This instance of \ctype{PyTypeObject} represents the Python file
type.  This is exposed to Python programs as \code{types.FileType}.
\withsubitem{(in module types)}{\ttindex{FileType}}
\end{cvardesc}

\begin{cfuncdesc}{int}{PyFile_Check}{PyObject *p}
Returns true if its argument is a \ctype{PyFileObject} or a subtype of
\ctype{PyFileObject}.
\versionchanged[Allowed subtypes to be accepted]{2.2}
\end{cfuncdesc}

\begin{cfuncdesc}{int}{PyFile_CheckExact}{PyObject *p}
Returns true if its argument is a \ctype{PyFileObject}, but not a
subtype of \ctype{PyFileObject}.
\versionadded{2.2}
\end{cfuncdesc}

\begin{cfuncdesc}{PyObject*}{PyFile_FromString}{char *filename, char *mode}
On success, returns a new file object that is opened on the
file given by \var{filename}, with a file mode given by \var{mode},
where \var{mode} has the same semantics as the standard C routine
\cfunction{fopen()}\ttindex{fopen()}.  On failure, returns \NULL.
\end{cfuncdesc}

\begin{cfuncdesc}{PyObject*}{PyFile_FromFile}{FILE *fp,
                                              char *name, char *mode,
                                              int (*close)(FILE*)}
Creates a new \ctype{PyFileObject} from the already-open standard C
file pointer, \var{fp}.  The function \var{close} will be called when
the file should be closed.  Returns \NULL{} on failure.
\end{cfuncdesc}

\begin{cfuncdesc}{FILE*}{PyFile_AsFile}{PyFileObject *p}
Returns the file object associated with \var{p} as a \ctype{FILE*}.
\end{cfuncdesc}

\begin{cfuncdesc}{PyObject*}{PyFile_GetLine}{PyObject *p, int n}
Equivalent to \code{\var{p}.readline(\optional{\var{n}})}, this
function reads one line from the object \var{p}.  \var{p} may be a
file object or any object with a \method{readline()} method.  If
\var{n} is \code{0}, exactly one line is read, regardless of the
length of the line.  If \var{n} is greater than \code{0}, no more than 
\var{n} bytes will be read from the file; a partial line can be
returned.  In both cases, an empty string is returned if the end of
the file is reached immediately.  If \var{n} is less than \code{0},
however, one line is read regardless of length, but
\exception{EOFError} is raised if the end of the file is reached
immediately.
\withsubitem{(built-in exception)}{\ttindex{EOFError}}
\end{cfuncdesc}

\begin{cfuncdesc}{PyObject*}{PyFile_Name}{PyObject *p}
Returns the name of the file specified by \var{p} as a string object.
\end{cfuncdesc}

\begin{cfuncdesc}{void}{PyFile_SetBufSize}{PyFileObject *p, int n}
Available on systems with \cfunction{setvbuf()}\ttindex{setvbuf()}
only.  This should only be called immediately after file object
creation.
\end{cfuncdesc}

\begin{cfuncdesc}{int}{PyFile_SoftSpace}{PyObject *p, int newflag}
This function exists for internal use by the interpreter.
Sets the \member{softspace} attribute of \var{p} to \var{newflag} and
\withsubitem{(file attribute)}{\ttindex{softspace}}returns the
previous value.  \var{p} does not have to be a file object
for this function to work properly; any object is supported (thought
its only interesting if the \member{softspace} attribute can be set).
This function clears any errors, and will return \code{0} as the
previous value if the attribute either does not exist or if there were
errors in retrieving it.  There is no way to detect errors from this
function, but doing so should not be needed.
\end{cfuncdesc}

\begin{cfuncdesc}{int}{PyFile_WriteObject}{PyObject *obj, PyFileObject *p,
                                           int flags}
Writes object \var{obj} to file object \var{p}.  The only supported
flag for \var{flags} is \constant{Py_PRINT_RAW}\ttindex{Py_PRINT_RAW};
if given, the \function{str()} of the object is written instead of the 
\function{repr()}.  Returns \code{0} on success or \code{-1} on
failure; the appropriate exception will be set.
\end{cfuncdesc}

\begin{cfuncdesc}{int}{PyFile_WriteString}{char *s, PyFileObject *p}
Writes string \var{s} to file object \var{p}.  Returns \code{0} on
success or \code{-1} on failure; the appropriate exception will be
set.
\end{cfuncdesc}


\subsection{Instance Objects \label{instanceObjects}}

\obindex{instance}
There are very few functions specific to instance objects.

\begin{cvardesc}{PyTypeObject}{PyInstance_Type}
  Type object for class instances.
\end{cvardesc}

\begin{cfuncdesc}{int}{PyInstance_Check}{PyObject *obj}
  Returns true if \var{obj} is an instance.
\end{cfuncdesc}

\begin{cfuncdesc}{PyObject*}{PyInstance_New}{PyObject *class,
                                             PyObject *arg,
                                             PyObject *kw}
  Create a new instance of a specific class.  The parameters \var{arg}
  and \var{kw} are used as the positional and keyword parameters to
  the object's constructor.
\end{cfuncdesc}

\begin{cfuncdesc}{PyObject*}{PyInstance_NewRaw}{PyObject *class,
                                                PyObject *dict}
  Create a new instance of a specific class without calling it's
  constructor.  \var{class} is the class of new object.  The
  \var{dict} parameter will be used as the object's \member{__dict__};
  if \NULL, a new dictionary will be created for the instance.
\end{cfuncdesc}


\subsection{Method Objects \label{method-objects}}

\obindex{method}
There are some useful functions that are useful for working with
method objects.

\begin{cvardesc}{PyTypeObject}{PyMethod_Type}
  This instance of \ctype{PyTypeObject} represents the Python method
  type.  This is exposed to Python programs as \code{types.MethodType}.
  \withsubitem{(in module types)}{\ttindex{MethodType}}
\end{cvardesc}

\begin{cfuncdesc}{int}{PyMethod_Check}{PyObject *o}
  Return true if \var{o} is a method object (has type
  \cdata{PyMethod_Type}).  The parameter must not be \NULL.
\end{cfuncdesc}

\begin{cfuncdesc}{PyObject*}{PyMethod_New}{PyObject *func.
                                           PyObject *self, PyObject *class}
  Return a new method object, with \var{func} being any callable
  object; this is the function that will be called when the method is
  called.  If this method should be bound to an instance, \var{self}
  should be the instance and \var{class} should be the class of
  \var{self}, otherwise \var{self} should be \NULL{} and \var{class}
  should be the class which provides the unbound method..
\end{cfuncdesc}

\begin{cfuncdesc}{PyObject*}{PyMethod_Class}{PyObject *meth}
  Return the class object from which the method \var{meth} was
  created; if this was created from an instance, it will be the class
  of the instance.
\end{cfuncdesc}

\begin{cfuncdesc}{PyObject*}{PyMethod_GET_CLASS}{PyObject *meth}
  Macro version of \cfunction{PyMethod_Class()} which avoids error
  checking.
\end{cfuncdesc}

\begin{cfuncdesc}{PyObject*}{PyMethod_Function}{PyObject *meth}
  Return the function object associated with the method \var{meth}.
\end{cfuncdesc}

\begin{cfuncdesc}{PyObject*}{PyMethod_GET_FUNCTION}{PyObject *meth}
  Macro version of \cfunction{PyMethod_Function()} which avoids error
  checking.
\end{cfuncdesc}

\begin{cfuncdesc}{PyObject*}{PyMethod_Self}{PyObject *meth}
  Return the instance associated with the method \var{meth} if it is
  bound, otherwise return \NULL.
\end{cfuncdesc}

\begin{cfuncdesc}{PyObject*}{PyMethod_GET_SELF}{PyObject *meth}
  Macro version of \cfunction{PyMethod_Self()} which avoids error
  checking.
\end{cfuncdesc}


\subsection{Module Objects \label{moduleObjects}}

\obindex{module}
There are only a few functions special to module objects.

\begin{cvardesc}{PyTypeObject}{PyModule_Type}
This instance of \ctype{PyTypeObject} represents the Python module
type.  This is exposed to Python programs as \code{types.ModuleType}.
\withsubitem{(in module types)}{\ttindex{ModuleType}}
\end{cvardesc}

\begin{cfuncdesc}{int}{PyModule_Check}{PyObject *p}
Returns true if \var{p} is a module object, or a subtype of a
module object.
\versionchanged[Allowed subtypes to be accepted]{2.2}
\end{cfuncdesc}

\begin{cfuncdesc}{int}{PyModule_CheckExact}{PyObject *p}
Returns true if \var{p} is a module object, but not a subtype of
\cdata{PyModule_Type}.
\versionadded{2.2}
\end{cfuncdesc}

\begin{cfuncdesc}{PyObject*}{PyModule_New}{char *name}
Return a new module object with the \member{__name__} attribute set to
\var{name}.  Only the module's \member{__doc__} and
\member{__name__} attributes are filled in; the caller is responsible
for providing a \member{__file__} attribute.
\withsubitem{(module attribute)}{
  \ttindex{__name__}\ttindex{__doc__}\ttindex{__file__}}
\end{cfuncdesc}

\begin{cfuncdesc}{PyObject*}{PyModule_GetDict}{PyObject *module}
Return the dictionary object that implements \var{module}'s namespace; 
this object is the same as the \member{__dict__} attribute of the
module object.  This function never fails.
\withsubitem{(module attribute)}{\ttindex{__dict__}}
\end{cfuncdesc}

\begin{cfuncdesc}{char*}{PyModule_GetName}{PyObject *module}
Return \var{module}'s \member{__name__} value.  If the module does not 
provide one, or if it is not a string, \exception{SystemError} is
raised and \NULL{} is returned.
\withsubitem{(module attribute)}{\ttindex{__name__}}
\withsubitem{(built-in exception)}{\ttindex{SystemError}}
\end{cfuncdesc}

\begin{cfuncdesc}{char*}{PyModule_GetFilename}{PyObject *module}
Return the name of the file from which \var{module} was loaded using
\var{module}'s \member{__file__} attribute.  If this is not defined,
or if it is not a string, raise \exception{SystemError} and return
\NULL.
\withsubitem{(module attribute)}{\ttindex{__file__}}
\withsubitem{(built-in exception)}{\ttindex{SystemError}}
\end{cfuncdesc}

\begin{cfuncdesc}{int}{PyModule_AddObject}{PyObject *module,
                                           char *name, PyObject *value}
Add an object to \var{module} as \var{name}.  This is a convenience
function which can be used from the module's initialization function.
This steals a reference to \var{value}.  Returns \code{-1} on error,
\code{0} on success.
\versionadded{2.0}
\end{cfuncdesc}

\begin{cfuncdesc}{int}{PyModule_AddIntConstant}{PyObject *module,
                                                char *name, int value}
Add an integer constant to \var{module} as \var{name}.  This convenience
function can be used from the module's initialization function.
Returns \code{-1} on error, \code{0} on success.
\versionadded{2.0}
\end{cfuncdesc}

\begin{cfuncdesc}{int}{PyModule_AddStringConstant}{PyObject *module,
                                                   char *name, char *value}
Add a string constant to \var{module} as \var{name}.  This convenience
function can be used from the module's initialization function.  The
string \var{value} must be null-terminated.  Returns \code{-1} on
error, \code{0} on success.
\versionadded{2.0}
\end{cfuncdesc}


\subsection{Iterator Objects \label{iterator-objects}}

Python provides two general-purpose iterator objects.  The first, a
sequence iterator, works with an arbitrary sequence supporting the
\method{__getitem__()} method.  The second works with a callable
object and a sentinel value, calling the callable for each item in the
sequence, and ending the iteration when the sentinel value is
returned.

\begin{cvardesc}{PyTypeObject}{PySeqIter_Type}
  Type object for iterator objects returned by
  \cfunction{PySeqIter_New()} and the one-argument form of the
  \function{iter()} built-in function for built-in sequence types.
  \versionadded{2.2}
\end{cvardesc}

\begin{cfuncdesc}{int}{PySeqIter_Check}{op}
  Return true if the type of \var{op} is \cdata{PySeqIter_Type}.
  \versionadded{2.2}
\end{cfuncdesc}

\begin{cfuncdesc}{PyObject*}{PySeqIter_New}{PyObject *seq}
  Return an iterator that works with a general sequence object,
  \var{seq}.  The iteration ends when the sequence raises
  \exception{IndexError} for the subscripting operation.
  \versionadded{2.2}
\end{cfuncdesc}

\begin{cvardesc}{PyTypeObject}{PyCallIter_Type}
  Type object for iterator objects returned by
  \cfunction{PyCallIter_New()} and the two-argument form of the
  \function{iter()} built-in function.
  \versionadded{2.2}
\end{cvardesc}

\begin{cfuncdesc}{int}{PyCallIter_Check}{op}
  Return true if the type of \var{op} is \cdata{PyCallIter_Type}.
  \versionadded{2.2}
\end{cfuncdesc}

\begin{cfuncdesc}{PyObject*}{PyCallIter_New}{PyObject *callable,
                                             PyObject *sentinel}
  Return a new iterator.  The first parameter, \var{callable}, can be
  any Python callable object that can be called with no parameters;
  each call to it should return the next item in the iteration.  When
  \var{callable} returns a value equal to \var{sentinel}, the
  iteration will be terminated.
  \versionadded{2.2}
\end{cfuncdesc}


\subsection{Descriptor Objects \label{descriptor-objects}}

\begin{cvardesc}{PyTypeObject}{PyProperty_Type}
  The type object for a descriptor.
  \versionadded{2.2}
\end{cvardesc}

\begin{cfuncdesc}{PyObject*}{PyDescr_NewGetSet}{PyTypeObject *type,
					        PyGetSetDef *getset}
  \versionadded{2.2}
\end{cfuncdesc}

\begin{cfuncdesc}{PyObject*}{PyDescr_NewMember}{PyTypeObject *type,
					        PyMemberDef *meth}
  \versionadded{2.2}
\end{cfuncdesc}

\begin{cfuncdesc}{PyObject*}{PyDescr_NewMethod}{PyTypeObject *type,
                                                PyMethodDef *meth}
  \versionadded{2.2}
\end{cfuncdesc}

\begin{cfuncdesc}{PyObject*}{PyDescr_NewWrapper}{PyTypeObject *type,
						 struct wrapperbase *wrapper,
                                                 void *wrapped}
  \versionadded{2.2}
\end{cfuncdesc}

\begin{cfuncdesc}{int}{PyDescr_IsData}{PyObject *descr}
  Returns true if the descriptor objects \var{descr} describes a data
  attribute, or false if it describes a method.  \var{descr} must be a
  descriptor object; there is no error checking.
  \versionadded{2.2}
\end{cfuncdesc}

\begin{cfuncdesc}{PyObject*}{PyWrapper_New}{PyObject *, PyObject *}
  \versionadded{2.2}
\end{cfuncdesc}


\subsection{Slice Objects \label{slice-objects}}

\begin{cvardesc}{PyTypeObject}{PySlice_Type}
  The type object for slice objects.  This is the same as
  \code{types.SliceType}.
  \withsubitem{(in module types)}{\ttindex{SliceType}}
\end{cvardesc}

\begin{cfuncdesc}{int}{PySlice_Check}{PyObject *ob}
  Returns true if \var{ob} is a slice object; \var{ob} must not be
  \NULL.
\end{cfuncdesc}

\begin{cfuncdesc}{PyObject*}{PySlice_New}{PyObject *start, PyObject *stop,
                                          PyObject *step}
  Return a new slice object with the given values.  The \var{start},
  \var{stop}, and \var{step} parameters are used as the values of the
  slice object attributes of the same names.  Any of the values may be
  \NULL, in which case the \code{None} will be used for the
  corresponding attribute.  Returns \NULL{} if the new object could
  not be allocated.
\end{cfuncdesc}

\begin{cfuncdesc}{int}{PySlice_GetIndices}{PySliceObject *slice, int length,
                                           int *start, int *stop, int *step}
\end{cfuncdesc}


\subsection{CObjects \label{cObjects}}

\obindex{CObject}
Refer to \emph{Extending and Embedding the Python Interpreter},
section 1.12 (``Providing a C API for an Extension Module), for more 
information on using these objects.


\begin{ctypedesc}{PyCObject}
This subtype of \ctype{PyObject} represents an opaque value, useful for
C extension modules who need to pass an opaque value (as a
\ctype{void*} pointer) through Python code to other C code.  It is
often used to make a C function pointer defined in one module
available to other modules, so the regular import mechanism can be
used to access C APIs defined in dynamically loaded modules.
\end{ctypedesc}

\begin{cfuncdesc}{int}{PyCObject_Check}{PyObject *p}
Returns true if its argument is a \ctype{PyCObject}.
\end{cfuncdesc}

\begin{cfuncdesc}{PyObject*}{PyCObject_FromVoidPtr}{void* cobj, 
	void (*destr)(void *)}
Creates a \ctype{PyCObject} from the \code{void *}\var{cobj}.  The
\var{destr} function will be called when the object is reclaimed, unless
it is \NULL.
\end{cfuncdesc}

\begin{cfuncdesc}{PyObject*}{PyCObject_FromVoidPtrAndDesc}{void* cobj,
	void* desc, void (*destr)(void *, void *) }
Creates a \ctype{PyCObject} from the \ctype{void *}\var{cobj}.  The
\var{destr} function will be called when the object is reclaimed.  The
\var{desc} argument can be used to pass extra callback data for the
destructor function.
\end{cfuncdesc}

\begin{cfuncdesc}{void*}{PyCObject_AsVoidPtr}{PyObject* self}
Returns the object \ctype{void *} that the
\ctype{PyCObject} \var{self} was created with.
\end{cfuncdesc}

\begin{cfuncdesc}{void*}{PyCObject_GetDesc}{PyObject* self}
Returns the description \ctype{void *} that the
\ctype{PyCObject} \var{self} was created with.
\end{cfuncdesc}


\chapter{Initialization, Finalization, and Threads
         \label{initialization}}

\begin{cfuncdesc}{void}{Py_Initialize}{}
Initialize the Python interpreter.  In an application embedding 
Python, this should be called before using any other Python/C API 
functions; with the exception of
\cfunction{Py_SetProgramName()}\ttindex{Py_SetProgramName()},
\cfunction{PyEval_InitThreads()}\ttindex{PyEval_InitThreads()},
\cfunction{PyEval_ReleaseLock()}\ttindex{PyEval_ReleaseLock()},
and \cfunction{PyEval_AcquireLock()}\ttindex{PyEval_AcquireLock()}.
This initializes the table of loaded modules (\code{sys.modules}), and
\withsubitem{(in module sys)}{\ttindex{modules}\ttindex{path}}creates the
fundamental modules \module{__builtin__}\refbimodindex{__builtin__},
\module{__main__}\refbimodindex{__main__} and
\module{sys}\refbimodindex{sys}.  It also initializes the module
search\indexiii{module}{search}{path} path (\code{sys.path}).
It does not set \code{sys.argv}; use
\cfunction{PySys_SetArgv()}\ttindex{PySys_SetArgv()} for that.  This
is a no-op when called for a second time (without calling
\cfunction{Py_Finalize()}\ttindex{Py_Finalize()} first).  There is no
return value; it is a fatal error if the initialization fails.
\end{cfuncdesc}

\begin{cfuncdesc}{int}{Py_IsInitialized}{}
Return true (nonzero) when the Python interpreter has been
initialized, false (zero) if not.  After \cfunction{Py_Finalize()} is
called, this returns false until \cfunction{Py_Initialize()} is called
again.
\end{cfuncdesc}

\begin{cfuncdesc}{void}{Py_Finalize}{}
Undo all initializations made by \cfunction{Py_Initialize()} and
subsequent use of Python/C API functions, and destroy all
sub-interpreters (see \cfunction{Py_NewInterpreter()} below) that were
created and not yet destroyed since the last call to
\cfunction{Py_Initialize()}.  Ideally, this frees all memory allocated
by the Python interpreter.  This is a no-op when called for a second
time (without calling \cfunction{Py_Initialize()} again first).  There
is no return value; errors during finalization are ignored.

This function is provided for a number of reasons.  An embedding 
application might want to restart Python without having to restart the 
application itself.  An application that has loaded the Python 
interpreter from a dynamically loadable library (or DLL) might want to 
free all memory allocated by Python before unloading the DLL. During a 
hunt for memory leaks in an application a developer might want to free 
all memory allocated by Python before exiting from the application.

\strong{Bugs and caveats:} The destruction of modules and objects in 
modules is done in random order; this may cause destructors 
(\method{__del__()} methods) to fail when they depend on other objects 
(even functions) or modules.  Dynamically loaded extension modules 
loaded by Python are not unloaded.  Small amounts of memory allocated 
by the Python interpreter may not be freed (if you find a leak, please 
report it).  Memory tied up in circular references between objects is 
not freed.  Some memory allocated by extension modules may not be 
freed.  Some extension may not work properly if their initialization 
routine is called more than once; this can happen if an applcation 
calls \cfunction{Py_Initialize()} and \cfunction{Py_Finalize()} more
than once.
\end{cfuncdesc}

\begin{cfuncdesc}{PyThreadState*}{Py_NewInterpreter}{}
Create a new sub-interpreter.  This is an (almost) totally separate
environment for the execution of Python code.  In particular, the new
interpreter has separate, independent versions of all imported
modules, including the fundamental modules
\module{__builtin__}\refbimodindex{__builtin__},
\module{__main__}\refbimodindex{__main__} and
\module{sys}\refbimodindex{sys}.  The table of loaded modules
(\code{sys.modules}) and the module search path (\code{sys.path}) are
also separate.  The new environment has no \code{sys.argv} variable.
It has new standard I/O stream file objects \code{sys.stdin},
\code{sys.stdout} and \code{sys.stderr} (however these refer to the
same underlying \ctype{FILE} structures in the C library).
\withsubitem{(in module sys)}{
  \ttindex{stdout}\ttindex{stderr}\ttindex{stdin}}

The return value points to the first thread state created in the new 
sub-interpreter.  This thread state is made the current thread state.  
Note that no actual thread is created; see the discussion of thread 
states below.  If creation of the new interpreter is unsuccessful, 
\NULL{} is returned; no exception is set since the exception state 
is stored in the current thread state and there may not be a current 
thread state.  (Like all other Python/C API functions, the global 
interpreter lock must be held before calling this function and is 
still held when it returns; however, unlike most other Python/C API 
functions, there needn't be a current thread state on entry.)

Extension modules are shared between (sub-)interpreters as follows: 
the first time a particular extension is imported, it is initialized 
normally, and a (shallow) copy of its module's dictionary is 
squirreled away.  When the same extension is imported by another 
(sub-)interpreter, a new module is initialized and filled with the 
contents of this copy; the extension's \code{init} function is not
called.  Note that this is different from what happens when an
extension is imported after the interpreter has been completely
re-initialized by calling
\cfunction{Py_Finalize()}\ttindex{Py_Finalize()} and
\cfunction{Py_Initialize()}\ttindex{Py_Initialize()}; in that case,
the extension's \code{init\var{module}} function \emph{is} called
again.

\strong{Bugs and caveats:} Because sub-interpreters (and the main 
interpreter) are part of the same process, the insulation between them 
isn't perfect --- for example, using low-level file operations like 
\withsubitem{(in module os)}{\ttindex{close()}}
\function{os.close()} they can (accidentally or maliciously) affect each 
other's open files.  Because of the way extensions are shared between 
(sub-)interpreters, some extensions may not work properly; this is 
especially likely when the extension makes use of (static) global 
variables, or when the extension manipulates its module's dictionary 
after its initialization.  It is possible to insert objects created in 
one sub-interpreter into a namespace of another sub-interpreter; this 
should be done with great care to avoid sharing user-defined 
functions, methods, instances or classes between sub-interpreters, 
since import operations executed by such objects may affect the 
wrong (sub-)interpreter's dictionary of loaded modules.  (XXX This is 
a hard-to-fix bug that will be addressed in a future release.)
\end{cfuncdesc}

\begin{cfuncdesc}{void}{Py_EndInterpreter}{PyThreadState *tstate}
Destroy the (sub-)interpreter represented by the given thread state.  
The given thread state must be the current thread state.  See the 
discussion of thread states below.  When the call returns, the current 
thread state is \NULL{}.  All thread states associated with this 
interpreted are destroyed.  (The global interpreter lock must be held 
before calling this function and is still held when it returns.)  
\cfunction{Py_Finalize()}\ttindex{Py_Finalize()} will destroy all
sub-interpreters that haven't been explicitly destroyed at that point.
\end{cfuncdesc}

\begin{cfuncdesc}{void}{Py_SetProgramName}{char *name}
This function should be called before
\cfunction{Py_Initialize()}\ttindex{Py_Initialize()} is called
for the first time, if it is called at all.  It tells the interpreter 
the value of the \code{argv[0]} argument to the
\cfunction{main()}\ttindex{main()} function of the program.  This is
used by \cfunction{Py_GetPath()}\ttindex{Py_GetPath()} and some other  
functions below to find the Python run-time libraries relative to the 
interpreter executable.  The default value is \code{'python'}.  The 
argument should point to a zero-terminated character string in static 
storage whose contents will not change for the duration of the 
program's execution.  No code in the Python interpreter will change 
the contents of this storage.
\end{cfuncdesc}

\begin{cfuncdesc}{char*}{Py_GetProgramName}{}
Return the program name set with
\cfunction{Py_SetProgramName()}\ttindex{Py_SetProgramName()}, or the
default.  The returned string points into static storage; the caller 
should not modify its value.
\end{cfuncdesc}

\begin{cfuncdesc}{char*}{Py_GetPrefix}{}
Return the \emph{prefix} for installed platform-independent files.  This 
is derived through a number of complicated rules from the program name 
set with \cfunction{Py_SetProgramName()} and some environment variables; 
for example, if the program name is \code{'/usr/local/bin/python'}, 
the prefix is \code{'/usr/local'}.  The returned string points into 
static storage; the caller should not modify its value.  This 
corresponds to the \makevar{prefix} variable in the top-level 
\file{Makefile} and the \longprogramopt{prefix} argument to the 
\program{configure} script at build time.  The value is available to 
Python code as \code{sys.prefix}.  It is only useful on \UNIX{}.  See 
also the next function.
\end{cfuncdesc}

\begin{cfuncdesc}{char*}{Py_GetExecPrefix}{}
Return the \emph{exec-prefix} for installed platform-\emph{de}pendent 
files.  This is derived through a number of complicated rules from the 
program name set with \cfunction{Py_SetProgramName()} and some environment 
variables; for example, if the program name is 
\code{'/usr/local/bin/python'}, the exec-prefix is 
\code{'/usr/local'}.  The returned string points into static storage; 
the caller should not modify its value.  This corresponds to the 
\makevar{exec_prefix} variable in the top-level \file{Makefile} and the 
\longprogramopt{exec-prefix} argument to the
\program{configure} script at build  time.  The value is available to
Python code as \code{sys.exec_prefix}.  It is only useful on \UNIX{}.

Background: The exec-prefix differs from the prefix when platform 
dependent files (such as executables and shared libraries) are 
installed in a different directory tree.  In a typical installation, 
platform dependent files may be installed in the 
\file{/usr/local/plat} subtree while platform independent may be 
installed in \file{/usr/local}.

Generally speaking, a platform is a combination of hardware and 
software families, e.g.  Sparc machines running the Solaris 2.x 
operating system are considered the same platform, but Intel machines 
running Solaris 2.x are another platform, and Intel machines running 
Linux are yet another platform.  Different major revisions of the same 
operating system generally also form different platforms.  Non-\UNIX{} 
operating systems are a different story; the installation strategies 
on those systems are so different that the prefix and exec-prefix are 
meaningless, and set to the empty string.  Note that compiled Python 
bytecode files are platform independent (but not independent from the 
Python version by which they were compiled!).

System administrators will know how to configure the \program{mount} or 
\program{automount} programs to share \file{/usr/local} between platforms 
while having \file{/usr/local/plat} be a different filesystem for each 
platform.
\end{cfuncdesc}

\begin{cfuncdesc}{char*}{Py_GetProgramFullPath}{}
Return the full program name of the Python executable; this is 
computed as a side-effect of deriving the default module search path 
from the program name (set by
\cfunction{Py_SetProgramName()}\ttindex{Py_SetProgramName()} above).
The returned string points into static storage; the caller should not 
modify its value.  The value is available to Python code as 
\code{sys.executable}.
\withsubitem{(in module sys)}{\ttindex{executable}}
\end{cfuncdesc}

\begin{cfuncdesc}{char*}{Py_GetPath}{}
\indexiii{module}{search}{path}
Return the default module search path; this is computed from the 
program name (set by \cfunction{Py_SetProgramName()} above) and some 
environment variables.  The returned string consists of a series of 
directory names separated by a platform dependent delimiter character.  
The delimiter character is \character{:} on \UNIX{}, \character{;} on
DOS/Windows, and \character{\e n} (the \ASCII{} newline character) on
Macintosh.  The returned string points into static storage; the caller
should not modify its value.  The value is available to Python code 
as the list \code{sys.path}\withsubitem{(in module sys)}{\ttindex{path}},
which may be modified to change the future search path for loaded
modules.

% XXX should give the exact rules
\end{cfuncdesc}

\begin{cfuncdesc}{const char*}{Py_GetVersion}{}
Return the version of this Python interpreter.  This is a string that 
looks something like

\begin{verbatim}
"1.5 (#67, Dec 31 1997, 22:34:28) [GCC 2.7.2.2]"
\end{verbatim}

The first word (up to the first space character) is the current Python 
version; the first three characters are the major and minor version 
separated by a period.  The returned string points into static storage; 
the caller should not modify its value.  The value is available to 
Python code as the list \code{sys.version}.
\withsubitem{(in module sys)}{\ttindex{version}}
\end{cfuncdesc}

\begin{cfuncdesc}{const char*}{Py_GetPlatform}{}
Return the platform identifier for the current platform.  On \UNIX{}, 
this is formed from the ``official'' name of the operating system, 
converted to lower case, followed by the major revision number; e.g., 
for Solaris 2.x, which is also known as SunOS 5.x, the value is 
\code{'sunos5'}.  On Macintosh, it is \code{'mac'}.  On Windows, it 
is \code{'win'}.  The returned string points into static storage; 
the caller should not modify its value.  The value is available to 
Python code as \code{sys.platform}.
\withsubitem{(in module sys)}{\ttindex{platform}}
\end{cfuncdesc}

\begin{cfuncdesc}{const char*}{Py_GetCopyright}{}
Return the official copyright string for the current Python version, 
for example

\code{'Copyright 1991-1995 Stichting Mathematisch Centrum, Amsterdam'}

The returned string points into static storage; the caller should not 
modify its value.  The value is available to Python code as the list 
\code{sys.copyright}.
\withsubitem{(in module sys)}{\ttindex{copyright}}
\end{cfuncdesc}

\begin{cfuncdesc}{const char*}{Py_GetCompiler}{}
Return an indication of the compiler used to build the current Python 
version, in square brackets, for example:

\begin{verbatim}
"[GCC 2.7.2.2]"
\end{verbatim}

The returned string points into static storage; the caller should not 
modify its value.  The value is available to Python code as part of 
the variable \code{sys.version}.
\withsubitem{(in module sys)}{\ttindex{version}}
\end{cfuncdesc}

\begin{cfuncdesc}{const char*}{Py_GetBuildInfo}{}
Return information about the sequence number and build date and time 
of the current Python interpreter instance, for example

\begin{verbatim}
"#67, Aug  1 1997, 22:34:28"
\end{verbatim}

The returned string points into static storage; the caller should not 
modify its value.  The value is available to Python code as part of 
the variable \code{sys.version}.
\withsubitem{(in module sys)}{\ttindex{version}}
\end{cfuncdesc}

\begin{cfuncdesc}{int}{PySys_SetArgv}{int argc, char **argv}
Set \code{sys.argv} based on \var{argc} and \var{argv}.  These
parameters are similar to those passed to the program's
\cfunction{main()}\ttindex{main()} function with the difference that
the first entry should refer to the script file to be executed rather
than the executable hosting the Python interpreter.  If there isn't a
script that will be run, the first entry in \var{argv} can be an empty
string.  If this function fails to initialize \code{sys.argv}, a fatal 
condition is signalled using
\cfunction{Py_FatalError()}\ttindex{Py_FatalError()}.
\withsubitem{(in module sys)}{\ttindex{argv}}
% XXX impl. doesn't seem consistent in allowing 0/NULL for the params; 
% check w/ Guido.
\end{cfuncdesc}

% XXX Other PySys thingies (doesn't really belong in this chapter)

\section{Thread State and the Global Interpreter Lock
         \label{threads}}

\index{global interpreter lock}
\index{interpreter lock}
\index{lock, interpreter}

The Python interpreter is not fully thread safe.  In order to support
multi-threaded Python programs, there's a global lock that must be
held by the current thread before it can safely access Python objects.
Without the lock, even the simplest operations could cause problems in
a multi-threaded program: for example, when two threads simultaneously
increment the reference count of the same object, the reference count
could end up being incremented only once instead of twice.

Therefore, the rule exists that only the thread that has acquired the
global interpreter lock may operate on Python objects or call Python/C
API functions.  In order to support multi-threaded Python programs,
the interpreter regularly releases and reacquires the lock --- by
default, every ten bytecode instructions (this can be changed with
\withsubitem{(in module sys)}{\ttindex{setcheckinterval()}}
\function{sys.setcheckinterval()}).  The lock is also released and
reacquired around potentially blocking I/O operations like reading or
writing a file, so that other threads can run while the thread that
requests the I/O is waiting for the I/O operation to complete.

The Python interpreter needs to keep some bookkeeping information
separate per thread --- for this it uses a data structure called
\ctype{PyThreadState}\ttindex{PyThreadState}.  This is new in Python
1.5; in earlier versions, such state was stored in global variables,
and switching threads could cause problems.  In particular, exception
handling is now thread safe, when the application uses
\withsubitem{(in module sys)}{\ttindex{exc_info()}}
\function{sys.exc_info()} to access the exception last raised in the
current thread.

There's one global variable left, however: the pointer to the current
\ctype{PyThreadState}\ttindex{PyThreadState} structure.  While most
thread packages have a way to store ``per-thread global data,''
Python's internal platform independent thread abstraction doesn't
support this yet.  Therefore, the current thread state must be
manipulated explicitly.

This is easy enough in most cases.  Most code manipulating the global
interpreter lock has the following simple structure:

\begin{verbatim}
Save the thread state in a local variable.
Release the interpreter lock.
...Do some blocking I/O operation...
Reacquire the interpreter lock.
Restore the thread state from the local variable.
\end{verbatim}

This is so common that a pair of macros exists to simplify it:

\begin{verbatim}
Py_BEGIN_ALLOW_THREADS
...Do some blocking I/O operation...
Py_END_ALLOW_THREADS
\end{verbatim}

The \code{Py_BEGIN_ALLOW_THREADS}\ttindex{Py_BEGIN_ALLOW_THREADS} macro
opens a new block and declares a hidden local variable; the
\code{Py_END_ALLOW_THREADS}\ttindex{Py_END_ALLOW_THREADS} macro closes 
the block.  Another advantage of using these two macros is that when
Python is compiled without thread support, they are defined empty,
thus saving the thread state and lock manipulations.

When thread support is enabled, the block above expands to the
following code:

\begin{verbatim}
    PyThreadState *_save;

    _save = PyEval_SaveThread();
    ...Do some blocking I/O operation...
    PyEval_RestoreThread(_save);
\end{verbatim}

Using even lower level primitives, we can get roughly the same effect
as follows:

\begin{verbatim}
    PyThreadState *_save;

    _save = PyThreadState_Swap(NULL);
    PyEval_ReleaseLock();
    ...Do some blocking I/O operation...
    PyEval_AcquireLock();
    PyThreadState_Swap(_save);
\end{verbatim}

There are some subtle differences; in particular,
\cfunction{PyEval_RestoreThread()}\ttindex{PyEval_RestoreThread()} saves
and restores the value of the  global variable
\cdata{errno}\ttindex{errno}, since the lock manipulation does not
guarantee that \cdata{errno} is left alone.  Also, when thread support
is disabled,
\cfunction{PyEval_SaveThread()}\ttindex{PyEval_SaveThread()} and
\cfunction{PyEval_RestoreThread()} don't manipulate the lock; in this
case, \cfunction{PyEval_ReleaseLock()}\ttindex{PyEval_ReleaseLock()} and
\cfunction{PyEval_AcquireLock()}\ttindex{PyEval_AcquireLock()} are not
available.  This is done so that dynamically loaded extensions
compiled with thread support enabled can be loaded by an interpreter
that was compiled with disabled thread support.

The global interpreter lock is used to protect the pointer to the
current thread state.  When releasing the lock and saving the thread
state, the current thread state pointer must be retrieved before the
lock is released (since another thread could immediately acquire the
lock and store its own thread state in the global variable).
Conversely, when acquiring the lock and restoring the thread state,
the lock must be acquired before storing the thread state pointer.

Why am I going on with so much detail about this?  Because when
threads are created from C, they don't have the global interpreter
lock, nor is there a thread state data structure for them.  Such
threads must bootstrap themselves into existence, by first creating a
thread state data structure, then acquiring the lock, and finally
storing their thread state pointer, before they can start using the
Python/C API.  When they are done, they should reset the thread state
pointer, release the lock, and finally free their thread state data
structure.

When creating a thread data structure, you need to provide an
interpreter state data structure.  The interpreter state data
structure hold global data that is shared by all threads in an
interpreter, for example the module administration
(\code{sys.modules}).  Depending on your needs, you can either create
a new interpreter state data structure, or share the interpreter state
data structure used by the Python main thread (to access the latter,
you must obtain the thread state and access its \member{interp} member;
this must be done by a thread that is created by Python or by the main
thread after Python is initialized).


\begin{ctypedesc}{PyInterpreterState}
This data structure represents the state shared by a number of
cooperating threads.  Threads belonging to the same interpreter
share their module administration and a few other internal items.
There are no public members in this structure.

Threads belonging to different interpreters initially share nothing,
except process state like available memory, open file descriptors and
such.  The global interpreter lock is also shared by all threads,
regardless of to which interpreter they belong.
\end{ctypedesc}

\begin{ctypedesc}{PyThreadState}
This data structure represents the state of a single thread.  The only
public data member is \ctype{PyInterpreterState *}\member{interp},
which points to this thread's interpreter state.
\end{ctypedesc}

\begin{cfuncdesc}{void}{PyEval_InitThreads}{}
Initialize and acquire the global interpreter lock.  It should be
called in the main thread before creating a second thread or engaging
in any other thread operations such as
\cfunction{PyEval_ReleaseLock()}\ttindex{PyEval_ReleaseLock()} or
\code{PyEval_ReleaseThread(\var{tstate})}\ttindex{PyEval_ReleaseThread()}.
It is not needed before calling
\cfunction{PyEval_SaveThread()}\ttindex{PyEval_SaveThread()} or
\cfunction{PyEval_RestoreThread()}\ttindex{PyEval_RestoreThread()}.

This is a no-op when called for a second time.  It is safe to call
this function before calling
\cfunction{Py_Initialize()}\ttindex{Py_Initialize()}.

When only the main thread exists, no lock operations are needed.  This
is a common situation (most Python programs do not use threads), and
the lock operations slow the interpreter down a bit.  Therefore, the
lock is not created initially.  This situation is equivalent to having
acquired the lock: when there is only a single thread, all object
accesses are safe.  Therefore, when this function initializes the
lock, it also acquires it.  Before the Python
\module{thread}\refbimodindex{thread} module creates a new thread,
knowing that either it has the lock or the lock hasn't been created
yet, it calls \cfunction{PyEval_InitThreads()}.  When this call
returns, it is guaranteed that the lock has been created and that it
has acquired it.

It is \strong{not} safe to call this function when it is unknown which
thread (if any) currently has the global interpreter lock.

This function is not available when thread support is disabled at
compile time.
\end{cfuncdesc}

\begin{cfuncdesc}{void}{PyEval_AcquireLock}{}
Acquire the global interpreter lock.  The lock must have been created
earlier.  If this thread already has the lock, a deadlock ensues.
This function is not available when thread support is disabled at
compile time.
\end{cfuncdesc}

\begin{cfuncdesc}{void}{PyEval_ReleaseLock}{}
Release the global interpreter lock.  The lock must have been created
earlier.  This function is not available when thread support is
disabled at compile time.
\end{cfuncdesc}

\begin{cfuncdesc}{void}{PyEval_AcquireThread}{PyThreadState *tstate}
Acquire the global interpreter lock and then set the current thread
state to \var{tstate}, which should not be \NULL{}.  The lock must
have been created earlier.  If this thread already has the lock,
deadlock ensues.  This function is not available when thread support
is disabled at compile time.
\end{cfuncdesc}

\begin{cfuncdesc}{void}{PyEval_ReleaseThread}{PyThreadState *tstate}
Reset the current thread state to \NULL{} and release the global
interpreter lock.  The lock must have been created earlier and must be
held by the current thread.  The \var{tstate} argument, which must not
be \NULL{}, is only used to check that it represents the current
thread state --- if it isn't, a fatal error is reported.  This
function is not available when thread support is disabled at compile
time.
\end{cfuncdesc}

\begin{cfuncdesc}{PyThreadState*}{PyEval_SaveThread}{}
Release the interpreter lock (if it has been created and thread
support is enabled) and reset the thread state to \NULL{},
returning the previous thread state (which is not \NULL{}).  If
the lock has been created, the current thread must have acquired it.
(This function is available even when thread support is disabled at
compile time.)
\end{cfuncdesc}

\begin{cfuncdesc}{void}{PyEval_RestoreThread}{PyThreadState *tstate}
Acquire the interpreter lock (if it has been created and thread
support is enabled) and set the thread state to \var{tstate}, which
must not be \NULL{}.  If the lock has been created, the current
thread must not have acquired it, otherwise deadlock ensues.  (This
function is available even when thread support is disabled at compile
time.)
\end{cfuncdesc}

The following macros are normally used without a trailing semicolon;
look for example usage in the Python source distribution.

\begin{csimplemacrodesc}{Py_BEGIN_ALLOW_THREADS}
This macro expands to
\samp{\{ PyThreadState *_save; _save = PyEval_SaveThread();}.
Note that it contains an opening brace; it must be matched with a
following \code{Py_END_ALLOW_THREADS} macro.  See above for further
discussion of this macro.  It is a no-op when thread support is
disabled at compile time.
\end{csimplemacrodesc}

\begin{csimplemacrodesc}{Py_END_ALLOW_THREADS}
This macro expands to
\samp{PyEval_RestoreThread(_save); \}}.
Note that it contains a closing brace; it must be matched with an
earlier \code{Py_BEGIN_ALLOW_THREADS} macro.  See above for further
discussion of this macro.  It is a no-op when thread support is
disabled at compile time.
\end{csimplemacrodesc}

\begin{csimplemacrodesc}{Py_BLOCK_THREADS}
This macro expands to \samp{PyEval_RestoreThread(_save);}: it
is equivalent to \code{Py_END_ALLOW_THREADS} without the closing
brace.  It is a no-op when thread support is disabled at compile
time.
\end{csimplemacrodesc}

\begin{csimplemacrodesc}{Py_UNBLOCK_THREADS}
This macro expands to \samp{_save = PyEval_SaveThread();}: it is
equivalent to \code{Py_BEGIN_ALLOW_THREADS} without the opening brace
and variable declaration.  It is a no-op when thread support is
disabled at compile time.
\end{csimplemacrodesc}

All of the following functions are only available when thread support
is enabled at compile time, and must be called only when the
interpreter lock has been created.

\begin{cfuncdesc}{PyInterpreterState*}{PyInterpreterState_New}{}
Create a new interpreter state object.  The interpreter lock need not
be held, but may be held if it is necessary to serialize calls to this
function.
\end{cfuncdesc}

\begin{cfuncdesc}{void}{PyInterpreterState_Clear}{PyInterpreterState *interp}
Reset all information in an interpreter state object.  The interpreter
lock must be held.
\end{cfuncdesc}

\begin{cfuncdesc}{void}{PyInterpreterState_Delete}{PyInterpreterState *interp}
Destroy an interpreter state object.  The interpreter lock need not be
held.  The interpreter state must have been reset with a previous
call to \cfunction{PyInterpreterState_Clear()}.
\end{cfuncdesc}

\begin{cfuncdesc}{PyThreadState*}{PyThreadState_New}{PyInterpreterState *interp}
Create a new thread state object belonging to the given interpreter
object.  The interpreter lock need not be held, but may be held if it
is necessary to serialize calls to this function.
\end{cfuncdesc}

\begin{cfuncdesc}{void}{PyThreadState_Clear}{PyThreadState *tstate}
Reset all information in a thread state object.  The interpreter lock
must be held.
\end{cfuncdesc}

\begin{cfuncdesc}{void}{PyThreadState_Delete}{PyThreadState *tstate}
Destroy a thread state object.  The interpreter lock need not be
held.  The thread state must have been reset with a previous
call to \cfunction{PyThreadState_Clear()}.
\end{cfuncdesc}

\begin{cfuncdesc}{PyThreadState*}{PyThreadState_Get}{}
Return the current thread state.  The interpreter lock must be held.
When the current thread state is \NULL{}, this issues a fatal
error (so that the caller needn't check for \NULL{}).
\end{cfuncdesc}

\begin{cfuncdesc}{PyThreadState*}{PyThreadState_Swap}{PyThreadState *tstate}
Swap the current thread state with the thread state given by the
argument \var{tstate}, which may be \NULL{}.  The interpreter lock
must be held.
\end{cfuncdesc}

\begin{cfuncdesc}{PyObject*}{PyThreadState_GetDict}{}
Return a dictionary in which extensions can store thread-specific
state information.  Each extension should use a unique key to use to
store state in the dictionary.  If this function returns \NULL, an
exception has been raised and the caller should allow it to
propogate.
\end{cfuncdesc}


\section{Profiling and Tracing \label{profiling}}

\sectionauthor{Fred L. Drake, Jr.}{fdrake@acm.org}

The Python interpreter provides some low-level support for attaching
profiling and execution tracing facilities.  These are used for
profiling, debugging, and coverage analysis tools.

Starting with Python 2.2, the implementation of this facility was
substantially revised, and an interface from C was added.  This C
interface allows the profiling or tracing code to avoid the overhead
of calling through Python-level callable objects, making a direct C
function call instead.  The essential attributes of the facility have
not changed; the interface allows trace functions to be installed
per-thread, and the basic events reported to the trace function are
the same as had been reported to the Python-level trace functions in
previous versions.

\begin{ctypedesc}[Py_tracefunc]{int (*Py_tracefunc)(PyObject *obj,
                                PyFrameObject *frame, int what,
                                PyObject *arg)}
  The type of the trace function registered using
  \cfunction{PyEval_SetProfile()} and \cfunction{PyEval_SetTrace()}.
  The first parameter is the object passed to the registration
  function, 
\end{ctypedesc}

\begin{cvardesc}{int}{PyTrace_CALL}
  The value of the \var{what} parameter to a \ctype{Py_tracefunc}
  function when a new function or method call is being reported.
\end{cvardesc}

\begin{cvardesc}{int}{PyTrace_EXCEPT}
\end{cvardesc}

\begin{cvardesc}{int}{PyTrace_LINE}
  The value passed as the \var{what} parameter to a trace function
  (but not a profiling function) when a line-number event is being
  reported.
\end{cvardesc}

\begin{cvardesc}{int}{PyTrace_RETURN}
  The value for the \var{what} parameter to \ctype{Py_tracefunc}
  functions when a call is returning without propogating an exception.
\end{cvardesc}

\begin{cfuncdesc}{void}{PyEval_SetProfile}{Py_tracefunc func, PyObject *obj}
  Set the profiler function to \var{func}.  The \var{obj} parameter is
  passed to the function as its first parameter, and may be any Python
  object, or \NULL.  If the profile function needs to maintain state,
  using a different value for \var{obj} for each thread provides a
  convenient and thread-safe place to store it.  The profile function
  is called for all monitored events except the line-number events.
\end{cfuncdesc}

\begin{cfuncdesc}{void}{PyEval_SetTrace}{Py_tracefunc func, PyObject *obj}
  Set the the tracing function to \var{func}.  This is similar to
  \cfunction{PyEval_SetProfile()}, except the tracing function does
  receive line-number events.
\end{cfuncdesc}


\section{Advanced Debugger Support \label{advanced-debugging}}
\sectionauthor{Fred L. Drake, Jr.}{fdrake@acm.org}

These functions are only intended to be used by advanced debugging
tools.

\begin{cfuncdesc}{PyInterpreterState*}{PyInterpreterState_Head}{}
Return the interpreter state object at the head of the list of all
such objects.
\versionadded{2.2}
\end{cfuncdesc}

\begin{cfuncdesc}{PyInterpreterState*}{PyInterpreterState_Next}{PyInterpreterState *interp}
Return the next interpreter state object after \var{interp} from the
list of all such objects.
\versionadded{2.2}
\end{cfuncdesc}

\begin{cfuncdesc}{PyThreadState *}{PyInterpreterState_ThreadHead}{PyInterpreterState *interp}
Return the a pointer to the first \ctype{PyThreadState} object in the
list of threads associated with the interpreter \var{interp}.
\versionadded{2.2}
\end{cfuncdesc}

\begin{cfuncdesc}{PyThreadState*}{PyThreadState_Next}{PyThreadState *tstate}
Return the next thread state object after \var{tstate} from the list
of all such objects belonging to the same \ctype{PyInterpreterState}
object.
\versionadded{2.2}
\end{cfuncdesc}


\chapter{Memory Management \label{memory}}
\sectionauthor{Vladimir Marangozov}{Vladimir.Marangozov@inrialpes.fr}


\section{Overview \label{memoryOverview}}

Memory management in Python involves a private heap containing all
Python objects and data structures. The management of this private
heap is ensured internally by the \emph{Python memory manager}.  The
Python memory manager has different components which deal with various
dynamic storage management aspects, like sharing, segmentation,
preallocation or caching.

At the lowest level, a raw memory allocator ensures that there is
enough room in the private heap for storing all Python-related data
by interacting with the memory manager of the operating system. On top
of the raw memory allocator, several object-specific allocators
operate on the same heap and implement distinct memory management
policies adapted to the peculiarities of every object type. For
example, integer objects are managed differently within the heap than
strings, tuples or dictionaries because integers imply different
storage requirements and speed/space tradeoffs. The Python memory
manager thus delegates some of the work to the object-specific
allocators, but ensures that the latter operate within the bounds of
the private heap.

It is important to understand that the management of the Python heap
is performed by the interpreter itself and that the user has no
control on it, even if she regularly manipulates object pointers to
memory blocks inside that heap.  The allocation of heap space for
Python objects and other internal buffers is performed on demand by
the Python memory manager through the Python/C API functions listed in
this document.

To avoid memory corruption, extension writers should never try to
operate on Python objects with the functions exported by the C
library: \cfunction{malloc()}\ttindex{malloc()},
\cfunction{calloc()}\ttindex{calloc()},
\cfunction{realloc()}\ttindex{realloc()} and
\cfunction{free()}\ttindex{free()}.  This will result in 
mixed calls between the C allocator and the Python memory manager
with fatal consequences, because they implement different algorithms
and operate on different heaps.  However, one may safely allocate and
release memory blocks with the C library allocator for individual
purposes, as shown in the following example:

\begin{verbatim}
    PyObject *res;
    char *buf = (char *) malloc(BUFSIZ); /* for I/O */

    if (buf == NULL)
        return PyErr_NoMemory();
    ...Do some I/O operation involving buf...
    res = PyString_FromString(buf);
    free(buf); /* malloc'ed */
    return res;
\end{verbatim}

In this example, the memory request for the I/O buffer is handled by
the C library allocator. The Python memory manager is involved only
in the allocation of the string object returned as a result.

In most situations, however, it is recommended to allocate memory from
the Python heap specifically because the latter is under control of
the Python memory manager. For example, this is required when the
interpreter is extended with new object types written in C. Another
reason for using the Python heap is the desire to \emph{inform} the
Python memory manager about the memory needs of the extension module.
Even when the requested memory is used exclusively for internal,
highly-specific purposes, delegating all memory requests to the Python
memory manager causes the interpreter to have a more accurate image of
its memory footprint as a whole. Consequently, under certain
circumstances, the Python memory manager may or may not trigger
appropriate actions, like garbage collection, memory compaction or
other preventive procedures. Note that by using the C library
allocator as shown in the previous example, the allocated memory for
the I/O buffer escapes completely the Python memory manager.


\section{Memory Interface \label{memoryInterface}}

The following function sets, modeled after the ANSI C standard, are
available for allocating and releasing memory from the Python heap:


\begin{cfuncdesc}{void*}{PyMem_Malloc}{size_t n}
Allocates \var{n} bytes and returns a pointer of type \ctype{void*} to
the allocated memory, or \NULL{} if the request fails.  Requesting zero
bytes returns a non-\NULL{} pointer.
The memory will not have been initialized in any way.
\end{cfuncdesc}

\begin{cfuncdesc}{void*}{PyMem_Realloc}{void *p, size_t n}
Resizes the memory block pointed to by \var{p} to \var{n} bytes. The
contents will be unchanged to the minimum of the old and the new
sizes. If \var{p} is \NULL{}, the call is equivalent to
\cfunction{PyMem_Malloc(\var{n})}; if \var{n} is equal to zero, the
memory block is resized but is not freed, and the returned pointer is
non-\NULL{}.  Unless \var{p} is \NULL{}, it must have been returned by
a previous call to \cfunction{PyMem_Malloc()} or
\cfunction{PyMem_Realloc()}.
\end{cfuncdesc}

\begin{cfuncdesc}{void}{PyMem_Free}{void *p}
Frees the memory block pointed to by \var{p}, which must have been
returned by a previous call to \cfunction{PyMem_Malloc()} or
\cfunction{PyMem_Realloc()}.  Otherwise, or if
\cfunction{PyMem_Free(p)} has been called before, undefined behaviour
occurs. If \var{p} is \NULL{}, no operation is performed.
\end{cfuncdesc}

The following type-oriented macros are provided for convenience.  Note 
that \var{TYPE} refers to any C type.

\begin{cfuncdesc}{\var{TYPE}*}{PyMem_New}{TYPE, size_t n}
Same as \cfunction{PyMem_Malloc()}, but allocates \code{(\var{n} *
sizeof(\var{TYPE}))} bytes of memory.  Returns a pointer cast to
\ctype{\var{TYPE}*}.
The memory will not have been initialized in any way.
\end{cfuncdesc}

\begin{cfuncdesc}{\var{TYPE}*}{PyMem_Resize}{void *p, TYPE, size_t n}
Same as \cfunction{PyMem_Realloc()}, but the memory block is resized
to \code{(\var{n} * sizeof(\var{TYPE}))} bytes.  Returns a pointer
cast to \ctype{\var{TYPE}*}.
\end{cfuncdesc}

\begin{cfuncdesc}{void}{PyMem_Del}{void *p}
Same as \cfunction{PyMem_Free()}.
\end{cfuncdesc}

In addition, the following macro sets are provided for calling the
Python memory allocator directly, without involving the C API functions
listed above. However, note that their use does not preserve binary
compatibility accross Python versions and is therefore deprecated in
extension modules.

\cfunction{PyMem_MALLOC()}, \cfunction{PyMem_REALLOC()}, \cfunction{PyMem_FREE()}.

\cfunction{PyMem_NEW()}, \cfunction{PyMem_RESIZE()}, \cfunction{PyMem_DEL()}.


\section{Examples \label{memoryExamples}}

Here is the example from section \ref{memoryOverview}, rewritten so
that the I/O buffer is allocated from the Python heap by using the
first function set:

\begin{verbatim}
    PyObject *res;
    char *buf = (char *) PyMem_Malloc(BUFSIZ); /* for I/O */

    if (buf == NULL)
        return PyErr_NoMemory();
    /* ...Do some I/O operation involving buf... */
    res = PyString_FromString(buf);
    PyMem_Free(buf); /* allocated with PyMem_Malloc */
    return res;
\end{verbatim}

The same code using the type-oriented function set:

\begin{verbatim}
    PyObject *res;
    char *buf = PyMem_New(char, BUFSIZ); /* for I/O */

    if (buf == NULL)
        return PyErr_NoMemory();
    /* ...Do some I/O operation involving buf... */
    res = PyString_FromString(buf);
    PyMem_Del(buf); /* allocated with PyMem_New */
    return res;
\end{verbatim}

Note that in the two examples above, the buffer is always
manipulated via functions belonging to the same set. Indeed, it
is required to use the same memory API family for a given
memory block, so that the risk of mixing different allocators is
reduced to a minimum. The following code sequence contains two errors,
one of which is labeled as \emph{fatal} because it mixes two different
allocators operating on different heaps.

\begin{verbatim}
char *buf1 = PyMem_New(char, BUFSIZ);
char *buf2 = (char *) malloc(BUFSIZ);
char *buf3 = (char *) PyMem_Malloc(BUFSIZ);
...
PyMem_Del(buf3);  /* Wrong -- should be PyMem_Free() */
free(buf2);       /* Right -- allocated via malloc() */
free(buf1);       /* Fatal -- should be PyMem_Del()  */
\end{verbatim}

In addition to the functions aimed at handling raw memory blocks from
the Python heap, objects in Python are allocated and released with
\cfunction{PyObject_New()}, \cfunction{PyObject_NewVar()} and
\cfunction{PyObject_Del()}, or with their corresponding macros
\cfunction{PyObject_NEW()}, \cfunction{PyObject_NEW_VAR()} and
\cfunction{PyObject_DEL()}.

These will be explained in the next chapter on defining and
implementing new object types in C.


\chapter{Defining New Object Types \label{newTypes}}


\section{Allocating Objects on the Heap
         \label{allocating-objects}}

\begin{cfuncdesc}{PyObject*}{_PyObject_New}{PyTypeObject *type}
\end{cfuncdesc}

\begin{cfuncdesc}{PyVarObject*}{_PyObject_NewVar}{PyTypeObject *type, int size}
\end{cfuncdesc}

\begin{cfuncdesc}{void}{_PyObject_Del}{PyObject *op}
\end{cfuncdesc}

\begin{cfuncdesc}{PyObject*}{PyObject_Init}{PyObject *op,
					    PyTypeObject *type}
  Initialize a newly-allocated object \var{op} with its type and
  initial reference.  Returns the initialized object.  If \var{type}
  indicates that the object participates in the cyclic garbage
  detector, it it added to the detector's set of observed objects.
  Other fields of the object are not affected.
\end{cfuncdesc}

\begin{cfuncdesc}{PyVarObject*}{PyObject_InitVar}{PyVarObject *op,
						  PyTypeObject *type, int size}
  This does everything \cfunction{PyObject_Init()} does, and also
  initializes the length information for a variable-size object.
\end{cfuncdesc}

\begin{cfuncdesc}{\var{TYPE}*}{PyObject_New}{TYPE, PyTypeObject *type}
  Allocate a new Python object using the C structure type \var{TYPE}
  and the Python type object \var{type}.  Fields not defined by the
  Python object header are not initialized; the object's reference
  count will be one.  The size of the memory
  allocation is determined from the \member{tp_basicsize} field of the
  type object.
\end{cfuncdesc}

\begin{cfuncdesc}{\var{TYPE}*}{PyObject_NewVar}{TYPE, PyTypeObject *type,
                                                int size}
  Allocate a new Python object using the C structure type \var{TYPE}
  and the Python type object \var{type}.  Fields not defined by the
  Python object header are not initialized.  The allocated memory
  allows for the \var{TYPE} structure plus \var{size} fields of the
  size given by the \member{tp_itemsize} field of \var{type}.  This is
  useful for implementing objects like tuples, which are able to
  determine their size at construction time.  Embedding the array of
  fields into the same allocation decreases the number of allocations,
  improving the memory management efficiency.
\end{cfuncdesc}

\begin{cfuncdesc}{void}{PyObject_Del}{PyObject *op}
  Releases memory allocated to an object using
  \cfunction{PyObject_New()} or \cfunction{PyObject_NewVar()}.  This
  is normally called from the \member{tp_dealloc} handler specified in
  the object's type.  The fields of the object should not be accessed
  after this call as the memory is no longer a valid Python object.
\end{cfuncdesc}

\begin{cfuncdesc}{\var{TYPE}*}{PyObject_NEW}{TYPE, PyTypeObject *type}
  Macro version of \cfunction{PyObject_New()}, to gain performance at
  the expense of safety.  This does not check \var{type} for a \NULL{}
  value.
\end{cfuncdesc}

\begin{cfuncdesc}{\var{TYPE}*}{PyObject_NEW_VAR}{TYPE, PyTypeObject *type,
                                                int size}
  Macro version of \cfunction{PyObject_NewVar()}, to gain performance
  at the expense of safety.  This does not check \var{type} for a
  \NULL{} value.
\end{cfuncdesc}

\begin{cfuncdesc}{void}{PyObject_DEL}{PyObject *op}
  Macro version of \cfunction{PyObject_Del()}.
\end{cfuncdesc}

\begin{cfuncdesc}{PyObject*}{Py_InitModule}{char *name,
                                            PyMethodDef *methods}
  Create a new module object based on a name and table of functions,
  returning the new module object.
\end{cfuncdesc}

\begin{cfuncdesc}{PyObject*}{Py_InitModule3}{char *name,
                                             PyMethodDef *methods,
                                             char *doc}
  Create a new module object based on a name and table of functions,
  returning the new module object.  If \var{doc} is non-\NULL, it will
  be used to define the docstring for the module.
\end{cfuncdesc}

\begin{cfuncdesc}{PyObject*}{Py_InitModule4}{char *name,
                                             PyMethodDef *methods,
                                             char *doc, PyObject *self,
                                             int apiver}
  Create a new module object based on a name and table of functions,
  returning the new module object.  If \var{doc} is non-\NULL, it will
  be used to define the docstring for the module.  If \var{self} is
  non-\NULL, it will passed to the functions of the module as their
  (otherwise \NULL) first parameter.  (This was added as an
  experimental feature, and there are no known uses in the current
  version of Python.)  For \var{apiver}, the only value which should
  be passed is defined by the constant \constant{PYTHON_API_VERSION}.

  \strong{Note:}  Most uses of this function should probably be using
  the \cfunction{Py_InitModule3()} instead; only use this if you are
  sure you need it.
\end{cfuncdesc}

DL_IMPORT

\begin{cvardesc}{PyObject}{_Py_NoneStruct}
  Object which is visible in Python as \code{None}.  This should only
  be accessed using the \code{Py_None} macro, which evaluates to a
  pointer to this object.
\end{cvardesc}


\section{Common Object Structures \label{common-structs}}

PyObject, PyVarObject

PyObject_HEAD, PyObject_HEAD_INIT, PyObject_VAR_HEAD

Typedefs:
unaryfunc, binaryfunc, ternaryfunc, inquiry, coercion, intargfunc,
intintargfunc, intobjargproc, intintobjargproc, objobjargproc,
destructor, printfunc, getattrfunc, getattrofunc, setattrfunc,
setattrofunc, cmpfunc, reprfunc, hashfunc

\begin{ctypedesc}{PyCFunction}
Type of the functions used to implement most Python callables in C.
\end{ctypedesc}

\begin{ctypedesc}{PyMethodDef}
Structure used to describe a method of an extension type.  This
structure has four fields:

\begin{tableiii}{l|l|l}{member}{Field}{C Type}{Meaning}
  \lineiii{ml_name}{char *}{name of the method}
  \lineiii{ml_meth}{PyCFunction}{pointer to the C implementation}
  \lineiii{ml_flags}{int}{flag bits indicating how the call should be
                          constructed}
  \lineiii{ml_doc}{char *}{points to the contents of the docstring}
\end{tableiii}
\end{ctypedesc}

The \var{ml_meth} is a C function pointer. The functions may be of
different types, but they always return \ctype{PyObject*}. If the
function is not of the \ctype{PyCFunction}, the compiler will require
a cast in the method table. Even though \ctype{PyCFunction} defines
the first parameter as \ctype{PyObject*}, it is common that the method
implementation uses a the specific C type of the \var{self} object.

The flags can have the following values. Only METH_VARARGS and
METH_KEYWORDS can be combined; the others can't.

\begin{datadesc}{METH_VARARGS}

This is the typical calling convention, where the methods have the
type \ctype{PyMethodDef}. The function expects two \ctype{PyObject*}.
The first one is the \var{self} object for methods; for module
functions, it has the value given to \cfunction{PyInitModule4} (or
\NULL{} if \cfunction{PyInitModule} was used). The second parameter
(often called \var{args}) is a tuple object representing all
arguments. This parameter is typically processed using
\cfunction{PyArg_ParseTuple}.

\end{datadesc}

\begin{datadesc}{METH_KEYWORDS}

Methods with these flags must be of type
\ctype{PyCFunctionWithKeywords}.  The function expects three
parameters: \var{self}, \var{args}, and a dictionary of all the keyword
arguments. The flag is typically combined with METH_VARARGS, and the
parameters are typically processed using
\cfunction{PyArg_ParseTupleAndKeywords}.

\end{datadesc}

\begin{datadesc}{METH_NOARGS}

Methods without parameters don't need to check whether arguments are
given if they are listed with the \code{METH_NOARGS} flag. They need
to be of type \ctype{PyNoArgsFunction}, i.e. they expect a single
\var{self} parameter.

\end{datadesc}

\begin{datadesc}{METH_O}

Methods with a single object argument can be listed with the
\code{METH_O} flag, instead of invoking \cfunction{PyArg_ParseTuple}
with a \code{``O''} argument. They have the type \ctype{PyCFunction},
with the \var{self} parameter, and a \ctype{PyObject*} parameter
representing the single argument.

\end{datadesc}

\begin{datadesc}{METH_OLDARGS}

This calling convention is deprecated. The method must be of type
\ctype{PyCFunction}. The second argument is \NULL{} if no arguments
are given, a single object if exactly one argument is given, and a
tuple of objects if more than one argument is given.

\end{datadesc}

\begin{cfuncdesc}{PyObject*}{Py_FindMethod}{PyMethodDef[] table,
                                            PyObject *ob, char *name}
Return a bound method object for an extension type implemented in C.
This function also handles the special attribute \member{__methods__},
returning a list of all the method names defined in \var{table}.
\end{cfuncdesc}


\section{Mapping Object Structures \label{mapping-structs}}

\begin{ctypedesc}{PyMappingMethods}
Structure used to hold pointers to the functions used to implement the 
mapping protocol for an extension type.
\end{ctypedesc}


\section{Number Object Structures \label{number-structs}}

\begin{ctypedesc}{PyNumberMethods}
Structure used to hold pointers to the functions an extension type
uses to implement the number protocol.
\end{ctypedesc}


\section{Sequence Object Structures \label{sequence-structs}}

\begin{ctypedesc}{PySequenceMethods}
Structure used to hold pointers to the functions which an object uses
to implement the sequence protocol.
\end{ctypedesc}


\section{Buffer Object Structures \label{buffer-structs}}
\sectionauthor{Greg J. Stein}{greg@lyra.org}

The buffer interface exports a model where an object can expose its
internal data as a set of chunks of data, where each chunk is
specified as a pointer/length pair.  These chunks are called
\dfn{segments} and are presumed to be non-contiguous in memory.

If an object does not export the buffer interface, then its
\member{tp_as_buffer} member in the \ctype{PyTypeObject} structure
should be \NULL{}.  Otherwise, the \member{tp_as_buffer} will point to
a \ctype{PyBufferProcs} structure.

\strong{Note:} It is very important that your
\ctype{PyTypeObject} structure uses \constant{Py_TPFLAGS_DEFAULT} for
the value of the \member{tp_flags} member rather than \code{0}.  This
tells the Python runtime that your \ctype{PyBufferProcs} structure
contains the \member{bf_getcharbuffer} slot. Older versions of Python
did not have this member, so a new Python interpreter using an old
extension needs to be able to test for its presence before using it.

\begin{ctypedesc}{PyBufferProcs}
Structure used to hold the function pointers which define an
implementation of the buffer protocol.

The first slot is \member{bf_getreadbuffer}, of type
\ctype{getreadbufferproc}.  If this slot is \NULL{}, then the object
does not support reading from the internal data.  This is
non-sensical, so implementors should fill this in, but callers should
test that the slot contains a non-\NULL{} value.

The next slot is \member{bf_getwritebuffer} having type
\ctype{getwritebufferproc}. This slot may be \NULL{} if the object
does not allow writing into its returned buffers.

The third slot is \member{bf_getsegcount}, with type
\ctype{getsegcountproc}.  This slot must not be \NULL{} and is used to 
inform the caller how many segments the object contains.  Simple
objects such as \ctype{PyString_Type} and
\ctype{PyBuffer_Type} objects contain a single segment.

The last slot is \member{bf_getcharbuffer}, of type
\ctype{getcharbufferproc}.  This slot will only be present if the
\constant{Py_TPFLAGS_HAVE_GETCHARBUFFER} flag is present in the
\member{tp_flags} field of the object's \ctype{PyTypeObject}.  Before using
this slot, the caller should test whether it is present by using the
\cfunction{PyType_HasFeature()}\ttindex{PyType_HasFeature()} function.
If present, it may be \NULL, indicating that the object's contents
cannot be used as \emph{8-bit characters}.
The slot function may also raise an error if the object's contents
cannot be interpreted as 8-bit characters.  For example, if the object
is an array which is configured to hold floating point values, an
exception may be raised if a caller attempts to use
\member{bf_getcharbuffer} to fetch a sequence of 8-bit characters.
This notion of exporting the internal buffers as ``text'' is used to
distinguish between objects that are binary in nature, and those which
have character-based content.

\strong{Note:} The current policy seems to state that these characters
may be multi-byte characters. This implies that a buffer size of
\var{N} does not mean there are \var{N} characters present.
\end{ctypedesc}

\begin{datadesc}{Py_TPFLAGS_HAVE_GETCHARBUFFER}
Flag bit set in the type structure to indicate that the
\member{bf_getcharbuffer} slot is known.  This being set does not
indicate that the object supports the buffer interface or that the
\member{bf_getcharbuffer} slot is non-\NULL.
\end{datadesc}

\begin{ctypedesc}[getreadbufferproc]{int (*getreadbufferproc)
                            (PyObject *self, int segment, void **ptrptr)}
Return a pointer to a readable segment of the buffer.  This function
is allowed to raise an exception, in which case it must return
\code{-1}.  The \var{segment} which is passed must be zero or
positive, and strictly less than the number of segments returned by
the \member{bf_getsegcount} slot function.  On success, it returns the
length of the buffer memory, and sets \code{*\var{ptrptr}} to a
pointer to that memory.
\end{ctypedesc}

\begin{ctypedesc}[getwritebufferproc]{int (*getwritebufferproc)
                            (PyObject *self, int segment, void **ptrptr)}
Return a pointer to a writable memory buffer in \code{*\var{ptrptr}},
and the length of that segment as the function return value.
The memory buffer must correspond to buffer segment \var{segment}.
Must return \code{-1} and set an exception on error.
\exception{TypeError} should be raised if the object only supports
read-only buffers, and \exception{SystemError} should be raised when
\var{segment} specifies a segment that doesn't exist.
% Why doesn't it raise ValueError for this one?
% GJS: because you shouldn't be calling it with an invalid
%      segment. That indicates a blatant programming error in the C
%      code.
\end{ctypedesc}

\begin{ctypedesc}[getsegcountproc]{int (*getsegcountproc)
                            (PyObject *self, int *lenp)}
Return the number of memory segments which comprise the buffer.  If
\var{lenp} is not \NULL, the implementation must report the sum of the 
sizes (in bytes) of all segments in \code{*\var{lenp}}.
The function cannot fail.
\end{ctypedesc}

\begin{ctypedesc}[getcharbufferproc]{int (*getcharbufferproc)
                            (PyObject *self, int segment, const char **ptrptr)}
\end{ctypedesc}


\section{Supporting the Iterator Protocol
         \label{supporting-iteration}}


\section{Supporting Cyclic Garbarge Collection
         \label{supporting-cycle-detection}}

Python's support for detecting and collecting garbage which involves
circular references requires support from object types which are
``containers'' for other objects which may also be containers.  Types
which do not store references to other objects, or which only store
references to atomic types (such as numbers or strings), do not need
to provide any explicit support for garbage collection.

To create a container type, the \member{tp_flags} field of the type
object must include the \constant{Py_TPFLAGS_HAVE_GC} and provide an
implementation of the \member{tp_traverse} handler.  If instances of the
type are mutable, a \member{tp_clear} implementation must also be
provided.

\begin{datadesc}{Py_TPFLAGS_HAVE_GC}
  Objects with a type with this flag set must conform with the rules
  documented here.  For convenience these objects will be referred to
  as container objects.
\end{datadesc}

Constructors for container types must conform to two rules:

\begin{enumerate}
\item  The memory for the object must be allocated using
       \cfunction{PyObject_GC_New()} or \cfunction{PyObject_GC_VarNew()}.

\item  Once all the fields which may contain references to other
       containers are initialized, it must call
       \cfunction{PyObject_GC_Track()}.
\end{enumerate}

\begin{cfuncdesc}{\var{TYPE}*}{PyObject_GC_New}{TYPE, PyTypeObject *type}
  Analogous to \cfunction{PyObject_New()} but for container objects with
  the \constant{Py_TPFLAGS_HAVE_GC} flag set.
\end{cfuncdesc}

\begin{cfuncdesc}{\var{TYPE}*}{PyObject_GC_NewVar}{TYPE, PyTypeObject *type,
                                                   int size}
  Analogous to \cfunction{PyObject_NewVar()} but for container objects
  with the \constant{Py_TPFLAGS_HAVE_GC} flag set.
\end{cfuncdesc}

\begin{cfuncdesc}{PyVarObject *}{PyObject_GC_Resize}{PyVarObject *op, int}
  Resize an object allocated by \cfunction{PyObject_NewVar()}.  Returns
  the resized object or \NULL{} on failure.
\end{cfuncdesc}

\begin{cfuncdesc}{void}{PyObject_GC_Track}{PyObject *op}
  Adds the object \var{op} to the set of container objects tracked by
  the collector.  The collector can run at unexpected times so objects
  must be valid while being tracked.  This should be called once all
  the fields followed by the \member{tp_traverse} handler become valid,
  usually near the end of the constructor.
\end{cfuncdesc}

\begin{cfuncdesc}{void}{_PyObject_GC_TRACK}{PyObject *op}
  A macro version of \cfunction{PyObject_GC_Track()}.  It should not be
  used for extension modules.
\end{cfuncdesc}

Similarly, the deallocator for the object must conform to a similar
pair of rules:

\begin{enumerate}
\item  Before fields which refer to other containers are invalidated,
       \cfunction{PyObject_GC_UnTrack()} must be called.

\item  The object's memory must be deallocated using
       \cfunction{PyObject_GC_Del()}.
\end{enumerate}

\begin{cfuncdesc}{void}{PyObject_GC_Del}{PyObject *op}
  Releases memory allocated to an object using
  \cfunction{PyObject_GC_New()} or \cfunction{PyObject_GC_NewVar()}.
\end{cfuncdesc}

\begin{cfuncdesc}{void}{PyObject_GC_UnTrack}{PyObject *op}
  Remove the object \var{op} from the set of container objects tracked
  by the collector.  Note that \cfunction{PyObject_GC_Track()} can be
  called again on this object to add it back to the set of tracked
  objects.  The deallocator (\member{tp_dealloc} handler) should call
  this for the object before any of the fields used by the
  \member{tp_traverse} handler become invalid.
\end{cfuncdesc}

\begin{cfuncdesc}{void}{_PyObject_GC_UNTRACK}{PyObject *op}
  A macro version of \cfunction{PyObject_GC_UnTrack()}.  It should not be
  used for extension modules.
\end{cfuncdesc}

The \member{tp_traverse} handler accepts a function parameter of this
type:

\begin{ctypedesc}[visitproc]{int (*visitproc)(PyObject *object, void *arg)}
  Type of the visitor function passed to the \member{tp_traverse}
  handler.  The function should be called with an object to traverse
  as \var{object} and the third parameter to the \member{tp_traverse}
  handler as \var{arg}.
\end{ctypedesc}

The \member{tp_traverse} handler must have the following type:

\begin{ctypedesc}[traverseproc]{int (*traverseproc)(PyObject *self,
                                visitproc visit, void *arg)}
  Traversal function for a container object.  Implementations must
  call the \var{visit} function for each object directly contained by
  \var{self}, with the parameters to \var{visit} being the contained
  object and the \var{arg} value passed to the handler.  If
  \var{visit} returns a non-zero value then an error has occurred and
  that value should be returned immediately.
\end{ctypedesc}

The \member{tp_clear} handler must be of the \ctype{inquiry} type, or
\NULL{} if the object is immutable.

\begin{ctypedesc}[inquiry]{int (*inquiry)(PyObject *self)}
  Drop references that may have created reference cycles.  Immutable
  objects do not have to define this method since they can never
  directly create reference cycles.  Note that the object must still
  be valid after calling this method (don't just call
  \cfunction{Py_DECREF()} on a reference).  The collector will call
  this method if it detects that this object is involved in a
  reference cycle.
\end{ctypedesc}


\subsection{Example Cycle Collector Support
            \label{example-cycle-support}}

This example shows only enough of the implementation of an extension
type to show how the garbage collector support needs to be added.  It
shows the definition of the object structure, the
\member{tp_traverse}, \member{tp_clear} and \member{tp_dealloc}
implementations, the type structure, and a constructor --- the module
initialization needed to export the constructor to Python is not shown
as there are no special considerations there for the collector.  To
make this interesting, assume that the module exposes ways for the
\member{container} field of the object to be modified.  Note that
since no checks are made on the type of the object used to initialize
\member{container}, we have to assume that it may be a container.

\begin{verbatim}
#include "Python.h"

typedef struct {
    PyObject_HEAD
    PyObject *container;
} MyObject;

static int
my_traverse(MyObject *self, visitproc visit, void *arg)
{
    if (self->container != NULL)
        return visit(self->container, arg);
    else
        return 0;
}

static int
my_clear(MyObject *self)
{
    Py_XDECREF(self->container);
    self->container = NULL;

    return 0;
}

static void
my_dealloc(MyObject *self)
{
    PyObject_GC_UnTrack((PyObject *) self);
    Py_XDECREF(self->container);
    PyObject_GC_Del(self);
}
\end{verbatim}

\begin{verbatim}
statichere PyTypeObject
MyObject_Type = {
    PyObject_HEAD_INIT(NULL)
    0,
    "MyObject",
    sizeof(MyObject),
    0,
    (destructor)my_dealloc,     /* tp_dealloc */
    0,                          /* tp_print */
    0,                          /* tp_getattr */
    0,                          /* tp_setattr */
    0,                          /* tp_compare */
    0,                          /* tp_repr */
    0,                          /* tp_as_number */
    0,                          /* tp_as_sequence */
    0,                          /* tp_as_mapping */
    0,                          /* tp_hash */
    0,                          /* tp_call */
    0,                          /* tp_str */
    0,                          /* tp_getattro */
    0,                          /* tp_setattro */
    0,                          /* tp_as_buffer */
    Py_TPFLAGS_DEFAULT | Py_TPFLAGS_HAVE_GC,
    0,                          /* tp_doc */
    (traverseproc)my_traverse,  /* tp_traverse */
    (inquiry)my_clear,          /* tp_clear */
    0,                          /* tp_richcompare */
    0,                          /* tp_weaklistoffset */
};

/* This constructor should be made accessible from Python. */
static PyObject *
new_object(PyObject *unused, PyObject *args)
{
    PyObject *container = NULL;
    MyObject *result = NULL;

    if (PyArg_ParseTuple(args, "|O:new_object", &container)) {
        result = PyObject_GC_New(MyObject, &MyObject_Type);
        if (result != NULL) {
            result->container = container;
            PyObject_GC_Track(result);
        }
    }
    return (PyObject *) result;
}
\end{verbatim}


% \chapter{Debugging \label{debugging}}
%
% XXX Explain Py_DEBUG, Py_TRACE_REFS, Py_REF_DEBUG.


\appendix
\chapter{Reporting Bugs}
\label{reporting-bugs}

Python is a mature programming language which has established a
reputation for stability.  In order to maintain this reputation, the
developers would like to know of any deficiencies you find in Python
or its documentation.

All bug reports should be submitted via the Python Bug Tracker on
SourceForge (\url{http://sourceforge.net/bugs/?group_id=5470}).  The
bug tracker offers a Web form which allows pertinent information to be
entered and submitted to the developers.

Before submitting a report, please log into SourceForge if you are a
member; this will make it possible for the developers to contact you
for additional information if needed.  If you are not a SourceForge
member but would not mind the developers contacting you, you may
include your email address in your bug description.  In this case,
please realize that the information is publically available and cannot
be protected.

The first step in filing a report is to determine whether the problem
has already been reported.  The advantage in doing so, aside from
saving the developers time, is that you learn what has been done to
fix it; it may be that the problem has already been fixed for the next
release, or additional information is needed (in which case you are
welcome to provide it if you can!).  To do this, search the bug
database using the search box near the bottom of the page.

If the problem you're reporting is not already in the bug tracker, go
back to the Python Bug Tracker
(\url{http://sourceforge.net/bugs/?group_id=5470}).  Select the
``Submit a Bug'' link at the top of the page to open the bug reporting
form.

The submission form has a number of fields.  The only fields that are
required are the ``Summary'' and ``Details'' fields.  For the summary,
enter a \emph{very} short description of the problem; less than ten
words is good.  In the Details field, describe the problem in detail,
including what you expected to happen and what did happen.  Be sure to
include the version of Python you used, whether any extension modules
were involved, and what hardware and software platform you were using
(including version information as appropriate).

The only other field that you may want to set is the ``Category''
field, which allows you to place the bug report into a broad category
(such as ``Documentation'' or ``Library'').

Each bug report will be assigned to a developer who will determine
what needs to be done to correct the problem.  If you have a
SourceForge account and logged in to report the problem, you will
receive an update each time action is taken on the bug.


\begin{seealso}
  \seetitle[http://www-mice.cs.ucl.ac.uk/multimedia/software/documentation/ReportingBugs.html]{How
        to Report Bugs Effectively}{Article which goes into some
        detail about how to create a useful bug report.  This
        describes what kind of information is useful and why it is
        useful.}

  \seetitle[http://www.mozilla.org/quality/bug-writing-guidelines.html]{Bug
        Writing Guidelines}{Information about writing a good bug
        report.  Some of this is specific to the Mozilla project, but
        describes general good practices.}
\end{seealso}


\chapter{History and License}
\section{History of the software}

Python was created in the early 1990s by Guido van Rossum at Stichting
Mathematisch Centrum (CWI, see \url{http://www.cwi.nl/}) in the Netherlands
as a successor of a language called ABC.  Guido remains Python's
principal author, although it includes many contributions from others.

In 1995, Guido continued his work on Python at the Corporation for
National Research Initiatives (CNRI, see \url{http://www.cnri.reston.va.us/})
in Reston, Virginia where he released several versions of the
software.

In May 2000, Guido and the Python core development team moved to
BeOpen.com to form the BeOpen PythonLabs team.  In October of the same
year, the PythonLabs team moved to Digital Creations (now Zope
Corporation; see \url{http://www.zope.com/}).  In 2001, the Python
Software Foundation (PSF, see \url{http://www.python.org/psf/}) was
formed, a non-profit organization created specifically to own
Python-related Intellectual Property.  Zope Corporation is a
sponsoring member of the PSF.

All Python releases are Open Source (see
\url{http://www.opensource.org/} for the Open Source Definition).
Historically, most, but not all, Python releases have also been
GPL-compatible; the table below summarizes the various releases.

\begin{tablev}{c|c|c|c|c}{textrm}{Release}{Derived from}{Year}{Owner}{GPL compatible?}
  \linev{0.9.0 thru 1.2}{n/a}{1991-1995}{CWI}{yes}
  \linev{1.3 thru 1.5.2}{1.2}{1995-1999}{CNRI}{yes}
  \linev{1.6}{1.5.2}{2000}{CNRI}{no}
  \linev{2.0}{1.6}{2000}{BeOpen.com}{no}
  \linev{1.6.1}{1.6}{2001}{CNRI}{no}
  \linev{2.1}{2.0+1.6.1}{2001}{PSF}{no}
  \linev{2.0.1}{2.0+1.6.1}{2001}{PSF}{yes}
  \linev{2.1.1}{2.1+2.0.1}{2001}{PSF}{yes}
  \linev{2.2}{2.1.1}{2001}{PSF}{yes}
  \linev{2.1.2}{2.1.1}{2002}{PSF}{yes}
  \linev{2.1.3}{2.1.2}{2002}{PSF}{yes}
  \linev{2.2.1}{2.2}{2002}{PSF}{yes}
\end{tablev}

\note{GPL-compatible doesn't mean that we're distributing
Python under the GPL.  All Python licenses, unlike the GPL, let you
distribute a modified version without making your changes open source.
The GPL-compatible licenses make it possible to combine Python with
other software that is released under the GPL; the others don't.}

Thanks to the many outside volunteers who have worked under Guido's
direction to make these releases possible.


\section{Terms and conditions for accessing or otherwise using Python}

\centerline{\strong{PSF LICENSE AGREEMENT FOR PYTHON 2.2}}

\begin{enumerate}
\item
This LICENSE AGREEMENT is between the Python Software Foundation
(``PSF''), and the Individual or Organization (``Licensee'') accessing
and otherwise using Python \version{} software in source or binary
form and its associated documentation.

\item
Subject to the terms and conditions of this License Agreement, PSF
hereby grants Licensee a nonexclusive, royalty-free, world-wide
license to reproduce, analyze, test, perform and/or display publicly,
prepare derivative works, distribute, and otherwise use Python
\version{} alone or in any derivative version, provided, however, that
PSF's License Agreement and PSF's notice of copyright, i.e.,
``Copyright \copyright{} 2001, 2002 Python Software Foundation; All
Rights Reserved'' are retained in Python \version{} alone or in any
derivative version prepared by Licensee.

\item
In the event Licensee prepares a derivative work that is based on
or incorporates Python \version{} or any part thereof, and wants to
make the derivative work available to others as provided herein, then
Licensee hereby agrees to include in any such work a brief summary of
the changes made to Python \version.

\item
PSF is making Python \version{} available to Licensee on an ``AS IS''
basis.  PSF MAKES NO REPRESENTATIONS OR WARRANTIES, EXPRESS OR
IMPLIED.  BY WAY OF EXAMPLE, BUT NOT LIMITATION, PSF MAKES NO AND
DISCLAIMS ANY REPRESENTATION OR WARRANTY OF MERCHANTABILITY OR FITNESS
FOR ANY PARTICULAR PURPOSE OR THAT THE USE OF PYTHON \version{} WILL
NOT INFRINGE ANY THIRD PARTY RIGHTS.

\item
PSF SHALL NOT BE LIABLE TO LICENSEE OR ANY OTHER USERS OF PYTHON
\version{} FOR ANY INCIDENTAL, SPECIAL, OR CONSEQUENTIAL DAMAGES OR
LOSS AS A RESULT OF MODIFYING, DISTRIBUTING, OR OTHERWISE USING PYTHON
\version, OR ANY DERIVATIVE THEREOF, EVEN IF ADVISED OF THE
POSSIBILITY THEREOF.

\item
This License Agreement will automatically terminate upon a material
breach of its terms and conditions.

\item
Nothing in this License Agreement shall be deemed to create any
relationship of agency, partnership, or joint venture between PSF and
Licensee.  This License Agreement does not grant permission to use PSF
trademarks or trade name in a trademark sense to endorse or promote
products or services of Licensee, or any third party.

\item
By copying, installing or otherwise using Python \version, Licensee
agrees to be bound by the terms and conditions of this License
Agreement.
\end{enumerate}


\centerline{\strong{BEOPEN.COM LICENSE AGREEMENT FOR PYTHON 2.0}}

\centerline{\strong{BEOPEN PYTHON OPEN SOURCE LICENSE AGREEMENT VERSION 1}}

\begin{enumerate}
\item
This LICENSE AGREEMENT is between BeOpen.com (``BeOpen''), having an
office at 160 Saratoga Avenue, Santa Clara, CA 95051, and the
Individual or Organization (``Licensee'') accessing and otherwise
using this software in source or binary form and its associated
documentation (``the Software'').

\item
Subject to the terms and conditions of this BeOpen Python License
Agreement, BeOpen hereby grants Licensee a non-exclusive,
royalty-free, world-wide license to reproduce, analyze, test, perform
and/or display publicly, prepare derivative works, distribute, and
otherwise use the Software alone or in any derivative version,
provided, however, that the BeOpen Python License is retained in the
Software, alone or in any derivative version prepared by Licensee.

\item
BeOpen is making the Software available to Licensee on an ``AS IS''
basis.  BEOPEN MAKES NO REPRESENTATIONS OR WARRANTIES, EXPRESS OR
IMPLIED.  BY WAY OF EXAMPLE, BUT NOT LIMITATION, BEOPEN MAKES NO AND
DISCLAIMS ANY REPRESENTATION OR WARRANTY OF MERCHANTABILITY OR FITNESS
FOR ANY PARTICULAR PURPOSE OR THAT THE USE OF THE SOFTWARE WILL NOT
INFRINGE ANY THIRD PARTY RIGHTS.

\item
BEOPEN SHALL NOT BE LIABLE TO LICENSEE OR ANY OTHER USERS OF THE
SOFTWARE FOR ANY INCIDENTAL, SPECIAL, OR CONSEQUENTIAL DAMAGES OR LOSS
AS A RESULT OF USING, MODIFYING OR DISTRIBUTING THE SOFTWARE, OR ANY
DERIVATIVE THEREOF, EVEN IF ADVISED OF THE POSSIBILITY THEREOF.

\item
This License Agreement will automatically terminate upon a material
breach of its terms and conditions.

\item
This License Agreement shall be governed by and interpreted in all
respects by the law of the State of California, excluding conflict of
law provisions.  Nothing in this License Agreement shall be deemed to
create any relationship of agency, partnership, or joint venture
between BeOpen and Licensee.  This License Agreement does not grant
permission to use BeOpen trademarks or trade names in a trademark
sense to endorse or promote products or services of Licensee, or any
third party.  As an exception, the ``BeOpen Python'' logos available
at http://www.pythonlabs.com/logos.html may be used according to the
permissions granted on that web page.

\item
By copying, installing or otherwise using the software, Licensee
agrees to be bound by the terms and conditions of this License
Agreement.
\end{enumerate}


\centerline{\strong{CNRI LICENSE AGREEMENT FOR PYTHON 1.6.1}}

\begin{enumerate}
\item
This LICENSE AGREEMENT is between the Corporation for National
Research Initiatives, having an office at 1895 Preston White Drive,
Reston, VA 20191 (``CNRI''), and the Individual or Organization
(``Licensee'') accessing and otherwise using Python 1.6.1 software in
source or binary form and its associated documentation.

\item
Subject to the terms and conditions of this License Agreement, CNRI
hereby grants Licensee a nonexclusive, royalty-free, world-wide
license to reproduce, analyze, test, perform and/or display publicly,
prepare derivative works, distribute, and otherwise use Python 1.6.1
alone or in any derivative version, provided, however, that CNRI's
License Agreement and CNRI's notice of copyright, i.e., ``Copyright
\copyright{} 1995-2001 Corporation for National Research Initiatives;
All Rights Reserved'' are retained in Python 1.6.1 alone or in any
derivative version prepared by Licensee.  Alternately, in lieu of
CNRI's License Agreement, Licensee may substitute the following text
(omitting the quotes): ``Python 1.6.1 is made available subject to the
terms and conditions in CNRI's License Agreement.  This Agreement
together with Python 1.6.1 may be located on the Internet using the
following unique, persistent identifier (known as a handle):
1895.22/1013.  This Agreement may also be obtained from a proxy server
on the Internet using the following URL:
\url{http://hdl.handle.net/1895.22/1013}.''

\item
In the event Licensee prepares a derivative work that is based on
or incorporates Python 1.6.1 or any part thereof, and wants to make
the derivative work available to others as provided herein, then
Licensee hereby agrees to include in any such work a brief summary of
the changes made to Python 1.6.1.

\item
CNRI is making Python 1.6.1 available to Licensee on an ``AS IS''
basis.  CNRI MAKES NO REPRESENTATIONS OR WARRANTIES, EXPRESS OR
IMPLIED.  BY WAY OF EXAMPLE, BUT NOT LIMITATION, CNRI MAKES NO AND
DISCLAIMS ANY REPRESENTATION OR WARRANTY OF MERCHANTABILITY OR FITNESS
FOR ANY PARTICULAR PURPOSE OR THAT THE USE OF PYTHON 1.6.1 WILL NOT
INFRINGE ANY THIRD PARTY RIGHTS.

\item
CNRI SHALL NOT BE LIABLE TO LICENSEE OR ANY OTHER USERS OF PYTHON
1.6.1 FOR ANY INCIDENTAL, SPECIAL, OR CONSEQUENTIAL DAMAGES OR LOSS AS
A RESULT OF MODIFYING, DISTRIBUTING, OR OTHERWISE USING PYTHON 1.6.1,
OR ANY DERIVATIVE THEREOF, EVEN IF ADVISED OF THE POSSIBILITY THEREOF.

\item
This License Agreement will automatically terminate upon a material
breach of its terms and conditions.

\item
This License Agreement shall be governed by the federal
intellectual property law of the United States, including without
limitation the federal copyright law, and, to the extent such
U.S. federal law does not apply, by the law of the Commonwealth of
Virginia, excluding Virginia's conflict of law provisions.
Notwithstanding the foregoing, with regard to derivative works based
on Python 1.6.1 that incorporate non-separable material that was
previously distributed under the GNU General Public License (GPL), the
law of the Commonwealth of Virginia shall govern this License
Agreement only as to issues arising under or with respect to
Paragraphs 4, 5, and 7 of this License Agreement.  Nothing in this
License Agreement shall be deemed to create any relationship of
agency, partnership, or joint venture between CNRI and Licensee.  This
License Agreement does not grant permission to use CNRI trademarks or
trade name in a trademark sense to endorse or promote products or
services of Licensee, or any third party.

\item
By clicking on the ``ACCEPT'' button where indicated, or by copying,
installing or otherwise using Python 1.6.1, Licensee agrees to be
bound by the terms and conditions of this License Agreement.
\end{enumerate}

\centerline{ACCEPT}



\centerline{\strong{CWI LICENSE AGREEMENT FOR PYTHON 0.9.0 THROUGH 1.2}}

Copyright \copyright{} 1991 - 1995, Stichting Mathematisch Centrum
Amsterdam, The Netherlands.  All rights reserved.

Permission to use, copy, modify, and distribute this software and its
documentation for any purpose and without fee is hereby granted,
provided that the above copyright notice appear in all copies and that
both that copyright notice and this permission notice appear in
supporting documentation, and that the name of Stichting Mathematisch
Centrum or CWI not be used in advertising or publicity pertaining to
distribution of the software without specific, written prior
permission.

STICHTING MATHEMATISCH CENTRUM DISCLAIMS ALL WARRANTIES WITH REGARD TO
THIS SOFTWARE, INCLUDING ALL IMPLIED WARRANTIES OF MERCHANTABILITY AND
FITNESS, IN NO EVENT SHALL STICHTING MATHEMATISCH CENTRUM BE LIABLE
FOR ANY SPECIAL, INDIRECT OR CONSEQUENTIAL DAMAGES OR ANY DAMAGES
WHATSOEVER RESULTING FROM LOSS OF USE, DATA OR PROFITS, WHETHER IN AN
ACTION OF CONTRACT, NEGLIGENCE OR OTHER TORTIOUS ACTION, ARISING OUT
OF OR IN CONNECTION WITH THE USE OR PERFORMANCE OF THIS SOFTWARE.


\documentclass{manual}

\title{Python/C API Reference Manual}

\author{Guido van Rossum\\
	Fred L. Drake, Jr., editor}
\authoraddress{
	\strong{Python Software Foundation}\\
	Email: \email{docs@python.org}
}

\date{20 June, 2006}			% XXX update before final release!
\input{patchlevel}		% include Python version information


\makeindex			% tell \index to actually write the .idx file


\begin{document}

\maketitle

\ifhtml
\chapter*{Front Matter\label{front}}
\fi

\leftline{Copyright \copyright{} 2000, BeOpen.com.}
\leftline{Copyright \copyright{} 1995-2000, Corporation for National Research Initiatives.}
\leftline{Copyright \copyright{} 1990-1995, Stichting Mathematisch Centrum.}
\leftline{All rights reserved.}

Redistribution and use in source and binary forms, with or without
modification, are permitted provided that the following conditions are
met:

\begin{itemize}
\item
Redistributions of source code must retain the above copyright
notice, this list of conditions and the following disclaimer.

\item
Redistributions in binary form must reproduce the above copyright
notice, this list of conditions and the following disclaimer in the
documentation and/or other materials provided with the distribution.

\item
Neither names of the copyright holders nor the names of their
contributors may be used to endorse or promote products derived from
this software without specific prior written permission.
\end{itemize}

THIS SOFTWARE IS PROVIDED BY THE COPYRIGHT HOLDERS AND CONTRIBUTORS
``AS IS'' AND ANY EXPRESS OR IMPLIED WARRANTIES, INCLUDING, BUT NOT
LIMITED TO, THE IMPLIED WARRANTIES OF MERCHANTABILITY AND FITNESS FOR
A PARTICULAR PURPOSE ARE DISCLAIMED.  IN NO EVENT SHALL THE COPYRIGHT
HOLDERS OR CONTRIBUTORS BE LIABLE FOR ANY DIRECT, INDIRECT,
INCIDENTAL, SPECIAL, EXEMPLARY, OR CONSEQUENTIAL DAMAGES (INCLUDING,
BUT NOT LIMITED TO, PROCUREMENT OF SUBSTITUTE GOODS OR SERVICES; LOSS
OF USE, DATA, OR PROFITS; OR BUSINESS INTERRUPTION) HOWEVER CAUSED AND
ON ANY THEORY OF LIABILITY, WHETHER IN CONTRACT, STRICT LIABILITY, OR
TORT (INCLUDING NEGLIGENCE OR OTHERWISE) ARISING IN ANY WAY OUT OF THE
USE OF THIS SOFTWARE, EVEN IF ADVISED OF THE POSSIBILITY OF SUCH
DAMAGE.


\begin{abstract}

\noindent
This manual documents the API used by C and \Cpp{} programmers who
want to write extension modules or embed Python.  It is a companion to
\citetitle[../ext/ext.html]{Extending and Embedding the Python
Interpreter}, which describes the general principles of extension
writing but does not document the API functions in detail.

\strong{Warning:} The current version of this document is incomplete.
I hope that it is nevertheless useful.  I will continue to work on it,
and release new versions from time to time, independent from Python
source code releases.

\end{abstract}

\tableofcontents

% XXX Consider moving all this back to ext.tex and giving api.tex
% XXX a *really* short intro only.

\chapter{Introduction \label{intro}}

The Application Programmer's Interface to Python gives C and
\Cpp{} programmers access to the Python interpreter at a variety of
levels.  The API is equally usable from \Cpp{}, but for brevity it is
generally referred to as the Python/C API.  There are two
fundamentally different reasons for using the Python/C API.  The first
reason is to write \emph{extension modules} for specific purposes;
these are C modules that extend the Python interpreter.  This is
probably the most common use.  The second reason is to use Python as a
component in a larger application; this technique is generally
referred to as \dfn{embedding} Python in an application.

Writing an extension module is a relatively well-understood process, 
where a ``cookbook'' approach works well.  There are several tools 
that automate the process to some extent.  While people have embedded 
Python in other applications since its early existence, the process of 
embedding Python is less straightforward than writing an extension.  

Many API functions are useful independent of whether you're embedding 
or extending Python; moreover, most applications that embed Python 
will need to provide a custom extension as well, so it's probably a 
good idea to become familiar with writing an extension before 
attempting to embed Python in a real application.


\section{Include Files \label{includes}}

All function, type and macro definitions needed to use the Python/C
API are included in your code by the following line:

\begin{verbatim}
#include "Python.h"
\end{verbatim}

This implies inclusion of the following standard headers:
\code{<stdio.h>}, \code{<string.h>}, \code{<errno.h>},
\code{<limits.h>}, and \code{<stdlib.h>} (if available).

All user visible names defined by Python.h (except those defined by
the included standard headers) have one of the prefixes \samp{Py} or
\samp{_Py}.  Names beginning with \samp{_Py} are for internal use by
the Python implementation and should not be used by extension writers.
Structure member names do not have a reserved prefix.

\strong{Important:} user code should never define names that begin
with \samp{Py} or \samp{_Py}.  This confuses the reader, and
jeopardizes the portability of the user code to future Python
versions, which may define additional names beginning with one of
these prefixes.

The header files are typically installed with Python.  On \UNIX, these 
are located in the directories
\file{\envvar{prefix}/include/python\var{version}/} and
\file{\envvar{exec_prefix}/include/python\var{version}/}, where
\envvar{prefix} and \envvar{exec_prefix} are defined by the
corresponding parameters to Python's \program{configure} script and
\var{version} is \code{sys.version[:3]}.  On Windows, the headers are
installed in \file{\envvar{prefix}/include}, where \envvar{prefix} is
the installation directory specified to the installer.

To include the headers, place both directories (if different) on your
compiler's search path for includes.  Do \emph{not} place the parent
directories on the search path and then use
\samp{\#include <python\shortversion/Python.h>}; this will break on
multi-platform builds since the platform independent headers under
\envvar{prefix} include the platform specific headers from
\envvar{exec_prefix}.


\section{Objects, Types and Reference Counts \label{objects}}

Most Python/C API functions have one or more arguments as well as a
return value of type \ctype{PyObject*}.  This type is a pointer
to an opaque data type representing an arbitrary Python
object.  Since all Python object types are treated the same way by the
Python language in most situations (e.g., assignments, scope rules,
and argument passing), it is only fitting that they should be
represented by a single C type.  Almost all Python objects live on the
heap: you never declare an automatic or static variable of type
\ctype{PyObject}, only pointer variables of type \ctype{PyObject*} can 
be declared.  The sole exception are the type objects\obindex{type};
since these must never be deallocated, they are typically static
\ctype{PyTypeObject} objects.

All Python objects (even Python integers) have a \dfn{type} and a
\dfn{reference count}.  An object's type determines what kind of object 
it is (e.g., an integer, a list, or a user-defined function; there are 
many more as explained in the \citetitle[../ref/ref.html]{Python
Reference Manual}).  For each of the well-known types there is a macro
to check whether an object is of that type; for instance,
\samp{PyList_Check(\var{a})} is true if (and only if) the object
pointed to by \var{a} is a Python list.


\subsection{Reference Counts \label{refcounts}}

The reference count is important because today's computers have a 
finite (and often severely limited) memory size; it counts how many 
different places there are that have a reference to an object.  Such a 
place could be another object, or a global (or static) C variable, or 
a local variable in some C function.  When an object's reference count 
becomes zero, the object is deallocated.  If it contains references to 
other objects, their reference count is decremented.  Those other 
objects may be deallocated in turn, if this decrement makes their 
reference count become zero, and so on.  (There's an obvious problem 
with objects that reference each other here; for now, the solution is 
``don't do that.'')

Reference counts are always manipulated explicitly.  The normal way is 
to use the macro \cfunction{Py_INCREF()}\ttindex{Py_INCREF()} to
increment an object's reference count by one, and
\cfunction{Py_DECREF()}\ttindex{Py_DECREF()} to decrement it by  
one.  The \cfunction{Py_DECREF()} macro is considerably more complex
than the incref one, since it must check whether the reference count
becomes zero and then cause the object's deallocator to be called.
The deallocator is a function pointer contained in the object's type
structure.  The type-specific deallocator takes care of decrementing
the reference counts for other objects contained in the object if this
is a compound object type, such as a list, as well as performing any
additional finalization that's needed.  There's no chance that the
reference count can overflow; at least as many bits are used to hold
the reference count as there are distinct memory locations in virtual
memory (assuming \code{sizeof(long) >= sizeof(char*)}).  Thus, the
reference count increment is a simple operation.

It is not necessary to increment an object's reference count for every 
local variable that contains a pointer to an object.  In theory, the 
object's reference count goes up by one when the variable is made to 
point to it and it goes down by one when the variable goes out of 
scope.  However, these two cancel each other out, so at the end the 
reference count hasn't changed.  The only real reason to use the 
reference count is to prevent the object from being deallocated as 
long as our variable is pointing to it.  If we know that there is at 
least one other reference to the object that lives at least as long as 
our variable, there is no need to increment the reference count 
temporarily.  An important situation where this arises is in objects 
that are passed as arguments to C functions in an extension module 
that are called from Python; the call mechanism guarantees to hold a 
reference to every argument for the duration of the call.

However, a common pitfall is to extract an object from a list and
hold on to it for a while without incrementing its reference count.
Some other operation might conceivably remove the object from the
list, decrementing its reference count and possible deallocating it.
The real danger is that innocent-looking operations may invoke
arbitrary Python code which could do this; there is a code path which
allows control to flow back to the user from a \cfunction{Py_DECREF()},
so almost any operation is potentially dangerous.

A safe approach is to always use the generic operations (functions 
whose name begins with \samp{PyObject_}, \samp{PyNumber_},
\samp{PySequence_} or \samp{PyMapping_}).  These operations always
increment the reference count of the object they return.  This leaves
the caller with the responsibility to call
\cfunction{Py_DECREF()} when they are done with the result; this soon
becomes second nature.


\subsubsection{Reference Count Details \label{refcountDetails}}

The reference count behavior of functions in the Python/C API is best 
explained in terms of \emph{ownership of references}.  Note that we 
talk of owning references, never of owning objects; objects are always 
shared!  When a function owns a reference, it has to dispose of it 
properly --- either by passing ownership on (usually to its caller) or 
by calling \cfunction{Py_DECREF()} or \cfunction{Py_XDECREF()}.  When
a function passes ownership of a reference on to its caller, the
caller is said to receive a \emph{new} reference.  When no ownership
is transferred, the caller is said to \emph{borrow} the reference.
Nothing needs to be done for a borrowed reference.

Conversely, when a calling function passes it a reference to an 
object, there are two possibilities: the function \emph{steals} a 
reference to the object, or it does not.  Few functions steal 
references; the two notable exceptions are
\cfunction{PyList_SetItem()}\ttindex{PyList_SetItem()} and
\cfunction{PyTuple_SetItem()}\ttindex{PyTuple_SetItem()}, which 
steal a reference to the item (but not to the tuple or list into which
the item is put!).  These functions were designed to steal a reference
because of a common idiom for populating a tuple or list with newly
created objects; for example, the code to create the tuple \code{(1,
2, "three")} could look like this (forgetting about error handling for
the moment; a better way to code this is shown below):

\begin{verbatim}
PyObject *t;

t = PyTuple_New(3);
PyTuple_SetItem(t, 0, PyInt_FromLong(1L));
PyTuple_SetItem(t, 1, PyInt_FromLong(2L));
PyTuple_SetItem(t, 2, PyString_FromString("three"));
\end{verbatim}

Incidentally, \cfunction{PyTuple_SetItem()} is the \emph{only} way to
set tuple items; \cfunction{PySequence_SetItem()} and
\cfunction{PyObject_SetItem()} refuse to do this since tuples are an
immutable data type.  You should only use
\cfunction{PyTuple_SetItem()} for tuples that you are creating
yourself.

Equivalent code for populating a list can be written using 
\cfunction{PyList_New()} and \cfunction{PyList_SetItem()}.  Such code
can also use \cfunction{PySequence_SetItem()}; this illustrates the
difference between the two (the extra \cfunction{Py_DECREF()} calls):

\begin{verbatim}
PyObject *l, *x;

l = PyList_New(3);
x = PyInt_FromLong(1L);
PySequence_SetItem(l, 0, x); Py_DECREF(x);
x = PyInt_FromLong(2L);
PySequence_SetItem(l, 1, x); Py_DECREF(x);
x = PyString_FromString("three");
PySequence_SetItem(l, 2, x); Py_DECREF(x);
\end{verbatim}

You might find it strange that the ``recommended'' approach takes more
code.  However, in practice, you will rarely use these ways of
creating and populating a tuple or list.  There's a generic function,
\cfunction{Py_BuildValue()}, that can create most common objects from
C values, directed by a \dfn{format string}.  For example, the
above two blocks of code could be replaced by the following (which
also takes care of the error checking):

\begin{verbatim}
PyObject *t, *l;

t = Py_BuildValue("(iis)", 1, 2, "three");
l = Py_BuildValue("[iis]", 1, 2, "three");
\end{verbatim}

It is much more common to use \cfunction{PyObject_SetItem()} and
friends with items whose references you are only borrowing, like
arguments that were passed in to the function you are writing.  In
that case, their behaviour regarding reference counts is much saner,
since you don't have to increment a reference count so you can give a
reference away (``have it be stolen'').  For example, this function
sets all items of a list (actually, any mutable sequence) to a given
item:

\begin{verbatim}
int set_all(PyObject *target, PyObject *item)
{
    int i, n;

    n = PyObject_Length(target);
    if (n < 0)
        return -1;
    for (i = 0; i < n; i++) {
        if (PyObject_SetItem(target, i, item) < 0)
            return -1;
    }
    return 0;
}
\end{verbatim}
\ttindex{set_all()}

The situation is slightly different for function return values.  
While passing a reference to most functions does not change your 
ownership responsibilities for that reference, many functions that 
return a referece to an object give you ownership of the reference.
The reason is simple: in many cases, the returned object is created 
on the fly, and the reference you get is the only reference to the 
object.  Therefore, the generic functions that return object 
references, like \cfunction{PyObject_GetItem()} and 
\cfunction{PySequence_GetItem()}, always return a new reference (i.e.,
the  caller becomes the owner of the reference).

It is important to realize that whether you own a reference returned 
by a function depends on which function you call only --- \emph{the
plumage} (i.e., the type of the type of the object passed as an
argument to the function) \emph{doesn't enter into it!}  Thus, if you 
extract an item from a list using \cfunction{PyList_GetItem()}, you
don't own the reference --- but if you obtain the same item from the
same list using \cfunction{PySequence_GetItem()} (which happens to
take exactly the same arguments), you do own a reference to the
returned object.

Here is an example of how you could write a function that computes the
sum of the items in a list of integers; once using 
\cfunction{PyList_GetItem()}\ttindex{PyList_GetItem()}, and once using
\cfunction{PySequence_GetItem()}\ttindex{PySequence_GetItem()}.

\begin{verbatim}
long sum_list(PyObject *list)
{
    int i, n;
    long total = 0;
    PyObject *item;

    n = PyList_Size(list);
    if (n < 0)
        return -1; /* Not a list */
    for (i = 0; i < n; i++) {
        item = PyList_GetItem(list, i); /* Can't fail */
        if (!PyInt_Check(item)) continue; /* Skip non-integers */
        total += PyInt_AsLong(item);
    }
    return total;
}
\end{verbatim}
\ttindex{sum_list()}

\begin{verbatim}
long sum_sequence(PyObject *sequence)
{
    int i, n;
    long total = 0;
    PyObject *item;
    n = PySequence_Length(sequence);
    if (n < 0)
        return -1; /* Has no length */
    for (i = 0; i < n; i++) {
        item = PySequence_GetItem(sequence, i);
        if (item == NULL)
            return -1; /* Not a sequence, or other failure */
        if (PyInt_Check(item))
            total += PyInt_AsLong(item);
        Py_DECREF(item); /* Discard reference ownership */
    }
    return total;
}
\end{verbatim}
\ttindex{sum_sequence()}


\subsection{Types \label{types}}

There are few other data types that play a significant role in 
the Python/C API; most are simple C types such as \ctype{int}, 
\ctype{long}, \ctype{double} and \ctype{char*}.  A few structure types 
are used to describe static tables used to list the functions exported 
by a module or the data attributes of a new object type, and another
is used to describe the value of a complex number.  These will 
be discussed together with the functions that use them.


\section{Exceptions \label{exceptions}}

The Python programmer only needs to deal with exceptions if specific 
error handling is required; unhandled exceptions are automatically 
propagated to the caller, then to the caller's caller, and so on, until
they reach the top-level interpreter, where they are reported to the 
user accompanied by a stack traceback.

For C programmers, however, error checking always has to be explicit.  
All functions in the Python/C API can raise exceptions, unless an 
explicit claim is made otherwise in a function's documentation.  In 
general, when a function encounters an error, it sets an exception, 
discards any object references that it owns, and returns an 
error indicator --- usually \NULL{} or \code{-1}.  A few functions 
return a Boolean true/false result, with false indicating an error.
Very few functions return no explicit error indicator or have an 
ambiguous return value, and require explicit testing for errors with 
\cfunction{PyErr_Occurred()}\ttindex{PyErr_Occurred()}.

Exception state is maintained in per-thread storage (this is 
equivalent to using global storage in an unthreaded application).  A 
thread can be in one of two states: an exception has occurred, or not.
The function \cfunction{PyErr_Occurred()} can be used to check for
this: it returns a borrowed reference to the exception type object
when an exception has occurred, and \NULL{} otherwise.  There are a
number of functions to set the exception state:
\cfunction{PyErr_SetString()}\ttindex{PyErr_SetString()} is the most
common (though not the most general) function to set the exception
state, and \cfunction{PyErr_Clear()}\ttindex{PyErr_Clear()} clears the
exception state.

The full exception state consists of three objects (all of which can 
be \NULL{}): the exception type, the corresponding exception 
value, and the traceback.  These have the same meanings as the Python
\withsubitem{(in module sys)}{
  \ttindex{exc_type}\ttindex{exc_value}\ttindex{exc_traceback}}
objects \code{sys.exc_type}, \code{sys.exc_value}, and
\code{sys.exc_traceback}; however, they are not the same: the Python
objects represent the last exception being handled by a Python 
\keyword{try} \ldots\ \keyword{except} statement, while the C level
exception state only exists while an exception is being passed on
between C functions until it reaches the Python bytecode interpreter's 
main loop, which takes care of transferring it to \code{sys.exc_type}
and friends.

Note that starting with Python 1.5, the preferred, thread-safe way to 
access the exception state from Python code is to call the function
\withsubitem{(in module sys)}{\ttindex{exc_info()}}
\function{sys.exc_info()}, which returns the per-thread exception state 
for Python code.  Also, the semantics of both ways to access the 
exception state have changed so that a function which catches an 
exception will save and restore its thread's exception state so as to 
preserve the exception state of its caller.  This prevents common bugs 
in exception handling code caused by an innocent-looking function 
overwriting the exception being handled; it also reduces the often 
unwanted lifetime extension for objects that are referenced by the 
stack frames in the traceback.

As a general principle, a function that calls another function to 
perform some task should check whether the called function raised an 
exception, and if so, pass the exception state on to its caller.  It 
should discard any object references that it owns, and return an 
error indicator, but it should \emph{not} set another exception ---
that would overwrite the exception that was just raised, and lose
important information about the exact cause of the error.

A simple example of detecting exceptions and passing them on is shown
in the \cfunction{sum_sequence()}\ttindex{sum_sequence()} example
above.  It so happens that that example doesn't need to clean up any
owned references when it detects an error.  The following example
function shows some error cleanup.  First, to remind you why you like
Python, we show the equivalent Python code:

\begin{verbatim}
def incr_item(dict, key):
    try:
        item = dict[key]
    except KeyError:
        item = 0
    dict[key] = item + 1
\end{verbatim}
\ttindex{incr_item()}

Here is the corresponding C code, in all its glory:

\begin{verbatim}
int incr_item(PyObject *dict, PyObject *key)
{
    /* Objects all initialized to NULL for Py_XDECREF */
    PyObject *item = NULL, *const_one = NULL, *incremented_item = NULL;
    int rv = -1; /* Return value initialized to -1 (failure) */

    item = PyObject_GetItem(dict, key);
    if (item == NULL) {
        /* Handle KeyError only: */
        if (!PyErr_ExceptionMatches(PyExc_KeyError))
            goto error;

        /* Clear the error and use zero: */
        PyErr_Clear();
        item = PyInt_FromLong(0L);
        if (item == NULL)
            goto error;
    }
    const_one = PyInt_FromLong(1L);
    if (const_one == NULL)
        goto error;

    incremented_item = PyNumber_Add(item, const_one);
    if (incremented_item == NULL)
        goto error;

    if (PyObject_SetItem(dict, key, incremented_item) < 0)
        goto error;
    rv = 0; /* Success */
    /* Continue with cleanup code */

 error:
    /* Cleanup code, shared by success and failure path */

    /* Use Py_XDECREF() to ignore NULL references */
    Py_XDECREF(item);
    Py_XDECREF(const_one);
    Py_XDECREF(incremented_item);

    return rv; /* -1 for error, 0 for success */
}
\end{verbatim}
\ttindex{incr_item()}

This example represents an endorsed use of the \keyword{goto} statement 
in C!  It illustrates the use of
\cfunction{PyErr_ExceptionMatches()}\ttindex{PyErr_ExceptionMatches()} and
\cfunction{PyErr_Clear()}\ttindex{PyErr_Clear()} to
handle specific exceptions, and the use of
\cfunction{Py_XDECREF()}\ttindex{Py_XDECREF()} to
dispose of owned references that may be \NULL{} (note the
\character{X} in the name; \cfunction{Py_DECREF()} would crash when
confronted with a \NULL{} reference).  It is important that the
variables used to hold owned references are initialized to \NULL{} for
this to work; likewise, the proposed return value is initialized to
\code{-1} (failure) and only set to success after the final call made
is successful.


\section{Embedding Python \label{embedding}}

The one important task that only embedders (as opposed to extension
writers) of the Python interpreter have to worry about is the
initialization, and possibly the finalization, of the Python
interpreter.  Most functionality of the interpreter can only be used
after the interpreter has been initialized.

The basic initialization function is
\cfunction{Py_Initialize()}\ttindex{Py_Initialize()}.
This initializes the table of loaded modules, and creates the
fundamental modules \module{__builtin__}\refbimodindex{__builtin__},
\module{__main__}\refbimodindex{__main__} and 
\module{sys}\refbimodindex{sys}.  It also initializes the module
search path (\code{sys.path}).%
\indexiii{module}{search}{path}
\withsubitem{(in module sys)}{\ttindex{path}}

\cfunction{Py_Initialize()} does not set the ``script argument list'' 
(\code{sys.argv}).  If this variable is needed by Python code that 
will be executed later, it must be set explicitly with a call to 
\code{PySys_SetArgv(\var{argc},
\var{argv})}\ttindex{PySys_SetArgv()} subsequent to the call to
\cfunction{Py_Initialize()}.

On most systems (in particular, on \UNIX{} and Windows, although the
details are slightly different),
\cfunction{Py_Initialize()} calculates the module search path based
upon its best guess for the location of the standard Python
interpreter executable, assuming that the Python library is found in a
fixed location relative to the Python interpreter executable.  In
particular, it looks for a directory named
\file{lib/python\shortversion} relative to the parent directory where
the executable named \file{python} is found on the shell command
search path (the environment variable \envvar{PATH}).

For instance, if the Python executable is found in
\file{/usr/local/bin/python}, it will assume that the libraries are in
\file{/usr/local/lib/python\shortversion}.  (In fact, this particular path
is also the ``fallback'' location, used when no executable file named
\file{python} is found along \envvar{PATH}.)  The user can override
this behavior by setting the environment variable \envvar{PYTHONHOME},
or insert additional directories in front of the standard path by
setting \envvar{PYTHONPATH}.

The embedding application can steer the search by calling 
\code{Py_SetProgramName(\var{file})}\ttindex{Py_SetProgramName()} \emph{before} calling 
\cfunction{Py_Initialize()}.  Note that \envvar{PYTHONHOME} still
overrides this and \envvar{PYTHONPATH} is still inserted in front of
the standard path.  An application that requires total control has to
provide its own implementation of
\cfunction{Py_GetPath()}\ttindex{Py_GetPath()},
\cfunction{Py_GetPrefix()}\ttindex{Py_GetPrefix()},
\cfunction{Py_GetExecPrefix()}\ttindex{Py_GetExecPrefix()}, and
\cfunction{Py_GetProgramFullPath()}\ttindex{Py_GetProgramFullPath()} (all
defined in \file{Modules/getpath.c}).

Sometimes, it is desirable to ``uninitialize'' Python.  For instance, 
the application may want to start over (make another call to 
\cfunction{Py_Initialize()}) or the application is simply done with its 
use of Python and wants to free all memory allocated by Python.  This
can be accomplished by calling \cfunction{Py_Finalize()}.  The function
\cfunction{Py_IsInitialized()}\ttindex{Py_IsInitialized()} returns
true if Python is currently in the initialized state.  More
information about these functions is given in a later chapter.


\chapter{The Very High Level Layer \label{veryhigh}}

The functions in this chapter will let you execute Python source code
given in a file or a buffer, but they will not let you interact in a
more detailed way with the interpreter.

Several of these functions accept a start symbol from the grammar as a 
parameter.  The available start symbols are \constant{Py_eval_input},
\constant{Py_file_input}, and \constant{Py_single_input}.  These are
described following the functions which accept them as parameters.

Note also that several of these functions take \ctype{FILE*}
parameters.  On particular issue which needs to be handled carefully
is that the \ctype{FILE} structure for different C libraries can be
different and incompatible.  Under Windows (at least), it is possible
for dynamically linked extensions to actually use different libraries,
so care should be taken that \ctype{FILE*} parameters are only passed
to these functions if it is certain that they were created by the same
library that the Python runtime is using.

\begin{cfuncdesc}{int}{PyRun_AnyFile}{FILE *fp, char *filename}
  If \var{fp} refers to a file associated with an interactive device
  (console or terminal input or \UNIX{} pseudo-terminal), return the
  value of \cfunction{PyRun_InteractiveLoop()}, otherwise return the
  result of \cfunction{PyRun_SimpleFile()}.  If \var{filename} is
  \NULL{}, this function uses \code{"???"} as the filename.
\end{cfuncdesc}

\begin{cfuncdesc}{int}{PyRun_SimpleString}{char *command}
  Executes the Python source code from \var{command} in the
  \module{__main__} module.  If \module{__main__} does not already
  exist, it is created.  Returns \code{0} on success or \code{-1} if
  an exception was raised.  If there was an error, there is no way to
  get the exception information.
\end{cfuncdesc}

\begin{cfuncdesc}{int}{PyRun_SimpleFile}{FILE *fp, char *filename}
  Similar to \cfunction{PyRun_SimpleString()}, but the Python source
  code is read from \var{fp} instead of an in-memory string.
  \var{filename} should be the name of the file.
\end{cfuncdesc}

\begin{cfuncdesc}{int}{PyRun_InteractiveOne}{FILE *fp, char *filename}
  Read and execute a single statement from a file associated with an
  interactive device.  If \var{filename} is \NULL, \code{"???"} is
  used instead.  The user will be prompted using \code{sys.ps1} and
  \code{sys.ps2}.  Returns \code{0} when the input was executed
  successfully, \code{-1} if there was an exception, or an error code
  from the \file{errcode.h} include file distributed as part of Python
  in case of a parse error.  (Note that \file{errcode.h} is not
  included by \file{Python.h}, so must be included specifically if
  needed.)
\end{cfuncdesc}

\begin{cfuncdesc}{int}{PyRun_InteractiveLoop}{FILE *fp, char *filename}
  Read and execute statements from a file associated with an
  interactive device until \EOF{} is reached.  If \var{filename} is
  \NULL, \code{"???"} is used instead.  The user will be prompted
  using \code{sys.ps1} and \code{sys.ps2}.  Returns \code{0} at \EOF.
\end{cfuncdesc}

\begin{cfuncdesc}{struct _node*}{PyParser_SimpleParseString}{char *str,
                                                             int start}
  Parse Python source code from \var{str} using the start token
  \var{start}.  The result can be used to create a code object which
  can be evaluated efficiently.  This is useful if a code fragment
  must be evaluated many times.
\end{cfuncdesc}

\begin{cfuncdesc}{struct _node*}{PyParser_SimpleParseFile}{FILE *fp,
                                 char *filename, int start}
  Similar to \cfunction{PyParser_SimpleParseString()}, but the Python
  source code is read from \var{fp} instead of an in-memory string.
  \var{filename} should be the name of the file.
\end{cfuncdesc}

\begin{cfuncdesc}{PyObject*}{PyRun_String}{char *str, int start,
                                           PyObject *globals,
                                           PyObject *locals}
  Execute Python source code from \var{str} in the context specified
  by the dictionaries \var{globals} and \var{locals}.  The parameter
  \var{start} specifies the start token that should be used to parse
  the source code.

  Returns the result of executing the code as a Python object, or
  \NULL{} if an exception was raised.
\end{cfuncdesc}

\begin{cfuncdesc}{PyObject*}{PyRun_File}{FILE *fp, char *filename,
                                         int start, PyObject *globals,
                                         PyObject *locals}
  Similar to \cfunction{PyRun_String()}, but the Python source code is 
  read from \var{fp} instead of an in-memory string.
  \var{filename} should be the name of the file.
\end{cfuncdesc}

\begin{cfuncdesc}{PyObject*}{Py_CompileString}{char *str, char *filename,
                                               int start}
  Parse and compile the Python source code in \var{str}, returning the 
  resulting code object.  The start token is given by \var{start};
  this can be used to constrain the code which can be compiled and should
  be \constant{Py_eval_input}, \constant{Py_file_input}, or
  \constant{Py_single_input}.  The filename specified by
  \var{filename} is used to construct the code object and may appear
  in tracebacks or \exception{SyntaxError} exception messages.  This
  returns \NULL{} if the code cannot be parsed or compiled.
\end{cfuncdesc}

\begin{cvardesc}{int}{Py_eval_input}
  The start symbol from the Python grammar for isolated expressions;
  for use with \cfunction{Py_CompileString()}\ttindex{Py_CompileString()}.
\end{cvardesc}

\begin{cvardesc}{int}{Py_file_input}
  The start symbol from the Python grammar for sequences of statements
  as read from a file or other source; for use with
  \cfunction{Py_CompileString()}\ttindex{Py_CompileString()}.  This is
  the symbol to use when compiling arbitrarily long Python source code.
\end{cvardesc}

\begin{cvardesc}{int}{Py_single_input}
  The start symbol from the Python grammar for a single statement; for 
  use with \cfunction{Py_CompileString()}\ttindex{Py_CompileString()}.
  This is the symbol used for the interactive interpreter loop.
\end{cvardesc}


\chapter{Reference Counting \label{countingRefs}}

The macros in this section are used for managing reference counts
of Python objects.

\begin{cfuncdesc}{void}{Py_INCREF}{PyObject *o}
Increment the reference count for object \var{o}.  The object must
not be \NULL{}; if you aren't sure that it isn't \NULL{}, use
\cfunction{Py_XINCREF()}.
\end{cfuncdesc}

\begin{cfuncdesc}{void}{Py_XINCREF}{PyObject *o}
Increment the reference count for object \var{o}.  The object may be
\NULL{}, in which case the macro has no effect.
\end{cfuncdesc}

\begin{cfuncdesc}{void}{Py_DECREF}{PyObject *o}
Decrement the reference count for object \var{o}.  The object must
not be \NULL{}; if you aren't sure that it isn't \NULL{}, use
\cfunction{Py_XDECREF()}.  If the reference count reaches zero, the
object's type's deallocation function (which must not be \NULL{}) is
invoked.

\strong{Warning:} The deallocation function can cause arbitrary Python
code to be invoked (e.g. when a class instance with a
\method{__del__()} method is deallocated).  While exceptions in such
code are not propagated, the executed code has free access to all
Python global variables.  This means that any object that is reachable
from a global variable should be in a consistent state before
\cfunction{Py_DECREF()} is invoked.  For example, code to delete an
object from a list should copy a reference to the deleted object in a
temporary variable, update the list data structure, and then call
\cfunction{Py_DECREF()} for the temporary variable.
\end{cfuncdesc}

\begin{cfuncdesc}{void}{Py_XDECREF}{PyObject *o}
Decrement the reference count for object \var{o}.  The object may be
\NULL{}, in which case the macro has no effect; otherwise the effect
is the same as for \cfunction{Py_DECREF()}, and the same warning
applies.
\end{cfuncdesc}

The following functions or macros are only for use within the
interpreter core: \cfunction{_Py_Dealloc()},
\cfunction{_Py_ForgetReference()}, \cfunction{_Py_NewReference()}, as
well as the global variable \cdata{_Py_RefTotal}.


\chapter{Exception Handling \label{exceptionHandling}}

The functions described in this chapter will let you handle and raise Python
exceptions.  It is important to understand some of the basics of
Python exception handling.  It works somewhat like the
\UNIX{} \cdata{errno} variable: there is a global indicator (per
thread) of the last error that occurred.  Most functions don't clear
this on success, but will set it to indicate the cause of the error on
failure.  Most functions also return an error indicator, usually
\NULL{} if they are supposed to return a pointer, or \code{-1} if they
return an integer (exception: the \cfunction{PyArg_Parse*()} functions
return \code{1} for success and \code{0} for failure).  When a
function must fail because some function it called failed, it
generally doesn't set the error indicator; the function it called
already set it.

The error indicator consists of three Python objects corresponding to
\withsubitem{(in module sys)}{
  \ttindex{exc_type}\ttindex{exc_value}\ttindex{exc_traceback}}
the Python variables \code{sys.exc_type}, \code{sys.exc_value} and
\code{sys.exc_traceback}.  API functions exist to interact with the
error indicator in various ways.  There is a separate error indicator
for each thread.

% XXX Order of these should be more thoughtful.
% Either alphabetical or some kind of structure.

\begin{cfuncdesc}{void}{PyErr_Print}{}
Print a standard traceback to \code{sys.stderr} and clear the error
indicator.  Call this function only when the error indicator is set.
(Otherwise it will cause a fatal error!)
\end{cfuncdesc}

\begin{cfuncdesc}{PyObject*}{PyErr_Occurred}{}
Test whether the error indicator is set.  If set, return the exception
\emph{type} (the first argument to the last call to one of the
\cfunction{PyErr_Set*()} functions or to \cfunction{PyErr_Restore()}).  If
not set, return \NULL{}.  You do not own a reference to the return
value, so you do not need to \cfunction{Py_DECREF()} it.
\strong{Note:}  Do not compare the return value to a specific
exception; use \cfunction{PyErr_ExceptionMatches()} instead, shown
below.  (The comparison could easily fail since the exception may be
an instance instead of a class, in the case of a class exception, or
it may the a subclass of the expected exception.)
\end{cfuncdesc}

\begin{cfuncdesc}{int}{PyErr_ExceptionMatches}{PyObject *exc}
Equivalent to
\samp{PyErr_GivenExceptionMatches(PyErr_Occurred(), \var{exc})}.
This should only be called when an exception is actually set; a memory 
access violation will occur if no exception has been raised.
\end{cfuncdesc}

\begin{cfuncdesc}{int}{PyErr_GivenExceptionMatches}{PyObject *given, PyObject *exc}
Return true if the \var{given} exception matches the exception in
\var{exc}.  If \var{exc} is a class object, this also returns true
when \var{given} is an instance of a subclass.  If \var{exc} is a tuple, all
exceptions in the tuple (and recursively in subtuples) are searched
for a match.  If \var{given} is \NULL, a memory access violation will
occur.
\end{cfuncdesc}

\begin{cfuncdesc}{void}{PyErr_NormalizeException}{PyObject**exc, PyObject**val, PyObject**tb}
Under certain circumstances, the values returned by
\cfunction{PyErr_Fetch()} below can be ``unnormalized'', meaning that
\code{*\var{exc}} is a class object but \code{*\var{val}} is not an
instance of the  same class.  This function can be used to instantiate
the class in that case.  If the values are already normalized, nothing
happens.  The delayed normalization is implemented to improve
performance.
\end{cfuncdesc}

\begin{cfuncdesc}{void}{PyErr_Clear}{}
Clear the error indicator.  If the error indicator is not set, there
is no effect.
\end{cfuncdesc}

\begin{cfuncdesc}{void}{PyErr_Fetch}{PyObject **ptype, PyObject **pvalue,
                                     PyObject **ptraceback}
Retrieve the error indicator into three variables whose addresses are
passed.  If the error indicator is not set, set all three variables to
\NULL{}.  If it is set, it will be cleared and you own a reference to
each object retrieved.  The value and traceback object may be
\NULL{} even when the type object is not.  \strong{Note:}  This
function is normally only used by code that needs to handle exceptions
or by code that needs to save and restore the error indicator
temporarily.
\end{cfuncdesc}

\begin{cfuncdesc}{void}{PyErr_Restore}{PyObject *type, PyObject *value,
                                       PyObject *traceback}
Set  the error indicator from the three objects.  If the error
indicator is already set, it is cleared first.  If the objects are
\NULL{}, the error indicator is cleared.  Do not pass a \NULL{} type
and non-\NULL{} value or traceback.  The exception type should be a
string or class; if it is a class, the value should be an instance of
that class.  Do not pass an invalid exception type or value.
(Violating these rules will cause subtle problems later.)  This call
takes away a reference to each object, i.e.\ you must own a reference
to each object before the call and after the call you no longer own
these references.  (If you don't understand this, don't use this
function.  I warned you.)  \strong{Note:}  This function is normally
only used by code that needs to save and restore the error indicator
temporarily.
\end{cfuncdesc}

\begin{cfuncdesc}{void}{PyErr_SetString}{PyObject *type, char *message}
This is the most common way to set the error indicator.  The first
argument specifies the exception type; it is normally one of the
standard exceptions, e.g. \cdata{PyExc_RuntimeError}.  You need not
increment its reference count.  The second argument is an error
message; it is converted to a string object.
\end{cfuncdesc}

\begin{cfuncdesc}{void}{PyErr_SetObject}{PyObject *type, PyObject *value}
This function is similar to \cfunction{PyErr_SetString()} but lets you
specify an arbitrary Python object for the ``value'' of the exception.
You need not increment its reference count.
\end{cfuncdesc}

\begin{cfuncdesc}{PyObject*}{PyErr_Format}{PyObject *exception,
                                           const char *format, \moreargs}
This function sets the error indicator.  \var{exception} should be a
Python exception (string or class, not an instance).
\var{format} should be a string, containing format codes, similar to 
\cfunction{printf}. The \code{width.precision} before a format code
is parsed, but the width part is ignored.

\begin{tableii}{c|l}{character}{Character}{Meaning}
  \lineii{c}{Character, as an \ctype{int} parameter}
  \lineii{d}{Number in decimal, as an \ctype{int} parameter}
  \lineii{x}{Number in hexadecimal, as an \ctype{int} parameter}
  \lineii{x}{A string, as a \ctype{char *} parameter}
\end{tableii}

An unrecognized format character causes all the rest of
the format string to be copied as-is to the result string,
and any extra arguments discarded.

A new reference is returned, which is owned by the caller.
\end{cfuncdesc}

\begin{cfuncdesc}{void}{PyErr_SetNone}{PyObject *type}
This is a shorthand for \samp{PyErr_SetObject(\var{type}, Py_None)}.
\end{cfuncdesc}

\begin{cfuncdesc}{int}{PyErr_BadArgument}{}
This is a shorthand for \samp{PyErr_SetString(PyExc_TypeError,
\var{message})}, where \var{message} indicates that a built-in operation
was invoked with an illegal argument.  It is mostly for internal use.
\end{cfuncdesc}

\begin{cfuncdesc}{PyObject*}{PyErr_NoMemory}{}
This is a shorthand for \samp{PyErr_SetNone(PyExc_MemoryError)}; it
returns \NULL{} so an object allocation function can write
\samp{return PyErr_NoMemory();} when it runs out of memory.
\end{cfuncdesc}

\begin{cfuncdesc}{PyObject*}{PyErr_SetFromErrno}{PyObject *type}
This is a convenience function to raise an exception when a C library
function has returned an error and set the C variable \cdata{errno}.
It constructs a tuple object whose first item is the integer
\cdata{errno} value and whose second item is the corresponding error
message (gotten from \cfunction{strerror()}\ttindex{strerror()}), and
then calls
\samp{PyErr_SetObject(\var{type}, \var{object})}.  On \UNIX{}, when
the \cdata{errno} value is \constant{EINTR}, indicating an interrupted
system call, this calls \cfunction{PyErr_CheckSignals()}, and if that set
the error indicator, leaves it set to that.  The function always
returns \NULL{}, so a wrapper function around a system call can write 
\samp{return PyErr_SetFromErrno();} when  the system call returns an
error.
\end{cfuncdesc}

\begin{cfuncdesc}{void}{PyErr_BadInternalCall}{}
This is a shorthand for \samp{PyErr_SetString(PyExc_TypeError,
\var{message})}, where \var{message} indicates that an internal
operation (e.g. a Python/C API function) was invoked with an illegal
argument.  It is mostly for internal use.
\end{cfuncdesc}

\begin{cfuncdesc}{int}{PyErr_Warn}{PyObject *category, char *message}
Issue a warning message.  The \var{category} argument is a warning
category (see below) or \NULL; the \var{message} argument is a message
string.

This function normally prints a warning message to \var{sys.stderr};
however, it is also possible that the user has specified that warnings
are to be turned into errors, and in that case this will raise an
exception.  It is also possible that the function raises an exception
because of a problem with the warning machinery (the implementation
imports the \module{warnings} module to do the heavy lifting).  The
return value is \code{0} if no exception is raised, or \code{-1} if
an exception is raised.  (It is not possible to determine whether a
warning message is actually printed, nor what the reason is for the
exception; this is intentional.)  If an exception is raised, the
caller should do its normal exception handling
(e.g. \cfunction{Py_DECREF()} owned references and return an error
value).

Warning categories must be subclasses of \cdata{Warning}; the default
warning category is \cdata{RuntimeWarning}.  The standard Python
warning categories are available as global variables whose names are
\samp{PyExc_} followed by the Python exception name.  These have the
type \ctype{PyObject*}; they are all class objects.  Their names are
\cdata{PyExc_Warning}, \cdata{PyExc_UserWarning},
\cdata{PyExc_DeprecationWarning}, \cdata{PyExc_SyntaxWarning}, and
\cdata{PyExc_RuntimeWarning}.  \cdata{PyExc_Warning} is a subclass of
\cdata{PyExc_Exception}; the other warning categories are subclasses
of \cdata{PyExc_Warning}.

For information about warning control, see the documentation for the
\module{warnings} module and the \programopt{-W} option in the command
line documentation.  There is no C API for warning control.
\end{cfuncdesc}

\begin{cfuncdesc}{int}{PyErr_WarnExplicit}{PyObject *category, char *message,
char *filename, int lineno, char *module, PyObject *registry}
Issue a warning message with explicit control over all warning
attributes.  This is a straightforward wrapper around the Python
function \function{warnings.warn_explicit()}, see there for more
information.  The \var{module} and \var{registry} arguments may be
set to \code{NULL} to get the default effect described there.
\end{cfuncdesc}

\begin{cfuncdesc}{int}{PyErr_CheckSignals}{}
This function interacts with Python's signal handling.  It checks
whether a signal has been sent to the processes and if so, invokes the
corresponding signal handler.  If the
\module{signal}\refbimodindex{signal} module is supported, this can
invoke a signal handler written in Python.  In all cases, the default
effect for \constant{SIGINT}\ttindex{SIGINT} is to raise the
\withsubitem{(built-in exception)}{\ttindex{KeyboardInterrupt}}
\exception{KeyboardInterrupt} exception.  If an exception is raised the 
error indicator is set and the function returns \code{1}; otherwise
the function returns \code{0}.  The error indicator may or may not be
cleared if it was previously set.
\end{cfuncdesc}

\begin{cfuncdesc}{void}{PyErr_SetInterrupt}{}
This function is obsolete.  It simulates the effect of a
\constant{SIGINT}\ttindex{SIGINT} signal arriving --- the next time
\cfunction{PyErr_CheckSignals()} is called,
\withsubitem{(built-in exception)}{\ttindex{KeyboardInterrupt}}
\exception{KeyboardInterrupt} will be raised.
It may be called without holding the interpreter lock.
\end{cfuncdesc}

\begin{cfuncdesc}{PyObject*}{PyErr_NewException}{char *name,
                                                 PyObject *base,
                                                 PyObject *dict}
This utility function creates and returns a new exception object.  The
\var{name} argument must be the name of the new exception, a C string
of the form \code{module.class}.  The \var{base} and
\var{dict} arguments are normally \NULL{}.  This creates a
class object derived from the root for all exceptions, the built-in
name \exception{Exception} (accessible in C as
\cdata{PyExc_Exception}).  The \member{__module__} attribute of the
new class is set to the first part (up to the last dot) of the
\var{name} argument, and the class name is set to the last part (after
the last dot).  The \var{base} argument can be used to specify an
alternate base class.  The \var{dict} argument can be used to specify
a dictionary of class variables and methods.
\end{cfuncdesc}

\begin{cfuncdesc}{void}{PyErr_WriteUnraisable}{PyObject *obj}
This utility function prints a warning message to \var{sys.stderr}
when an exception has been set but it is impossible for the
interpreter to actually raise the exception.  It is used, for example,
when an exception occurs in an \member{__del__} method.

The function is called with a single argument \var{obj} that
identifies where the context in which the unraisable exception
occurred.  The repr of \var{obj} will be printed in the warning
message.
\end{cfuncdesc}

\section{Standard Exceptions \label{standardExceptions}}

All standard Python exceptions are available as global variables whose
names are \samp{PyExc_} followed by the Python exception name.  These
have the type \ctype{PyObject*}; they are all class objects.  For
completeness, here are all the variables:

\begin{tableiii}{l|l|c}{cdata}{C Name}{Python Name}{Notes}
  \lineiii{PyExc_Exception}{\exception{Exception}}{(1)}
  \lineiii{PyExc_StandardError}{\exception{StandardError}}{(1)}
  \lineiii{PyExc_ArithmeticError}{\exception{ArithmeticError}}{(1)}
  \lineiii{PyExc_LookupError}{\exception{LookupError}}{(1)}
  \lineiii{PyExc_AssertionError}{\exception{AssertionError}}{}
  \lineiii{PyExc_AttributeError}{\exception{AttributeError}}{}
  \lineiii{PyExc_EOFError}{\exception{EOFError}}{}
  \lineiii{PyExc_EnvironmentError}{\exception{EnvironmentError}}{(1)}
  \lineiii{PyExc_FloatingPointError}{\exception{FloatingPointError}}{}
  \lineiii{PyExc_IOError}{\exception{IOError}}{}
  \lineiii{PyExc_ImportError}{\exception{ImportError}}{}
  \lineiii{PyExc_IndexError}{\exception{IndexError}}{}
  \lineiii{PyExc_KeyError}{\exception{KeyError}}{}
  \lineiii{PyExc_KeyboardInterrupt}{\exception{KeyboardInterrupt}}{}
  \lineiii{PyExc_MemoryError}{\exception{MemoryError}}{}
  \lineiii{PyExc_NameError}{\exception{NameError}}{}
  \lineiii{PyExc_NotImplementedError}{\exception{NotImplementedError}}{}
  \lineiii{PyExc_OSError}{\exception{OSError}}{}
  \lineiii{PyExc_OverflowError}{\exception{OverflowError}}{}
  \lineiii{PyExc_RuntimeError}{\exception{RuntimeError}}{}
  \lineiii{PyExc_SyntaxError}{\exception{SyntaxError}}{}
  \lineiii{PyExc_SystemError}{\exception{SystemError}}{}
  \lineiii{PyExc_SystemExit}{\exception{SystemExit}}{}
  \lineiii{PyExc_TypeError}{\exception{TypeError}}{}
  \lineiii{PyExc_ValueError}{\exception{ValueError}}{}
  \lineiii{PyExc_WindowsError}{\exception{WindowsError}}{(2)}
  \lineiii{PyExc_ZeroDivisionError}{\exception{ZeroDivisionError}}{}
\end{tableiii}

\noindent
Notes:
\begin{description}
\item[(1)]
  This is a base class for other standard exceptions.

\item[(2)]
  Only defined on Windows; protect code that uses this by testing that
  the preprocessor macro \code{MS_WINDOWS} is defined.
\end{description}


\section{Deprecation of String Exceptions}

All exceptions built into Python or provided in the standard library
are derived from \exception{Exception}.
\withsubitem{(built-in exception)}{\ttindex{Exception}}

String exceptions are still supported in the interpreter to allow
existing code to run unmodified, but this will also change in a future 
release.


\chapter{Utilities \label{utilities}}

The functions in this chapter perform various utility tasks, such as
parsing function arguments and constructing Python values from C
values.

\section{OS Utilities \label{os}}

\begin{cfuncdesc}{int}{Py_FdIsInteractive}{FILE *fp, char *filename}
Return true (nonzero) if the standard I/O file \var{fp} with name
\var{filename} is deemed interactive.  This is the case for files for
which \samp{isatty(fileno(\var{fp}))} is true.  If the global flag
\cdata{Py_InteractiveFlag} is true, this function also returns true if
the \var{filename} pointer is \NULL{} or if the name is equal to one of
the strings \code{'<stdin>'} or \code{'???'}.
\end{cfuncdesc}

\begin{cfuncdesc}{long}{PyOS_GetLastModificationTime}{char *filename}
Return the time of last modification of the file \var{filename}.
The result is encoded in the same way as the timestamp returned by
the standard C library function \cfunction{time()}.
\end{cfuncdesc}

\begin{cfuncdesc}{void}{PyOS_AfterFork}{}
Function to update some internal state after a process fork; this
should be called in the new process if the Python interpreter will
continue to be used.  If a new executable is loaded into the new
process, this function does not need to be called.
\end{cfuncdesc}

\begin{cfuncdesc}{int}{PyOS_CheckStack}{}
Return true when the interpreter runs out of stack space.  This is a
reliable check, but is only available when \code{USE_STACKCHECK} is
defined (currently on Windows using the Microsoft Visual C++ compiler
and on the Macintosh).  \code{USE_CHECKSTACK} will be defined
automatically; you should never change the definition in your own
code.
\end{cfuncdesc}

\begin{cfuncdesc}{PyOS_sighandler_t}{PyOS_getsig}{int i}
Return the current signal handler for signal \var{i}.
This is a thin wrapper around either \cfunction{sigaction} or
\cfunction{signal}.  Do not call those functions directly!
\ctype{PyOS_sighandler_t} is a typedef alias for \ctype{void (*)(int)}.
\end{cfuncdesc}

\begin{cfuncdesc}{PyOS_sighandler_t}{PyOS_setsig}{int i, PyOS_sighandler_t h}
Set the signal handler for signal \var{i} to be \var{h};
return the old signal handler.
This is a thin wrapper around either \cfunction{sigaction} or
\cfunction{signal}.  Do not call those functions directly!
\ctype{PyOS_sighandler_t} is a typedef alias for \ctype{void (*)(int)}.
\end{cfuncdesc}


\section{Process Control \label{processControl}}

\begin{cfuncdesc}{void}{Py_FatalError}{char *message}
Print a fatal error message and kill the process.  No cleanup is
performed.  This function should only be invoked when a condition is
detected that would make it dangerous to continue using the Python
interpreter; e.g., when the object administration appears to be
corrupted.  On \UNIX{}, the standard C library function
\cfunction{abort()}\ttindex{abort()} is called which will attempt to
produce a \file{core} file.
\end{cfuncdesc}

\begin{cfuncdesc}{void}{Py_Exit}{int status}
Exit the current process.  This calls
\cfunction{Py_Finalize()}\ttindex{Py_Finalize()} and
then calls the standard C library function
\code{exit(\var{status})}\ttindex{exit()}.
\end{cfuncdesc}

\begin{cfuncdesc}{int}{Py_AtExit}{void (*func) ()}
Register a cleanup function to be called by
\cfunction{Py_Finalize()}\ttindex{Py_Finalize()}.
The cleanup function will be called with no arguments and should
return no value.  At most 32 \index{cleanup functions}cleanup
functions can be registered.
When the registration is successful, \cfunction{Py_AtExit()} returns
\code{0}; on failure, it returns \code{-1}.  The cleanup function
registered last is called first.  Each cleanup function will be called
at most once.  Since Python's internal finallization will have
completed before the cleanup function, no Python APIs should be called
by \var{func}.
\end{cfuncdesc}


\section{Importing Modules \label{importing}}

\begin{cfuncdesc}{PyObject*}{PyImport_ImportModule}{char *name}
This is a simplified interface to
\cfunction{PyImport_ImportModuleEx()} below, leaving the
\var{globals} and \var{locals} arguments set to \NULL{}.  When the
\var{name} argument contains a dot (i.e., when it specifies a
submodule of a package), the \var{fromlist} argument is set to the
list \code{['*']} so that the return value is the named module rather
than the top-level package containing it as would otherwise be the
case.  (Unfortunately, this has an additional side effect when
\var{name} in fact specifies a subpackage instead of a submodule: the
submodules specified in the package's \code{__all__} variable are
\index{package variable!\code{__all__}}
\withsubitem{(package variable)}{\ttindex{__all__}}loaded.)  Return a
new reference to the imported module, or
\NULL{} with an exception set on failure (the module may still be
created in this case --- examine \code{sys.modules} to find out).
\withsubitem{(in module sys)}{\ttindex{modules}}
\end{cfuncdesc}

\begin{cfuncdesc}{PyObject*}{PyImport_ImportModuleEx}{char *name, PyObject *globals, PyObject *locals, PyObject *fromlist}
Import a module.  This is best described by referring to the built-in
Python function \function{__import__()}\bifuncindex{__import__}, as
the standard \function{__import__()} function calls this function
directly.

The return value is a new reference to the imported module or
top-level package, or \NULL{} with an exception set on failure
(the module may still be created in this case).  Like for
\function{__import__()}, the return value when a submodule of a
package was requested is normally the top-level package, unless a
non-empty \var{fromlist} was given.
\end{cfuncdesc}

\begin{cfuncdesc}{PyObject*}{PyImport_Import}{PyObject *name}
This is a higher-level interface that calls the current ``import hook
function''.  It invokes the \function{__import__()} function from the
\code{__builtins__} of the current globals.  This means that the
import is done using whatever import hooks are installed in the
current environment, e.g. by \module{rexec}\refstmodindex{rexec} or
\module{ihooks}\refstmodindex{ihooks}.
\end{cfuncdesc}

\begin{cfuncdesc}{PyObject*}{PyImport_ReloadModule}{PyObject *m}
Reload a module.  This is best described by referring to the built-in
Python function \function{reload()}\bifuncindex{reload}, as the standard
\function{reload()} function calls this function directly.  Return a
new reference to the reloaded module, or \NULL{} with an exception set
on failure (the module still exists in this case).
\end{cfuncdesc}

\begin{cfuncdesc}{PyObject*}{PyImport_AddModule}{char *name}
Return the module object corresponding to a module name.  The
\var{name} argument may be of the form \code{package.module}).  First
check the modules dictionary if there's one there, and if not, create
a new one and insert in in the modules dictionary.
Warning: this function does not load or import the module; if the
module wasn't already loaded, you will get an empty module object.
Use \cfunction{PyImport_ImportModule()} or one of its variants to
import a module.
Return \NULL{} with an exception set on failure.
\end{cfuncdesc}

\begin{cfuncdesc}{PyObject*}{PyImport_ExecCodeModule}{char *name, PyObject *co}
Given a module name (possibly of the form \code{package.module}) and a
code object read from a Python bytecode file or obtained from the
built-in function \function{compile()}\bifuncindex{compile}, load the
module.  Return a new reference to the module object, or \NULL{} with
an exception set if an error occurred (the module may still be created
in this case).  (This function would reload the module if it was
already imported.)
\end{cfuncdesc}

\begin{cfuncdesc}{long}{PyImport_GetMagicNumber}{}
Return the magic number for Python bytecode files (a.k.a.
\file{.pyc} and \file{.pyo} files).  The magic number should be
present in the first four bytes of the bytecode file, in little-endian
byte order.
\end{cfuncdesc}

\begin{cfuncdesc}{PyObject*}{PyImport_GetModuleDict}{}
Return the dictionary used for the module administration
(a.k.a. \code{sys.modules}).  Note that this is a per-interpreter
variable.
\end{cfuncdesc}

\begin{cfuncdesc}{void}{_PyImport_Init}{}
Initialize the import mechanism.  For internal use only.
\end{cfuncdesc}

\begin{cfuncdesc}{void}{PyImport_Cleanup}{}
Empty the module table.  For internal use only.
\end{cfuncdesc}

\begin{cfuncdesc}{void}{_PyImport_Fini}{}
Finalize the import mechanism.  For internal use only.
\end{cfuncdesc}

\begin{cfuncdesc}{PyObject*}{_PyImport_FindExtension}{char *, char *}
For internal use only.
\end{cfuncdesc}

\begin{cfuncdesc}{PyObject*}{_PyImport_FixupExtension}{char *, char *}
For internal use only.
\end{cfuncdesc}

\begin{cfuncdesc}{int}{PyImport_ImportFrozenModule}{char *name}
Load a frozen module named \var{name}.  Return \code{1} for success,
\code{0} if the module is not found, and \code{-1} with an exception
set if the initialization failed.  To access the imported module on a
successful load, use \cfunction{PyImport_ImportModule()}.
(Note the misnomer --- this function would reload the module if it was
already imported.)
\end{cfuncdesc}

\begin{ctypedesc}[_frozen]{struct _frozen}
This is the structure type definition for frozen module descriptors,
as generated by the \program{freeze}\index{freeze utility} utility
(see \file{Tools/freeze/} in the Python source distribution).  Its
definition, found in \file{Include/import.h}, is:

\begin{verbatim}
struct _frozen {
    char *name;
    unsigned char *code;
    int size;
};
\end{verbatim}
\end{ctypedesc}

\begin{cvardesc}{struct _frozen*}{PyImport_FrozenModules}
This pointer is initialized to point to an array of \ctype{struct
_frozen} records, terminated by one whose members are all
\NULL{} or zero.  When a frozen module is imported, it is searched in
this table.  Third-party code could play tricks with this to provide a 
dynamically created collection of frozen modules.
\end{cvardesc}

\begin{cfuncdesc}{int}{PyImport_AppendInittab}{char *name,
                                               void (*initfunc)(void)}
Add a single module to the existing table of built-in modules.  This
is a convenience wrapper around \cfunction{PyImport_ExtendInittab()},
returning \code{-1} if the table could not be extended.  The new
module can be imported by the name \var{name}, and uses the function
\var{initfunc} as the initialization function called on the first
attempted import.  This should be called before
\cfunction{Py_Initialize()}.
\end{cfuncdesc}

\begin{ctypedesc}[_inittab]{struct _inittab}
Structure describing a single entry in the list of built-in modules.
Each of these structures gives the name and initialization function
for a module built into the interpreter.  Programs which embed Python
may use an array of these structures in conjunction with
\cfunction{PyImport_ExtendInittab()} to provide additional built-in
modules.  The structure is defined in \file{Include/import.h} as:

\begin{verbatim}
struct _inittab {
    char *name;
    void (*initfunc)(void);
};
\end{verbatim}
\end{ctypedesc}

\begin{cfuncdesc}{int}{PyImport_ExtendInittab}{struct _inittab *newtab}
Add a collection of modules to the table of built-in modules.  The
\var{newtab} array must end with a sentinel entry which contains
\NULL{} for the \member{name} field; failure to provide the sentinel
value can result in a memory fault.  Returns \code{0} on success or
\code{-1} if insufficient memory could be allocated to extend the
internal table.  In the event of failure, no modules are added to the
internal table.  This should be called before
\cfunction{Py_Initialize()}.
\end{cfuncdesc}


\chapter{Abstract Objects Layer \label{abstract}}

The functions in this chapter interact with Python objects regardless
of their type, or with wide classes of object types (e.g. all
numerical types, or all sequence types).  When used on object types
for which they do not apply, they will raise a Python exception.

\section{Object Protocol \label{object}}

\begin{cfuncdesc}{int}{PyObject_Print}{PyObject *o, FILE *fp, int flags}
Print an object \var{o}, on file \var{fp}.  Returns \code{-1} on error.
The flags argument is used to enable certain printing options.  The
only option currently supported is \constant{Py_PRINT_RAW}; if given,
the \function{str()} of the object is written instead of the
\function{repr()}.
\end{cfuncdesc}

\begin{cfuncdesc}{int}{PyObject_HasAttrString}{PyObject *o, char *attr_name}
Returns \code{1} if \var{o} has the attribute \var{attr_name}, and
\code{0} otherwise.  This is equivalent to the Python expression
\samp{hasattr(\var{o}, \var{attr_name})}.
This function always succeeds.
\end{cfuncdesc}

\begin{cfuncdesc}{PyObject*}{PyObject_GetAttrString}{PyObject *o,
                                                     char *attr_name}
Retrieve an attribute named \var{attr_name} from object \var{o}.
Returns the attribute value on success, or \NULL{} on failure.
This is the equivalent of the Python expression
\samp{\var{o}.\var{attr_name}}.
\end{cfuncdesc}


\begin{cfuncdesc}{int}{PyObject_HasAttr}{PyObject *o, PyObject *attr_name}
Returns \code{1} if \var{o} has the attribute \var{attr_name}, and
\code{0} otherwise.  This is equivalent to the Python expression
\samp{hasattr(\var{o}, \var{attr_name})}. 
This function always succeeds.
\end{cfuncdesc}


\begin{cfuncdesc}{PyObject*}{PyObject_GetAttr}{PyObject *o,
                                               PyObject *attr_name}
Retrieve an attribute named \var{attr_name} from object \var{o}.
Returns the attribute value on success, or \NULL{} on failure.
This is the equivalent of the Python expression
\samp{\var{o}.\var{attr_name}}.
\end{cfuncdesc}


\begin{cfuncdesc}{int}{PyObject_SetAttrString}{PyObject *o, char *attr_name, PyObject *v}
Set the value of the attribute named \var{attr_name}, for object
\var{o}, to the value \var{v}. Returns \code{-1} on failure.  This is
the equivalent of the Python statement \samp{\var{o}.\var{attr_name} =
\var{v}}.
\end{cfuncdesc}


\begin{cfuncdesc}{int}{PyObject_SetAttr}{PyObject *o, PyObject *attr_name, PyObject *v}
Set the value of the attribute named \var{attr_name}, for
object \var{o},
to the value \var{v}. Returns \code{-1} on failure.  This is
the equivalent of the Python statement \samp{\var{o}.\var{attr_name} =
\var{v}}.
\end{cfuncdesc}


\begin{cfuncdesc}{int}{PyObject_DelAttrString}{PyObject *o, char *attr_name}
Delete attribute named \var{attr_name}, for object \var{o}. Returns
\code{-1} on failure.  This is the equivalent of the Python
statement: \samp{del \var{o}.\var{attr_name}}.
\end{cfuncdesc}


\begin{cfuncdesc}{int}{PyObject_DelAttr}{PyObject *o, PyObject *attr_name}
Delete attribute named \var{attr_name}, for object \var{o}. Returns
\code{-1} on failure.  This is the equivalent of the Python
statement \samp{del \var{o}.\var{attr_name}}.
\end{cfuncdesc}


\begin{cfuncdesc}{int}{PyObject_Cmp}{PyObject *o1, PyObject *o2, int *result}
Compare the values of \var{o1} and \var{o2} using a routine provided
by \var{o1}, if one exists, otherwise with a routine provided by
\var{o2}.  The result of the comparison is returned in \var{result}.
Returns \code{-1} on failure.  This is the equivalent of the Python
statement\bifuncindex{cmp} \samp{\var{result} = cmp(\var{o1}, \var{o2})}.
\end{cfuncdesc}


\begin{cfuncdesc}{int}{PyObject_Compare}{PyObject *o1, PyObject *o2}
Compare the values of \var{o1} and \var{o2} using a routine provided
by \var{o1}, if one exists, otherwise with a routine provided by
\var{o2}.  Returns the result of the comparison on success.  On error,
the value returned is undefined; use \cfunction{PyErr_Occurred()} to
detect an error.  This is equivalent to the Python
expression\bifuncindex{cmp} \samp{cmp(\var{o1}, \var{o2})}.
\end{cfuncdesc}


\begin{cfuncdesc}{PyObject*}{PyObject_Repr}{PyObject *o}
Compute a string representation of object \var{o}.  Returns the
string representation on success, \NULL{} on failure.  This is
the equivalent of the Python expression \samp{repr(\var{o})}.
Called by the \function{repr()}\bifuncindex{repr} built-in function
and by reverse quotes.
\end{cfuncdesc}


\begin{cfuncdesc}{PyObject*}{PyObject_Str}{PyObject *o}
Compute a string representation of object \var{o}.  Returns the
string representation on success, \NULL{} on failure.  This is
the equivalent of the Python expression \samp{str(\var{o})}.
Called by the \function{str()}\bifuncindex{str} built-in function and
by the \keyword{print} statement.
\end{cfuncdesc}


\begin{cfuncdesc}{PyObject*}{PyObject_Unicode}{PyObject *o}
Compute a Unicode string representation of object \var{o}.  Returns the
Unicode string representation on success, \NULL{} on failure.  This is
the equivalent of the Python expression \samp{unistr(\var{o})}.
Called by the \function{unistr()}\bifuncindex{unistr} built-in function.
\end{cfuncdesc}

\begin{cfuncdesc}{int}{PyObject_IsInstance}{PyObject *inst, PyObject *cls}
Return \code{1} if \var{inst} is an instance of the class \var{cls} or
a subclass of \var{cls}.  If \var{cls} is a type object rather than a
class object, \cfunction{PyObject_IsInstance()} returns \code{1} if
\var{inst} is of type \var{cls}.  If \var{inst} is not a class
instance and \var{cls} is neither a type object or class object,
\var{inst} must have a \member{__class__} attribute --- the class
relationship of the value of that attribute with \var{cls} will be
used to determine the result of this function.
\versionadded{2.1}
\end{cfuncdesc}

Subclass determination is done in a fairly straightforward way, but
includes a wrinkle that implementors of extensions to the class system
may want to be aware of.  If \class{A} and \class{B} are class
objects, \class{B} is a subclass of \class{A} if it inherits from
\class{A} either directly or indirectly.  If either is not a class
object, a more general mechanism is used to determine the class
relationship of the two objects.  When testing if \var{B} is a
subclass of \var{A}, if \var{A} is \var{B},
\cfunction{PyObject_IsSubclass()} returns true.  If \var{A} and
\var{B} are different objects, \var{B}'s \member{__bases__} attribute
is searched in a depth-first fashion for \var{A} --- the presence of
the \member{__bases__} attribute is considered sufficient for this
determination.

\begin{cfuncdesc}{int}{PyObject_IsSubclass}{PyObject *derived,
                                            PyObject *cls}
Returns \code{1} if the class \var{derived} is identical to or derived
from the class \var{cls}, otherwise returns \code{0}.  In case of an
error, returns \code{-1}.  If either \var{derived} or \var{cls} is not
an actual class object, this function uses the generic algorithm
described above.
\versionadded{2.1}
\end{cfuncdesc}


\begin{cfuncdesc}{int}{PyCallable_Check}{PyObject *o}
Determine if the object \var{o} is callable.  Return \code{1} if the
object is callable and \code{0} otherwise.
This function always succeeds.
\end{cfuncdesc}


\begin{cfuncdesc}{PyObject*}{PyObject_CallObject}{PyObject *callable_object,
                                                  PyObject *args}
Call a callable Python object \var{callable_object}, with
arguments given by the tuple \var{args}.  If no arguments are
needed, then \var{args} may be \NULL{}.  Returns the result of the
call on success, or \NULL{} on failure.  This is the equivalent
of the Python expression \samp{apply(\var{callable_object}, \var{args})}.
\bifuncindex{apply}
\end{cfuncdesc}

\begin{cfuncdesc}{PyObject*}{PyObject_CallFunction}{PyObject *callable_object,
                                                    char *format, ...}
Call a callable Python object \var{callable_object}, with a
variable number of C arguments. The C arguments are described
using a \cfunction{Py_BuildValue()} style format string. The format may
be \NULL{}, indicating that no arguments are provided.  Returns the
result of the call on success, or \NULL{} on failure.  This is
the equivalent of the Python expression \samp{apply(\var{callable_object},
\var{args})}.\bifuncindex{apply}
\end{cfuncdesc}


\begin{cfuncdesc}{PyObject*}{PyObject_CallMethod}{PyObject *o,
                                           char *method, char *format, ...}
Call the method named \var{m} of object \var{o} with a variable number
of C arguments.  The C arguments are described by a
\cfunction{Py_BuildValue()} format string.  The format may be \NULL{},
indicating that no arguments are provided. Returns the result of the
call on success, or \NULL{} on failure.  This is the equivalent of the
Python expression \samp{\var{o}.\var{method}(\var{args})}.
Note that special method names, such as \method{__add__()},
\method{__getitem__()}, and so on are not supported.  The specific
abstract-object routines for these must be used.
\end{cfuncdesc}


\begin{cfuncdesc}{int}{PyObject_Hash}{PyObject *o}
Compute and return the hash value of an object \var{o}.  On
failure, return \code{-1}.  This is the equivalent of the Python
expression \samp{hash(\var{o})}.\bifuncindex{hash}
\end{cfuncdesc}


\begin{cfuncdesc}{int}{PyObject_IsTrue}{PyObject *o}
Returns \code{1} if the object \var{o} is considered to be true, and
\code{0} otherwise. This is equivalent to the Python expression
\samp{not not \var{o}}.
This function always succeeds.
\end{cfuncdesc}


\begin{cfuncdesc}{PyObject*}{PyObject_Type}{PyObject *o}
On success, returns a type object corresponding to the object
type of object \var{o}. On failure, returns \NULL{}.  This is
equivalent to the Python expression \samp{type(\var{o})}.
\bifuncindex{type}
\end{cfuncdesc}

\begin{cfuncdesc}{int}{PyObject_Length}{PyObject *o}
Return the length of object \var{o}.  If the object \var{o} provides
both sequence and mapping protocols, the sequence length is
returned.  On error, \code{-1} is returned.  This is the equivalent
to the Python expression \samp{len(\var{o})}.\bifuncindex{len}
\end{cfuncdesc}


\begin{cfuncdesc}{PyObject*}{PyObject_GetItem}{PyObject *o, PyObject *key}
Return element of \var{o} corresponding to the object \var{key} or
\NULL{} on failure. This is the equivalent of the Python expression
\samp{\var{o}[\var{key}]}.
\end{cfuncdesc}


\begin{cfuncdesc}{int}{PyObject_SetItem}{PyObject *o, PyObject *key, PyObject *v}
Map the object \var{key} to the value \var{v}.
Returns \code{-1} on failure.  This is the equivalent
of the Python statement \samp{\var{o}[\var{key}] = \var{v}}.
\end{cfuncdesc}


\begin{cfuncdesc}{int}{PyObject_DelItem}{PyObject *o, PyObject *key}
Delete the mapping for \var{key} from \var{o}.  Returns \code{-1} on
failure. This is the equivalent of the Python statement \samp{del
\var{o}[\var{key}]}.
\end{cfuncdesc}

\begin{cfuncdesc}{int}{PyObject_AsFileDescriptor}{PyObject *o}
Derives a file-descriptor from a Python object.  If the object
is an integer or long integer, its value is returned.  If not, the
object's \method{fileno()} method is called if it exists; the method
must return an integer or long integer, which is returned as the file
descriptor value.  Returns \code{-1} on failure.
\end{cfuncdesc}

\section{Number Protocol \label{number}}

\begin{cfuncdesc}{int}{PyNumber_Check}{PyObject *o}
Returns \code{1} if the object \var{o} provides numeric protocols, and
false otherwise. 
This function always succeeds.
\end{cfuncdesc}


\begin{cfuncdesc}{PyObject*}{PyNumber_Add}{PyObject *o1, PyObject *o2}
Returns the result of adding \var{o1} and \var{o2}, or \NULL{} on
failure.  This is the equivalent of the Python expression
\samp{\var{o1} + \var{o2}}.
\end{cfuncdesc}


\begin{cfuncdesc}{PyObject*}{PyNumber_Subtract}{PyObject *o1, PyObject *o2}
Returns the result of subtracting \var{o2} from \var{o1}, or
\NULL{} on failure.  This is the equivalent of the Python expression
\samp{\var{o1} - \var{o2}}.
\end{cfuncdesc}


\begin{cfuncdesc}{PyObject*}{PyNumber_Multiply}{PyObject *o1, PyObject *o2}
Returns the result of multiplying \var{o1} and \var{o2}, or \NULL{} on
failure.  This is the equivalent of the Python expression
\samp{\var{o1} * \var{o2}}.
\end{cfuncdesc}


\begin{cfuncdesc}{PyObject*}{PyNumber_Divide}{PyObject *o1, PyObject *o2}
Returns the result of dividing \var{o1} by \var{o2}, or \NULL{} on
failure. 
This is the equivalent of the Python expression \samp{\var{o1} /
\var{o2}}.
\end{cfuncdesc}


\begin{cfuncdesc}{PyObject*}{PyNumber_Remainder}{PyObject *o1, PyObject *o2}
Returns the remainder of dividing \var{o1} by \var{o2}, or \NULL{} on
failure.  This is the equivalent of the Python expression
\samp{\var{o1} \%\ \var{o2}}.
\end{cfuncdesc}


\begin{cfuncdesc}{PyObject*}{PyNumber_Divmod}{PyObject *o1, PyObject *o2}
See the built-in function \function{divmod()}\bifuncindex{divmod}.
Returns \NULL{} on failure.  This is the equivalent of the Python
expression \samp{divmod(\var{o1}, \var{o2})}.
\end{cfuncdesc}


\begin{cfuncdesc}{PyObject*}{PyNumber_Power}{PyObject *o1, PyObject *o2, PyObject *o3}
See the built-in function \function{pow()}\bifuncindex{pow}.  Returns
\NULL{} on failure. This is the equivalent of the Python expression
\samp{pow(\var{o1}, \var{o2}, \var{o3})}, where \var{o3} is optional.
If \var{o3} is to be ignored, pass \cdata{Py_None} in its place
(passing \NULL{} for \var{o3} would cause an illegal memory access).
\end{cfuncdesc}


\begin{cfuncdesc}{PyObject*}{PyNumber_Negative}{PyObject *o}
Returns the negation of \var{o} on success, or \NULL{} on failure.
This is the equivalent of the Python expression \samp{-\var{o}}.
\end{cfuncdesc}


\begin{cfuncdesc}{PyObject*}{PyNumber_Positive}{PyObject *o}
Returns \var{o} on success, or \NULL{} on failure.
This is the equivalent of the Python expression \samp{+\var{o}}.
\end{cfuncdesc}


\begin{cfuncdesc}{PyObject*}{PyNumber_Absolute}{PyObject *o}
Returns the absolute value of \var{o}, or \NULL{} on failure.  This is
the equivalent of the Python expression \samp{abs(\var{o})}.
\bifuncindex{abs}
\end{cfuncdesc}


\begin{cfuncdesc}{PyObject*}{PyNumber_Invert}{PyObject *o}
Returns the bitwise negation of \var{o} on success, or \NULL{} on
failure.  This is the equivalent of the Python expression
\samp{\~\var{o}}.
\end{cfuncdesc}


\begin{cfuncdesc}{PyObject*}{PyNumber_Lshift}{PyObject *o1, PyObject *o2}
Returns the result of left shifting \var{o1} by \var{o2} on success,
or \NULL{} on failure.  This is the equivalent of the Python
expression \samp{\var{o1} <\code{<} \var{o2}}.
\end{cfuncdesc}


\begin{cfuncdesc}{PyObject*}{PyNumber_Rshift}{PyObject *o1, PyObject *o2}
Returns the result of right shifting \var{o1} by \var{o2} on success,
or \NULL{} on failure.  This is the equivalent of the Python
expression \samp{\var{o1} >\code{>} \var{o2}}.
\end{cfuncdesc}


\begin{cfuncdesc}{PyObject*}{PyNumber_And}{PyObject *o1, PyObject *o2}
Returns the ``bitwise and'' of \var{o2} and \var{o2} on success and
\NULL{} on failure. This is the equivalent of the Python expression
\samp{\var{o1} \&\ \var{o2}}.
\end{cfuncdesc}


\begin{cfuncdesc}{PyObject*}{PyNumber_Xor}{PyObject *o1, PyObject *o2}
Returns the ``bitwise exclusive or'' of \var{o1} by \var{o2} on success,
or \NULL{} on failure.  This is the equivalent of the Python
expression \samp{\var{o1} \^{ }\var{o2}}.
\end{cfuncdesc}

\begin{cfuncdesc}{PyObject*}{PyNumber_Or}{PyObject *o1, PyObject *o2}
Returns the ``bitwise or'' of \var{o1} and \var{o2} on success, or
\NULL{} on failure.  This is the equivalent of the Python expression
\samp{\var{o1} | \var{o2}}.
\end{cfuncdesc}


\begin{cfuncdesc}{PyObject*}{PyNumber_InPlaceAdd}{PyObject *o1, PyObject *o2}
Returns the result of adding \var{o1} and \var{o2}, or \NULL{} on failure. 
The operation is done \emph{in-place} when \var{o1} supports it.  This is the
equivalent of the Python expression \samp{\var{o1} += \var{o2}}.
\end{cfuncdesc}


\begin{cfuncdesc}{PyObject*}{PyNumber_InPlaceSubtract}{PyObject *o1, PyObject *o2}
Returns the result of subtracting \var{o2} from \var{o1}, or
\NULL{} on failure.  The operation is done \emph{in-place} when \var{o1}
supports it.  This is the equivalent of the Python expression \samp{\var{o1}
-= \var{o2}}.
\end{cfuncdesc}


\begin{cfuncdesc}{PyObject*}{PyNumber_InPlaceMultiply}{PyObject *o1, PyObject *o2}
Returns the result of multiplying \var{o1} and \var{o2}, or \NULL{} on
failure.  The operation is done \emph{in-place} when \var{o1} supports it. 
This is the equivalent of the Python expression \samp{\var{o1} *= \var{o2}}.
\end{cfuncdesc}


\begin{cfuncdesc}{PyObject*}{PyNumber_InPlaceDivide}{PyObject *o1, PyObject *o2}
Returns the result of dividing \var{o1} by \var{o2}, or \NULL{} on failure. 
The operation is done \emph{in-place} when \var{o1} supports it. This is the
equivalent of the Python expression \samp{\var{o1} /= \var{o2}}.
\end{cfuncdesc}


\begin{cfuncdesc}{PyObject*}{PyNumber_InPlaceRemainder}{PyObject *o1, PyObject *o2}
Returns the remainder of dividing \var{o1} by \var{o2}, or \NULL{} on
failure.  The operation is done \emph{in-place} when \var{o1} supports it. 
This is the equivalent of the Python expression \samp{\var{o1} \%= \var{o2}}.
\end{cfuncdesc}


\begin{cfuncdesc}{PyObject*}{PyNumber_InPlacePower}{PyObject *o1, PyObject *o2, PyObject *o3}
See the built-in function \function{pow()}\bifuncindex{pow}.  Returns
\NULL{} on failure.  The operation is done \emph{in-place} when \var{o1}
supports it.  This is the equivalent of the Python expression \samp{\var{o1}
**= \var{o2}} when o3 is \cdata{Py_None}, or an in-place variant of
\samp{pow(\var{o1}, \var{o2}, \var{o3})} otherwise. If \var{o3} is to be
ignored, pass \cdata{Py_None} in its place (passing \NULL{} for \var{o3}
would cause an illegal memory access).
\end{cfuncdesc}

\begin{cfuncdesc}{PyObject*}{PyNumber_InPlaceLshift}{PyObject *o1, PyObject *o2}
Returns the result of left shifting \var{o1} by \var{o2} on success, or
\NULL{} on failure.  The operation is done \emph{in-place} when \var{o1}
supports it.  This is the equivalent of the Python expression \samp{\var{o1}
<\code{<=} \var{o2}}.
\end{cfuncdesc}


\begin{cfuncdesc}{PyObject*}{PyNumber_InPlaceRshift}{PyObject *o1, PyObject *o2}
Returns the result of right shifting \var{o1} by \var{o2} on success, or
\NULL{} on failure.  The operation is done \emph{in-place} when \var{o1}
supports it.  This is the equivalent of the Python expression \samp{\var{o1}
>\code{>=} \var{o2}}.
\end{cfuncdesc}


\begin{cfuncdesc}{PyObject*}{PyNumber_InPlaceAnd}{PyObject *o1, PyObject *o2}
Returns the ``bitwise and'' of \var{o1} and \var{o2} on success
and \NULL{} on failure. The operation is done \emph{in-place} when
\var{o1} supports it.  This is the equivalent of the Python expression
\samp{\var{o1} \&= \var{o2}}.
\end{cfuncdesc}


\begin{cfuncdesc}{PyObject*}{PyNumber_InPlaceXor}{PyObject *o1, PyObject *o2}
Returns the ``bitwise exclusive or'' of \var{o1} by \var{o2} on success, or
\NULL{} on failure.  The operation is done \emph{in-place} when \var{o1}
supports it.  This is the equivalent of the Python expression \samp{\var{o1}
\^= \var{o2}}.
\end{cfuncdesc}

\begin{cfuncdesc}{PyObject*}{PyNumber_InPlaceOr}{PyObject *o1, PyObject *o2}
Returns the ``bitwise or'' of \var{o1} and \var{o2} on success, or \NULL{}
on failure.  The operation is done \emph{in-place} when \var{o1} supports
it.  This is the equivalent of the Python expression \samp{\var{o1} |=
\var{o2}}.
\end{cfuncdesc}

\begin{cfuncdesc}{int}{PyNumber_Coerce}{PyObject **p1, PyObject **p2}
This function takes the addresses of two variables of type
\ctype{PyObject*}.  If the objects pointed to by \code{*\var{p1}} and
\code{*\var{p2}} have the same type, increment their reference count
and return \code{0} (success). If the objects can be converted to a
common numeric type, replace \code{*p1} and \code{*p2} by their
converted value (with 'new' reference counts), and return \code{0}.
If no conversion is possible, or if some other error occurs, return
\code{-1} (failure) and don't increment the reference counts.  The
call \code{PyNumber_Coerce(\&o1, \&o2)} is equivalent to the Python
statement \samp{\var{o1}, \var{o2} = coerce(\var{o1}, \var{o2})}.
\bifuncindex{coerce}
\end{cfuncdesc}

\begin{cfuncdesc}{PyObject*}{PyNumber_Int}{PyObject *o}
Returns the \var{o} converted to an integer object on success, or
\NULL{} on failure.  This is the equivalent of the Python
expression \samp{int(\var{o})}.\bifuncindex{int}
\end{cfuncdesc}

\begin{cfuncdesc}{PyObject*}{PyNumber_Long}{PyObject *o}
Returns the \var{o} converted to a long integer object on success,
or \NULL{} on failure.  This is the equivalent of the Python
expression \samp{long(\var{o})}.\bifuncindex{long}
\end{cfuncdesc}

\begin{cfuncdesc}{PyObject*}{PyNumber_Float}{PyObject *o}
Returns the \var{o} converted to a float object on success, or
\NULL{} on failure.  This is the equivalent of the Python expression
\samp{float(\var{o})}.\bifuncindex{float}
\end{cfuncdesc}


\section{Sequence Protocol \label{sequence}}

\begin{cfuncdesc}{int}{PySequence_Check}{PyObject *o}
Return \code{1} if the object provides sequence protocol, and
\code{0} otherwise.  This function always succeeds.
\end{cfuncdesc}

\begin{cfuncdesc}{int}{PySequence_Size}{PyObject *o}
Returns the number of objects in sequence \var{o} on success, and
\code{-1} on failure.  For objects that do not provide sequence
protocol, this is equivalent to the Python expression
\samp{len(\var{o})}.\bifuncindex{len}
\end{cfuncdesc}

\begin{cfuncdesc}{int}{PySequence_Length}{PyObject *o}
Alternate name for \cfunction{PySequence_Size()}.
\end{cfuncdesc}

\begin{cfuncdesc}{PyObject*}{PySequence_Concat}{PyObject *o1, PyObject *o2}
Return the concatenation of \var{o1} and \var{o2} on success, and \NULL{} on
failure.   This is the equivalent of the Python
expression \samp{\var{o1} + \var{o2}}.
\end{cfuncdesc}


\begin{cfuncdesc}{PyObject*}{PySequence_Repeat}{PyObject *o, int count}
Return the result of repeating sequence object
\var{o} \var{count} times, or \NULL{} on failure.  This is the
equivalent of the Python expression \samp{\var{o} * \var{count}}.
\end{cfuncdesc}

\begin{cfuncdesc}{PyObject*}{PySequence_InPlaceConcat}{PyObject *o1, PyObject *o2}
Return the concatenation of \var{o1} and \var{o2} on success, and \NULL{} on
failure.  The operation is done \emph{in-place} when \var{o1} supports it. 
This is the equivalent of the Python expression \samp{\var{o1} += \var{o2}}.
\end{cfuncdesc}


\begin{cfuncdesc}{PyObject*}{PySequence_InPlaceRepeat}{PyObject *o, int count}
Return the result of repeating sequence object \var{o} \var{count} times, or
\NULL{} on failure.  The operation is done \emph{in-place} when \var{o}
supports it.  This is the equivalent of the Python expression \samp{\var{o}
*= \var{count}}.
\end{cfuncdesc}


\begin{cfuncdesc}{PyObject*}{PySequence_GetItem}{PyObject *o, int i}
Return the \var{i}th element of \var{o}, or \NULL{} on failure. This
is the equivalent of the Python expression \samp{\var{o}[\var{i}]}.
\end{cfuncdesc}


\begin{cfuncdesc}{PyObject*}{PySequence_GetSlice}{PyObject *o, int i1, int i2}
Return the slice of sequence object \var{o} between \var{i1} and
\var{i2}, or \NULL{} on failure. This is the equivalent of the Python
expression \samp{\var{o}[\var{i1}:\var{i2}]}.
\end{cfuncdesc}


\begin{cfuncdesc}{int}{PySequence_SetItem}{PyObject *o, int i, PyObject *v}
Assign object \var{v} to the \var{i}th element of \var{o}.
Returns \code{-1} on failure.  This is the equivalent of the Python
statement \samp{\var{o}[\var{i}] = \var{v}}.
\end{cfuncdesc}

\begin{cfuncdesc}{int}{PySequence_DelItem}{PyObject *o, int i}
Delete the \var{i}th element of object \var{o}.  Returns
\code{-1} on failure.  This is the equivalent of the Python
statement \samp{del \var{o}[\var{i}]}.
\end{cfuncdesc}

\begin{cfuncdesc}{int}{PySequence_SetSlice}{PyObject *o, int i1,
                                            int i2, PyObject *v}
Assign the sequence object \var{v} to the slice in sequence
object \var{o} from \var{i1} to \var{i2}.  This is the equivalent of
the Python statement \samp{\var{o}[\var{i1}:\var{i2}] = \var{v}}.
\end{cfuncdesc}

\begin{cfuncdesc}{int}{PySequence_DelSlice}{PyObject *o, int i1, int i2}
Delete the slice in sequence object \var{o} from \var{i1} to \var{i2}.
Returns \code{-1} on failure. This is the equivalent of the Python
statement \samp{del \var{o}[\var{i1}:\var{i2}]}.
\end{cfuncdesc}

\begin{cfuncdesc}{PyObject*}{PySequence_Tuple}{PyObject *o}
Returns the \var{o} as a tuple on success, and \NULL{} on failure.
This is equivalent to the Python expression \samp{tuple(\var{o})}.
\bifuncindex{tuple}
\end{cfuncdesc}

\begin{cfuncdesc}{int}{PySequence_Count}{PyObject *o, PyObject *value}
Return the number of occurrences of \var{value} in \var{o}, that is,
return the number of keys for which \code{\var{o}[\var{key}] ==
\var{value}}.  On failure, return \code{-1}.  This is equivalent to
the Python expression \samp{\var{o}.count(\var{value})}.
\end{cfuncdesc}

\begin{cfuncdesc}{int}{PySequence_Contains}{PyObject *o, PyObject *value}
Determine if \var{o} contains \var{value}.  If an item in \var{o} is
equal to \var{value}, return \code{1}, otherwise return \code{0}.  On
error, return \code{-1}.  This is equivalent to the Python expression
\samp{\var{value} in \var{o}}.
\end{cfuncdesc}

\begin{cfuncdesc}{int}{PySequence_Index}{PyObject *o, PyObject *value}
Return the first index \var{i} for which \code{\var{o}[\var{i}] ==
\var{value}}.  On error, return \code{-1}.    This is equivalent to
the Python expression \samp{\var{o}.index(\var{value})}.
\end{cfuncdesc}

\begin{cfuncdesc}{PyObject*}{PySequence_List}{PyObject *o}
Return a list object with the same contents as the arbitrary sequence
\var{o}.  The returned list is guaranteed to be new.
\end{cfuncdesc}

\begin{cfuncdesc}{PyObject*}{PySequence_Tuple}{PyObject *o}
Return a tuple object with the same contents as the arbitrary sequence
\var{o}.  If \var{o} is a tuple, a new reference will be returned,
otherwise a tuple will be constructed with the appropriate contents.
\end{cfuncdesc}


\begin{cfuncdesc}{PyObject*}{PySequence_Fast}{PyObject *o, const char *m}
Returns the sequence \var{o} as a tuple, unless it is already a
tuple or list, in which case \var{o} is returned.  Use
\cfunction{PySequence_Fast_GET_ITEM()} to access the members of the
result.  Returns \NULL{} on failure.  If the object is not a sequence,
raises \exception{TypeError} with \var{m} as the message text.
\end{cfuncdesc}

\begin{cfuncdesc}{PyObject*}{PySequence_Fast_GET_ITEM}{PyObject *o, int i}
Return the \var{i}th element of \var{o}, assuming that \var{o} was
returned by \cfunction{PySequence_Fast()}, and that \var{i} is within
bounds.  The caller is expected to get the length of the sequence by
calling \cfunction{PyObject_Size()} on \var{o}, since lists and tuples
are guaranteed to always return their true length.
\end{cfuncdesc}


\section{Mapping Protocol \label{mapping}}

\begin{cfuncdesc}{int}{PyMapping_Check}{PyObject *o}
Return \code{1} if the object provides mapping protocol, and
\code{0} otherwise.  This function always succeeds.
\end{cfuncdesc}


\begin{cfuncdesc}{int}{PyMapping_Length}{PyObject *o}
Returns the number of keys in object \var{o} on success, and
\code{-1} on failure.  For objects that do not provide mapping
protocol, this is equivalent to the Python expression
\samp{len(\var{o})}.\bifuncindex{len}
\end{cfuncdesc}


\begin{cfuncdesc}{int}{PyMapping_DelItemString}{PyObject *o, char *key}
Remove the mapping for object \var{key} from the object \var{o}.
Return \code{-1} on failure.  This is equivalent to
the Python statement \samp{del \var{o}[\var{key}]}.
\end{cfuncdesc}


\begin{cfuncdesc}{int}{PyMapping_DelItem}{PyObject *o, PyObject *key}
Remove the mapping for object \var{key} from the object \var{o}.
Return \code{-1} on failure.  This is equivalent to
the Python statement \samp{del \var{o}[\var{key}]}.
\end{cfuncdesc}


\begin{cfuncdesc}{int}{PyMapping_HasKeyString}{PyObject *o, char *key}
On success, return \code{1} if the mapping object has the key
\var{key} and \code{0} otherwise.  This is equivalent to the Python
expression \samp{\var{o}.has_key(\var{key})}. 
This function always succeeds.
\end{cfuncdesc}


\begin{cfuncdesc}{int}{PyMapping_HasKey}{PyObject *o, PyObject *key}
Return \code{1} if the mapping object has the key \var{key} and
\code{0} otherwise.  This is equivalent to the Python expression
\samp{\var{o}.has_key(\var{key})}. 
This function always succeeds.
\end{cfuncdesc}


\begin{cfuncdesc}{PyObject*}{PyMapping_Keys}{PyObject *o}
On success, return a list of the keys in object \var{o}.  On
failure, return \NULL{}. This is equivalent to the Python
expression \samp{\var{o}.keys()}.
\end{cfuncdesc}


\begin{cfuncdesc}{PyObject*}{PyMapping_Values}{PyObject *o}
On success, return a list of the values in object \var{o}.  On
failure, return \NULL{}. This is equivalent to the Python
expression \samp{\var{o}.values()}.
\end{cfuncdesc}


\begin{cfuncdesc}{PyObject*}{PyMapping_Items}{PyObject *o}
On success, return a list of the items in object \var{o}, where
each item is a tuple containing a key-value pair.  On
failure, return \NULL{}. This is equivalent to the Python
expression \samp{\var{o}.items()}.
\end{cfuncdesc}


\begin{cfuncdesc}{PyObject*}{PyMapping_GetItemString}{PyObject *o, char *key}
Return element of \var{o} corresponding to the object \var{key} or
\NULL{} on failure. This is the equivalent of the Python expression
\samp{\var{o}[\var{key}]}.
\end{cfuncdesc}

\begin{cfuncdesc}{int}{PyMapping_SetItemString}{PyObject *o, char *key,
                                                PyObject *v}
Map the object \var{key} to the value \var{v} in object \var{o}.
Returns \code{-1} on failure.  This is the equivalent of the Python
statement \samp{\var{o}[\var{key}] = \var{v}}.
\end{cfuncdesc}


\section{Iterator Protocol \label{iterator}}

\versionadded{2.2}

There are only a couple of functions specifically for working with
iterators.

\begin{cfuncdesc}{int}{PyIter_Check}{PyObject *o}
  Return true if the object \var{o} supports the iterator protocol.
\end{cfuncdesc}

\begin{cfuncdesc}{PyObject*}{PyIter_Next}{PyObject *o}
  Return the next value from the iteration \var{o}.  If the object is
  an iterator, this retrieves the next value from the iteration, and
  returns \NULL{} with no exception set if there are no remaining
  items.  If the object is not an iterator, \exception{TypeError} is
  raised, or if there is an error in retrieving the item, returns
  \NULL{} and passes along the exception.
\end{cfuncdesc}

To write a loop which iterates over an iterator, the C code should
look something like this:

\begin{verbatim}
PyObject *iterator = ...;
PyObject *item;

while (item = PyIter_Next(iter)) {
    /* do something with item */
}
if (PyErr_Occurred()) {
    /* propogate error */
}
else {
    /* continue doing useful work */
}
\end{verbatim}


\chapter{Concrete Objects Layer \label{concrete}}

The functions in this chapter are specific to certain Python object
types.  Passing them an object of the wrong type is not a good idea;
if you receive an object from a Python program and you are not sure
that it has the right type, you must perform a type check first;
for example, to check that an object is a dictionary, use
\cfunction{PyDict_Check()}.  The chapter is structured like the
``family tree'' of Python object types.

\strong{Warning:}
While the functions described in this chapter carefully check the type
of the objects which are passed in, many of them do not check for
\NULL{} being passed instead of a valid object.  Allowing \NULL{} to
be passed in can cause memory access violations and immediate
termination of the interpreter.


\section{Fundamental Objects \label{fundamental}}

This section describes Python type objects and the singleton object 
\code{None}.


\subsection{Type Objects \label{typeObjects}}

\obindex{type}
\begin{ctypedesc}{PyTypeObject}
The C structure of the objects used to describe built-in types.
\end{ctypedesc}

\begin{cvardesc}{PyObject*}{PyType_Type}
This is the type object for type objects; it is the same object as
\code{types.TypeType} in the Python layer.
\withsubitem{(in module types)}{\ttindex{TypeType}}
\end{cvardesc}

\begin{cfuncdesc}{int}{PyType_Check}{PyObject *o}
Returns true is the object \var{o} is a type object.
\end{cfuncdesc}

\begin{cfuncdesc}{int}{PyType_HasFeature}{PyObject *o, int feature}
Returns true if the type object \var{o} sets the feature
\var{feature}.  Type features are denoted by single bit flags.
\end{cfuncdesc}


\subsection{The None Object \label{noneObject}}

\obindex{None@\texttt{None}}
Note that the \ctype{PyTypeObject} for \code{None} is not directly
exposed in the Python/C API.  Since \code{None} is a singleton,
testing for object identity (using \samp{==} in C) is sufficient.
There is no \cfunction{PyNone_Check()} function for the same reason.

\begin{cvardesc}{PyObject*}{Py_None}
The Python \code{None} object, denoting lack of value.  This object has
no methods.
\end{cvardesc}


\section{Sequence Objects \label{sequenceObjects}}

\obindex{sequence}
Generic operations on sequence objects were discussed in the previous 
chapter; this section deals with the specific kinds of sequence 
objects that are intrinsic to the Python language.


\subsection{String Objects \label{stringObjects}}

These functions raise \exception{TypeError} when expecting a string
parameter and are called with a non-string parameter.

\obindex{string}
\begin{ctypedesc}{PyStringObject}
This subtype of \ctype{PyObject} represents a Python string object.
\end{ctypedesc}

\begin{cvardesc}{PyTypeObject}{PyString_Type}
This instance of \ctype{PyTypeObject} represents the Python string
type; it is the same object as \code{types.TypeType} in the Python
layer.\withsubitem{(in module types)}{\ttindex{StringType}}.
\end{cvardesc}

\begin{cfuncdesc}{int}{PyString_Check}{PyObject *o}
Returns true if the object \var{o} is a string object.
\end{cfuncdesc}

\begin{cfuncdesc}{PyObject*}{PyString_FromString}{const char *v}
Returns a new string object with the value \var{v} on success, and
\NULL{} on failure.
\end{cfuncdesc}

\begin{cfuncdesc}{PyObject*}{PyString_FromStringAndSize}{const char *v,
                                                         int len}
Returns a new string object with the value \var{v} and length
\var{len} on success, and \NULL{} on failure.  If \var{v} is \NULL{},
the contents of the string are uninitialized.
\end{cfuncdesc}

\begin{cfuncdesc}{int}{PyString_Size}{PyObject *string}
Returns the length of the string in string object \var{string}.
\end{cfuncdesc}

\begin{cfuncdesc}{int}{PyString_GET_SIZE}{PyObject *string}
Macro form of \cfunction{PyString_Size()} but without error
checking.
\end{cfuncdesc}

\begin{cfuncdesc}{char*}{PyString_AsString}{PyObject *string}
Returns a null-terminated representation of the contents of
\var{string}.  The pointer refers to the internal buffer of
\var{string}, not a copy.  The data must not be modified in any way,
unless the string was just created using
\code{PyString_FromStringAndSize(NULL, \var{size})}.
It must not be deallocated.
\end{cfuncdesc}

\begin{cfuncdesc}{char*}{PyString_AS_STRING}{PyObject *string}
Macro form of \cfunction{PyString_AsString()} but without error
checking.
\end{cfuncdesc}

\begin{cfuncdesc}{int}{PyString_AsStringAndSize}{PyObject *obj,
                                                 char **buffer,
                                                 int *length}
Returns a null-terminated representation of the contents of the object
\var{obj} through the output variables \var{buffer} and \var{length}.

The function accepts both string and Unicode objects as input. For
Unicode objects it returns the default encoded version of the object.
If \var{length} is set to \NULL{}, the resulting buffer may not contain
null characters; if it does, the function returns -1 and a
TypeError is raised.

The buffer refers to an internal string buffer of \var{obj}, not a
copy. The data must not be modified in any way, unless the string was
just created using \code{PyString_FromStringAndSize(NULL,
\var{size})}.  It must not be deallocated.
\end{cfuncdesc}

\begin{cfuncdesc}{void}{PyString_Concat}{PyObject **string,
                                         PyObject *newpart}
Creates a new string object in \var{*string} containing the
contents of \var{newpart} appended to \var{string}; the caller will
own the new reference.  The reference to the old value of \var{string}
will be stolen.  If the new string
cannot be created, the old reference to \var{string} will still be
discarded and the value of \var{*string} will be set to
\NULL{}; the appropriate exception will be set.
\end{cfuncdesc}

\begin{cfuncdesc}{void}{PyString_ConcatAndDel}{PyObject **string,
                                               PyObject *newpart}
Creates a new string object in \var{*string} containing the contents
of \var{newpart} appended to \var{string}.  This version decrements
the reference count of \var{newpart}.
\end{cfuncdesc}

\begin{cfuncdesc}{int}{_PyString_Resize}{PyObject **string, int newsize}
A way to resize a string object even though it is ``immutable''.  
Only use this to build up a brand new string object; don't use this if
the string may already be known in other parts of the code.
\end{cfuncdesc}

\begin{cfuncdesc}{PyObject*}{PyString_Format}{PyObject *format,
                                              PyObject *args}
Returns a new string object from \var{format} and \var{args}.  Analogous
to \code{\var{format} \%\ \var{args}}.  The \var{args} argument must be
a tuple.
\end{cfuncdesc}

\begin{cfuncdesc}{void}{PyString_InternInPlace}{PyObject **string}
Intern the argument \var{*string} in place.  The argument must be the
address of a pointer variable pointing to a Python string object.
If there is an existing interned string that is the same as
\var{*string}, it sets \var{*string} to it (decrementing the reference 
count of the old string object and incrementing the reference count of
the interned string object), otherwise it leaves \var{*string} alone
and interns it (incrementing its reference count).  (Clarification:
even though there is a lot of talk about reference counts, think of
this function as reference-count-neutral; you own the object after
the call if and only if you owned it before the call.)
\end{cfuncdesc}

\begin{cfuncdesc}{PyObject*}{PyString_InternFromString}{const char *v}
A combination of \cfunction{PyString_FromString()} and
\cfunction{PyString_InternInPlace()}, returning either a new string object
that has been interned, or a new (``owned'') reference to an earlier
interned string object with the same value.
\end{cfuncdesc}

\begin{cfuncdesc}{PyObject*}{PyString_Decode}{const char *s,
                                               int size,
                                               const char *encoding,
                                               const char *errors}
Creates an object by decoding \var{size} bytes of the encoded
buffer \var{s} using the codec registered
for \var{encoding}. \var{encoding} and \var{errors} have the same meaning
as the parameters of the same name in the unicode() builtin
function. The codec to be used is looked up using the Python codec
registry. Returns \NULL{} in case an exception was raised by the
codec.
\end{cfuncdesc}

\begin{cfuncdesc}{PyObject*}{PyString_AsDecodedObject}{PyObject *str,
                                               const char *encoding,
                                               const char *errors}
Decodes a string object by passing it to the codec registered
for \var{encoding} and returns the result as Python 
object. \var{encoding} and \var{errors} have the same meaning as the
parameters of the same name in the string .encode() method. The codec
to be used is looked up using the Python codec registry. Returns
\NULL{} in case an exception was raised by the codec.
\end{cfuncdesc}

\begin{cfuncdesc}{PyObject*}{PyString_Encode}{const char *s,
                                               int size,
                                               const char *encoding,
                                               const char *errors}
Encodes the \ctype{char} buffer of the given size by passing it to 
the codec registered for \var{encoding} and returns a Python object. 
\var{encoding} and \var{errors} have the same
meaning as the parameters of the same name in the string .encode()
method. The codec to be used is looked up using the Python codec
registry. Returns \NULL{} in case an exception was raised by the
codec.
\end{cfuncdesc}

\begin{cfuncdesc}{PyObject*}{PyString_AsEncodedObject}{PyObject *str,
                                               const char *encoding,
                                               const char *errors}
Encodes a string object using the codec registered
for \var{encoding} and returns the result as Python 
object. \var{encoding} and \var{errors} have the same meaning as the
parameters of the same name in the string .encode() method. The codec
to be used is looked up using the Python codec registry. Returns
\NULL{} in case an exception was raised by the codec.
\end{cfuncdesc}


\subsection{Unicode Objects \label{unicodeObjects}}
\sectionauthor{Marc-Andre Lemburg}{mal@lemburg.com}

%--- Unicode Type -------------------------------------------------------

These are the basic Unicode object types used for the Unicode
implementation in Python:

\begin{ctypedesc}{Py_UNICODE}
This type represents a 16-bit unsigned storage type which is used by
Python internally as basis for holding Unicode ordinals. On platforms
where \ctype{wchar_t} is available and also has 16-bits,
\ctype{Py_UNICODE} is a typedef alias for \ctype{wchar_t} to enhance
native platform compatibility. On all other platforms,
\ctype{Py_UNICODE} is a typedef alias for \ctype{unsigned short}.
\end{ctypedesc}

\begin{ctypedesc}{PyUnicodeObject}
This subtype of \ctype{PyObject} represents a Python Unicode object.
\end{ctypedesc}

\begin{cvardesc}{PyTypeObject}{PyUnicode_Type}
This instance of \ctype{PyTypeObject} represents the Python Unicode type.
\end{cvardesc}

%--- These are really C macros... is there a macrodesc TeX macro ?

The following APIs are really C macros and can be used to do fast
checks and to access internal read-only data of Unicode objects:

\begin{cfuncdesc}{int}{PyUnicode_Check}{PyObject *o}
Returns true if the object \var{o} is a Unicode object.
\end{cfuncdesc}

\begin{cfuncdesc}{int}{PyUnicode_GET_SIZE}{PyObject *o}
Returns the size of the object.  o has to be a
PyUnicodeObject (not checked).
\end{cfuncdesc}

\begin{cfuncdesc}{int}{PyUnicode_GET_DATA_SIZE}{PyObject *o}
Returns the size of the object's internal buffer in bytes. o has to be
a PyUnicodeObject (not checked).
\end{cfuncdesc}

\begin{cfuncdesc}{Py_UNICODE*}{PyUnicode_AS_UNICODE}{PyObject *o}
Returns a pointer to the internal Py_UNICODE buffer of the object. o
has to be a PyUnicodeObject (not checked).
\end{cfuncdesc}

\begin{cfuncdesc}{const char*}{PyUnicode_AS_DATA}{PyObject *o}
Returns a (const char *) pointer to the internal buffer of the object.
o has to be a PyUnicodeObject (not checked).
\end{cfuncdesc}

% --- Unicode character properties ---------------------------------------

Unicode provides many different character properties. The most often
needed ones are available through these macros which are mapped to C
functions depending on the Python configuration.

\begin{cfuncdesc}{int}{Py_UNICODE_ISSPACE}{Py_UNICODE ch}
Returns 1/0 depending on whether \var{ch} is a whitespace character.
\end{cfuncdesc}

\begin{cfuncdesc}{int}{Py_UNICODE_ISLOWER}{Py_UNICODE ch}
Returns 1/0 depending on whether \var{ch} is a lowercase character.
\end{cfuncdesc}

\begin{cfuncdesc}{int}{Py_UNICODE_ISUPPER}{Py_UNICODE ch}
Returns 1/0 depending on whether \var{ch} is an uppercase character.
\end{cfuncdesc}

\begin{cfuncdesc}{int}{Py_UNICODE_ISTITLE}{Py_UNICODE ch}
Returns 1/0 depending on whether \var{ch} is a titlecase character.
\end{cfuncdesc}

\begin{cfuncdesc}{int}{Py_UNICODE_ISLINEBREAK}{Py_UNICODE ch}
Returns 1/0 depending on whether \var{ch} is a linebreak character.
\end{cfuncdesc}

\begin{cfuncdesc}{int}{Py_UNICODE_ISDECIMAL}{Py_UNICODE ch}
Returns 1/0 depending on whether \var{ch} is a decimal character.
\end{cfuncdesc}

\begin{cfuncdesc}{int}{Py_UNICODE_ISDIGIT}{Py_UNICODE ch}
Returns 1/0 depending on whether \var{ch} is a digit character.
\end{cfuncdesc}

\begin{cfuncdesc}{int}{Py_UNICODE_ISNUMERIC}{Py_UNICODE ch}
Returns 1/0 depending on whether \var{ch} is a numeric character.
\end{cfuncdesc}

\begin{cfuncdesc}{int}{Py_UNICODE_ISALPHA}{Py_UNICODE ch}
Returns 1/0 depending on whether \var{ch} is an alphabetic character.
\end{cfuncdesc}

\begin{cfuncdesc}{int}{Py_UNICODE_ISALNUM}{Py_UNICODE ch}
Returns 1/0 depending on whether \var{ch} is an alphanumeric character.
\end{cfuncdesc}

These APIs can be used for fast direct character conversions:

\begin{cfuncdesc}{Py_UNICODE}{Py_UNICODE_TOLOWER}{Py_UNICODE ch}
Returns the character \var{ch} converted to lower case.
\end{cfuncdesc}

\begin{cfuncdesc}{Py_UNICODE}{Py_UNICODE_TOUPPER}{Py_UNICODE ch}
Returns the character \var{ch} converted to upper case.
\end{cfuncdesc}

\begin{cfuncdesc}{Py_UNICODE}{Py_UNICODE_TOTITLE}{Py_UNICODE ch}
Returns the character \var{ch} converted to title case.
\end{cfuncdesc}

\begin{cfuncdesc}{int}{Py_UNICODE_TODECIMAL}{Py_UNICODE ch}
Returns the character \var{ch} converted to a decimal positive integer.
Returns -1 in case this is not possible. Does not raise exceptions.
\end{cfuncdesc}

\begin{cfuncdesc}{int}{Py_UNICODE_TODIGIT}{Py_UNICODE ch}
Returns the character \var{ch} converted to a single digit integer.
Returns -1 in case this is not possible. Does not raise exceptions.
\end{cfuncdesc}

\begin{cfuncdesc}{double}{Py_UNICODE_TONUMERIC}{Py_UNICODE ch}
Returns the character \var{ch} converted to a (positive) double.
Returns -1.0 in case this is not possible. Does not raise exceptions.
\end{cfuncdesc}

% --- Plain Py_UNICODE ---------------------------------------------------

To create Unicode objects and access their basic sequence properties,
use these APIs:

\begin{cfuncdesc}{PyObject*}{PyUnicode_FromUnicode}{const Py_UNICODE *u,
                                                    int size} 

Create a Unicode Object from the Py_UNICODE buffer \var{u} of the
given size. \var{u} may be \NULL{} which causes the contents to be
undefined. It is the user's responsibility to fill in the needed data.
The buffer is copied into the new object. If the buffer is not \NULL{},
the return value might be a shared object. Therefore, modification of
the resulting Unicode Object is only allowed when \var{u} is \NULL{}.
\end{cfuncdesc}

\begin{cfuncdesc}{Py_UNICODE*}{PyUnicode_AsUnicode}{PyObject *unicode}
Return a read-only pointer to the Unicode object's internal
\ctype{Py_UNICODE} buffer.
\end{cfuncdesc}

\begin{cfuncdesc}{int}{PyUnicode_GetSize}{PyObject *unicode}
Return the length of the Unicode object.
\end{cfuncdesc}

\begin{cfuncdesc}{PyObject*}{PyUnicode_FromEncodedObject}{PyObject *obj,
                                                      const char *encoding,
                                                      const char *errors}

Coerce an encoded object obj to an Unicode object and return a
reference with incremented refcount.

Coercion is done in the following way:
\begin{enumerate}
\item  Unicode objects are passed back as-is with incremented
      refcount. Note: these cannot be decoded; passing a non-NULL
      value for encoding will result in a TypeError.

\item String and other char buffer compatible objects are decoded
      according to the given encoding and using the error handling
      defined by errors. Both can be NULL to have the interface use
      the default values (see the next section for details).

\item All other objects cause an exception.
\end{enumerate}
The API returns NULL in case of an error. The caller is responsible
for decref'ing the returned objects.
\end{cfuncdesc}

\begin{cfuncdesc}{PyObject*}{PyUnicode_FromObject}{PyObject *obj}

Shortcut for PyUnicode_FromEncodedObject(obj, NULL, ``strict'')
which is used throughout the interpreter whenever coercion to
Unicode is needed.
\end{cfuncdesc}

% --- wchar_t support for platforms which support it ---------------------

If the platform supports \ctype{wchar_t} and provides a header file
wchar.h, Python can interface directly to this type using the
following functions. Support is optimized if Python's own
\ctype{Py_UNICODE} type is identical to the system's \ctype{wchar_t}.

\begin{cfuncdesc}{PyObject*}{PyUnicode_FromWideChar}{const wchar_t *w,
                                                     int size}
Create a Unicode Object from the \ctype{whcar_t} buffer \var{w} of the
given size. Returns \NULL{} on failure.
\end{cfuncdesc}

\begin{cfuncdesc}{int}{PyUnicode_AsWideChar}{PyUnicodeObject *unicode,
                                             wchar_t *w,
                                             int size}
Copies the Unicode Object contents into the \ctype{whcar_t} buffer
\var{w}.  At most \var{size} \ctype{whcar_t} characters are copied.
Returns the number of \ctype{whcar_t} characters copied or -1 in case
of an error.
\end{cfuncdesc}


\subsubsection{Builtin Codecs \label{builtinCodecs}}

Python provides a set of builtin codecs which are written in C
for speed. All of these codecs are directly usable via the
following functions.

Many of the following APIs take two arguments encoding and
errors. These parameters encoding and errors have the same semantics
as the ones of the builtin unicode() Unicode object constructor.

Setting encoding to NULL causes the default encoding to be used which
is UTF-8.

Error handling is set by errors which may also be set to NULL meaning
to use the default handling defined for the codec. Default error
handling for all builtin codecs is ``strict'' (ValueErrors are raised).

The codecs all use a similar interface. Only deviation from the
following generic ones are documented for simplicity.

% --- Generic Codecs -----------------------------------------------------

These are the generic codec APIs:

\begin{cfuncdesc}{PyObject*}{PyUnicode_Decode}{const char *s,
                                               int size,
                                               const char *encoding,
                                               const char *errors}
Create a Unicode object by decoding \var{size} bytes of the encoded
string \var{s}. \var{encoding} and \var{errors} have the same meaning
as the parameters of the same name in the unicode() builtin
function. The codec to be used is looked up using the Python codec
registry. Returns \NULL{} in case an exception was raised by the
codec.
\end{cfuncdesc}

\begin{cfuncdesc}{PyObject*}{PyUnicode_Encode}{const Py_UNICODE *s,
                                               int size,
                                               const char *encoding,
                                               const char *errors}
Encodes the \ctype{Py_UNICODE} buffer of the given size and returns a
Python string object. \var{encoding} and \var{errors} have the same
meaning as the parameters of the same name in the Unicode .encode()
method. The codec to be used is looked up using the Python codec
registry. Returns \NULL{} in case an exception was raised by the
codec.
\end{cfuncdesc}

\begin{cfuncdesc}{PyObject*}{PyUnicode_AsEncodedString}{PyObject *unicode,
                                               const char *encoding,
                                               const char *errors}
Encodes a Unicode object and returns the result as Python string
object. \var{encoding} and \var{errors} have the same meaning as the
parameters of the same name in the Unicode .encode() method. The codec
to be used is looked up using the Python codec registry. Returns
\NULL{} in case an exception was raised by the codec.
\end{cfuncdesc}

% --- UTF-8 Codecs -------------------------------------------------------

These are the UTF-8 codec APIs:

\begin{cfuncdesc}{PyObject*}{PyUnicode_DecodeUTF8}{const char *s,
                                               int size,
                                               const char *errors}
Creates a Unicode object by decoding \var{size} bytes of the UTF-8
encoded string \var{s}. Returns \NULL{} in case an exception was
raised by the codec.
\end{cfuncdesc}

\begin{cfuncdesc}{PyObject*}{PyUnicode_EncodeUTF8}{const Py_UNICODE *s,
                                               int size,
                                               const char *errors}
Encodes the \ctype{Py_UNICODE} buffer of the given size using UTF-8
and returns a Python string object.  Returns \NULL{} in case an
exception was raised by the codec.
\end{cfuncdesc}

\begin{cfuncdesc}{PyObject*}{PyUnicode_AsUTF8String}{PyObject *unicode}
Encodes a Unicode objects using UTF-8 and returns the result as Python
string object. Error handling is ``strict''. Returns
\NULL{} in case an exception was raised by the codec.
\end{cfuncdesc}

% --- UTF-16 Codecs ------------------------------------------------------ */

These are the UTF-16 codec APIs:

\begin{cfuncdesc}{PyObject*}{PyUnicode_DecodeUTF16}{const char *s,
                                               int size,
                                               const char *errors,
                                               int *byteorder}
Decodes \var{length} bytes from a UTF-16 encoded buffer string and
returns the corresponding Unicode object.

\var{errors} (if non-NULL) defines the error handling. It defaults
to ``strict''.

If \var{byteorder} is non-\NULL{}, the decoder starts decoding using
the given byte order:

\begin{verbatim}
   *byteorder == -1: little endian
   *byteorder == 0:  native order
   *byteorder == 1:  big endian
\end{verbatim}

and then switches according to all byte order marks (BOM) it finds in
the input data. BOM marks are not copied into the resulting Unicode
string.  After completion, \var{*byteorder} is set to the current byte
order at the end of input data.

If \var{byteorder} is \NULL{}, the codec starts in native order mode.

Returns \NULL{} in case an exception was raised by the codec.
\end{cfuncdesc}

\begin{cfuncdesc}{PyObject*}{PyUnicode_EncodeUTF16}{const Py_UNICODE *s,
                                               int size,
                                               const char *errors,
                                               int byteorder}
Returns a Python string object holding the UTF-16 encoded value of the
Unicode data in \var{s}.

If \var{byteorder} is not \code{0}, output is written according to the
following byte order:

\begin{verbatim}
   byteorder == -1: little endian
   byteorder == 0:  native byte order (writes a BOM mark)
   byteorder == 1:  big endian
\end{verbatim}

If byteorder is \code{0}, the output string will always start with the
Unicode BOM mark (U+FEFF). In the other two modes, no BOM mark is
prepended.

Note that \ctype{Py_UNICODE} data is being interpreted as UTF-16
reduced to UCS-2. This trick makes it possible to add full UTF-16
capabilities at a later point without comprimising the APIs.

Returns \NULL{} in case an exception was raised by the codec.
\end{cfuncdesc}

\begin{cfuncdesc}{PyObject*}{PyUnicode_AsUTF16String}{PyObject *unicode}
Returns a Python string using the UTF-16 encoding in native byte
order. The string always starts with a BOM mark. Error handling is
``strict''. Returns \NULL{} in case an exception was raised by the
codec.
\end{cfuncdesc}

% --- Unicode-Escape Codecs ----------------------------------------------

These are the ``Unicode Esacpe'' codec APIs:

\begin{cfuncdesc}{PyObject*}{PyUnicode_DecodeUnicodeEscape}{const char *s,
                                               int size,
                                               const char *errors}
Creates a Unicode object by decoding \var{size} bytes of the Unicode-Esacpe
encoded string \var{s}. Returns \NULL{} in case an exception was
raised by the codec.
\end{cfuncdesc}

\begin{cfuncdesc}{PyObject*}{PyUnicode_EncodeUnicodeEscape}{const Py_UNICODE *s,
                                               int size,
                                               const char *errors}
Encodes the \ctype{Py_UNICODE} buffer of the given size using Unicode-Escape
and returns a Python string object.  Returns \NULL{} in case an
exception was raised by the codec.
\end{cfuncdesc}

\begin{cfuncdesc}{PyObject*}{PyUnicode_AsUnicodeEscapeString}{PyObject *unicode}
Encodes a Unicode objects using Unicode-Escape and returns the result
as Python string object. Error handling is ``strict''. Returns
\NULL{} in case an exception was raised by the codec.
\end{cfuncdesc}

% --- Raw-Unicode-Escape Codecs ------------------------------------------

These are the ``Raw Unicode Esacpe'' codec APIs:

\begin{cfuncdesc}{PyObject*}{PyUnicode_DecodeRawUnicodeEscape}{const char *s,
                                               int size,
                                               const char *errors}
Creates a Unicode object by decoding \var{size} bytes of the Raw-Unicode-Esacpe
encoded string \var{s}. Returns \NULL{} in case an exception was
raised by the codec.
\end{cfuncdesc}

\begin{cfuncdesc}{PyObject*}{PyUnicode_EncodeRawUnicodeEscape}{const Py_UNICODE *s,
                                               int size,
                                               const char *errors}
Encodes the \ctype{Py_UNICODE} buffer of the given size using Raw-Unicode-Escape
and returns a Python string object.  Returns \NULL{} in case an
exception was raised by the codec.
\end{cfuncdesc}

\begin{cfuncdesc}{PyObject*}{PyUnicode_AsRawUnicodeEscapeString}{PyObject *unicode}
Encodes a Unicode objects using Raw-Unicode-Escape and returns the result
as Python string object. Error handling is ``strict''. Returns
\NULL{} in case an exception was raised by the codec.
\end{cfuncdesc}

% --- Latin-1 Codecs ----------------------------------------------------- 

These are the Latin-1 codec APIs:

Latin-1 corresponds to the first 256 Unicode ordinals and only these
are accepted by the codecs during encoding.

\begin{cfuncdesc}{PyObject*}{PyUnicode_DecodeLatin1}{const char *s,
                                                     int size,
                                                     const char *errors}
Creates a Unicode object by decoding \var{size} bytes of the Latin-1
encoded string \var{s}. Returns \NULL{} in case an exception was
raised by the codec.
\end{cfuncdesc}

\begin{cfuncdesc}{PyObject*}{PyUnicode_EncodeLatin1}{const Py_UNICODE *s,
                                                     int size,
                                                     const char *errors}
Encodes the \ctype{Py_UNICODE} buffer of the given size using Latin-1
and returns a Python string object.  Returns \NULL{} in case an
exception was raised by the codec.
\end{cfuncdesc}

\begin{cfuncdesc}{PyObject*}{PyUnicode_AsLatin1String}{PyObject *unicode}
Encodes a Unicode objects using Latin-1 and returns the result as
Python string object. Error handling is ``strict''. Returns
\NULL{} in case an exception was raised by the codec.
\end{cfuncdesc}

% --- ASCII Codecs ------------------------------------------------------- 

These are the \ASCII{} codec APIs.  Only 7-bit \ASCII{} data is
accepted. All other codes generate errors.

\begin{cfuncdesc}{PyObject*}{PyUnicode_DecodeASCII}{const char *s,
                                                    int size,
                                                    const char *errors}
Creates a Unicode object by decoding \var{size} bytes of the
\ASCII{} encoded string \var{s}. Returns \NULL{} in case an exception
was raised by the codec.
\end{cfuncdesc}

\begin{cfuncdesc}{PyObject*}{PyUnicode_EncodeASCII}{const Py_UNICODE *s,
                                                    int size,
                                                    const char *errors}
Encodes the \ctype{Py_UNICODE} buffer of the given size using
\ASCII{} and returns a Python string object.  Returns \NULL{} in case
an exception was raised by the codec.
\end{cfuncdesc}

\begin{cfuncdesc}{PyObject*}{PyUnicode_AsASCIIString}{PyObject *unicode}
Encodes a Unicode objects using \ASCII{} and returns the result as Python
string object. Error handling is ``strict''. Returns
\NULL{} in case an exception was raised by the codec.
\end{cfuncdesc}

% --- Character Map Codecs ----------------------------------------------- 

These are the mapping codec APIs:

This codec is special in that it can be used to implement many
different codecs (and this is in fact what was done to obtain most of
the standard codecs included in the \module{encodings} package). The
codec uses mapping to encode and decode characters.

Decoding mappings must map single string characters to single Unicode
characters, integers (which are then interpreted as Unicode ordinals)
or None (meaning "undefined mapping" and causing an error). 

Encoding mappings must map single Unicode characters to single string
characters, integers (which are then interpreted as Latin-1 ordinals)
or None (meaning "undefined mapping" and causing an error).

The mapping objects provided must only support the __getitem__ mapping
interface.

If a character lookup fails with a LookupError, the character is
copied as-is meaning that its ordinal value will be interpreted as
Unicode or Latin-1 ordinal resp. Because of this, mappings only need
to contain those mappings which map characters to different code
points.

\begin{cfuncdesc}{PyObject*}{PyUnicode_DecodeCharmap}{const char *s,
                                               int size,
                                               PyObject *mapping,
                                               const char *errors}
Creates a Unicode object by decoding \var{size} bytes of the encoded
string \var{s} using the given \var{mapping} object.  Returns \NULL{}
in case an exception was raised by the codec.
\end{cfuncdesc}

\begin{cfuncdesc}{PyObject*}{PyUnicode_EncodeCharmap}{const Py_UNICODE *s,
                                               int size,
                                               PyObject *mapping,
                                               const char *errors}
Encodes the \ctype{Py_UNICODE} buffer of the given size using the
given \var{mapping} object and returns a Python string object.
Returns \NULL{} in case an exception was raised by the codec.
\end{cfuncdesc}

\begin{cfuncdesc}{PyObject*}{PyUnicode_AsCharmapString}{PyObject *unicode,
                                                        PyObject *mapping}
Encodes a Unicode objects using the given \var{mapping} object and
returns the result as Python string object. Error handling is
``strict''. Returns \NULL{} in case an exception was raised by the
codec.
\end{cfuncdesc}

The following codec API is special in that maps Unicode to Unicode.

\begin{cfuncdesc}{PyObject*}{PyUnicode_TranslateCharmap}{const Py_UNICODE *s,
                                               int size,
                                               PyObject *table,
                                               const char *errors}
Translates a \ctype{Py_UNICODE} buffer of the given length by applying
a character mapping \var{table} to it and returns the resulting
Unicode object.  Returns \NULL{} when an exception was raised by the
codec.

The \var{mapping} table must map Unicode ordinal integers to Unicode
ordinal integers or None (causing deletion of the character).

Mapping tables must only provide the __getitem__ interface,
e.g. dictionaries or sequences. Unmapped character ordinals (ones
which cause a LookupError) are left untouched and are copied as-is.
\end{cfuncdesc}

% --- MBCS codecs for Windows --------------------------------------------

These are the MBCS codec APIs. They are currently only available on
Windows and use the Win32 MBCS converters to implement the
conversions.  Note that MBCS (or DBCS) is a class of encodings, not
just one.  The target encoding is defined by the user settings on the
machine running the codec.

\begin{cfuncdesc}{PyObject*}{PyUnicode_DecodeMBCS}{const char *s,
                                               int size,
                                               const char *errors}
Creates a Unicode object by decoding \var{size} bytes of the MBCS
encoded string \var{s}.  Returns \NULL{} in case an exception was
raised by the codec.
\end{cfuncdesc}

\begin{cfuncdesc}{PyObject*}{PyUnicode_EncodeMBCS}{const Py_UNICODE *s,
                                               int size,
                                               const char *errors}
Encodes the \ctype{Py_UNICODE} buffer of the given size using MBCS
and returns a Python string object.  Returns \NULL{} in case an
exception was raised by the codec.
\end{cfuncdesc}

\begin{cfuncdesc}{PyObject*}{PyUnicode_AsMBCSString}{PyObject *unicode}
Encodes a Unicode objects using MBCS and returns the result as Python
string object.  Error handling is ``strict''.  Returns \NULL{} in case
an exception was raised by the codec.
\end{cfuncdesc}

% --- Methods & Slots ----------------------------------------------------

\subsubsection{Methods and Slot Functions \label{unicodeMethodsAndSlots}}

The following APIs are capable of handling Unicode objects and strings
on input (we refer to them as strings in the descriptions) and return
Unicode objects or integers as apporpriate.

They all return \NULL{} or -1 in case an exception occurrs.

\begin{cfuncdesc}{PyObject*}{PyUnicode_Concat}{PyObject *left,
                                               PyObject *right}
Concat two strings giving a new Unicode string.
\end{cfuncdesc}

\begin{cfuncdesc}{PyObject*}{PyUnicode_Split}{PyObject *s,
                                              PyObject *sep,
                                              int maxsplit}
Split a string giving a list of Unicode strings.

If sep is NULL, splitting will be done at all whitespace
substrings. Otherwise, splits occur at the given separator.

At most maxsplit splits will be done. If negative, no limit is set.

Separators are not included in the resulting list.
\end{cfuncdesc}

\begin{cfuncdesc}{PyObject*}{PyUnicode_Splitlines}{PyObject *s,
                                                   int maxsplit}
Split a Unicode string at line breaks, returning a list of Unicode
strings.  CRLF is considered to be one line break.  The Line break
characters are not included in the resulting strings.
\end{cfuncdesc}

\begin{cfuncdesc}{PyObject*}{PyUnicode_Translate}{PyObject *str,
                                                  PyObject *table,
                                                  const char *errors}
Translate a string by applying a character mapping table to it and
return the resulting Unicode object.

The mapping table must map Unicode ordinal integers to Unicode ordinal
integers or None (causing deletion of the character).

Mapping tables must only provide the __getitem__ interface,
e.g. dictionaries or sequences. Unmapped character ordinals (ones
which cause a LookupError) are left untouched and are copied as-is.

\var{errors} has the usual meaning for codecs. It may be \NULL{}
which indicates to use the default error handling.
\end{cfuncdesc}

\begin{cfuncdesc}{PyObject*}{PyUnicode_Join}{PyObject *separator,
                                             PyObject *seq}
Join a sequence of strings using the given separator and return
the resulting Unicode string.
\end{cfuncdesc}

\begin{cfuncdesc}{PyObject*}{PyUnicode_Tailmatch}{PyObject *str,
                                                  PyObject *substr,
                                                  int start,
                                                  int end,
                                                  int direction}
Return 1 if \var{substr} matches \var{str}[\var{start}:\var{end}] at
the given tail end (\var{direction} == -1 means to do a prefix match,
\var{direction} == 1 a suffix match), 0 otherwise.
\end{cfuncdesc}

\begin{cfuncdesc}{PyObject*}{PyUnicode_Find}{PyObject *str,
                                                  PyObject *substr,
                                                  int start,
                                                  int end,
                                                  int direction}
Return the first position of \var{substr} in
\var{str}[\var{start}:\var{end}] using the given \var{direction}
(\var{direction} == 1 means to do a forward search,
\var{direction} == -1 a backward search), 0 otherwise.
\end{cfuncdesc}

\begin{cfuncdesc}{PyObject*}{PyUnicode_Count}{PyObject *str,
                                                  PyObject *substr,
                                                  int start,
                                                  int end}
Count the number of occurrences of \var{substr} in
\var{str}[\var{start}:\var{end}]
\end{cfuncdesc}

\begin{cfuncdesc}{PyObject*}{PyUnicode_Replace}{PyObject *str,
                                                PyObject *substr,
                                                PyObject *replstr,
                                                int maxcount}
Replace at most \var{maxcount} occurrences of \var{substr} in
\var{str} with \var{replstr} and return the resulting Unicode object.
\var{maxcount} == -1 means: replace all occurrences.
\end{cfuncdesc}

\begin{cfuncdesc}{int}{PyUnicode_Compare}{PyObject *left, PyObject *right}
Compare two strings and return -1, 0, 1 for less than, equal,
greater than resp.
\end{cfuncdesc}

\begin{cfuncdesc}{PyObject*}{PyUnicode_Format}{PyObject *format,
                                              PyObject *args}
Returns a new string object from \var{format} and \var{args}; this is
analogous to \code{\var{format} \%\ \var{args}}.  The
\var{args} argument must be a tuple.
\end{cfuncdesc}

\begin{cfuncdesc}{int}{PyUnicode_Contains}{PyObject *container,
                                           PyObject *element}
Checks whether \var{element} is contained in \var{container} and
returns true or false accordingly.

\var{element} has to coerce to a one element Unicode string. \code{-1} is
returned in case of an error.
\end{cfuncdesc}


\subsection{Buffer Objects \label{bufferObjects}}
\sectionauthor{Greg Stein}{gstein@lyra.org}

\obindex{buffer}
Python objects implemented in C can export a group of functions called
the ``buffer\index{buffer interface} interface.''  These functions can
be used by an object to expose its data in a raw, byte-oriented
format. Clients of the object can use the buffer interface to access
the object data directly, without needing to copy it first.

Two examples of objects that support 
the buffer interface are strings and arrays. The string object exposes 
the character contents in the buffer interface's byte-oriented
form. An array can also expose its contents, but it should be noted
that array elements may be multi-byte values.

An example user of the buffer interface is the file object's
\method{write()} method. Any object that can export a series of bytes
through the buffer interface can be written to a file. There are a
number of format codes to \cfunction{PyArgs_ParseTuple()} that operate 
against an object's buffer interface, returning data from the target
object.

More information on the buffer interface is provided in the section
``Buffer Object Structures'' (section \ref{buffer-structs}), under
the description for \ctype{PyBufferProcs}\ttindex{PyBufferProcs}.

A ``buffer object'' is defined in the \file{bufferobject.h} header
(included by \file{Python.h}). These objects look very similar to
string objects at the Python programming level: they support slicing,
indexing, concatenation, and some other standard string
operations. However, their data can come from one of two sources: from
a block of memory, or from another object which exports the buffer
interface.

Buffer objects are useful as a way to expose the data from another
object's buffer interface to the Python programmer. They can also be
used as a zero-copy slicing mechanism. Using their ability to
reference a block of memory, it is possible to expose any data to the
Python programmer quite easily. The memory could be a large, constant
array in a C extension, it could be a raw block of memory for
manipulation before passing to an operating system library, or it
could be used to pass around structured data in its native, in-memory
format.

\begin{ctypedesc}{PyBufferObject}
This subtype of \ctype{PyObject} represents a buffer object.
\end{ctypedesc}

\begin{cvardesc}{PyTypeObject}{PyBuffer_Type}
The instance of \ctype{PyTypeObject} which represents the Python
buffer type; it is the same object as \code{types.BufferType} in the
Python layer.\withsubitem{(in module types)}{\ttindex{BufferType}}.
\end{cvardesc}

\begin{cvardesc}{int}{Py_END_OF_BUFFER}
This constant may be passed as the \var{size} parameter to
\cfunction{PyBuffer_FromObject()} or
\cfunction{PyBuffer_FromReadWriteObject()}. It indicates that the new
\ctype{PyBufferObject} should refer to \var{base} object from the
specified \var{offset} to the end of its exported buffer. Using this
enables the caller to avoid querying the \var{base} object for its
length.
\end{cvardesc}

\begin{cfuncdesc}{int}{PyBuffer_Check}{PyObject *p}
Return true if the argument has type \cdata{PyBuffer_Type}.
\end{cfuncdesc}

\begin{cfuncdesc}{PyObject*}{PyBuffer_FromObject}{PyObject *base,
                                                  int offset, int size}
Return a new read-only buffer object.  This raises
\exception{TypeError} if \var{base} doesn't support the read-only
buffer protocol or doesn't provide exactly one buffer segment, or it
raises \exception{ValueError} if \var{offset} is less than zero. The
buffer will hold a reference to the \var{base} object, and the
buffer's contents will refer to the \var{base} object's buffer
interface, starting as position \var{offset} and extending for
\var{size} bytes. If \var{size} is \constant{Py_END_OF_BUFFER}, then
the new buffer's contents extend to the length of the
\var{base} object's exported buffer data.
\end{cfuncdesc}

\begin{cfuncdesc}{PyObject*}{PyBuffer_FromReadWriteObject}{PyObject *base,
                                                           int offset,
                                                           int size}
Return a new writable buffer object.  Parameters and exceptions are
similar to those for \cfunction{PyBuffer_FromObject()}.
If the \var{base} object does not export the writeable buffer
protocol, then \exception{TypeError} is raised.
\end{cfuncdesc}

\begin{cfuncdesc}{PyObject*}{PyBuffer_FromMemory}{void *ptr, int size}
Return a new read-only buffer object that reads from a specified
location in memory, with a specified size.
The caller is responsible for ensuring that the memory buffer, passed
in as \var{ptr}, is not deallocated while the returned buffer object
exists.  Raises \exception{ValueError} if \var{size} is less than
zero.  Note that \constant{Py_END_OF_BUFFER} may \emph{not} be passed
for the \var{size} parameter; \exception{ValueError} will be raised in 
that case.
\end{cfuncdesc}

\begin{cfuncdesc}{PyObject*}{PyBuffer_FromReadWriteMemory}{void *ptr, int size}
Similar to \cfunction{PyBuffer_FromMemory()}, but the returned buffer
is writable.
\end{cfuncdesc}

\begin{cfuncdesc}{PyObject*}{PyBuffer_New}{int size}
Returns a new writable buffer object that maintains its own memory
buffer of \var{size} bytes.  \exception{ValueError} is returned if
\var{size} is not zero or positive.
\end{cfuncdesc}


\subsection{Tuple Objects \label{tupleObjects}}

\obindex{tuple}
\begin{ctypedesc}{PyTupleObject}
This subtype of \ctype{PyObject} represents a Python tuple object.
\end{ctypedesc}

\begin{cvardesc}{PyTypeObject}{PyTuple_Type}
This instance of \ctype{PyTypeObject} represents the Python tuple
type; it is the same object as \code{types.TupleType} in the Python
layer.\withsubitem{(in module types)}{\ttindex{TupleType}}.
\end{cvardesc}

\begin{cfuncdesc}{int}{PyTuple_Check}{PyObject *p}
Return true if the argument is a tuple object.
\end{cfuncdesc}

\begin{cfuncdesc}{PyObject*}{PyTuple_New}{int len}
Return a new tuple object of size \var{len}, or \NULL{} on failure.
\end{cfuncdesc}

\begin{cfuncdesc}{int}{PyTuple_Size}{PyObject *p}
Takes a pointer to a tuple object, and returns the size
of that tuple.
\end{cfuncdesc}

\begin{cfuncdesc}{PyObject*}{PyTuple_GetItem}{PyObject *p, int pos}
Returns the object at position \var{pos} in the tuple pointed
to by \var{p}.  If \var{pos} is out of bounds, returns \NULL{} and
sets an \exception{IndexError} exception.
\end{cfuncdesc}

\begin{cfuncdesc}{PyObject*}{PyTuple_GET_ITEM}{PyObject *p, int pos}
Does the same, but does no checking of its arguments.
\end{cfuncdesc}

\begin{cfuncdesc}{PyObject*}{PyTuple_GetSlice}{PyObject *p,
                                               int low, int high}
Takes a slice of the tuple pointed to by \var{p} from
\var{low} to \var{high} and returns it as a new tuple.
\end{cfuncdesc}

\begin{cfuncdesc}{int}{PyTuple_SetItem}{PyObject *p,
                                        int pos, PyObject *o}
Inserts a reference to object \var{o} at position \var{pos} of
the tuple pointed to by \var{p}. It returns \code{0} on success.
\strong{Note:}  This function ``steals'' a reference to \var{o}.
\end{cfuncdesc}

\begin{cfuncdesc}{void}{PyTuple_SET_ITEM}{PyObject *p,
                                          int pos, PyObject *o}
Does the same, but does no error checking, and
should \emph{only} be used to fill in brand new tuples.
\strong{Note:}  This function ``steals'' a reference to \var{o}.
\end{cfuncdesc}

\begin{cfuncdesc}{int}{_PyTuple_Resize}{PyObject **p,
                                        int newsize, int last_is_sticky}
Can be used to resize a tuple.  \var{newsize} will be the new length
of the tuple.  Because tuples are \emph{supposed} to be immutable,
this should only be used if there is only one reference to the object.
Do \emph{not} use this if the tuple may already be known to some other
part of the code.  The tuple will always grow or shrink at the end.  The
\var{last_is_sticky} flag is not used and should always be false.  Think
of this as destroying the old tuple and creating a new one, only more
efficiently.  Returns \code{0} on success and \code{-1} on failure (in
which case a \exception{MemoryError} or \exception{SystemError} will be
raised).
\end{cfuncdesc}


\subsection{List Objects \label{listObjects}}

\obindex{list}
\begin{ctypedesc}{PyListObject}
This subtype of \ctype{PyObject} represents a Python list object.
\end{ctypedesc}

\begin{cvardesc}{PyTypeObject}{PyList_Type}
This instance of \ctype{PyTypeObject} represents the Python list
type.  This is the same object as \code{types.ListType}.
\withsubitem{(in module types)}{\ttindex{ListType}}
\end{cvardesc}

\begin{cfuncdesc}{int}{PyList_Check}{PyObject *p}
Returns true if its argument is a \ctype{PyListObject}.
\end{cfuncdesc}

\begin{cfuncdesc}{PyObject*}{PyList_New}{int len}
Returns a new list of length \var{len} on success, or \NULL{} on
failure.
\end{cfuncdesc}

\begin{cfuncdesc}{int}{PyList_Size}{PyObject *list}
Returns the length of the list object in \var{list}; this is
equivalent to \samp{len(\var{list})} on a list object.
\bifuncindex{len}
\end{cfuncdesc}

\begin{cfuncdesc}{int}{PyList_GET_SIZE}{PyObject *list}
Macro form of \cfunction{PyList_Size()} without error checking.
\end{cfuncdesc}

\begin{cfuncdesc}{PyObject*}{PyList_GetItem}{PyObject *list, int index}
Returns the object at position \var{pos} in the list pointed
to by \var{p}.  If \var{pos} is out of bounds, returns \NULL{} and
sets an \exception{IndexError} exception.
\end{cfuncdesc}

\begin{cfuncdesc}{PyObject*}{PyList_GET_ITEM}{PyObject *list, int i}
Macro form of \cfunction{PyList_GetItem()} without error checking.
\end{cfuncdesc}

\begin{cfuncdesc}{int}{PyList_SetItem}{PyObject *list, int index,
                                       PyObject *item}
Sets the item at index \var{index} in list to \var{item}.
\strong{Note:}  This function ``steals'' a reference to \var{item}.
\end{cfuncdesc}

\begin{cfuncdesc}{PyObject*}{PyList_SET_ITEM}{PyObject *list, int i,
                                              PyObject *o}
Macro form of \cfunction{PyList_SetItem()} without error checking.
\strong{Note:}  This function ``steals'' a reference to \var{item}.
\end{cfuncdesc}

\begin{cfuncdesc}{int}{PyList_Insert}{PyObject *list, int index,
                                      PyObject *item}
Inserts the item \var{item} into list \var{list} in front of index
\var{index}.  Returns \code{0} if successful; returns \code{-1} and
raises an exception if unsuccessful.  Analogous to
\code{\var{list}.insert(\var{index}, \var{item})}.
\end{cfuncdesc}

\begin{cfuncdesc}{int}{PyList_Append}{PyObject *list, PyObject *item}
Appends the object \var{item} at the end of list \var{list}.  Returns
\code{0} if successful; returns \code{-1} and sets an exception if
unsuccessful.  Analogous to \code{\var{list}.append(\var{item})}.
\end{cfuncdesc}

\begin{cfuncdesc}{PyObject*}{PyList_GetSlice}{PyObject *list,
                                              int low, int high}
Returns a list of the objects in \var{list} containing the objects 
\emph{between} \var{low} and \var{high}.  Returns NULL and sets an
exception if unsuccessful.
Analogous to \code{\var{list}[\var{low}:\var{high}]}.
\end{cfuncdesc}

\begin{cfuncdesc}{int}{PyList_SetSlice}{PyObject *list,
                                        int low, int high,
                                        PyObject *itemlist}
Sets the slice of \var{list} between \var{low} and \var{high} to the
contents of \var{itemlist}.  Analogous to
\code{\var{list}[\var{low}:\var{high}] = \var{itemlist}}.  Returns
\code{0} on success, \code{-1} on failure.
\end{cfuncdesc}

\begin{cfuncdesc}{int}{PyList_Sort}{PyObject *list}
Sorts the items of \var{list} in place.  Returns \code{0} on success,
\code{-1} on failure.  This is equivalent to
\samp{\var{list}.sort()}.
\end{cfuncdesc}

\begin{cfuncdesc}{int}{PyList_Reverse}{PyObject *list}
Reverses the items of \var{list} in place.  Returns \code{0} on
success, \code{-1} on failure.  This is the equivalent of
\samp{\var{list}.reverse()}.
\end{cfuncdesc}

\begin{cfuncdesc}{PyObject*}{PyList_AsTuple}{PyObject *list}
Returns a new tuple object containing the contents of \var{list};
equivalent to \samp{tuple(\var{list})}.\bifuncindex{tuple}
\end{cfuncdesc}


\section{Mapping Objects \label{mapObjects}}

\obindex{mapping}


\subsection{Dictionary Objects \label{dictObjects}}

\obindex{dictionary}
\begin{ctypedesc}{PyDictObject}
This subtype of \ctype{PyObject} represents a Python dictionary object.
\end{ctypedesc}

\begin{cvardesc}{PyTypeObject}{PyDict_Type}
This instance of \ctype{PyTypeObject} represents the Python dictionary 
type.  This is exposed to Python programs as \code{types.DictType} and 
\code{types.DictionaryType}.
\withsubitem{(in module types)}{\ttindex{DictType}\ttindex{DictionaryType}}
\end{cvardesc}

\begin{cfuncdesc}{int}{PyDict_Check}{PyObject *p}
Returns true if its argument is a \ctype{PyDictObject}.
\end{cfuncdesc}

\begin{cfuncdesc}{PyObject*}{PyDict_New}{}
Returns a new empty dictionary, or \NULL{} on failure.
\end{cfuncdesc}

\begin{cfuncdesc}{void}{PyDict_Clear}{PyObject *p}
Empties an existing dictionary of all key-value pairs.
\end{cfuncdesc}

\begin{cfuncdesc}{PyObject*}{PyDict_Copy}{PyObject *p}
Returns a new dictionary that contains the same key-value pairs as p.
Empties an existing dictionary of all key-value pairs.
\end{cfuncdesc}

\begin{cfuncdesc}{int}{PyDict_SetItem}{PyObject *p, PyObject *key,
                                       PyObject *val}
Inserts \var{value} into the dictionary with a key of \var{key}.
\var{key} must be hashable; if it isn't, \exception{TypeError} will be 
raised.
\end{cfuncdesc}

\begin{cfuncdesc}{int}{PyDict_SetItemString}{PyObject *p,
            char *key,
            PyObject *val}
Inserts \var{value} into the dictionary using \var{key}
as a key. \var{key} should be a \ctype{char*}.  The key object is
created using \code{PyString_FromString(\var{key})}.
\ttindex{PyString_FromString()}
\end{cfuncdesc}

\begin{cfuncdesc}{int}{PyDict_DelItem}{PyObject *p, PyObject *key}
Removes the entry in dictionary \var{p} with key \var{key}.
\var{key} must be hashable; if it isn't, \exception{TypeError} is
raised.
\end{cfuncdesc}

\begin{cfuncdesc}{int}{PyDict_DelItemString}{PyObject *p, char *key}
Removes the entry in dictionary \var{p} which has a key
specified by the string \var{key}.
\end{cfuncdesc}

\begin{cfuncdesc}{PyObject*}{PyDict_GetItem}{PyObject *p, PyObject *key}
Returns the object from dictionary \var{p} which has a key
\var{key}.  Returns \NULL{} if the key \var{key} is not present, but
\emph{without} setting an exception.
\end{cfuncdesc}

\begin{cfuncdesc}{PyObject*}{PyDict_GetItemString}{PyObject *p, char *key}
This is the same as \cfunction{PyDict_GetItem()}, but \var{key} is
specified as a \ctype{char*}, rather than a \ctype{PyObject*}.
\end{cfuncdesc}

\begin{cfuncdesc}{PyObject*}{PyDict_Items}{PyObject *p}
Returns a \ctype{PyListObject} containing all the items 
from the dictionary, as in the dictinoary method \method{items()} (see
the \citetitle[../lib/lib.html]{Python Library Reference}).
\end{cfuncdesc}

\begin{cfuncdesc}{PyObject*}{PyDict_Keys}{PyObject *p}
Returns a \ctype{PyListObject} containing all the keys 
from the dictionary, as in the dictionary method \method{keys()} (see the
\citetitle[../lib/lib.html]{Python Library Reference}).
\end{cfuncdesc}

\begin{cfuncdesc}{PyObject*}{PyDict_Values}{PyObject *p}
Returns a \ctype{PyListObject} containing all the values 
from the dictionary \var{p}, as in the dictionary method
\method{values()} (see the \citetitle[../lib/lib.html]{Python Library
Reference}).
\end{cfuncdesc}

\begin{cfuncdesc}{int}{PyDict_Size}{PyObject *p}
Returns the number of items in the dictionary.  This is equivalent to
\samp{len(\var{p})} on a dictionary.\bifuncindex{len}
\end{cfuncdesc}

\begin{cfuncdesc}{int}{PyDict_Next}{PyObject *p, int *ppos,
                                    PyObject **pkey, PyObject **pvalue}
Iterate over all key-value pairs in the dictionary \var{p}.  The
\ctype{int} referred to by \var{ppos} must be initialized to \code{0}
prior to the first call to this function to start the iteration; the
function returns true for each pair in the dictionary, and false once
all pairs have been reported.  The parameters \var{pkey} and
\var{pvalue} should either point to \ctype{PyObject*} variables that
will be filled in with each key and value, respectively, or may be
\NULL.

For example:

\begin{verbatim}
PyObject *key, *value;
int pos = 0;

while (PyDict_Next(self->dict, &pos, &key, &value)) {
    /* do something interesting with the values... */
    ...
}
\end{verbatim}

The dictionary \var{p} should not be mutated during iteration.  It is
safe (since Python 2.1) to modify the values of the keys as you
iterate over the dictionary, for example:

\begin{verbatim}
PyObject *key, *value;
int pos = 0;

while (PyDict_Next(self->dict, &pos, &key, &value)) {
    int i = PyInt_AS_LONG(value) + 1;
    PyObject *o = PyInt_FromLong(i);
    if (o == NULL)
        return -1;
    if (PyDict_SetItem(self->dict, key, o) < 0) {
        Py_DECREF(o);
        return -1;
    }
    Py_DECREF(o);
}
\end{verbatim}
\end{cfuncdesc}


\section{Numeric Objects \label{numericObjects}}

\obindex{numeric}


\subsection{Plain Integer Objects \label{intObjects}}

\obindex{integer}
\begin{ctypedesc}{PyIntObject}
This subtype of \ctype{PyObject} represents a Python integer object.
\end{ctypedesc}

\begin{cvardesc}{PyTypeObject}{PyInt_Type}
This instance of \ctype{PyTypeObject} represents the Python plain 
integer type.  This is the same object as \code{types.IntType}.
\withsubitem{(in modules types)}{\ttindex{IntType}}
\end{cvardesc}

\begin{cfuncdesc}{int}{PyInt_Check}{PyObject* o}
Returns true if \var{o} is of type \cdata{PyInt_Type}.
\end{cfuncdesc}

\begin{cfuncdesc}{PyObject*}{PyInt_FromLong}{long ival}
Creates a new integer object with a value of \var{ival}.

The current implementation keeps an array of integer objects for all
integers between \code{-1} and \code{100}, when you create an int in
that range you actually just get back a reference to the existing
object. So it should be possible to change the value of \code{1}. I
suspect the behaviour of Python in this case is undefined. :-)
\end{cfuncdesc}

\begin{cfuncdesc}{long}{PyInt_AsLong}{PyObject *io}
Will first attempt to cast the object to a \ctype{PyIntObject}, if
it is not already one, and then return its value.
\end{cfuncdesc}

\begin{cfuncdesc}{long}{PyInt_AS_LONG}{PyObject *io}
Returns the value of the object \var{io}.  No error checking is
performed.
\end{cfuncdesc}

\begin{cfuncdesc}{long}{PyInt_GetMax}{}
Returns the system's idea of the largest integer it can handle
(\constant{LONG_MAX}\ttindex{LONG_MAX}, as defined in the system
header files).
\end{cfuncdesc}


\subsection{Long Integer Objects \label{longObjects}}

\obindex{long integer}
\begin{ctypedesc}{PyLongObject}
This subtype of \ctype{PyObject} represents a Python long integer
object.
\end{ctypedesc}

\begin{cvardesc}{PyTypeObject}{PyLong_Type}
This instance of \ctype{PyTypeObject} represents the Python long
integer type.  This is the same object as \code{types.LongType}.
\withsubitem{(in modules types)}{\ttindex{LongType}}
\end{cvardesc}

\begin{cfuncdesc}{int}{PyLong_Check}{PyObject *p}
Returns true if its argument is a \ctype{PyLongObject}.
\end{cfuncdesc}

\begin{cfuncdesc}{PyObject*}{PyLong_FromLong}{long v}
Returns a new \ctype{PyLongObject} object from \var{v}, or \NULL{} on
failure.
\end{cfuncdesc}

\begin{cfuncdesc}{PyObject*}{PyLong_FromUnsignedLong}{unsigned long v}
Returns a new \ctype{PyLongObject} object from a C \ctype{unsigned
long}, or \NULL{} on failure.
\end{cfuncdesc}

\begin{cfuncdesc}{PyObject*}{PyLong_FromDouble}{double v}
Returns a new \ctype{PyLongObject} object from the integer part of
\var{v}, or \NULL{} on failure.
\end{cfuncdesc}

\begin{cfuncdesc}{long}{PyLong_AsLong}{PyObject *pylong}
Returns a C \ctype{long} representation of the contents of
\var{pylong}.  If \var{pylong} is greater than
\constant{LONG_MAX}\ttindex{LONG_MAX}, an \exception{OverflowError} is
raised.\withsubitem{(built-in exception)}{OverflowError}
\end{cfuncdesc}

\begin{cfuncdesc}{unsigned long}{PyLong_AsUnsignedLong}{PyObject *pylong}
Returns a C \ctype{unsigned long} representation of the contents of 
\var{pylong}.  If \var{pylong} is greater than
\constant{ULONG_MAX}\ttindex{ULONG_MAX}, an \exception{OverflowError}
is raised.\withsubitem{(built-in exception)}{OverflowError}
\end{cfuncdesc}

\begin{cfuncdesc}{double}{PyLong_AsDouble}{PyObject *pylong}
Returns a C \ctype{double} representation of the contents of \var{pylong}.
\end{cfuncdesc}

\begin{cfuncdesc}{PyObject*}{PyLong_FromString}{char *str, char **pend,
                                                int base}
Return a new \ctype{PyLongObject} based on the string value in
\var{str}, which is interpreted according to the radix in \var{base}.
If \var{pend} is non-\NULL, \code{*\var{pend}} will point to the first 
character in \var{str} which follows the representation of the
number.  If \var{base} is \code{0}, the radix will be determined base
on the leading characters of \var{str}: if \var{str} starts with
\code{'0x'} or \code{'0X'}, radix 16 will be used; if \var{str} starts 
with \code{'0'}, radix 8 will be used; otherwise radix 10 will be
used.  If \var{base} is not \code{0}, it must be between \code{2} and
\code{36}, inclusive.  Leading spaces are ignored.  If there are no
digits, \exception{ValueError} will be raised.
\end{cfuncdesc}


\subsection{Floating Point Objects \label{floatObjects}}

\obindex{floating point}
\begin{ctypedesc}{PyFloatObject}
This subtype of \ctype{PyObject} represents a Python floating point
object.
\end{ctypedesc}

\begin{cvardesc}{PyTypeObject}{PyFloat_Type}
This instance of \ctype{PyTypeObject} represents the Python floating
point type.  This is the same object as \code{types.FloatType}.
\withsubitem{(in modules types)}{\ttindex{FloatType}}
\end{cvardesc}

\begin{cfuncdesc}{int}{PyFloat_Check}{PyObject *p}
Returns true if its argument is a \ctype{PyFloatObject}.
\end{cfuncdesc}

\begin{cfuncdesc}{PyObject*}{PyFloat_FromDouble}{double v}
Creates a \ctype{PyFloatObject} object from \var{v}, or \NULL{} on
failure.
\end{cfuncdesc}

\begin{cfuncdesc}{double}{PyFloat_AsDouble}{PyObject *pyfloat}
Returns a C \ctype{double} representation of the contents of \var{pyfloat}.
\end{cfuncdesc}

\begin{cfuncdesc}{double}{PyFloat_AS_DOUBLE}{PyObject *pyfloat}
Returns a C \ctype{double} representation of the contents of
\var{pyfloat}, but without error checking.
\end{cfuncdesc}


\subsection{Complex Number Objects \label{complexObjects}}

\obindex{complex number}
Python's complex number objects are implemented as two distinct types
when viewed from the C API:  one is the Python object exposed to
Python programs, and the other is a C structure which represents the
actual complex number value.  The API provides functions for working
with both.

\subsubsection{Complex Numbers as C Structures}

Note that the functions which accept these structures as parameters
and return them as results do so \emph{by value} rather than
dereferencing them through pointers.  This is consistent throughout
the API.

\begin{ctypedesc}{Py_complex}
The C structure which corresponds to the value portion of a Python
complex number object.  Most of the functions for dealing with complex
number objects use structures of this type as input or output values,
as appropriate.  It is defined as:

\begin{verbatim}
typedef struct {
   double real;
   double imag;
} Py_complex;
\end{verbatim}
\end{ctypedesc}

\begin{cfuncdesc}{Py_complex}{_Py_c_sum}{Py_complex left, Py_complex right}
Return the sum of two complex numbers, using the C
\ctype{Py_complex} representation.
\end{cfuncdesc}

\begin{cfuncdesc}{Py_complex}{_Py_c_diff}{Py_complex left, Py_complex right}
Return the difference between two complex numbers, using the C
\ctype{Py_complex} representation.
\end{cfuncdesc}

\begin{cfuncdesc}{Py_complex}{_Py_c_neg}{Py_complex complex}
Return the negation of the complex number \var{complex}, using the C
\ctype{Py_complex} representation.
\end{cfuncdesc}

\begin{cfuncdesc}{Py_complex}{_Py_c_prod}{Py_complex left, Py_complex right}
Return the product of two complex numbers, using the C
\ctype{Py_complex} representation.
\end{cfuncdesc}

\begin{cfuncdesc}{Py_complex}{_Py_c_quot}{Py_complex dividend,
                                          Py_complex divisor}
Return the quotient of two complex numbers, using the C
\ctype{Py_complex} representation.
\end{cfuncdesc}

\begin{cfuncdesc}{Py_complex}{_Py_c_pow}{Py_complex num, Py_complex exp}
Return the exponentiation of \var{num} by \var{exp}, using the C
\ctype{Py_complex} representation.
\end{cfuncdesc}


\subsubsection{Complex Numbers as Python Objects}

\begin{ctypedesc}{PyComplexObject}
This subtype of \ctype{PyObject} represents a Python complex number object.
\end{ctypedesc}

\begin{cvardesc}{PyTypeObject}{PyComplex_Type}
This instance of \ctype{PyTypeObject} represents the Python complex 
number type.
\end{cvardesc}

\begin{cfuncdesc}{int}{PyComplex_Check}{PyObject *p}
Returns true if its argument is a \ctype{PyComplexObject}.
\end{cfuncdesc}

\begin{cfuncdesc}{PyObject*}{PyComplex_FromCComplex}{Py_complex v}
Create a new Python complex number object from a C
\ctype{Py_complex} value.
\end{cfuncdesc}

\begin{cfuncdesc}{PyObject*}{PyComplex_FromDoubles}{double real, double imag}
Returns a new \ctype{PyComplexObject} object from \var{real} and \var{imag}.
\end{cfuncdesc}

\begin{cfuncdesc}{double}{PyComplex_RealAsDouble}{PyObject *op}
Returns the real part of \var{op} as a C \ctype{double}.
\end{cfuncdesc}

\begin{cfuncdesc}{double}{PyComplex_ImagAsDouble}{PyObject *op}
Returns the imaginary part of \var{op} as a C \ctype{double}.
\end{cfuncdesc}

\begin{cfuncdesc}{Py_complex}{PyComplex_AsCComplex}{PyObject *op}
Returns the \ctype{Py_complex} value of the complex number \var{op}.
\end{cfuncdesc}



\section{Other Objects \label{otherObjects}}

\subsection{File Objects \label{fileObjects}}

\obindex{file}
Python's built-in file objects are implemented entirely on the
\ctype{FILE*} support from the C standard library.  This is an
implementation detail and may change in future releases of Python.

\begin{ctypedesc}{PyFileObject}
This subtype of \ctype{PyObject} represents a Python file object.
\end{ctypedesc}

\begin{cvardesc}{PyTypeObject}{PyFile_Type}
This instance of \ctype{PyTypeObject} represents the Python file
type.  This is exposed to Python programs as \code{types.FileType}.
\withsubitem{(in module types)}{\ttindex{FileType}}
\end{cvardesc}

\begin{cfuncdesc}{int}{PyFile_Check}{PyObject *p}
Returns true if its argument is a \ctype{PyFileObject}.
\end{cfuncdesc}

\begin{cfuncdesc}{PyObject*}{PyFile_FromString}{char *filename, char *mode}
On success, returns a new file object that is opened on the
file given by \var{filename}, with a file mode given by \var{mode},
where \var{mode} has the same semantics as the standard C routine
\cfunction{fopen()}\ttindex{fopen()}.  On failure, returns \NULL.
\end{cfuncdesc}

\begin{cfuncdesc}{PyObject*}{PyFile_FromFile}{FILE *fp,
                                              char *name, char *mode,
                                              int (*close)(FILE*)}
Creates a new \ctype{PyFileObject} from the already-open standard C
file pointer, \var{fp}.  The function \var{close} will be called when
the file should be closed.  Returns \NULL{} on failure.
\end{cfuncdesc}

\begin{cfuncdesc}{FILE*}{PyFile_AsFile}{PyFileObject *p}
Returns the file object associated with \var{p} as a \ctype{FILE*}.
\end{cfuncdesc}

\begin{cfuncdesc}{PyObject*}{PyFile_GetLine}{PyObject *p, int n}
Equivalent to \code{\var{p}.readline(\optional{\var{n}})}, this
function reads one line from the object \var{p}.  \var{p} may be a
file object or any object with a \method{readline()} method.  If
\var{n} is \code{0}, exactly one line is read, regardless of the
length of the line.  If \var{n} is greater than \code{0}, no more than 
\var{n} bytes will be read from the file; a partial line can be
returned.  In both cases, an empty string is returned if the end of
the file is reached immediately.  If \var{n} is less than \code{0},
however, one line is read regardless of length, but
\exception{EOFError} is raised if the end of the file is reached
immediately.
\withsubitem{(built-in exception)}{\ttindex{EOFError}}
\end{cfuncdesc}

\begin{cfuncdesc}{PyObject*}{PyFile_Name}{PyObject *p}
Returns the name of the file specified by \var{p} as a string object.
\end{cfuncdesc}

\begin{cfuncdesc}{void}{PyFile_SetBufSize}{PyFileObject *p, int n}
Available on systems with \cfunction{setvbuf()}\ttindex{setvbuf()}
only.  This should only be called immediately after file object
creation.
\end{cfuncdesc}

\begin{cfuncdesc}{int}{PyFile_SoftSpace}{PyObject *p, int newflag}
This function exists for internal use by the interpreter.
Sets the \member{softspace} attribute of \var{p} to \var{newflag} and
\withsubitem{(file attribute)}{\ttindex{softspace}}returns the
previous value.  \var{p} does not have to be a file object
for this function to work properly; any object is supported (thought
its only interesting if the \member{softspace} attribute can be set).
This function clears any errors, and will return \code{0} as the
previous value if the attribute either does not exist or if there were
errors in retrieving it.  There is no way to detect errors from this
function, but doing so should not be needed.
\end{cfuncdesc}

\begin{cfuncdesc}{int}{PyFile_WriteObject}{PyObject *obj, PyFileObject *p,
                                           int flags}
Writes object \var{obj} to file object \var{p}.  The only supported
flag for \var{flags} is \constant{Py_PRINT_RAW}\ttindex{Py_PRINT_RAW};
if given, the \function{str()} of the object is written instead of the 
\function{repr()}.  Returns \code{0} on success or \code{-1} on
failure; the appropriate exception will be set.
\end{cfuncdesc}

\begin{cfuncdesc}{int}{PyFile_WriteString}{char *s, PyFileObject *p,
                                           int flags}
Writes string \var{s} to file object \var{p}.  Returns \code{0} on
success or \code{-1} on failure; the appropriate exception will be
set.
\end{cfuncdesc}


\subsection{Instance Objects \label{instanceObjects}}

\obindex{instance}
There are very few functions specific to instance objects.

\begin{cvardesc}{PyTypeObject}{PyInstance_Type}
  Type object for class instances.
\end{cvardesc}

\begin{cfuncdesc}{int}{PyInstance_Check}{PyObject *obj}
  Returns true if \var{obj} is an instance.
\end{cfuncdesc}

\begin{cfuncdesc}{PyObject*}{PyInstance_New}{PyObject *class,
                                             PyObject *arg,
                                             PyObject *kw}
  Create a new instance of a specific class.  The parameters \var{arg}
  and \var{kw} are used as the positional and keyword parameters to
  the object's constructor.
\end{cfuncdesc}

\begin{cfuncdesc}{PyObject*}{PyInstance_NewRaw}{PyObject *class,
                                                PyObject *dict}
  Create a new instance of a specific class without calling it's
  constructor.  \var{class} is the class of new object.  The
  \var{dict} parameter will be used as the object's \member{__dict__};
  if \NULL, a new dictionary will be created for the instance.
\end{cfuncdesc}


\subsection{Module Objects \label{moduleObjects}}

\obindex{module}
There are only a few functions special to module objects.

\begin{cvardesc}{PyTypeObject}{PyModule_Type}
This instance of \ctype{PyTypeObject} represents the Python module
type.  This is exposed to Python programs as \code{types.ModuleType}.
\withsubitem{(in module types)}{\ttindex{ModuleType}}
\end{cvardesc}

\begin{cfuncdesc}{int}{PyModule_Check}{PyObject *p}
Returns true if its argument is a module object.
\end{cfuncdesc}

\begin{cfuncdesc}{PyObject*}{PyModule_New}{char *name}
Return a new module object with the \member{__name__} attribute set to
\var{name}.  Only the module's \member{__doc__} and
\member{__name__} attributes are filled in; the caller is responsible
for providing a \member{__file__} attribute.
\withsubitem{(module attribute)}{
  \ttindex{__name__}\ttindex{__doc__}\ttindex{__file__}}
\end{cfuncdesc}

\begin{cfuncdesc}{PyObject*}{PyModule_GetDict}{PyObject *module}
Return the dictionary object that implements \var{module}'s namespace; 
this object is the same as the \member{__dict__} attribute of the
module object.  This function never fails.
\withsubitem{(module attribute)}{\ttindex{__dict__}}
\end{cfuncdesc}

\begin{cfuncdesc}{char*}{PyModule_GetName}{PyObject *module}
Return \var{module}'s \member{__name__} value.  If the module does not 
provide one, or if it is not a string, \exception{SystemError} is
raised and \NULL{} is returned.
\withsubitem{(module attribute)}{\ttindex{__name__}}
\withsubitem{(built-in exception)}{\ttindex{SystemError}}
\end{cfuncdesc}

\begin{cfuncdesc}{char*}{PyModule_GetFilename}{PyObject *module}
Return the name of the file from which \var{module} was loaded using
\var{module}'s \member{__file__} attribute.  If this is not defined,
or if it is not a string, raise \exception{SystemError} and return
\NULL.
\withsubitem{(module attribute)}{\ttindex{__file__}}
\withsubitem{(built-in exception)}{\ttindex{SystemError}}
\end{cfuncdesc}

\begin{cfuncdesc}{int}{PyModule_AddObject}{PyObject *module,
                                           char *name, PyObject *value}
Add an object to \var{module} as \var{name}.  This is a convenience
function which can be used from the module's initialization function.
This steals a reference to \var{value}.  Returns \code{-1} on error,
\code{0} on success.
\versionadded{2.0}
\end{cfuncdesc}

\begin{cfuncdesc}{int}{PyModule_AddIntConstant}{PyObject *module,
                                                char *name, int value}
Add an integer constant to \var{module} as \var{name}.  This convenience
function can be used from the module's initialization function.
Returns \code{-1} on error, \code{0} on success.
\versionadded{2.0}
\end{cfuncdesc}

\begin{cfuncdesc}{int}{PyModule_AddStringConstant}{PyObject *module,
                                                   char *name, char *value}
Add a string constant to \var{module} as \var{name}.  This convenience
function can be used from the module's initialization function.  The
string \var{value} must be null-terminated.  Returns \code{-1} on
error, \code{0} on success.
\versionadded{2.0}
\end{cfuncdesc}


\subsection{CObjects \label{cObjects}}

\obindex{CObject}
Refer to \emph{Extending and Embedding the Python Interpreter},
section 1.12 (``Providing a C API for an Extension Module''), for more 
information on using these objects.


\begin{ctypedesc}{PyCObject}
This subtype of \ctype{PyObject} represents an opaque value, useful for
C extension modules who need to pass an opaque value (as a
\ctype{void*} pointer) through Python code to other C code.  It is
often used to make a C function pointer defined in one module
available to other modules, so the regular import mechanism can be
used to access C APIs defined in dynamically loaded modules.
\end{ctypedesc}

\begin{cfuncdesc}{int}{PyCObject_Check}{PyObject *p}
Returns true if its argument is a \ctype{PyCObject}.
\end{cfuncdesc}

\begin{cfuncdesc}{PyObject*}{PyCObject_FromVoidPtr}{void* cobj, 
	void (*destr)(void *)}
Creates a \ctype{PyCObject} from the \code{void *}\var{cobj}.  The
\var{destr} function will be called when the object is reclaimed, unless
it is \NULL.
\end{cfuncdesc}

\begin{cfuncdesc}{PyObject*}{PyCObject_FromVoidPtrAndDesc}{void* cobj,
	void* desc, void (*destr)(void *, void *) }
Creates a \ctype{PyCObject} from the \ctype{void *}\var{cobj}.  The
\var{destr} function will be called when the object is reclaimed.  The
\var{desc} argument can be used to pass extra callback data for the
destructor function.
\end{cfuncdesc}

\begin{cfuncdesc}{void*}{PyCObject_AsVoidPtr}{PyObject* self}
Returns the object \ctype{void *} that the
\ctype{PyCObject} \var{self} was created with.
\end{cfuncdesc}

\begin{cfuncdesc}{void*}{PyCObject_GetDesc}{PyObject* self}
Returns the description \ctype{void *} that the
\ctype{PyCObject} \var{self} was created with.
\end{cfuncdesc}


\chapter{Initialization, Finalization, and Threads
         \label{initialization}}

\begin{cfuncdesc}{void}{Py_Initialize}{}
Initialize the Python interpreter.  In an application embedding 
Python, this should be called before using any other Python/C API 
functions; with the exception of
\cfunction{Py_SetProgramName()}\ttindex{Py_SetProgramName()},
\cfunction{PyEval_InitThreads()}\ttindex{PyEval_InitThreads()},
\cfunction{PyEval_ReleaseLock()}\ttindex{PyEval_ReleaseLock()},
and \cfunction{PyEval_AcquireLock()}\ttindex{PyEval_AcquireLock()}.
This initializes the table of loaded modules (\code{sys.modules}), and
\withsubitem{(in module sys)}{\ttindex{modules}\ttindex{path}}creates the
fundamental modules \module{__builtin__}\refbimodindex{__builtin__},
\module{__main__}\refbimodindex{__main__} and
\module{sys}\refbimodindex{sys}.  It also initializes the module
search\indexiii{module}{search}{path} path (\code{sys.path}).
It does not set \code{sys.argv}; use
\cfunction{PySys_SetArgv()}\ttindex{PySys_SetArgv()} for that.  This
is a no-op when called for a second time (without calling
\cfunction{Py_Finalize()}\ttindex{Py_Finalize()} first).  There is no
return value; it is a fatal error if the initialization fails.
\end{cfuncdesc}

\begin{cfuncdesc}{int}{Py_IsInitialized}{}
Return true (nonzero) when the Python interpreter has been
initialized, false (zero) if not.  After \cfunction{Py_Finalize()} is
called, this returns false until \cfunction{Py_Initialize()} is called
again.
\end{cfuncdesc}

\begin{cfuncdesc}{void}{Py_Finalize}{}
Undo all initializations made by \cfunction{Py_Initialize()} and
subsequent use of Python/C API functions, and destroy all
sub-interpreters (see \cfunction{Py_NewInterpreter()} below) that were
created and not yet destroyed since the last call to
\cfunction{Py_Initialize()}.  Ideally, this frees all memory allocated
by the Python interpreter.  This is a no-op when called for a second
time (without calling \cfunction{Py_Initialize()} again first).  There
is no return value; errors during finalization are ignored.

This function is provided for a number of reasons.  An embedding 
application might want to restart Python without having to restart the 
application itself.  An application that has loaded the Python 
interpreter from a dynamically loadable library (or DLL) might want to 
free all memory allocated by Python before unloading the DLL. During a 
hunt for memory leaks in an application a developer might want to free 
all memory allocated by Python before exiting from the application.

\strong{Bugs and caveats:} The destruction of modules and objects in 
modules is done in random order; this may cause destructors 
(\method{__del__()} methods) to fail when they depend on other objects 
(even functions) or modules.  Dynamically loaded extension modules 
loaded by Python are not unloaded.  Small amounts of memory allocated 
by the Python interpreter may not be freed (if you find a leak, please 
report it).  Memory tied up in circular references between objects is 
not freed.  Some memory allocated by extension modules may not be 
freed.  Some extension may not work properly if their initialization 
routine is called more than once; this can happen if an applcation 
calls \cfunction{Py_Initialize()} and \cfunction{Py_Finalize()} more
than once.
\end{cfuncdesc}

\begin{cfuncdesc}{PyThreadState*}{Py_NewInterpreter}{}
Create a new sub-interpreter.  This is an (almost) totally separate
environment for the execution of Python code.  In particular, the new
interpreter has separate, independent versions of all imported
modules, including the fundamental modules
\module{__builtin__}\refbimodindex{__builtin__},
\module{__main__}\refbimodindex{__main__} and
\module{sys}\refbimodindex{sys}.  The table of loaded modules
(\code{sys.modules}) and the module search path (\code{sys.path}) are
also separate.  The new environment has no \code{sys.argv} variable.
It has new standard I/O stream file objects \code{sys.stdin},
\code{sys.stdout} and \code{sys.stderr} (however these refer to the
same underlying \ctype{FILE} structures in the C library).
\withsubitem{(in module sys)}{
  \ttindex{stdout}\ttindex{stderr}\ttindex{stdin}}

The return value points to the first thread state created in the new 
sub-interpreter.  This thread state is made the current thread state.  
Note that no actual thread is created; see the discussion of thread 
states below.  If creation of the new interpreter is unsuccessful, 
\NULL{} is returned; no exception is set since the exception state 
is stored in the current thread state and there may not be a current 
thread state.  (Like all other Python/C API functions, the global 
interpreter lock must be held before calling this function and is 
still held when it returns; however, unlike most other Python/C API 
functions, there needn't be a current thread state on entry.)

Extension modules are shared between (sub-)interpreters as follows: 
the first time a particular extension is imported, it is initialized 
normally, and a (shallow) copy of its module's dictionary is 
squirreled away.  When the same extension is imported by another 
(sub-)interpreter, a new module is initialized and filled with the 
contents of this copy; the extension's \code{init} function is not
called.  Note that this is different from what happens when an
extension is imported after the interpreter has been completely
re-initialized by calling
\cfunction{Py_Finalize()}\ttindex{Py_Finalize()} and
\cfunction{Py_Initialize()}\ttindex{Py_Initialize()}; in that case,
the extension's \code{init\var{module}} function \emph{is} called
again.

\strong{Bugs and caveats:} Because sub-interpreters (and the main 
interpreter) are part of the same process, the insulation between them 
isn't perfect --- for example, using low-level file operations like 
\withsubitem{(in module os)}{\ttindex{close()}}
\function{os.close()} they can (accidentally or maliciously) affect each 
other's open files.  Because of the way extensions are shared between 
(sub-)interpreters, some extensions may not work properly; this is 
especially likely when the extension makes use of (static) global 
variables, or when the extension manipulates its module's dictionary 
after its initialization.  It is possible to insert objects created in 
one sub-interpreter into a namespace of another sub-interpreter; this 
should be done with great care to avoid sharing user-defined 
functions, methods, instances or classes between sub-interpreters, 
since import operations executed by such objects may affect the 
wrong (sub-)interpreter's dictionary of loaded modules.  (XXX This is 
a hard-to-fix bug that will be addressed in a future release.)
\end{cfuncdesc}

\begin{cfuncdesc}{void}{Py_EndInterpreter}{PyThreadState *tstate}
Destroy the (sub-)interpreter represented by the given thread state.  
The given thread state must be the current thread state.  See the 
discussion of thread states below.  When the call returns, the current 
thread state is \NULL{}.  All thread states associated with this 
interpreted are destroyed.  (The global interpreter lock must be held 
before calling this function and is still held when it returns.)  
\cfunction{Py_Finalize()}\ttindex{Py_Finalize()} will destroy all
sub-interpreters that haven't been explicitly destroyed at that point.
\end{cfuncdesc}

\begin{cfuncdesc}{void}{Py_SetProgramName}{char *name}
This function should be called before
\cfunction{Py_Initialize()}\ttindex{Py_Initialize()} is called
for the first time, if it is called at all.  It tells the interpreter 
the value of the \code{argv[0]} argument to the
\cfunction{main()}\ttindex{main()} function of the program.  This is
used by \cfunction{Py_GetPath()}\ttindex{Py_GetPath()} and some other  
functions below to find the Python run-time libraries relative to the 
interpreter executable.  The default value is \code{'python'}.  The 
argument should point to a zero-terminated character string in static 
storage whose contents will not change for the duration of the 
program's execution.  No code in the Python interpreter will change 
the contents of this storage.
\end{cfuncdesc}

\begin{cfuncdesc}{char*}{Py_GetProgramName}{}
Return the program name set with
\cfunction{Py_SetProgramName()}\ttindex{Py_SetProgramName()}, or the
default.  The returned string points into static storage; the caller 
should not modify its value.
\end{cfuncdesc}

\begin{cfuncdesc}{char*}{Py_GetPrefix}{}
Return the \emph{prefix} for installed platform-independent files.  This 
is derived through a number of complicated rules from the program name 
set with \cfunction{Py_SetProgramName()} and some environment variables; 
for example, if the program name is \code{'/usr/local/bin/python'}, 
the prefix is \code{'/usr/local'}.  The returned string points into 
static storage; the caller should not modify its value.  This 
corresponds to the \makevar{prefix} variable in the top-level 
\file{Makefile} and the \longprogramopt{prefix} argument to the 
\program{configure} script at build time.  The value is available to 
Python code as \code{sys.prefix}.  It is only useful on \UNIX{}.  See 
also the next function.
\end{cfuncdesc}

\begin{cfuncdesc}{char*}{Py_GetExecPrefix}{}
Return the \emph{exec-prefix} for installed platform-\emph{de}pendent 
files.  This is derived through a number of complicated rules from the 
program name set with \cfunction{Py_SetProgramName()} and some environment 
variables; for example, if the program name is 
\code{'/usr/local/bin/python'}, the exec-prefix is 
\code{'/usr/local'}.  The returned string points into static storage; 
the caller should not modify its value.  This corresponds to the 
\makevar{exec_prefix} variable in the top-level \file{Makefile} and the 
\longprogramopt{exec-prefix} argument to the
\program{configure} script at build  time.  The value is available to
Python code as \code{sys.exec_prefix}.  It is only useful on \UNIX{}.

Background: The exec-prefix differs from the prefix when platform 
dependent files (such as executables and shared libraries) are 
installed in a different directory tree.  In a typical installation, 
platform dependent files may be installed in the 
\file{/usr/local/plat} subtree while platform independent may be 
installed in \file{/usr/local}.

Generally speaking, a platform is a combination of hardware and 
software families, e.g.  Sparc machines running the Solaris 2.x 
operating system are considered the same platform, but Intel machines 
running Solaris 2.x are another platform, and Intel machines running 
Linux are yet another platform.  Different major revisions of the same 
operating system generally also form different platforms.  Non-\UNIX{} 
operating systems are a different story; the installation strategies 
on those systems are so different that the prefix and exec-prefix are 
meaningless, and set to the empty string.  Note that compiled Python 
bytecode files are platform independent (but not independent from the 
Python version by which they were compiled!).

System administrators will know how to configure the \program{mount} or 
\program{automount} programs to share \file{/usr/local} between platforms 
while having \file{/usr/local/plat} be a different filesystem for each 
platform.
\end{cfuncdesc}

\begin{cfuncdesc}{char*}{Py_GetProgramFullPath}{}
Return the full program name of the Python executable; this is 
computed as a side-effect of deriving the default module search path 
from the program name (set by
\cfunction{Py_SetProgramName()}\ttindex{Py_SetProgramName()} above).
The returned string points into static storage; the caller should not 
modify its value.  The value is available to Python code as 
\code{sys.executable}.
\withsubitem{(in module sys)}{\ttindex{executable}}
\end{cfuncdesc}

\begin{cfuncdesc}{char*}{Py_GetPath}{}
\indexiii{module}{search}{path}
Return the default module search path; this is computed from the 
program name (set by \cfunction{Py_SetProgramName()} above) and some 
environment variables.  The returned string consists of a series of 
directory names separated by a platform dependent delimiter character.  
The delimiter character is \character{:} on \UNIX{}, \character{;} on
DOS/Windows, and \character{\e n} (the \ASCII{} newline character) on
Macintosh.  The returned string points into static storage; the caller
should not modify its value.  The value is available to Python code 
as the list \code{sys.path}\withsubitem{(in module sys)}{\ttindex{path}},
which may be modified to change the future search path for loaded
modules.

% XXX should give the exact rules
\end{cfuncdesc}

\begin{cfuncdesc}{const char*}{Py_GetVersion}{}
Return the version of this Python interpreter.  This is a string that 
looks something like

\begin{verbatim}
"1.5 (#67, Dec 31 1997, 22:34:28) [GCC 2.7.2.2]"
\end{verbatim}

The first word (up to the first space character) is the current Python 
version; the first three characters are the major and minor version 
separated by a period.  The returned string points into static storage; 
the caller should not modify its value.  The value is available to 
Python code as the list \code{sys.version}.
\withsubitem{(in module sys)}{\ttindex{version}}
\end{cfuncdesc}

\begin{cfuncdesc}{const char*}{Py_GetPlatform}{}
Return the platform identifier for the current platform.  On \UNIX{}, 
this is formed from the ``official'' name of the operating system, 
converted to lower case, followed by the major revision number; e.g., 
for Solaris 2.x, which is also known as SunOS 5.x, the value is 
\code{'sunos5'}.  On Macintosh, it is \code{'mac'}.  On Windows, it 
is \code{'win'}.  The returned string points into static storage; 
the caller should not modify its value.  The value is available to 
Python code as \code{sys.platform}.
\withsubitem{(in module sys)}{\ttindex{platform}}
\end{cfuncdesc}

\begin{cfuncdesc}{const char*}{Py_GetCopyright}{}
Return the official copyright string for the current Python version, 
for example

\code{'Copyright 1991-1995 Stichting Mathematisch Centrum, Amsterdam'}

The returned string points into static storage; the caller should not 
modify its value.  The value is available to Python code as the list 
\code{sys.copyright}.
\withsubitem{(in module sys)}{\ttindex{copyright}}
\end{cfuncdesc}

\begin{cfuncdesc}{const char*}{Py_GetCompiler}{}
Return an indication of the compiler used to build the current Python 
version, in square brackets, for example:

\begin{verbatim}
"[GCC 2.7.2.2]"
\end{verbatim}

The returned string points into static storage; the caller should not 
modify its value.  The value is available to Python code as part of 
the variable \code{sys.version}.
\withsubitem{(in module sys)}{\ttindex{version}}
\end{cfuncdesc}

\begin{cfuncdesc}{const char*}{Py_GetBuildInfo}{}
Return information about the sequence number and build date and time 
of the current Python interpreter instance, for example

\begin{verbatim}
"#67, Aug  1 1997, 22:34:28"
\end{verbatim}

The returned string points into static storage; the caller should not 
modify its value.  The value is available to Python code as part of 
the variable \code{sys.version}.
\withsubitem{(in module sys)}{\ttindex{version}}
\end{cfuncdesc}

\begin{cfuncdesc}{int}{PySys_SetArgv}{int argc, char **argv}
Set \code{sys.argv} based on \var{argc} and \var{argv}.  These
parameters are similar to those passed to the program's
\cfunction{main()}\ttindex{main()} function with the difference that
the first entry should refer to the script file to be executed rather
than the executable hosting the Python interpreter.  If there isn't a
script that will be run, the first entry in \var{argv} can be an empty
string.  If this function fails to initialize \code{sys.argv}, a fatal 
condition is signalled using
\cfunction{Py_FatalError()}\ttindex{Py_FatalError()}.
\withsubitem{(in module sys)}{\ttindex{argv}}
% XXX impl. doesn't seem consistent in allowing 0/NULL for the params; 
% check w/ Guido.
\end{cfuncdesc}

% XXX Other PySys thingies (doesn't really belong in this chapter)

\section{Thread State and the Global Interpreter Lock
         \label{threads}}

\index{global interpreter lock}
\index{interpreter lock}
\index{lock, interpreter}

The Python interpreter is not fully thread safe.  In order to support
multi-threaded Python programs, there's a global lock that must be
held by the current thread before it can safely access Python objects.
Without the lock, even the simplest operations could cause problems in
a multi-threaded program: for example, when two threads simultaneously
increment the reference count of the same object, the reference count
could end up being incremented only once instead of twice.

Therefore, the rule exists that only the thread that has acquired the
global interpreter lock may operate on Python objects or call Python/C
API functions.  In order to support multi-threaded Python programs,
the interpreter regularly releases and reacquires the lock --- by
default, every ten bytecode instructions (this can be changed with
\withsubitem{(in module sys)}{\ttindex{setcheckinterval()}}
\function{sys.setcheckinterval()}).  The lock is also released and
reacquired around potentially blocking I/O operations like reading or
writing a file, so that other threads can run while the thread that
requests the I/O is waiting for the I/O operation to complete.

The Python interpreter needs to keep some bookkeeping information
separate per thread --- for this it uses a data structure called
\ctype{PyThreadState}\ttindex{PyThreadState}.  This is new in Python
1.5; in earlier versions, such state was stored in global variables,
and switching threads could cause problems.  In particular, exception
handling is now thread safe, when the application uses
\withsubitem{(in module sys)}{\ttindex{exc_info()}}
\function{sys.exc_info()} to access the exception last raised in the
current thread.

There's one global variable left, however: the pointer to the current
\ctype{PyThreadState}\ttindex{PyThreadState} structure.  While most
thread packages have a way to store ``per-thread global data,''
Python's internal platform independent thread abstraction doesn't
support this yet.  Therefore, the current thread state must be
manipulated explicitly.

This is easy enough in most cases.  Most code manipulating the global
interpreter lock has the following simple structure:

\begin{verbatim}
Save the thread state in a local variable.
Release the interpreter lock.
...Do some blocking I/O operation...
Reacquire the interpreter lock.
Restore the thread state from the local variable.
\end{verbatim}

This is so common that a pair of macros exists to simplify it:

\begin{verbatim}
Py_BEGIN_ALLOW_THREADS
...Do some blocking I/O operation...
Py_END_ALLOW_THREADS
\end{verbatim}

The \code{Py_BEGIN_ALLOW_THREADS}\ttindex{Py_BEGIN_ALLOW_THREADS} macro
opens a new block and declares a hidden local variable; the
\code{Py_END_ALLOW_THREADS}\ttindex{Py_END_ALLOW_THREADS} macro closes 
the block.  Another advantage of using these two macros is that when
Python is compiled without thread support, they are defined empty,
thus saving the thread state and lock manipulations.

When thread support is enabled, the block above expands to the
following code:

\begin{verbatim}
    PyThreadState *_save;

    _save = PyEval_SaveThread();
    ...Do some blocking I/O operation...
    PyEval_RestoreThread(_save);
\end{verbatim}

Using even lower level primitives, we can get roughly the same effect
as follows:

\begin{verbatim}
    PyThreadState *_save;

    _save = PyThreadState_Swap(NULL);
    PyEval_ReleaseLock();
    ...Do some blocking I/O operation...
    PyEval_AcquireLock();
    PyThreadState_Swap(_save);
\end{verbatim}

There are some subtle differences; in particular,
\cfunction{PyEval_RestoreThread()}\ttindex{PyEval_RestoreThread()} saves
and restores the value of the  global variable
\cdata{errno}\ttindex{errno}, since the lock manipulation does not
guarantee that \cdata{errno} is left alone.  Also, when thread support
is disabled,
\cfunction{PyEval_SaveThread()}\ttindex{PyEval_SaveThread()} and
\cfunction{PyEval_RestoreThread()} don't manipulate the lock; in this
case, \cfunction{PyEval_ReleaseLock()}\ttindex{PyEval_ReleaseLock()} and
\cfunction{PyEval_AcquireLock()}\ttindex{PyEval_AcquireLock()} are not
available.  This is done so that dynamically loaded extensions
compiled with thread support enabled can be loaded by an interpreter
that was compiled with disabled thread support.

The global interpreter lock is used to protect the pointer to the
current thread state.  When releasing the lock and saving the thread
state, the current thread state pointer must be retrieved before the
lock is released (since another thread could immediately acquire the
lock and store its own thread state in the global variable).
Conversely, when acquiring the lock and restoring the thread state,
the lock must be acquired before storing the thread state pointer.

Why am I going on with so much detail about this?  Because when
threads are created from C, they don't have the global interpreter
lock, nor is there a thread state data structure for them.  Such
threads must bootstrap themselves into existence, by first creating a
thread state data structure, then acquiring the lock, and finally
storing their thread state pointer, before they can start using the
Python/C API.  When they are done, they should reset the thread state
pointer, release the lock, and finally free their thread state data
structure.

When creating a thread data structure, you need to provide an
interpreter state data structure.  The interpreter state data
structure hold global data that is shared by all threads in an
interpreter, for example the module administration
(\code{sys.modules}).  Depending on your needs, you can either create
a new interpreter state data structure, or share the interpreter state
data structure used by the Python main thread (to access the latter,
you must obtain the thread state and access its \member{interp} member;
this must be done by a thread that is created by Python or by the main
thread after Python is initialized).


\begin{ctypedesc}{PyInterpreterState}
This data structure represents the state shared by a number of
cooperating threads.  Threads belonging to the same interpreter
share their module administration and a few other internal items.
There are no public members in this structure.

Threads belonging to different interpreters initially share nothing,
except process state like available memory, open file descriptors and
such.  The global interpreter lock is also shared by all threads,
regardless of to which interpreter they belong.
\end{ctypedesc}

\begin{ctypedesc}{PyThreadState}
This data structure represents the state of a single thread.  The only
public data member is \ctype{PyInterpreterState *}\member{interp},
which points to this thread's interpreter state.
\end{ctypedesc}

\begin{cfuncdesc}{void}{PyEval_InitThreads}{}
Initialize and acquire the global interpreter lock.  It should be
called in the main thread before creating a second thread or engaging
in any other thread operations such as
\cfunction{PyEval_ReleaseLock()}\ttindex{PyEval_ReleaseLock()} or
\code{PyEval_ReleaseThread(\var{tstate})}\ttindex{PyEval_ReleaseThread()}.
It is not needed before calling
\cfunction{PyEval_SaveThread()}\ttindex{PyEval_SaveThread()} or
\cfunction{PyEval_RestoreThread()}\ttindex{PyEval_RestoreThread()}.

This is a no-op when called for a second time.  It is safe to call
this function before calling
\cfunction{Py_Initialize()}\ttindex{Py_Initialize()}.

When only the main thread exists, no lock operations are needed.  This
is a common situation (most Python programs do not use threads), and
the lock operations slow the interpreter down a bit.  Therefore, the
lock is not created initially.  This situation is equivalent to having
acquired the lock: when there is only a single thread, all object
accesses are safe.  Therefore, when this function initializes the
lock, it also acquires it.  Before the Python
\module{thread}\refbimodindex{thread} module creates a new thread,
knowing that either it has the lock or the lock hasn't been created
yet, it calls \cfunction{PyEval_InitThreads()}.  When this call
returns, it is guaranteed that the lock has been created and that it
has acquired it.

It is \strong{not} safe to call this function when it is unknown which
thread (if any) currently has the global interpreter lock.

This function is not available when thread support is disabled at
compile time.
\end{cfuncdesc}

\begin{cfuncdesc}{void}{PyEval_AcquireLock}{}
Acquire the global interpreter lock.  The lock must have been created
earlier.  If this thread already has the lock, a deadlock ensues.
This function is not available when thread support is disabled at
compile time.
\end{cfuncdesc}

\begin{cfuncdesc}{void}{PyEval_ReleaseLock}{}
Release the global interpreter lock.  The lock must have been created
earlier.  This function is not available when thread support is
disabled at compile time.
\end{cfuncdesc}

\begin{cfuncdesc}{void}{PyEval_AcquireThread}{PyThreadState *tstate}
Acquire the global interpreter lock and then set the current thread
state to \var{tstate}, which should not be \NULL{}.  The lock must
have been created earlier.  If this thread already has the lock,
deadlock ensues.  This function is not available when thread support
is disabled at compile time.
\end{cfuncdesc}

\begin{cfuncdesc}{void}{PyEval_ReleaseThread}{PyThreadState *tstate}
Reset the current thread state to \NULL{} and release the global
interpreter lock.  The lock must have been created earlier and must be
held by the current thread.  The \var{tstate} argument, which must not
be \NULL{}, is only used to check that it represents the current
thread state --- if it isn't, a fatal error is reported.  This
function is not available when thread support is disabled at compile
time.
\end{cfuncdesc}

\begin{cfuncdesc}{PyThreadState*}{PyEval_SaveThread}{}
Release the interpreter lock (if it has been created and thread
support is enabled) and reset the thread state to \NULL{},
returning the previous thread state (which is not \NULL{}).  If
the lock has been created, the current thread must have acquired it.
(This function is available even when thread support is disabled at
compile time.)
\end{cfuncdesc}

\begin{cfuncdesc}{void}{PyEval_RestoreThread}{PyThreadState *tstate}
Acquire the interpreter lock (if it has been created and thread
support is enabled) and set the thread state to \var{tstate}, which
must not be \NULL{}.  If the lock has been created, the current
thread must not have acquired it, otherwise deadlock ensues.  (This
function is available even when thread support is disabled at compile
time.)
\end{cfuncdesc}

The following macros are normally used without a trailing semicolon;
look for example usage in the Python source distribution.

\begin{csimplemacrodesc}{Py_BEGIN_ALLOW_THREADS}
This macro expands to
\samp{\{ PyThreadState *_save; _save = PyEval_SaveThread();}.
Note that it contains an opening brace; it must be matched with a
following \code{Py_END_ALLOW_THREADS} macro.  See above for further
discussion of this macro.  It is a no-op when thread support is
disabled at compile time.
\end{csimplemacrodesc}

\begin{csimplemacrodesc}{Py_END_ALLOW_THREADS}
This macro expands to
\samp{PyEval_RestoreThread(_save); \}}.
Note that it contains a closing brace; it must be matched with an
earlier \code{Py_BEGIN_ALLOW_THREADS} macro.  See above for further
discussion of this macro.  It is a no-op when thread support is
disabled at compile time.
\end{csimplemacrodesc}

\begin{csimplemacrodesc}{Py_BEGIN_BLOCK_THREADS}
This macro expands to \samp{PyEval_RestoreThread(_save);} i.e. it
is equivalent to \code{Py_END_ALLOW_THREADS} without the closing
brace.  It is a no-op when thread support is disabled at compile
time.
\end{csimplemacrodesc}

\begin{csimplemacrodesc}{Py_BEGIN_UNBLOCK_THREADS}
This macro expands to \samp{_save = PyEval_SaveThread();} i.e. it is
equivalent to \code{Py_BEGIN_ALLOW_THREADS} without the opening brace
and variable declaration.  It is a no-op when thread support is
disabled at compile time.
\end{csimplemacrodesc}

All of the following functions are only available when thread support
is enabled at compile time, and must be called only when the
interpreter lock has been created.

\begin{cfuncdesc}{PyInterpreterState*}{PyInterpreterState_New}{}
Create a new interpreter state object.  The interpreter lock need not
be held, but may be held if it is necessary to serialize calls to this
function.
\end{cfuncdesc}

\begin{cfuncdesc}{void}{PyInterpreterState_Clear}{PyInterpreterState *interp}
Reset all information in an interpreter state object.  The interpreter
lock must be held.
\end{cfuncdesc}

\begin{cfuncdesc}{void}{PyInterpreterState_Delete}{PyInterpreterState *interp}
Destroy an interpreter state object.  The interpreter lock need not be
held.  The interpreter state must have been reset with a previous
call to \cfunction{PyInterpreterState_Clear()}.
\end{cfuncdesc}

\begin{cfuncdesc}{PyThreadState*}{PyThreadState_New}{PyInterpreterState *interp}
Create a new thread state object belonging to the given interpreter
object.  The interpreter lock need not be held, but may be held if it
is necessary to serialize calls to this function.
\end{cfuncdesc}

\begin{cfuncdesc}{void}{PyThreadState_Clear}{PyThreadState *tstate}
Reset all information in a thread state object.  The interpreter lock
must be held.
\end{cfuncdesc}

\begin{cfuncdesc}{void}{PyThreadState_Delete}{PyThreadState *tstate}
Destroy a thread state object.  The interpreter lock need not be
held.  The thread state must have been reset with a previous
call to \cfunction{PyThreadState_Clear()}.
\end{cfuncdesc}

\begin{cfuncdesc}{PyThreadState*}{PyThreadState_Get}{}
Return the current thread state.  The interpreter lock must be held.
When the current thread state is \NULL{}, this issues a fatal
error (so that the caller needn't check for \NULL{}).
\end{cfuncdesc}

\begin{cfuncdesc}{PyThreadState*}{PyThreadState_Swap}{PyThreadState *tstate}
Swap the current thread state with the thread state given by the
argument \var{tstate}, which may be \NULL{}.  The interpreter lock
must be held.
\end{cfuncdesc}


\chapter{Memory Management \label{memory}}
\sectionauthor{Vladimir Marangozov}{Vladimir.Marangozov@inrialpes.fr}


\section{Overview \label{memoryOverview}}

Memory management in Python involves a private heap containing all
Python objects and data structures. The management of this private
heap is ensured internally by the \emph{Python memory manager}.  The
Python memory manager has different components which deal with various
dynamic storage management aspects, like sharing, segmentation,
preallocation or caching.

At the lowest level, a raw memory allocator ensures that there is
enough room in the private heap for storing all Python-related data
by interacting with the memory manager of the operating system. On top
of the raw memory allocator, several object-specific allocators
operate on the same heap and implement distinct memory management
policies adapted to the peculiarities of every object type. For
example, integer objects are managed differently within the heap than
strings, tuples or dictionaries because integers imply different
storage requirements and speed/space tradeoffs. The Python memory
manager thus delegates some of the work to the object-specific
allocators, but ensures that the latter operate within the bounds of
the private heap.

It is important to understand that the management of the Python heap
is performed by the interpreter itself and that the user has no
control on it, even if she regularly manipulates object pointers to
memory blocks inside that heap.  The allocation of heap space for
Python objects and other internal buffers is performed on demand by
the Python memory manager through the Python/C API functions listed in
this document.

To avoid memory corruption, extension writers should never try to
operate on Python objects with the functions exported by the C
library: \cfunction{malloc()}\ttindex{malloc()},
\cfunction{calloc()}\ttindex{calloc()},
\cfunction{realloc()}\ttindex{realloc()} and
\cfunction{free()}\ttindex{free()}.  This will result in 
mixed calls between the C allocator and the Python memory manager
with fatal consequences, because they implement different algorithms
and operate on different heaps.  However, one may safely allocate and
release memory blocks with the C library allocator for individual
purposes, as shown in the following example:

\begin{verbatim}
    PyObject *res;
    char *buf = (char *) malloc(BUFSIZ); /* for I/O */

    if (buf == NULL)
        return PyErr_NoMemory();
    ...Do some I/O operation involving buf...
    res = PyString_FromString(buf);
    free(buf); /* malloc'ed */
    return res;
\end{verbatim}

In this example, the memory request for the I/O buffer is handled by
the C library allocator. The Python memory manager is involved only
in the allocation of the string object returned as a result.

In most situations, however, it is recommended to allocate memory from
the Python heap specifically because the latter is under control of
the Python memory manager. For example, this is required when the
interpreter is extended with new object types written in C. Another
reason for using the Python heap is the desire to \emph{inform} the
Python memory manager about the memory needs of the extension module.
Even when the requested memory is used exclusively for internal,
highly-specific purposes, delegating all memory requests to the Python
memory manager causes the interpreter to have a more accurate image of
its memory footprint as a whole. Consequently, under certain
circumstances, the Python memory manager may or may not trigger
appropriate actions, like garbage collection, memory compaction or
other preventive procedures. Note that by using the C library
allocator as shown in the previous example, the allocated memory for
the I/O buffer escapes completely the Python memory manager.


\section{Memory Interface \label{memoryInterface}}

The following function sets, modeled after the ANSI C standard, are
available for allocating and releasing memory from the Python heap:


\begin{cfuncdesc}{void*}{PyMem_Malloc}{size_t n}
Allocates \var{n} bytes and returns a pointer of type \ctype{void*} to
the allocated memory, or \NULL{} if the request fails. Requesting zero
bytes returns a non-\NULL{} pointer.
\end{cfuncdesc}

\begin{cfuncdesc}{void*}{PyMem_Realloc}{void *p, size_t n}
Resizes the memory block pointed to by \var{p} to \var{n} bytes. The
contents will be unchanged to the minimum of the old and the new
sizes. If \var{p} is \NULL{}, the call is equivalent to
\cfunction{PyMem_Malloc(\var{n})}; if \var{n} is equal to zero, the memory block
is resized but is not freed, and the returned pointer is non-\NULL{}.
Unless \var{p} is \NULL{}, it must have been returned by a previous
call to \cfunction{PyMem_Malloc()} or \cfunction{PyMem_Realloc()}.
\end{cfuncdesc}

\begin{cfuncdesc}{void}{PyMem_Free}{void *p}
Frees the memory block pointed to by \var{p}, which must have been
returned by a previous call to \cfunction{PyMem_Malloc()} or
\cfunction{PyMem_Realloc()}.  Otherwise, or if
\cfunction{PyMem_Free(p)} has been called before, undefined behaviour
occurs. If \var{p} is \NULL{}, no operation is performed.
\end{cfuncdesc}

The following type-oriented macros are provided for convenience.  Note 
that \var{TYPE} refers to any C type.

\begin{cfuncdesc}{\var{TYPE}*}{PyMem_New}{TYPE, size_t n}
Same as \cfunction{PyMem_Malloc()}, but allocates \code{(\var{n} *
sizeof(\var{TYPE}))} bytes of memory.  Returns a pointer cast to
\ctype{\var{TYPE}*}.
\end{cfuncdesc}

\begin{cfuncdesc}{\var{TYPE}*}{PyMem_Resize}{void *p, TYPE, size_t n}
Same as \cfunction{PyMem_Realloc()}, but the memory block is resized
to \code{(\var{n} * sizeof(\var{TYPE}))} bytes.  Returns a pointer
cast to \ctype{\var{TYPE}*}.
\end{cfuncdesc}

\begin{cfuncdesc}{void}{PyMem_Del}{void *p}
Same as \cfunction{PyMem_Free()}.
\end{cfuncdesc}

In addition, the following macro sets are provided for calling the
Python memory allocator directly, without involving the C API functions
listed above. However, note that their use does not preserve binary
compatibility accross Python versions and is therefore deprecated in
extension modules.

\cfunction{PyMem_MALLOC()}, \cfunction{PyMem_REALLOC()}, \cfunction{PyMem_FREE()}.

\cfunction{PyMem_NEW()}, \cfunction{PyMem_RESIZE()}, \cfunction{PyMem_DEL()}.


\section{Examples \label{memoryExamples}}

Here is the example from section \ref{memoryOverview}, rewritten so
that the I/O buffer is allocated from the Python heap by using the
first function set:

\begin{verbatim}
    PyObject *res;
    char *buf = (char *) PyMem_Malloc(BUFSIZ); /* for I/O */

    if (buf == NULL)
        return PyErr_NoMemory();
    /* ...Do some I/O operation involving buf... */
    res = PyString_FromString(buf);
    PyMem_Free(buf); /* allocated with PyMem_Malloc */
    return res;
\end{verbatim}

The same code using the type-oriented function set:

\begin{verbatim}
    PyObject *res;
    char *buf = PyMem_New(char, BUFSIZ); /* for I/O */

    if (buf == NULL)
        return PyErr_NoMemory();
    /* ...Do some I/O operation involving buf... */
    res = PyString_FromString(buf);
    PyMem_Del(buf); /* allocated with PyMem_New */
    return res;
\end{verbatim}

Note that in the two examples above, the buffer is always
manipulated via functions belonging to the same set. Indeed, it
is required to use the same memory API family for a given
memory block, so that the risk of mixing different allocators is
reduced to a minimum. The following code sequence contains two errors,
one of which is labeled as \emph{fatal} because it mixes two different
allocators operating on different heaps.

\begin{verbatim}
char *buf1 = PyMem_New(char, BUFSIZ);
char *buf2 = (char *) malloc(BUFSIZ);
char *buf3 = (char *) PyMem_Malloc(BUFSIZ);
...
PyMem_Del(buf3);  /* Wrong -- should be PyMem_Free() */
free(buf2);       /* Right -- allocated via malloc() */
free(buf1);       /* Fatal -- should be PyMem_Del()  */
\end{verbatim}

In addition to the functions aimed at handling raw memory blocks from
the Python heap, objects in Python are allocated and released with
\cfunction{PyObject_New()}, \cfunction{PyObject_NewVar()} and
\cfunction{PyObject_Del()}, or with their corresponding macros
\cfunction{PyObject_NEW()}, \cfunction{PyObject_NEW_VAR()} and
\cfunction{PyObject_DEL()}.

These will be explained in the next chapter on defining and
implementing new object types in C.


\chapter{Defining New Object Types \label{newTypes}}

\begin{cfuncdesc}{PyObject*}{_PyObject_New}{PyTypeObject *type}
\end{cfuncdesc}

\begin{cfuncdesc}{PyVarObject*}{_PyObject_NewVar}{PyTypeObject *type, int size}
\end{cfuncdesc}

\begin{cfuncdesc}{void}{_PyObject_Del}{PyObject *op}
\end{cfuncdesc}

\begin{cfuncdesc}{PyObject*}{PyObject_Init}{PyObject *op,
						PyTypeObject *type}
\end{cfuncdesc}

\begin{cfuncdesc}{PyVarObject*}{PyObject_InitVar}{PyVarObject *op,
						PyTypeObject *type, int size}
\end{cfuncdesc}

\begin{cfuncdesc}{\var{TYPE}*}{PyObject_New}{TYPE, PyTypeObject *type}
\end{cfuncdesc}

\begin{cfuncdesc}{\var{TYPE}*}{PyObject_NewVar}{TYPE, PyTypeObject *type,
                                                int size}
\end{cfuncdesc}

\begin{cfuncdesc}{void}{PyObject_Del}{PyObject *op}
\end{cfuncdesc}

\begin{cfuncdesc}{\var{TYPE}*}{PyObject_NEW}{TYPE, PyTypeObject *type}
\end{cfuncdesc}

\begin{cfuncdesc}{\var{TYPE}*}{PyObject_NEW_VAR}{TYPE, PyTypeObject *type,
                                                int size}
\end{cfuncdesc}

\begin{cfuncdesc}{void}{PyObject_DEL}{PyObject *op}
\end{cfuncdesc}

\begin{cfuncdesc}{PyObject*}{Py_InitModule}{char *name,
                                            PyMethodDef *methods}
  Create a new module object based on a name and table of functions,
  returning the new module object.
\end{cfuncdesc}

\begin{cfuncdesc}{PyObject*}{Py_InitModule3}{char *name,
                                             PyMethodDef *methods,
                                             char *doc}
  Create a new module object based on a name and table of functions,
  returning the new module object.  If \var{doc} is non-\NULL, it will
  be used to define the docstring for the module.
\end{cfuncdesc}

\begin{cfuncdesc}{PyObject*}{Py_InitModule4}{char *name,
                                             PyMethodDef *methods,
                                             char *doc, PyObject *self,
                                             int apiver}
  Create a new module object based on a name and table of functions,
  returning the new module object.  If \var{doc} is non-\NULL, it will
  be used to define the docstring for the module.  If \var{self} is
  non-\NULL, it will passed to the functions of the module as their
  (otherwise \NULL) first parameter.  (This was added as an
  experimental feature, and there are no known uses in the current
  version of Python.)  For \var{apiver}, the only value which should
  be passed is defined by the constant \constant{PYTHON_API_VERSION}.

  \strong{Note:}  Most uses of this function should probably be using
  the \cfunction{Py_InitModule3()} instead; only use this if you are
  sure you need it.
\end{cfuncdesc}

PyArg_ParseTupleAndKeywords, PyArg_ParseTuple, PyArg_Parse

Py_BuildValue

DL_IMPORT

_Py_NoneStruct


\section{Common Object Structures \label{common-structs}}

PyObject, PyVarObject

PyObject_HEAD, PyObject_HEAD_INIT, PyObject_VAR_HEAD

Typedefs:
unaryfunc, binaryfunc, ternaryfunc, inquiry, coercion, intargfunc,
intintargfunc, intobjargproc, intintobjargproc, objobjargproc,
destructor, printfunc, getattrfunc, getattrofunc, setattrfunc,
setattrofunc, cmpfunc, reprfunc, hashfunc

\begin{ctypedesc}{PyCFunction}
Type of the functions used to implement most Python callables in C.
\end{ctypedesc}

\begin{ctypedesc}{PyMethodDef}
Structure used to describe a method of an extension type.  This
structure has four fields:

\begin{tableiii}{l|l|l}{member}{Field}{C Type}{Meaning}
  \lineiii{ml_name}{char *}{name of the method}
  \lineiii{ml_meth}{PyCFunction}{pointer to the C implementation}
  \lineiii{ml_flags}{int}{flag bits indicating how the call should be
                          constructed}
  \lineiii{ml_doc}{char *}{points to the contents of the docstring}
\end{tableiii}
\end{ctypedesc}

\begin{cfuncdesc}{PyObject*}{Py_FindMethod}{PyMethodDef[] table,
                                            PyObject *ob, char *name}
Return a bound method object for an extension type implemented in C.
This function also handles the special attribute \member{__methods__},
returning a list of all the method names defined in \var{table}.
\end{cfuncdesc}


\section{Mapping Object Structures \label{mapping-structs}}

\begin{ctypedesc}{PyMappingMethods}
Structure used to hold pointers to the functions used to implement the 
mapping protocol for an extension type.
\end{ctypedesc}


\section{Number Object Structures \label{number-structs}}

\begin{ctypedesc}{PyNumberMethods}
Structure used to hold pointers to the functions an extension type
uses to implement the number protocol.
\end{ctypedesc}


\section{Sequence Object Structures \label{sequence-structs}}

\begin{ctypedesc}{PySequenceMethods}
Structure used to hold pointers to the functions which an object uses
to implement the sequence protocol.
\end{ctypedesc}


\section{Buffer Object Structures \label{buffer-structs}}
\sectionauthor{Greg J. Stein}{greg@lyra.org}

The buffer interface exports a model where an object can expose its
internal data as a set of chunks of data, where each chunk is
specified as a pointer/length pair.  These chunks are called
\dfn{segments} and are presumed to be non-contiguous in memory.

If an object does not export the buffer interface, then its
\member{tp_as_buffer} member in the \ctype{PyTypeObject} structure
should be \NULL{}.  Otherwise, the \member{tp_as_buffer} will point to
a \ctype{PyBufferProcs} structure.

\strong{Note:} It is very important that your
\ctype{PyTypeObject} structure uses \constant{Py_TPFLAGS_DEFAULT} for
the value of the \member{tp_flags} member rather than \code{0}.  This
tells the Python runtime that your \ctype{PyBufferProcs} structure
contains the \member{bf_getcharbuffer} slot. Older versions of Python
did not have this member, so a new Python interpreter using an old
extension needs to be able to test for its presence before using it.

\begin{ctypedesc}{PyBufferProcs}
Structure used to hold the function pointers which define an
implementation of the buffer protocol.

The first slot is \member{bf_getreadbuffer}, of type
\ctype{getreadbufferproc}.  If this slot is \NULL{}, then the object
does not support reading from the internal data.  This is
non-sensical, so implementors should fill this in, but callers should
test that the slot contains a non-\NULL{} value.

The next slot is \member{bf_getwritebuffer} having type
\ctype{getwritebufferproc}. This slot may be \NULL{} if the object
does not allow writing into its returned buffers.

The third slot is \member{bf_getsegcount}, with type
\ctype{getsegcountproc}.  This slot must not be \NULL{} and is used to 
inform the caller how many segments the object contains.  Simple
objects such as \ctype{PyString_Type} and
\ctype{PyBuffer_Type} objects contain a single segment.

The last slot is \member{bf_getcharbuffer}, of type
\ctype{getcharbufferproc}.  This slot will only be present if the
\constant{Py_TPFLAGS_HAVE_GETCHARBUFFER} flag is present in the
\member{tp_flags} field of the object's \ctype{PyTypeObject}.  Before using
this slot, the caller should test whether it is present by using the
\cfunction{PyType_HasFeature()}\ttindex{PyType_HasFeature()} function.
If present, it may be \NULL, indicating that the object's contents
cannot be used as \emph{8-bit characters}.
The slot function may also raise an error if the object's contents
cannot be interpreted as 8-bit characters.  For example, if the object
is an array which is configured to hold floating point values, an
exception may be raised if a caller attempts to use
\member{bf_getcharbuffer} to fetch a sequence of 8-bit characters.
This notion of exporting the internal buffers as ``text'' is used to
distinguish between objects that are binary in nature, and those which
have character-based content.

\strong{Note:} The current policy seems to state that these characters
may be multi-byte characters. This implies that a buffer size of
\var{N} does not mean there are \var{N} characters present.
\end{ctypedesc}

\begin{datadesc}{Py_TPFLAGS_HAVE_GETCHARBUFFER}
Flag bit set in the type structure to indicate that the
\member{bf_getcharbuffer} slot is known.  This being set does not
indicate that the object supports the buffer interface or that the
\member{bf_getcharbuffer} slot is non-\NULL.
\end{datadesc}

\begin{ctypedesc}[getreadbufferproc]{int (*getreadbufferproc)
                            (PyObject *self, int segment, void **ptrptr)}
Return a pointer to a readable segment of the buffer.  This function
is allowed to raise an exception, in which case it must return
\code{-1}.  The \var{segment} which is passed must be zero or
positive, and strictly less than the number of segments returned by
the \member{bf_getsegcount} slot function.  On success, it returns the
length of the buffer memory, and sets \code{*\var{ptrptr}} to a
pointer to that memory.
\end{ctypedesc}

\begin{ctypedesc}[getwritebufferproc]{int (*getwritebufferproc)
                            (PyObject *self, int segment, void **ptrptr)}
Return a pointer to a writable memory buffer in \code{*\var{ptrptr}},
and the length of that segment as the function return value.
The memory buffer must correspond to buffer segment \var{segment}.
Must return \code{-1} and set an exception on error.
\exception{TypeError} should be raised if the object only supports
read-only buffers, and \exception{SystemError} should be raised when
\var{segment} specifies a segment that doesn't exist.
% Why doesn't it raise ValueError for this one?
% GJS: because you shouldn't be calling it with an invalid
%      segment. That indicates a blatant programming error in the C
%      code.
\end{ctypedesc}

\begin{ctypedesc}[getsegcountproc]{int (*getsegcountproc)
                            (PyObject *self, int *lenp)}
Return the number of memory segments which comprise the buffer.  If
\var{lenp} is not \NULL, the implementation must report the sum of the 
sizes (in bytes) of all segments in \code{*\var{lenp}}.
The function cannot fail.
\end{ctypedesc}

\begin{ctypedesc}[getcharbufferproc]{int (*getcharbufferproc)
                            (PyObject *self, int segment, const char **ptrptr)}
\end{ctypedesc}


\section{Supporting Cyclic Garbarge Collection
         \label{supporting-cycle-detection}}

Python's support for detecting and collecting garbage which involves
circular references requires support from object types which are
``containers'' for other objects which may also be containers.  Types
which do not store references to other objects, or which only store
references to atomic types (such as numbers or strings), do not need
to provide any explicit support for garbage collection.

To create a container type, the \member{tp_flags} field of the type
object must include the \constant{Py_TPFLAGS_GC} and provide an
implementation of the \member{tp_traverse} handler.  The computed
value of the \member{tp_basicsize} field must include
\constant{PyGC_HEAD_SIZE} as well.  If instances of the type are
mutable, a \member{tp_clear} implementation must also be provided.

\begin{datadesc}{Py_TPFLAGS_GC}
  Objects with a type with this flag set must conform with the rules
  documented here.  For convenience these objects will be referred to
  as container objects.
\end{datadesc}

\begin{datadesc}{PyGC_HEAD_SIZE}
  Extra memory needed for the garbage collector.  Container objects
  must include this in the calculation of their tp_basicsize.  If the
  collector is disabled at compile time then this is \code{0}.
\end{datadesc}

Constructors for container types must conform to two rules:

\begin{enumerate}
\item  The memory for the object must be allocated using
       \cfunction{PyObject_New()} or \cfunction{PyObject_VarNew()}.

\item  Once all the fields which may contain references to other
       containers are initialized, it must call
       \cfunction{PyObject_GC_Init()}.
\end{enumerate}

\begin{cfuncdesc}{void}{PyObject_GC_Init}{PyObject *op}
  Adds the object \var{op} to the set of container objects tracked by
  the collector.  The collector can run at unexpected times so objects
  must be valid while being tracked.  This should be called once all
  the fields followed by the \member{tp_traverse} handler become valid,
  usually near the end of the constructor.
\end{cfuncdesc}

Similarly, the deallocator for the object must conform to a similar
pair of rules:

\begin{enumerate}
\item  Before fields which refer to other containers are invalidated,
       \cfunction{PyObject_GC_Fini()} must be called.

\item  The object's memory must be deallocated using
       \cfunction{PyObject_Del()}.
\end{enumerate}

\begin{cfuncdesc}{void}{PyObject_GC_Fini}{PyObject *op}
  Remove the object \var{op} from the set of container objects tracked
  by the collector.  Note that \cfunction{PyObject_GC_Init()} can be
  called again on this object to add it back to the set of tracked
  objects.  The deallocator (\member{tp_dealloc} handler) should call
  this for the object before any of the fields used by the
  \member{tp_traverse} handler become invalid.

  \strong{Note:}  Any container which may be referenced from another
  object reachable by the collector must itself be tracked by the
  collector, so it is generally not safe to call this function
  anywhere but in the object's deallocator.
\end{cfuncdesc}

The \member{tp_traverse} handler accepts a function parameter of this
type:

\begin{ctypedesc}[visitproc]{int (*visitproc)(PyObject *object, void *arg)}
  Type of the visitor function passed to the \member{tp_traverse}
  handler.  The function should be called with an object to traverse
  as \var{object} and the third parameter to the \member{tp_traverse}
  handler as \var{arg}.
\end{ctypedesc}

The \member{tp_traverse} handler must have the following type:

\begin{ctypedesc}[traverseproc]{int (*traverseproc)(PyObject *self,
                                visitproc visit, void *arg)}
  Traversal function for a container object.  Implementations must
  call the \var{visit} function for each object directly contained by
  \var{self}, with the parameters to \var{visit} being the contained
  object and the \var{arg} value passed to the handler.  If
  \var{visit} returns a non-zero value then an error has occurred and
  that value should be returned immediately.
\end{ctypedesc}

The \member{tp_clear} handler must be of the \ctype{inquiry} type, or
\NULL{} if the object is immutable.

\begin{ctypedesc}[inquiry]{int (*inquiry)(PyObject *self)}
  Drop references that may have created reference cycles.  Immutable
  objects do not have to define this method since they can never
  directly create reference cycles.  Note that the object must still
  be valid after calling this method (i.e., don't just call
  \cfunction{Py_DECREF()} on a reference).  The collector will call
  this method if it detects that this object is involved in a
  reference cycle.
\end{ctypedesc}


\subsection{Example Cycle Collector Support
            \label{example-cycle-support}}

This example shows only enough of the implementation of an extension
type to show how the garbage collector support needs to be added.  It
shows the definition of the object structure, the
\member{tp_traverse}, \member{tp_clear} and \member{tp_dealloc}
implementations, the type structure, and a constructor --- the module
initialization needed to export the constructor to Python is not shown
as there are no special considerations there for the collector.  To
make this interesting, assume that the module exposes ways for the
\member{container} field of the object to be modified.  Note that
since no checks are made on the type of the object used to initialize
\member{container}, we have to assume that it may be a container.

\begin{verbatim}
#include "Python.h"

typedef struct {
    PyObject_HEAD
    PyObject *container;
} MyObject;

static int
my_traverse(MyObject *self, visitproc visit, void *arg)
{
    if (self->container != NULL)
        return visit(self->container, arg);
    else
        return 0;
}

static int
my_clear(MyObject *self)
{
    Py_XDECREF(self->container);
    self->container = NULL;

    return 0;
}

static void
my_dealloc(MyObject *self)
{
    PyObject_GC_Fini((PyObject *) self);
    Py_XDECREF(self->container);
    PyObject_Del(self);
}
\end{verbatim}

\begin{verbatim}
statichere PyTypeObject
MyObject_Type = {
    PyObject_HEAD_INIT(NULL)
    0,
    "MyObject",
    sizeof(MyObject) + PyGC_HEAD_SIZE,
    0,
    (destructor)my_dealloc,     /* tp_dealloc */
    0,                          /* tp_print */
    0,                          /* tp_getattr */
    0,                          /* tp_setattr */
    0,                          /* tp_compare */
    0,                          /* tp_repr */
    0,                          /* tp_as_number */
    0,                          /* tp_as_sequence */
    0,                          /* tp_as_mapping */
    0,                          /* tp_hash */
    0,                          /* tp_call */
    0,                          /* tp_str */
    0,                          /* tp_getattro */
    0,                          /* tp_setattro */
    0,                          /* tp_as_buffer */
    Py_TPFLAGS_DEFAULT | Py_TPFLAGS_GC,
    0,                          /* tp_doc */
    (traverseproc)my_traverse,  /* tp_traverse */
    (inquiry)my_clear,          /* tp_clear */
    0,                          /* tp_richcompare */
    0,                          /* tp_weaklistoffset */
};

/* This constructor should be made accessible from Python. */
static PyObject *
new_object(PyObject *unused, PyObject *args)
{
    PyObject *container = NULL;
    MyObject *result = NULL;

    if (PyArg_ParseTuple(args, "|O:new_object", &container)) {
        result = PyObject_New(MyObject, &MyObject_Type);
        if (result != NULL) {
            result->container = container;
            PyObject_GC_Init();
        }
    }
    return (PyObject *) result;
}
\end{verbatim}


% \chapter{Debugging \label{debugging}}
%
% XXX Explain Py_DEBUG, Py_TRACE_REFS, Py_REF_DEBUG.


\appendix
\chapter{Reporting Bugs}
\label{reporting-bugs}

Python is a mature programming language which has established a
reputation for stability.  In order to maintain this reputation, the
developers would like to know of any deficiencies you find in Python
or its documentation.

All bug reports should be submitted via the Python Bug Tracker on
SourceForge (\url{http://sourceforge.net/bugs/?group_id=5470}).  The
bug tracker offers a Web form which allows pertinent information to be
entered and submitted to the developers.

Before submitting a report, please log into SourceForge if you are a
member; this will make it possible for the developers to contact you
for additional information if needed.  If you are not a SourceForge
member but would not mind the developers contacting you, you may
include your email address in your bug description.  In this case,
please realize that the information is publically available and cannot
be protected.

The first step in filing a report is to determine whether the problem
has already been reported.  The advantage in doing so, aside from
saving the developers time, is that you learn what has been done to
fix it; it may be that the problem has already been fixed for the next
release, or additional information is needed (in which case you are
welcome to provide it if you can!).  To do this, search the bug
database using the search box near the bottom of the page.

If the problem you're reporting is not already in the bug tracker, go
back to the Python Bug Tracker
(\url{http://sourceforge.net/bugs/?group_id=5470}).  Select the
``Submit a Bug'' link at the top of the page to open the bug reporting
form.

The submission form has a number of fields.  The only fields that are
required are the ``Summary'' and ``Details'' fields.  For the summary,
enter a \emph{very} short description of the problem; less than ten
words is good.  In the Details field, describe the problem in detail,
including what you expected to happen and what did happen.  Be sure to
include the version of Python you used, whether any extension modules
were involved, and what hardware and software platform you were using
(including version information as appropriate).

The only other field that you may want to set is the ``Category''
field, which allows you to place the bug report into a broad category
(such as ``Documentation'' or ``Library'').

Each bug report will be assigned to a developer who will determine
what needs to be done to correct the problem.  If you have a
SourceForge account and logged in to report the problem, you will
receive an update each time action is taken on the bug.


\begin{seealso}
  \seetitle[http://www-mice.cs.ucl.ac.uk/multimedia/software/documentation/ReportingBugs.html]{How
        to Report Bugs Effectively}{Article which goes into some
        detail about how to create a useful bug report.  This
        describes what kind of information is useful and why it is
        useful.}

  \seetitle[http://www.mozilla.org/quality/bug-writing-guidelines.html]{Bug
        Writing Guidelines}{Information about writing a good bug
        report.  Some of this is specific to the Mozilla project, but
        describes general good practices.}
\end{seealso}


\documentclass{manual}

\title{Python/C API Reference Manual}

\author{Guido van Rossum\\
	Fred L. Drake, Jr., editor}
\authoraddress{
	\strong{Python Software Foundation}\\
	Email: \email{docs@python.org}
}

\date{20 June, 2006}			% XXX update before final release!
\input{patchlevel}		% include Python version information


\makeindex			% tell \index to actually write the .idx file


\begin{document}

\maketitle

\ifhtml
\chapter*{Front Matter\label{front}}
\fi

\leftline{Copyright \copyright{} 2000, BeOpen.com.}
\leftline{Copyright \copyright{} 1995-2000, Corporation for National Research Initiatives.}
\leftline{Copyright \copyright{} 1990-1995, Stichting Mathematisch Centrum.}
\leftline{All rights reserved.}

Redistribution and use in source and binary forms, with or without
modification, are permitted provided that the following conditions are
met:

\begin{itemize}
\item
Redistributions of source code must retain the above copyright
notice, this list of conditions and the following disclaimer.

\item
Redistributions in binary form must reproduce the above copyright
notice, this list of conditions and the following disclaimer in the
documentation and/or other materials provided with the distribution.

\item
Neither names of the copyright holders nor the names of their
contributors may be used to endorse or promote products derived from
this software without specific prior written permission.
\end{itemize}

THIS SOFTWARE IS PROVIDED BY THE COPYRIGHT HOLDERS AND CONTRIBUTORS
``AS IS'' AND ANY EXPRESS OR IMPLIED WARRANTIES, INCLUDING, BUT NOT
LIMITED TO, THE IMPLIED WARRANTIES OF MERCHANTABILITY AND FITNESS FOR
A PARTICULAR PURPOSE ARE DISCLAIMED.  IN NO EVENT SHALL THE COPYRIGHT
HOLDERS OR CONTRIBUTORS BE LIABLE FOR ANY DIRECT, INDIRECT,
INCIDENTAL, SPECIAL, EXEMPLARY, OR CONSEQUENTIAL DAMAGES (INCLUDING,
BUT NOT LIMITED TO, PROCUREMENT OF SUBSTITUTE GOODS OR SERVICES; LOSS
OF USE, DATA, OR PROFITS; OR BUSINESS INTERRUPTION) HOWEVER CAUSED AND
ON ANY THEORY OF LIABILITY, WHETHER IN CONTRACT, STRICT LIABILITY, OR
TORT (INCLUDING NEGLIGENCE OR OTHERWISE) ARISING IN ANY WAY OUT OF THE
USE OF THIS SOFTWARE, EVEN IF ADVISED OF THE POSSIBILITY OF SUCH
DAMAGE.


\begin{abstract}

\noindent
This manual documents the API used by C and \Cpp{} programmers who
want to write extension modules or embed Python.  It is a companion to
\citetitle[../ext/ext.html]{Extending and Embedding the Python
Interpreter}, which describes the general principles of extension
writing but does not document the API functions in detail.

\strong{Warning:} The current version of this document is incomplete.
I hope that it is nevertheless useful.  I will continue to work on it,
and release new versions from time to time, independent from Python
source code releases.

\end{abstract}

\tableofcontents

% XXX Consider moving all this back to ext.tex and giving api.tex
% XXX a *really* short intro only.

\chapter{Introduction \label{intro}}

The Application Programmer's Interface to Python gives C and
\Cpp{} programmers access to the Python interpreter at a variety of
levels.  The API is equally usable from \Cpp{}, but for brevity it is
generally referred to as the Python/C API.  There are two
fundamentally different reasons for using the Python/C API.  The first
reason is to write \emph{extension modules} for specific purposes;
these are C modules that extend the Python interpreter.  This is
probably the most common use.  The second reason is to use Python as a
component in a larger application; this technique is generally
referred to as \dfn{embedding} Python in an application.

Writing an extension module is a relatively well-understood process, 
where a ``cookbook'' approach works well.  There are several tools 
that automate the process to some extent.  While people have embedded 
Python in other applications since its early existence, the process of 
embedding Python is less straightforward than writing an extension.  

Many API functions are useful independent of whether you're embedding 
or extending Python; moreover, most applications that embed Python 
will need to provide a custom extension as well, so it's probably a 
good idea to become familiar with writing an extension before 
attempting to embed Python in a real application.


\section{Include Files \label{includes}}

All function, type and macro definitions needed to use the Python/C
API are included in your code by the following line:

\begin{verbatim}
#include "Python.h"
\end{verbatim}

This implies inclusion of the following standard headers:
\code{<stdio.h>}, \code{<string.h>}, \code{<errno.h>},
\code{<limits.h>}, and \code{<stdlib.h>} (if available).

All user visible names defined by Python.h (except those defined by
the included standard headers) have one of the prefixes \samp{Py} or
\samp{_Py}.  Names beginning with \samp{_Py} are for internal use by
the Python implementation and should not be used by extension writers.
Structure member names do not have a reserved prefix.

\strong{Important:} user code should never define names that begin
with \samp{Py} or \samp{_Py}.  This confuses the reader, and
jeopardizes the portability of the user code to future Python
versions, which may define additional names beginning with one of
these prefixes.

The header files are typically installed with Python.  On \UNIX, these 
are located in the directories
\file{\envvar{prefix}/include/python\var{version}/} and
\file{\envvar{exec_prefix}/include/python\var{version}/}, where
\envvar{prefix} and \envvar{exec_prefix} are defined by the
corresponding parameters to Python's \program{configure} script and
\var{version} is \code{sys.version[:3]}.  On Windows, the headers are
installed in \file{\envvar{prefix}/include}, where \envvar{prefix} is
the installation directory specified to the installer.

To include the headers, place both directories (if different) on your
compiler's search path for includes.  Do \emph{not} place the parent
directories on the search path and then use
\samp{\#include <python\shortversion/Python.h>}; this will break on
multi-platform builds since the platform independent headers under
\envvar{prefix} include the platform specific headers from
\envvar{exec_prefix}.


\section{Objects, Types and Reference Counts \label{objects}}

Most Python/C API functions have one or more arguments as well as a
return value of type \ctype{PyObject*}.  This type is a pointer
to an opaque data type representing an arbitrary Python
object.  Since all Python object types are treated the same way by the
Python language in most situations (e.g., assignments, scope rules,
and argument passing), it is only fitting that they should be
represented by a single C type.  Almost all Python objects live on the
heap: you never declare an automatic or static variable of type
\ctype{PyObject}, only pointer variables of type \ctype{PyObject*} can 
be declared.  The sole exception are the type objects\obindex{type};
since these must never be deallocated, they are typically static
\ctype{PyTypeObject} objects.

All Python objects (even Python integers) have a \dfn{type} and a
\dfn{reference count}.  An object's type determines what kind of object 
it is (e.g., an integer, a list, or a user-defined function; there are 
many more as explained in the \citetitle[../ref/ref.html]{Python
Reference Manual}).  For each of the well-known types there is a macro
to check whether an object is of that type; for instance,
\samp{PyList_Check(\var{a})} is true if (and only if) the object
pointed to by \var{a} is a Python list.


\subsection{Reference Counts \label{refcounts}}

The reference count is important because today's computers have a 
finite (and often severely limited) memory size; it counts how many 
different places there are that have a reference to an object.  Such a 
place could be another object, or a global (or static) C variable, or 
a local variable in some C function.  When an object's reference count 
becomes zero, the object is deallocated.  If it contains references to 
other objects, their reference count is decremented.  Those other 
objects may be deallocated in turn, if this decrement makes their 
reference count become zero, and so on.  (There's an obvious problem 
with objects that reference each other here; for now, the solution is 
``don't do that.'')

Reference counts are always manipulated explicitly.  The normal way is 
to use the macro \cfunction{Py_INCREF()}\ttindex{Py_INCREF()} to
increment an object's reference count by one, and
\cfunction{Py_DECREF()}\ttindex{Py_DECREF()} to decrement it by  
one.  The \cfunction{Py_DECREF()} macro is considerably more complex
than the incref one, since it must check whether the reference count
becomes zero and then cause the object's deallocator to be called.
The deallocator is a function pointer contained in the object's type
structure.  The type-specific deallocator takes care of decrementing
the reference counts for other objects contained in the object if this
is a compound object type, such as a list, as well as performing any
additional finalization that's needed.  There's no chance that the
reference count can overflow; at least as many bits are used to hold
the reference count as there are distinct memory locations in virtual
memory (assuming \code{sizeof(long) >= sizeof(char*)}).  Thus, the
reference count increment is a simple operation.

It is not necessary to increment an object's reference count for every 
local variable that contains a pointer to an object.  In theory, the 
object's reference count goes up by one when the variable is made to 
point to it and it goes down by one when the variable goes out of 
scope.  However, these two cancel each other out, so at the end the 
reference count hasn't changed.  The only real reason to use the 
reference count is to prevent the object from being deallocated as 
long as our variable is pointing to it.  If we know that there is at 
least one other reference to the object that lives at least as long as 
our variable, there is no need to increment the reference count 
temporarily.  An important situation where this arises is in objects 
that are passed as arguments to C functions in an extension module 
that are called from Python; the call mechanism guarantees to hold a 
reference to every argument for the duration of the call.

However, a common pitfall is to extract an object from a list and
hold on to it for a while without incrementing its reference count.
Some other operation might conceivably remove the object from the
list, decrementing its reference count and possible deallocating it.
The real danger is that innocent-looking operations may invoke
arbitrary Python code which could do this; there is a code path which
allows control to flow back to the user from a \cfunction{Py_DECREF()},
so almost any operation is potentially dangerous.

A safe approach is to always use the generic operations (functions 
whose name begins with \samp{PyObject_}, \samp{PyNumber_},
\samp{PySequence_} or \samp{PyMapping_}).  These operations always
increment the reference count of the object they return.  This leaves
the caller with the responsibility to call
\cfunction{Py_DECREF()} when they are done with the result; this soon
becomes second nature.


\subsubsection{Reference Count Details \label{refcountDetails}}

The reference count behavior of functions in the Python/C API is best 
explained in terms of \emph{ownership of references}.  Note that we 
talk of owning references, never of owning objects; objects are always 
shared!  When a function owns a reference, it has to dispose of it 
properly --- either by passing ownership on (usually to its caller) or 
by calling \cfunction{Py_DECREF()} or \cfunction{Py_XDECREF()}.  When
a function passes ownership of a reference on to its caller, the
caller is said to receive a \emph{new} reference.  When no ownership
is transferred, the caller is said to \emph{borrow} the reference.
Nothing needs to be done for a borrowed reference.

Conversely, when a calling function passes it a reference to an 
object, there are two possibilities: the function \emph{steals} a 
reference to the object, or it does not.  Few functions steal 
references; the two notable exceptions are
\cfunction{PyList_SetItem()}\ttindex{PyList_SetItem()} and
\cfunction{PyTuple_SetItem()}\ttindex{PyTuple_SetItem()}, which 
steal a reference to the item (but not to the tuple or list into which
the item is put!).  These functions were designed to steal a reference
because of a common idiom for populating a tuple or list with newly
created objects; for example, the code to create the tuple \code{(1,
2, "three")} could look like this (forgetting about error handling for
the moment; a better way to code this is shown below):

\begin{verbatim}
PyObject *t;

t = PyTuple_New(3);
PyTuple_SetItem(t, 0, PyInt_FromLong(1L));
PyTuple_SetItem(t, 1, PyInt_FromLong(2L));
PyTuple_SetItem(t, 2, PyString_FromString("three"));
\end{verbatim}

Incidentally, \cfunction{PyTuple_SetItem()} is the \emph{only} way to
set tuple items; \cfunction{PySequence_SetItem()} and
\cfunction{PyObject_SetItem()} refuse to do this since tuples are an
immutable data type.  You should only use
\cfunction{PyTuple_SetItem()} for tuples that you are creating
yourself.

Equivalent code for populating a list can be written using 
\cfunction{PyList_New()} and \cfunction{PyList_SetItem()}.  Such code
can also use \cfunction{PySequence_SetItem()}; this illustrates the
difference between the two (the extra \cfunction{Py_DECREF()} calls):

\begin{verbatim}
PyObject *l, *x;

l = PyList_New(3);
x = PyInt_FromLong(1L);
PySequence_SetItem(l, 0, x); Py_DECREF(x);
x = PyInt_FromLong(2L);
PySequence_SetItem(l, 1, x); Py_DECREF(x);
x = PyString_FromString("three");
PySequence_SetItem(l, 2, x); Py_DECREF(x);
\end{verbatim}

You might find it strange that the ``recommended'' approach takes more
code.  However, in practice, you will rarely use these ways of
creating and populating a tuple or list.  There's a generic function,
\cfunction{Py_BuildValue()}, that can create most common objects from
C values, directed by a \dfn{format string}.  For example, the
above two blocks of code could be replaced by the following (which
also takes care of the error checking):

\begin{verbatim}
PyObject *t, *l;

t = Py_BuildValue("(iis)", 1, 2, "three");
l = Py_BuildValue("[iis]", 1, 2, "three");
\end{verbatim}

It is much more common to use \cfunction{PyObject_SetItem()} and
friends with items whose references you are only borrowing, like
arguments that were passed in to the function you are writing.  In
that case, their behaviour regarding reference counts is much saner,
since you don't have to increment a reference count so you can give a
reference away (``have it be stolen'').  For example, this function
sets all items of a list (actually, any mutable sequence) to a given
item:

\begin{verbatim}
int set_all(PyObject *target, PyObject *item)
{
    int i, n;

    n = PyObject_Length(target);
    if (n < 0)
        return -1;
    for (i = 0; i < n; i++) {
        if (PyObject_SetItem(target, i, item) < 0)
            return -1;
    }
    return 0;
}
\end{verbatim}
\ttindex{set_all()}

The situation is slightly different for function return values.  
While passing a reference to most functions does not change your 
ownership responsibilities for that reference, many functions that 
return a referece to an object give you ownership of the reference.
The reason is simple: in many cases, the returned object is created 
on the fly, and the reference you get is the only reference to the 
object.  Therefore, the generic functions that return object 
references, like \cfunction{PyObject_GetItem()} and 
\cfunction{PySequence_GetItem()}, always return a new reference (i.e.,
the  caller becomes the owner of the reference).

It is important to realize that whether you own a reference returned 
by a function depends on which function you call only --- \emph{the
plumage} (i.e., the type of the type of the object passed as an
argument to the function) \emph{doesn't enter into it!}  Thus, if you 
extract an item from a list using \cfunction{PyList_GetItem()}, you
don't own the reference --- but if you obtain the same item from the
same list using \cfunction{PySequence_GetItem()} (which happens to
take exactly the same arguments), you do own a reference to the
returned object.

Here is an example of how you could write a function that computes the
sum of the items in a list of integers; once using 
\cfunction{PyList_GetItem()}\ttindex{PyList_GetItem()}, and once using
\cfunction{PySequence_GetItem()}\ttindex{PySequence_GetItem()}.

\begin{verbatim}
long sum_list(PyObject *list)
{
    int i, n;
    long total = 0;
    PyObject *item;

    n = PyList_Size(list);
    if (n < 0)
        return -1; /* Not a list */
    for (i = 0; i < n; i++) {
        item = PyList_GetItem(list, i); /* Can't fail */
        if (!PyInt_Check(item)) continue; /* Skip non-integers */
        total += PyInt_AsLong(item);
    }
    return total;
}
\end{verbatim}
\ttindex{sum_list()}

\begin{verbatim}
long sum_sequence(PyObject *sequence)
{
    int i, n;
    long total = 0;
    PyObject *item;
    n = PySequence_Length(sequence);
    if (n < 0)
        return -1; /* Has no length */
    for (i = 0; i < n; i++) {
        item = PySequence_GetItem(sequence, i);
        if (item == NULL)
            return -1; /* Not a sequence, or other failure */
        if (PyInt_Check(item))
            total += PyInt_AsLong(item);
        Py_DECREF(item); /* Discard reference ownership */
    }
    return total;
}
\end{verbatim}
\ttindex{sum_sequence()}


\subsection{Types \label{types}}

There are few other data types that play a significant role in 
the Python/C API; most are simple C types such as \ctype{int}, 
\ctype{long}, \ctype{double} and \ctype{char*}.  A few structure types 
are used to describe static tables used to list the functions exported 
by a module or the data attributes of a new object type, and another
is used to describe the value of a complex number.  These will 
be discussed together with the functions that use them.


\section{Exceptions \label{exceptions}}

The Python programmer only needs to deal with exceptions if specific 
error handling is required; unhandled exceptions are automatically 
propagated to the caller, then to the caller's caller, and so on, until
they reach the top-level interpreter, where they are reported to the 
user accompanied by a stack traceback.

For C programmers, however, error checking always has to be explicit.  
All functions in the Python/C API can raise exceptions, unless an 
explicit claim is made otherwise in a function's documentation.  In 
general, when a function encounters an error, it sets an exception, 
discards any object references that it owns, and returns an 
error indicator --- usually \NULL{} or \code{-1}.  A few functions 
return a Boolean true/false result, with false indicating an error.
Very few functions return no explicit error indicator or have an 
ambiguous return value, and require explicit testing for errors with 
\cfunction{PyErr_Occurred()}\ttindex{PyErr_Occurred()}.

Exception state is maintained in per-thread storage (this is 
equivalent to using global storage in an unthreaded application).  A 
thread can be in one of two states: an exception has occurred, or not.
The function \cfunction{PyErr_Occurred()} can be used to check for
this: it returns a borrowed reference to the exception type object
when an exception has occurred, and \NULL{} otherwise.  There are a
number of functions to set the exception state:
\cfunction{PyErr_SetString()}\ttindex{PyErr_SetString()} is the most
common (though not the most general) function to set the exception
state, and \cfunction{PyErr_Clear()}\ttindex{PyErr_Clear()} clears the
exception state.

The full exception state consists of three objects (all of which can 
be \NULL{}): the exception type, the corresponding exception 
value, and the traceback.  These have the same meanings as the Python
\withsubitem{(in module sys)}{
  \ttindex{exc_type}\ttindex{exc_value}\ttindex{exc_traceback}}
objects \code{sys.exc_type}, \code{sys.exc_value}, and
\code{sys.exc_traceback}; however, they are not the same: the Python
objects represent the last exception being handled by a Python 
\keyword{try} \ldots\ \keyword{except} statement, while the C level
exception state only exists while an exception is being passed on
between C functions until it reaches the Python bytecode interpreter's 
main loop, which takes care of transferring it to \code{sys.exc_type}
and friends.

Note that starting with Python 1.5, the preferred, thread-safe way to 
access the exception state from Python code is to call the function
\withsubitem{(in module sys)}{\ttindex{exc_info()}}
\function{sys.exc_info()}, which returns the per-thread exception state 
for Python code.  Also, the semantics of both ways to access the 
exception state have changed so that a function which catches an 
exception will save and restore its thread's exception state so as to 
preserve the exception state of its caller.  This prevents common bugs 
in exception handling code caused by an innocent-looking function 
overwriting the exception being handled; it also reduces the often 
unwanted lifetime extension for objects that are referenced by the 
stack frames in the traceback.

As a general principle, a function that calls another function to 
perform some task should check whether the called function raised an 
exception, and if so, pass the exception state on to its caller.  It 
should discard any object references that it owns, and return an 
error indicator, but it should \emph{not} set another exception ---
that would overwrite the exception that was just raised, and lose
important information about the exact cause of the error.

A simple example of detecting exceptions and passing them on is shown
in the \cfunction{sum_sequence()}\ttindex{sum_sequence()} example
above.  It so happens that that example doesn't need to clean up any
owned references when it detects an error.  The following example
function shows some error cleanup.  First, to remind you why you like
Python, we show the equivalent Python code:

\begin{verbatim}
def incr_item(dict, key):
    try:
        item = dict[key]
    except KeyError:
        item = 0
    dict[key] = item + 1
\end{verbatim}
\ttindex{incr_item()}

Here is the corresponding C code, in all its glory:

\begin{verbatim}
int incr_item(PyObject *dict, PyObject *key)
{
    /* Objects all initialized to NULL for Py_XDECREF */
    PyObject *item = NULL, *const_one = NULL, *incremented_item = NULL;
    int rv = -1; /* Return value initialized to -1 (failure) */

    item = PyObject_GetItem(dict, key);
    if (item == NULL) {
        /* Handle KeyError only: */
        if (!PyErr_ExceptionMatches(PyExc_KeyError))
            goto error;

        /* Clear the error and use zero: */
        PyErr_Clear();
        item = PyInt_FromLong(0L);
        if (item == NULL)
            goto error;
    }
    const_one = PyInt_FromLong(1L);
    if (const_one == NULL)
        goto error;

    incremented_item = PyNumber_Add(item, const_one);
    if (incremented_item == NULL)
        goto error;

    if (PyObject_SetItem(dict, key, incremented_item) < 0)
        goto error;
    rv = 0; /* Success */
    /* Continue with cleanup code */

 error:
    /* Cleanup code, shared by success and failure path */

    /* Use Py_XDECREF() to ignore NULL references */
    Py_XDECREF(item);
    Py_XDECREF(const_one);
    Py_XDECREF(incremented_item);

    return rv; /* -1 for error, 0 for success */
}
\end{verbatim}
\ttindex{incr_item()}

This example represents an endorsed use of the \keyword{goto} statement 
in C!  It illustrates the use of
\cfunction{PyErr_ExceptionMatches()}\ttindex{PyErr_ExceptionMatches()} and
\cfunction{PyErr_Clear()}\ttindex{PyErr_Clear()} to
handle specific exceptions, and the use of
\cfunction{Py_XDECREF()}\ttindex{Py_XDECREF()} to
dispose of owned references that may be \NULL{} (note the
\character{X} in the name; \cfunction{Py_DECREF()} would crash when
confronted with a \NULL{} reference).  It is important that the
variables used to hold owned references are initialized to \NULL{} for
this to work; likewise, the proposed return value is initialized to
\code{-1} (failure) and only set to success after the final call made
is successful.


\section{Embedding Python \label{embedding}}

The one important task that only embedders (as opposed to extension
writers) of the Python interpreter have to worry about is the
initialization, and possibly the finalization, of the Python
interpreter.  Most functionality of the interpreter can only be used
after the interpreter has been initialized.

The basic initialization function is
\cfunction{Py_Initialize()}\ttindex{Py_Initialize()}.
This initializes the table of loaded modules, and creates the
fundamental modules \module{__builtin__}\refbimodindex{__builtin__},
\module{__main__}\refbimodindex{__main__} and 
\module{sys}\refbimodindex{sys}.  It also initializes the module
search path (\code{sys.path}).%
\indexiii{module}{search}{path}
\withsubitem{(in module sys)}{\ttindex{path}}

\cfunction{Py_Initialize()} does not set the ``script argument list'' 
(\code{sys.argv}).  If this variable is needed by Python code that 
will be executed later, it must be set explicitly with a call to 
\code{PySys_SetArgv(\var{argc},
\var{argv})}\ttindex{PySys_SetArgv()} subsequent to the call to
\cfunction{Py_Initialize()}.

On most systems (in particular, on \UNIX{} and Windows, although the
details are slightly different),
\cfunction{Py_Initialize()} calculates the module search path based
upon its best guess for the location of the standard Python
interpreter executable, assuming that the Python library is found in a
fixed location relative to the Python interpreter executable.  In
particular, it looks for a directory named
\file{lib/python\shortversion} relative to the parent directory where
the executable named \file{python} is found on the shell command
search path (the environment variable \envvar{PATH}).

For instance, if the Python executable is found in
\file{/usr/local/bin/python}, it will assume that the libraries are in
\file{/usr/local/lib/python\shortversion}.  (In fact, this particular path
is also the ``fallback'' location, used when no executable file named
\file{python} is found along \envvar{PATH}.)  The user can override
this behavior by setting the environment variable \envvar{PYTHONHOME},
or insert additional directories in front of the standard path by
setting \envvar{PYTHONPATH}.

The embedding application can steer the search by calling 
\code{Py_SetProgramName(\var{file})}\ttindex{Py_SetProgramName()} \emph{before} calling 
\cfunction{Py_Initialize()}.  Note that \envvar{PYTHONHOME} still
overrides this and \envvar{PYTHONPATH} is still inserted in front of
the standard path.  An application that requires total control has to
provide its own implementation of
\cfunction{Py_GetPath()}\ttindex{Py_GetPath()},
\cfunction{Py_GetPrefix()}\ttindex{Py_GetPrefix()},
\cfunction{Py_GetExecPrefix()}\ttindex{Py_GetExecPrefix()}, and
\cfunction{Py_GetProgramFullPath()}\ttindex{Py_GetProgramFullPath()} (all
defined in \file{Modules/getpath.c}).

Sometimes, it is desirable to ``uninitialize'' Python.  For instance, 
the application may want to start over (make another call to 
\cfunction{Py_Initialize()}) or the application is simply done with its 
use of Python and wants to free all memory allocated by Python.  This
can be accomplished by calling \cfunction{Py_Finalize()}.  The function
\cfunction{Py_IsInitialized()}\ttindex{Py_IsInitialized()} returns
true if Python is currently in the initialized state.  More
information about these functions is given in a later chapter.


\chapter{The Very High Level Layer \label{veryhigh}}

The functions in this chapter will let you execute Python source code
given in a file or a buffer, but they will not let you interact in a
more detailed way with the interpreter.

Several of these functions accept a start symbol from the grammar as a 
parameter.  The available start symbols are \constant{Py_eval_input},
\constant{Py_file_input}, and \constant{Py_single_input}.  These are
described following the functions which accept them as parameters.

Note also that several of these functions take \ctype{FILE*}
parameters.  On particular issue which needs to be handled carefully
is that the \ctype{FILE} structure for different C libraries can be
different and incompatible.  Under Windows (at least), it is possible
for dynamically linked extensions to actually use different libraries,
so care should be taken that \ctype{FILE*} parameters are only passed
to these functions if it is certain that they were created by the same
library that the Python runtime is using.

\begin{cfuncdesc}{int}{PyRun_AnyFile}{FILE *fp, char *filename}
  If \var{fp} refers to a file associated with an interactive device
  (console or terminal input or \UNIX{} pseudo-terminal), return the
  value of \cfunction{PyRun_InteractiveLoop()}, otherwise return the
  result of \cfunction{PyRun_SimpleFile()}.  If \var{filename} is
  \NULL{}, this function uses \code{"???"} as the filename.
\end{cfuncdesc}

\begin{cfuncdesc}{int}{PyRun_SimpleString}{char *command}
  Executes the Python source code from \var{command} in the
  \module{__main__} module.  If \module{__main__} does not already
  exist, it is created.  Returns \code{0} on success or \code{-1} if
  an exception was raised.  If there was an error, there is no way to
  get the exception information.
\end{cfuncdesc}

\begin{cfuncdesc}{int}{PyRun_SimpleFile}{FILE *fp, char *filename}
  Similar to \cfunction{PyRun_SimpleString()}, but the Python source
  code is read from \var{fp} instead of an in-memory string.
  \var{filename} should be the name of the file.
\end{cfuncdesc}

\begin{cfuncdesc}{int}{PyRun_InteractiveOne}{FILE *fp, char *filename}
  Read and execute a single statement from a file associated with an
  interactive device.  If \var{filename} is \NULL, \code{"???"} is
  used instead.  The user will be prompted using \code{sys.ps1} and
  \code{sys.ps2}.  Returns \code{0} when the input was executed
  successfully, \code{-1} if there was an exception, or an error code
  from the \file{errcode.h} include file distributed as part of Python
  in case of a parse error.  (Note that \file{errcode.h} is not
  included by \file{Python.h}, so must be included specifically if
  needed.)
\end{cfuncdesc}

\begin{cfuncdesc}{int}{PyRun_InteractiveLoop}{FILE *fp, char *filename}
  Read and execute statements from a file associated with an
  interactive device until \EOF{} is reached.  If \var{filename} is
  \NULL, \code{"???"} is used instead.  The user will be prompted
  using \code{sys.ps1} and \code{sys.ps2}.  Returns \code{0} at \EOF.
\end{cfuncdesc}

\begin{cfuncdesc}{struct _node*}{PyParser_SimpleParseString}{char *str,
                                                             int start}
  Parse Python source code from \var{str} using the start token
  \var{start}.  The result can be used to create a code object which
  can be evaluated efficiently.  This is useful if a code fragment
  must be evaluated many times.
\end{cfuncdesc}

\begin{cfuncdesc}{struct _node*}{PyParser_SimpleParseFile}{FILE *fp,
                                 char *filename, int start}
  Similar to \cfunction{PyParser_SimpleParseString()}, but the Python
  source code is read from \var{fp} instead of an in-memory string.
  \var{filename} should be the name of the file.
\end{cfuncdesc}

\begin{cfuncdesc}{PyObject*}{PyRun_String}{char *str, int start,
                                           PyObject *globals,
                                           PyObject *locals}
  Execute Python source code from \var{str} in the context specified
  by the dictionaries \var{globals} and \var{locals}.  The parameter
  \var{start} specifies the start token that should be used to parse
  the source code.

  Returns the result of executing the code as a Python object, or
  \NULL{} if an exception was raised.
\end{cfuncdesc}

\begin{cfuncdesc}{PyObject*}{PyRun_File}{FILE *fp, char *filename,
                                         int start, PyObject *globals,
                                         PyObject *locals}
  Similar to \cfunction{PyRun_String()}, but the Python source code is 
  read from \var{fp} instead of an in-memory string.
  \var{filename} should be the name of the file.
\end{cfuncdesc}

\begin{cfuncdesc}{PyObject*}{Py_CompileString}{char *str, char *filename,
                                               int start}
  Parse and compile the Python source code in \var{str}, returning the 
  resulting code object.  The start token is given by \var{start};
  this can be used to constrain the code which can be compiled and should
  be \constant{Py_eval_input}, \constant{Py_file_input}, or
  \constant{Py_single_input}.  The filename specified by
  \var{filename} is used to construct the code object and may appear
  in tracebacks or \exception{SyntaxError} exception messages.  This
  returns \NULL{} if the code cannot be parsed or compiled.
\end{cfuncdesc}

\begin{cvardesc}{int}{Py_eval_input}
  The start symbol from the Python grammar for isolated expressions;
  for use with \cfunction{Py_CompileString()}\ttindex{Py_CompileString()}.
\end{cvardesc}

\begin{cvardesc}{int}{Py_file_input}
  The start symbol from the Python grammar for sequences of statements
  as read from a file or other source; for use with
  \cfunction{Py_CompileString()}\ttindex{Py_CompileString()}.  This is
  the symbol to use when compiling arbitrarily long Python source code.
\end{cvardesc}

\begin{cvardesc}{int}{Py_single_input}
  The start symbol from the Python grammar for a single statement; for 
  use with \cfunction{Py_CompileString()}\ttindex{Py_CompileString()}.
  This is the symbol used for the interactive interpreter loop.
\end{cvardesc}


\chapter{Reference Counting \label{countingRefs}}

The macros in this section are used for managing reference counts
of Python objects.

\begin{cfuncdesc}{void}{Py_INCREF}{PyObject *o}
Increment the reference count for object \var{o}.  The object must
not be \NULL{}; if you aren't sure that it isn't \NULL{}, use
\cfunction{Py_XINCREF()}.
\end{cfuncdesc}

\begin{cfuncdesc}{void}{Py_XINCREF}{PyObject *o}
Increment the reference count for object \var{o}.  The object may be
\NULL{}, in which case the macro has no effect.
\end{cfuncdesc}

\begin{cfuncdesc}{void}{Py_DECREF}{PyObject *o}
Decrement the reference count for object \var{o}.  The object must
not be \NULL{}; if you aren't sure that it isn't \NULL{}, use
\cfunction{Py_XDECREF()}.  If the reference count reaches zero, the
object's type's deallocation function (which must not be \NULL{}) is
invoked.

\strong{Warning:} The deallocation function can cause arbitrary Python
code to be invoked (e.g. when a class instance with a
\method{__del__()} method is deallocated).  While exceptions in such
code are not propagated, the executed code has free access to all
Python global variables.  This means that any object that is reachable
from a global variable should be in a consistent state before
\cfunction{Py_DECREF()} is invoked.  For example, code to delete an
object from a list should copy a reference to the deleted object in a
temporary variable, update the list data structure, and then call
\cfunction{Py_DECREF()} for the temporary variable.
\end{cfuncdesc}

\begin{cfuncdesc}{void}{Py_XDECREF}{PyObject *o}
Decrement the reference count for object \var{o}.  The object may be
\NULL{}, in which case the macro has no effect; otherwise the effect
is the same as for \cfunction{Py_DECREF()}, and the same warning
applies.
\end{cfuncdesc}

The following functions or macros are only for use within the
interpreter core: \cfunction{_Py_Dealloc()},
\cfunction{_Py_ForgetReference()}, \cfunction{_Py_NewReference()}, as
well as the global variable \cdata{_Py_RefTotal}.


\chapter{Exception Handling \label{exceptionHandling}}

The functions described in this chapter will let you handle and raise Python
exceptions.  It is important to understand some of the basics of
Python exception handling.  It works somewhat like the
\UNIX{} \cdata{errno} variable: there is a global indicator (per
thread) of the last error that occurred.  Most functions don't clear
this on success, but will set it to indicate the cause of the error on
failure.  Most functions also return an error indicator, usually
\NULL{} if they are supposed to return a pointer, or \code{-1} if they
return an integer (exception: the \cfunction{PyArg_Parse*()} functions
return \code{1} for success and \code{0} for failure).  When a
function must fail because some function it called failed, it
generally doesn't set the error indicator; the function it called
already set it.

The error indicator consists of three Python objects corresponding to
\withsubitem{(in module sys)}{
  \ttindex{exc_type}\ttindex{exc_value}\ttindex{exc_traceback}}
the Python variables \code{sys.exc_type}, \code{sys.exc_value} and
\code{sys.exc_traceback}.  API functions exist to interact with the
error indicator in various ways.  There is a separate error indicator
for each thread.

% XXX Order of these should be more thoughtful.
% Either alphabetical or some kind of structure.

\begin{cfuncdesc}{void}{PyErr_Print}{}
Print a standard traceback to \code{sys.stderr} and clear the error
indicator.  Call this function only when the error indicator is set.
(Otherwise it will cause a fatal error!)
\end{cfuncdesc}

\begin{cfuncdesc}{PyObject*}{PyErr_Occurred}{}
Test whether the error indicator is set.  If set, return the exception
\emph{type} (the first argument to the last call to one of the
\cfunction{PyErr_Set*()} functions or to \cfunction{PyErr_Restore()}).  If
not set, return \NULL{}.  You do not own a reference to the return
value, so you do not need to \cfunction{Py_DECREF()} it.
\strong{Note:}  Do not compare the return value to a specific
exception; use \cfunction{PyErr_ExceptionMatches()} instead, shown
below.  (The comparison could easily fail since the exception may be
an instance instead of a class, in the case of a class exception, or
it may the a subclass of the expected exception.)
\end{cfuncdesc}

\begin{cfuncdesc}{int}{PyErr_ExceptionMatches}{PyObject *exc}
Equivalent to
\samp{PyErr_GivenExceptionMatches(PyErr_Occurred(), \var{exc})}.
This should only be called when an exception is actually set; a memory 
access violation will occur if no exception has been raised.
\end{cfuncdesc}

\begin{cfuncdesc}{int}{PyErr_GivenExceptionMatches}{PyObject *given, PyObject *exc}
Return true if the \var{given} exception matches the exception in
\var{exc}.  If \var{exc} is a class object, this also returns true
when \var{given} is an instance of a subclass.  If \var{exc} is a tuple, all
exceptions in the tuple (and recursively in subtuples) are searched
for a match.  If \var{given} is \NULL, a memory access violation will
occur.
\end{cfuncdesc}

\begin{cfuncdesc}{void}{PyErr_NormalizeException}{PyObject**exc, PyObject**val, PyObject**tb}
Under certain circumstances, the values returned by
\cfunction{PyErr_Fetch()} below can be ``unnormalized'', meaning that
\code{*\var{exc}} is a class object but \code{*\var{val}} is not an
instance of the  same class.  This function can be used to instantiate
the class in that case.  If the values are already normalized, nothing
happens.  The delayed normalization is implemented to improve
performance.
\end{cfuncdesc}

\begin{cfuncdesc}{void}{PyErr_Clear}{}
Clear the error indicator.  If the error indicator is not set, there
is no effect.
\end{cfuncdesc}

\begin{cfuncdesc}{void}{PyErr_Fetch}{PyObject **ptype, PyObject **pvalue,
                                     PyObject **ptraceback}
Retrieve the error indicator into three variables whose addresses are
passed.  If the error indicator is not set, set all three variables to
\NULL{}.  If it is set, it will be cleared and you own a reference to
each object retrieved.  The value and traceback object may be
\NULL{} even when the type object is not.  \strong{Note:}  This
function is normally only used by code that needs to handle exceptions
or by code that needs to save and restore the error indicator
temporarily.
\end{cfuncdesc}

\begin{cfuncdesc}{void}{PyErr_Restore}{PyObject *type, PyObject *value,
                                       PyObject *traceback}
Set  the error indicator from the three objects.  If the error
indicator is already set, it is cleared first.  If the objects are
\NULL{}, the error indicator is cleared.  Do not pass a \NULL{} type
and non-\NULL{} value or traceback.  The exception type should be a
string or class; if it is a class, the value should be an instance of
that class.  Do not pass an invalid exception type or value.
(Violating these rules will cause subtle problems later.)  This call
takes away a reference to each object, i.e.\ you must own a reference
to each object before the call and after the call you no longer own
these references.  (If you don't understand this, don't use this
function.  I warned you.)  \strong{Note:}  This function is normally
only used by code that needs to save and restore the error indicator
temporarily.
\end{cfuncdesc}

\begin{cfuncdesc}{void}{PyErr_SetString}{PyObject *type, char *message}
This is the most common way to set the error indicator.  The first
argument specifies the exception type; it is normally one of the
standard exceptions, e.g. \cdata{PyExc_RuntimeError}.  You need not
increment its reference count.  The second argument is an error
message; it is converted to a string object.
\end{cfuncdesc}

\begin{cfuncdesc}{void}{PyErr_SetObject}{PyObject *type, PyObject *value}
This function is similar to \cfunction{PyErr_SetString()} but lets you
specify an arbitrary Python object for the ``value'' of the exception.
You need not increment its reference count.
\end{cfuncdesc}

\begin{cfuncdesc}{PyObject*}{PyErr_Format}{PyObject *exception,
                                           const char *format, \moreargs}
This function sets the error indicator.  \var{exception} should be a
Python exception (string or class, not an instance).
\var{format} should be a string, containing format codes, similar to 
\cfunction{printf}. The \code{width.precision} before a format code
is parsed, but the width part is ignored.

\begin{tableii}{c|l}{character}{Character}{Meaning}
  \lineii{c}{Character, as an \ctype{int} parameter}
  \lineii{d}{Number in decimal, as an \ctype{int} parameter}
  \lineii{x}{Number in hexadecimal, as an \ctype{int} parameter}
  \lineii{x}{A string, as a \ctype{char *} parameter}
\end{tableii}

An unrecognized format character causes all the rest of
the format string to be copied as-is to the result string,
and any extra arguments discarded.

A new reference is returned, which is owned by the caller.
\end{cfuncdesc}

\begin{cfuncdesc}{void}{PyErr_SetNone}{PyObject *type}
This is a shorthand for \samp{PyErr_SetObject(\var{type}, Py_None)}.
\end{cfuncdesc}

\begin{cfuncdesc}{int}{PyErr_BadArgument}{}
This is a shorthand for \samp{PyErr_SetString(PyExc_TypeError,
\var{message})}, where \var{message} indicates that a built-in operation
was invoked with an illegal argument.  It is mostly for internal use.
\end{cfuncdesc}

\begin{cfuncdesc}{PyObject*}{PyErr_NoMemory}{}
This is a shorthand for \samp{PyErr_SetNone(PyExc_MemoryError)}; it
returns \NULL{} so an object allocation function can write
\samp{return PyErr_NoMemory();} when it runs out of memory.
\end{cfuncdesc}

\begin{cfuncdesc}{PyObject*}{PyErr_SetFromErrno}{PyObject *type}
This is a convenience function to raise an exception when a C library
function has returned an error and set the C variable \cdata{errno}.
It constructs a tuple object whose first item is the integer
\cdata{errno} value and whose second item is the corresponding error
message (gotten from \cfunction{strerror()}\ttindex{strerror()}), and
then calls
\samp{PyErr_SetObject(\var{type}, \var{object})}.  On \UNIX{}, when
the \cdata{errno} value is \constant{EINTR}, indicating an interrupted
system call, this calls \cfunction{PyErr_CheckSignals()}, and if that set
the error indicator, leaves it set to that.  The function always
returns \NULL{}, so a wrapper function around a system call can write 
\samp{return PyErr_SetFromErrno();} when  the system call returns an
error.
\end{cfuncdesc}

\begin{cfuncdesc}{void}{PyErr_BadInternalCall}{}
This is a shorthand for \samp{PyErr_SetString(PyExc_TypeError,
\var{message})}, where \var{message} indicates that an internal
operation (e.g. a Python/C API function) was invoked with an illegal
argument.  It is mostly for internal use.
\end{cfuncdesc}

\begin{cfuncdesc}{int}{PyErr_Warn}{PyObject *category, char *message}
Issue a warning message.  The \var{category} argument is a warning
category (see below) or \NULL; the \var{message} argument is a message
string.

This function normally prints a warning message to \var{sys.stderr};
however, it is also possible that the user has specified that warnings
are to be turned into errors, and in that case this will raise an
exception.  It is also possible that the function raises an exception
because of a problem with the warning machinery (the implementation
imports the \module{warnings} module to do the heavy lifting).  The
return value is \code{0} if no exception is raised, or \code{-1} if
an exception is raised.  (It is not possible to determine whether a
warning message is actually printed, nor what the reason is for the
exception; this is intentional.)  If an exception is raised, the
caller should do its normal exception handling
(e.g. \cfunction{Py_DECREF()} owned references and return an error
value).

Warning categories must be subclasses of \cdata{Warning}; the default
warning category is \cdata{RuntimeWarning}.  The standard Python
warning categories are available as global variables whose names are
\samp{PyExc_} followed by the Python exception name.  These have the
type \ctype{PyObject*}; they are all class objects.  Their names are
\cdata{PyExc_Warning}, \cdata{PyExc_UserWarning},
\cdata{PyExc_DeprecationWarning}, \cdata{PyExc_SyntaxWarning}, and
\cdata{PyExc_RuntimeWarning}.  \cdata{PyExc_Warning} is a subclass of
\cdata{PyExc_Exception}; the other warning categories are subclasses
of \cdata{PyExc_Warning}.

For information about warning control, see the documentation for the
\module{warnings} module and the \programopt{-W} option in the command
line documentation.  There is no C API for warning control.
\end{cfuncdesc}

\begin{cfuncdesc}{int}{PyErr_WarnExplicit}{PyObject *category, char *message,
char *filename, int lineno, char *module, PyObject *registry}
Issue a warning message with explicit control over all warning
attributes.  This is a straightforward wrapper around the Python
function \function{warnings.warn_explicit()}, see there for more
information.  The \var{module} and \var{registry} arguments may be
set to \code{NULL} to get the default effect described there.
\end{cfuncdesc}

\begin{cfuncdesc}{int}{PyErr_CheckSignals}{}
This function interacts with Python's signal handling.  It checks
whether a signal has been sent to the processes and if so, invokes the
corresponding signal handler.  If the
\module{signal}\refbimodindex{signal} module is supported, this can
invoke a signal handler written in Python.  In all cases, the default
effect for \constant{SIGINT}\ttindex{SIGINT} is to raise the
\withsubitem{(built-in exception)}{\ttindex{KeyboardInterrupt}}
\exception{KeyboardInterrupt} exception.  If an exception is raised the 
error indicator is set and the function returns \code{1}; otherwise
the function returns \code{0}.  The error indicator may or may not be
cleared if it was previously set.
\end{cfuncdesc}

\begin{cfuncdesc}{void}{PyErr_SetInterrupt}{}
This function is obsolete.  It simulates the effect of a
\constant{SIGINT}\ttindex{SIGINT} signal arriving --- the next time
\cfunction{PyErr_CheckSignals()} is called,
\withsubitem{(built-in exception)}{\ttindex{KeyboardInterrupt}}
\exception{KeyboardInterrupt} will be raised.
It may be called without holding the interpreter lock.
\end{cfuncdesc}

\begin{cfuncdesc}{PyObject*}{PyErr_NewException}{char *name,
                                                 PyObject *base,
                                                 PyObject *dict}
This utility function creates and returns a new exception object.  The
\var{name} argument must be the name of the new exception, a C string
of the form \code{module.class}.  The \var{base} and
\var{dict} arguments are normally \NULL{}.  This creates a
class object derived from the root for all exceptions, the built-in
name \exception{Exception} (accessible in C as
\cdata{PyExc_Exception}).  The \member{__module__} attribute of the
new class is set to the first part (up to the last dot) of the
\var{name} argument, and the class name is set to the last part (after
the last dot).  The \var{base} argument can be used to specify an
alternate base class.  The \var{dict} argument can be used to specify
a dictionary of class variables and methods.
\end{cfuncdesc}

\begin{cfuncdesc}{void}{PyErr_WriteUnraisable}{PyObject *obj}
This utility function prints a warning message to \var{sys.stderr}
when an exception has been set but it is impossible for the
interpreter to actually raise the exception.  It is used, for example,
when an exception occurs in an \member{__del__} method.

The function is called with a single argument \var{obj} that
identifies where the context in which the unraisable exception
occurred.  The repr of \var{obj} will be printed in the warning
message.
\end{cfuncdesc}

\section{Standard Exceptions \label{standardExceptions}}

All standard Python exceptions are available as global variables whose
names are \samp{PyExc_} followed by the Python exception name.  These
have the type \ctype{PyObject*}; they are all class objects.  For
completeness, here are all the variables:

\begin{tableiii}{l|l|c}{cdata}{C Name}{Python Name}{Notes}
  \lineiii{PyExc_Exception}{\exception{Exception}}{(1)}
  \lineiii{PyExc_StandardError}{\exception{StandardError}}{(1)}
  \lineiii{PyExc_ArithmeticError}{\exception{ArithmeticError}}{(1)}
  \lineiii{PyExc_LookupError}{\exception{LookupError}}{(1)}
  \lineiii{PyExc_AssertionError}{\exception{AssertionError}}{}
  \lineiii{PyExc_AttributeError}{\exception{AttributeError}}{}
  \lineiii{PyExc_EOFError}{\exception{EOFError}}{}
  \lineiii{PyExc_EnvironmentError}{\exception{EnvironmentError}}{(1)}
  \lineiii{PyExc_FloatingPointError}{\exception{FloatingPointError}}{}
  \lineiii{PyExc_IOError}{\exception{IOError}}{}
  \lineiii{PyExc_ImportError}{\exception{ImportError}}{}
  \lineiii{PyExc_IndexError}{\exception{IndexError}}{}
  \lineiii{PyExc_KeyError}{\exception{KeyError}}{}
  \lineiii{PyExc_KeyboardInterrupt}{\exception{KeyboardInterrupt}}{}
  \lineiii{PyExc_MemoryError}{\exception{MemoryError}}{}
  \lineiii{PyExc_NameError}{\exception{NameError}}{}
  \lineiii{PyExc_NotImplementedError}{\exception{NotImplementedError}}{}
  \lineiii{PyExc_OSError}{\exception{OSError}}{}
  \lineiii{PyExc_OverflowError}{\exception{OverflowError}}{}
  \lineiii{PyExc_RuntimeError}{\exception{RuntimeError}}{}
  \lineiii{PyExc_SyntaxError}{\exception{SyntaxError}}{}
  \lineiii{PyExc_SystemError}{\exception{SystemError}}{}
  \lineiii{PyExc_SystemExit}{\exception{SystemExit}}{}
  \lineiii{PyExc_TypeError}{\exception{TypeError}}{}
  \lineiii{PyExc_ValueError}{\exception{ValueError}}{}
  \lineiii{PyExc_WindowsError}{\exception{WindowsError}}{(2)}
  \lineiii{PyExc_ZeroDivisionError}{\exception{ZeroDivisionError}}{}
\end{tableiii}

\noindent
Notes:
\begin{description}
\item[(1)]
  This is a base class for other standard exceptions.

\item[(2)]
  Only defined on Windows; protect code that uses this by testing that
  the preprocessor macro \code{MS_WINDOWS} is defined.
\end{description}


\section{Deprecation of String Exceptions}

All exceptions built into Python or provided in the standard library
are derived from \exception{Exception}.
\withsubitem{(built-in exception)}{\ttindex{Exception}}

String exceptions are still supported in the interpreter to allow
existing code to run unmodified, but this will also change in a future 
release.


\chapter{Utilities \label{utilities}}

The functions in this chapter perform various utility tasks, such as
parsing function arguments and constructing Python values from C
values.

\section{OS Utilities \label{os}}

\begin{cfuncdesc}{int}{Py_FdIsInteractive}{FILE *fp, char *filename}
Return true (nonzero) if the standard I/O file \var{fp} with name
\var{filename} is deemed interactive.  This is the case for files for
which \samp{isatty(fileno(\var{fp}))} is true.  If the global flag
\cdata{Py_InteractiveFlag} is true, this function also returns true if
the \var{filename} pointer is \NULL{} or if the name is equal to one of
the strings \code{'<stdin>'} or \code{'???'}.
\end{cfuncdesc}

\begin{cfuncdesc}{long}{PyOS_GetLastModificationTime}{char *filename}
Return the time of last modification of the file \var{filename}.
The result is encoded in the same way as the timestamp returned by
the standard C library function \cfunction{time()}.
\end{cfuncdesc}

\begin{cfuncdesc}{void}{PyOS_AfterFork}{}
Function to update some internal state after a process fork; this
should be called in the new process if the Python interpreter will
continue to be used.  If a new executable is loaded into the new
process, this function does not need to be called.
\end{cfuncdesc}

\begin{cfuncdesc}{int}{PyOS_CheckStack}{}
Return true when the interpreter runs out of stack space.  This is a
reliable check, but is only available when \code{USE_STACKCHECK} is
defined (currently on Windows using the Microsoft Visual C++ compiler
and on the Macintosh).  \code{USE_CHECKSTACK} will be defined
automatically; you should never change the definition in your own
code.
\end{cfuncdesc}

\begin{cfuncdesc}{PyOS_sighandler_t}{PyOS_getsig}{int i}
Return the current signal handler for signal \var{i}.
This is a thin wrapper around either \cfunction{sigaction} or
\cfunction{signal}.  Do not call those functions directly!
\ctype{PyOS_sighandler_t} is a typedef alias for \ctype{void (*)(int)}.
\end{cfuncdesc}

\begin{cfuncdesc}{PyOS_sighandler_t}{PyOS_setsig}{int i, PyOS_sighandler_t h}
Set the signal handler for signal \var{i} to be \var{h};
return the old signal handler.
This is a thin wrapper around either \cfunction{sigaction} or
\cfunction{signal}.  Do not call those functions directly!
\ctype{PyOS_sighandler_t} is a typedef alias for \ctype{void (*)(int)}.
\end{cfuncdesc}


\section{Process Control \label{processControl}}

\begin{cfuncdesc}{void}{Py_FatalError}{char *message}
Print a fatal error message and kill the process.  No cleanup is
performed.  This function should only be invoked when a condition is
detected that would make it dangerous to continue using the Python
interpreter; e.g., when the object administration appears to be
corrupted.  On \UNIX{}, the standard C library function
\cfunction{abort()}\ttindex{abort()} is called which will attempt to
produce a \file{core} file.
\end{cfuncdesc}

\begin{cfuncdesc}{void}{Py_Exit}{int status}
Exit the current process.  This calls
\cfunction{Py_Finalize()}\ttindex{Py_Finalize()} and
then calls the standard C library function
\code{exit(\var{status})}\ttindex{exit()}.
\end{cfuncdesc}

\begin{cfuncdesc}{int}{Py_AtExit}{void (*func) ()}
Register a cleanup function to be called by
\cfunction{Py_Finalize()}\ttindex{Py_Finalize()}.
The cleanup function will be called with no arguments and should
return no value.  At most 32 \index{cleanup functions}cleanup
functions can be registered.
When the registration is successful, \cfunction{Py_AtExit()} returns
\code{0}; on failure, it returns \code{-1}.  The cleanup function
registered last is called first.  Each cleanup function will be called
at most once.  Since Python's internal finallization will have
completed before the cleanup function, no Python APIs should be called
by \var{func}.
\end{cfuncdesc}


\section{Importing Modules \label{importing}}

\begin{cfuncdesc}{PyObject*}{PyImport_ImportModule}{char *name}
This is a simplified interface to
\cfunction{PyImport_ImportModuleEx()} below, leaving the
\var{globals} and \var{locals} arguments set to \NULL{}.  When the
\var{name} argument contains a dot (i.e., when it specifies a
submodule of a package), the \var{fromlist} argument is set to the
list \code{['*']} so that the return value is the named module rather
than the top-level package containing it as would otherwise be the
case.  (Unfortunately, this has an additional side effect when
\var{name} in fact specifies a subpackage instead of a submodule: the
submodules specified in the package's \code{__all__} variable are
\index{package variable!\code{__all__}}
\withsubitem{(package variable)}{\ttindex{__all__}}loaded.)  Return a
new reference to the imported module, or
\NULL{} with an exception set on failure (the module may still be
created in this case --- examine \code{sys.modules} to find out).
\withsubitem{(in module sys)}{\ttindex{modules}}
\end{cfuncdesc}

\begin{cfuncdesc}{PyObject*}{PyImport_ImportModuleEx}{char *name, PyObject *globals, PyObject *locals, PyObject *fromlist}
Import a module.  This is best described by referring to the built-in
Python function \function{__import__()}\bifuncindex{__import__}, as
the standard \function{__import__()} function calls this function
directly.

The return value is a new reference to the imported module or
top-level package, or \NULL{} with an exception set on failure
(the module may still be created in this case).  Like for
\function{__import__()}, the return value when a submodule of a
package was requested is normally the top-level package, unless a
non-empty \var{fromlist} was given.
\end{cfuncdesc}

\begin{cfuncdesc}{PyObject*}{PyImport_Import}{PyObject *name}
This is a higher-level interface that calls the current ``import hook
function''.  It invokes the \function{__import__()} function from the
\code{__builtins__} of the current globals.  This means that the
import is done using whatever import hooks are installed in the
current environment, e.g. by \module{rexec}\refstmodindex{rexec} or
\module{ihooks}\refstmodindex{ihooks}.
\end{cfuncdesc}

\begin{cfuncdesc}{PyObject*}{PyImport_ReloadModule}{PyObject *m}
Reload a module.  This is best described by referring to the built-in
Python function \function{reload()}\bifuncindex{reload}, as the standard
\function{reload()} function calls this function directly.  Return a
new reference to the reloaded module, or \NULL{} with an exception set
on failure (the module still exists in this case).
\end{cfuncdesc}

\begin{cfuncdesc}{PyObject*}{PyImport_AddModule}{char *name}
Return the module object corresponding to a module name.  The
\var{name} argument may be of the form \code{package.module}).  First
check the modules dictionary if there's one there, and if not, create
a new one and insert in in the modules dictionary.
Warning: this function does not load or import the module; if the
module wasn't already loaded, you will get an empty module object.
Use \cfunction{PyImport_ImportModule()} or one of its variants to
import a module.
Return \NULL{} with an exception set on failure.
\end{cfuncdesc}

\begin{cfuncdesc}{PyObject*}{PyImport_ExecCodeModule}{char *name, PyObject *co}
Given a module name (possibly of the form \code{package.module}) and a
code object read from a Python bytecode file or obtained from the
built-in function \function{compile()}\bifuncindex{compile}, load the
module.  Return a new reference to the module object, or \NULL{} with
an exception set if an error occurred (the module may still be created
in this case).  (This function would reload the module if it was
already imported.)
\end{cfuncdesc}

\begin{cfuncdesc}{long}{PyImport_GetMagicNumber}{}
Return the magic number for Python bytecode files (a.k.a.
\file{.pyc} and \file{.pyo} files).  The magic number should be
present in the first four bytes of the bytecode file, in little-endian
byte order.
\end{cfuncdesc}

\begin{cfuncdesc}{PyObject*}{PyImport_GetModuleDict}{}
Return the dictionary used for the module administration
(a.k.a. \code{sys.modules}).  Note that this is a per-interpreter
variable.
\end{cfuncdesc}

\begin{cfuncdesc}{void}{_PyImport_Init}{}
Initialize the import mechanism.  For internal use only.
\end{cfuncdesc}

\begin{cfuncdesc}{void}{PyImport_Cleanup}{}
Empty the module table.  For internal use only.
\end{cfuncdesc}

\begin{cfuncdesc}{void}{_PyImport_Fini}{}
Finalize the import mechanism.  For internal use only.
\end{cfuncdesc}

\begin{cfuncdesc}{PyObject*}{_PyImport_FindExtension}{char *, char *}
For internal use only.
\end{cfuncdesc}

\begin{cfuncdesc}{PyObject*}{_PyImport_FixupExtension}{char *, char *}
For internal use only.
\end{cfuncdesc}

\begin{cfuncdesc}{int}{PyImport_ImportFrozenModule}{char *name}
Load a frozen module named \var{name}.  Return \code{1} for success,
\code{0} if the module is not found, and \code{-1} with an exception
set if the initialization failed.  To access the imported module on a
successful load, use \cfunction{PyImport_ImportModule()}.
(Note the misnomer --- this function would reload the module if it was
already imported.)
\end{cfuncdesc}

\begin{ctypedesc}[_frozen]{struct _frozen}
This is the structure type definition for frozen module descriptors,
as generated by the \program{freeze}\index{freeze utility} utility
(see \file{Tools/freeze/} in the Python source distribution).  Its
definition, found in \file{Include/import.h}, is:

\begin{verbatim}
struct _frozen {
    char *name;
    unsigned char *code;
    int size;
};
\end{verbatim}
\end{ctypedesc}

\begin{cvardesc}{struct _frozen*}{PyImport_FrozenModules}
This pointer is initialized to point to an array of \ctype{struct
_frozen} records, terminated by one whose members are all
\NULL{} or zero.  When a frozen module is imported, it is searched in
this table.  Third-party code could play tricks with this to provide a 
dynamically created collection of frozen modules.
\end{cvardesc}

\begin{cfuncdesc}{int}{PyImport_AppendInittab}{char *name,
                                               void (*initfunc)(void)}
Add a single module to the existing table of built-in modules.  This
is a convenience wrapper around \cfunction{PyImport_ExtendInittab()},
returning \code{-1} if the table could not be extended.  The new
module can be imported by the name \var{name}, and uses the function
\var{initfunc} as the initialization function called on the first
attempted import.  This should be called before
\cfunction{Py_Initialize()}.
\end{cfuncdesc}

\begin{ctypedesc}[_inittab]{struct _inittab}
Structure describing a single entry in the list of built-in modules.
Each of these structures gives the name and initialization function
for a module built into the interpreter.  Programs which embed Python
may use an array of these structures in conjunction with
\cfunction{PyImport_ExtendInittab()} to provide additional built-in
modules.  The structure is defined in \file{Include/import.h} as:

\begin{verbatim}
struct _inittab {
    char *name;
    void (*initfunc)(void);
};
\end{verbatim}
\end{ctypedesc}

\begin{cfuncdesc}{int}{PyImport_ExtendInittab}{struct _inittab *newtab}
Add a collection of modules to the table of built-in modules.  The
\var{newtab} array must end with a sentinel entry which contains
\NULL{} for the \member{name} field; failure to provide the sentinel
value can result in a memory fault.  Returns \code{0} on success or
\code{-1} if insufficient memory could be allocated to extend the
internal table.  In the event of failure, no modules are added to the
internal table.  This should be called before
\cfunction{Py_Initialize()}.
\end{cfuncdesc}


\chapter{Abstract Objects Layer \label{abstract}}

The functions in this chapter interact with Python objects regardless
of their type, or with wide classes of object types (e.g. all
numerical types, or all sequence types).  When used on object types
for which they do not apply, they will raise a Python exception.

\section{Object Protocol \label{object}}

\begin{cfuncdesc}{int}{PyObject_Print}{PyObject *o, FILE *fp, int flags}
Print an object \var{o}, on file \var{fp}.  Returns \code{-1} on error.
The flags argument is used to enable certain printing options.  The
only option currently supported is \constant{Py_PRINT_RAW}; if given,
the \function{str()} of the object is written instead of the
\function{repr()}.
\end{cfuncdesc}

\begin{cfuncdesc}{int}{PyObject_HasAttrString}{PyObject *o, char *attr_name}
Returns \code{1} if \var{o} has the attribute \var{attr_name}, and
\code{0} otherwise.  This is equivalent to the Python expression
\samp{hasattr(\var{o}, \var{attr_name})}.
This function always succeeds.
\end{cfuncdesc}

\begin{cfuncdesc}{PyObject*}{PyObject_GetAttrString}{PyObject *o,
                                                     char *attr_name}
Retrieve an attribute named \var{attr_name} from object \var{o}.
Returns the attribute value on success, or \NULL{} on failure.
This is the equivalent of the Python expression
\samp{\var{o}.\var{attr_name}}.
\end{cfuncdesc}


\begin{cfuncdesc}{int}{PyObject_HasAttr}{PyObject *o, PyObject *attr_name}
Returns \code{1} if \var{o} has the attribute \var{attr_name}, and
\code{0} otherwise.  This is equivalent to the Python expression
\samp{hasattr(\var{o}, \var{attr_name})}. 
This function always succeeds.
\end{cfuncdesc}


\begin{cfuncdesc}{PyObject*}{PyObject_GetAttr}{PyObject *o,
                                               PyObject *attr_name}
Retrieve an attribute named \var{attr_name} from object \var{o}.
Returns the attribute value on success, or \NULL{} on failure.
This is the equivalent of the Python expression
\samp{\var{o}.\var{attr_name}}.
\end{cfuncdesc}


\begin{cfuncdesc}{int}{PyObject_SetAttrString}{PyObject *o, char *attr_name, PyObject *v}
Set the value of the attribute named \var{attr_name}, for object
\var{o}, to the value \var{v}. Returns \code{-1} on failure.  This is
the equivalent of the Python statement \samp{\var{o}.\var{attr_name} =
\var{v}}.
\end{cfuncdesc}


\begin{cfuncdesc}{int}{PyObject_SetAttr}{PyObject *o, PyObject *attr_name, PyObject *v}
Set the value of the attribute named \var{attr_name}, for
object \var{o},
to the value \var{v}. Returns \code{-1} on failure.  This is
the equivalent of the Python statement \samp{\var{o}.\var{attr_name} =
\var{v}}.
\end{cfuncdesc}


\begin{cfuncdesc}{int}{PyObject_DelAttrString}{PyObject *o, char *attr_name}
Delete attribute named \var{attr_name}, for object \var{o}. Returns
\code{-1} on failure.  This is the equivalent of the Python
statement: \samp{del \var{o}.\var{attr_name}}.
\end{cfuncdesc}


\begin{cfuncdesc}{int}{PyObject_DelAttr}{PyObject *o, PyObject *attr_name}
Delete attribute named \var{attr_name}, for object \var{o}. Returns
\code{-1} on failure.  This is the equivalent of the Python
statement \samp{del \var{o}.\var{attr_name}}.
\end{cfuncdesc}


\begin{cfuncdesc}{int}{PyObject_Cmp}{PyObject *o1, PyObject *o2, int *result}
Compare the values of \var{o1} and \var{o2} using a routine provided
by \var{o1}, if one exists, otherwise with a routine provided by
\var{o2}.  The result of the comparison is returned in \var{result}.
Returns \code{-1} on failure.  This is the equivalent of the Python
statement\bifuncindex{cmp} \samp{\var{result} = cmp(\var{o1}, \var{o2})}.
\end{cfuncdesc}


\begin{cfuncdesc}{int}{PyObject_Compare}{PyObject *o1, PyObject *o2}
Compare the values of \var{o1} and \var{o2} using a routine provided
by \var{o1}, if one exists, otherwise with a routine provided by
\var{o2}.  Returns the result of the comparison on success.  On error,
the value returned is undefined; use \cfunction{PyErr_Occurred()} to
detect an error.  This is equivalent to the Python
expression\bifuncindex{cmp} \samp{cmp(\var{o1}, \var{o2})}.
\end{cfuncdesc}


\begin{cfuncdesc}{PyObject*}{PyObject_Repr}{PyObject *o}
Compute a string representation of object \var{o}.  Returns the
string representation on success, \NULL{} on failure.  This is
the equivalent of the Python expression \samp{repr(\var{o})}.
Called by the \function{repr()}\bifuncindex{repr} built-in function
and by reverse quotes.
\end{cfuncdesc}


\begin{cfuncdesc}{PyObject*}{PyObject_Str}{PyObject *o}
Compute a string representation of object \var{o}.  Returns the
string representation on success, \NULL{} on failure.  This is
the equivalent of the Python expression \samp{str(\var{o})}.
Called by the \function{str()}\bifuncindex{str} built-in function and
by the \keyword{print} statement.
\end{cfuncdesc}


\begin{cfuncdesc}{PyObject*}{PyObject_Unicode}{PyObject *o}
Compute a Unicode string representation of object \var{o}.  Returns the
Unicode string representation on success, \NULL{} on failure.  This is
the equivalent of the Python expression \samp{unistr(\var{o})}.
Called by the \function{unistr()}\bifuncindex{unistr} built-in function.
\end{cfuncdesc}

\begin{cfuncdesc}{int}{PyObject_IsInstance}{PyObject *inst, PyObject *cls}
Return \code{1} if \var{inst} is an instance of the class \var{cls} or
a subclass of \var{cls}.  If \var{cls} is a type object rather than a
class object, \cfunction{PyObject_IsInstance()} returns \code{1} if
\var{inst} is of type \var{cls}.  If \var{inst} is not a class
instance and \var{cls} is neither a type object or class object,
\var{inst} must have a \member{__class__} attribute --- the class
relationship of the value of that attribute with \var{cls} will be
used to determine the result of this function.
\versionadded{2.1}
\end{cfuncdesc}

Subclass determination is done in a fairly straightforward way, but
includes a wrinkle that implementors of extensions to the class system
may want to be aware of.  If \class{A} and \class{B} are class
objects, \class{B} is a subclass of \class{A} if it inherits from
\class{A} either directly or indirectly.  If either is not a class
object, a more general mechanism is used to determine the class
relationship of the two objects.  When testing if \var{B} is a
subclass of \var{A}, if \var{A} is \var{B},
\cfunction{PyObject_IsSubclass()} returns true.  If \var{A} and
\var{B} are different objects, \var{B}'s \member{__bases__} attribute
is searched in a depth-first fashion for \var{A} --- the presence of
the \member{__bases__} attribute is considered sufficient for this
determination.

\begin{cfuncdesc}{int}{PyObject_IsSubclass}{PyObject *derived,
                                            PyObject *cls}
Returns \code{1} if the class \var{derived} is identical to or derived
from the class \var{cls}, otherwise returns \code{0}.  In case of an
error, returns \code{-1}.  If either \var{derived} or \var{cls} is not
an actual class object, this function uses the generic algorithm
described above.
\versionadded{2.1}
\end{cfuncdesc}


\begin{cfuncdesc}{int}{PyCallable_Check}{PyObject *o}
Determine if the object \var{o} is callable.  Return \code{1} if the
object is callable and \code{0} otherwise.
This function always succeeds.
\end{cfuncdesc}


\begin{cfuncdesc}{PyObject*}{PyObject_CallObject}{PyObject *callable_object,
                                                  PyObject *args}
Call a callable Python object \var{callable_object}, with
arguments given by the tuple \var{args}.  If no arguments are
needed, then \var{args} may be \NULL{}.  Returns the result of the
call on success, or \NULL{} on failure.  This is the equivalent
of the Python expression \samp{apply(\var{callable_object}, \var{args})}.
\bifuncindex{apply}
\end{cfuncdesc}

\begin{cfuncdesc}{PyObject*}{PyObject_CallFunction}{PyObject *callable_object,
                                                    char *format, ...}
Call a callable Python object \var{callable_object}, with a
variable number of C arguments. The C arguments are described
using a \cfunction{Py_BuildValue()} style format string. The format may
be \NULL{}, indicating that no arguments are provided.  Returns the
result of the call on success, or \NULL{} on failure.  This is
the equivalent of the Python expression \samp{apply(\var{callable_object},
\var{args})}.\bifuncindex{apply}
\end{cfuncdesc}


\begin{cfuncdesc}{PyObject*}{PyObject_CallMethod}{PyObject *o,
                                           char *method, char *format, ...}
Call the method named \var{m} of object \var{o} with a variable number
of C arguments.  The C arguments are described by a
\cfunction{Py_BuildValue()} format string.  The format may be \NULL{},
indicating that no arguments are provided. Returns the result of the
call on success, or \NULL{} on failure.  This is the equivalent of the
Python expression \samp{\var{o}.\var{method}(\var{args})}.
Note that special method names, such as \method{__add__()},
\method{__getitem__()}, and so on are not supported.  The specific
abstract-object routines for these must be used.
\end{cfuncdesc}


\begin{cfuncdesc}{int}{PyObject_Hash}{PyObject *o}
Compute and return the hash value of an object \var{o}.  On
failure, return \code{-1}.  This is the equivalent of the Python
expression \samp{hash(\var{o})}.\bifuncindex{hash}
\end{cfuncdesc}


\begin{cfuncdesc}{int}{PyObject_IsTrue}{PyObject *o}
Returns \code{1} if the object \var{o} is considered to be true, and
\code{0} otherwise. This is equivalent to the Python expression
\samp{not not \var{o}}.
This function always succeeds.
\end{cfuncdesc}


\begin{cfuncdesc}{PyObject*}{PyObject_Type}{PyObject *o}
On success, returns a type object corresponding to the object
type of object \var{o}. On failure, returns \NULL{}.  This is
equivalent to the Python expression \samp{type(\var{o})}.
\bifuncindex{type}
\end{cfuncdesc}

\begin{cfuncdesc}{int}{PyObject_Length}{PyObject *o}
Return the length of object \var{o}.  If the object \var{o} provides
both sequence and mapping protocols, the sequence length is
returned.  On error, \code{-1} is returned.  This is the equivalent
to the Python expression \samp{len(\var{o})}.\bifuncindex{len}
\end{cfuncdesc}


\begin{cfuncdesc}{PyObject*}{PyObject_GetItem}{PyObject *o, PyObject *key}
Return element of \var{o} corresponding to the object \var{key} or
\NULL{} on failure. This is the equivalent of the Python expression
\samp{\var{o}[\var{key}]}.
\end{cfuncdesc}


\begin{cfuncdesc}{int}{PyObject_SetItem}{PyObject *o, PyObject *key, PyObject *v}
Map the object \var{key} to the value \var{v}.
Returns \code{-1} on failure.  This is the equivalent
of the Python statement \samp{\var{o}[\var{key}] = \var{v}}.
\end{cfuncdesc}


\begin{cfuncdesc}{int}{PyObject_DelItem}{PyObject *o, PyObject *key}
Delete the mapping for \var{key} from \var{o}.  Returns \code{-1} on
failure. This is the equivalent of the Python statement \samp{del
\var{o}[\var{key}]}.
\end{cfuncdesc}

\begin{cfuncdesc}{int}{PyObject_AsFileDescriptor}{PyObject *o}
Derives a file-descriptor from a Python object.  If the object
is an integer or long integer, its value is returned.  If not, the
object's \method{fileno()} method is called if it exists; the method
must return an integer or long integer, which is returned as the file
descriptor value.  Returns \code{-1} on failure.
\end{cfuncdesc}

\section{Number Protocol \label{number}}

\begin{cfuncdesc}{int}{PyNumber_Check}{PyObject *o}
Returns \code{1} if the object \var{o} provides numeric protocols, and
false otherwise. 
This function always succeeds.
\end{cfuncdesc}


\begin{cfuncdesc}{PyObject*}{PyNumber_Add}{PyObject *o1, PyObject *o2}
Returns the result of adding \var{o1} and \var{o2}, or \NULL{} on
failure.  This is the equivalent of the Python expression
\samp{\var{o1} + \var{o2}}.
\end{cfuncdesc}


\begin{cfuncdesc}{PyObject*}{PyNumber_Subtract}{PyObject *o1, PyObject *o2}
Returns the result of subtracting \var{o2} from \var{o1}, or
\NULL{} on failure.  This is the equivalent of the Python expression
\samp{\var{o1} - \var{o2}}.
\end{cfuncdesc}


\begin{cfuncdesc}{PyObject*}{PyNumber_Multiply}{PyObject *o1, PyObject *o2}
Returns the result of multiplying \var{o1} and \var{o2}, or \NULL{} on
failure.  This is the equivalent of the Python expression
\samp{\var{o1} * \var{o2}}.
\end{cfuncdesc}


\begin{cfuncdesc}{PyObject*}{PyNumber_Divide}{PyObject *o1, PyObject *o2}
Returns the result of dividing \var{o1} by \var{o2}, or \NULL{} on
failure. 
This is the equivalent of the Python expression \samp{\var{o1} /
\var{o2}}.
\end{cfuncdesc}


\begin{cfuncdesc}{PyObject*}{PyNumber_Remainder}{PyObject *o1, PyObject *o2}
Returns the remainder of dividing \var{o1} by \var{o2}, or \NULL{} on
failure.  This is the equivalent of the Python expression
\samp{\var{o1} \%\ \var{o2}}.
\end{cfuncdesc}


\begin{cfuncdesc}{PyObject*}{PyNumber_Divmod}{PyObject *o1, PyObject *o2}
See the built-in function \function{divmod()}\bifuncindex{divmod}.
Returns \NULL{} on failure.  This is the equivalent of the Python
expression \samp{divmod(\var{o1}, \var{o2})}.
\end{cfuncdesc}


\begin{cfuncdesc}{PyObject*}{PyNumber_Power}{PyObject *o1, PyObject *o2, PyObject *o3}
See the built-in function \function{pow()}\bifuncindex{pow}.  Returns
\NULL{} on failure. This is the equivalent of the Python expression
\samp{pow(\var{o1}, \var{o2}, \var{o3})}, where \var{o3} is optional.
If \var{o3} is to be ignored, pass \cdata{Py_None} in its place
(passing \NULL{} for \var{o3} would cause an illegal memory access).
\end{cfuncdesc}


\begin{cfuncdesc}{PyObject*}{PyNumber_Negative}{PyObject *o}
Returns the negation of \var{o} on success, or \NULL{} on failure.
This is the equivalent of the Python expression \samp{-\var{o}}.
\end{cfuncdesc}


\begin{cfuncdesc}{PyObject*}{PyNumber_Positive}{PyObject *o}
Returns \var{o} on success, or \NULL{} on failure.
This is the equivalent of the Python expression \samp{+\var{o}}.
\end{cfuncdesc}


\begin{cfuncdesc}{PyObject*}{PyNumber_Absolute}{PyObject *o}
Returns the absolute value of \var{o}, or \NULL{} on failure.  This is
the equivalent of the Python expression \samp{abs(\var{o})}.
\bifuncindex{abs}
\end{cfuncdesc}


\begin{cfuncdesc}{PyObject*}{PyNumber_Invert}{PyObject *o}
Returns the bitwise negation of \var{o} on success, or \NULL{} on
failure.  This is the equivalent of the Python expression
\samp{\~\var{o}}.
\end{cfuncdesc}


\begin{cfuncdesc}{PyObject*}{PyNumber_Lshift}{PyObject *o1, PyObject *o2}
Returns the result of left shifting \var{o1} by \var{o2} on success,
or \NULL{} on failure.  This is the equivalent of the Python
expression \samp{\var{o1} <\code{<} \var{o2}}.
\end{cfuncdesc}


\begin{cfuncdesc}{PyObject*}{PyNumber_Rshift}{PyObject *o1, PyObject *o2}
Returns the result of right shifting \var{o1} by \var{o2} on success,
or \NULL{} on failure.  This is the equivalent of the Python
expression \samp{\var{o1} >\code{>} \var{o2}}.
\end{cfuncdesc}


\begin{cfuncdesc}{PyObject*}{PyNumber_And}{PyObject *o1, PyObject *o2}
Returns the ``bitwise and'' of \var{o2} and \var{o2} on success and
\NULL{} on failure. This is the equivalent of the Python expression
\samp{\var{o1} \&\ \var{o2}}.
\end{cfuncdesc}


\begin{cfuncdesc}{PyObject*}{PyNumber_Xor}{PyObject *o1, PyObject *o2}
Returns the ``bitwise exclusive or'' of \var{o1} by \var{o2} on success,
or \NULL{} on failure.  This is the equivalent of the Python
expression \samp{\var{o1} \^{ }\var{o2}}.
\end{cfuncdesc}

\begin{cfuncdesc}{PyObject*}{PyNumber_Or}{PyObject *o1, PyObject *o2}
Returns the ``bitwise or'' of \var{o1} and \var{o2} on success, or
\NULL{} on failure.  This is the equivalent of the Python expression
\samp{\var{o1} | \var{o2}}.
\end{cfuncdesc}


\begin{cfuncdesc}{PyObject*}{PyNumber_InPlaceAdd}{PyObject *o1, PyObject *o2}
Returns the result of adding \var{o1} and \var{o2}, or \NULL{} on failure. 
The operation is done \emph{in-place} when \var{o1} supports it.  This is the
equivalent of the Python expression \samp{\var{o1} += \var{o2}}.
\end{cfuncdesc}


\begin{cfuncdesc}{PyObject*}{PyNumber_InPlaceSubtract}{PyObject *o1, PyObject *o2}
Returns the result of subtracting \var{o2} from \var{o1}, or
\NULL{} on failure.  The operation is done \emph{in-place} when \var{o1}
supports it.  This is the equivalent of the Python expression \samp{\var{o1}
-= \var{o2}}.
\end{cfuncdesc}


\begin{cfuncdesc}{PyObject*}{PyNumber_InPlaceMultiply}{PyObject *o1, PyObject *o2}
Returns the result of multiplying \var{o1} and \var{o2}, or \NULL{} on
failure.  The operation is done \emph{in-place} when \var{o1} supports it. 
This is the equivalent of the Python expression \samp{\var{o1} *= \var{o2}}.
\end{cfuncdesc}


\begin{cfuncdesc}{PyObject*}{PyNumber_InPlaceDivide}{PyObject *o1, PyObject *o2}
Returns the result of dividing \var{o1} by \var{o2}, or \NULL{} on failure. 
The operation is done \emph{in-place} when \var{o1} supports it. This is the
equivalent of the Python expression \samp{\var{o1} /= \var{o2}}.
\end{cfuncdesc}


\begin{cfuncdesc}{PyObject*}{PyNumber_InPlaceRemainder}{PyObject *o1, PyObject *o2}
Returns the remainder of dividing \var{o1} by \var{o2}, or \NULL{} on
failure.  The operation is done \emph{in-place} when \var{o1} supports it. 
This is the equivalent of the Python expression \samp{\var{o1} \%= \var{o2}}.
\end{cfuncdesc}


\begin{cfuncdesc}{PyObject*}{PyNumber_InPlacePower}{PyObject *o1, PyObject *o2, PyObject *o3}
See the built-in function \function{pow()}\bifuncindex{pow}.  Returns
\NULL{} on failure.  The operation is done \emph{in-place} when \var{o1}
supports it.  This is the equivalent of the Python expression \samp{\var{o1}
**= \var{o2}} when o3 is \cdata{Py_None}, or an in-place variant of
\samp{pow(\var{o1}, \var{o2}, \var{o3})} otherwise. If \var{o3} is to be
ignored, pass \cdata{Py_None} in its place (passing \NULL{} for \var{o3}
would cause an illegal memory access).
\end{cfuncdesc}

\begin{cfuncdesc}{PyObject*}{PyNumber_InPlaceLshift}{PyObject *o1, PyObject *o2}
Returns the result of left shifting \var{o1} by \var{o2} on success, or
\NULL{} on failure.  The operation is done \emph{in-place} when \var{o1}
supports it.  This is the equivalent of the Python expression \samp{\var{o1}
<\code{<=} \var{o2}}.
\end{cfuncdesc}


\begin{cfuncdesc}{PyObject*}{PyNumber_InPlaceRshift}{PyObject *o1, PyObject *o2}
Returns the result of right shifting \var{o1} by \var{o2} on success, or
\NULL{} on failure.  The operation is done \emph{in-place} when \var{o1}
supports it.  This is the equivalent of the Python expression \samp{\var{o1}
>\code{>=} \var{o2}}.
\end{cfuncdesc}


\begin{cfuncdesc}{PyObject*}{PyNumber_InPlaceAnd}{PyObject *o1, PyObject *o2}
Returns the ``bitwise and'' of \var{o1} and \var{o2} on success
and \NULL{} on failure. The operation is done \emph{in-place} when
\var{o1} supports it.  This is the equivalent of the Python expression
\samp{\var{o1} \&= \var{o2}}.
\end{cfuncdesc}


\begin{cfuncdesc}{PyObject*}{PyNumber_InPlaceXor}{PyObject *o1, PyObject *o2}
Returns the ``bitwise exclusive or'' of \var{o1} by \var{o2} on success, or
\NULL{} on failure.  The operation is done \emph{in-place} when \var{o1}
supports it.  This is the equivalent of the Python expression \samp{\var{o1}
\^= \var{o2}}.
\end{cfuncdesc}

\begin{cfuncdesc}{PyObject*}{PyNumber_InPlaceOr}{PyObject *o1, PyObject *o2}
Returns the ``bitwise or'' of \var{o1} and \var{o2} on success, or \NULL{}
on failure.  The operation is done \emph{in-place} when \var{o1} supports
it.  This is the equivalent of the Python expression \samp{\var{o1} |=
\var{o2}}.
\end{cfuncdesc}

\begin{cfuncdesc}{int}{PyNumber_Coerce}{PyObject **p1, PyObject **p2}
This function takes the addresses of two variables of type
\ctype{PyObject*}.  If the objects pointed to by \code{*\var{p1}} and
\code{*\var{p2}} have the same type, increment their reference count
and return \code{0} (success). If the objects can be converted to a
common numeric type, replace \code{*p1} and \code{*p2} by their
converted value (with 'new' reference counts), and return \code{0}.
If no conversion is possible, or if some other error occurs, return
\code{-1} (failure) and don't increment the reference counts.  The
call \code{PyNumber_Coerce(\&o1, \&o2)} is equivalent to the Python
statement \samp{\var{o1}, \var{o2} = coerce(\var{o1}, \var{o2})}.
\bifuncindex{coerce}
\end{cfuncdesc}

\begin{cfuncdesc}{PyObject*}{PyNumber_Int}{PyObject *o}
Returns the \var{o} converted to an integer object on success, or
\NULL{} on failure.  This is the equivalent of the Python
expression \samp{int(\var{o})}.\bifuncindex{int}
\end{cfuncdesc}

\begin{cfuncdesc}{PyObject*}{PyNumber_Long}{PyObject *o}
Returns the \var{o} converted to a long integer object on success,
or \NULL{} on failure.  This is the equivalent of the Python
expression \samp{long(\var{o})}.\bifuncindex{long}
\end{cfuncdesc}

\begin{cfuncdesc}{PyObject*}{PyNumber_Float}{PyObject *o}
Returns the \var{o} converted to a float object on success, or
\NULL{} on failure.  This is the equivalent of the Python expression
\samp{float(\var{o})}.\bifuncindex{float}
\end{cfuncdesc}


\section{Sequence Protocol \label{sequence}}

\begin{cfuncdesc}{int}{PySequence_Check}{PyObject *o}
Return \code{1} if the object provides sequence protocol, and
\code{0} otherwise.  This function always succeeds.
\end{cfuncdesc}

\begin{cfuncdesc}{int}{PySequence_Size}{PyObject *o}
Returns the number of objects in sequence \var{o} on success, and
\code{-1} on failure.  For objects that do not provide sequence
protocol, this is equivalent to the Python expression
\samp{len(\var{o})}.\bifuncindex{len}
\end{cfuncdesc}

\begin{cfuncdesc}{int}{PySequence_Length}{PyObject *o}
Alternate name for \cfunction{PySequence_Size()}.
\end{cfuncdesc}

\begin{cfuncdesc}{PyObject*}{PySequence_Concat}{PyObject *o1, PyObject *o2}
Return the concatenation of \var{o1} and \var{o2} on success, and \NULL{} on
failure.   This is the equivalent of the Python
expression \samp{\var{o1} + \var{o2}}.
\end{cfuncdesc}


\begin{cfuncdesc}{PyObject*}{PySequence_Repeat}{PyObject *o, int count}
Return the result of repeating sequence object
\var{o} \var{count} times, or \NULL{} on failure.  This is the
equivalent of the Python expression \samp{\var{o} * \var{count}}.
\end{cfuncdesc}

\begin{cfuncdesc}{PyObject*}{PySequence_InPlaceConcat}{PyObject *o1, PyObject *o2}
Return the concatenation of \var{o1} and \var{o2} on success, and \NULL{} on
failure.  The operation is done \emph{in-place} when \var{o1} supports it. 
This is the equivalent of the Python expression \samp{\var{o1} += \var{o2}}.
\end{cfuncdesc}


\begin{cfuncdesc}{PyObject*}{PySequence_InPlaceRepeat}{PyObject *o, int count}
Return the result of repeating sequence object \var{o} \var{count} times, or
\NULL{} on failure.  The operation is done \emph{in-place} when \var{o}
supports it.  This is the equivalent of the Python expression \samp{\var{o}
*= \var{count}}.
\end{cfuncdesc}


\begin{cfuncdesc}{PyObject*}{PySequence_GetItem}{PyObject *o, int i}
Return the \var{i}th element of \var{o}, or \NULL{} on failure. This
is the equivalent of the Python expression \samp{\var{o}[\var{i}]}.
\end{cfuncdesc}


\begin{cfuncdesc}{PyObject*}{PySequence_GetSlice}{PyObject *o, int i1, int i2}
Return the slice of sequence object \var{o} between \var{i1} and
\var{i2}, or \NULL{} on failure. This is the equivalent of the Python
expression \samp{\var{o}[\var{i1}:\var{i2}]}.
\end{cfuncdesc}


\begin{cfuncdesc}{int}{PySequence_SetItem}{PyObject *o, int i, PyObject *v}
Assign object \var{v} to the \var{i}th element of \var{o}.
Returns \code{-1} on failure.  This is the equivalent of the Python
statement \samp{\var{o}[\var{i}] = \var{v}}.
\end{cfuncdesc}

\begin{cfuncdesc}{int}{PySequence_DelItem}{PyObject *o, int i}
Delete the \var{i}th element of object \var{o}.  Returns
\code{-1} on failure.  This is the equivalent of the Python
statement \samp{del \var{o}[\var{i}]}.
\end{cfuncdesc}

\begin{cfuncdesc}{int}{PySequence_SetSlice}{PyObject *o, int i1,
                                            int i2, PyObject *v}
Assign the sequence object \var{v} to the slice in sequence
object \var{o} from \var{i1} to \var{i2}.  This is the equivalent of
the Python statement \samp{\var{o}[\var{i1}:\var{i2}] = \var{v}}.
\end{cfuncdesc}

\begin{cfuncdesc}{int}{PySequence_DelSlice}{PyObject *o, int i1, int i2}
Delete the slice in sequence object \var{o} from \var{i1} to \var{i2}.
Returns \code{-1} on failure. This is the equivalent of the Python
statement \samp{del \var{o}[\var{i1}:\var{i2}]}.
\end{cfuncdesc}

\begin{cfuncdesc}{PyObject*}{PySequence_Tuple}{PyObject *o}
Returns the \var{o} as a tuple on success, and \NULL{} on failure.
This is equivalent to the Python expression \samp{tuple(\var{o})}.
\bifuncindex{tuple}
\end{cfuncdesc}

\begin{cfuncdesc}{int}{PySequence_Count}{PyObject *o, PyObject *value}
Return the number of occurrences of \var{value} in \var{o}, that is,
return the number of keys for which \code{\var{o}[\var{key}] ==
\var{value}}.  On failure, return \code{-1}.  This is equivalent to
the Python expression \samp{\var{o}.count(\var{value})}.
\end{cfuncdesc}

\begin{cfuncdesc}{int}{PySequence_Contains}{PyObject *o, PyObject *value}
Determine if \var{o} contains \var{value}.  If an item in \var{o} is
equal to \var{value}, return \code{1}, otherwise return \code{0}.  On
error, return \code{-1}.  This is equivalent to the Python expression
\samp{\var{value} in \var{o}}.
\end{cfuncdesc}

\begin{cfuncdesc}{int}{PySequence_Index}{PyObject *o, PyObject *value}
Return the first index \var{i} for which \code{\var{o}[\var{i}] ==
\var{value}}.  On error, return \code{-1}.    This is equivalent to
the Python expression \samp{\var{o}.index(\var{value})}.
\end{cfuncdesc}

\begin{cfuncdesc}{PyObject*}{PySequence_List}{PyObject *o}
Return a list object with the same contents as the arbitrary sequence
\var{o}.  The returned list is guaranteed to be new.
\end{cfuncdesc}

\begin{cfuncdesc}{PyObject*}{PySequence_Tuple}{PyObject *o}
Return a tuple object with the same contents as the arbitrary sequence
\var{o}.  If \var{o} is a tuple, a new reference will be returned,
otherwise a tuple will be constructed with the appropriate contents.
\end{cfuncdesc}


\begin{cfuncdesc}{PyObject*}{PySequence_Fast}{PyObject *o, const char *m}
Returns the sequence \var{o} as a tuple, unless it is already a
tuple or list, in which case \var{o} is returned.  Use
\cfunction{PySequence_Fast_GET_ITEM()} to access the members of the
result.  Returns \NULL{} on failure.  If the object is not a sequence,
raises \exception{TypeError} with \var{m} as the message text.
\end{cfuncdesc}

\begin{cfuncdesc}{PyObject*}{PySequence_Fast_GET_ITEM}{PyObject *o, int i}
Return the \var{i}th element of \var{o}, assuming that \var{o} was
returned by \cfunction{PySequence_Fast()}, and that \var{i} is within
bounds.  The caller is expected to get the length of the sequence by
calling \cfunction{PyObject_Size()} on \var{o}, since lists and tuples
are guaranteed to always return their true length.
\end{cfuncdesc}


\section{Mapping Protocol \label{mapping}}

\begin{cfuncdesc}{int}{PyMapping_Check}{PyObject *o}
Return \code{1} if the object provides mapping protocol, and
\code{0} otherwise.  This function always succeeds.
\end{cfuncdesc}


\begin{cfuncdesc}{int}{PyMapping_Length}{PyObject *o}
Returns the number of keys in object \var{o} on success, and
\code{-1} on failure.  For objects that do not provide mapping
protocol, this is equivalent to the Python expression
\samp{len(\var{o})}.\bifuncindex{len}
\end{cfuncdesc}


\begin{cfuncdesc}{int}{PyMapping_DelItemString}{PyObject *o, char *key}
Remove the mapping for object \var{key} from the object \var{o}.
Return \code{-1} on failure.  This is equivalent to
the Python statement \samp{del \var{o}[\var{key}]}.
\end{cfuncdesc}


\begin{cfuncdesc}{int}{PyMapping_DelItem}{PyObject *o, PyObject *key}
Remove the mapping for object \var{key} from the object \var{o}.
Return \code{-1} on failure.  This is equivalent to
the Python statement \samp{del \var{o}[\var{key}]}.
\end{cfuncdesc}


\begin{cfuncdesc}{int}{PyMapping_HasKeyString}{PyObject *o, char *key}
On success, return \code{1} if the mapping object has the key
\var{key} and \code{0} otherwise.  This is equivalent to the Python
expression \samp{\var{o}.has_key(\var{key})}. 
This function always succeeds.
\end{cfuncdesc}


\begin{cfuncdesc}{int}{PyMapping_HasKey}{PyObject *o, PyObject *key}
Return \code{1} if the mapping object has the key \var{key} and
\code{0} otherwise.  This is equivalent to the Python expression
\samp{\var{o}.has_key(\var{key})}. 
This function always succeeds.
\end{cfuncdesc}


\begin{cfuncdesc}{PyObject*}{PyMapping_Keys}{PyObject *o}
On success, return a list of the keys in object \var{o}.  On
failure, return \NULL{}. This is equivalent to the Python
expression \samp{\var{o}.keys()}.
\end{cfuncdesc}


\begin{cfuncdesc}{PyObject*}{PyMapping_Values}{PyObject *o}
On success, return a list of the values in object \var{o}.  On
failure, return \NULL{}. This is equivalent to the Python
expression \samp{\var{o}.values()}.
\end{cfuncdesc}


\begin{cfuncdesc}{PyObject*}{PyMapping_Items}{PyObject *o}
On success, return a list of the items in object \var{o}, where
each item is a tuple containing a key-value pair.  On
failure, return \NULL{}. This is equivalent to the Python
expression \samp{\var{o}.items()}.
\end{cfuncdesc}


\begin{cfuncdesc}{PyObject*}{PyMapping_GetItemString}{PyObject *o, char *key}
Return element of \var{o} corresponding to the object \var{key} or
\NULL{} on failure. This is the equivalent of the Python expression
\samp{\var{o}[\var{key}]}.
\end{cfuncdesc}

\begin{cfuncdesc}{int}{PyMapping_SetItemString}{PyObject *o, char *key,
                                                PyObject *v}
Map the object \var{key} to the value \var{v} in object \var{o}.
Returns \code{-1} on failure.  This is the equivalent of the Python
statement \samp{\var{o}[\var{key}] = \var{v}}.
\end{cfuncdesc}


\section{Iterator Protocol \label{iterator}}

\versionadded{2.2}

There are only a couple of functions specifically for working with
iterators.

\begin{cfuncdesc}{int}{PyIter_Check}{PyObject *o}
  Return true if the object \var{o} supports the iterator protocol.
\end{cfuncdesc}

\begin{cfuncdesc}{PyObject*}{PyIter_Next}{PyObject *o}
  Return the next value from the iteration \var{o}.  If the object is
  an iterator, this retrieves the next value from the iteration, and
  returns \NULL{} with no exception set if there are no remaining
  items.  If the object is not an iterator, \exception{TypeError} is
  raised, or if there is an error in retrieving the item, returns
  \NULL{} and passes along the exception.
\end{cfuncdesc}

To write a loop which iterates over an iterator, the C code should
look something like this:

\begin{verbatim}
PyObject *iterator = ...;
PyObject *item;

while (item = PyIter_Next(iter)) {
    /* do something with item */
}
if (PyErr_Occurred()) {
    /* propogate error */
}
else {
    /* continue doing useful work */
}
\end{verbatim}


\chapter{Concrete Objects Layer \label{concrete}}

The functions in this chapter are specific to certain Python object
types.  Passing them an object of the wrong type is not a good idea;
if you receive an object from a Python program and you are not sure
that it has the right type, you must perform a type check first;
for example, to check that an object is a dictionary, use
\cfunction{PyDict_Check()}.  The chapter is structured like the
``family tree'' of Python object types.

\strong{Warning:}
While the functions described in this chapter carefully check the type
of the objects which are passed in, many of them do not check for
\NULL{} being passed instead of a valid object.  Allowing \NULL{} to
be passed in can cause memory access violations and immediate
termination of the interpreter.


\section{Fundamental Objects \label{fundamental}}

This section describes Python type objects and the singleton object 
\code{None}.


\subsection{Type Objects \label{typeObjects}}

\obindex{type}
\begin{ctypedesc}{PyTypeObject}
The C structure of the objects used to describe built-in types.
\end{ctypedesc}

\begin{cvardesc}{PyObject*}{PyType_Type}
This is the type object for type objects; it is the same object as
\code{types.TypeType} in the Python layer.
\withsubitem{(in module types)}{\ttindex{TypeType}}
\end{cvardesc}

\begin{cfuncdesc}{int}{PyType_Check}{PyObject *o}
Returns true is the object \var{o} is a type object.
\end{cfuncdesc}

\begin{cfuncdesc}{int}{PyType_HasFeature}{PyObject *o, int feature}
Returns true if the type object \var{o} sets the feature
\var{feature}.  Type features are denoted by single bit flags.
\end{cfuncdesc}


\subsection{The None Object \label{noneObject}}

\obindex{None@\texttt{None}}
Note that the \ctype{PyTypeObject} for \code{None} is not directly
exposed in the Python/C API.  Since \code{None} is a singleton,
testing for object identity (using \samp{==} in C) is sufficient.
There is no \cfunction{PyNone_Check()} function for the same reason.

\begin{cvardesc}{PyObject*}{Py_None}
The Python \code{None} object, denoting lack of value.  This object has
no methods.
\end{cvardesc}


\section{Sequence Objects \label{sequenceObjects}}

\obindex{sequence}
Generic operations on sequence objects were discussed in the previous 
chapter; this section deals with the specific kinds of sequence 
objects that are intrinsic to the Python language.


\subsection{String Objects \label{stringObjects}}

These functions raise \exception{TypeError} when expecting a string
parameter and are called with a non-string parameter.

\obindex{string}
\begin{ctypedesc}{PyStringObject}
This subtype of \ctype{PyObject} represents a Python string object.
\end{ctypedesc}

\begin{cvardesc}{PyTypeObject}{PyString_Type}
This instance of \ctype{PyTypeObject} represents the Python string
type; it is the same object as \code{types.TypeType} in the Python
layer.\withsubitem{(in module types)}{\ttindex{StringType}}.
\end{cvardesc}

\begin{cfuncdesc}{int}{PyString_Check}{PyObject *o}
Returns true if the object \var{o} is a string object.
\end{cfuncdesc}

\begin{cfuncdesc}{PyObject*}{PyString_FromString}{const char *v}
Returns a new string object with the value \var{v} on success, and
\NULL{} on failure.
\end{cfuncdesc}

\begin{cfuncdesc}{PyObject*}{PyString_FromStringAndSize}{const char *v,
                                                         int len}
Returns a new string object with the value \var{v} and length
\var{len} on success, and \NULL{} on failure.  If \var{v} is \NULL{},
the contents of the string are uninitialized.
\end{cfuncdesc}

\begin{cfuncdesc}{int}{PyString_Size}{PyObject *string}
Returns the length of the string in string object \var{string}.
\end{cfuncdesc}

\begin{cfuncdesc}{int}{PyString_GET_SIZE}{PyObject *string}
Macro form of \cfunction{PyString_Size()} but without error
checking.
\end{cfuncdesc}

\begin{cfuncdesc}{char*}{PyString_AsString}{PyObject *string}
Returns a null-terminated representation of the contents of
\var{string}.  The pointer refers to the internal buffer of
\var{string}, not a copy.  The data must not be modified in any way,
unless the string was just created using
\code{PyString_FromStringAndSize(NULL, \var{size})}.
It must not be deallocated.
\end{cfuncdesc}

\begin{cfuncdesc}{char*}{PyString_AS_STRING}{PyObject *string}
Macro form of \cfunction{PyString_AsString()} but without error
checking.
\end{cfuncdesc}

\begin{cfuncdesc}{int}{PyString_AsStringAndSize}{PyObject *obj,
                                                 char **buffer,
                                                 int *length}
Returns a null-terminated representation of the contents of the object
\var{obj} through the output variables \var{buffer} and \var{length}.

The function accepts both string and Unicode objects as input. For
Unicode objects it returns the default encoded version of the object.
If \var{length} is set to \NULL{}, the resulting buffer may not contain
null characters; if it does, the function returns -1 and a
TypeError is raised.

The buffer refers to an internal string buffer of \var{obj}, not a
copy. The data must not be modified in any way, unless the string was
just created using \code{PyString_FromStringAndSize(NULL,
\var{size})}.  It must not be deallocated.
\end{cfuncdesc}

\begin{cfuncdesc}{void}{PyString_Concat}{PyObject **string,
                                         PyObject *newpart}
Creates a new string object in \var{*string} containing the
contents of \var{newpart} appended to \var{string}; the caller will
own the new reference.  The reference to the old value of \var{string}
will be stolen.  If the new string
cannot be created, the old reference to \var{string} will still be
discarded and the value of \var{*string} will be set to
\NULL{}; the appropriate exception will be set.
\end{cfuncdesc}

\begin{cfuncdesc}{void}{PyString_ConcatAndDel}{PyObject **string,
                                               PyObject *newpart}
Creates a new string object in \var{*string} containing the contents
of \var{newpart} appended to \var{string}.  This version decrements
the reference count of \var{newpart}.
\end{cfuncdesc}

\begin{cfuncdesc}{int}{_PyString_Resize}{PyObject **string, int newsize}
A way to resize a string object even though it is ``immutable''.  
Only use this to build up a brand new string object; don't use this if
the string may already be known in other parts of the code.
\end{cfuncdesc}

\begin{cfuncdesc}{PyObject*}{PyString_Format}{PyObject *format,
                                              PyObject *args}
Returns a new string object from \var{format} and \var{args}.  Analogous
to \code{\var{format} \%\ \var{args}}.  The \var{args} argument must be
a tuple.
\end{cfuncdesc}

\begin{cfuncdesc}{void}{PyString_InternInPlace}{PyObject **string}
Intern the argument \var{*string} in place.  The argument must be the
address of a pointer variable pointing to a Python string object.
If there is an existing interned string that is the same as
\var{*string}, it sets \var{*string} to it (decrementing the reference 
count of the old string object and incrementing the reference count of
the interned string object), otherwise it leaves \var{*string} alone
and interns it (incrementing its reference count).  (Clarification:
even though there is a lot of talk about reference counts, think of
this function as reference-count-neutral; you own the object after
the call if and only if you owned it before the call.)
\end{cfuncdesc}

\begin{cfuncdesc}{PyObject*}{PyString_InternFromString}{const char *v}
A combination of \cfunction{PyString_FromString()} and
\cfunction{PyString_InternInPlace()}, returning either a new string object
that has been interned, or a new (``owned'') reference to an earlier
interned string object with the same value.
\end{cfuncdesc}

\begin{cfuncdesc}{PyObject*}{PyString_Decode}{const char *s,
                                               int size,
                                               const char *encoding,
                                               const char *errors}
Creates an object by decoding \var{size} bytes of the encoded
buffer \var{s} using the codec registered
for \var{encoding}. \var{encoding} and \var{errors} have the same meaning
as the parameters of the same name in the unicode() builtin
function. The codec to be used is looked up using the Python codec
registry. Returns \NULL{} in case an exception was raised by the
codec.
\end{cfuncdesc}

\begin{cfuncdesc}{PyObject*}{PyString_AsDecodedObject}{PyObject *str,
                                               const char *encoding,
                                               const char *errors}
Decodes a string object by passing it to the codec registered
for \var{encoding} and returns the result as Python 
object. \var{encoding} and \var{errors} have the same meaning as the
parameters of the same name in the string .encode() method. The codec
to be used is looked up using the Python codec registry. Returns
\NULL{} in case an exception was raised by the codec.
\end{cfuncdesc}

\begin{cfuncdesc}{PyObject*}{PyString_Encode}{const char *s,
                                               int size,
                                               const char *encoding,
                                               const char *errors}
Encodes the \ctype{char} buffer of the given size by passing it to 
the codec registered for \var{encoding} and returns a Python object. 
\var{encoding} and \var{errors} have the same
meaning as the parameters of the same name in the string .encode()
method. The codec to be used is looked up using the Python codec
registry. Returns \NULL{} in case an exception was raised by the
codec.
\end{cfuncdesc}

\begin{cfuncdesc}{PyObject*}{PyString_AsEncodedObject}{PyObject *str,
                                               const char *encoding,
                                               const char *errors}
Encodes a string object using the codec registered
for \var{encoding} and returns the result as Python 
object. \var{encoding} and \var{errors} have the same meaning as the
parameters of the same name in the string .encode() method. The codec
to be used is looked up using the Python codec registry. Returns
\NULL{} in case an exception was raised by the codec.
\end{cfuncdesc}


\subsection{Unicode Objects \label{unicodeObjects}}
\sectionauthor{Marc-Andre Lemburg}{mal@lemburg.com}

%--- Unicode Type -------------------------------------------------------

These are the basic Unicode object types used for the Unicode
implementation in Python:

\begin{ctypedesc}{Py_UNICODE}
This type represents a 16-bit unsigned storage type which is used by
Python internally as basis for holding Unicode ordinals. On platforms
where \ctype{wchar_t} is available and also has 16-bits,
\ctype{Py_UNICODE} is a typedef alias for \ctype{wchar_t} to enhance
native platform compatibility. On all other platforms,
\ctype{Py_UNICODE} is a typedef alias for \ctype{unsigned short}.
\end{ctypedesc}

\begin{ctypedesc}{PyUnicodeObject}
This subtype of \ctype{PyObject} represents a Python Unicode object.
\end{ctypedesc}

\begin{cvardesc}{PyTypeObject}{PyUnicode_Type}
This instance of \ctype{PyTypeObject} represents the Python Unicode type.
\end{cvardesc}

%--- These are really C macros... is there a macrodesc TeX macro ?

The following APIs are really C macros and can be used to do fast
checks and to access internal read-only data of Unicode objects:

\begin{cfuncdesc}{int}{PyUnicode_Check}{PyObject *o}
Returns true if the object \var{o} is a Unicode object.
\end{cfuncdesc}

\begin{cfuncdesc}{int}{PyUnicode_GET_SIZE}{PyObject *o}
Returns the size of the object.  o has to be a
PyUnicodeObject (not checked).
\end{cfuncdesc}

\begin{cfuncdesc}{int}{PyUnicode_GET_DATA_SIZE}{PyObject *o}
Returns the size of the object's internal buffer in bytes. o has to be
a PyUnicodeObject (not checked).
\end{cfuncdesc}

\begin{cfuncdesc}{Py_UNICODE*}{PyUnicode_AS_UNICODE}{PyObject *o}
Returns a pointer to the internal Py_UNICODE buffer of the object. o
has to be a PyUnicodeObject (not checked).
\end{cfuncdesc}

\begin{cfuncdesc}{const char*}{PyUnicode_AS_DATA}{PyObject *o}
Returns a (const char *) pointer to the internal buffer of the object.
o has to be a PyUnicodeObject (not checked).
\end{cfuncdesc}

% --- Unicode character properties ---------------------------------------

Unicode provides many different character properties. The most often
needed ones are available through these macros which are mapped to C
functions depending on the Python configuration.

\begin{cfuncdesc}{int}{Py_UNICODE_ISSPACE}{Py_UNICODE ch}
Returns 1/0 depending on whether \var{ch} is a whitespace character.
\end{cfuncdesc}

\begin{cfuncdesc}{int}{Py_UNICODE_ISLOWER}{Py_UNICODE ch}
Returns 1/0 depending on whether \var{ch} is a lowercase character.
\end{cfuncdesc}

\begin{cfuncdesc}{int}{Py_UNICODE_ISUPPER}{Py_UNICODE ch}
Returns 1/0 depending on whether \var{ch} is an uppercase character.
\end{cfuncdesc}

\begin{cfuncdesc}{int}{Py_UNICODE_ISTITLE}{Py_UNICODE ch}
Returns 1/0 depending on whether \var{ch} is a titlecase character.
\end{cfuncdesc}

\begin{cfuncdesc}{int}{Py_UNICODE_ISLINEBREAK}{Py_UNICODE ch}
Returns 1/0 depending on whether \var{ch} is a linebreak character.
\end{cfuncdesc}

\begin{cfuncdesc}{int}{Py_UNICODE_ISDECIMAL}{Py_UNICODE ch}
Returns 1/0 depending on whether \var{ch} is a decimal character.
\end{cfuncdesc}

\begin{cfuncdesc}{int}{Py_UNICODE_ISDIGIT}{Py_UNICODE ch}
Returns 1/0 depending on whether \var{ch} is a digit character.
\end{cfuncdesc}

\begin{cfuncdesc}{int}{Py_UNICODE_ISNUMERIC}{Py_UNICODE ch}
Returns 1/0 depending on whether \var{ch} is a numeric character.
\end{cfuncdesc}

\begin{cfuncdesc}{int}{Py_UNICODE_ISALPHA}{Py_UNICODE ch}
Returns 1/0 depending on whether \var{ch} is an alphabetic character.
\end{cfuncdesc}

\begin{cfuncdesc}{int}{Py_UNICODE_ISALNUM}{Py_UNICODE ch}
Returns 1/0 depending on whether \var{ch} is an alphanumeric character.
\end{cfuncdesc}

These APIs can be used for fast direct character conversions:

\begin{cfuncdesc}{Py_UNICODE}{Py_UNICODE_TOLOWER}{Py_UNICODE ch}
Returns the character \var{ch} converted to lower case.
\end{cfuncdesc}

\begin{cfuncdesc}{Py_UNICODE}{Py_UNICODE_TOUPPER}{Py_UNICODE ch}
Returns the character \var{ch} converted to upper case.
\end{cfuncdesc}

\begin{cfuncdesc}{Py_UNICODE}{Py_UNICODE_TOTITLE}{Py_UNICODE ch}
Returns the character \var{ch} converted to title case.
\end{cfuncdesc}

\begin{cfuncdesc}{int}{Py_UNICODE_TODECIMAL}{Py_UNICODE ch}
Returns the character \var{ch} converted to a decimal positive integer.
Returns -1 in case this is not possible. Does not raise exceptions.
\end{cfuncdesc}

\begin{cfuncdesc}{int}{Py_UNICODE_TODIGIT}{Py_UNICODE ch}
Returns the character \var{ch} converted to a single digit integer.
Returns -1 in case this is not possible. Does not raise exceptions.
\end{cfuncdesc}

\begin{cfuncdesc}{double}{Py_UNICODE_TONUMERIC}{Py_UNICODE ch}
Returns the character \var{ch} converted to a (positive) double.
Returns -1.0 in case this is not possible. Does not raise exceptions.
\end{cfuncdesc}

% --- Plain Py_UNICODE ---------------------------------------------------

To create Unicode objects and access their basic sequence properties,
use these APIs:

\begin{cfuncdesc}{PyObject*}{PyUnicode_FromUnicode}{const Py_UNICODE *u,
                                                    int size} 

Create a Unicode Object from the Py_UNICODE buffer \var{u} of the
given size. \var{u} may be \NULL{} which causes the contents to be
undefined. It is the user's responsibility to fill in the needed data.
The buffer is copied into the new object. If the buffer is not \NULL{},
the return value might be a shared object. Therefore, modification of
the resulting Unicode Object is only allowed when \var{u} is \NULL{}.
\end{cfuncdesc}

\begin{cfuncdesc}{Py_UNICODE*}{PyUnicode_AsUnicode}{PyObject *unicode}
Return a read-only pointer to the Unicode object's internal
\ctype{Py_UNICODE} buffer.
\end{cfuncdesc}

\begin{cfuncdesc}{int}{PyUnicode_GetSize}{PyObject *unicode}
Return the length of the Unicode object.
\end{cfuncdesc}

\begin{cfuncdesc}{PyObject*}{PyUnicode_FromEncodedObject}{PyObject *obj,
                                                      const char *encoding,
                                                      const char *errors}

Coerce an encoded object obj to an Unicode object and return a
reference with incremented refcount.

Coercion is done in the following way:
\begin{enumerate}
\item  Unicode objects are passed back as-is with incremented
      refcount. Note: these cannot be decoded; passing a non-NULL
      value for encoding will result in a TypeError.

\item String and other char buffer compatible objects are decoded
      according to the given encoding and using the error handling
      defined by errors. Both can be NULL to have the interface use
      the default values (see the next section for details).

\item All other objects cause an exception.
\end{enumerate}
The API returns NULL in case of an error. The caller is responsible
for decref'ing the returned objects.
\end{cfuncdesc}

\begin{cfuncdesc}{PyObject*}{PyUnicode_FromObject}{PyObject *obj}

Shortcut for PyUnicode_FromEncodedObject(obj, NULL, ``strict'')
which is used throughout the interpreter whenever coercion to
Unicode is needed.
\end{cfuncdesc}

% --- wchar_t support for platforms which support it ---------------------

If the platform supports \ctype{wchar_t} and provides a header file
wchar.h, Python can interface directly to this type using the
following functions. Support is optimized if Python's own
\ctype{Py_UNICODE} type is identical to the system's \ctype{wchar_t}.

\begin{cfuncdesc}{PyObject*}{PyUnicode_FromWideChar}{const wchar_t *w,
                                                     int size}
Create a Unicode Object from the \ctype{whcar_t} buffer \var{w} of the
given size. Returns \NULL{} on failure.
\end{cfuncdesc}

\begin{cfuncdesc}{int}{PyUnicode_AsWideChar}{PyUnicodeObject *unicode,
                                             wchar_t *w,
                                             int size}
Copies the Unicode Object contents into the \ctype{whcar_t} buffer
\var{w}.  At most \var{size} \ctype{whcar_t} characters are copied.
Returns the number of \ctype{whcar_t} characters copied or -1 in case
of an error.
\end{cfuncdesc}


\subsubsection{Builtin Codecs \label{builtinCodecs}}

Python provides a set of builtin codecs which are written in C
for speed. All of these codecs are directly usable via the
following functions.

Many of the following APIs take two arguments encoding and
errors. These parameters encoding and errors have the same semantics
as the ones of the builtin unicode() Unicode object constructor.

Setting encoding to NULL causes the default encoding to be used which
is UTF-8.

Error handling is set by errors which may also be set to NULL meaning
to use the default handling defined for the codec. Default error
handling for all builtin codecs is ``strict'' (ValueErrors are raised).

The codecs all use a similar interface. Only deviation from the
following generic ones are documented for simplicity.

% --- Generic Codecs -----------------------------------------------------

These are the generic codec APIs:

\begin{cfuncdesc}{PyObject*}{PyUnicode_Decode}{const char *s,
                                               int size,
                                               const char *encoding,
                                               const char *errors}
Create a Unicode object by decoding \var{size} bytes of the encoded
string \var{s}. \var{encoding} and \var{errors} have the same meaning
as the parameters of the same name in the unicode() builtin
function. The codec to be used is looked up using the Python codec
registry. Returns \NULL{} in case an exception was raised by the
codec.
\end{cfuncdesc}

\begin{cfuncdesc}{PyObject*}{PyUnicode_Encode}{const Py_UNICODE *s,
                                               int size,
                                               const char *encoding,
                                               const char *errors}
Encodes the \ctype{Py_UNICODE} buffer of the given size and returns a
Python string object. \var{encoding} and \var{errors} have the same
meaning as the parameters of the same name in the Unicode .encode()
method. The codec to be used is looked up using the Python codec
registry. Returns \NULL{} in case an exception was raised by the
codec.
\end{cfuncdesc}

\begin{cfuncdesc}{PyObject*}{PyUnicode_AsEncodedString}{PyObject *unicode,
                                               const char *encoding,
                                               const char *errors}
Encodes a Unicode object and returns the result as Python string
object. \var{encoding} and \var{errors} have the same meaning as the
parameters of the same name in the Unicode .encode() method. The codec
to be used is looked up using the Python codec registry. Returns
\NULL{} in case an exception was raised by the codec.
\end{cfuncdesc}

% --- UTF-8 Codecs -------------------------------------------------------

These are the UTF-8 codec APIs:

\begin{cfuncdesc}{PyObject*}{PyUnicode_DecodeUTF8}{const char *s,
                                               int size,
                                               const char *errors}
Creates a Unicode object by decoding \var{size} bytes of the UTF-8
encoded string \var{s}. Returns \NULL{} in case an exception was
raised by the codec.
\end{cfuncdesc}

\begin{cfuncdesc}{PyObject*}{PyUnicode_EncodeUTF8}{const Py_UNICODE *s,
                                               int size,
                                               const char *errors}
Encodes the \ctype{Py_UNICODE} buffer of the given size using UTF-8
and returns a Python string object.  Returns \NULL{} in case an
exception was raised by the codec.
\end{cfuncdesc}

\begin{cfuncdesc}{PyObject*}{PyUnicode_AsUTF8String}{PyObject *unicode}
Encodes a Unicode objects using UTF-8 and returns the result as Python
string object. Error handling is ``strict''. Returns
\NULL{} in case an exception was raised by the codec.
\end{cfuncdesc}

% --- UTF-16 Codecs ------------------------------------------------------ */

These are the UTF-16 codec APIs:

\begin{cfuncdesc}{PyObject*}{PyUnicode_DecodeUTF16}{const char *s,
                                               int size,
                                               const char *errors,
                                               int *byteorder}
Decodes \var{length} bytes from a UTF-16 encoded buffer string and
returns the corresponding Unicode object.

\var{errors} (if non-NULL) defines the error handling. It defaults
to ``strict''.

If \var{byteorder} is non-\NULL{}, the decoder starts decoding using
the given byte order:

\begin{verbatim}
   *byteorder == -1: little endian
   *byteorder == 0:  native order
   *byteorder == 1:  big endian
\end{verbatim}

and then switches according to all byte order marks (BOM) it finds in
the input data. BOM marks are not copied into the resulting Unicode
string.  After completion, \var{*byteorder} is set to the current byte
order at the end of input data.

If \var{byteorder} is \NULL{}, the codec starts in native order mode.

Returns \NULL{} in case an exception was raised by the codec.
\end{cfuncdesc}

\begin{cfuncdesc}{PyObject*}{PyUnicode_EncodeUTF16}{const Py_UNICODE *s,
                                               int size,
                                               const char *errors,
                                               int byteorder}
Returns a Python string object holding the UTF-16 encoded value of the
Unicode data in \var{s}.

If \var{byteorder} is not \code{0}, output is written according to the
following byte order:

\begin{verbatim}
   byteorder == -1: little endian
   byteorder == 0:  native byte order (writes a BOM mark)
   byteorder == 1:  big endian
\end{verbatim}

If byteorder is \code{0}, the output string will always start with the
Unicode BOM mark (U+FEFF). In the other two modes, no BOM mark is
prepended.

Note that \ctype{Py_UNICODE} data is being interpreted as UTF-16
reduced to UCS-2. This trick makes it possible to add full UTF-16
capabilities at a later point without comprimising the APIs.

Returns \NULL{} in case an exception was raised by the codec.
\end{cfuncdesc}

\begin{cfuncdesc}{PyObject*}{PyUnicode_AsUTF16String}{PyObject *unicode}
Returns a Python string using the UTF-16 encoding in native byte
order. The string always starts with a BOM mark. Error handling is
``strict''. Returns \NULL{} in case an exception was raised by the
codec.
\end{cfuncdesc}

% --- Unicode-Escape Codecs ----------------------------------------------

These are the ``Unicode Esacpe'' codec APIs:

\begin{cfuncdesc}{PyObject*}{PyUnicode_DecodeUnicodeEscape}{const char *s,
                                               int size,
                                               const char *errors}
Creates a Unicode object by decoding \var{size} bytes of the Unicode-Esacpe
encoded string \var{s}. Returns \NULL{} in case an exception was
raised by the codec.
\end{cfuncdesc}

\begin{cfuncdesc}{PyObject*}{PyUnicode_EncodeUnicodeEscape}{const Py_UNICODE *s,
                                               int size,
                                               const char *errors}
Encodes the \ctype{Py_UNICODE} buffer of the given size using Unicode-Escape
and returns a Python string object.  Returns \NULL{} in case an
exception was raised by the codec.
\end{cfuncdesc}

\begin{cfuncdesc}{PyObject*}{PyUnicode_AsUnicodeEscapeString}{PyObject *unicode}
Encodes a Unicode objects using Unicode-Escape and returns the result
as Python string object. Error handling is ``strict''. Returns
\NULL{} in case an exception was raised by the codec.
\end{cfuncdesc}

% --- Raw-Unicode-Escape Codecs ------------------------------------------

These are the ``Raw Unicode Esacpe'' codec APIs:

\begin{cfuncdesc}{PyObject*}{PyUnicode_DecodeRawUnicodeEscape}{const char *s,
                                               int size,
                                               const char *errors}
Creates a Unicode object by decoding \var{size} bytes of the Raw-Unicode-Esacpe
encoded string \var{s}. Returns \NULL{} in case an exception was
raised by the codec.
\end{cfuncdesc}

\begin{cfuncdesc}{PyObject*}{PyUnicode_EncodeRawUnicodeEscape}{const Py_UNICODE *s,
                                               int size,
                                               const char *errors}
Encodes the \ctype{Py_UNICODE} buffer of the given size using Raw-Unicode-Escape
and returns a Python string object.  Returns \NULL{} in case an
exception was raised by the codec.
\end{cfuncdesc}

\begin{cfuncdesc}{PyObject*}{PyUnicode_AsRawUnicodeEscapeString}{PyObject *unicode}
Encodes a Unicode objects using Raw-Unicode-Escape and returns the result
as Python string object. Error handling is ``strict''. Returns
\NULL{} in case an exception was raised by the codec.
\end{cfuncdesc}

% --- Latin-1 Codecs ----------------------------------------------------- 

These are the Latin-1 codec APIs:

Latin-1 corresponds to the first 256 Unicode ordinals and only these
are accepted by the codecs during encoding.

\begin{cfuncdesc}{PyObject*}{PyUnicode_DecodeLatin1}{const char *s,
                                                     int size,
                                                     const char *errors}
Creates a Unicode object by decoding \var{size} bytes of the Latin-1
encoded string \var{s}. Returns \NULL{} in case an exception was
raised by the codec.
\end{cfuncdesc}

\begin{cfuncdesc}{PyObject*}{PyUnicode_EncodeLatin1}{const Py_UNICODE *s,
                                                     int size,
                                                     const char *errors}
Encodes the \ctype{Py_UNICODE} buffer of the given size using Latin-1
and returns a Python string object.  Returns \NULL{} in case an
exception was raised by the codec.
\end{cfuncdesc}

\begin{cfuncdesc}{PyObject*}{PyUnicode_AsLatin1String}{PyObject *unicode}
Encodes a Unicode objects using Latin-1 and returns the result as
Python string object. Error handling is ``strict''. Returns
\NULL{} in case an exception was raised by the codec.
\end{cfuncdesc}

% --- ASCII Codecs ------------------------------------------------------- 

These are the \ASCII{} codec APIs.  Only 7-bit \ASCII{} data is
accepted. All other codes generate errors.

\begin{cfuncdesc}{PyObject*}{PyUnicode_DecodeASCII}{const char *s,
                                                    int size,
                                                    const char *errors}
Creates a Unicode object by decoding \var{size} bytes of the
\ASCII{} encoded string \var{s}. Returns \NULL{} in case an exception
was raised by the codec.
\end{cfuncdesc}

\begin{cfuncdesc}{PyObject*}{PyUnicode_EncodeASCII}{const Py_UNICODE *s,
                                                    int size,
                                                    const char *errors}
Encodes the \ctype{Py_UNICODE} buffer of the given size using
\ASCII{} and returns a Python string object.  Returns \NULL{} in case
an exception was raised by the codec.
\end{cfuncdesc}

\begin{cfuncdesc}{PyObject*}{PyUnicode_AsASCIIString}{PyObject *unicode}
Encodes a Unicode objects using \ASCII{} and returns the result as Python
string object. Error handling is ``strict''. Returns
\NULL{} in case an exception was raised by the codec.
\end{cfuncdesc}

% --- Character Map Codecs ----------------------------------------------- 

These are the mapping codec APIs:

This codec is special in that it can be used to implement many
different codecs (and this is in fact what was done to obtain most of
the standard codecs included in the \module{encodings} package). The
codec uses mapping to encode and decode characters.

Decoding mappings must map single string characters to single Unicode
characters, integers (which are then interpreted as Unicode ordinals)
or None (meaning "undefined mapping" and causing an error). 

Encoding mappings must map single Unicode characters to single string
characters, integers (which are then interpreted as Latin-1 ordinals)
or None (meaning "undefined mapping" and causing an error).

The mapping objects provided must only support the __getitem__ mapping
interface.

If a character lookup fails with a LookupError, the character is
copied as-is meaning that its ordinal value will be interpreted as
Unicode or Latin-1 ordinal resp. Because of this, mappings only need
to contain those mappings which map characters to different code
points.

\begin{cfuncdesc}{PyObject*}{PyUnicode_DecodeCharmap}{const char *s,
                                               int size,
                                               PyObject *mapping,
                                               const char *errors}
Creates a Unicode object by decoding \var{size} bytes of the encoded
string \var{s} using the given \var{mapping} object.  Returns \NULL{}
in case an exception was raised by the codec.
\end{cfuncdesc}

\begin{cfuncdesc}{PyObject*}{PyUnicode_EncodeCharmap}{const Py_UNICODE *s,
                                               int size,
                                               PyObject *mapping,
                                               const char *errors}
Encodes the \ctype{Py_UNICODE} buffer of the given size using the
given \var{mapping} object and returns a Python string object.
Returns \NULL{} in case an exception was raised by the codec.
\end{cfuncdesc}

\begin{cfuncdesc}{PyObject*}{PyUnicode_AsCharmapString}{PyObject *unicode,
                                                        PyObject *mapping}
Encodes a Unicode objects using the given \var{mapping} object and
returns the result as Python string object. Error handling is
``strict''. Returns \NULL{} in case an exception was raised by the
codec.
\end{cfuncdesc}

The following codec API is special in that maps Unicode to Unicode.

\begin{cfuncdesc}{PyObject*}{PyUnicode_TranslateCharmap}{const Py_UNICODE *s,
                                               int size,
                                               PyObject *table,
                                               const char *errors}
Translates a \ctype{Py_UNICODE} buffer of the given length by applying
a character mapping \var{table} to it and returns the resulting
Unicode object.  Returns \NULL{} when an exception was raised by the
codec.

The \var{mapping} table must map Unicode ordinal integers to Unicode
ordinal integers or None (causing deletion of the character).

Mapping tables must only provide the __getitem__ interface,
e.g. dictionaries or sequences. Unmapped character ordinals (ones
which cause a LookupError) are left untouched and are copied as-is.
\end{cfuncdesc}

% --- MBCS codecs for Windows --------------------------------------------

These are the MBCS codec APIs. They are currently only available on
Windows and use the Win32 MBCS converters to implement the
conversions.  Note that MBCS (or DBCS) is a class of encodings, not
just one.  The target encoding is defined by the user settings on the
machine running the codec.

\begin{cfuncdesc}{PyObject*}{PyUnicode_DecodeMBCS}{const char *s,
                                               int size,
                                               const char *errors}
Creates a Unicode object by decoding \var{size} bytes of the MBCS
encoded string \var{s}.  Returns \NULL{} in case an exception was
raised by the codec.
\end{cfuncdesc}

\begin{cfuncdesc}{PyObject*}{PyUnicode_EncodeMBCS}{const Py_UNICODE *s,
                                               int size,
                                               const char *errors}
Encodes the \ctype{Py_UNICODE} buffer of the given size using MBCS
and returns a Python string object.  Returns \NULL{} in case an
exception was raised by the codec.
\end{cfuncdesc}

\begin{cfuncdesc}{PyObject*}{PyUnicode_AsMBCSString}{PyObject *unicode}
Encodes a Unicode objects using MBCS and returns the result as Python
string object.  Error handling is ``strict''.  Returns \NULL{} in case
an exception was raised by the codec.
\end{cfuncdesc}

% --- Methods & Slots ----------------------------------------------------

\subsubsection{Methods and Slot Functions \label{unicodeMethodsAndSlots}}

The following APIs are capable of handling Unicode objects and strings
on input (we refer to them as strings in the descriptions) and return
Unicode objects or integers as apporpriate.

They all return \NULL{} or -1 in case an exception occurrs.

\begin{cfuncdesc}{PyObject*}{PyUnicode_Concat}{PyObject *left,
                                               PyObject *right}
Concat two strings giving a new Unicode string.
\end{cfuncdesc}

\begin{cfuncdesc}{PyObject*}{PyUnicode_Split}{PyObject *s,
                                              PyObject *sep,
                                              int maxsplit}
Split a string giving a list of Unicode strings.

If sep is NULL, splitting will be done at all whitespace
substrings. Otherwise, splits occur at the given separator.

At most maxsplit splits will be done. If negative, no limit is set.

Separators are not included in the resulting list.
\end{cfuncdesc}

\begin{cfuncdesc}{PyObject*}{PyUnicode_Splitlines}{PyObject *s,
                                                   int maxsplit}
Split a Unicode string at line breaks, returning a list of Unicode
strings.  CRLF is considered to be one line break.  The Line break
characters are not included in the resulting strings.
\end{cfuncdesc}

\begin{cfuncdesc}{PyObject*}{PyUnicode_Translate}{PyObject *str,
                                                  PyObject *table,
                                                  const char *errors}
Translate a string by applying a character mapping table to it and
return the resulting Unicode object.

The mapping table must map Unicode ordinal integers to Unicode ordinal
integers or None (causing deletion of the character).

Mapping tables must only provide the __getitem__ interface,
e.g. dictionaries or sequences. Unmapped character ordinals (ones
which cause a LookupError) are left untouched and are copied as-is.

\var{errors} has the usual meaning for codecs. It may be \NULL{}
which indicates to use the default error handling.
\end{cfuncdesc}

\begin{cfuncdesc}{PyObject*}{PyUnicode_Join}{PyObject *separator,
                                             PyObject *seq}
Join a sequence of strings using the given separator and return
the resulting Unicode string.
\end{cfuncdesc}

\begin{cfuncdesc}{PyObject*}{PyUnicode_Tailmatch}{PyObject *str,
                                                  PyObject *substr,
                                                  int start,
                                                  int end,
                                                  int direction}
Return 1 if \var{substr} matches \var{str}[\var{start}:\var{end}] at
the given tail end (\var{direction} == -1 means to do a prefix match,
\var{direction} == 1 a suffix match), 0 otherwise.
\end{cfuncdesc}

\begin{cfuncdesc}{PyObject*}{PyUnicode_Find}{PyObject *str,
                                                  PyObject *substr,
                                                  int start,
                                                  int end,
                                                  int direction}
Return the first position of \var{substr} in
\var{str}[\var{start}:\var{end}] using the given \var{direction}
(\var{direction} == 1 means to do a forward search,
\var{direction} == -1 a backward search), 0 otherwise.
\end{cfuncdesc}

\begin{cfuncdesc}{PyObject*}{PyUnicode_Count}{PyObject *str,
                                                  PyObject *substr,
                                                  int start,
                                                  int end}
Count the number of occurrences of \var{substr} in
\var{str}[\var{start}:\var{end}]
\end{cfuncdesc}

\begin{cfuncdesc}{PyObject*}{PyUnicode_Replace}{PyObject *str,
                                                PyObject *substr,
                                                PyObject *replstr,
                                                int maxcount}
Replace at most \var{maxcount} occurrences of \var{substr} in
\var{str} with \var{replstr} and return the resulting Unicode object.
\var{maxcount} == -1 means: replace all occurrences.
\end{cfuncdesc}

\begin{cfuncdesc}{int}{PyUnicode_Compare}{PyObject *left, PyObject *right}
Compare two strings and return -1, 0, 1 for less than, equal,
greater than resp.
\end{cfuncdesc}

\begin{cfuncdesc}{PyObject*}{PyUnicode_Format}{PyObject *format,
                                              PyObject *args}
Returns a new string object from \var{format} and \var{args}; this is
analogous to \code{\var{format} \%\ \var{args}}.  The
\var{args} argument must be a tuple.
\end{cfuncdesc}

\begin{cfuncdesc}{int}{PyUnicode_Contains}{PyObject *container,
                                           PyObject *element}
Checks whether \var{element} is contained in \var{container} and
returns true or false accordingly.

\var{element} has to coerce to a one element Unicode string. \code{-1} is
returned in case of an error.
\end{cfuncdesc}


\subsection{Buffer Objects \label{bufferObjects}}
\sectionauthor{Greg Stein}{gstein@lyra.org}

\obindex{buffer}
Python objects implemented in C can export a group of functions called
the ``buffer\index{buffer interface} interface.''  These functions can
be used by an object to expose its data in a raw, byte-oriented
format. Clients of the object can use the buffer interface to access
the object data directly, without needing to copy it first.

Two examples of objects that support 
the buffer interface are strings and arrays. The string object exposes 
the character contents in the buffer interface's byte-oriented
form. An array can also expose its contents, but it should be noted
that array elements may be multi-byte values.

An example user of the buffer interface is the file object's
\method{write()} method. Any object that can export a series of bytes
through the buffer interface can be written to a file. There are a
number of format codes to \cfunction{PyArgs_ParseTuple()} that operate 
against an object's buffer interface, returning data from the target
object.

More information on the buffer interface is provided in the section
``Buffer Object Structures'' (section \ref{buffer-structs}), under
the description for \ctype{PyBufferProcs}\ttindex{PyBufferProcs}.

A ``buffer object'' is defined in the \file{bufferobject.h} header
(included by \file{Python.h}). These objects look very similar to
string objects at the Python programming level: they support slicing,
indexing, concatenation, and some other standard string
operations. However, their data can come from one of two sources: from
a block of memory, or from another object which exports the buffer
interface.

Buffer objects are useful as a way to expose the data from another
object's buffer interface to the Python programmer. They can also be
used as a zero-copy slicing mechanism. Using their ability to
reference a block of memory, it is possible to expose any data to the
Python programmer quite easily. The memory could be a large, constant
array in a C extension, it could be a raw block of memory for
manipulation before passing to an operating system library, or it
could be used to pass around structured data in its native, in-memory
format.

\begin{ctypedesc}{PyBufferObject}
This subtype of \ctype{PyObject} represents a buffer object.
\end{ctypedesc}

\begin{cvardesc}{PyTypeObject}{PyBuffer_Type}
The instance of \ctype{PyTypeObject} which represents the Python
buffer type; it is the same object as \code{types.BufferType} in the
Python layer.\withsubitem{(in module types)}{\ttindex{BufferType}}.
\end{cvardesc}

\begin{cvardesc}{int}{Py_END_OF_BUFFER}
This constant may be passed as the \var{size} parameter to
\cfunction{PyBuffer_FromObject()} or
\cfunction{PyBuffer_FromReadWriteObject()}. It indicates that the new
\ctype{PyBufferObject} should refer to \var{base} object from the
specified \var{offset} to the end of its exported buffer. Using this
enables the caller to avoid querying the \var{base} object for its
length.
\end{cvardesc}

\begin{cfuncdesc}{int}{PyBuffer_Check}{PyObject *p}
Return true if the argument has type \cdata{PyBuffer_Type}.
\end{cfuncdesc}

\begin{cfuncdesc}{PyObject*}{PyBuffer_FromObject}{PyObject *base,
                                                  int offset, int size}
Return a new read-only buffer object.  This raises
\exception{TypeError} if \var{base} doesn't support the read-only
buffer protocol or doesn't provide exactly one buffer segment, or it
raises \exception{ValueError} if \var{offset} is less than zero. The
buffer will hold a reference to the \var{base} object, and the
buffer's contents will refer to the \var{base} object's buffer
interface, starting as position \var{offset} and extending for
\var{size} bytes. If \var{size} is \constant{Py_END_OF_BUFFER}, then
the new buffer's contents extend to the length of the
\var{base} object's exported buffer data.
\end{cfuncdesc}

\begin{cfuncdesc}{PyObject*}{PyBuffer_FromReadWriteObject}{PyObject *base,
                                                           int offset,
                                                           int size}
Return a new writable buffer object.  Parameters and exceptions are
similar to those for \cfunction{PyBuffer_FromObject()}.
If the \var{base} object does not export the writeable buffer
protocol, then \exception{TypeError} is raised.
\end{cfuncdesc}

\begin{cfuncdesc}{PyObject*}{PyBuffer_FromMemory}{void *ptr, int size}
Return a new read-only buffer object that reads from a specified
location in memory, with a specified size.
The caller is responsible for ensuring that the memory buffer, passed
in as \var{ptr}, is not deallocated while the returned buffer object
exists.  Raises \exception{ValueError} if \var{size} is less than
zero.  Note that \constant{Py_END_OF_BUFFER} may \emph{not} be passed
for the \var{size} parameter; \exception{ValueError} will be raised in 
that case.
\end{cfuncdesc}

\begin{cfuncdesc}{PyObject*}{PyBuffer_FromReadWriteMemory}{void *ptr, int size}
Similar to \cfunction{PyBuffer_FromMemory()}, but the returned buffer
is writable.
\end{cfuncdesc}

\begin{cfuncdesc}{PyObject*}{PyBuffer_New}{int size}
Returns a new writable buffer object that maintains its own memory
buffer of \var{size} bytes.  \exception{ValueError} is returned if
\var{size} is not zero or positive.
\end{cfuncdesc}


\subsection{Tuple Objects \label{tupleObjects}}

\obindex{tuple}
\begin{ctypedesc}{PyTupleObject}
This subtype of \ctype{PyObject} represents a Python tuple object.
\end{ctypedesc}

\begin{cvardesc}{PyTypeObject}{PyTuple_Type}
This instance of \ctype{PyTypeObject} represents the Python tuple
type; it is the same object as \code{types.TupleType} in the Python
layer.\withsubitem{(in module types)}{\ttindex{TupleType}}.
\end{cvardesc}

\begin{cfuncdesc}{int}{PyTuple_Check}{PyObject *p}
Return true if the argument is a tuple object.
\end{cfuncdesc}

\begin{cfuncdesc}{PyObject*}{PyTuple_New}{int len}
Return a new tuple object of size \var{len}, or \NULL{} on failure.
\end{cfuncdesc}

\begin{cfuncdesc}{int}{PyTuple_Size}{PyObject *p}
Takes a pointer to a tuple object, and returns the size
of that tuple.
\end{cfuncdesc}

\begin{cfuncdesc}{PyObject*}{PyTuple_GetItem}{PyObject *p, int pos}
Returns the object at position \var{pos} in the tuple pointed
to by \var{p}.  If \var{pos} is out of bounds, returns \NULL{} and
sets an \exception{IndexError} exception.
\end{cfuncdesc}

\begin{cfuncdesc}{PyObject*}{PyTuple_GET_ITEM}{PyObject *p, int pos}
Does the same, but does no checking of its arguments.
\end{cfuncdesc}

\begin{cfuncdesc}{PyObject*}{PyTuple_GetSlice}{PyObject *p,
                                               int low, int high}
Takes a slice of the tuple pointed to by \var{p} from
\var{low} to \var{high} and returns it as a new tuple.
\end{cfuncdesc}

\begin{cfuncdesc}{int}{PyTuple_SetItem}{PyObject *p,
                                        int pos, PyObject *o}
Inserts a reference to object \var{o} at position \var{pos} of
the tuple pointed to by \var{p}. It returns \code{0} on success.
\strong{Note:}  This function ``steals'' a reference to \var{o}.
\end{cfuncdesc}

\begin{cfuncdesc}{void}{PyTuple_SET_ITEM}{PyObject *p,
                                          int pos, PyObject *o}
Does the same, but does no error checking, and
should \emph{only} be used to fill in brand new tuples.
\strong{Note:}  This function ``steals'' a reference to \var{o}.
\end{cfuncdesc}

\begin{cfuncdesc}{int}{_PyTuple_Resize}{PyObject **p,
                                        int newsize, int last_is_sticky}
Can be used to resize a tuple.  \var{newsize} will be the new length
of the tuple.  Because tuples are \emph{supposed} to be immutable,
this should only be used if there is only one reference to the object.
Do \emph{not} use this if the tuple may already be known to some other
part of the code.  The tuple will always grow or shrink at the end.  The
\var{last_is_sticky} flag is not used and should always be false.  Think
of this as destroying the old tuple and creating a new one, only more
efficiently.  Returns \code{0} on success and \code{-1} on failure (in
which case a \exception{MemoryError} or \exception{SystemError} will be
raised).
\end{cfuncdesc}


\subsection{List Objects \label{listObjects}}

\obindex{list}
\begin{ctypedesc}{PyListObject}
This subtype of \ctype{PyObject} represents a Python list object.
\end{ctypedesc}

\begin{cvardesc}{PyTypeObject}{PyList_Type}
This instance of \ctype{PyTypeObject} represents the Python list
type.  This is the same object as \code{types.ListType}.
\withsubitem{(in module types)}{\ttindex{ListType}}
\end{cvardesc}

\begin{cfuncdesc}{int}{PyList_Check}{PyObject *p}
Returns true if its argument is a \ctype{PyListObject}.
\end{cfuncdesc}

\begin{cfuncdesc}{PyObject*}{PyList_New}{int len}
Returns a new list of length \var{len} on success, or \NULL{} on
failure.
\end{cfuncdesc}

\begin{cfuncdesc}{int}{PyList_Size}{PyObject *list}
Returns the length of the list object in \var{list}; this is
equivalent to \samp{len(\var{list})} on a list object.
\bifuncindex{len}
\end{cfuncdesc}

\begin{cfuncdesc}{int}{PyList_GET_SIZE}{PyObject *list}
Macro form of \cfunction{PyList_Size()} without error checking.
\end{cfuncdesc}

\begin{cfuncdesc}{PyObject*}{PyList_GetItem}{PyObject *list, int index}
Returns the object at position \var{pos} in the list pointed
to by \var{p}.  If \var{pos} is out of bounds, returns \NULL{} and
sets an \exception{IndexError} exception.
\end{cfuncdesc}

\begin{cfuncdesc}{PyObject*}{PyList_GET_ITEM}{PyObject *list, int i}
Macro form of \cfunction{PyList_GetItem()} without error checking.
\end{cfuncdesc}

\begin{cfuncdesc}{int}{PyList_SetItem}{PyObject *list, int index,
                                       PyObject *item}
Sets the item at index \var{index} in list to \var{item}.
\strong{Note:}  This function ``steals'' a reference to \var{item}.
\end{cfuncdesc}

\begin{cfuncdesc}{PyObject*}{PyList_SET_ITEM}{PyObject *list, int i,
                                              PyObject *o}
Macro form of \cfunction{PyList_SetItem()} without error checking.
\strong{Note:}  This function ``steals'' a reference to \var{item}.
\end{cfuncdesc}

\begin{cfuncdesc}{int}{PyList_Insert}{PyObject *list, int index,
                                      PyObject *item}
Inserts the item \var{item} into list \var{list} in front of index
\var{index}.  Returns \code{0} if successful; returns \code{-1} and
raises an exception if unsuccessful.  Analogous to
\code{\var{list}.insert(\var{index}, \var{item})}.
\end{cfuncdesc}

\begin{cfuncdesc}{int}{PyList_Append}{PyObject *list, PyObject *item}
Appends the object \var{item} at the end of list \var{list}.  Returns
\code{0} if successful; returns \code{-1} and sets an exception if
unsuccessful.  Analogous to \code{\var{list}.append(\var{item})}.
\end{cfuncdesc}

\begin{cfuncdesc}{PyObject*}{PyList_GetSlice}{PyObject *list,
                                              int low, int high}
Returns a list of the objects in \var{list} containing the objects 
\emph{between} \var{low} and \var{high}.  Returns NULL and sets an
exception if unsuccessful.
Analogous to \code{\var{list}[\var{low}:\var{high}]}.
\end{cfuncdesc}

\begin{cfuncdesc}{int}{PyList_SetSlice}{PyObject *list,
                                        int low, int high,
                                        PyObject *itemlist}
Sets the slice of \var{list} between \var{low} and \var{high} to the
contents of \var{itemlist}.  Analogous to
\code{\var{list}[\var{low}:\var{high}] = \var{itemlist}}.  Returns
\code{0} on success, \code{-1} on failure.
\end{cfuncdesc}

\begin{cfuncdesc}{int}{PyList_Sort}{PyObject *list}
Sorts the items of \var{list} in place.  Returns \code{0} on success,
\code{-1} on failure.  This is equivalent to
\samp{\var{list}.sort()}.
\end{cfuncdesc}

\begin{cfuncdesc}{int}{PyList_Reverse}{PyObject *list}
Reverses the items of \var{list} in place.  Returns \code{0} on
success, \code{-1} on failure.  This is the equivalent of
\samp{\var{list}.reverse()}.
\end{cfuncdesc}

\begin{cfuncdesc}{PyObject*}{PyList_AsTuple}{PyObject *list}
Returns a new tuple object containing the contents of \var{list};
equivalent to \samp{tuple(\var{list})}.\bifuncindex{tuple}
\end{cfuncdesc}


\section{Mapping Objects \label{mapObjects}}

\obindex{mapping}


\subsection{Dictionary Objects \label{dictObjects}}

\obindex{dictionary}
\begin{ctypedesc}{PyDictObject}
This subtype of \ctype{PyObject} represents a Python dictionary object.
\end{ctypedesc}

\begin{cvardesc}{PyTypeObject}{PyDict_Type}
This instance of \ctype{PyTypeObject} represents the Python dictionary 
type.  This is exposed to Python programs as \code{types.DictType} and 
\code{types.DictionaryType}.
\withsubitem{(in module types)}{\ttindex{DictType}\ttindex{DictionaryType}}
\end{cvardesc}

\begin{cfuncdesc}{int}{PyDict_Check}{PyObject *p}
Returns true if its argument is a \ctype{PyDictObject}.
\end{cfuncdesc}

\begin{cfuncdesc}{PyObject*}{PyDict_New}{}
Returns a new empty dictionary, or \NULL{} on failure.
\end{cfuncdesc}

\begin{cfuncdesc}{void}{PyDict_Clear}{PyObject *p}
Empties an existing dictionary of all key-value pairs.
\end{cfuncdesc}

\begin{cfuncdesc}{PyObject*}{PyDict_Copy}{PyObject *p}
Returns a new dictionary that contains the same key-value pairs as p.
Empties an existing dictionary of all key-value pairs.
\end{cfuncdesc}

\begin{cfuncdesc}{int}{PyDict_SetItem}{PyObject *p, PyObject *key,
                                       PyObject *val}
Inserts \var{value} into the dictionary with a key of \var{key}.
\var{key} must be hashable; if it isn't, \exception{TypeError} will be 
raised.
\end{cfuncdesc}

\begin{cfuncdesc}{int}{PyDict_SetItemString}{PyObject *p,
            char *key,
            PyObject *val}
Inserts \var{value} into the dictionary using \var{key}
as a key. \var{key} should be a \ctype{char*}.  The key object is
created using \code{PyString_FromString(\var{key})}.
\ttindex{PyString_FromString()}
\end{cfuncdesc}

\begin{cfuncdesc}{int}{PyDict_DelItem}{PyObject *p, PyObject *key}
Removes the entry in dictionary \var{p} with key \var{key}.
\var{key} must be hashable; if it isn't, \exception{TypeError} is
raised.
\end{cfuncdesc}

\begin{cfuncdesc}{int}{PyDict_DelItemString}{PyObject *p, char *key}
Removes the entry in dictionary \var{p} which has a key
specified by the string \var{key}.
\end{cfuncdesc}

\begin{cfuncdesc}{PyObject*}{PyDict_GetItem}{PyObject *p, PyObject *key}
Returns the object from dictionary \var{p} which has a key
\var{key}.  Returns \NULL{} if the key \var{key} is not present, but
\emph{without} setting an exception.
\end{cfuncdesc}

\begin{cfuncdesc}{PyObject*}{PyDict_GetItemString}{PyObject *p, char *key}
This is the same as \cfunction{PyDict_GetItem()}, but \var{key} is
specified as a \ctype{char*}, rather than a \ctype{PyObject*}.
\end{cfuncdesc}

\begin{cfuncdesc}{PyObject*}{PyDict_Items}{PyObject *p}
Returns a \ctype{PyListObject} containing all the items 
from the dictionary, as in the dictinoary method \method{items()} (see
the \citetitle[../lib/lib.html]{Python Library Reference}).
\end{cfuncdesc}

\begin{cfuncdesc}{PyObject*}{PyDict_Keys}{PyObject *p}
Returns a \ctype{PyListObject} containing all the keys 
from the dictionary, as in the dictionary method \method{keys()} (see the
\citetitle[../lib/lib.html]{Python Library Reference}).
\end{cfuncdesc}

\begin{cfuncdesc}{PyObject*}{PyDict_Values}{PyObject *p}
Returns a \ctype{PyListObject} containing all the values 
from the dictionary \var{p}, as in the dictionary method
\method{values()} (see the \citetitle[../lib/lib.html]{Python Library
Reference}).
\end{cfuncdesc}

\begin{cfuncdesc}{int}{PyDict_Size}{PyObject *p}
Returns the number of items in the dictionary.  This is equivalent to
\samp{len(\var{p})} on a dictionary.\bifuncindex{len}
\end{cfuncdesc}

\begin{cfuncdesc}{int}{PyDict_Next}{PyObject *p, int *ppos,
                                    PyObject **pkey, PyObject **pvalue}
Iterate over all key-value pairs in the dictionary \var{p}.  The
\ctype{int} referred to by \var{ppos} must be initialized to \code{0}
prior to the first call to this function to start the iteration; the
function returns true for each pair in the dictionary, and false once
all pairs have been reported.  The parameters \var{pkey} and
\var{pvalue} should either point to \ctype{PyObject*} variables that
will be filled in with each key and value, respectively, or may be
\NULL.

For example:

\begin{verbatim}
PyObject *key, *value;
int pos = 0;

while (PyDict_Next(self->dict, &pos, &key, &value)) {
    /* do something interesting with the values... */
    ...
}
\end{verbatim}

The dictionary \var{p} should not be mutated during iteration.  It is
safe (since Python 2.1) to modify the values of the keys as you
iterate over the dictionary, for example:

\begin{verbatim}
PyObject *key, *value;
int pos = 0;

while (PyDict_Next(self->dict, &pos, &key, &value)) {
    int i = PyInt_AS_LONG(value) + 1;
    PyObject *o = PyInt_FromLong(i);
    if (o == NULL)
        return -1;
    if (PyDict_SetItem(self->dict, key, o) < 0) {
        Py_DECREF(o);
        return -1;
    }
    Py_DECREF(o);
}
\end{verbatim}
\end{cfuncdesc}


\section{Numeric Objects \label{numericObjects}}

\obindex{numeric}


\subsection{Plain Integer Objects \label{intObjects}}

\obindex{integer}
\begin{ctypedesc}{PyIntObject}
This subtype of \ctype{PyObject} represents a Python integer object.
\end{ctypedesc}

\begin{cvardesc}{PyTypeObject}{PyInt_Type}
This instance of \ctype{PyTypeObject} represents the Python plain 
integer type.  This is the same object as \code{types.IntType}.
\withsubitem{(in modules types)}{\ttindex{IntType}}
\end{cvardesc}

\begin{cfuncdesc}{int}{PyInt_Check}{PyObject* o}
Returns true if \var{o} is of type \cdata{PyInt_Type}.
\end{cfuncdesc}

\begin{cfuncdesc}{PyObject*}{PyInt_FromLong}{long ival}
Creates a new integer object with a value of \var{ival}.

The current implementation keeps an array of integer objects for all
integers between \code{-1} and \code{100}, when you create an int in
that range you actually just get back a reference to the existing
object. So it should be possible to change the value of \code{1}. I
suspect the behaviour of Python in this case is undefined. :-)
\end{cfuncdesc}

\begin{cfuncdesc}{long}{PyInt_AsLong}{PyObject *io}
Will first attempt to cast the object to a \ctype{PyIntObject}, if
it is not already one, and then return its value.
\end{cfuncdesc}

\begin{cfuncdesc}{long}{PyInt_AS_LONG}{PyObject *io}
Returns the value of the object \var{io}.  No error checking is
performed.
\end{cfuncdesc}

\begin{cfuncdesc}{long}{PyInt_GetMax}{}
Returns the system's idea of the largest integer it can handle
(\constant{LONG_MAX}\ttindex{LONG_MAX}, as defined in the system
header files).
\end{cfuncdesc}


\subsection{Long Integer Objects \label{longObjects}}

\obindex{long integer}
\begin{ctypedesc}{PyLongObject}
This subtype of \ctype{PyObject} represents a Python long integer
object.
\end{ctypedesc}

\begin{cvardesc}{PyTypeObject}{PyLong_Type}
This instance of \ctype{PyTypeObject} represents the Python long
integer type.  This is the same object as \code{types.LongType}.
\withsubitem{(in modules types)}{\ttindex{LongType}}
\end{cvardesc}

\begin{cfuncdesc}{int}{PyLong_Check}{PyObject *p}
Returns true if its argument is a \ctype{PyLongObject}.
\end{cfuncdesc}

\begin{cfuncdesc}{PyObject*}{PyLong_FromLong}{long v}
Returns a new \ctype{PyLongObject} object from \var{v}, or \NULL{} on
failure.
\end{cfuncdesc}

\begin{cfuncdesc}{PyObject*}{PyLong_FromUnsignedLong}{unsigned long v}
Returns a new \ctype{PyLongObject} object from a C \ctype{unsigned
long}, or \NULL{} on failure.
\end{cfuncdesc}

\begin{cfuncdesc}{PyObject*}{PyLong_FromDouble}{double v}
Returns a new \ctype{PyLongObject} object from the integer part of
\var{v}, or \NULL{} on failure.
\end{cfuncdesc}

\begin{cfuncdesc}{long}{PyLong_AsLong}{PyObject *pylong}
Returns a C \ctype{long} representation of the contents of
\var{pylong}.  If \var{pylong} is greater than
\constant{LONG_MAX}\ttindex{LONG_MAX}, an \exception{OverflowError} is
raised.\withsubitem{(built-in exception)}{OverflowError}
\end{cfuncdesc}

\begin{cfuncdesc}{unsigned long}{PyLong_AsUnsignedLong}{PyObject *pylong}
Returns a C \ctype{unsigned long} representation of the contents of 
\var{pylong}.  If \var{pylong} is greater than
\constant{ULONG_MAX}\ttindex{ULONG_MAX}, an \exception{OverflowError}
is raised.\withsubitem{(built-in exception)}{OverflowError}
\end{cfuncdesc}

\begin{cfuncdesc}{double}{PyLong_AsDouble}{PyObject *pylong}
Returns a C \ctype{double} representation of the contents of \var{pylong}.
\end{cfuncdesc}

\begin{cfuncdesc}{PyObject*}{PyLong_FromString}{char *str, char **pend,
                                                int base}
Return a new \ctype{PyLongObject} based on the string value in
\var{str}, which is interpreted according to the radix in \var{base}.
If \var{pend} is non-\NULL, \code{*\var{pend}} will point to the first 
character in \var{str} which follows the representation of the
number.  If \var{base} is \code{0}, the radix will be determined base
on the leading characters of \var{str}: if \var{str} starts with
\code{'0x'} or \code{'0X'}, radix 16 will be used; if \var{str} starts 
with \code{'0'}, radix 8 will be used; otherwise radix 10 will be
used.  If \var{base} is not \code{0}, it must be between \code{2} and
\code{36}, inclusive.  Leading spaces are ignored.  If there are no
digits, \exception{ValueError} will be raised.
\end{cfuncdesc}


\subsection{Floating Point Objects \label{floatObjects}}

\obindex{floating point}
\begin{ctypedesc}{PyFloatObject}
This subtype of \ctype{PyObject} represents a Python floating point
object.
\end{ctypedesc}

\begin{cvardesc}{PyTypeObject}{PyFloat_Type}
This instance of \ctype{PyTypeObject} represents the Python floating
point type.  This is the same object as \code{types.FloatType}.
\withsubitem{(in modules types)}{\ttindex{FloatType}}
\end{cvardesc}

\begin{cfuncdesc}{int}{PyFloat_Check}{PyObject *p}
Returns true if its argument is a \ctype{PyFloatObject}.
\end{cfuncdesc}

\begin{cfuncdesc}{PyObject*}{PyFloat_FromDouble}{double v}
Creates a \ctype{PyFloatObject} object from \var{v}, or \NULL{} on
failure.
\end{cfuncdesc}

\begin{cfuncdesc}{double}{PyFloat_AsDouble}{PyObject *pyfloat}
Returns a C \ctype{double} representation of the contents of \var{pyfloat}.
\end{cfuncdesc}

\begin{cfuncdesc}{double}{PyFloat_AS_DOUBLE}{PyObject *pyfloat}
Returns a C \ctype{double} representation of the contents of
\var{pyfloat}, but without error checking.
\end{cfuncdesc}


\subsection{Complex Number Objects \label{complexObjects}}

\obindex{complex number}
Python's complex number objects are implemented as two distinct types
when viewed from the C API:  one is the Python object exposed to
Python programs, and the other is a C structure which represents the
actual complex number value.  The API provides functions for working
with both.

\subsubsection{Complex Numbers as C Structures}

Note that the functions which accept these structures as parameters
and return them as results do so \emph{by value} rather than
dereferencing them through pointers.  This is consistent throughout
the API.

\begin{ctypedesc}{Py_complex}
The C structure which corresponds to the value portion of a Python
complex number object.  Most of the functions for dealing with complex
number objects use structures of this type as input or output values,
as appropriate.  It is defined as:

\begin{verbatim}
typedef struct {
   double real;
   double imag;
} Py_complex;
\end{verbatim}
\end{ctypedesc}

\begin{cfuncdesc}{Py_complex}{_Py_c_sum}{Py_complex left, Py_complex right}
Return the sum of two complex numbers, using the C
\ctype{Py_complex} representation.
\end{cfuncdesc}

\begin{cfuncdesc}{Py_complex}{_Py_c_diff}{Py_complex left, Py_complex right}
Return the difference between two complex numbers, using the C
\ctype{Py_complex} representation.
\end{cfuncdesc}

\begin{cfuncdesc}{Py_complex}{_Py_c_neg}{Py_complex complex}
Return the negation of the complex number \var{complex}, using the C
\ctype{Py_complex} representation.
\end{cfuncdesc}

\begin{cfuncdesc}{Py_complex}{_Py_c_prod}{Py_complex left, Py_complex right}
Return the product of two complex numbers, using the C
\ctype{Py_complex} representation.
\end{cfuncdesc}

\begin{cfuncdesc}{Py_complex}{_Py_c_quot}{Py_complex dividend,
                                          Py_complex divisor}
Return the quotient of two complex numbers, using the C
\ctype{Py_complex} representation.
\end{cfuncdesc}

\begin{cfuncdesc}{Py_complex}{_Py_c_pow}{Py_complex num, Py_complex exp}
Return the exponentiation of \var{num} by \var{exp}, using the C
\ctype{Py_complex} representation.
\end{cfuncdesc}


\subsubsection{Complex Numbers as Python Objects}

\begin{ctypedesc}{PyComplexObject}
This subtype of \ctype{PyObject} represents a Python complex number object.
\end{ctypedesc}

\begin{cvardesc}{PyTypeObject}{PyComplex_Type}
This instance of \ctype{PyTypeObject} represents the Python complex 
number type.
\end{cvardesc}

\begin{cfuncdesc}{int}{PyComplex_Check}{PyObject *p}
Returns true if its argument is a \ctype{PyComplexObject}.
\end{cfuncdesc}

\begin{cfuncdesc}{PyObject*}{PyComplex_FromCComplex}{Py_complex v}
Create a new Python complex number object from a C
\ctype{Py_complex} value.
\end{cfuncdesc}

\begin{cfuncdesc}{PyObject*}{PyComplex_FromDoubles}{double real, double imag}
Returns a new \ctype{PyComplexObject} object from \var{real} and \var{imag}.
\end{cfuncdesc}

\begin{cfuncdesc}{double}{PyComplex_RealAsDouble}{PyObject *op}
Returns the real part of \var{op} as a C \ctype{double}.
\end{cfuncdesc}

\begin{cfuncdesc}{double}{PyComplex_ImagAsDouble}{PyObject *op}
Returns the imaginary part of \var{op} as a C \ctype{double}.
\end{cfuncdesc}

\begin{cfuncdesc}{Py_complex}{PyComplex_AsCComplex}{PyObject *op}
Returns the \ctype{Py_complex} value of the complex number \var{op}.
\end{cfuncdesc}



\section{Other Objects \label{otherObjects}}

\subsection{File Objects \label{fileObjects}}

\obindex{file}
Python's built-in file objects are implemented entirely on the
\ctype{FILE*} support from the C standard library.  This is an
implementation detail and may change in future releases of Python.

\begin{ctypedesc}{PyFileObject}
This subtype of \ctype{PyObject} represents a Python file object.
\end{ctypedesc}

\begin{cvardesc}{PyTypeObject}{PyFile_Type}
This instance of \ctype{PyTypeObject} represents the Python file
type.  This is exposed to Python programs as \code{types.FileType}.
\withsubitem{(in module types)}{\ttindex{FileType}}
\end{cvardesc}

\begin{cfuncdesc}{int}{PyFile_Check}{PyObject *p}
Returns true if its argument is a \ctype{PyFileObject}.
\end{cfuncdesc}

\begin{cfuncdesc}{PyObject*}{PyFile_FromString}{char *filename, char *mode}
On success, returns a new file object that is opened on the
file given by \var{filename}, with a file mode given by \var{mode},
where \var{mode} has the same semantics as the standard C routine
\cfunction{fopen()}\ttindex{fopen()}.  On failure, returns \NULL.
\end{cfuncdesc}

\begin{cfuncdesc}{PyObject*}{PyFile_FromFile}{FILE *fp,
                                              char *name, char *mode,
                                              int (*close)(FILE*)}
Creates a new \ctype{PyFileObject} from the already-open standard C
file pointer, \var{fp}.  The function \var{close} will be called when
the file should be closed.  Returns \NULL{} on failure.
\end{cfuncdesc}

\begin{cfuncdesc}{FILE*}{PyFile_AsFile}{PyFileObject *p}
Returns the file object associated with \var{p} as a \ctype{FILE*}.
\end{cfuncdesc}

\begin{cfuncdesc}{PyObject*}{PyFile_GetLine}{PyObject *p, int n}
Equivalent to \code{\var{p}.readline(\optional{\var{n}})}, this
function reads one line from the object \var{p}.  \var{p} may be a
file object or any object with a \method{readline()} method.  If
\var{n} is \code{0}, exactly one line is read, regardless of the
length of the line.  If \var{n} is greater than \code{0}, no more than 
\var{n} bytes will be read from the file; a partial line can be
returned.  In both cases, an empty string is returned if the end of
the file is reached immediately.  If \var{n} is less than \code{0},
however, one line is read regardless of length, but
\exception{EOFError} is raised if the end of the file is reached
immediately.
\withsubitem{(built-in exception)}{\ttindex{EOFError}}
\end{cfuncdesc}

\begin{cfuncdesc}{PyObject*}{PyFile_Name}{PyObject *p}
Returns the name of the file specified by \var{p} as a string object.
\end{cfuncdesc}

\begin{cfuncdesc}{void}{PyFile_SetBufSize}{PyFileObject *p, int n}
Available on systems with \cfunction{setvbuf()}\ttindex{setvbuf()}
only.  This should only be called immediately after file object
creation.
\end{cfuncdesc}

\begin{cfuncdesc}{int}{PyFile_SoftSpace}{PyObject *p, int newflag}
This function exists for internal use by the interpreter.
Sets the \member{softspace} attribute of \var{p} to \var{newflag} and
\withsubitem{(file attribute)}{\ttindex{softspace}}returns the
previous value.  \var{p} does not have to be a file object
for this function to work properly; any object is supported (thought
its only interesting if the \member{softspace} attribute can be set).
This function clears any errors, and will return \code{0} as the
previous value if the attribute either does not exist or if there were
errors in retrieving it.  There is no way to detect errors from this
function, but doing so should not be needed.
\end{cfuncdesc}

\begin{cfuncdesc}{int}{PyFile_WriteObject}{PyObject *obj, PyFileObject *p,
                                           int flags}
Writes object \var{obj} to file object \var{p}.  The only supported
flag for \var{flags} is \constant{Py_PRINT_RAW}\ttindex{Py_PRINT_RAW};
if given, the \function{str()} of the object is written instead of the 
\function{repr()}.  Returns \code{0} on success or \code{-1} on
failure; the appropriate exception will be set.
\end{cfuncdesc}

\begin{cfuncdesc}{int}{PyFile_WriteString}{char *s, PyFileObject *p,
                                           int flags}
Writes string \var{s} to file object \var{p}.  Returns \code{0} on
success or \code{-1} on failure; the appropriate exception will be
set.
\end{cfuncdesc}


\subsection{Instance Objects \label{instanceObjects}}

\obindex{instance}
There are very few functions specific to instance objects.

\begin{cvardesc}{PyTypeObject}{PyInstance_Type}
  Type object for class instances.
\end{cvardesc}

\begin{cfuncdesc}{int}{PyInstance_Check}{PyObject *obj}
  Returns true if \var{obj} is an instance.
\end{cfuncdesc}

\begin{cfuncdesc}{PyObject*}{PyInstance_New}{PyObject *class,
                                             PyObject *arg,
                                             PyObject *kw}
  Create a new instance of a specific class.  The parameters \var{arg}
  and \var{kw} are used as the positional and keyword parameters to
  the object's constructor.
\end{cfuncdesc}

\begin{cfuncdesc}{PyObject*}{PyInstance_NewRaw}{PyObject *class,
                                                PyObject *dict}
  Create a new instance of a specific class without calling it's
  constructor.  \var{class} is the class of new object.  The
  \var{dict} parameter will be used as the object's \member{__dict__};
  if \NULL, a new dictionary will be created for the instance.
\end{cfuncdesc}


\subsection{Module Objects \label{moduleObjects}}

\obindex{module}
There are only a few functions special to module objects.

\begin{cvardesc}{PyTypeObject}{PyModule_Type}
This instance of \ctype{PyTypeObject} represents the Python module
type.  This is exposed to Python programs as \code{types.ModuleType}.
\withsubitem{(in module types)}{\ttindex{ModuleType}}
\end{cvardesc}

\begin{cfuncdesc}{int}{PyModule_Check}{PyObject *p}
Returns true if its argument is a module object.
\end{cfuncdesc}

\begin{cfuncdesc}{PyObject*}{PyModule_New}{char *name}
Return a new module object with the \member{__name__} attribute set to
\var{name}.  Only the module's \member{__doc__} and
\member{__name__} attributes are filled in; the caller is responsible
for providing a \member{__file__} attribute.
\withsubitem{(module attribute)}{
  \ttindex{__name__}\ttindex{__doc__}\ttindex{__file__}}
\end{cfuncdesc}

\begin{cfuncdesc}{PyObject*}{PyModule_GetDict}{PyObject *module}
Return the dictionary object that implements \var{module}'s namespace; 
this object is the same as the \member{__dict__} attribute of the
module object.  This function never fails.
\withsubitem{(module attribute)}{\ttindex{__dict__}}
\end{cfuncdesc}

\begin{cfuncdesc}{char*}{PyModule_GetName}{PyObject *module}
Return \var{module}'s \member{__name__} value.  If the module does not 
provide one, or if it is not a string, \exception{SystemError} is
raised and \NULL{} is returned.
\withsubitem{(module attribute)}{\ttindex{__name__}}
\withsubitem{(built-in exception)}{\ttindex{SystemError}}
\end{cfuncdesc}

\begin{cfuncdesc}{char*}{PyModule_GetFilename}{PyObject *module}
Return the name of the file from which \var{module} was loaded using
\var{module}'s \member{__file__} attribute.  If this is not defined,
or if it is not a string, raise \exception{SystemError} and return
\NULL.
\withsubitem{(module attribute)}{\ttindex{__file__}}
\withsubitem{(built-in exception)}{\ttindex{SystemError}}
\end{cfuncdesc}

\begin{cfuncdesc}{int}{PyModule_AddObject}{PyObject *module,
                                           char *name, PyObject *value}
Add an object to \var{module} as \var{name}.  This is a convenience
function which can be used from the module's initialization function.
This steals a reference to \var{value}.  Returns \code{-1} on error,
\code{0} on success.
\versionadded{2.0}
\end{cfuncdesc}

\begin{cfuncdesc}{int}{PyModule_AddIntConstant}{PyObject *module,
                                                char *name, int value}
Add an integer constant to \var{module} as \var{name}.  This convenience
function can be used from the module's initialization function.
Returns \code{-1} on error, \code{0} on success.
\versionadded{2.0}
\end{cfuncdesc}

\begin{cfuncdesc}{int}{PyModule_AddStringConstant}{PyObject *module,
                                                   char *name, char *value}
Add a string constant to \var{module} as \var{name}.  This convenience
function can be used from the module's initialization function.  The
string \var{value} must be null-terminated.  Returns \code{-1} on
error, \code{0} on success.
\versionadded{2.0}
\end{cfuncdesc}


\subsection{CObjects \label{cObjects}}

\obindex{CObject}
Refer to \emph{Extending and Embedding the Python Interpreter},
section 1.12 (``Providing a C API for an Extension Module''), for more 
information on using these objects.


\begin{ctypedesc}{PyCObject}
This subtype of \ctype{PyObject} represents an opaque value, useful for
C extension modules who need to pass an opaque value (as a
\ctype{void*} pointer) through Python code to other C code.  It is
often used to make a C function pointer defined in one module
available to other modules, so the regular import mechanism can be
used to access C APIs defined in dynamically loaded modules.
\end{ctypedesc}

\begin{cfuncdesc}{int}{PyCObject_Check}{PyObject *p}
Returns true if its argument is a \ctype{PyCObject}.
\end{cfuncdesc}

\begin{cfuncdesc}{PyObject*}{PyCObject_FromVoidPtr}{void* cobj, 
	void (*destr)(void *)}
Creates a \ctype{PyCObject} from the \code{void *}\var{cobj}.  The
\var{destr} function will be called when the object is reclaimed, unless
it is \NULL.
\end{cfuncdesc}

\begin{cfuncdesc}{PyObject*}{PyCObject_FromVoidPtrAndDesc}{void* cobj,
	void* desc, void (*destr)(void *, void *) }
Creates a \ctype{PyCObject} from the \ctype{void *}\var{cobj}.  The
\var{destr} function will be called when the object is reclaimed.  The
\var{desc} argument can be used to pass extra callback data for the
destructor function.
\end{cfuncdesc}

\begin{cfuncdesc}{void*}{PyCObject_AsVoidPtr}{PyObject* self}
Returns the object \ctype{void *} that the
\ctype{PyCObject} \var{self} was created with.
\end{cfuncdesc}

\begin{cfuncdesc}{void*}{PyCObject_GetDesc}{PyObject* self}
Returns the description \ctype{void *} that the
\ctype{PyCObject} \var{self} was created with.
\end{cfuncdesc}


\chapter{Initialization, Finalization, and Threads
         \label{initialization}}

\begin{cfuncdesc}{void}{Py_Initialize}{}
Initialize the Python interpreter.  In an application embedding 
Python, this should be called before using any other Python/C API 
functions; with the exception of
\cfunction{Py_SetProgramName()}\ttindex{Py_SetProgramName()},
\cfunction{PyEval_InitThreads()}\ttindex{PyEval_InitThreads()},
\cfunction{PyEval_ReleaseLock()}\ttindex{PyEval_ReleaseLock()},
and \cfunction{PyEval_AcquireLock()}\ttindex{PyEval_AcquireLock()}.
This initializes the table of loaded modules (\code{sys.modules}), and
\withsubitem{(in module sys)}{\ttindex{modules}\ttindex{path}}creates the
fundamental modules \module{__builtin__}\refbimodindex{__builtin__},
\module{__main__}\refbimodindex{__main__} and
\module{sys}\refbimodindex{sys}.  It also initializes the module
search\indexiii{module}{search}{path} path (\code{sys.path}).
It does not set \code{sys.argv}; use
\cfunction{PySys_SetArgv()}\ttindex{PySys_SetArgv()} for that.  This
is a no-op when called for a second time (without calling
\cfunction{Py_Finalize()}\ttindex{Py_Finalize()} first).  There is no
return value; it is a fatal error if the initialization fails.
\end{cfuncdesc}

\begin{cfuncdesc}{int}{Py_IsInitialized}{}
Return true (nonzero) when the Python interpreter has been
initialized, false (zero) if not.  After \cfunction{Py_Finalize()} is
called, this returns false until \cfunction{Py_Initialize()} is called
again.
\end{cfuncdesc}

\begin{cfuncdesc}{void}{Py_Finalize}{}
Undo all initializations made by \cfunction{Py_Initialize()} and
subsequent use of Python/C API functions, and destroy all
sub-interpreters (see \cfunction{Py_NewInterpreter()} below) that were
created and not yet destroyed since the last call to
\cfunction{Py_Initialize()}.  Ideally, this frees all memory allocated
by the Python interpreter.  This is a no-op when called for a second
time (without calling \cfunction{Py_Initialize()} again first).  There
is no return value; errors during finalization are ignored.

This function is provided for a number of reasons.  An embedding 
application might want to restart Python without having to restart the 
application itself.  An application that has loaded the Python 
interpreter from a dynamically loadable library (or DLL) might want to 
free all memory allocated by Python before unloading the DLL. During a 
hunt for memory leaks in an application a developer might want to free 
all memory allocated by Python before exiting from the application.

\strong{Bugs and caveats:} The destruction of modules and objects in 
modules is done in random order; this may cause destructors 
(\method{__del__()} methods) to fail when they depend on other objects 
(even functions) or modules.  Dynamically loaded extension modules 
loaded by Python are not unloaded.  Small amounts of memory allocated 
by the Python interpreter may not be freed (if you find a leak, please 
report it).  Memory tied up in circular references between objects is 
not freed.  Some memory allocated by extension modules may not be 
freed.  Some extension may not work properly if their initialization 
routine is called more than once; this can happen if an applcation 
calls \cfunction{Py_Initialize()} and \cfunction{Py_Finalize()} more
than once.
\end{cfuncdesc}

\begin{cfuncdesc}{PyThreadState*}{Py_NewInterpreter}{}
Create a new sub-interpreter.  This is an (almost) totally separate
environment for the execution of Python code.  In particular, the new
interpreter has separate, independent versions of all imported
modules, including the fundamental modules
\module{__builtin__}\refbimodindex{__builtin__},
\module{__main__}\refbimodindex{__main__} and
\module{sys}\refbimodindex{sys}.  The table of loaded modules
(\code{sys.modules}) and the module search path (\code{sys.path}) are
also separate.  The new environment has no \code{sys.argv} variable.
It has new standard I/O stream file objects \code{sys.stdin},
\code{sys.stdout} and \code{sys.stderr} (however these refer to the
same underlying \ctype{FILE} structures in the C library).
\withsubitem{(in module sys)}{
  \ttindex{stdout}\ttindex{stderr}\ttindex{stdin}}

The return value points to the first thread state created in the new 
sub-interpreter.  This thread state is made the current thread state.  
Note that no actual thread is created; see the discussion of thread 
states below.  If creation of the new interpreter is unsuccessful, 
\NULL{} is returned; no exception is set since the exception state 
is stored in the current thread state and there may not be a current 
thread state.  (Like all other Python/C API functions, the global 
interpreter lock must be held before calling this function and is 
still held when it returns; however, unlike most other Python/C API 
functions, there needn't be a current thread state on entry.)

Extension modules are shared between (sub-)interpreters as follows: 
the first time a particular extension is imported, it is initialized 
normally, and a (shallow) copy of its module's dictionary is 
squirreled away.  When the same extension is imported by another 
(sub-)interpreter, a new module is initialized and filled with the 
contents of this copy; the extension's \code{init} function is not
called.  Note that this is different from what happens when an
extension is imported after the interpreter has been completely
re-initialized by calling
\cfunction{Py_Finalize()}\ttindex{Py_Finalize()} and
\cfunction{Py_Initialize()}\ttindex{Py_Initialize()}; in that case,
the extension's \code{init\var{module}} function \emph{is} called
again.

\strong{Bugs and caveats:} Because sub-interpreters (and the main 
interpreter) are part of the same process, the insulation between them 
isn't perfect --- for example, using low-level file operations like 
\withsubitem{(in module os)}{\ttindex{close()}}
\function{os.close()} they can (accidentally or maliciously) affect each 
other's open files.  Because of the way extensions are shared between 
(sub-)interpreters, some extensions may not work properly; this is 
especially likely when the extension makes use of (static) global 
variables, or when the extension manipulates its module's dictionary 
after its initialization.  It is possible to insert objects created in 
one sub-interpreter into a namespace of another sub-interpreter; this 
should be done with great care to avoid sharing user-defined 
functions, methods, instances or classes between sub-interpreters, 
since import operations executed by such objects may affect the 
wrong (sub-)interpreter's dictionary of loaded modules.  (XXX This is 
a hard-to-fix bug that will be addressed in a future release.)
\end{cfuncdesc}

\begin{cfuncdesc}{void}{Py_EndInterpreter}{PyThreadState *tstate}
Destroy the (sub-)interpreter represented by the given thread state.  
The given thread state must be the current thread state.  See the 
discussion of thread states below.  When the call returns, the current 
thread state is \NULL{}.  All thread states associated with this 
interpreted are destroyed.  (The global interpreter lock must be held 
before calling this function and is still held when it returns.)  
\cfunction{Py_Finalize()}\ttindex{Py_Finalize()} will destroy all
sub-interpreters that haven't been explicitly destroyed at that point.
\end{cfuncdesc}

\begin{cfuncdesc}{void}{Py_SetProgramName}{char *name}
This function should be called before
\cfunction{Py_Initialize()}\ttindex{Py_Initialize()} is called
for the first time, if it is called at all.  It tells the interpreter 
the value of the \code{argv[0]} argument to the
\cfunction{main()}\ttindex{main()} function of the program.  This is
used by \cfunction{Py_GetPath()}\ttindex{Py_GetPath()} and some other  
functions below to find the Python run-time libraries relative to the 
interpreter executable.  The default value is \code{'python'}.  The 
argument should point to a zero-terminated character string in static 
storage whose contents will not change for the duration of the 
program's execution.  No code in the Python interpreter will change 
the contents of this storage.
\end{cfuncdesc}

\begin{cfuncdesc}{char*}{Py_GetProgramName}{}
Return the program name set with
\cfunction{Py_SetProgramName()}\ttindex{Py_SetProgramName()}, or the
default.  The returned string points into static storage; the caller 
should not modify its value.
\end{cfuncdesc}

\begin{cfuncdesc}{char*}{Py_GetPrefix}{}
Return the \emph{prefix} for installed platform-independent files.  This 
is derived through a number of complicated rules from the program name 
set with \cfunction{Py_SetProgramName()} and some environment variables; 
for example, if the program name is \code{'/usr/local/bin/python'}, 
the prefix is \code{'/usr/local'}.  The returned string points into 
static storage; the caller should not modify its value.  This 
corresponds to the \makevar{prefix} variable in the top-level 
\file{Makefile} and the \longprogramopt{prefix} argument to the 
\program{configure} script at build time.  The value is available to 
Python code as \code{sys.prefix}.  It is only useful on \UNIX{}.  See 
also the next function.
\end{cfuncdesc}

\begin{cfuncdesc}{char*}{Py_GetExecPrefix}{}
Return the \emph{exec-prefix} for installed platform-\emph{de}pendent 
files.  This is derived through a number of complicated rules from the 
program name set with \cfunction{Py_SetProgramName()} and some environment 
variables; for example, if the program name is 
\code{'/usr/local/bin/python'}, the exec-prefix is 
\code{'/usr/local'}.  The returned string points into static storage; 
the caller should not modify its value.  This corresponds to the 
\makevar{exec_prefix} variable in the top-level \file{Makefile} and the 
\longprogramopt{exec-prefix} argument to the
\program{configure} script at build  time.  The value is available to
Python code as \code{sys.exec_prefix}.  It is only useful on \UNIX{}.

Background: The exec-prefix differs from the prefix when platform 
dependent files (such as executables and shared libraries) are 
installed in a different directory tree.  In a typical installation, 
platform dependent files may be installed in the 
\file{/usr/local/plat} subtree while platform independent may be 
installed in \file{/usr/local}.

Generally speaking, a platform is a combination of hardware and 
software families, e.g.  Sparc machines running the Solaris 2.x 
operating system are considered the same platform, but Intel machines 
running Solaris 2.x are another platform, and Intel machines running 
Linux are yet another platform.  Different major revisions of the same 
operating system generally also form different platforms.  Non-\UNIX{} 
operating systems are a different story; the installation strategies 
on those systems are so different that the prefix and exec-prefix are 
meaningless, and set to the empty string.  Note that compiled Python 
bytecode files are platform independent (but not independent from the 
Python version by which they were compiled!).

System administrators will know how to configure the \program{mount} or 
\program{automount} programs to share \file{/usr/local} between platforms 
while having \file{/usr/local/plat} be a different filesystem for each 
platform.
\end{cfuncdesc}

\begin{cfuncdesc}{char*}{Py_GetProgramFullPath}{}
Return the full program name of the Python executable; this is 
computed as a side-effect of deriving the default module search path 
from the program name (set by
\cfunction{Py_SetProgramName()}\ttindex{Py_SetProgramName()} above).
The returned string points into static storage; the caller should not 
modify its value.  The value is available to Python code as 
\code{sys.executable}.
\withsubitem{(in module sys)}{\ttindex{executable}}
\end{cfuncdesc}

\begin{cfuncdesc}{char*}{Py_GetPath}{}
\indexiii{module}{search}{path}
Return the default module search path; this is computed from the 
program name (set by \cfunction{Py_SetProgramName()} above) and some 
environment variables.  The returned string consists of a series of 
directory names separated by a platform dependent delimiter character.  
The delimiter character is \character{:} on \UNIX{}, \character{;} on
DOS/Windows, and \character{\e n} (the \ASCII{} newline character) on
Macintosh.  The returned string points into static storage; the caller
should not modify its value.  The value is available to Python code 
as the list \code{sys.path}\withsubitem{(in module sys)}{\ttindex{path}},
which may be modified to change the future search path for loaded
modules.

% XXX should give the exact rules
\end{cfuncdesc}

\begin{cfuncdesc}{const char*}{Py_GetVersion}{}
Return the version of this Python interpreter.  This is a string that 
looks something like

\begin{verbatim}
"1.5 (#67, Dec 31 1997, 22:34:28) [GCC 2.7.2.2]"
\end{verbatim}

The first word (up to the first space character) is the current Python 
version; the first three characters are the major and minor version 
separated by a period.  The returned string points into static storage; 
the caller should not modify its value.  The value is available to 
Python code as the list \code{sys.version}.
\withsubitem{(in module sys)}{\ttindex{version}}
\end{cfuncdesc}

\begin{cfuncdesc}{const char*}{Py_GetPlatform}{}
Return the platform identifier for the current platform.  On \UNIX{}, 
this is formed from the ``official'' name of the operating system, 
converted to lower case, followed by the major revision number; e.g., 
for Solaris 2.x, which is also known as SunOS 5.x, the value is 
\code{'sunos5'}.  On Macintosh, it is \code{'mac'}.  On Windows, it 
is \code{'win'}.  The returned string points into static storage; 
the caller should not modify its value.  The value is available to 
Python code as \code{sys.platform}.
\withsubitem{(in module sys)}{\ttindex{platform}}
\end{cfuncdesc}

\begin{cfuncdesc}{const char*}{Py_GetCopyright}{}
Return the official copyright string for the current Python version, 
for example

\code{'Copyright 1991-1995 Stichting Mathematisch Centrum, Amsterdam'}

The returned string points into static storage; the caller should not 
modify its value.  The value is available to Python code as the list 
\code{sys.copyright}.
\withsubitem{(in module sys)}{\ttindex{copyright}}
\end{cfuncdesc}

\begin{cfuncdesc}{const char*}{Py_GetCompiler}{}
Return an indication of the compiler used to build the current Python 
version, in square brackets, for example:

\begin{verbatim}
"[GCC 2.7.2.2]"
\end{verbatim}

The returned string points into static storage; the caller should not 
modify its value.  The value is available to Python code as part of 
the variable \code{sys.version}.
\withsubitem{(in module sys)}{\ttindex{version}}
\end{cfuncdesc}

\begin{cfuncdesc}{const char*}{Py_GetBuildInfo}{}
Return information about the sequence number and build date and time 
of the current Python interpreter instance, for example

\begin{verbatim}
"#67, Aug  1 1997, 22:34:28"
\end{verbatim}

The returned string points into static storage; the caller should not 
modify its value.  The value is available to Python code as part of 
the variable \code{sys.version}.
\withsubitem{(in module sys)}{\ttindex{version}}
\end{cfuncdesc}

\begin{cfuncdesc}{int}{PySys_SetArgv}{int argc, char **argv}
Set \code{sys.argv} based on \var{argc} and \var{argv}.  These
parameters are similar to those passed to the program's
\cfunction{main()}\ttindex{main()} function with the difference that
the first entry should refer to the script file to be executed rather
than the executable hosting the Python interpreter.  If there isn't a
script that will be run, the first entry in \var{argv} can be an empty
string.  If this function fails to initialize \code{sys.argv}, a fatal 
condition is signalled using
\cfunction{Py_FatalError()}\ttindex{Py_FatalError()}.
\withsubitem{(in module sys)}{\ttindex{argv}}
% XXX impl. doesn't seem consistent in allowing 0/NULL for the params; 
% check w/ Guido.
\end{cfuncdesc}

% XXX Other PySys thingies (doesn't really belong in this chapter)

\section{Thread State and the Global Interpreter Lock
         \label{threads}}

\index{global interpreter lock}
\index{interpreter lock}
\index{lock, interpreter}

The Python interpreter is not fully thread safe.  In order to support
multi-threaded Python programs, there's a global lock that must be
held by the current thread before it can safely access Python objects.
Without the lock, even the simplest operations could cause problems in
a multi-threaded program: for example, when two threads simultaneously
increment the reference count of the same object, the reference count
could end up being incremented only once instead of twice.

Therefore, the rule exists that only the thread that has acquired the
global interpreter lock may operate on Python objects or call Python/C
API functions.  In order to support multi-threaded Python programs,
the interpreter regularly releases and reacquires the lock --- by
default, every ten bytecode instructions (this can be changed with
\withsubitem{(in module sys)}{\ttindex{setcheckinterval()}}
\function{sys.setcheckinterval()}).  The lock is also released and
reacquired around potentially blocking I/O operations like reading or
writing a file, so that other threads can run while the thread that
requests the I/O is waiting for the I/O operation to complete.

The Python interpreter needs to keep some bookkeeping information
separate per thread --- for this it uses a data structure called
\ctype{PyThreadState}\ttindex{PyThreadState}.  This is new in Python
1.5; in earlier versions, such state was stored in global variables,
and switching threads could cause problems.  In particular, exception
handling is now thread safe, when the application uses
\withsubitem{(in module sys)}{\ttindex{exc_info()}}
\function{sys.exc_info()} to access the exception last raised in the
current thread.

There's one global variable left, however: the pointer to the current
\ctype{PyThreadState}\ttindex{PyThreadState} structure.  While most
thread packages have a way to store ``per-thread global data,''
Python's internal platform independent thread abstraction doesn't
support this yet.  Therefore, the current thread state must be
manipulated explicitly.

This is easy enough in most cases.  Most code manipulating the global
interpreter lock has the following simple structure:

\begin{verbatim}
Save the thread state in a local variable.
Release the interpreter lock.
...Do some blocking I/O operation...
Reacquire the interpreter lock.
Restore the thread state from the local variable.
\end{verbatim}

This is so common that a pair of macros exists to simplify it:

\begin{verbatim}
Py_BEGIN_ALLOW_THREADS
...Do some blocking I/O operation...
Py_END_ALLOW_THREADS
\end{verbatim}

The \code{Py_BEGIN_ALLOW_THREADS}\ttindex{Py_BEGIN_ALLOW_THREADS} macro
opens a new block and declares a hidden local variable; the
\code{Py_END_ALLOW_THREADS}\ttindex{Py_END_ALLOW_THREADS} macro closes 
the block.  Another advantage of using these two macros is that when
Python is compiled without thread support, they are defined empty,
thus saving the thread state and lock manipulations.

When thread support is enabled, the block above expands to the
following code:

\begin{verbatim}
    PyThreadState *_save;

    _save = PyEval_SaveThread();
    ...Do some blocking I/O operation...
    PyEval_RestoreThread(_save);
\end{verbatim}

Using even lower level primitives, we can get roughly the same effect
as follows:

\begin{verbatim}
    PyThreadState *_save;

    _save = PyThreadState_Swap(NULL);
    PyEval_ReleaseLock();
    ...Do some blocking I/O operation...
    PyEval_AcquireLock();
    PyThreadState_Swap(_save);
\end{verbatim}

There are some subtle differences; in particular,
\cfunction{PyEval_RestoreThread()}\ttindex{PyEval_RestoreThread()} saves
and restores the value of the  global variable
\cdata{errno}\ttindex{errno}, since the lock manipulation does not
guarantee that \cdata{errno} is left alone.  Also, when thread support
is disabled,
\cfunction{PyEval_SaveThread()}\ttindex{PyEval_SaveThread()} and
\cfunction{PyEval_RestoreThread()} don't manipulate the lock; in this
case, \cfunction{PyEval_ReleaseLock()}\ttindex{PyEval_ReleaseLock()} and
\cfunction{PyEval_AcquireLock()}\ttindex{PyEval_AcquireLock()} are not
available.  This is done so that dynamically loaded extensions
compiled with thread support enabled can be loaded by an interpreter
that was compiled with disabled thread support.

The global interpreter lock is used to protect the pointer to the
current thread state.  When releasing the lock and saving the thread
state, the current thread state pointer must be retrieved before the
lock is released (since another thread could immediately acquire the
lock and store its own thread state in the global variable).
Conversely, when acquiring the lock and restoring the thread state,
the lock must be acquired before storing the thread state pointer.

Why am I going on with so much detail about this?  Because when
threads are created from C, they don't have the global interpreter
lock, nor is there a thread state data structure for them.  Such
threads must bootstrap themselves into existence, by first creating a
thread state data structure, then acquiring the lock, and finally
storing their thread state pointer, before they can start using the
Python/C API.  When they are done, they should reset the thread state
pointer, release the lock, and finally free their thread state data
structure.

When creating a thread data structure, you need to provide an
interpreter state data structure.  The interpreter state data
structure hold global data that is shared by all threads in an
interpreter, for example the module administration
(\code{sys.modules}).  Depending on your needs, you can either create
a new interpreter state data structure, or share the interpreter state
data structure used by the Python main thread (to access the latter,
you must obtain the thread state and access its \member{interp} member;
this must be done by a thread that is created by Python or by the main
thread after Python is initialized).


\begin{ctypedesc}{PyInterpreterState}
This data structure represents the state shared by a number of
cooperating threads.  Threads belonging to the same interpreter
share their module administration and a few other internal items.
There are no public members in this structure.

Threads belonging to different interpreters initially share nothing,
except process state like available memory, open file descriptors and
such.  The global interpreter lock is also shared by all threads,
regardless of to which interpreter they belong.
\end{ctypedesc}

\begin{ctypedesc}{PyThreadState}
This data structure represents the state of a single thread.  The only
public data member is \ctype{PyInterpreterState *}\member{interp},
which points to this thread's interpreter state.
\end{ctypedesc}

\begin{cfuncdesc}{void}{PyEval_InitThreads}{}
Initialize and acquire the global interpreter lock.  It should be
called in the main thread before creating a second thread or engaging
in any other thread operations such as
\cfunction{PyEval_ReleaseLock()}\ttindex{PyEval_ReleaseLock()} or
\code{PyEval_ReleaseThread(\var{tstate})}\ttindex{PyEval_ReleaseThread()}.
It is not needed before calling
\cfunction{PyEval_SaveThread()}\ttindex{PyEval_SaveThread()} or
\cfunction{PyEval_RestoreThread()}\ttindex{PyEval_RestoreThread()}.

This is a no-op when called for a second time.  It is safe to call
this function before calling
\cfunction{Py_Initialize()}\ttindex{Py_Initialize()}.

When only the main thread exists, no lock operations are needed.  This
is a common situation (most Python programs do not use threads), and
the lock operations slow the interpreter down a bit.  Therefore, the
lock is not created initially.  This situation is equivalent to having
acquired the lock: when there is only a single thread, all object
accesses are safe.  Therefore, when this function initializes the
lock, it also acquires it.  Before the Python
\module{thread}\refbimodindex{thread} module creates a new thread,
knowing that either it has the lock or the lock hasn't been created
yet, it calls \cfunction{PyEval_InitThreads()}.  When this call
returns, it is guaranteed that the lock has been created and that it
has acquired it.

It is \strong{not} safe to call this function when it is unknown which
thread (if any) currently has the global interpreter lock.

This function is not available when thread support is disabled at
compile time.
\end{cfuncdesc}

\begin{cfuncdesc}{void}{PyEval_AcquireLock}{}
Acquire the global interpreter lock.  The lock must have been created
earlier.  If this thread already has the lock, a deadlock ensues.
This function is not available when thread support is disabled at
compile time.
\end{cfuncdesc}

\begin{cfuncdesc}{void}{PyEval_ReleaseLock}{}
Release the global interpreter lock.  The lock must have been created
earlier.  This function is not available when thread support is
disabled at compile time.
\end{cfuncdesc}

\begin{cfuncdesc}{void}{PyEval_AcquireThread}{PyThreadState *tstate}
Acquire the global interpreter lock and then set the current thread
state to \var{tstate}, which should not be \NULL{}.  The lock must
have been created earlier.  If this thread already has the lock,
deadlock ensues.  This function is not available when thread support
is disabled at compile time.
\end{cfuncdesc}

\begin{cfuncdesc}{void}{PyEval_ReleaseThread}{PyThreadState *tstate}
Reset the current thread state to \NULL{} and release the global
interpreter lock.  The lock must have been created earlier and must be
held by the current thread.  The \var{tstate} argument, which must not
be \NULL{}, is only used to check that it represents the current
thread state --- if it isn't, a fatal error is reported.  This
function is not available when thread support is disabled at compile
time.
\end{cfuncdesc}

\begin{cfuncdesc}{PyThreadState*}{PyEval_SaveThread}{}
Release the interpreter lock (if it has been created and thread
support is enabled) and reset the thread state to \NULL{},
returning the previous thread state (which is not \NULL{}).  If
the lock has been created, the current thread must have acquired it.
(This function is available even when thread support is disabled at
compile time.)
\end{cfuncdesc}

\begin{cfuncdesc}{void}{PyEval_RestoreThread}{PyThreadState *tstate}
Acquire the interpreter lock (if it has been created and thread
support is enabled) and set the thread state to \var{tstate}, which
must not be \NULL{}.  If the lock has been created, the current
thread must not have acquired it, otherwise deadlock ensues.  (This
function is available even when thread support is disabled at compile
time.)
\end{cfuncdesc}

The following macros are normally used without a trailing semicolon;
look for example usage in the Python source distribution.

\begin{csimplemacrodesc}{Py_BEGIN_ALLOW_THREADS}
This macro expands to
\samp{\{ PyThreadState *_save; _save = PyEval_SaveThread();}.
Note that it contains an opening brace; it must be matched with a
following \code{Py_END_ALLOW_THREADS} macro.  See above for further
discussion of this macro.  It is a no-op when thread support is
disabled at compile time.
\end{csimplemacrodesc}

\begin{csimplemacrodesc}{Py_END_ALLOW_THREADS}
This macro expands to
\samp{PyEval_RestoreThread(_save); \}}.
Note that it contains a closing brace; it must be matched with an
earlier \code{Py_BEGIN_ALLOW_THREADS} macro.  See above for further
discussion of this macro.  It is a no-op when thread support is
disabled at compile time.
\end{csimplemacrodesc}

\begin{csimplemacrodesc}{Py_BEGIN_BLOCK_THREADS}
This macro expands to \samp{PyEval_RestoreThread(_save);} i.e. it
is equivalent to \code{Py_END_ALLOW_THREADS} without the closing
brace.  It is a no-op when thread support is disabled at compile
time.
\end{csimplemacrodesc}

\begin{csimplemacrodesc}{Py_BEGIN_UNBLOCK_THREADS}
This macro expands to \samp{_save = PyEval_SaveThread();} i.e. it is
equivalent to \code{Py_BEGIN_ALLOW_THREADS} without the opening brace
and variable declaration.  It is a no-op when thread support is
disabled at compile time.
\end{csimplemacrodesc}

All of the following functions are only available when thread support
is enabled at compile time, and must be called only when the
interpreter lock has been created.

\begin{cfuncdesc}{PyInterpreterState*}{PyInterpreterState_New}{}
Create a new interpreter state object.  The interpreter lock need not
be held, but may be held if it is necessary to serialize calls to this
function.
\end{cfuncdesc}

\begin{cfuncdesc}{void}{PyInterpreterState_Clear}{PyInterpreterState *interp}
Reset all information in an interpreter state object.  The interpreter
lock must be held.
\end{cfuncdesc}

\begin{cfuncdesc}{void}{PyInterpreterState_Delete}{PyInterpreterState *interp}
Destroy an interpreter state object.  The interpreter lock need not be
held.  The interpreter state must have been reset with a previous
call to \cfunction{PyInterpreterState_Clear()}.
\end{cfuncdesc}

\begin{cfuncdesc}{PyThreadState*}{PyThreadState_New}{PyInterpreterState *interp}
Create a new thread state object belonging to the given interpreter
object.  The interpreter lock need not be held, but may be held if it
is necessary to serialize calls to this function.
\end{cfuncdesc}

\begin{cfuncdesc}{void}{PyThreadState_Clear}{PyThreadState *tstate}
Reset all information in a thread state object.  The interpreter lock
must be held.
\end{cfuncdesc}

\begin{cfuncdesc}{void}{PyThreadState_Delete}{PyThreadState *tstate}
Destroy a thread state object.  The interpreter lock need not be
held.  The thread state must have been reset with a previous
call to \cfunction{PyThreadState_Clear()}.
\end{cfuncdesc}

\begin{cfuncdesc}{PyThreadState*}{PyThreadState_Get}{}
Return the current thread state.  The interpreter lock must be held.
When the current thread state is \NULL{}, this issues a fatal
error (so that the caller needn't check for \NULL{}).
\end{cfuncdesc}

\begin{cfuncdesc}{PyThreadState*}{PyThreadState_Swap}{PyThreadState *tstate}
Swap the current thread state with the thread state given by the
argument \var{tstate}, which may be \NULL{}.  The interpreter lock
must be held.
\end{cfuncdesc}


\chapter{Memory Management \label{memory}}
\sectionauthor{Vladimir Marangozov}{Vladimir.Marangozov@inrialpes.fr}


\section{Overview \label{memoryOverview}}

Memory management in Python involves a private heap containing all
Python objects and data structures. The management of this private
heap is ensured internally by the \emph{Python memory manager}.  The
Python memory manager has different components which deal with various
dynamic storage management aspects, like sharing, segmentation,
preallocation or caching.

At the lowest level, a raw memory allocator ensures that there is
enough room in the private heap for storing all Python-related data
by interacting with the memory manager of the operating system. On top
of the raw memory allocator, several object-specific allocators
operate on the same heap and implement distinct memory management
policies adapted to the peculiarities of every object type. For
example, integer objects are managed differently within the heap than
strings, tuples or dictionaries because integers imply different
storage requirements and speed/space tradeoffs. The Python memory
manager thus delegates some of the work to the object-specific
allocators, but ensures that the latter operate within the bounds of
the private heap.

It is important to understand that the management of the Python heap
is performed by the interpreter itself and that the user has no
control on it, even if she regularly manipulates object pointers to
memory blocks inside that heap.  The allocation of heap space for
Python objects and other internal buffers is performed on demand by
the Python memory manager through the Python/C API functions listed in
this document.

To avoid memory corruption, extension writers should never try to
operate on Python objects with the functions exported by the C
library: \cfunction{malloc()}\ttindex{malloc()},
\cfunction{calloc()}\ttindex{calloc()},
\cfunction{realloc()}\ttindex{realloc()} and
\cfunction{free()}\ttindex{free()}.  This will result in 
mixed calls between the C allocator and the Python memory manager
with fatal consequences, because they implement different algorithms
and operate on different heaps.  However, one may safely allocate and
release memory blocks with the C library allocator for individual
purposes, as shown in the following example:

\begin{verbatim}
    PyObject *res;
    char *buf = (char *) malloc(BUFSIZ); /* for I/O */

    if (buf == NULL)
        return PyErr_NoMemory();
    ...Do some I/O operation involving buf...
    res = PyString_FromString(buf);
    free(buf); /* malloc'ed */
    return res;
\end{verbatim}

In this example, the memory request for the I/O buffer is handled by
the C library allocator. The Python memory manager is involved only
in the allocation of the string object returned as a result.

In most situations, however, it is recommended to allocate memory from
the Python heap specifically because the latter is under control of
the Python memory manager. For example, this is required when the
interpreter is extended with new object types written in C. Another
reason for using the Python heap is the desire to \emph{inform} the
Python memory manager about the memory needs of the extension module.
Even when the requested memory is used exclusively for internal,
highly-specific purposes, delegating all memory requests to the Python
memory manager causes the interpreter to have a more accurate image of
its memory footprint as a whole. Consequently, under certain
circumstances, the Python memory manager may or may not trigger
appropriate actions, like garbage collection, memory compaction or
other preventive procedures. Note that by using the C library
allocator as shown in the previous example, the allocated memory for
the I/O buffer escapes completely the Python memory manager.


\section{Memory Interface \label{memoryInterface}}

The following function sets, modeled after the ANSI C standard, are
available for allocating and releasing memory from the Python heap:


\begin{cfuncdesc}{void*}{PyMem_Malloc}{size_t n}
Allocates \var{n} bytes and returns a pointer of type \ctype{void*} to
the allocated memory, or \NULL{} if the request fails. Requesting zero
bytes returns a non-\NULL{} pointer.
\end{cfuncdesc}

\begin{cfuncdesc}{void*}{PyMem_Realloc}{void *p, size_t n}
Resizes the memory block pointed to by \var{p} to \var{n} bytes. The
contents will be unchanged to the minimum of the old and the new
sizes. If \var{p} is \NULL{}, the call is equivalent to
\cfunction{PyMem_Malloc(\var{n})}; if \var{n} is equal to zero, the memory block
is resized but is not freed, and the returned pointer is non-\NULL{}.
Unless \var{p} is \NULL{}, it must have been returned by a previous
call to \cfunction{PyMem_Malloc()} or \cfunction{PyMem_Realloc()}.
\end{cfuncdesc}

\begin{cfuncdesc}{void}{PyMem_Free}{void *p}
Frees the memory block pointed to by \var{p}, which must have been
returned by a previous call to \cfunction{PyMem_Malloc()} or
\cfunction{PyMem_Realloc()}.  Otherwise, or if
\cfunction{PyMem_Free(p)} has been called before, undefined behaviour
occurs. If \var{p} is \NULL{}, no operation is performed.
\end{cfuncdesc}

The following type-oriented macros are provided for convenience.  Note 
that \var{TYPE} refers to any C type.

\begin{cfuncdesc}{\var{TYPE}*}{PyMem_New}{TYPE, size_t n}
Same as \cfunction{PyMem_Malloc()}, but allocates \code{(\var{n} *
sizeof(\var{TYPE}))} bytes of memory.  Returns a pointer cast to
\ctype{\var{TYPE}*}.
\end{cfuncdesc}

\begin{cfuncdesc}{\var{TYPE}*}{PyMem_Resize}{void *p, TYPE, size_t n}
Same as \cfunction{PyMem_Realloc()}, but the memory block is resized
to \code{(\var{n} * sizeof(\var{TYPE}))} bytes.  Returns a pointer
cast to \ctype{\var{TYPE}*}.
\end{cfuncdesc}

\begin{cfuncdesc}{void}{PyMem_Del}{void *p}
Same as \cfunction{PyMem_Free()}.
\end{cfuncdesc}

In addition, the following macro sets are provided for calling the
Python memory allocator directly, without involving the C API functions
listed above. However, note that their use does not preserve binary
compatibility accross Python versions and is therefore deprecated in
extension modules.

\cfunction{PyMem_MALLOC()}, \cfunction{PyMem_REALLOC()}, \cfunction{PyMem_FREE()}.

\cfunction{PyMem_NEW()}, \cfunction{PyMem_RESIZE()}, \cfunction{PyMem_DEL()}.


\section{Examples \label{memoryExamples}}

Here is the example from section \ref{memoryOverview}, rewritten so
that the I/O buffer is allocated from the Python heap by using the
first function set:

\begin{verbatim}
    PyObject *res;
    char *buf = (char *) PyMem_Malloc(BUFSIZ); /* for I/O */

    if (buf == NULL)
        return PyErr_NoMemory();
    /* ...Do some I/O operation involving buf... */
    res = PyString_FromString(buf);
    PyMem_Free(buf); /* allocated with PyMem_Malloc */
    return res;
\end{verbatim}

The same code using the type-oriented function set:

\begin{verbatim}
    PyObject *res;
    char *buf = PyMem_New(char, BUFSIZ); /* for I/O */

    if (buf == NULL)
        return PyErr_NoMemory();
    /* ...Do some I/O operation involving buf... */
    res = PyString_FromString(buf);
    PyMem_Del(buf); /* allocated with PyMem_New */
    return res;
\end{verbatim}

Note that in the two examples above, the buffer is always
manipulated via functions belonging to the same set. Indeed, it
is required to use the same memory API family for a given
memory block, so that the risk of mixing different allocators is
reduced to a minimum. The following code sequence contains two errors,
one of which is labeled as \emph{fatal} because it mixes two different
allocators operating on different heaps.

\begin{verbatim}
char *buf1 = PyMem_New(char, BUFSIZ);
char *buf2 = (char *) malloc(BUFSIZ);
char *buf3 = (char *) PyMem_Malloc(BUFSIZ);
...
PyMem_Del(buf3);  /* Wrong -- should be PyMem_Free() */
free(buf2);       /* Right -- allocated via malloc() */
free(buf1);       /* Fatal -- should be PyMem_Del()  */
\end{verbatim}

In addition to the functions aimed at handling raw memory blocks from
the Python heap, objects in Python are allocated and released with
\cfunction{PyObject_New()}, \cfunction{PyObject_NewVar()} and
\cfunction{PyObject_Del()}, or with their corresponding macros
\cfunction{PyObject_NEW()}, \cfunction{PyObject_NEW_VAR()} and
\cfunction{PyObject_DEL()}.

These will be explained in the next chapter on defining and
implementing new object types in C.


\chapter{Defining New Object Types \label{newTypes}}

\begin{cfuncdesc}{PyObject*}{_PyObject_New}{PyTypeObject *type}
\end{cfuncdesc}

\begin{cfuncdesc}{PyVarObject*}{_PyObject_NewVar}{PyTypeObject *type, int size}
\end{cfuncdesc}

\begin{cfuncdesc}{void}{_PyObject_Del}{PyObject *op}
\end{cfuncdesc}

\begin{cfuncdesc}{PyObject*}{PyObject_Init}{PyObject *op,
						PyTypeObject *type}
\end{cfuncdesc}

\begin{cfuncdesc}{PyVarObject*}{PyObject_InitVar}{PyVarObject *op,
						PyTypeObject *type, int size}
\end{cfuncdesc}

\begin{cfuncdesc}{\var{TYPE}*}{PyObject_New}{TYPE, PyTypeObject *type}
\end{cfuncdesc}

\begin{cfuncdesc}{\var{TYPE}*}{PyObject_NewVar}{TYPE, PyTypeObject *type,
                                                int size}
\end{cfuncdesc}

\begin{cfuncdesc}{void}{PyObject_Del}{PyObject *op}
\end{cfuncdesc}

\begin{cfuncdesc}{\var{TYPE}*}{PyObject_NEW}{TYPE, PyTypeObject *type}
\end{cfuncdesc}

\begin{cfuncdesc}{\var{TYPE}*}{PyObject_NEW_VAR}{TYPE, PyTypeObject *type,
                                                int size}
\end{cfuncdesc}

\begin{cfuncdesc}{void}{PyObject_DEL}{PyObject *op}
\end{cfuncdesc}

\begin{cfuncdesc}{PyObject*}{Py_InitModule}{char *name,
                                            PyMethodDef *methods}
  Create a new module object based on a name and table of functions,
  returning the new module object.
\end{cfuncdesc}

\begin{cfuncdesc}{PyObject*}{Py_InitModule3}{char *name,
                                             PyMethodDef *methods,
                                             char *doc}
  Create a new module object based on a name and table of functions,
  returning the new module object.  If \var{doc} is non-\NULL, it will
  be used to define the docstring for the module.
\end{cfuncdesc}

\begin{cfuncdesc}{PyObject*}{Py_InitModule4}{char *name,
                                             PyMethodDef *methods,
                                             char *doc, PyObject *self,
                                             int apiver}
  Create a new module object based on a name and table of functions,
  returning the new module object.  If \var{doc} is non-\NULL, it will
  be used to define the docstring for the module.  If \var{self} is
  non-\NULL, it will passed to the functions of the module as their
  (otherwise \NULL) first parameter.  (This was added as an
  experimental feature, and there are no known uses in the current
  version of Python.)  For \var{apiver}, the only value which should
  be passed is defined by the constant \constant{PYTHON_API_VERSION}.

  \strong{Note:}  Most uses of this function should probably be using
  the \cfunction{Py_InitModule3()} instead; only use this if you are
  sure you need it.
\end{cfuncdesc}

PyArg_ParseTupleAndKeywords, PyArg_ParseTuple, PyArg_Parse

Py_BuildValue

DL_IMPORT

_Py_NoneStruct


\section{Common Object Structures \label{common-structs}}

PyObject, PyVarObject

PyObject_HEAD, PyObject_HEAD_INIT, PyObject_VAR_HEAD

Typedefs:
unaryfunc, binaryfunc, ternaryfunc, inquiry, coercion, intargfunc,
intintargfunc, intobjargproc, intintobjargproc, objobjargproc,
destructor, printfunc, getattrfunc, getattrofunc, setattrfunc,
setattrofunc, cmpfunc, reprfunc, hashfunc

\begin{ctypedesc}{PyCFunction}
Type of the functions used to implement most Python callables in C.
\end{ctypedesc}

\begin{ctypedesc}{PyMethodDef}
Structure used to describe a method of an extension type.  This
structure has four fields:

\begin{tableiii}{l|l|l}{member}{Field}{C Type}{Meaning}
  \lineiii{ml_name}{char *}{name of the method}
  \lineiii{ml_meth}{PyCFunction}{pointer to the C implementation}
  \lineiii{ml_flags}{int}{flag bits indicating how the call should be
                          constructed}
  \lineiii{ml_doc}{char *}{points to the contents of the docstring}
\end{tableiii}
\end{ctypedesc}

\begin{cfuncdesc}{PyObject*}{Py_FindMethod}{PyMethodDef[] table,
                                            PyObject *ob, char *name}
Return a bound method object for an extension type implemented in C.
This function also handles the special attribute \member{__methods__},
returning a list of all the method names defined in \var{table}.
\end{cfuncdesc}


\section{Mapping Object Structures \label{mapping-structs}}

\begin{ctypedesc}{PyMappingMethods}
Structure used to hold pointers to the functions used to implement the 
mapping protocol for an extension type.
\end{ctypedesc}


\section{Number Object Structures \label{number-structs}}

\begin{ctypedesc}{PyNumberMethods}
Structure used to hold pointers to the functions an extension type
uses to implement the number protocol.
\end{ctypedesc}


\section{Sequence Object Structures \label{sequence-structs}}

\begin{ctypedesc}{PySequenceMethods}
Structure used to hold pointers to the functions which an object uses
to implement the sequence protocol.
\end{ctypedesc}


\section{Buffer Object Structures \label{buffer-structs}}
\sectionauthor{Greg J. Stein}{greg@lyra.org}

The buffer interface exports a model where an object can expose its
internal data as a set of chunks of data, where each chunk is
specified as a pointer/length pair.  These chunks are called
\dfn{segments} and are presumed to be non-contiguous in memory.

If an object does not export the buffer interface, then its
\member{tp_as_buffer} member in the \ctype{PyTypeObject} structure
should be \NULL{}.  Otherwise, the \member{tp_as_buffer} will point to
a \ctype{PyBufferProcs} structure.

\strong{Note:} It is very important that your
\ctype{PyTypeObject} structure uses \constant{Py_TPFLAGS_DEFAULT} for
the value of the \member{tp_flags} member rather than \code{0}.  This
tells the Python runtime that your \ctype{PyBufferProcs} structure
contains the \member{bf_getcharbuffer} slot. Older versions of Python
did not have this member, so a new Python interpreter using an old
extension needs to be able to test for its presence before using it.

\begin{ctypedesc}{PyBufferProcs}
Structure used to hold the function pointers which define an
implementation of the buffer protocol.

The first slot is \member{bf_getreadbuffer}, of type
\ctype{getreadbufferproc}.  If this slot is \NULL{}, then the object
does not support reading from the internal data.  This is
non-sensical, so implementors should fill this in, but callers should
test that the slot contains a non-\NULL{} value.

The next slot is \member{bf_getwritebuffer} having type
\ctype{getwritebufferproc}. This slot may be \NULL{} if the object
does not allow writing into its returned buffers.

The third slot is \member{bf_getsegcount}, with type
\ctype{getsegcountproc}.  This slot must not be \NULL{} and is used to 
inform the caller how many segments the object contains.  Simple
objects such as \ctype{PyString_Type} and
\ctype{PyBuffer_Type} objects contain a single segment.

The last slot is \member{bf_getcharbuffer}, of type
\ctype{getcharbufferproc}.  This slot will only be present if the
\constant{Py_TPFLAGS_HAVE_GETCHARBUFFER} flag is present in the
\member{tp_flags} field of the object's \ctype{PyTypeObject}.  Before using
this slot, the caller should test whether it is present by using the
\cfunction{PyType_HasFeature()}\ttindex{PyType_HasFeature()} function.
If present, it may be \NULL, indicating that the object's contents
cannot be used as \emph{8-bit characters}.
The slot function may also raise an error if the object's contents
cannot be interpreted as 8-bit characters.  For example, if the object
is an array which is configured to hold floating point values, an
exception may be raised if a caller attempts to use
\member{bf_getcharbuffer} to fetch a sequence of 8-bit characters.
This notion of exporting the internal buffers as ``text'' is used to
distinguish between objects that are binary in nature, and those which
have character-based content.

\strong{Note:} The current policy seems to state that these characters
may be multi-byte characters. This implies that a buffer size of
\var{N} does not mean there are \var{N} characters present.
\end{ctypedesc}

\begin{datadesc}{Py_TPFLAGS_HAVE_GETCHARBUFFER}
Flag bit set in the type structure to indicate that the
\member{bf_getcharbuffer} slot is known.  This being set does not
indicate that the object supports the buffer interface or that the
\member{bf_getcharbuffer} slot is non-\NULL.
\end{datadesc}

\begin{ctypedesc}[getreadbufferproc]{int (*getreadbufferproc)
                            (PyObject *self, int segment, void **ptrptr)}
Return a pointer to a readable segment of the buffer.  This function
is allowed to raise an exception, in which case it must return
\code{-1}.  The \var{segment} which is passed must be zero or
positive, and strictly less than the number of segments returned by
the \member{bf_getsegcount} slot function.  On success, it returns the
length of the buffer memory, and sets \code{*\var{ptrptr}} to a
pointer to that memory.
\end{ctypedesc}

\begin{ctypedesc}[getwritebufferproc]{int (*getwritebufferproc)
                            (PyObject *self, int segment, void **ptrptr)}
Return a pointer to a writable memory buffer in \code{*\var{ptrptr}},
and the length of that segment as the function return value.
The memory buffer must correspond to buffer segment \var{segment}.
Must return \code{-1} and set an exception on error.
\exception{TypeError} should be raised if the object only supports
read-only buffers, and \exception{SystemError} should be raised when
\var{segment} specifies a segment that doesn't exist.
% Why doesn't it raise ValueError for this one?
% GJS: because you shouldn't be calling it with an invalid
%      segment. That indicates a blatant programming error in the C
%      code.
\end{ctypedesc}

\begin{ctypedesc}[getsegcountproc]{int (*getsegcountproc)
                            (PyObject *self, int *lenp)}
Return the number of memory segments which comprise the buffer.  If
\var{lenp} is not \NULL, the implementation must report the sum of the 
sizes (in bytes) of all segments in \code{*\var{lenp}}.
The function cannot fail.
\end{ctypedesc}

\begin{ctypedesc}[getcharbufferproc]{int (*getcharbufferproc)
                            (PyObject *self, int segment, const char **ptrptr)}
\end{ctypedesc}


\section{Supporting Cyclic Garbarge Collection
         \label{supporting-cycle-detection}}

Python's support for detecting and collecting garbage which involves
circular references requires support from object types which are
``containers'' for other objects which may also be containers.  Types
which do not store references to other objects, or which only store
references to atomic types (such as numbers or strings), do not need
to provide any explicit support for garbage collection.

To create a container type, the \member{tp_flags} field of the type
object must include the \constant{Py_TPFLAGS_GC} and provide an
implementation of the \member{tp_traverse} handler.  The computed
value of the \member{tp_basicsize} field must include
\constant{PyGC_HEAD_SIZE} as well.  If instances of the type are
mutable, a \member{tp_clear} implementation must also be provided.

\begin{datadesc}{Py_TPFLAGS_GC}
  Objects with a type with this flag set must conform with the rules
  documented here.  For convenience these objects will be referred to
  as container objects.
\end{datadesc}

\begin{datadesc}{PyGC_HEAD_SIZE}
  Extra memory needed for the garbage collector.  Container objects
  must include this in the calculation of their tp_basicsize.  If the
  collector is disabled at compile time then this is \code{0}.
\end{datadesc}

Constructors for container types must conform to two rules:

\begin{enumerate}
\item  The memory for the object must be allocated using
       \cfunction{PyObject_New()} or \cfunction{PyObject_VarNew()}.

\item  Once all the fields which may contain references to other
       containers are initialized, it must call
       \cfunction{PyObject_GC_Init()}.
\end{enumerate}

\begin{cfuncdesc}{void}{PyObject_GC_Init}{PyObject *op}
  Adds the object \var{op} to the set of container objects tracked by
  the collector.  The collector can run at unexpected times so objects
  must be valid while being tracked.  This should be called once all
  the fields followed by the \member{tp_traverse} handler become valid,
  usually near the end of the constructor.
\end{cfuncdesc}

Similarly, the deallocator for the object must conform to a similar
pair of rules:

\begin{enumerate}
\item  Before fields which refer to other containers are invalidated,
       \cfunction{PyObject_GC_Fini()} must be called.

\item  The object's memory must be deallocated using
       \cfunction{PyObject_Del()}.
\end{enumerate}

\begin{cfuncdesc}{void}{PyObject_GC_Fini}{PyObject *op}
  Remove the object \var{op} from the set of container objects tracked
  by the collector.  Note that \cfunction{PyObject_GC_Init()} can be
  called again on this object to add it back to the set of tracked
  objects.  The deallocator (\member{tp_dealloc} handler) should call
  this for the object before any of the fields used by the
  \member{tp_traverse} handler become invalid.

  \strong{Note:}  Any container which may be referenced from another
  object reachable by the collector must itself be tracked by the
  collector, so it is generally not safe to call this function
  anywhere but in the object's deallocator.
\end{cfuncdesc}

The \member{tp_traverse} handler accepts a function parameter of this
type:

\begin{ctypedesc}[visitproc]{int (*visitproc)(PyObject *object, void *arg)}
  Type of the visitor function passed to the \member{tp_traverse}
  handler.  The function should be called with an object to traverse
  as \var{object} and the third parameter to the \member{tp_traverse}
  handler as \var{arg}.
\end{ctypedesc}

The \member{tp_traverse} handler must have the following type:

\begin{ctypedesc}[traverseproc]{int (*traverseproc)(PyObject *self,
                                visitproc visit, void *arg)}
  Traversal function for a container object.  Implementations must
  call the \var{visit} function for each object directly contained by
  \var{self}, with the parameters to \var{visit} being the contained
  object and the \var{arg} value passed to the handler.  If
  \var{visit} returns a non-zero value then an error has occurred and
  that value should be returned immediately.
\end{ctypedesc}

The \member{tp_clear} handler must be of the \ctype{inquiry} type, or
\NULL{} if the object is immutable.

\begin{ctypedesc}[inquiry]{int (*inquiry)(PyObject *self)}
  Drop references that may have created reference cycles.  Immutable
  objects do not have to define this method since they can never
  directly create reference cycles.  Note that the object must still
  be valid after calling this method (i.e., don't just call
  \cfunction{Py_DECREF()} on a reference).  The collector will call
  this method if it detects that this object is involved in a
  reference cycle.
\end{ctypedesc}


\subsection{Example Cycle Collector Support
            \label{example-cycle-support}}

This example shows only enough of the implementation of an extension
type to show how the garbage collector support needs to be added.  It
shows the definition of the object structure, the
\member{tp_traverse}, \member{tp_clear} and \member{tp_dealloc}
implementations, the type structure, and a constructor --- the module
initialization needed to export the constructor to Python is not shown
as there are no special considerations there for the collector.  To
make this interesting, assume that the module exposes ways for the
\member{container} field of the object to be modified.  Note that
since no checks are made on the type of the object used to initialize
\member{container}, we have to assume that it may be a container.

\begin{verbatim}
#include "Python.h"

typedef struct {
    PyObject_HEAD
    PyObject *container;
} MyObject;

static int
my_traverse(MyObject *self, visitproc visit, void *arg)
{
    if (self->container != NULL)
        return visit(self->container, arg);
    else
        return 0;
}

static int
my_clear(MyObject *self)
{
    Py_XDECREF(self->container);
    self->container = NULL;

    return 0;
}

static void
my_dealloc(MyObject *self)
{
    PyObject_GC_Fini((PyObject *) self);
    Py_XDECREF(self->container);
    PyObject_Del(self);
}
\end{verbatim}

\begin{verbatim}
statichere PyTypeObject
MyObject_Type = {
    PyObject_HEAD_INIT(NULL)
    0,
    "MyObject",
    sizeof(MyObject) + PyGC_HEAD_SIZE,
    0,
    (destructor)my_dealloc,     /* tp_dealloc */
    0,                          /* tp_print */
    0,                          /* tp_getattr */
    0,                          /* tp_setattr */
    0,                          /* tp_compare */
    0,                          /* tp_repr */
    0,                          /* tp_as_number */
    0,                          /* tp_as_sequence */
    0,                          /* tp_as_mapping */
    0,                          /* tp_hash */
    0,                          /* tp_call */
    0,                          /* tp_str */
    0,                          /* tp_getattro */
    0,                          /* tp_setattro */
    0,                          /* tp_as_buffer */
    Py_TPFLAGS_DEFAULT | Py_TPFLAGS_GC,
    0,                          /* tp_doc */
    (traverseproc)my_traverse,  /* tp_traverse */
    (inquiry)my_clear,          /* tp_clear */
    0,                          /* tp_richcompare */
    0,                          /* tp_weaklistoffset */
};

/* This constructor should be made accessible from Python. */
static PyObject *
new_object(PyObject *unused, PyObject *args)
{
    PyObject *container = NULL;
    MyObject *result = NULL;

    if (PyArg_ParseTuple(args, "|O:new_object", &container)) {
        result = PyObject_New(MyObject, &MyObject_Type);
        if (result != NULL) {
            result->container = container;
            PyObject_GC_Init();
        }
    }
    return (PyObject *) result;
}
\end{verbatim}


% \chapter{Debugging \label{debugging}}
%
% XXX Explain Py_DEBUG, Py_TRACE_REFS, Py_REF_DEBUG.


\appendix
\chapter{Reporting Bugs}
\label{reporting-bugs}

Python is a mature programming language which has established a
reputation for stability.  In order to maintain this reputation, the
developers would like to know of any deficiencies you find in Python
or its documentation.

All bug reports should be submitted via the Python Bug Tracker on
SourceForge (\url{http://sourceforge.net/bugs/?group_id=5470}).  The
bug tracker offers a Web form which allows pertinent information to be
entered and submitted to the developers.

Before submitting a report, please log into SourceForge if you are a
member; this will make it possible for the developers to contact you
for additional information if needed.  If you are not a SourceForge
member but would not mind the developers contacting you, you may
include your email address in your bug description.  In this case,
please realize that the information is publically available and cannot
be protected.

The first step in filing a report is to determine whether the problem
has already been reported.  The advantage in doing so, aside from
saving the developers time, is that you learn what has been done to
fix it; it may be that the problem has already been fixed for the next
release, or additional information is needed (in which case you are
welcome to provide it if you can!).  To do this, search the bug
database using the search box near the bottom of the page.

If the problem you're reporting is not already in the bug tracker, go
back to the Python Bug Tracker
(\url{http://sourceforge.net/bugs/?group_id=5470}).  Select the
``Submit a Bug'' link at the top of the page to open the bug reporting
form.

The submission form has a number of fields.  The only fields that are
required are the ``Summary'' and ``Details'' fields.  For the summary,
enter a \emph{very} short description of the problem; less than ten
words is good.  In the Details field, describe the problem in detail,
including what you expected to happen and what did happen.  Be sure to
include the version of Python you used, whether any extension modules
were involved, and what hardware and software platform you were using
(including version information as appropriate).

The only other field that you may want to set is the ``Category''
field, which allows you to place the bug report into a broad category
(such as ``Documentation'' or ``Library'').

Each bug report will be assigned to a developer who will determine
what needs to be done to correct the problem.  If you have a
SourceForge account and logged in to report the problem, you will
receive an update each time action is taken on the bug.


\begin{seealso}
  \seetitle[http://www-mice.cs.ucl.ac.uk/multimedia/software/documentation/ReportingBugs.html]{How
        to Report Bugs Effectively}{Article which goes into some
        detail about how to create a useful bug report.  This
        describes what kind of information is useful and why it is
        useful.}

  \seetitle[http://www.mozilla.org/quality/bug-writing-guidelines.html]{Bug
        Writing Guidelines}{Information about writing a good bug
        report.  Some of this is specific to the Mozilla project, but
        describes general good practices.}
\end{seealso}


\documentclass{manual}

\title{Python/C API Reference Manual}

\input{boilerplate}

\makeindex			% tell \index to actually write the .idx file


\begin{document}

\maketitle

\ifhtml
\chapter*{Front Matter\label{front}}
\fi

\input{copyright}

\begin{abstract}

\noindent
This manual documents the API used by C and \Cpp{} programmers who
want to write extension modules or embed Python.  It is a companion to
\citetitle[../ext/ext.html]{Extending and Embedding the Python
Interpreter}, which describes the general principles of extension
writing but does not document the API functions in detail.

\strong{Warning:} The current version of this document is incomplete.
I hope that it is nevertheless useful.  I will continue to work on it,
and release new versions from time to time, independent from Python
source code releases.

\end{abstract}

\tableofcontents

% XXX Consider moving all this back to ext.tex and giving api.tex
% XXX a *really* short intro only.

\chapter{Introduction \label{intro}}

The Application Programmer's Interface to Python gives C and
\Cpp{} programmers access to the Python interpreter at a variety of
levels.  The API is equally usable from \Cpp{}, but for brevity it is
generally referred to as the Python/C API.  There are two
fundamentally different reasons for using the Python/C API.  The first
reason is to write \emph{extension modules} for specific purposes;
these are C modules that extend the Python interpreter.  This is
probably the most common use.  The second reason is to use Python as a
component in a larger application; this technique is generally
referred to as \dfn{embedding} Python in an application.

Writing an extension module is a relatively well-understood process, 
where a ``cookbook'' approach works well.  There are several tools 
that automate the process to some extent.  While people have embedded 
Python in other applications since its early existence, the process of 
embedding Python is less straightforward than writing an extension.  

Many API functions are useful independent of whether you're embedding 
or extending Python; moreover, most applications that embed Python 
will need to provide a custom extension as well, so it's probably a 
good idea to become familiar with writing an extension before 
attempting to embed Python in a real application.


\section{Include Files \label{includes}}

All function, type and macro definitions needed to use the Python/C
API are included in your code by the following line:

\begin{verbatim}
#include "Python.h"
\end{verbatim}

This implies inclusion of the following standard headers:
\code{<stdio.h>}, \code{<string.h>}, \code{<errno.h>},
\code{<limits.h>}, and \code{<stdlib.h>} (if available).

All user visible names defined by Python.h (except those defined by
the included standard headers) have one of the prefixes \samp{Py} or
\samp{_Py}.  Names beginning with \samp{_Py} are for internal use by
the Python implementation and should not be used by extension writers.
Structure member names do not have a reserved prefix.

\strong{Important:} user code should never define names that begin
with \samp{Py} or \samp{_Py}.  This confuses the reader, and
jeopardizes the portability of the user code to future Python
versions, which may define additional names beginning with one of
these prefixes.

The header files are typically installed with Python.  On \UNIX, these 
are located in the directories
\file{\envvar{prefix}/include/python\var{version}/} and
\file{\envvar{exec_prefix}/include/python\var{version}/}, where
\envvar{prefix} and \envvar{exec_prefix} are defined by the
corresponding parameters to Python's \program{configure} script and
\var{version} is \code{sys.version[:3]}.  On Windows, the headers are
installed in \file{\envvar{prefix}/include}, where \envvar{prefix} is
the installation directory specified to the installer.

To include the headers, place both directories (if different) on your
compiler's search path for includes.  Do \emph{not} place the parent
directories on the search path and then use
\samp{\#include <python\shortversion/Python.h>}; this will break on
multi-platform builds since the platform independent headers under
\envvar{prefix} include the platform specific headers from
\envvar{exec_prefix}.


\section{Objects, Types and Reference Counts \label{objects}}

Most Python/C API functions have one or more arguments as well as a
return value of type \ctype{PyObject*}.  This type is a pointer
to an opaque data type representing an arbitrary Python
object.  Since all Python object types are treated the same way by the
Python language in most situations (e.g., assignments, scope rules,
and argument passing), it is only fitting that they should be
represented by a single C type.  Almost all Python objects live on the
heap: you never declare an automatic or static variable of type
\ctype{PyObject}, only pointer variables of type \ctype{PyObject*} can 
be declared.  The sole exception are the type objects\obindex{type};
since these must never be deallocated, they are typically static
\ctype{PyTypeObject} objects.

All Python objects (even Python integers) have a \dfn{type} and a
\dfn{reference count}.  An object's type determines what kind of object 
it is (e.g., an integer, a list, or a user-defined function; there are 
many more as explained in the \citetitle[../ref/ref.html]{Python
Reference Manual}).  For each of the well-known types there is a macro
to check whether an object is of that type; for instance,
\samp{PyList_Check(\var{a})} is true if (and only if) the object
pointed to by \var{a} is a Python list.


\subsection{Reference Counts \label{refcounts}}

The reference count is important because today's computers have a 
finite (and often severely limited) memory size; it counts how many 
different places there are that have a reference to an object.  Such a 
place could be another object, or a global (or static) C variable, or 
a local variable in some C function.  When an object's reference count 
becomes zero, the object is deallocated.  If it contains references to 
other objects, their reference count is decremented.  Those other 
objects may be deallocated in turn, if this decrement makes their 
reference count become zero, and so on.  (There's an obvious problem 
with objects that reference each other here; for now, the solution is 
``don't do that.'')

Reference counts are always manipulated explicitly.  The normal way is 
to use the macro \cfunction{Py_INCREF()}\ttindex{Py_INCREF()} to
increment an object's reference count by one, and
\cfunction{Py_DECREF()}\ttindex{Py_DECREF()} to decrement it by  
one.  The \cfunction{Py_DECREF()} macro is considerably more complex
than the incref one, since it must check whether the reference count
becomes zero and then cause the object's deallocator to be called.
The deallocator is a function pointer contained in the object's type
structure.  The type-specific deallocator takes care of decrementing
the reference counts for other objects contained in the object if this
is a compound object type, such as a list, as well as performing any
additional finalization that's needed.  There's no chance that the
reference count can overflow; at least as many bits are used to hold
the reference count as there are distinct memory locations in virtual
memory (assuming \code{sizeof(long) >= sizeof(char*)}).  Thus, the
reference count increment is a simple operation.

It is not necessary to increment an object's reference count for every 
local variable that contains a pointer to an object.  In theory, the 
object's reference count goes up by one when the variable is made to 
point to it and it goes down by one when the variable goes out of 
scope.  However, these two cancel each other out, so at the end the 
reference count hasn't changed.  The only real reason to use the 
reference count is to prevent the object from being deallocated as 
long as our variable is pointing to it.  If we know that there is at 
least one other reference to the object that lives at least as long as 
our variable, there is no need to increment the reference count 
temporarily.  An important situation where this arises is in objects 
that are passed as arguments to C functions in an extension module 
that are called from Python; the call mechanism guarantees to hold a 
reference to every argument for the duration of the call.

However, a common pitfall is to extract an object from a list and
hold on to it for a while without incrementing its reference count.
Some other operation might conceivably remove the object from the
list, decrementing its reference count and possible deallocating it.
The real danger is that innocent-looking operations may invoke
arbitrary Python code which could do this; there is a code path which
allows control to flow back to the user from a \cfunction{Py_DECREF()},
so almost any operation is potentially dangerous.

A safe approach is to always use the generic operations (functions 
whose name begins with \samp{PyObject_}, \samp{PyNumber_},
\samp{PySequence_} or \samp{PyMapping_}).  These operations always
increment the reference count of the object they return.  This leaves
the caller with the responsibility to call
\cfunction{Py_DECREF()} when they are done with the result; this soon
becomes second nature.


\subsubsection{Reference Count Details \label{refcountDetails}}

The reference count behavior of functions in the Python/C API is best 
explained in terms of \emph{ownership of references}.  Note that we 
talk of owning references, never of owning objects; objects are always 
shared!  When a function owns a reference, it has to dispose of it 
properly --- either by passing ownership on (usually to its caller) or 
by calling \cfunction{Py_DECREF()} or \cfunction{Py_XDECREF()}.  When
a function passes ownership of a reference on to its caller, the
caller is said to receive a \emph{new} reference.  When no ownership
is transferred, the caller is said to \emph{borrow} the reference.
Nothing needs to be done for a borrowed reference.

Conversely, when a calling function passes it a reference to an 
object, there are two possibilities: the function \emph{steals} a 
reference to the object, or it does not.  Few functions steal 
references; the two notable exceptions are
\cfunction{PyList_SetItem()}\ttindex{PyList_SetItem()} and
\cfunction{PyTuple_SetItem()}\ttindex{PyTuple_SetItem()}, which 
steal a reference to the item (but not to the tuple or list into which
the item is put!).  These functions were designed to steal a reference
because of a common idiom for populating a tuple or list with newly
created objects; for example, the code to create the tuple \code{(1,
2, "three")} could look like this (forgetting about error handling for
the moment; a better way to code this is shown below):

\begin{verbatim}
PyObject *t;

t = PyTuple_New(3);
PyTuple_SetItem(t, 0, PyInt_FromLong(1L));
PyTuple_SetItem(t, 1, PyInt_FromLong(2L));
PyTuple_SetItem(t, 2, PyString_FromString("three"));
\end{verbatim}

Incidentally, \cfunction{PyTuple_SetItem()} is the \emph{only} way to
set tuple items; \cfunction{PySequence_SetItem()} and
\cfunction{PyObject_SetItem()} refuse to do this since tuples are an
immutable data type.  You should only use
\cfunction{PyTuple_SetItem()} for tuples that you are creating
yourself.

Equivalent code for populating a list can be written using 
\cfunction{PyList_New()} and \cfunction{PyList_SetItem()}.  Such code
can also use \cfunction{PySequence_SetItem()}; this illustrates the
difference between the two (the extra \cfunction{Py_DECREF()} calls):

\begin{verbatim}
PyObject *l, *x;

l = PyList_New(3);
x = PyInt_FromLong(1L);
PySequence_SetItem(l, 0, x); Py_DECREF(x);
x = PyInt_FromLong(2L);
PySequence_SetItem(l, 1, x); Py_DECREF(x);
x = PyString_FromString("three");
PySequence_SetItem(l, 2, x); Py_DECREF(x);
\end{verbatim}

You might find it strange that the ``recommended'' approach takes more
code.  However, in practice, you will rarely use these ways of
creating and populating a tuple or list.  There's a generic function,
\cfunction{Py_BuildValue()}, that can create most common objects from
C values, directed by a \dfn{format string}.  For example, the
above two blocks of code could be replaced by the following (which
also takes care of the error checking):

\begin{verbatim}
PyObject *t, *l;

t = Py_BuildValue("(iis)", 1, 2, "three");
l = Py_BuildValue("[iis]", 1, 2, "three");
\end{verbatim}

It is much more common to use \cfunction{PyObject_SetItem()} and
friends with items whose references you are only borrowing, like
arguments that were passed in to the function you are writing.  In
that case, their behaviour regarding reference counts is much saner,
since you don't have to increment a reference count so you can give a
reference away (``have it be stolen'').  For example, this function
sets all items of a list (actually, any mutable sequence) to a given
item:

\begin{verbatim}
int set_all(PyObject *target, PyObject *item)
{
    int i, n;

    n = PyObject_Length(target);
    if (n < 0)
        return -1;
    for (i = 0; i < n; i++) {
        if (PyObject_SetItem(target, i, item) < 0)
            return -1;
    }
    return 0;
}
\end{verbatim}
\ttindex{set_all()}

The situation is slightly different for function return values.  
While passing a reference to most functions does not change your 
ownership responsibilities for that reference, many functions that 
return a referece to an object give you ownership of the reference.
The reason is simple: in many cases, the returned object is created 
on the fly, and the reference you get is the only reference to the 
object.  Therefore, the generic functions that return object 
references, like \cfunction{PyObject_GetItem()} and 
\cfunction{PySequence_GetItem()}, always return a new reference (i.e.,
the  caller becomes the owner of the reference).

It is important to realize that whether you own a reference returned 
by a function depends on which function you call only --- \emph{the
plumage} (i.e., the type of the type of the object passed as an
argument to the function) \emph{doesn't enter into it!}  Thus, if you 
extract an item from a list using \cfunction{PyList_GetItem()}, you
don't own the reference --- but if you obtain the same item from the
same list using \cfunction{PySequence_GetItem()} (which happens to
take exactly the same arguments), you do own a reference to the
returned object.

Here is an example of how you could write a function that computes the
sum of the items in a list of integers; once using 
\cfunction{PyList_GetItem()}\ttindex{PyList_GetItem()}, and once using
\cfunction{PySequence_GetItem()}\ttindex{PySequence_GetItem()}.

\begin{verbatim}
long sum_list(PyObject *list)
{
    int i, n;
    long total = 0;
    PyObject *item;

    n = PyList_Size(list);
    if (n < 0)
        return -1; /* Not a list */
    for (i = 0; i < n; i++) {
        item = PyList_GetItem(list, i); /* Can't fail */
        if (!PyInt_Check(item)) continue; /* Skip non-integers */
        total += PyInt_AsLong(item);
    }
    return total;
}
\end{verbatim}
\ttindex{sum_list()}

\begin{verbatim}
long sum_sequence(PyObject *sequence)
{
    int i, n;
    long total = 0;
    PyObject *item;
    n = PySequence_Length(sequence);
    if (n < 0)
        return -1; /* Has no length */
    for (i = 0; i < n; i++) {
        item = PySequence_GetItem(sequence, i);
        if (item == NULL)
            return -1; /* Not a sequence, or other failure */
        if (PyInt_Check(item))
            total += PyInt_AsLong(item);
        Py_DECREF(item); /* Discard reference ownership */
    }
    return total;
}
\end{verbatim}
\ttindex{sum_sequence()}


\subsection{Types \label{types}}

There are few other data types that play a significant role in 
the Python/C API; most are simple C types such as \ctype{int}, 
\ctype{long}, \ctype{double} and \ctype{char*}.  A few structure types 
are used to describe static tables used to list the functions exported 
by a module or the data attributes of a new object type, and another
is used to describe the value of a complex number.  These will 
be discussed together with the functions that use them.


\section{Exceptions \label{exceptions}}

The Python programmer only needs to deal with exceptions if specific 
error handling is required; unhandled exceptions are automatically 
propagated to the caller, then to the caller's caller, and so on, until
they reach the top-level interpreter, where they are reported to the 
user accompanied by a stack traceback.

For C programmers, however, error checking always has to be explicit.  
All functions in the Python/C API can raise exceptions, unless an 
explicit claim is made otherwise in a function's documentation.  In 
general, when a function encounters an error, it sets an exception, 
discards any object references that it owns, and returns an 
error indicator --- usually \NULL{} or \code{-1}.  A few functions 
return a Boolean true/false result, with false indicating an error.
Very few functions return no explicit error indicator or have an 
ambiguous return value, and require explicit testing for errors with 
\cfunction{PyErr_Occurred()}\ttindex{PyErr_Occurred()}.

Exception state is maintained in per-thread storage (this is 
equivalent to using global storage in an unthreaded application).  A 
thread can be in one of two states: an exception has occurred, or not.
The function \cfunction{PyErr_Occurred()} can be used to check for
this: it returns a borrowed reference to the exception type object
when an exception has occurred, and \NULL{} otherwise.  There are a
number of functions to set the exception state:
\cfunction{PyErr_SetString()}\ttindex{PyErr_SetString()} is the most
common (though not the most general) function to set the exception
state, and \cfunction{PyErr_Clear()}\ttindex{PyErr_Clear()} clears the
exception state.

The full exception state consists of three objects (all of which can 
be \NULL{}): the exception type, the corresponding exception 
value, and the traceback.  These have the same meanings as the Python
\withsubitem{(in module sys)}{
  \ttindex{exc_type}\ttindex{exc_value}\ttindex{exc_traceback}}
objects \code{sys.exc_type}, \code{sys.exc_value}, and
\code{sys.exc_traceback}; however, they are not the same: the Python
objects represent the last exception being handled by a Python 
\keyword{try} \ldots\ \keyword{except} statement, while the C level
exception state only exists while an exception is being passed on
between C functions until it reaches the Python bytecode interpreter's 
main loop, which takes care of transferring it to \code{sys.exc_type}
and friends.

Note that starting with Python 1.5, the preferred, thread-safe way to 
access the exception state from Python code is to call the function
\withsubitem{(in module sys)}{\ttindex{exc_info()}}
\function{sys.exc_info()}, which returns the per-thread exception state 
for Python code.  Also, the semantics of both ways to access the 
exception state have changed so that a function which catches an 
exception will save and restore its thread's exception state so as to 
preserve the exception state of its caller.  This prevents common bugs 
in exception handling code caused by an innocent-looking function 
overwriting the exception being handled; it also reduces the often 
unwanted lifetime extension for objects that are referenced by the 
stack frames in the traceback.

As a general principle, a function that calls another function to 
perform some task should check whether the called function raised an 
exception, and if so, pass the exception state on to its caller.  It 
should discard any object references that it owns, and return an 
error indicator, but it should \emph{not} set another exception ---
that would overwrite the exception that was just raised, and lose
important information about the exact cause of the error.

A simple example of detecting exceptions and passing them on is shown
in the \cfunction{sum_sequence()}\ttindex{sum_sequence()} example
above.  It so happens that that example doesn't need to clean up any
owned references when it detects an error.  The following example
function shows some error cleanup.  First, to remind you why you like
Python, we show the equivalent Python code:

\begin{verbatim}
def incr_item(dict, key):
    try:
        item = dict[key]
    except KeyError:
        item = 0
    dict[key] = item + 1
\end{verbatim}
\ttindex{incr_item()}

Here is the corresponding C code, in all its glory:

\begin{verbatim}
int incr_item(PyObject *dict, PyObject *key)
{
    /* Objects all initialized to NULL for Py_XDECREF */
    PyObject *item = NULL, *const_one = NULL, *incremented_item = NULL;
    int rv = -1; /* Return value initialized to -1 (failure) */

    item = PyObject_GetItem(dict, key);
    if (item == NULL) {
        /* Handle KeyError only: */
        if (!PyErr_ExceptionMatches(PyExc_KeyError))
            goto error;

        /* Clear the error and use zero: */
        PyErr_Clear();
        item = PyInt_FromLong(0L);
        if (item == NULL)
            goto error;
    }
    const_one = PyInt_FromLong(1L);
    if (const_one == NULL)
        goto error;

    incremented_item = PyNumber_Add(item, const_one);
    if (incremented_item == NULL)
        goto error;

    if (PyObject_SetItem(dict, key, incremented_item) < 0)
        goto error;
    rv = 0; /* Success */
    /* Continue with cleanup code */

 error:
    /* Cleanup code, shared by success and failure path */

    /* Use Py_XDECREF() to ignore NULL references */
    Py_XDECREF(item);
    Py_XDECREF(const_one);
    Py_XDECREF(incremented_item);

    return rv; /* -1 for error, 0 for success */
}
\end{verbatim}
\ttindex{incr_item()}

This example represents an endorsed use of the \keyword{goto} statement 
in C!  It illustrates the use of
\cfunction{PyErr_ExceptionMatches()}\ttindex{PyErr_ExceptionMatches()} and
\cfunction{PyErr_Clear()}\ttindex{PyErr_Clear()} to
handle specific exceptions, and the use of
\cfunction{Py_XDECREF()}\ttindex{Py_XDECREF()} to
dispose of owned references that may be \NULL{} (note the
\character{X} in the name; \cfunction{Py_DECREF()} would crash when
confronted with a \NULL{} reference).  It is important that the
variables used to hold owned references are initialized to \NULL{} for
this to work; likewise, the proposed return value is initialized to
\code{-1} (failure) and only set to success after the final call made
is successful.


\section{Embedding Python \label{embedding}}

The one important task that only embedders (as opposed to extension
writers) of the Python interpreter have to worry about is the
initialization, and possibly the finalization, of the Python
interpreter.  Most functionality of the interpreter can only be used
after the interpreter has been initialized.

The basic initialization function is
\cfunction{Py_Initialize()}\ttindex{Py_Initialize()}.
This initializes the table of loaded modules, and creates the
fundamental modules \module{__builtin__}\refbimodindex{__builtin__},
\module{__main__}\refbimodindex{__main__} and 
\module{sys}\refbimodindex{sys}.  It also initializes the module
search path (\code{sys.path}).%
\indexiii{module}{search}{path}
\withsubitem{(in module sys)}{\ttindex{path}}

\cfunction{Py_Initialize()} does not set the ``script argument list'' 
(\code{sys.argv}).  If this variable is needed by Python code that 
will be executed later, it must be set explicitly with a call to 
\code{PySys_SetArgv(\var{argc},
\var{argv})}\ttindex{PySys_SetArgv()} subsequent to the call to
\cfunction{Py_Initialize()}.

On most systems (in particular, on \UNIX{} and Windows, although the
details are slightly different),
\cfunction{Py_Initialize()} calculates the module search path based
upon its best guess for the location of the standard Python
interpreter executable, assuming that the Python library is found in a
fixed location relative to the Python interpreter executable.  In
particular, it looks for a directory named
\file{lib/python\shortversion} relative to the parent directory where
the executable named \file{python} is found on the shell command
search path (the environment variable \envvar{PATH}).

For instance, if the Python executable is found in
\file{/usr/local/bin/python}, it will assume that the libraries are in
\file{/usr/local/lib/python\shortversion}.  (In fact, this particular path
is also the ``fallback'' location, used when no executable file named
\file{python} is found along \envvar{PATH}.)  The user can override
this behavior by setting the environment variable \envvar{PYTHONHOME},
or insert additional directories in front of the standard path by
setting \envvar{PYTHONPATH}.

The embedding application can steer the search by calling 
\code{Py_SetProgramName(\var{file})}\ttindex{Py_SetProgramName()} \emph{before} calling 
\cfunction{Py_Initialize()}.  Note that \envvar{PYTHONHOME} still
overrides this and \envvar{PYTHONPATH} is still inserted in front of
the standard path.  An application that requires total control has to
provide its own implementation of
\cfunction{Py_GetPath()}\ttindex{Py_GetPath()},
\cfunction{Py_GetPrefix()}\ttindex{Py_GetPrefix()},
\cfunction{Py_GetExecPrefix()}\ttindex{Py_GetExecPrefix()}, and
\cfunction{Py_GetProgramFullPath()}\ttindex{Py_GetProgramFullPath()} (all
defined in \file{Modules/getpath.c}).

Sometimes, it is desirable to ``uninitialize'' Python.  For instance, 
the application may want to start over (make another call to 
\cfunction{Py_Initialize()}) or the application is simply done with its 
use of Python and wants to free all memory allocated by Python.  This
can be accomplished by calling \cfunction{Py_Finalize()}.  The function
\cfunction{Py_IsInitialized()}\ttindex{Py_IsInitialized()} returns
true if Python is currently in the initialized state.  More
information about these functions is given in a later chapter.


\chapter{The Very High Level Layer \label{veryhigh}}

The functions in this chapter will let you execute Python source code
given in a file or a buffer, but they will not let you interact in a
more detailed way with the interpreter.

Several of these functions accept a start symbol from the grammar as a 
parameter.  The available start symbols are \constant{Py_eval_input},
\constant{Py_file_input}, and \constant{Py_single_input}.  These are
described following the functions which accept them as parameters.

Note also that several of these functions take \ctype{FILE*}
parameters.  On particular issue which needs to be handled carefully
is that the \ctype{FILE} structure for different C libraries can be
different and incompatible.  Under Windows (at least), it is possible
for dynamically linked extensions to actually use different libraries,
so care should be taken that \ctype{FILE*} parameters are only passed
to these functions if it is certain that they were created by the same
library that the Python runtime is using.

\begin{cfuncdesc}{int}{PyRun_AnyFile}{FILE *fp, char *filename}
  If \var{fp} refers to a file associated with an interactive device
  (console or terminal input or \UNIX{} pseudo-terminal), return the
  value of \cfunction{PyRun_InteractiveLoop()}, otherwise return the
  result of \cfunction{PyRun_SimpleFile()}.  If \var{filename} is
  \NULL{}, this function uses \code{"???"} as the filename.
\end{cfuncdesc}

\begin{cfuncdesc}{int}{PyRun_SimpleString}{char *command}
  Executes the Python source code from \var{command} in the
  \module{__main__} module.  If \module{__main__} does not already
  exist, it is created.  Returns \code{0} on success or \code{-1} if
  an exception was raised.  If there was an error, there is no way to
  get the exception information.
\end{cfuncdesc}

\begin{cfuncdesc}{int}{PyRun_SimpleFile}{FILE *fp, char *filename}
  Similar to \cfunction{PyRun_SimpleString()}, but the Python source
  code is read from \var{fp} instead of an in-memory string.
  \var{filename} should be the name of the file.
\end{cfuncdesc}

\begin{cfuncdesc}{int}{PyRun_InteractiveOne}{FILE *fp, char *filename}
  Read and execute a single statement from a file associated with an
  interactive device.  If \var{filename} is \NULL, \code{"???"} is
  used instead.  The user will be prompted using \code{sys.ps1} and
  \code{sys.ps2}.  Returns \code{0} when the input was executed
  successfully, \code{-1} if there was an exception, or an error code
  from the \file{errcode.h} include file distributed as part of Python
  in case of a parse error.  (Note that \file{errcode.h} is not
  included by \file{Python.h}, so must be included specifically if
  needed.)
\end{cfuncdesc}

\begin{cfuncdesc}{int}{PyRun_InteractiveLoop}{FILE *fp, char *filename}
  Read and execute statements from a file associated with an
  interactive device until \EOF{} is reached.  If \var{filename} is
  \NULL, \code{"???"} is used instead.  The user will be prompted
  using \code{sys.ps1} and \code{sys.ps2}.  Returns \code{0} at \EOF.
\end{cfuncdesc}

\begin{cfuncdesc}{struct _node*}{PyParser_SimpleParseString}{char *str,
                                                             int start}
  Parse Python source code from \var{str} using the start token
  \var{start}.  The result can be used to create a code object which
  can be evaluated efficiently.  This is useful if a code fragment
  must be evaluated many times.
\end{cfuncdesc}

\begin{cfuncdesc}{struct _node*}{PyParser_SimpleParseFile}{FILE *fp,
                                 char *filename, int start}
  Similar to \cfunction{PyParser_SimpleParseString()}, but the Python
  source code is read from \var{fp} instead of an in-memory string.
  \var{filename} should be the name of the file.
\end{cfuncdesc}

\begin{cfuncdesc}{PyObject*}{PyRun_String}{char *str, int start,
                                           PyObject *globals,
                                           PyObject *locals}
  Execute Python source code from \var{str} in the context specified
  by the dictionaries \var{globals} and \var{locals}.  The parameter
  \var{start} specifies the start token that should be used to parse
  the source code.

  Returns the result of executing the code as a Python object, or
  \NULL{} if an exception was raised.
\end{cfuncdesc}

\begin{cfuncdesc}{PyObject*}{PyRun_File}{FILE *fp, char *filename,
                                         int start, PyObject *globals,
                                         PyObject *locals}
  Similar to \cfunction{PyRun_String()}, but the Python source code is 
  read from \var{fp} instead of an in-memory string.
  \var{filename} should be the name of the file.
\end{cfuncdesc}

\begin{cfuncdesc}{PyObject*}{Py_CompileString}{char *str, char *filename,
                                               int start}
  Parse and compile the Python source code in \var{str}, returning the 
  resulting code object.  The start token is given by \var{start};
  this can be used to constrain the code which can be compiled and should
  be \constant{Py_eval_input}, \constant{Py_file_input}, or
  \constant{Py_single_input}.  The filename specified by
  \var{filename} is used to construct the code object and may appear
  in tracebacks or \exception{SyntaxError} exception messages.  This
  returns \NULL{} if the code cannot be parsed or compiled.
\end{cfuncdesc}

\begin{cvardesc}{int}{Py_eval_input}
  The start symbol from the Python grammar for isolated expressions;
  for use with \cfunction{Py_CompileString()}\ttindex{Py_CompileString()}.
\end{cvardesc}

\begin{cvardesc}{int}{Py_file_input}
  The start symbol from the Python grammar for sequences of statements
  as read from a file or other source; for use with
  \cfunction{Py_CompileString()}\ttindex{Py_CompileString()}.  This is
  the symbol to use when compiling arbitrarily long Python source code.
\end{cvardesc}

\begin{cvardesc}{int}{Py_single_input}
  The start symbol from the Python grammar for a single statement; for 
  use with \cfunction{Py_CompileString()}\ttindex{Py_CompileString()}.
  This is the symbol used for the interactive interpreter loop.
\end{cvardesc}


\chapter{Reference Counting \label{countingRefs}}

The macros in this section are used for managing reference counts
of Python objects.

\begin{cfuncdesc}{void}{Py_INCREF}{PyObject *o}
Increment the reference count for object \var{o}.  The object must
not be \NULL{}; if you aren't sure that it isn't \NULL{}, use
\cfunction{Py_XINCREF()}.
\end{cfuncdesc}

\begin{cfuncdesc}{void}{Py_XINCREF}{PyObject *o}
Increment the reference count for object \var{o}.  The object may be
\NULL{}, in which case the macro has no effect.
\end{cfuncdesc}

\begin{cfuncdesc}{void}{Py_DECREF}{PyObject *o}
Decrement the reference count for object \var{o}.  The object must
not be \NULL{}; if you aren't sure that it isn't \NULL{}, use
\cfunction{Py_XDECREF()}.  If the reference count reaches zero, the
object's type's deallocation function (which must not be \NULL{}) is
invoked.

\strong{Warning:} The deallocation function can cause arbitrary Python
code to be invoked (e.g. when a class instance with a
\method{__del__()} method is deallocated).  While exceptions in such
code are not propagated, the executed code has free access to all
Python global variables.  This means that any object that is reachable
from a global variable should be in a consistent state before
\cfunction{Py_DECREF()} is invoked.  For example, code to delete an
object from a list should copy a reference to the deleted object in a
temporary variable, update the list data structure, and then call
\cfunction{Py_DECREF()} for the temporary variable.
\end{cfuncdesc}

\begin{cfuncdesc}{void}{Py_XDECREF}{PyObject *o}
Decrement the reference count for object \var{o}.  The object may be
\NULL{}, in which case the macro has no effect; otherwise the effect
is the same as for \cfunction{Py_DECREF()}, and the same warning
applies.
\end{cfuncdesc}

The following functions or macros are only for use within the
interpreter core: \cfunction{_Py_Dealloc()},
\cfunction{_Py_ForgetReference()}, \cfunction{_Py_NewReference()}, as
well as the global variable \cdata{_Py_RefTotal}.


\chapter{Exception Handling \label{exceptionHandling}}

The functions described in this chapter will let you handle and raise Python
exceptions.  It is important to understand some of the basics of
Python exception handling.  It works somewhat like the
\UNIX{} \cdata{errno} variable: there is a global indicator (per
thread) of the last error that occurred.  Most functions don't clear
this on success, but will set it to indicate the cause of the error on
failure.  Most functions also return an error indicator, usually
\NULL{} if they are supposed to return a pointer, or \code{-1} if they
return an integer (exception: the \cfunction{PyArg_Parse*()} functions
return \code{1} for success and \code{0} for failure).  When a
function must fail because some function it called failed, it
generally doesn't set the error indicator; the function it called
already set it.

The error indicator consists of three Python objects corresponding to
\withsubitem{(in module sys)}{
  \ttindex{exc_type}\ttindex{exc_value}\ttindex{exc_traceback}}
the Python variables \code{sys.exc_type}, \code{sys.exc_value} and
\code{sys.exc_traceback}.  API functions exist to interact with the
error indicator in various ways.  There is a separate error indicator
for each thread.

% XXX Order of these should be more thoughtful.
% Either alphabetical or some kind of structure.

\begin{cfuncdesc}{void}{PyErr_Print}{}
Print a standard traceback to \code{sys.stderr} and clear the error
indicator.  Call this function only when the error indicator is set.
(Otherwise it will cause a fatal error!)
\end{cfuncdesc}

\begin{cfuncdesc}{PyObject*}{PyErr_Occurred}{}
Test whether the error indicator is set.  If set, return the exception
\emph{type} (the first argument to the last call to one of the
\cfunction{PyErr_Set*()} functions or to \cfunction{PyErr_Restore()}).  If
not set, return \NULL{}.  You do not own a reference to the return
value, so you do not need to \cfunction{Py_DECREF()} it.
\strong{Note:}  Do not compare the return value to a specific
exception; use \cfunction{PyErr_ExceptionMatches()} instead, shown
below.  (The comparison could easily fail since the exception may be
an instance instead of a class, in the case of a class exception, or
it may the a subclass of the expected exception.)
\end{cfuncdesc}

\begin{cfuncdesc}{int}{PyErr_ExceptionMatches}{PyObject *exc}
Equivalent to
\samp{PyErr_GivenExceptionMatches(PyErr_Occurred(), \var{exc})}.
This should only be called when an exception is actually set; a memory 
access violation will occur if no exception has been raised.
\end{cfuncdesc}

\begin{cfuncdesc}{int}{PyErr_GivenExceptionMatches}{PyObject *given, PyObject *exc}
Return true if the \var{given} exception matches the exception in
\var{exc}.  If \var{exc} is a class object, this also returns true
when \var{given} is an instance of a subclass.  If \var{exc} is a tuple, all
exceptions in the tuple (and recursively in subtuples) are searched
for a match.  If \var{given} is \NULL, a memory access violation will
occur.
\end{cfuncdesc}

\begin{cfuncdesc}{void}{PyErr_NormalizeException}{PyObject**exc, PyObject**val, PyObject**tb}
Under certain circumstances, the values returned by
\cfunction{PyErr_Fetch()} below can be ``unnormalized'', meaning that
\code{*\var{exc}} is a class object but \code{*\var{val}} is not an
instance of the  same class.  This function can be used to instantiate
the class in that case.  If the values are already normalized, nothing
happens.  The delayed normalization is implemented to improve
performance.
\end{cfuncdesc}

\begin{cfuncdesc}{void}{PyErr_Clear}{}
Clear the error indicator.  If the error indicator is not set, there
is no effect.
\end{cfuncdesc}

\begin{cfuncdesc}{void}{PyErr_Fetch}{PyObject **ptype, PyObject **pvalue,
                                     PyObject **ptraceback}
Retrieve the error indicator into three variables whose addresses are
passed.  If the error indicator is not set, set all three variables to
\NULL{}.  If it is set, it will be cleared and you own a reference to
each object retrieved.  The value and traceback object may be
\NULL{} even when the type object is not.  \strong{Note:}  This
function is normally only used by code that needs to handle exceptions
or by code that needs to save and restore the error indicator
temporarily.
\end{cfuncdesc}

\begin{cfuncdesc}{void}{PyErr_Restore}{PyObject *type, PyObject *value,
                                       PyObject *traceback}
Set  the error indicator from the three objects.  If the error
indicator is already set, it is cleared first.  If the objects are
\NULL{}, the error indicator is cleared.  Do not pass a \NULL{} type
and non-\NULL{} value or traceback.  The exception type should be a
string or class; if it is a class, the value should be an instance of
that class.  Do not pass an invalid exception type or value.
(Violating these rules will cause subtle problems later.)  This call
takes away a reference to each object, i.e.\ you must own a reference
to each object before the call and after the call you no longer own
these references.  (If you don't understand this, don't use this
function.  I warned you.)  \strong{Note:}  This function is normally
only used by code that needs to save and restore the error indicator
temporarily.
\end{cfuncdesc}

\begin{cfuncdesc}{void}{PyErr_SetString}{PyObject *type, char *message}
This is the most common way to set the error indicator.  The first
argument specifies the exception type; it is normally one of the
standard exceptions, e.g. \cdata{PyExc_RuntimeError}.  You need not
increment its reference count.  The second argument is an error
message; it is converted to a string object.
\end{cfuncdesc}

\begin{cfuncdesc}{void}{PyErr_SetObject}{PyObject *type, PyObject *value}
This function is similar to \cfunction{PyErr_SetString()} but lets you
specify an arbitrary Python object for the ``value'' of the exception.
You need not increment its reference count.
\end{cfuncdesc}

\begin{cfuncdesc}{PyObject*}{PyErr_Format}{PyObject *exception,
                                           const char *format, \moreargs}
This function sets the error indicator.  \var{exception} should be a
Python exception (string or class, not an instance).
\var{format} should be a string, containing format codes, similar to 
\cfunction{printf}. The \code{width.precision} before a format code
is parsed, but the width part is ignored.

\begin{tableii}{c|l}{character}{Character}{Meaning}
  \lineii{c}{Character, as an \ctype{int} parameter}
  \lineii{d}{Number in decimal, as an \ctype{int} parameter}
  \lineii{x}{Number in hexadecimal, as an \ctype{int} parameter}
  \lineii{x}{A string, as a \ctype{char *} parameter}
\end{tableii}

An unrecognized format character causes all the rest of
the format string to be copied as-is to the result string,
and any extra arguments discarded.

A new reference is returned, which is owned by the caller.
\end{cfuncdesc}

\begin{cfuncdesc}{void}{PyErr_SetNone}{PyObject *type}
This is a shorthand for \samp{PyErr_SetObject(\var{type}, Py_None)}.
\end{cfuncdesc}

\begin{cfuncdesc}{int}{PyErr_BadArgument}{}
This is a shorthand for \samp{PyErr_SetString(PyExc_TypeError,
\var{message})}, where \var{message} indicates that a built-in operation
was invoked with an illegal argument.  It is mostly for internal use.
\end{cfuncdesc}

\begin{cfuncdesc}{PyObject*}{PyErr_NoMemory}{}
This is a shorthand for \samp{PyErr_SetNone(PyExc_MemoryError)}; it
returns \NULL{} so an object allocation function can write
\samp{return PyErr_NoMemory();} when it runs out of memory.
\end{cfuncdesc}

\begin{cfuncdesc}{PyObject*}{PyErr_SetFromErrno}{PyObject *type}
This is a convenience function to raise an exception when a C library
function has returned an error and set the C variable \cdata{errno}.
It constructs a tuple object whose first item is the integer
\cdata{errno} value and whose second item is the corresponding error
message (gotten from \cfunction{strerror()}\ttindex{strerror()}), and
then calls
\samp{PyErr_SetObject(\var{type}, \var{object})}.  On \UNIX{}, when
the \cdata{errno} value is \constant{EINTR}, indicating an interrupted
system call, this calls \cfunction{PyErr_CheckSignals()}, and if that set
the error indicator, leaves it set to that.  The function always
returns \NULL{}, so a wrapper function around a system call can write 
\samp{return PyErr_SetFromErrno();} when  the system call returns an
error.
\end{cfuncdesc}

\begin{cfuncdesc}{void}{PyErr_BadInternalCall}{}
This is a shorthand for \samp{PyErr_SetString(PyExc_TypeError,
\var{message})}, where \var{message} indicates that an internal
operation (e.g. a Python/C API function) was invoked with an illegal
argument.  It is mostly for internal use.
\end{cfuncdesc}

\begin{cfuncdesc}{int}{PyErr_Warn}{PyObject *category, char *message}
Issue a warning message.  The \var{category} argument is a warning
category (see below) or \NULL; the \var{message} argument is a message
string.

This function normally prints a warning message to \var{sys.stderr};
however, it is also possible that the user has specified that warnings
are to be turned into errors, and in that case this will raise an
exception.  It is also possible that the function raises an exception
because of a problem with the warning machinery (the implementation
imports the \module{warnings} module to do the heavy lifting).  The
return value is \code{0} if no exception is raised, or \code{-1} if
an exception is raised.  (It is not possible to determine whether a
warning message is actually printed, nor what the reason is for the
exception; this is intentional.)  If an exception is raised, the
caller should do its normal exception handling
(e.g. \cfunction{Py_DECREF()} owned references and return an error
value).

Warning categories must be subclasses of \cdata{Warning}; the default
warning category is \cdata{RuntimeWarning}.  The standard Python
warning categories are available as global variables whose names are
\samp{PyExc_} followed by the Python exception name.  These have the
type \ctype{PyObject*}; they are all class objects.  Their names are
\cdata{PyExc_Warning}, \cdata{PyExc_UserWarning},
\cdata{PyExc_DeprecationWarning}, \cdata{PyExc_SyntaxWarning}, and
\cdata{PyExc_RuntimeWarning}.  \cdata{PyExc_Warning} is a subclass of
\cdata{PyExc_Exception}; the other warning categories are subclasses
of \cdata{PyExc_Warning}.

For information about warning control, see the documentation for the
\module{warnings} module and the \programopt{-W} option in the command
line documentation.  There is no C API for warning control.
\end{cfuncdesc}

\begin{cfuncdesc}{int}{PyErr_WarnExplicit}{PyObject *category, char *message,
char *filename, int lineno, char *module, PyObject *registry}
Issue a warning message with explicit control over all warning
attributes.  This is a straightforward wrapper around the Python
function \function{warnings.warn_explicit()}, see there for more
information.  The \var{module} and \var{registry} arguments may be
set to \code{NULL} to get the default effect described there.
\end{cfuncdesc}

\begin{cfuncdesc}{int}{PyErr_CheckSignals}{}
This function interacts with Python's signal handling.  It checks
whether a signal has been sent to the processes and if so, invokes the
corresponding signal handler.  If the
\module{signal}\refbimodindex{signal} module is supported, this can
invoke a signal handler written in Python.  In all cases, the default
effect for \constant{SIGINT}\ttindex{SIGINT} is to raise the
\withsubitem{(built-in exception)}{\ttindex{KeyboardInterrupt}}
\exception{KeyboardInterrupt} exception.  If an exception is raised the 
error indicator is set and the function returns \code{1}; otherwise
the function returns \code{0}.  The error indicator may or may not be
cleared if it was previously set.
\end{cfuncdesc}

\begin{cfuncdesc}{void}{PyErr_SetInterrupt}{}
This function is obsolete.  It simulates the effect of a
\constant{SIGINT}\ttindex{SIGINT} signal arriving --- the next time
\cfunction{PyErr_CheckSignals()} is called,
\withsubitem{(built-in exception)}{\ttindex{KeyboardInterrupt}}
\exception{KeyboardInterrupt} will be raised.
It may be called without holding the interpreter lock.
\end{cfuncdesc}

\begin{cfuncdesc}{PyObject*}{PyErr_NewException}{char *name,
                                                 PyObject *base,
                                                 PyObject *dict}
This utility function creates and returns a new exception object.  The
\var{name} argument must be the name of the new exception, a C string
of the form \code{module.class}.  The \var{base} and
\var{dict} arguments are normally \NULL{}.  This creates a
class object derived from the root for all exceptions, the built-in
name \exception{Exception} (accessible in C as
\cdata{PyExc_Exception}).  The \member{__module__} attribute of the
new class is set to the first part (up to the last dot) of the
\var{name} argument, and the class name is set to the last part (after
the last dot).  The \var{base} argument can be used to specify an
alternate base class.  The \var{dict} argument can be used to specify
a dictionary of class variables and methods.
\end{cfuncdesc}

\begin{cfuncdesc}{void}{PyErr_WriteUnraisable}{PyObject *obj}
This utility function prints a warning message to \var{sys.stderr}
when an exception has been set but it is impossible for the
interpreter to actually raise the exception.  It is used, for example,
when an exception occurs in an \member{__del__} method.

The function is called with a single argument \var{obj} that
identifies where the context in which the unraisable exception
occurred.  The repr of \var{obj} will be printed in the warning
message.
\end{cfuncdesc}

\section{Standard Exceptions \label{standardExceptions}}

All standard Python exceptions are available as global variables whose
names are \samp{PyExc_} followed by the Python exception name.  These
have the type \ctype{PyObject*}; they are all class objects.  For
completeness, here are all the variables:

\begin{tableiii}{l|l|c}{cdata}{C Name}{Python Name}{Notes}
  \lineiii{PyExc_Exception}{\exception{Exception}}{(1)}
  \lineiii{PyExc_StandardError}{\exception{StandardError}}{(1)}
  \lineiii{PyExc_ArithmeticError}{\exception{ArithmeticError}}{(1)}
  \lineiii{PyExc_LookupError}{\exception{LookupError}}{(1)}
  \lineiii{PyExc_AssertionError}{\exception{AssertionError}}{}
  \lineiii{PyExc_AttributeError}{\exception{AttributeError}}{}
  \lineiii{PyExc_EOFError}{\exception{EOFError}}{}
  \lineiii{PyExc_EnvironmentError}{\exception{EnvironmentError}}{(1)}
  \lineiii{PyExc_FloatingPointError}{\exception{FloatingPointError}}{}
  \lineiii{PyExc_IOError}{\exception{IOError}}{}
  \lineiii{PyExc_ImportError}{\exception{ImportError}}{}
  \lineiii{PyExc_IndexError}{\exception{IndexError}}{}
  \lineiii{PyExc_KeyError}{\exception{KeyError}}{}
  \lineiii{PyExc_KeyboardInterrupt}{\exception{KeyboardInterrupt}}{}
  \lineiii{PyExc_MemoryError}{\exception{MemoryError}}{}
  \lineiii{PyExc_NameError}{\exception{NameError}}{}
  \lineiii{PyExc_NotImplementedError}{\exception{NotImplementedError}}{}
  \lineiii{PyExc_OSError}{\exception{OSError}}{}
  \lineiii{PyExc_OverflowError}{\exception{OverflowError}}{}
  \lineiii{PyExc_RuntimeError}{\exception{RuntimeError}}{}
  \lineiii{PyExc_SyntaxError}{\exception{SyntaxError}}{}
  \lineiii{PyExc_SystemError}{\exception{SystemError}}{}
  \lineiii{PyExc_SystemExit}{\exception{SystemExit}}{}
  \lineiii{PyExc_TypeError}{\exception{TypeError}}{}
  \lineiii{PyExc_ValueError}{\exception{ValueError}}{}
  \lineiii{PyExc_WindowsError}{\exception{WindowsError}}{(2)}
  \lineiii{PyExc_ZeroDivisionError}{\exception{ZeroDivisionError}}{}
\end{tableiii}

\noindent
Notes:
\begin{description}
\item[(1)]
  This is a base class for other standard exceptions.

\item[(2)]
  Only defined on Windows; protect code that uses this by testing that
  the preprocessor macro \code{MS_WINDOWS} is defined.
\end{description}


\section{Deprecation of String Exceptions}

All exceptions built into Python or provided in the standard library
are derived from \exception{Exception}.
\withsubitem{(built-in exception)}{\ttindex{Exception}}

String exceptions are still supported in the interpreter to allow
existing code to run unmodified, but this will also change in a future 
release.


\chapter{Utilities \label{utilities}}

The functions in this chapter perform various utility tasks, such as
parsing function arguments and constructing Python values from C
values.

\section{OS Utilities \label{os}}

\begin{cfuncdesc}{int}{Py_FdIsInteractive}{FILE *fp, char *filename}
Return true (nonzero) if the standard I/O file \var{fp} with name
\var{filename} is deemed interactive.  This is the case for files for
which \samp{isatty(fileno(\var{fp}))} is true.  If the global flag
\cdata{Py_InteractiveFlag} is true, this function also returns true if
the \var{filename} pointer is \NULL{} or if the name is equal to one of
the strings \code{'<stdin>'} or \code{'???'}.
\end{cfuncdesc}

\begin{cfuncdesc}{long}{PyOS_GetLastModificationTime}{char *filename}
Return the time of last modification of the file \var{filename}.
The result is encoded in the same way as the timestamp returned by
the standard C library function \cfunction{time()}.
\end{cfuncdesc}

\begin{cfuncdesc}{void}{PyOS_AfterFork}{}
Function to update some internal state after a process fork; this
should be called in the new process if the Python interpreter will
continue to be used.  If a new executable is loaded into the new
process, this function does not need to be called.
\end{cfuncdesc}

\begin{cfuncdesc}{int}{PyOS_CheckStack}{}
Return true when the interpreter runs out of stack space.  This is a
reliable check, but is only available when \code{USE_STACKCHECK} is
defined (currently on Windows using the Microsoft Visual C++ compiler
and on the Macintosh).  \code{USE_CHECKSTACK} will be defined
automatically; you should never change the definition in your own
code.
\end{cfuncdesc}

\begin{cfuncdesc}{PyOS_sighandler_t}{PyOS_getsig}{int i}
Return the current signal handler for signal \var{i}.
This is a thin wrapper around either \cfunction{sigaction} or
\cfunction{signal}.  Do not call those functions directly!
\ctype{PyOS_sighandler_t} is a typedef alias for \ctype{void (*)(int)}.
\end{cfuncdesc}

\begin{cfuncdesc}{PyOS_sighandler_t}{PyOS_setsig}{int i, PyOS_sighandler_t h}
Set the signal handler for signal \var{i} to be \var{h};
return the old signal handler.
This is a thin wrapper around either \cfunction{sigaction} or
\cfunction{signal}.  Do not call those functions directly!
\ctype{PyOS_sighandler_t} is a typedef alias for \ctype{void (*)(int)}.
\end{cfuncdesc}


\section{Process Control \label{processControl}}

\begin{cfuncdesc}{void}{Py_FatalError}{char *message}
Print a fatal error message and kill the process.  No cleanup is
performed.  This function should only be invoked when a condition is
detected that would make it dangerous to continue using the Python
interpreter; e.g., when the object administration appears to be
corrupted.  On \UNIX{}, the standard C library function
\cfunction{abort()}\ttindex{abort()} is called which will attempt to
produce a \file{core} file.
\end{cfuncdesc}

\begin{cfuncdesc}{void}{Py_Exit}{int status}
Exit the current process.  This calls
\cfunction{Py_Finalize()}\ttindex{Py_Finalize()} and
then calls the standard C library function
\code{exit(\var{status})}\ttindex{exit()}.
\end{cfuncdesc}

\begin{cfuncdesc}{int}{Py_AtExit}{void (*func) ()}
Register a cleanup function to be called by
\cfunction{Py_Finalize()}\ttindex{Py_Finalize()}.
The cleanup function will be called with no arguments and should
return no value.  At most 32 \index{cleanup functions}cleanup
functions can be registered.
When the registration is successful, \cfunction{Py_AtExit()} returns
\code{0}; on failure, it returns \code{-1}.  The cleanup function
registered last is called first.  Each cleanup function will be called
at most once.  Since Python's internal finallization will have
completed before the cleanup function, no Python APIs should be called
by \var{func}.
\end{cfuncdesc}


\section{Importing Modules \label{importing}}

\begin{cfuncdesc}{PyObject*}{PyImport_ImportModule}{char *name}
This is a simplified interface to
\cfunction{PyImport_ImportModuleEx()} below, leaving the
\var{globals} and \var{locals} arguments set to \NULL{}.  When the
\var{name} argument contains a dot (i.e., when it specifies a
submodule of a package), the \var{fromlist} argument is set to the
list \code{['*']} so that the return value is the named module rather
than the top-level package containing it as would otherwise be the
case.  (Unfortunately, this has an additional side effect when
\var{name} in fact specifies a subpackage instead of a submodule: the
submodules specified in the package's \code{__all__} variable are
\index{package variable!\code{__all__}}
\withsubitem{(package variable)}{\ttindex{__all__}}loaded.)  Return a
new reference to the imported module, or
\NULL{} with an exception set on failure (the module may still be
created in this case --- examine \code{sys.modules} to find out).
\withsubitem{(in module sys)}{\ttindex{modules}}
\end{cfuncdesc}

\begin{cfuncdesc}{PyObject*}{PyImport_ImportModuleEx}{char *name, PyObject *globals, PyObject *locals, PyObject *fromlist}
Import a module.  This is best described by referring to the built-in
Python function \function{__import__()}\bifuncindex{__import__}, as
the standard \function{__import__()} function calls this function
directly.

The return value is a new reference to the imported module or
top-level package, or \NULL{} with an exception set on failure
(the module may still be created in this case).  Like for
\function{__import__()}, the return value when a submodule of a
package was requested is normally the top-level package, unless a
non-empty \var{fromlist} was given.
\end{cfuncdesc}

\begin{cfuncdesc}{PyObject*}{PyImport_Import}{PyObject *name}
This is a higher-level interface that calls the current ``import hook
function''.  It invokes the \function{__import__()} function from the
\code{__builtins__} of the current globals.  This means that the
import is done using whatever import hooks are installed in the
current environment, e.g. by \module{rexec}\refstmodindex{rexec} or
\module{ihooks}\refstmodindex{ihooks}.
\end{cfuncdesc}

\begin{cfuncdesc}{PyObject*}{PyImport_ReloadModule}{PyObject *m}
Reload a module.  This is best described by referring to the built-in
Python function \function{reload()}\bifuncindex{reload}, as the standard
\function{reload()} function calls this function directly.  Return a
new reference to the reloaded module, or \NULL{} with an exception set
on failure (the module still exists in this case).
\end{cfuncdesc}

\begin{cfuncdesc}{PyObject*}{PyImport_AddModule}{char *name}
Return the module object corresponding to a module name.  The
\var{name} argument may be of the form \code{package.module}).  First
check the modules dictionary if there's one there, and if not, create
a new one and insert in in the modules dictionary.
Warning: this function does not load or import the module; if the
module wasn't already loaded, you will get an empty module object.
Use \cfunction{PyImport_ImportModule()} or one of its variants to
import a module.
Return \NULL{} with an exception set on failure.
\end{cfuncdesc}

\begin{cfuncdesc}{PyObject*}{PyImport_ExecCodeModule}{char *name, PyObject *co}
Given a module name (possibly of the form \code{package.module}) and a
code object read from a Python bytecode file or obtained from the
built-in function \function{compile()}\bifuncindex{compile}, load the
module.  Return a new reference to the module object, or \NULL{} with
an exception set if an error occurred (the module may still be created
in this case).  (This function would reload the module if it was
already imported.)
\end{cfuncdesc}

\begin{cfuncdesc}{long}{PyImport_GetMagicNumber}{}
Return the magic number for Python bytecode files (a.k.a.
\file{.pyc} and \file{.pyo} files).  The magic number should be
present in the first four bytes of the bytecode file, in little-endian
byte order.
\end{cfuncdesc}

\begin{cfuncdesc}{PyObject*}{PyImport_GetModuleDict}{}
Return the dictionary used for the module administration
(a.k.a. \code{sys.modules}).  Note that this is a per-interpreter
variable.
\end{cfuncdesc}

\begin{cfuncdesc}{void}{_PyImport_Init}{}
Initialize the import mechanism.  For internal use only.
\end{cfuncdesc}

\begin{cfuncdesc}{void}{PyImport_Cleanup}{}
Empty the module table.  For internal use only.
\end{cfuncdesc}

\begin{cfuncdesc}{void}{_PyImport_Fini}{}
Finalize the import mechanism.  For internal use only.
\end{cfuncdesc}

\begin{cfuncdesc}{PyObject*}{_PyImport_FindExtension}{char *, char *}
For internal use only.
\end{cfuncdesc}

\begin{cfuncdesc}{PyObject*}{_PyImport_FixupExtension}{char *, char *}
For internal use only.
\end{cfuncdesc}

\begin{cfuncdesc}{int}{PyImport_ImportFrozenModule}{char *name}
Load a frozen module named \var{name}.  Return \code{1} for success,
\code{0} if the module is not found, and \code{-1} with an exception
set if the initialization failed.  To access the imported module on a
successful load, use \cfunction{PyImport_ImportModule()}.
(Note the misnomer --- this function would reload the module if it was
already imported.)
\end{cfuncdesc}

\begin{ctypedesc}[_frozen]{struct _frozen}
This is the structure type definition for frozen module descriptors,
as generated by the \program{freeze}\index{freeze utility} utility
(see \file{Tools/freeze/} in the Python source distribution).  Its
definition, found in \file{Include/import.h}, is:

\begin{verbatim}
struct _frozen {
    char *name;
    unsigned char *code;
    int size;
};
\end{verbatim}
\end{ctypedesc}

\begin{cvardesc}{struct _frozen*}{PyImport_FrozenModules}
This pointer is initialized to point to an array of \ctype{struct
_frozen} records, terminated by one whose members are all
\NULL{} or zero.  When a frozen module is imported, it is searched in
this table.  Third-party code could play tricks with this to provide a 
dynamically created collection of frozen modules.
\end{cvardesc}

\begin{cfuncdesc}{int}{PyImport_AppendInittab}{char *name,
                                               void (*initfunc)(void)}
Add a single module to the existing table of built-in modules.  This
is a convenience wrapper around \cfunction{PyImport_ExtendInittab()},
returning \code{-1} if the table could not be extended.  The new
module can be imported by the name \var{name}, and uses the function
\var{initfunc} as the initialization function called on the first
attempted import.  This should be called before
\cfunction{Py_Initialize()}.
\end{cfuncdesc}

\begin{ctypedesc}[_inittab]{struct _inittab}
Structure describing a single entry in the list of built-in modules.
Each of these structures gives the name and initialization function
for a module built into the interpreter.  Programs which embed Python
may use an array of these structures in conjunction with
\cfunction{PyImport_ExtendInittab()} to provide additional built-in
modules.  The structure is defined in \file{Include/import.h} as:

\begin{verbatim}
struct _inittab {
    char *name;
    void (*initfunc)(void);
};
\end{verbatim}
\end{ctypedesc}

\begin{cfuncdesc}{int}{PyImport_ExtendInittab}{struct _inittab *newtab}
Add a collection of modules to the table of built-in modules.  The
\var{newtab} array must end with a sentinel entry which contains
\NULL{} for the \member{name} field; failure to provide the sentinel
value can result in a memory fault.  Returns \code{0} on success or
\code{-1} if insufficient memory could be allocated to extend the
internal table.  In the event of failure, no modules are added to the
internal table.  This should be called before
\cfunction{Py_Initialize()}.
\end{cfuncdesc}


\chapter{Abstract Objects Layer \label{abstract}}

The functions in this chapter interact with Python objects regardless
of their type, or with wide classes of object types (e.g. all
numerical types, or all sequence types).  When used on object types
for which they do not apply, they will raise a Python exception.

\section{Object Protocol \label{object}}

\begin{cfuncdesc}{int}{PyObject_Print}{PyObject *o, FILE *fp, int flags}
Print an object \var{o}, on file \var{fp}.  Returns \code{-1} on error.
The flags argument is used to enable certain printing options.  The
only option currently supported is \constant{Py_PRINT_RAW}; if given,
the \function{str()} of the object is written instead of the
\function{repr()}.
\end{cfuncdesc}

\begin{cfuncdesc}{int}{PyObject_HasAttrString}{PyObject *o, char *attr_name}
Returns \code{1} if \var{o} has the attribute \var{attr_name}, and
\code{0} otherwise.  This is equivalent to the Python expression
\samp{hasattr(\var{o}, \var{attr_name})}.
This function always succeeds.
\end{cfuncdesc}

\begin{cfuncdesc}{PyObject*}{PyObject_GetAttrString}{PyObject *o,
                                                     char *attr_name}
Retrieve an attribute named \var{attr_name} from object \var{o}.
Returns the attribute value on success, or \NULL{} on failure.
This is the equivalent of the Python expression
\samp{\var{o}.\var{attr_name}}.
\end{cfuncdesc}


\begin{cfuncdesc}{int}{PyObject_HasAttr}{PyObject *o, PyObject *attr_name}
Returns \code{1} if \var{o} has the attribute \var{attr_name}, and
\code{0} otherwise.  This is equivalent to the Python expression
\samp{hasattr(\var{o}, \var{attr_name})}. 
This function always succeeds.
\end{cfuncdesc}


\begin{cfuncdesc}{PyObject*}{PyObject_GetAttr}{PyObject *o,
                                               PyObject *attr_name}
Retrieve an attribute named \var{attr_name} from object \var{o}.
Returns the attribute value on success, or \NULL{} on failure.
This is the equivalent of the Python expression
\samp{\var{o}.\var{attr_name}}.
\end{cfuncdesc}


\begin{cfuncdesc}{int}{PyObject_SetAttrString}{PyObject *o, char *attr_name, PyObject *v}
Set the value of the attribute named \var{attr_name}, for object
\var{o}, to the value \var{v}. Returns \code{-1} on failure.  This is
the equivalent of the Python statement \samp{\var{o}.\var{attr_name} =
\var{v}}.
\end{cfuncdesc}


\begin{cfuncdesc}{int}{PyObject_SetAttr}{PyObject *o, PyObject *attr_name, PyObject *v}
Set the value of the attribute named \var{attr_name}, for
object \var{o},
to the value \var{v}. Returns \code{-1} on failure.  This is
the equivalent of the Python statement \samp{\var{o}.\var{attr_name} =
\var{v}}.
\end{cfuncdesc}


\begin{cfuncdesc}{int}{PyObject_DelAttrString}{PyObject *o, char *attr_name}
Delete attribute named \var{attr_name}, for object \var{o}. Returns
\code{-1} on failure.  This is the equivalent of the Python
statement: \samp{del \var{o}.\var{attr_name}}.
\end{cfuncdesc}


\begin{cfuncdesc}{int}{PyObject_DelAttr}{PyObject *o, PyObject *attr_name}
Delete attribute named \var{attr_name}, for object \var{o}. Returns
\code{-1} on failure.  This is the equivalent of the Python
statement \samp{del \var{o}.\var{attr_name}}.
\end{cfuncdesc}


\begin{cfuncdesc}{int}{PyObject_Cmp}{PyObject *o1, PyObject *o2, int *result}
Compare the values of \var{o1} and \var{o2} using a routine provided
by \var{o1}, if one exists, otherwise with a routine provided by
\var{o2}.  The result of the comparison is returned in \var{result}.
Returns \code{-1} on failure.  This is the equivalent of the Python
statement\bifuncindex{cmp} \samp{\var{result} = cmp(\var{o1}, \var{o2})}.
\end{cfuncdesc}


\begin{cfuncdesc}{int}{PyObject_Compare}{PyObject *o1, PyObject *o2}
Compare the values of \var{o1} and \var{o2} using a routine provided
by \var{o1}, if one exists, otherwise with a routine provided by
\var{o2}.  Returns the result of the comparison on success.  On error,
the value returned is undefined; use \cfunction{PyErr_Occurred()} to
detect an error.  This is equivalent to the Python
expression\bifuncindex{cmp} \samp{cmp(\var{o1}, \var{o2})}.
\end{cfuncdesc}


\begin{cfuncdesc}{PyObject*}{PyObject_Repr}{PyObject *o}
Compute a string representation of object \var{o}.  Returns the
string representation on success, \NULL{} on failure.  This is
the equivalent of the Python expression \samp{repr(\var{o})}.
Called by the \function{repr()}\bifuncindex{repr} built-in function
and by reverse quotes.
\end{cfuncdesc}


\begin{cfuncdesc}{PyObject*}{PyObject_Str}{PyObject *o}
Compute a string representation of object \var{o}.  Returns the
string representation on success, \NULL{} on failure.  This is
the equivalent of the Python expression \samp{str(\var{o})}.
Called by the \function{str()}\bifuncindex{str} built-in function and
by the \keyword{print} statement.
\end{cfuncdesc}


\begin{cfuncdesc}{PyObject*}{PyObject_Unicode}{PyObject *o}
Compute a Unicode string representation of object \var{o}.  Returns the
Unicode string representation on success, \NULL{} on failure.  This is
the equivalent of the Python expression \samp{unistr(\var{o})}.
Called by the \function{unistr()}\bifuncindex{unistr} built-in function.
\end{cfuncdesc}

\begin{cfuncdesc}{int}{PyObject_IsInstance}{PyObject *inst, PyObject *cls}
Return \code{1} if \var{inst} is an instance of the class \var{cls} or
a subclass of \var{cls}.  If \var{cls} is a type object rather than a
class object, \cfunction{PyObject_IsInstance()} returns \code{1} if
\var{inst} is of type \var{cls}.  If \var{inst} is not a class
instance and \var{cls} is neither a type object or class object,
\var{inst} must have a \member{__class__} attribute --- the class
relationship of the value of that attribute with \var{cls} will be
used to determine the result of this function.
\versionadded{2.1}
\end{cfuncdesc}

Subclass determination is done in a fairly straightforward way, but
includes a wrinkle that implementors of extensions to the class system
may want to be aware of.  If \class{A} and \class{B} are class
objects, \class{B} is a subclass of \class{A} if it inherits from
\class{A} either directly or indirectly.  If either is not a class
object, a more general mechanism is used to determine the class
relationship of the two objects.  When testing if \var{B} is a
subclass of \var{A}, if \var{A} is \var{B},
\cfunction{PyObject_IsSubclass()} returns true.  If \var{A} and
\var{B} are different objects, \var{B}'s \member{__bases__} attribute
is searched in a depth-first fashion for \var{A} --- the presence of
the \member{__bases__} attribute is considered sufficient for this
determination.

\begin{cfuncdesc}{int}{PyObject_IsSubclass}{PyObject *derived,
                                            PyObject *cls}
Returns \code{1} if the class \var{derived} is identical to or derived
from the class \var{cls}, otherwise returns \code{0}.  In case of an
error, returns \code{-1}.  If either \var{derived} or \var{cls} is not
an actual class object, this function uses the generic algorithm
described above.
\versionadded{2.1}
\end{cfuncdesc}


\begin{cfuncdesc}{int}{PyCallable_Check}{PyObject *o}
Determine if the object \var{o} is callable.  Return \code{1} if the
object is callable and \code{0} otherwise.
This function always succeeds.
\end{cfuncdesc}


\begin{cfuncdesc}{PyObject*}{PyObject_CallObject}{PyObject *callable_object,
                                                  PyObject *args}
Call a callable Python object \var{callable_object}, with
arguments given by the tuple \var{args}.  If no arguments are
needed, then \var{args} may be \NULL{}.  Returns the result of the
call on success, or \NULL{} on failure.  This is the equivalent
of the Python expression \samp{apply(\var{callable_object}, \var{args})}.
\bifuncindex{apply}
\end{cfuncdesc}

\begin{cfuncdesc}{PyObject*}{PyObject_CallFunction}{PyObject *callable_object,
                                                    char *format, ...}
Call a callable Python object \var{callable_object}, with a
variable number of C arguments. The C arguments are described
using a \cfunction{Py_BuildValue()} style format string. The format may
be \NULL{}, indicating that no arguments are provided.  Returns the
result of the call on success, or \NULL{} on failure.  This is
the equivalent of the Python expression \samp{apply(\var{callable_object},
\var{args})}.\bifuncindex{apply}
\end{cfuncdesc}


\begin{cfuncdesc}{PyObject*}{PyObject_CallMethod}{PyObject *o,
                                           char *method, char *format, ...}
Call the method named \var{m} of object \var{o} with a variable number
of C arguments.  The C arguments are described by a
\cfunction{Py_BuildValue()} format string.  The format may be \NULL{},
indicating that no arguments are provided. Returns the result of the
call on success, or \NULL{} on failure.  This is the equivalent of the
Python expression \samp{\var{o}.\var{method}(\var{args})}.
Note that special method names, such as \method{__add__()},
\method{__getitem__()}, and so on are not supported.  The specific
abstract-object routines for these must be used.
\end{cfuncdesc}


\begin{cfuncdesc}{int}{PyObject_Hash}{PyObject *o}
Compute and return the hash value of an object \var{o}.  On
failure, return \code{-1}.  This is the equivalent of the Python
expression \samp{hash(\var{o})}.\bifuncindex{hash}
\end{cfuncdesc}


\begin{cfuncdesc}{int}{PyObject_IsTrue}{PyObject *o}
Returns \code{1} if the object \var{o} is considered to be true, and
\code{0} otherwise. This is equivalent to the Python expression
\samp{not not \var{o}}.
This function always succeeds.
\end{cfuncdesc}


\begin{cfuncdesc}{PyObject*}{PyObject_Type}{PyObject *o}
On success, returns a type object corresponding to the object
type of object \var{o}. On failure, returns \NULL{}.  This is
equivalent to the Python expression \samp{type(\var{o})}.
\bifuncindex{type}
\end{cfuncdesc}

\begin{cfuncdesc}{int}{PyObject_Length}{PyObject *o}
Return the length of object \var{o}.  If the object \var{o} provides
both sequence and mapping protocols, the sequence length is
returned.  On error, \code{-1} is returned.  This is the equivalent
to the Python expression \samp{len(\var{o})}.\bifuncindex{len}
\end{cfuncdesc}


\begin{cfuncdesc}{PyObject*}{PyObject_GetItem}{PyObject *o, PyObject *key}
Return element of \var{o} corresponding to the object \var{key} or
\NULL{} on failure. This is the equivalent of the Python expression
\samp{\var{o}[\var{key}]}.
\end{cfuncdesc}


\begin{cfuncdesc}{int}{PyObject_SetItem}{PyObject *o, PyObject *key, PyObject *v}
Map the object \var{key} to the value \var{v}.
Returns \code{-1} on failure.  This is the equivalent
of the Python statement \samp{\var{o}[\var{key}] = \var{v}}.
\end{cfuncdesc}


\begin{cfuncdesc}{int}{PyObject_DelItem}{PyObject *o, PyObject *key}
Delete the mapping for \var{key} from \var{o}.  Returns \code{-1} on
failure. This is the equivalent of the Python statement \samp{del
\var{o}[\var{key}]}.
\end{cfuncdesc}

\begin{cfuncdesc}{int}{PyObject_AsFileDescriptor}{PyObject *o}
Derives a file-descriptor from a Python object.  If the object
is an integer or long integer, its value is returned.  If not, the
object's \method{fileno()} method is called if it exists; the method
must return an integer or long integer, which is returned as the file
descriptor value.  Returns \code{-1} on failure.
\end{cfuncdesc}

\section{Number Protocol \label{number}}

\begin{cfuncdesc}{int}{PyNumber_Check}{PyObject *o}
Returns \code{1} if the object \var{o} provides numeric protocols, and
false otherwise. 
This function always succeeds.
\end{cfuncdesc}


\begin{cfuncdesc}{PyObject*}{PyNumber_Add}{PyObject *o1, PyObject *o2}
Returns the result of adding \var{o1} and \var{o2}, or \NULL{} on
failure.  This is the equivalent of the Python expression
\samp{\var{o1} + \var{o2}}.
\end{cfuncdesc}


\begin{cfuncdesc}{PyObject*}{PyNumber_Subtract}{PyObject *o1, PyObject *o2}
Returns the result of subtracting \var{o2} from \var{o1}, or
\NULL{} on failure.  This is the equivalent of the Python expression
\samp{\var{o1} - \var{o2}}.
\end{cfuncdesc}


\begin{cfuncdesc}{PyObject*}{PyNumber_Multiply}{PyObject *o1, PyObject *o2}
Returns the result of multiplying \var{o1} and \var{o2}, or \NULL{} on
failure.  This is the equivalent of the Python expression
\samp{\var{o1} * \var{o2}}.
\end{cfuncdesc}


\begin{cfuncdesc}{PyObject*}{PyNumber_Divide}{PyObject *o1, PyObject *o2}
Returns the result of dividing \var{o1} by \var{o2}, or \NULL{} on
failure. 
This is the equivalent of the Python expression \samp{\var{o1} /
\var{o2}}.
\end{cfuncdesc}


\begin{cfuncdesc}{PyObject*}{PyNumber_Remainder}{PyObject *o1, PyObject *o2}
Returns the remainder of dividing \var{o1} by \var{o2}, or \NULL{} on
failure.  This is the equivalent of the Python expression
\samp{\var{o1} \%\ \var{o2}}.
\end{cfuncdesc}


\begin{cfuncdesc}{PyObject*}{PyNumber_Divmod}{PyObject *o1, PyObject *o2}
See the built-in function \function{divmod()}\bifuncindex{divmod}.
Returns \NULL{} on failure.  This is the equivalent of the Python
expression \samp{divmod(\var{o1}, \var{o2})}.
\end{cfuncdesc}


\begin{cfuncdesc}{PyObject*}{PyNumber_Power}{PyObject *o1, PyObject *o2, PyObject *o3}
See the built-in function \function{pow()}\bifuncindex{pow}.  Returns
\NULL{} on failure. This is the equivalent of the Python expression
\samp{pow(\var{o1}, \var{o2}, \var{o3})}, where \var{o3} is optional.
If \var{o3} is to be ignored, pass \cdata{Py_None} in its place
(passing \NULL{} for \var{o3} would cause an illegal memory access).
\end{cfuncdesc}


\begin{cfuncdesc}{PyObject*}{PyNumber_Negative}{PyObject *o}
Returns the negation of \var{o} on success, or \NULL{} on failure.
This is the equivalent of the Python expression \samp{-\var{o}}.
\end{cfuncdesc}


\begin{cfuncdesc}{PyObject*}{PyNumber_Positive}{PyObject *o}
Returns \var{o} on success, or \NULL{} on failure.
This is the equivalent of the Python expression \samp{+\var{o}}.
\end{cfuncdesc}


\begin{cfuncdesc}{PyObject*}{PyNumber_Absolute}{PyObject *o}
Returns the absolute value of \var{o}, or \NULL{} on failure.  This is
the equivalent of the Python expression \samp{abs(\var{o})}.
\bifuncindex{abs}
\end{cfuncdesc}


\begin{cfuncdesc}{PyObject*}{PyNumber_Invert}{PyObject *o}
Returns the bitwise negation of \var{o} on success, or \NULL{} on
failure.  This is the equivalent of the Python expression
\samp{\~\var{o}}.
\end{cfuncdesc}


\begin{cfuncdesc}{PyObject*}{PyNumber_Lshift}{PyObject *o1, PyObject *o2}
Returns the result of left shifting \var{o1} by \var{o2} on success,
or \NULL{} on failure.  This is the equivalent of the Python
expression \samp{\var{o1} <\code{<} \var{o2}}.
\end{cfuncdesc}


\begin{cfuncdesc}{PyObject*}{PyNumber_Rshift}{PyObject *o1, PyObject *o2}
Returns the result of right shifting \var{o1} by \var{o2} on success,
or \NULL{} on failure.  This is the equivalent of the Python
expression \samp{\var{o1} >\code{>} \var{o2}}.
\end{cfuncdesc}


\begin{cfuncdesc}{PyObject*}{PyNumber_And}{PyObject *o1, PyObject *o2}
Returns the ``bitwise and'' of \var{o2} and \var{o2} on success and
\NULL{} on failure. This is the equivalent of the Python expression
\samp{\var{o1} \&\ \var{o2}}.
\end{cfuncdesc}


\begin{cfuncdesc}{PyObject*}{PyNumber_Xor}{PyObject *o1, PyObject *o2}
Returns the ``bitwise exclusive or'' of \var{o1} by \var{o2} on success,
or \NULL{} on failure.  This is the equivalent of the Python
expression \samp{\var{o1} \^{ }\var{o2}}.
\end{cfuncdesc}

\begin{cfuncdesc}{PyObject*}{PyNumber_Or}{PyObject *o1, PyObject *o2}
Returns the ``bitwise or'' of \var{o1} and \var{o2} on success, or
\NULL{} on failure.  This is the equivalent of the Python expression
\samp{\var{o1} | \var{o2}}.
\end{cfuncdesc}


\begin{cfuncdesc}{PyObject*}{PyNumber_InPlaceAdd}{PyObject *o1, PyObject *o2}
Returns the result of adding \var{o1} and \var{o2}, or \NULL{} on failure. 
The operation is done \emph{in-place} when \var{o1} supports it.  This is the
equivalent of the Python expression \samp{\var{o1} += \var{o2}}.
\end{cfuncdesc}


\begin{cfuncdesc}{PyObject*}{PyNumber_InPlaceSubtract}{PyObject *o1, PyObject *o2}
Returns the result of subtracting \var{o2} from \var{o1}, or
\NULL{} on failure.  The operation is done \emph{in-place} when \var{o1}
supports it.  This is the equivalent of the Python expression \samp{\var{o1}
-= \var{o2}}.
\end{cfuncdesc}


\begin{cfuncdesc}{PyObject*}{PyNumber_InPlaceMultiply}{PyObject *o1, PyObject *o2}
Returns the result of multiplying \var{o1} and \var{o2}, or \NULL{} on
failure.  The operation is done \emph{in-place} when \var{o1} supports it. 
This is the equivalent of the Python expression \samp{\var{o1} *= \var{o2}}.
\end{cfuncdesc}


\begin{cfuncdesc}{PyObject*}{PyNumber_InPlaceDivide}{PyObject *o1, PyObject *o2}
Returns the result of dividing \var{o1} by \var{o2}, or \NULL{} on failure. 
The operation is done \emph{in-place} when \var{o1} supports it. This is the
equivalent of the Python expression \samp{\var{o1} /= \var{o2}}.
\end{cfuncdesc}


\begin{cfuncdesc}{PyObject*}{PyNumber_InPlaceRemainder}{PyObject *o1, PyObject *o2}
Returns the remainder of dividing \var{o1} by \var{o2}, or \NULL{} on
failure.  The operation is done \emph{in-place} when \var{o1} supports it. 
This is the equivalent of the Python expression \samp{\var{o1} \%= \var{o2}}.
\end{cfuncdesc}


\begin{cfuncdesc}{PyObject*}{PyNumber_InPlacePower}{PyObject *o1, PyObject *o2, PyObject *o3}
See the built-in function \function{pow()}\bifuncindex{pow}.  Returns
\NULL{} on failure.  The operation is done \emph{in-place} when \var{o1}
supports it.  This is the equivalent of the Python expression \samp{\var{o1}
**= \var{o2}} when o3 is \cdata{Py_None}, or an in-place variant of
\samp{pow(\var{o1}, \var{o2}, \var{o3})} otherwise. If \var{o3} is to be
ignored, pass \cdata{Py_None} in its place (passing \NULL{} for \var{o3}
would cause an illegal memory access).
\end{cfuncdesc}

\begin{cfuncdesc}{PyObject*}{PyNumber_InPlaceLshift}{PyObject *o1, PyObject *o2}
Returns the result of left shifting \var{o1} by \var{o2} on success, or
\NULL{} on failure.  The operation is done \emph{in-place} when \var{o1}
supports it.  This is the equivalent of the Python expression \samp{\var{o1}
<\code{<=} \var{o2}}.
\end{cfuncdesc}


\begin{cfuncdesc}{PyObject*}{PyNumber_InPlaceRshift}{PyObject *o1, PyObject *o2}
Returns the result of right shifting \var{o1} by \var{o2} on success, or
\NULL{} on failure.  The operation is done \emph{in-place} when \var{o1}
supports it.  This is the equivalent of the Python expression \samp{\var{o1}
>\code{>=} \var{o2}}.
\end{cfuncdesc}


\begin{cfuncdesc}{PyObject*}{PyNumber_InPlaceAnd}{PyObject *o1, PyObject *o2}
Returns the ``bitwise and'' of \var{o1} and \var{o2} on success
and \NULL{} on failure. The operation is done \emph{in-place} when
\var{o1} supports it.  This is the equivalent of the Python expression
\samp{\var{o1} \&= \var{o2}}.
\end{cfuncdesc}


\begin{cfuncdesc}{PyObject*}{PyNumber_InPlaceXor}{PyObject *o1, PyObject *o2}
Returns the ``bitwise exclusive or'' of \var{o1} by \var{o2} on success, or
\NULL{} on failure.  The operation is done \emph{in-place} when \var{o1}
supports it.  This is the equivalent of the Python expression \samp{\var{o1}
\^= \var{o2}}.
\end{cfuncdesc}

\begin{cfuncdesc}{PyObject*}{PyNumber_InPlaceOr}{PyObject *o1, PyObject *o2}
Returns the ``bitwise or'' of \var{o1} and \var{o2} on success, or \NULL{}
on failure.  The operation is done \emph{in-place} when \var{o1} supports
it.  This is the equivalent of the Python expression \samp{\var{o1} |=
\var{o2}}.
\end{cfuncdesc}

\begin{cfuncdesc}{int}{PyNumber_Coerce}{PyObject **p1, PyObject **p2}
This function takes the addresses of two variables of type
\ctype{PyObject*}.  If the objects pointed to by \code{*\var{p1}} and
\code{*\var{p2}} have the same type, increment their reference count
and return \code{0} (success). If the objects can be converted to a
common numeric type, replace \code{*p1} and \code{*p2} by their
converted value (with 'new' reference counts), and return \code{0}.
If no conversion is possible, or if some other error occurs, return
\code{-1} (failure) and don't increment the reference counts.  The
call \code{PyNumber_Coerce(\&o1, \&o2)} is equivalent to the Python
statement \samp{\var{o1}, \var{o2} = coerce(\var{o1}, \var{o2})}.
\bifuncindex{coerce}
\end{cfuncdesc}

\begin{cfuncdesc}{PyObject*}{PyNumber_Int}{PyObject *o}
Returns the \var{o} converted to an integer object on success, or
\NULL{} on failure.  This is the equivalent of the Python
expression \samp{int(\var{o})}.\bifuncindex{int}
\end{cfuncdesc}

\begin{cfuncdesc}{PyObject*}{PyNumber_Long}{PyObject *o}
Returns the \var{o} converted to a long integer object on success,
or \NULL{} on failure.  This is the equivalent of the Python
expression \samp{long(\var{o})}.\bifuncindex{long}
\end{cfuncdesc}

\begin{cfuncdesc}{PyObject*}{PyNumber_Float}{PyObject *o}
Returns the \var{o} converted to a float object on success, or
\NULL{} on failure.  This is the equivalent of the Python expression
\samp{float(\var{o})}.\bifuncindex{float}
\end{cfuncdesc}


\section{Sequence Protocol \label{sequence}}

\begin{cfuncdesc}{int}{PySequence_Check}{PyObject *o}
Return \code{1} if the object provides sequence protocol, and
\code{0} otherwise.  This function always succeeds.
\end{cfuncdesc}

\begin{cfuncdesc}{int}{PySequence_Size}{PyObject *o}
Returns the number of objects in sequence \var{o} on success, and
\code{-1} on failure.  For objects that do not provide sequence
protocol, this is equivalent to the Python expression
\samp{len(\var{o})}.\bifuncindex{len}
\end{cfuncdesc}

\begin{cfuncdesc}{int}{PySequence_Length}{PyObject *o}
Alternate name for \cfunction{PySequence_Size()}.
\end{cfuncdesc}

\begin{cfuncdesc}{PyObject*}{PySequence_Concat}{PyObject *o1, PyObject *o2}
Return the concatenation of \var{o1} and \var{o2} on success, and \NULL{} on
failure.   This is the equivalent of the Python
expression \samp{\var{o1} + \var{o2}}.
\end{cfuncdesc}


\begin{cfuncdesc}{PyObject*}{PySequence_Repeat}{PyObject *o, int count}
Return the result of repeating sequence object
\var{o} \var{count} times, or \NULL{} on failure.  This is the
equivalent of the Python expression \samp{\var{o} * \var{count}}.
\end{cfuncdesc}

\begin{cfuncdesc}{PyObject*}{PySequence_InPlaceConcat}{PyObject *o1, PyObject *o2}
Return the concatenation of \var{o1} and \var{o2} on success, and \NULL{} on
failure.  The operation is done \emph{in-place} when \var{o1} supports it. 
This is the equivalent of the Python expression \samp{\var{o1} += \var{o2}}.
\end{cfuncdesc}


\begin{cfuncdesc}{PyObject*}{PySequence_InPlaceRepeat}{PyObject *o, int count}
Return the result of repeating sequence object \var{o} \var{count} times, or
\NULL{} on failure.  The operation is done \emph{in-place} when \var{o}
supports it.  This is the equivalent of the Python expression \samp{\var{o}
*= \var{count}}.
\end{cfuncdesc}


\begin{cfuncdesc}{PyObject*}{PySequence_GetItem}{PyObject *o, int i}
Return the \var{i}th element of \var{o}, or \NULL{} on failure. This
is the equivalent of the Python expression \samp{\var{o}[\var{i}]}.
\end{cfuncdesc}


\begin{cfuncdesc}{PyObject*}{PySequence_GetSlice}{PyObject *o, int i1, int i2}
Return the slice of sequence object \var{o} between \var{i1} and
\var{i2}, or \NULL{} on failure. This is the equivalent of the Python
expression \samp{\var{o}[\var{i1}:\var{i2}]}.
\end{cfuncdesc}


\begin{cfuncdesc}{int}{PySequence_SetItem}{PyObject *o, int i, PyObject *v}
Assign object \var{v} to the \var{i}th element of \var{o}.
Returns \code{-1} on failure.  This is the equivalent of the Python
statement \samp{\var{o}[\var{i}] = \var{v}}.
\end{cfuncdesc}

\begin{cfuncdesc}{int}{PySequence_DelItem}{PyObject *o, int i}
Delete the \var{i}th element of object \var{o}.  Returns
\code{-1} on failure.  This is the equivalent of the Python
statement \samp{del \var{o}[\var{i}]}.
\end{cfuncdesc}

\begin{cfuncdesc}{int}{PySequence_SetSlice}{PyObject *o, int i1,
                                            int i2, PyObject *v}
Assign the sequence object \var{v} to the slice in sequence
object \var{o} from \var{i1} to \var{i2}.  This is the equivalent of
the Python statement \samp{\var{o}[\var{i1}:\var{i2}] = \var{v}}.
\end{cfuncdesc}

\begin{cfuncdesc}{int}{PySequence_DelSlice}{PyObject *o, int i1, int i2}
Delete the slice in sequence object \var{o} from \var{i1} to \var{i2}.
Returns \code{-1} on failure. This is the equivalent of the Python
statement \samp{del \var{o}[\var{i1}:\var{i2}]}.
\end{cfuncdesc}

\begin{cfuncdesc}{PyObject*}{PySequence_Tuple}{PyObject *o}
Returns the \var{o} as a tuple on success, and \NULL{} on failure.
This is equivalent to the Python expression \samp{tuple(\var{o})}.
\bifuncindex{tuple}
\end{cfuncdesc}

\begin{cfuncdesc}{int}{PySequence_Count}{PyObject *o, PyObject *value}
Return the number of occurrences of \var{value} in \var{o}, that is,
return the number of keys for which \code{\var{o}[\var{key}] ==
\var{value}}.  On failure, return \code{-1}.  This is equivalent to
the Python expression \samp{\var{o}.count(\var{value})}.
\end{cfuncdesc}

\begin{cfuncdesc}{int}{PySequence_Contains}{PyObject *o, PyObject *value}
Determine if \var{o} contains \var{value}.  If an item in \var{o} is
equal to \var{value}, return \code{1}, otherwise return \code{0}.  On
error, return \code{-1}.  This is equivalent to the Python expression
\samp{\var{value} in \var{o}}.
\end{cfuncdesc}

\begin{cfuncdesc}{int}{PySequence_Index}{PyObject *o, PyObject *value}
Return the first index \var{i} for which \code{\var{o}[\var{i}] ==
\var{value}}.  On error, return \code{-1}.    This is equivalent to
the Python expression \samp{\var{o}.index(\var{value})}.
\end{cfuncdesc}

\begin{cfuncdesc}{PyObject*}{PySequence_List}{PyObject *o}
Return a list object with the same contents as the arbitrary sequence
\var{o}.  The returned list is guaranteed to be new.
\end{cfuncdesc}

\begin{cfuncdesc}{PyObject*}{PySequence_Tuple}{PyObject *o}
Return a tuple object with the same contents as the arbitrary sequence
\var{o}.  If \var{o} is a tuple, a new reference will be returned,
otherwise a tuple will be constructed with the appropriate contents.
\end{cfuncdesc}


\begin{cfuncdesc}{PyObject*}{PySequence_Fast}{PyObject *o, const char *m}
Returns the sequence \var{o} as a tuple, unless it is already a
tuple or list, in which case \var{o} is returned.  Use
\cfunction{PySequence_Fast_GET_ITEM()} to access the members of the
result.  Returns \NULL{} on failure.  If the object is not a sequence,
raises \exception{TypeError} with \var{m} as the message text.
\end{cfuncdesc}

\begin{cfuncdesc}{PyObject*}{PySequence_Fast_GET_ITEM}{PyObject *o, int i}
Return the \var{i}th element of \var{o}, assuming that \var{o} was
returned by \cfunction{PySequence_Fast()}, and that \var{i} is within
bounds.  The caller is expected to get the length of the sequence by
calling \cfunction{PyObject_Size()} on \var{o}, since lists and tuples
are guaranteed to always return their true length.
\end{cfuncdesc}


\section{Mapping Protocol \label{mapping}}

\begin{cfuncdesc}{int}{PyMapping_Check}{PyObject *o}
Return \code{1} if the object provides mapping protocol, and
\code{0} otherwise.  This function always succeeds.
\end{cfuncdesc}


\begin{cfuncdesc}{int}{PyMapping_Length}{PyObject *o}
Returns the number of keys in object \var{o} on success, and
\code{-1} on failure.  For objects that do not provide mapping
protocol, this is equivalent to the Python expression
\samp{len(\var{o})}.\bifuncindex{len}
\end{cfuncdesc}


\begin{cfuncdesc}{int}{PyMapping_DelItemString}{PyObject *o, char *key}
Remove the mapping for object \var{key} from the object \var{o}.
Return \code{-1} on failure.  This is equivalent to
the Python statement \samp{del \var{o}[\var{key}]}.
\end{cfuncdesc}


\begin{cfuncdesc}{int}{PyMapping_DelItem}{PyObject *o, PyObject *key}
Remove the mapping for object \var{key} from the object \var{o}.
Return \code{-1} on failure.  This is equivalent to
the Python statement \samp{del \var{o}[\var{key}]}.
\end{cfuncdesc}


\begin{cfuncdesc}{int}{PyMapping_HasKeyString}{PyObject *o, char *key}
On success, return \code{1} if the mapping object has the key
\var{key} and \code{0} otherwise.  This is equivalent to the Python
expression \samp{\var{o}.has_key(\var{key})}. 
This function always succeeds.
\end{cfuncdesc}


\begin{cfuncdesc}{int}{PyMapping_HasKey}{PyObject *o, PyObject *key}
Return \code{1} if the mapping object has the key \var{key} and
\code{0} otherwise.  This is equivalent to the Python expression
\samp{\var{o}.has_key(\var{key})}. 
This function always succeeds.
\end{cfuncdesc}


\begin{cfuncdesc}{PyObject*}{PyMapping_Keys}{PyObject *o}
On success, return a list of the keys in object \var{o}.  On
failure, return \NULL{}. This is equivalent to the Python
expression \samp{\var{o}.keys()}.
\end{cfuncdesc}


\begin{cfuncdesc}{PyObject*}{PyMapping_Values}{PyObject *o}
On success, return a list of the values in object \var{o}.  On
failure, return \NULL{}. This is equivalent to the Python
expression \samp{\var{o}.values()}.
\end{cfuncdesc}


\begin{cfuncdesc}{PyObject*}{PyMapping_Items}{PyObject *o}
On success, return a list of the items in object \var{o}, where
each item is a tuple containing a key-value pair.  On
failure, return \NULL{}. This is equivalent to the Python
expression \samp{\var{o}.items()}.
\end{cfuncdesc}


\begin{cfuncdesc}{PyObject*}{PyMapping_GetItemString}{PyObject *o, char *key}
Return element of \var{o} corresponding to the object \var{key} or
\NULL{} on failure. This is the equivalent of the Python expression
\samp{\var{o}[\var{key}]}.
\end{cfuncdesc}

\begin{cfuncdesc}{int}{PyMapping_SetItemString}{PyObject *o, char *key,
                                                PyObject *v}
Map the object \var{key} to the value \var{v} in object \var{o}.
Returns \code{-1} on failure.  This is the equivalent of the Python
statement \samp{\var{o}[\var{key}] = \var{v}}.
\end{cfuncdesc}


\section{Iterator Protocol \label{iterator}}

\versionadded{2.2}

There are only a couple of functions specifically for working with
iterators.

\begin{cfuncdesc}{int}{PyIter_Check}{PyObject *o}
  Return true if the object \var{o} supports the iterator protocol.
\end{cfuncdesc}

\begin{cfuncdesc}{PyObject*}{PyIter_Next}{PyObject *o}
  Return the next value from the iteration \var{o}.  If the object is
  an iterator, this retrieves the next value from the iteration, and
  returns \NULL{} with no exception set if there are no remaining
  items.  If the object is not an iterator, \exception{TypeError} is
  raised, or if there is an error in retrieving the item, returns
  \NULL{} and passes along the exception.
\end{cfuncdesc}

To write a loop which iterates over an iterator, the C code should
look something like this:

\begin{verbatim}
PyObject *iterator = ...;
PyObject *item;

while (item = PyIter_Next(iter)) {
    /* do something with item */
}
if (PyErr_Occurred()) {
    /* propogate error */
}
else {
    /* continue doing useful work */
}
\end{verbatim}


\chapter{Concrete Objects Layer \label{concrete}}

The functions in this chapter are specific to certain Python object
types.  Passing them an object of the wrong type is not a good idea;
if you receive an object from a Python program and you are not sure
that it has the right type, you must perform a type check first;
for example, to check that an object is a dictionary, use
\cfunction{PyDict_Check()}.  The chapter is structured like the
``family tree'' of Python object types.

\strong{Warning:}
While the functions described in this chapter carefully check the type
of the objects which are passed in, many of them do not check for
\NULL{} being passed instead of a valid object.  Allowing \NULL{} to
be passed in can cause memory access violations and immediate
termination of the interpreter.


\section{Fundamental Objects \label{fundamental}}

This section describes Python type objects and the singleton object 
\code{None}.


\subsection{Type Objects \label{typeObjects}}

\obindex{type}
\begin{ctypedesc}{PyTypeObject}
The C structure of the objects used to describe built-in types.
\end{ctypedesc}

\begin{cvardesc}{PyObject*}{PyType_Type}
This is the type object for type objects; it is the same object as
\code{types.TypeType} in the Python layer.
\withsubitem{(in module types)}{\ttindex{TypeType}}
\end{cvardesc}

\begin{cfuncdesc}{int}{PyType_Check}{PyObject *o}
Returns true is the object \var{o} is a type object.
\end{cfuncdesc}

\begin{cfuncdesc}{int}{PyType_HasFeature}{PyObject *o, int feature}
Returns true if the type object \var{o} sets the feature
\var{feature}.  Type features are denoted by single bit flags.
\end{cfuncdesc}


\subsection{The None Object \label{noneObject}}

\obindex{None@\texttt{None}}
Note that the \ctype{PyTypeObject} for \code{None} is not directly
exposed in the Python/C API.  Since \code{None} is a singleton,
testing for object identity (using \samp{==} in C) is sufficient.
There is no \cfunction{PyNone_Check()} function for the same reason.

\begin{cvardesc}{PyObject*}{Py_None}
The Python \code{None} object, denoting lack of value.  This object has
no methods.
\end{cvardesc}


\section{Sequence Objects \label{sequenceObjects}}

\obindex{sequence}
Generic operations on sequence objects were discussed in the previous 
chapter; this section deals with the specific kinds of sequence 
objects that are intrinsic to the Python language.


\subsection{String Objects \label{stringObjects}}

These functions raise \exception{TypeError} when expecting a string
parameter and are called with a non-string parameter.

\obindex{string}
\begin{ctypedesc}{PyStringObject}
This subtype of \ctype{PyObject} represents a Python string object.
\end{ctypedesc}

\begin{cvardesc}{PyTypeObject}{PyString_Type}
This instance of \ctype{PyTypeObject} represents the Python string
type; it is the same object as \code{types.TypeType} in the Python
layer.\withsubitem{(in module types)}{\ttindex{StringType}}.
\end{cvardesc}

\begin{cfuncdesc}{int}{PyString_Check}{PyObject *o}
Returns true if the object \var{o} is a string object.
\end{cfuncdesc}

\begin{cfuncdesc}{PyObject*}{PyString_FromString}{const char *v}
Returns a new string object with the value \var{v} on success, and
\NULL{} on failure.
\end{cfuncdesc}

\begin{cfuncdesc}{PyObject*}{PyString_FromStringAndSize}{const char *v,
                                                         int len}
Returns a new string object with the value \var{v} and length
\var{len} on success, and \NULL{} on failure.  If \var{v} is \NULL{},
the contents of the string are uninitialized.
\end{cfuncdesc}

\begin{cfuncdesc}{int}{PyString_Size}{PyObject *string}
Returns the length of the string in string object \var{string}.
\end{cfuncdesc}

\begin{cfuncdesc}{int}{PyString_GET_SIZE}{PyObject *string}
Macro form of \cfunction{PyString_Size()} but without error
checking.
\end{cfuncdesc}

\begin{cfuncdesc}{char*}{PyString_AsString}{PyObject *string}
Returns a null-terminated representation of the contents of
\var{string}.  The pointer refers to the internal buffer of
\var{string}, not a copy.  The data must not be modified in any way,
unless the string was just created using
\code{PyString_FromStringAndSize(NULL, \var{size})}.
It must not be deallocated.
\end{cfuncdesc}

\begin{cfuncdesc}{char*}{PyString_AS_STRING}{PyObject *string}
Macro form of \cfunction{PyString_AsString()} but without error
checking.
\end{cfuncdesc}

\begin{cfuncdesc}{int}{PyString_AsStringAndSize}{PyObject *obj,
                                                 char **buffer,
                                                 int *length}
Returns a null-terminated representation of the contents of the object
\var{obj} through the output variables \var{buffer} and \var{length}.

The function accepts both string and Unicode objects as input. For
Unicode objects it returns the default encoded version of the object.
If \var{length} is set to \NULL{}, the resulting buffer may not contain
null characters; if it does, the function returns -1 and a
TypeError is raised.

The buffer refers to an internal string buffer of \var{obj}, not a
copy. The data must not be modified in any way, unless the string was
just created using \code{PyString_FromStringAndSize(NULL,
\var{size})}.  It must not be deallocated.
\end{cfuncdesc}

\begin{cfuncdesc}{void}{PyString_Concat}{PyObject **string,
                                         PyObject *newpart}
Creates a new string object in \var{*string} containing the
contents of \var{newpart} appended to \var{string}; the caller will
own the new reference.  The reference to the old value of \var{string}
will be stolen.  If the new string
cannot be created, the old reference to \var{string} will still be
discarded and the value of \var{*string} will be set to
\NULL{}; the appropriate exception will be set.
\end{cfuncdesc}

\begin{cfuncdesc}{void}{PyString_ConcatAndDel}{PyObject **string,
                                               PyObject *newpart}
Creates a new string object in \var{*string} containing the contents
of \var{newpart} appended to \var{string}.  This version decrements
the reference count of \var{newpart}.
\end{cfuncdesc}

\begin{cfuncdesc}{int}{_PyString_Resize}{PyObject **string, int newsize}
A way to resize a string object even though it is ``immutable''.  
Only use this to build up a brand new string object; don't use this if
the string may already be known in other parts of the code.
\end{cfuncdesc}

\begin{cfuncdesc}{PyObject*}{PyString_Format}{PyObject *format,
                                              PyObject *args}
Returns a new string object from \var{format} and \var{args}.  Analogous
to \code{\var{format} \%\ \var{args}}.  The \var{args} argument must be
a tuple.
\end{cfuncdesc}

\begin{cfuncdesc}{void}{PyString_InternInPlace}{PyObject **string}
Intern the argument \var{*string} in place.  The argument must be the
address of a pointer variable pointing to a Python string object.
If there is an existing interned string that is the same as
\var{*string}, it sets \var{*string} to it (decrementing the reference 
count of the old string object and incrementing the reference count of
the interned string object), otherwise it leaves \var{*string} alone
and interns it (incrementing its reference count).  (Clarification:
even though there is a lot of talk about reference counts, think of
this function as reference-count-neutral; you own the object after
the call if and only if you owned it before the call.)
\end{cfuncdesc}

\begin{cfuncdesc}{PyObject*}{PyString_InternFromString}{const char *v}
A combination of \cfunction{PyString_FromString()} and
\cfunction{PyString_InternInPlace()}, returning either a new string object
that has been interned, or a new (``owned'') reference to an earlier
interned string object with the same value.
\end{cfuncdesc}

\begin{cfuncdesc}{PyObject*}{PyString_Decode}{const char *s,
                                               int size,
                                               const char *encoding,
                                               const char *errors}
Creates an object by decoding \var{size} bytes of the encoded
buffer \var{s} using the codec registered
for \var{encoding}. \var{encoding} and \var{errors} have the same meaning
as the parameters of the same name in the unicode() builtin
function. The codec to be used is looked up using the Python codec
registry. Returns \NULL{} in case an exception was raised by the
codec.
\end{cfuncdesc}

\begin{cfuncdesc}{PyObject*}{PyString_AsDecodedObject}{PyObject *str,
                                               const char *encoding,
                                               const char *errors}
Decodes a string object by passing it to the codec registered
for \var{encoding} and returns the result as Python 
object. \var{encoding} and \var{errors} have the same meaning as the
parameters of the same name in the string .encode() method. The codec
to be used is looked up using the Python codec registry. Returns
\NULL{} in case an exception was raised by the codec.
\end{cfuncdesc}

\begin{cfuncdesc}{PyObject*}{PyString_Encode}{const char *s,
                                               int size,
                                               const char *encoding,
                                               const char *errors}
Encodes the \ctype{char} buffer of the given size by passing it to 
the codec registered for \var{encoding} and returns a Python object. 
\var{encoding} and \var{errors} have the same
meaning as the parameters of the same name in the string .encode()
method. The codec to be used is looked up using the Python codec
registry. Returns \NULL{} in case an exception was raised by the
codec.
\end{cfuncdesc}

\begin{cfuncdesc}{PyObject*}{PyString_AsEncodedObject}{PyObject *str,
                                               const char *encoding,
                                               const char *errors}
Encodes a string object using the codec registered
for \var{encoding} and returns the result as Python 
object. \var{encoding} and \var{errors} have the same meaning as the
parameters of the same name in the string .encode() method. The codec
to be used is looked up using the Python codec registry. Returns
\NULL{} in case an exception was raised by the codec.
\end{cfuncdesc}


\subsection{Unicode Objects \label{unicodeObjects}}
\sectionauthor{Marc-Andre Lemburg}{mal@lemburg.com}

%--- Unicode Type -------------------------------------------------------

These are the basic Unicode object types used for the Unicode
implementation in Python:

\begin{ctypedesc}{Py_UNICODE}
This type represents a 16-bit unsigned storage type which is used by
Python internally as basis for holding Unicode ordinals. On platforms
where \ctype{wchar_t} is available and also has 16-bits,
\ctype{Py_UNICODE} is a typedef alias for \ctype{wchar_t} to enhance
native platform compatibility. On all other platforms,
\ctype{Py_UNICODE} is a typedef alias for \ctype{unsigned short}.
\end{ctypedesc}

\begin{ctypedesc}{PyUnicodeObject}
This subtype of \ctype{PyObject} represents a Python Unicode object.
\end{ctypedesc}

\begin{cvardesc}{PyTypeObject}{PyUnicode_Type}
This instance of \ctype{PyTypeObject} represents the Python Unicode type.
\end{cvardesc}

%--- These are really C macros... is there a macrodesc TeX macro ?

The following APIs are really C macros and can be used to do fast
checks and to access internal read-only data of Unicode objects:

\begin{cfuncdesc}{int}{PyUnicode_Check}{PyObject *o}
Returns true if the object \var{o} is a Unicode object.
\end{cfuncdesc}

\begin{cfuncdesc}{int}{PyUnicode_GET_SIZE}{PyObject *o}
Returns the size of the object.  o has to be a
PyUnicodeObject (not checked).
\end{cfuncdesc}

\begin{cfuncdesc}{int}{PyUnicode_GET_DATA_SIZE}{PyObject *o}
Returns the size of the object's internal buffer in bytes. o has to be
a PyUnicodeObject (not checked).
\end{cfuncdesc}

\begin{cfuncdesc}{Py_UNICODE*}{PyUnicode_AS_UNICODE}{PyObject *o}
Returns a pointer to the internal Py_UNICODE buffer of the object. o
has to be a PyUnicodeObject (not checked).
\end{cfuncdesc}

\begin{cfuncdesc}{const char*}{PyUnicode_AS_DATA}{PyObject *o}
Returns a (const char *) pointer to the internal buffer of the object.
o has to be a PyUnicodeObject (not checked).
\end{cfuncdesc}

% --- Unicode character properties ---------------------------------------

Unicode provides many different character properties. The most often
needed ones are available through these macros which are mapped to C
functions depending on the Python configuration.

\begin{cfuncdesc}{int}{Py_UNICODE_ISSPACE}{Py_UNICODE ch}
Returns 1/0 depending on whether \var{ch} is a whitespace character.
\end{cfuncdesc}

\begin{cfuncdesc}{int}{Py_UNICODE_ISLOWER}{Py_UNICODE ch}
Returns 1/0 depending on whether \var{ch} is a lowercase character.
\end{cfuncdesc}

\begin{cfuncdesc}{int}{Py_UNICODE_ISUPPER}{Py_UNICODE ch}
Returns 1/0 depending on whether \var{ch} is an uppercase character.
\end{cfuncdesc}

\begin{cfuncdesc}{int}{Py_UNICODE_ISTITLE}{Py_UNICODE ch}
Returns 1/0 depending on whether \var{ch} is a titlecase character.
\end{cfuncdesc}

\begin{cfuncdesc}{int}{Py_UNICODE_ISLINEBREAK}{Py_UNICODE ch}
Returns 1/0 depending on whether \var{ch} is a linebreak character.
\end{cfuncdesc}

\begin{cfuncdesc}{int}{Py_UNICODE_ISDECIMAL}{Py_UNICODE ch}
Returns 1/0 depending on whether \var{ch} is a decimal character.
\end{cfuncdesc}

\begin{cfuncdesc}{int}{Py_UNICODE_ISDIGIT}{Py_UNICODE ch}
Returns 1/0 depending on whether \var{ch} is a digit character.
\end{cfuncdesc}

\begin{cfuncdesc}{int}{Py_UNICODE_ISNUMERIC}{Py_UNICODE ch}
Returns 1/0 depending on whether \var{ch} is a numeric character.
\end{cfuncdesc}

\begin{cfuncdesc}{int}{Py_UNICODE_ISALPHA}{Py_UNICODE ch}
Returns 1/0 depending on whether \var{ch} is an alphabetic character.
\end{cfuncdesc}

\begin{cfuncdesc}{int}{Py_UNICODE_ISALNUM}{Py_UNICODE ch}
Returns 1/0 depending on whether \var{ch} is an alphanumeric character.
\end{cfuncdesc}

These APIs can be used for fast direct character conversions:

\begin{cfuncdesc}{Py_UNICODE}{Py_UNICODE_TOLOWER}{Py_UNICODE ch}
Returns the character \var{ch} converted to lower case.
\end{cfuncdesc}

\begin{cfuncdesc}{Py_UNICODE}{Py_UNICODE_TOUPPER}{Py_UNICODE ch}
Returns the character \var{ch} converted to upper case.
\end{cfuncdesc}

\begin{cfuncdesc}{Py_UNICODE}{Py_UNICODE_TOTITLE}{Py_UNICODE ch}
Returns the character \var{ch} converted to title case.
\end{cfuncdesc}

\begin{cfuncdesc}{int}{Py_UNICODE_TODECIMAL}{Py_UNICODE ch}
Returns the character \var{ch} converted to a decimal positive integer.
Returns -1 in case this is not possible. Does not raise exceptions.
\end{cfuncdesc}

\begin{cfuncdesc}{int}{Py_UNICODE_TODIGIT}{Py_UNICODE ch}
Returns the character \var{ch} converted to a single digit integer.
Returns -1 in case this is not possible. Does not raise exceptions.
\end{cfuncdesc}

\begin{cfuncdesc}{double}{Py_UNICODE_TONUMERIC}{Py_UNICODE ch}
Returns the character \var{ch} converted to a (positive) double.
Returns -1.0 in case this is not possible. Does not raise exceptions.
\end{cfuncdesc}

% --- Plain Py_UNICODE ---------------------------------------------------

To create Unicode objects and access their basic sequence properties,
use these APIs:

\begin{cfuncdesc}{PyObject*}{PyUnicode_FromUnicode}{const Py_UNICODE *u,
                                                    int size} 

Create a Unicode Object from the Py_UNICODE buffer \var{u} of the
given size. \var{u} may be \NULL{} which causes the contents to be
undefined. It is the user's responsibility to fill in the needed data.
The buffer is copied into the new object. If the buffer is not \NULL{},
the return value might be a shared object. Therefore, modification of
the resulting Unicode Object is only allowed when \var{u} is \NULL{}.
\end{cfuncdesc}

\begin{cfuncdesc}{Py_UNICODE*}{PyUnicode_AsUnicode}{PyObject *unicode}
Return a read-only pointer to the Unicode object's internal
\ctype{Py_UNICODE} buffer.
\end{cfuncdesc}

\begin{cfuncdesc}{int}{PyUnicode_GetSize}{PyObject *unicode}
Return the length of the Unicode object.
\end{cfuncdesc}

\begin{cfuncdesc}{PyObject*}{PyUnicode_FromEncodedObject}{PyObject *obj,
                                                      const char *encoding,
                                                      const char *errors}

Coerce an encoded object obj to an Unicode object and return a
reference with incremented refcount.

Coercion is done in the following way:
\begin{enumerate}
\item  Unicode objects are passed back as-is with incremented
      refcount. Note: these cannot be decoded; passing a non-NULL
      value for encoding will result in a TypeError.

\item String and other char buffer compatible objects are decoded
      according to the given encoding and using the error handling
      defined by errors. Both can be NULL to have the interface use
      the default values (see the next section for details).

\item All other objects cause an exception.
\end{enumerate}
The API returns NULL in case of an error. The caller is responsible
for decref'ing the returned objects.
\end{cfuncdesc}

\begin{cfuncdesc}{PyObject*}{PyUnicode_FromObject}{PyObject *obj}

Shortcut for PyUnicode_FromEncodedObject(obj, NULL, ``strict'')
which is used throughout the interpreter whenever coercion to
Unicode is needed.
\end{cfuncdesc}

% --- wchar_t support for platforms which support it ---------------------

If the platform supports \ctype{wchar_t} and provides a header file
wchar.h, Python can interface directly to this type using the
following functions. Support is optimized if Python's own
\ctype{Py_UNICODE} type is identical to the system's \ctype{wchar_t}.

\begin{cfuncdesc}{PyObject*}{PyUnicode_FromWideChar}{const wchar_t *w,
                                                     int size}
Create a Unicode Object from the \ctype{whcar_t} buffer \var{w} of the
given size. Returns \NULL{} on failure.
\end{cfuncdesc}

\begin{cfuncdesc}{int}{PyUnicode_AsWideChar}{PyUnicodeObject *unicode,
                                             wchar_t *w,
                                             int size}
Copies the Unicode Object contents into the \ctype{whcar_t} buffer
\var{w}.  At most \var{size} \ctype{whcar_t} characters are copied.
Returns the number of \ctype{whcar_t} characters copied or -1 in case
of an error.
\end{cfuncdesc}


\subsubsection{Builtin Codecs \label{builtinCodecs}}

Python provides a set of builtin codecs which are written in C
for speed. All of these codecs are directly usable via the
following functions.

Many of the following APIs take two arguments encoding and
errors. These parameters encoding and errors have the same semantics
as the ones of the builtin unicode() Unicode object constructor.

Setting encoding to NULL causes the default encoding to be used which
is UTF-8.

Error handling is set by errors which may also be set to NULL meaning
to use the default handling defined for the codec. Default error
handling for all builtin codecs is ``strict'' (ValueErrors are raised).

The codecs all use a similar interface. Only deviation from the
following generic ones are documented for simplicity.

% --- Generic Codecs -----------------------------------------------------

These are the generic codec APIs:

\begin{cfuncdesc}{PyObject*}{PyUnicode_Decode}{const char *s,
                                               int size,
                                               const char *encoding,
                                               const char *errors}
Create a Unicode object by decoding \var{size} bytes of the encoded
string \var{s}. \var{encoding} and \var{errors} have the same meaning
as the parameters of the same name in the unicode() builtin
function. The codec to be used is looked up using the Python codec
registry. Returns \NULL{} in case an exception was raised by the
codec.
\end{cfuncdesc}

\begin{cfuncdesc}{PyObject*}{PyUnicode_Encode}{const Py_UNICODE *s,
                                               int size,
                                               const char *encoding,
                                               const char *errors}
Encodes the \ctype{Py_UNICODE} buffer of the given size and returns a
Python string object. \var{encoding} and \var{errors} have the same
meaning as the parameters of the same name in the Unicode .encode()
method. The codec to be used is looked up using the Python codec
registry. Returns \NULL{} in case an exception was raised by the
codec.
\end{cfuncdesc}

\begin{cfuncdesc}{PyObject*}{PyUnicode_AsEncodedString}{PyObject *unicode,
                                               const char *encoding,
                                               const char *errors}
Encodes a Unicode object and returns the result as Python string
object. \var{encoding} and \var{errors} have the same meaning as the
parameters of the same name in the Unicode .encode() method. The codec
to be used is looked up using the Python codec registry. Returns
\NULL{} in case an exception was raised by the codec.
\end{cfuncdesc}

% --- UTF-8 Codecs -------------------------------------------------------

These are the UTF-8 codec APIs:

\begin{cfuncdesc}{PyObject*}{PyUnicode_DecodeUTF8}{const char *s,
                                               int size,
                                               const char *errors}
Creates a Unicode object by decoding \var{size} bytes of the UTF-8
encoded string \var{s}. Returns \NULL{} in case an exception was
raised by the codec.
\end{cfuncdesc}

\begin{cfuncdesc}{PyObject*}{PyUnicode_EncodeUTF8}{const Py_UNICODE *s,
                                               int size,
                                               const char *errors}
Encodes the \ctype{Py_UNICODE} buffer of the given size using UTF-8
and returns a Python string object.  Returns \NULL{} in case an
exception was raised by the codec.
\end{cfuncdesc}

\begin{cfuncdesc}{PyObject*}{PyUnicode_AsUTF8String}{PyObject *unicode}
Encodes a Unicode objects using UTF-8 and returns the result as Python
string object. Error handling is ``strict''. Returns
\NULL{} in case an exception was raised by the codec.
\end{cfuncdesc}

% --- UTF-16 Codecs ------------------------------------------------------ */

These are the UTF-16 codec APIs:

\begin{cfuncdesc}{PyObject*}{PyUnicode_DecodeUTF16}{const char *s,
                                               int size,
                                               const char *errors,
                                               int *byteorder}
Decodes \var{length} bytes from a UTF-16 encoded buffer string and
returns the corresponding Unicode object.

\var{errors} (if non-NULL) defines the error handling. It defaults
to ``strict''.

If \var{byteorder} is non-\NULL{}, the decoder starts decoding using
the given byte order:

\begin{verbatim}
   *byteorder == -1: little endian
   *byteorder == 0:  native order
   *byteorder == 1:  big endian
\end{verbatim}

and then switches according to all byte order marks (BOM) it finds in
the input data. BOM marks are not copied into the resulting Unicode
string.  After completion, \var{*byteorder} is set to the current byte
order at the end of input data.

If \var{byteorder} is \NULL{}, the codec starts in native order mode.

Returns \NULL{} in case an exception was raised by the codec.
\end{cfuncdesc}

\begin{cfuncdesc}{PyObject*}{PyUnicode_EncodeUTF16}{const Py_UNICODE *s,
                                               int size,
                                               const char *errors,
                                               int byteorder}
Returns a Python string object holding the UTF-16 encoded value of the
Unicode data in \var{s}.

If \var{byteorder} is not \code{0}, output is written according to the
following byte order:

\begin{verbatim}
   byteorder == -1: little endian
   byteorder == 0:  native byte order (writes a BOM mark)
   byteorder == 1:  big endian
\end{verbatim}

If byteorder is \code{0}, the output string will always start with the
Unicode BOM mark (U+FEFF). In the other two modes, no BOM mark is
prepended.

Note that \ctype{Py_UNICODE} data is being interpreted as UTF-16
reduced to UCS-2. This trick makes it possible to add full UTF-16
capabilities at a later point without comprimising the APIs.

Returns \NULL{} in case an exception was raised by the codec.
\end{cfuncdesc}

\begin{cfuncdesc}{PyObject*}{PyUnicode_AsUTF16String}{PyObject *unicode}
Returns a Python string using the UTF-16 encoding in native byte
order. The string always starts with a BOM mark. Error handling is
``strict''. Returns \NULL{} in case an exception was raised by the
codec.
\end{cfuncdesc}

% --- Unicode-Escape Codecs ----------------------------------------------

These are the ``Unicode Esacpe'' codec APIs:

\begin{cfuncdesc}{PyObject*}{PyUnicode_DecodeUnicodeEscape}{const char *s,
                                               int size,
                                               const char *errors}
Creates a Unicode object by decoding \var{size} bytes of the Unicode-Esacpe
encoded string \var{s}. Returns \NULL{} in case an exception was
raised by the codec.
\end{cfuncdesc}

\begin{cfuncdesc}{PyObject*}{PyUnicode_EncodeUnicodeEscape}{const Py_UNICODE *s,
                                               int size,
                                               const char *errors}
Encodes the \ctype{Py_UNICODE} buffer of the given size using Unicode-Escape
and returns a Python string object.  Returns \NULL{} in case an
exception was raised by the codec.
\end{cfuncdesc}

\begin{cfuncdesc}{PyObject*}{PyUnicode_AsUnicodeEscapeString}{PyObject *unicode}
Encodes a Unicode objects using Unicode-Escape and returns the result
as Python string object. Error handling is ``strict''. Returns
\NULL{} in case an exception was raised by the codec.
\end{cfuncdesc}

% --- Raw-Unicode-Escape Codecs ------------------------------------------

These are the ``Raw Unicode Esacpe'' codec APIs:

\begin{cfuncdesc}{PyObject*}{PyUnicode_DecodeRawUnicodeEscape}{const char *s,
                                               int size,
                                               const char *errors}
Creates a Unicode object by decoding \var{size} bytes of the Raw-Unicode-Esacpe
encoded string \var{s}. Returns \NULL{} in case an exception was
raised by the codec.
\end{cfuncdesc}

\begin{cfuncdesc}{PyObject*}{PyUnicode_EncodeRawUnicodeEscape}{const Py_UNICODE *s,
                                               int size,
                                               const char *errors}
Encodes the \ctype{Py_UNICODE} buffer of the given size using Raw-Unicode-Escape
and returns a Python string object.  Returns \NULL{} in case an
exception was raised by the codec.
\end{cfuncdesc}

\begin{cfuncdesc}{PyObject*}{PyUnicode_AsRawUnicodeEscapeString}{PyObject *unicode}
Encodes a Unicode objects using Raw-Unicode-Escape and returns the result
as Python string object. Error handling is ``strict''. Returns
\NULL{} in case an exception was raised by the codec.
\end{cfuncdesc}

% --- Latin-1 Codecs ----------------------------------------------------- 

These are the Latin-1 codec APIs:

Latin-1 corresponds to the first 256 Unicode ordinals and only these
are accepted by the codecs during encoding.

\begin{cfuncdesc}{PyObject*}{PyUnicode_DecodeLatin1}{const char *s,
                                                     int size,
                                                     const char *errors}
Creates a Unicode object by decoding \var{size} bytes of the Latin-1
encoded string \var{s}. Returns \NULL{} in case an exception was
raised by the codec.
\end{cfuncdesc}

\begin{cfuncdesc}{PyObject*}{PyUnicode_EncodeLatin1}{const Py_UNICODE *s,
                                                     int size,
                                                     const char *errors}
Encodes the \ctype{Py_UNICODE} buffer of the given size using Latin-1
and returns a Python string object.  Returns \NULL{} in case an
exception was raised by the codec.
\end{cfuncdesc}

\begin{cfuncdesc}{PyObject*}{PyUnicode_AsLatin1String}{PyObject *unicode}
Encodes a Unicode objects using Latin-1 and returns the result as
Python string object. Error handling is ``strict''. Returns
\NULL{} in case an exception was raised by the codec.
\end{cfuncdesc}

% --- ASCII Codecs ------------------------------------------------------- 

These are the \ASCII{} codec APIs.  Only 7-bit \ASCII{} data is
accepted. All other codes generate errors.

\begin{cfuncdesc}{PyObject*}{PyUnicode_DecodeASCII}{const char *s,
                                                    int size,
                                                    const char *errors}
Creates a Unicode object by decoding \var{size} bytes of the
\ASCII{} encoded string \var{s}. Returns \NULL{} in case an exception
was raised by the codec.
\end{cfuncdesc}

\begin{cfuncdesc}{PyObject*}{PyUnicode_EncodeASCII}{const Py_UNICODE *s,
                                                    int size,
                                                    const char *errors}
Encodes the \ctype{Py_UNICODE} buffer of the given size using
\ASCII{} and returns a Python string object.  Returns \NULL{} in case
an exception was raised by the codec.
\end{cfuncdesc}

\begin{cfuncdesc}{PyObject*}{PyUnicode_AsASCIIString}{PyObject *unicode}
Encodes a Unicode objects using \ASCII{} and returns the result as Python
string object. Error handling is ``strict''. Returns
\NULL{} in case an exception was raised by the codec.
\end{cfuncdesc}

% --- Character Map Codecs ----------------------------------------------- 

These are the mapping codec APIs:

This codec is special in that it can be used to implement many
different codecs (and this is in fact what was done to obtain most of
the standard codecs included in the \module{encodings} package). The
codec uses mapping to encode and decode characters.

Decoding mappings must map single string characters to single Unicode
characters, integers (which are then interpreted as Unicode ordinals)
or None (meaning "undefined mapping" and causing an error). 

Encoding mappings must map single Unicode characters to single string
characters, integers (which are then interpreted as Latin-1 ordinals)
or None (meaning "undefined mapping" and causing an error).

The mapping objects provided must only support the __getitem__ mapping
interface.

If a character lookup fails with a LookupError, the character is
copied as-is meaning that its ordinal value will be interpreted as
Unicode or Latin-1 ordinal resp. Because of this, mappings only need
to contain those mappings which map characters to different code
points.

\begin{cfuncdesc}{PyObject*}{PyUnicode_DecodeCharmap}{const char *s,
                                               int size,
                                               PyObject *mapping,
                                               const char *errors}
Creates a Unicode object by decoding \var{size} bytes of the encoded
string \var{s} using the given \var{mapping} object.  Returns \NULL{}
in case an exception was raised by the codec.
\end{cfuncdesc}

\begin{cfuncdesc}{PyObject*}{PyUnicode_EncodeCharmap}{const Py_UNICODE *s,
                                               int size,
                                               PyObject *mapping,
                                               const char *errors}
Encodes the \ctype{Py_UNICODE} buffer of the given size using the
given \var{mapping} object and returns a Python string object.
Returns \NULL{} in case an exception was raised by the codec.
\end{cfuncdesc}

\begin{cfuncdesc}{PyObject*}{PyUnicode_AsCharmapString}{PyObject *unicode,
                                                        PyObject *mapping}
Encodes a Unicode objects using the given \var{mapping} object and
returns the result as Python string object. Error handling is
``strict''. Returns \NULL{} in case an exception was raised by the
codec.
\end{cfuncdesc}

The following codec API is special in that maps Unicode to Unicode.

\begin{cfuncdesc}{PyObject*}{PyUnicode_TranslateCharmap}{const Py_UNICODE *s,
                                               int size,
                                               PyObject *table,
                                               const char *errors}
Translates a \ctype{Py_UNICODE} buffer of the given length by applying
a character mapping \var{table} to it and returns the resulting
Unicode object.  Returns \NULL{} when an exception was raised by the
codec.

The \var{mapping} table must map Unicode ordinal integers to Unicode
ordinal integers or None (causing deletion of the character).

Mapping tables must only provide the __getitem__ interface,
e.g. dictionaries or sequences. Unmapped character ordinals (ones
which cause a LookupError) are left untouched and are copied as-is.
\end{cfuncdesc}

% --- MBCS codecs for Windows --------------------------------------------

These are the MBCS codec APIs. They are currently only available on
Windows and use the Win32 MBCS converters to implement the
conversions.  Note that MBCS (or DBCS) is a class of encodings, not
just one.  The target encoding is defined by the user settings on the
machine running the codec.

\begin{cfuncdesc}{PyObject*}{PyUnicode_DecodeMBCS}{const char *s,
                                               int size,
                                               const char *errors}
Creates a Unicode object by decoding \var{size} bytes of the MBCS
encoded string \var{s}.  Returns \NULL{} in case an exception was
raised by the codec.
\end{cfuncdesc}

\begin{cfuncdesc}{PyObject*}{PyUnicode_EncodeMBCS}{const Py_UNICODE *s,
                                               int size,
                                               const char *errors}
Encodes the \ctype{Py_UNICODE} buffer of the given size using MBCS
and returns a Python string object.  Returns \NULL{} in case an
exception was raised by the codec.
\end{cfuncdesc}

\begin{cfuncdesc}{PyObject*}{PyUnicode_AsMBCSString}{PyObject *unicode}
Encodes a Unicode objects using MBCS and returns the result as Python
string object.  Error handling is ``strict''.  Returns \NULL{} in case
an exception was raised by the codec.
\end{cfuncdesc}

% --- Methods & Slots ----------------------------------------------------

\subsubsection{Methods and Slot Functions \label{unicodeMethodsAndSlots}}

The following APIs are capable of handling Unicode objects and strings
on input (we refer to them as strings in the descriptions) and return
Unicode objects or integers as apporpriate.

They all return \NULL{} or -1 in case an exception occurrs.

\begin{cfuncdesc}{PyObject*}{PyUnicode_Concat}{PyObject *left,
                                               PyObject *right}
Concat two strings giving a new Unicode string.
\end{cfuncdesc}

\begin{cfuncdesc}{PyObject*}{PyUnicode_Split}{PyObject *s,
                                              PyObject *sep,
                                              int maxsplit}
Split a string giving a list of Unicode strings.

If sep is NULL, splitting will be done at all whitespace
substrings. Otherwise, splits occur at the given separator.

At most maxsplit splits will be done. If negative, no limit is set.

Separators are not included in the resulting list.
\end{cfuncdesc}

\begin{cfuncdesc}{PyObject*}{PyUnicode_Splitlines}{PyObject *s,
                                                   int maxsplit}
Split a Unicode string at line breaks, returning a list of Unicode
strings.  CRLF is considered to be one line break.  The Line break
characters are not included in the resulting strings.
\end{cfuncdesc}

\begin{cfuncdesc}{PyObject*}{PyUnicode_Translate}{PyObject *str,
                                                  PyObject *table,
                                                  const char *errors}
Translate a string by applying a character mapping table to it and
return the resulting Unicode object.

The mapping table must map Unicode ordinal integers to Unicode ordinal
integers or None (causing deletion of the character).

Mapping tables must only provide the __getitem__ interface,
e.g. dictionaries or sequences. Unmapped character ordinals (ones
which cause a LookupError) are left untouched and are copied as-is.

\var{errors} has the usual meaning for codecs. It may be \NULL{}
which indicates to use the default error handling.
\end{cfuncdesc}

\begin{cfuncdesc}{PyObject*}{PyUnicode_Join}{PyObject *separator,
                                             PyObject *seq}
Join a sequence of strings using the given separator and return
the resulting Unicode string.
\end{cfuncdesc}

\begin{cfuncdesc}{PyObject*}{PyUnicode_Tailmatch}{PyObject *str,
                                                  PyObject *substr,
                                                  int start,
                                                  int end,
                                                  int direction}
Return 1 if \var{substr} matches \var{str}[\var{start}:\var{end}] at
the given tail end (\var{direction} == -1 means to do a prefix match,
\var{direction} == 1 a suffix match), 0 otherwise.
\end{cfuncdesc}

\begin{cfuncdesc}{PyObject*}{PyUnicode_Find}{PyObject *str,
                                                  PyObject *substr,
                                                  int start,
                                                  int end,
                                                  int direction}
Return the first position of \var{substr} in
\var{str}[\var{start}:\var{end}] using the given \var{direction}
(\var{direction} == 1 means to do a forward search,
\var{direction} == -1 a backward search), 0 otherwise.
\end{cfuncdesc}

\begin{cfuncdesc}{PyObject*}{PyUnicode_Count}{PyObject *str,
                                                  PyObject *substr,
                                                  int start,
                                                  int end}
Count the number of occurrences of \var{substr} in
\var{str}[\var{start}:\var{end}]
\end{cfuncdesc}

\begin{cfuncdesc}{PyObject*}{PyUnicode_Replace}{PyObject *str,
                                                PyObject *substr,
                                                PyObject *replstr,
                                                int maxcount}
Replace at most \var{maxcount} occurrences of \var{substr} in
\var{str} with \var{replstr} and return the resulting Unicode object.
\var{maxcount} == -1 means: replace all occurrences.
\end{cfuncdesc}

\begin{cfuncdesc}{int}{PyUnicode_Compare}{PyObject *left, PyObject *right}
Compare two strings and return -1, 0, 1 for less than, equal,
greater than resp.
\end{cfuncdesc}

\begin{cfuncdesc}{PyObject*}{PyUnicode_Format}{PyObject *format,
                                              PyObject *args}
Returns a new string object from \var{format} and \var{args}; this is
analogous to \code{\var{format} \%\ \var{args}}.  The
\var{args} argument must be a tuple.
\end{cfuncdesc}

\begin{cfuncdesc}{int}{PyUnicode_Contains}{PyObject *container,
                                           PyObject *element}
Checks whether \var{element} is contained in \var{container} and
returns true or false accordingly.

\var{element} has to coerce to a one element Unicode string. \code{-1} is
returned in case of an error.
\end{cfuncdesc}


\subsection{Buffer Objects \label{bufferObjects}}
\sectionauthor{Greg Stein}{gstein@lyra.org}

\obindex{buffer}
Python objects implemented in C can export a group of functions called
the ``buffer\index{buffer interface} interface.''  These functions can
be used by an object to expose its data in a raw, byte-oriented
format. Clients of the object can use the buffer interface to access
the object data directly, without needing to copy it first.

Two examples of objects that support 
the buffer interface are strings and arrays. The string object exposes 
the character contents in the buffer interface's byte-oriented
form. An array can also expose its contents, but it should be noted
that array elements may be multi-byte values.

An example user of the buffer interface is the file object's
\method{write()} method. Any object that can export a series of bytes
through the buffer interface can be written to a file. There are a
number of format codes to \cfunction{PyArgs_ParseTuple()} that operate 
against an object's buffer interface, returning data from the target
object.

More information on the buffer interface is provided in the section
``Buffer Object Structures'' (section \ref{buffer-structs}), under
the description for \ctype{PyBufferProcs}\ttindex{PyBufferProcs}.

A ``buffer object'' is defined in the \file{bufferobject.h} header
(included by \file{Python.h}). These objects look very similar to
string objects at the Python programming level: they support slicing,
indexing, concatenation, and some other standard string
operations. However, their data can come from one of two sources: from
a block of memory, or from another object which exports the buffer
interface.

Buffer objects are useful as a way to expose the data from another
object's buffer interface to the Python programmer. They can also be
used as a zero-copy slicing mechanism. Using their ability to
reference a block of memory, it is possible to expose any data to the
Python programmer quite easily. The memory could be a large, constant
array in a C extension, it could be a raw block of memory for
manipulation before passing to an operating system library, or it
could be used to pass around structured data in its native, in-memory
format.

\begin{ctypedesc}{PyBufferObject}
This subtype of \ctype{PyObject} represents a buffer object.
\end{ctypedesc}

\begin{cvardesc}{PyTypeObject}{PyBuffer_Type}
The instance of \ctype{PyTypeObject} which represents the Python
buffer type; it is the same object as \code{types.BufferType} in the
Python layer.\withsubitem{(in module types)}{\ttindex{BufferType}}.
\end{cvardesc}

\begin{cvardesc}{int}{Py_END_OF_BUFFER}
This constant may be passed as the \var{size} parameter to
\cfunction{PyBuffer_FromObject()} or
\cfunction{PyBuffer_FromReadWriteObject()}. It indicates that the new
\ctype{PyBufferObject} should refer to \var{base} object from the
specified \var{offset} to the end of its exported buffer. Using this
enables the caller to avoid querying the \var{base} object for its
length.
\end{cvardesc}

\begin{cfuncdesc}{int}{PyBuffer_Check}{PyObject *p}
Return true if the argument has type \cdata{PyBuffer_Type}.
\end{cfuncdesc}

\begin{cfuncdesc}{PyObject*}{PyBuffer_FromObject}{PyObject *base,
                                                  int offset, int size}
Return a new read-only buffer object.  This raises
\exception{TypeError} if \var{base} doesn't support the read-only
buffer protocol or doesn't provide exactly one buffer segment, or it
raises \exception{ValueError} if \var{offset} is less than zero. The
buffer will hold a reference to the \var{base} object, and the
buffer's contents will refer to the \var{base} object's buffer
interface, starting as position \var{offset} and extending for
\var{size} bytes. If \var{size} is \constant{Py_END_OF_BUFFER}, then
the new buffer's contents extend to the length of the
\var{base} object's exported buffer data.
\end{cfuncdesc}

\begin{cfuncdesc}{PyObject*}{PyBuffer_FromReadWriteObject}{PyObject *base,
                                                           int offset,
                                                           int size}
Return a new writable buffer object.  Parameters and exceptions are
similar to those for \cfunction{PyBuffer_FromObject()}.
If the \var{base} object does not export the writeable buffer
protocol, then \exception{TypeError} is raised.
\end{cfuncdesc}

\begin{cfuncdesc}{PyObject*}{PyBuffer_FromMemory}{void *ptr, int size}
Return a new read-only buffer object that reads from a specified
location in memory, with a specified size.
The caller is responsible for ensuring that the memory buffer, passed
in as \var{ptr}, is not deallocated while the returned buffer object
exists.  Raises \exception{ValueError} if \var{size} is less than
zero.  Note that \constant{Py_END_OF_BUFFER} may \emph{not} be passed
for the \var{size} parameter; \exception{ValueError} will be raised in 
that case.
\end{cfuncdesc}

\begin{cfuncdesc}{PyObject*}{PyBuffer_FromReadWriteMemory}{void *ptr, int size}
Similar to \cfunction{PyBuffer_FromMemory()}, but the returned buffer
is writable.
\end{cfuncdesc}

\begin{cfuncdesc}{PyObject*}{PyBuffer_New}{int size}
Returns a new writable buffer object that maintains its own memory
buffer of \var{size} bytes.  \exception{ValueError} is returned if
\var{size} is not zero or positive.
\end{cfuncdesc}


\subsection{Tuple Objects \label{tupleObjects}}

\obindex{tuple}
\begin{ctypedesc}{PyTupleObject}
This subtype of \ctype{PyObject} represents a Python tuple object.
\end{ctypedesc}

\begin{cvardesc}{PyTypeObject}{PyTuple_Type}
This instance of \ctype{PyTypeObject} represents the Python tuple
type; it is the same object as \code{types.TupleType} in the Python
layer.\withsubitem{(in module types)}{\ttindex{TupleType}}.
\end{cvardesc}

\begin{cfuncdesc}{int}{PyTuple_Check}{PyObject *p}
Return true if the argument is a tuple object.
\end{cfuncdesc}

\begin{cfuncdesc}{PyObject*}{PyTuple_New}{int len}
Return a new tuple object of size \var{len}, or \NULL{} on failure.
\end{cfuncdesc}

\begin{cfuncdesc}{int}{PyTuple_Size}{PyObject *p}
Takes a pointer to a tuple object, and returns the size
of that tuple.
\end{cfuncdesc}

\begin{cfuncdesc}{PyObject*}{PyTuple_GetItem}{PyObject *p, int pos}
Returns the object at position \var{pos} in the tuple pointed
to by \var{p}.  If \var{pos} is out of bounds, returns \NULL{} and
sets an \exception{IndexError} exception.
\end{cfuncdesc}

\begin{cfuncdesc}{PyObject*}{PyTuple_GET_ITEM}{PyObject *p, int pos}
Does the same, but does no checking of its arguments.
\end{cfuncdesc}

\begin{cfuncdesc}{PyObject*}{PyTuple_GetSlice}{PyObject *p,
                                               int low, int high}
Takes a slice of the tuple pointed to by \var{p} from
\var{low} to \var{high} and returns it as a new tuple.
\end{cfuncdesc}

\begin{cfuncdesc}{int}{PyTuple_SetItem}{PyObject *p,
                                        int pos, PyObject *o}
Inserts a reference to object \var{o} at position \var{pos} of
the tuple pointed to by \var{p}. It returns \code{0} on success.
\strong{Note:}  This function ``steals'' a reference to \var{o}.
\end{cfuncdesc}

\begin{cfuncdesc}{void}{PyTuple_SET_ITEM}{PyObject *p,
                                          int pos, PyObject *o}
Does the same, but does no error checking, and
should \emph{only} be used to fill in brand new tuples.
\strong{Note:}  This function ``steals'' a reference to \var{o}.
\end{cfuncdesc}

\begin{cfuncdesc}{int}{_PyTuple_Resize}{PyObject **p,
                                        int newsize, int last_is_sticky}
Can be used to resize a tuple.  \var{newsize} will be the new length
of the tuple.  Because tuples are \emph{supposed} to be immutable,
this should only be used if there is only one reference to the object.
Do \emph{not} use this if the tuple may already be known to some other
part of the code.  The tuple will always grow or shrink at the end.  The
\var{last_is_sticky} flag is not used and should always be false.  Think
of this as destroying the old tuple and creating a new one, only more
efficiently.  Returns \code{0} on success and \code{-1} on failure (in
which case a \exception{MemoryError} or \exception{SystemError} will be
raised).
\end{cfuncdesc}


\subsection{List Objects \label{listObjects}}

\obindex{list}
\begin{ctypedesc}{PyListObject}
This subtype of \ctype{PyObject} represents a Python list object.
\end{ctypedesc}

\begin{cvardesc}{PyTypeObject}{PyList_Type}
This instance of \ctype{PyTypeObject} represents the Python list
type.  This is the same object as \code{types.ListType}.
\withsubitem{(in module types)}{\ttindex{ListType}}
\end{cvardesc}

\begin{cfuncdesc}{int}{PyList_Check}{PyObject *p}
Returns true if its argument is a \ctype{PyListObject}.
\end{cfuncdesc}

\begin{cfuncdesc}{PyObject*}{PyList_New}{int len}
Returns a new list of length \var{len} on success, or \NULL{} on
failure.
\end{cfuncdesc}

\begin{cfuncdesc}{int}{PyList_Size}{PyObject *list}
Returns the length of the list object in \var{list}; this is
equivalent to \samp{len(\var{list})} on a list object.
\bifuncindex{len}
\end{cfuncdesc}

\begin{cfuncdesc}{int}{PyList_GET_SIZE}{PyObject *list}
Macro form of \cfunction{PyList_Size()} without error checking.
\end{cfuncdesc}

\begin{cfuncdesc}{PyObject*}{PyList_GetItem}{PyObject *list, int index}
Returns the object at position \var{pos} in the list pointed
to by \var{p}.  If \var{pos} is out of bounds, returns \NULL{} and
sets an \exception{IndexError} exception.
\end{cfuncdesc}

\begin{cfuncdesc}{PyObject*}{PyList_GET_ITEM}{PyObject *list, int i}
Macro form of \cfunction{PyList_GetItem()} without error checking.
\end{cfuncdesc}

\begin{cfuncdesc}{int}{PyList_SetItem}{PyObject *list, int index,
                                       PyObject *item}
Sets the item at index \var{index} in list to \var{item}.
\strong{Note:}  This function ``steals'' a reference to \var{item}.
\end{cfuncdesc}

\begin{cfuncdesc}{PyObject*}{PyList_SET_ITEM}{PyObject *list, int i,
                                              PyObject *o}
Macro form of \cfunction{PyList_SetItem()} without error checking.
\strong{Note:}  This function ``steals'' a reference to \var{item}.
\end{cfuncdesc}

\begin{cfuncdesc}{int}{PyList_Insert}{PyObject *list, int index,
                                      PyObject *item}
Inserts the item \var{item} into list \var{list} in front of index
\var{index}.  Returns \code{0} if successful; returns \code{-1} and
raises an exception if unsuccessful.  Analogous to
\code{\var{list}.insert(\var{index}, \var{item})}.
\end{cfuncdesc}

\begin{cfuncdesc}{int}{PyList_Append}{PyObject *list, PyObject *item}
Appends the object \var{item} at the end of list \var{list}.  Returns
\code{0} if successful; returns \code{-1} and sets an exception if
unsuccessful.  Analogous to \code{\var{list}.append(\var{item})}.
\end{cfuncdesc}

\begin{cfuncdesc}{PyObject*}{PyList_GetSlice}{PyObject *list,
                                              int low, int high}
Returns a list of the objects in \var{list} containing the objects 
\emph{between} \var{low} and \var{high}.  Returns NULL and sets an
exception if unsuccessful.
Analogous to \code{\var{list}[\var{low}:\var{high}]}.
\end{cfuncdesc}

\begin{cfuncdesc}{int}{PyList_SetSlice}{PyObject *list,
                                        int low, int high,
                                        PyObject *itemlist}
Sets the slice of \var{list} between \var{low} and \var{high} to the
contents of \var{itemlist}.  Analogous to
\code{\var{list}[\var{low}:\var{high}] = \var{itemlist}}.  Returns
\code{0} on success, \code{-1} on failure.
\end{cfuncdesc}

\begin{cfuncdesc}{int}{PyList_Sort}{PyObject *list}
Sorts the items of \var{list} in place.  Returns \code{0} on success,
\code{-1} on failure.  This is equivalent to
\samp{\var{list}.sort()}.
\end{cfuncdesc}

\begin{cfuncdesc}{int}{PyList_Reverse}{PyObject *list}
Reverses the items of \var{list} in place.  Returns \code{0} on
success, \code{-1} on failure.  This is the equivalent of
\samp{\var{list}.reverse()}.
\end{cfuncdesc}

\begin{cfuncdesc}{PyObject*}{PyList_AsTuple}{PyObject *list}
Returns a new tuple object containing the contents of \var{list};
equivalent to \samp{tuple(\var{list})}.\bifuncindex{tuple}
\end{cfuncdesc}


\section{Mapping Objects \label{mapObjects}}

\obindex{mapping}


\subsection{Dictionary Objects \label{dictObjects}}

\obindex{dictionary}
\begin{ctypedesc}{PyDictObject}
This subtype of \ctype{PyObject} represents a Python dictionary object.
\end{ctypedesc}

\begin{cvardesc}{PyTypeObject}{PyDict_Type}
This instance of \ctype{PyTypeObject} represents the Python dictionary 
type.  This is exposed to Python programs as \code{types.DictType} and 
\code{types.DictionaryType}.
\withsubitem{(in module types)}{\ttindex{DictType}\ttindex{DictionaryType}}
\end{cvardesc}

\begin{cfuncdesc}{int}{PyDict_Check}{PyObject *p}
Returns true if its argument is a \ctype{PyDictObject}.
\end{cfuncdesc}

\begin{cfuncdesc}{PyObject*}{PyDict_New}{}
Returns a new empty dictionary, or \NULL{} on failure.
\end{cfuncdesc}

\begin{cfuncdesc}{void}{PyDict_Clear}{PyObject *p}
Empties an existing dictionary of all key-value pairs.
\end{cfuncdesc}

\begin{cfuncdesc}{PyObject*}{PyDict_Copy}{PyObject *p}
Returns a new dictionary that contains the same key-value pairs as p.
Empties an existing dictionary of all key-value pairs.
\end{cfuncdesc}

\begin{cfuncdesc}{int}{PyDict_SetItem}{PyObject *p, PyObject *key,
                                       PyObject *val}
Inserts \var{value} into the dictionary with a key of \var{key}.
\var{key} must be hashable; if it isn't, \exception{TypeError} will be 
raised.
\end{cfuncdesc}

\begin{cfuncdesc}{int}{PyDict_SetItemString}{PyObject *p,
            char *key,
            PyObject *val}
Inserts \var{value} into the dictionary using \var{key}
as a key. \var{key} should be a \ctype{char*}.  The key object is
created using \code{PyString_FromString(\var{key})}.
\ttindex{PyString_FromString()}
\end{cfuncdesc}

\begin{cfuncdesc}{int}{PyDict_DelItem}{PyObject *p, PyObject *key}
Removes the entry in dictionary \var{p} with key \var{key}.
\var{key} must be hashable; if it isn't, \exception{TypeError} is
raised.
\end{cfuncdesc}

\begin{cfuncdesc}{int}{PyDict_DelItemString}{PyObject *p, char *key}
Removes the entry in dictionary \var{p} which has a key
specified by the string \var{key}.
\end{cfuncdesc}

\begin{cfuncdesc}{PyObject*}{PyDict_GetItem}{PyObject *p, PyObject *key}
Returns the object from dictionary \var{p} which has a key
\var{key}.  Returns \NULL{} if the key \var{key} is not present, but
\emph{without} setting an exception.
\end{cfuncdesc}

\begin{cfuncdesc}{PyObject*}{PyDict_GetItemString}{PyObject *p, char *key}
This is the same as \cfunction{PyDict_GetItem()}, but \var{key} is
specified as a \ctype{char*}, rather than a \ctype{PyObject*}.
\end{cfuncdesc}

\begin{cfuncdesc}{PyObject*}{PyDict_Items}{PyObject *p}
Returns a \ctype{PyListObject} containing all the items 
from the dictionary, as in the dictinoary method \method{items()} (see
the \citetitle[../lib/lib.html]{Python Library Reference}).
\end{cfuncdesc}

\begin{cfuncdesc}{PyObject*}{PyDict_Keys}{PyObject *p}
Returns a \ctype{PyListObject} containing all the keys 
from the dictionary, as in the dictionary method \method{keys()} (see the
\citetitle[../lib/lib.html]{Python Library Reference}).
\end{cfuncdesc}

\begin{cfuncdesc}{PyObject*}{PyDict_Values}{PyObject *p}
Returns a \ctype{PyListObject} containing all the values 
from the dictionary \var{p}, as in the dictionary method
\method{values()} (see the \citetitle[../lib/lib.html]{Python Library
Reference}).
\end{cfuncdesc}

\begin{cfuncdesc}{int}{PyDict_Size}{PyObject *p}
Returns the number of items in the dictionary.  This is equivalent to
\samp{len(\var{p})} on a dictionary.\bifuncindex{len}
\end{cfuncdesc}

\begin{cfuncdesc}{int}{PyDict_Next}{PyObject *p, int *ppos,
                                    PyObject **pkey, PyObject **pvalue}
Iterate over all key-value pairs in the dictionary \var{p}.  The
\ctype{int} referred to by \var{ppos} must be initialized to \code{0}
prior to the first call to this function to start the iteration; the
function returns true for each pair in the dictionary, and false once
all pairs have been reported.  The parameters \var{pkey} and
\var{pvalue} should either point to \ctype{PyObject*} variables that
will be filled in with each key and value, respectively, or may be
\NULL.

For example:

\begin{verbatim}
PyObject *key, *value;
int pos = 0;

while (PyDict_Next(self->dict, &pos, &key, &value)) {
    /* do something interesting with the values... */
    ...
}
\end{verbatim}

The dictionary \var{p} should not be mutated during iteration.  It is
safe (since Python 2.1) to modify the values of the keys as you
iterate over the dictionary, for example:

\begin{verbatim}
PyObject *key, *value;
int pos = 0;

while (PyDict_Next(self->dict, &pos, &key, &value)) {
    int i = PyInt_AS_LONG(value) + 1;
    PyObject *o = PyInt_FromLong(i);
    if (o == NULL)
        return -1;
    if (PyDict_SetItem(self->dict, key, o) < 0) {
        Py_DECREF(o);
        return -1;
    }
    Py_DECREF(o);
}
\end{verbatim}
\end{cfuncdesc}


\section{Numeric Objects \label{numericObjects}}

\obindex{numeric}


\subsection{Plain Integer Objects \label{intObjects}}

\obindex{integer}
\begin{ctypedesc}{PyIntObject}
This subtype of \ctype{PyObject} represents a Python integer object.
\end{ctypedesc}

\begin{cvardesc}{PyTypeObject}{PyInt_Type}
This instance of \ctype{PyTypeObject} represents the Python plain 
integer type.  This is the same object as \code{types.IntType}.
\withsubitem{(in modules types)}{\ttindex{IntType}}
\end{cvardesc}

\begin{cfuncdesc}{int}{PyInt_Check}{PyObject* o}
Returns true if \var{o} is of type \cdata{PyInt_Type}.
\end{cfuncdesc}

\begin{cfuncdesc}{PyObject*}{PyInt_FromLong}{long ival}
Creates a new integer object with a value of \var{ival}.

The current implementation keeps an array of integer objects for all
integers between \code{-1} and \code{100}, when you create an int in
that range you actually just get back a reference to the existing
object. So it should be possible to change the value of \code{1}. I
suspect the behaviour of Python in this case is undefined. :-)
\end{cfuncdesc}

\begin{cfuncdesc}{long}{PyInt_AsLong}{PyObject *io}
Will first attempt to cast the object to a \ctype{PyIntObject}, if
it is not already one, and then return its value.
\end{cfuncdesc}

\begin{cfuncdesc}{long}{PyInt_AS_LONG}{PyObject *io}
Returns the value of the object \var{io}.  No error checking is
performed.
\end{cfuncdesc}

\begin{cfuncdesc}{long}{PyInt_GetMax}{}
Returns the system's idea of the largest integer it can handle
(\constant{LONG_MAX}\ttindex{LONG_MAX}, as defined in the system
header files).
\end{cfuncdesc}


\subsection{Long Integer Objects \label{longObjects}}

\obindex{long integer}
\begin{ctypedesc}{PyLongObject}
This subtype of \ctype{PyObject} represents a Python long integer
object.
\end{ctypedesc}

\begin{cvardesc}{PyTypeObject}{PyLong_Type}
This instance of \ctype{PyTypeObject} represents the Python long
integer type.  This is the same object as \code{types.LongType}.
\withsubitem{(in modules types)}{\ttindex{LongType}}
\end{cvardesc}

\begin{cfuncdesc}{int}{PyLong_Check}{PyObject *p}
Returns true if its argument is a \ctype{PyLongObject}.
\end{cfuncdesc}

\begin{cfuncdesc}{PyObject*}{PyLong_FromLong}{long v}
Returns a new \ctype{PyLongObject} object from \var{v}, or \NULL{} on
failure.
\end{cfuncdesc}

\begin{cfuncdesc}{PyObject*}{PyLong_FromUnsignedLong}{unsigned long v}
Returns a new \ctype{PyLongObject} object from a C \ctype{unsigned
long}, or \NULL{} on failure.
\end{cfuncdesc}

\begin{cfuncdesc}{PyObject*}{PyLong_FromDouble}{double v}
Returns a new \ctype{PyLongObject} object from the integer part of
\var{v}, or \NULL{} on failure.
\end{cfuncdesc}

\begin{cfuncdesc}{long}{PyLong_AsLong}{PyObject *pylong}
Returns a C \ctype{long} representation of the contents of
\var{pylong}.  If \var{pylong} is greater than
\constant{LONG_MAX}\ttindex{LONG_MAX}, an \exception{OverflowError} is
raised.\withsubitem{(built-in exception)}{OverflowError}
\end{cfuncdesc}

\begin{cfuncdesc}{unsigned long}{PyLong_AsUnsignedLong}{PyObject *pylong}
Returns a C \ctype{unsigned long} representation of the contents of 
\var{pylong}.  If \var{pylong} is greater than
\constant{ULONG_MAX}\ttindex{ULONG_MAX}, an \exception{OverflowError}
is raised.\withsubitem{(built-in exception)}{OverflowError}
\end{cfuncdesc}

\begin{cfuncdesc}{double}{PyLong_AsDouble}{PyObject *pylong}
Returns a C \ctype{double} representation of the contents of \var{pylong}.
\end{cfuncdesc}

\begin{cfuncdesc}{PyObject*}{PyLong_FromString}{char *str, char **pend,
                                                int base}
Return a new \ctype{PyLongObject} based on the string value in
\var{str}, which is interpreted according to the radix in \var{base}.
If \var{pend} is non-\NULL, \code{*\var{pend}} will point to the first 
character in \var{str} which follows the representation of the
number.  If \var{base} is \code{0}, the radix will be determined base
on the leading characters of \var{str}: if \var{str} starts with
\code{'0x'} or \code{'0X'}, radix 16 will be used; if \var{str} starts 
with \code{'0'}, radix 8 will be used; otherwise radix 10 will be
used.  If \var{base} is not \code{0}, it must be between \code{2} and
\code{36}, inclusive.  Leading spaces are ignored.  If there are no
digits, \exception{ValueError} will be raised.
\end{cfuncdesc}


\subsection{Floating Point Objects \label{floatObjects}}

\obindex{floating point}
\begin{ctypedesc}{PyFloatObject}
This subtype of \ctype{PyObject} represents a Python floating point
object.
\end{ctypedesc}

\begin{cvardesc}{PyTypeObject}{PyFloat_Type}
This instance of \ctype{PyTypeObject} represents the Python floating
point type.  This is the same object as \code{types.FloatType}.
\withsubitem{(in modules types)}{\ttindex{FloatType}}
\end{cvardesc}

\begin{cfuncdesc}{int}{PyFloat_Check}{PyObject *p}
Returns true if its argument is a \ctype{PyFloatObject}.
\end{cfuncdesc}

\begin{cfuncdesc}{PyObject*}{PyFloat_FromDouble}{double v}
Creates a \ctype{PyFloatObject} object from \var{v}, or \NULL{} on
failure.
\end{cfuncdesc}

\begin{cfuncdesc}{double}{PyFloat_AsDouble}{PyObject *pyfloat}
Returns a C \ctype{double} representation of the contents of \var{pyfloat}.
\end{cfuncdesc}

\begin{cfuncdesc}{double}{PyFloat_AS_DOUBLE}{PyObject *pyfloat}
Returns a C \ctype{double} representation of the contents of
\var{pyfloat}, but without error checking.
\end{cfuncdesc}


\subsection{Complex Number Objects \label{complexObjects}}

\obindex{complex number}
Python's complex number objects are implemented as two distinct types
when viewed from the C API:  one is the Python object exposed to
Python programs, and the other is a C structure which represents the
actual complex number value.  The API provides functions for working
with both.

\subsubsection{Complex Numbers as C Structures}

Note that the functions which accept these structures as parameters
and return them as results do so \emph{by value} rather than
dereferencing them through pointers.  This is consistent throughout
the API.

\begin{ctypedesc}{Py_complex}
The C structure which corresponds to the value portion of a Python
complex number object.  Most of the functions for dealing with complex
number objects use structures of this type as input or output values,
as appropriate.  It is defined as:

\begin{verbatim}
typedef struct {
   double real;
   double imag;
} Py_complex;
\end{verbatim}
\end{ctypedesc}

\begin{cfuncdesc}{Py_complex}{_Py_c_sum}{Py_complex left, Py_complex right}
Return the sum of two complex numbers, using the C
\ctype{Py_complex} representation.
\end{cfuncdesc}

\begin{cfuncdesc}{Py_complex}{_Py_c_diff}{Py_complex left, Py_complex right}
Return the difference between two complex numbers, using the C
\ctype{Py_complex} representation.
\end{cfuncdesc}

\begin{cfuncdesc}{Py_complex}{_Py_c_neg}{Py_complex complex}
Return the negation of the complex number \var{complex}, using the C
\ctype{Py_complex} representation.
\end{cfuncdesc}

\begin{cfuncdesc}{Py_complex}{_Py_c_prod}{Py_complex left, Py_complex right}
Return the product of two complex numbers, using the C
\ctype{Py_complex} representation.
\end{cfuncdesc}

\begin{cfuncdesc}{Py_complex}{_Py_c_quot}{Py_complex dividend,
                                          Py_complex divisor}
Return the quotient of two complex numbers, using the C
\ctype{Py_complex} representation.
\end{cfuncdesc}

\begin{cfuncdesc}{Py_complex}{_Py_c_pow}{Py_complex num, Py_complex exp}
Return the exponentiation of \var{num} by \var{exp}, using the C
\ctype{Py_complex} representation.
\end{cfuncdesc}


\subsubsection{Complex Numbers as Python Objects}

\begin{ctypedesc}{PyComplexObject}
This subtype of \ctype{PyObject} represents a Python complex number object.
\end{ctypedesc}

\begin{cvardesc}{PyTypeObject}{PyComplex_Type}
This instance of \ctype{PyTypeObject} represents the Python complex 
number type.
\end{cvardesc}

\begin{cfuncdesc}{int}{PyComplex_Check}{PyObject *p}
Returns true if its argument is a \ctype{PyComplexObject}.
\end{cfuncdesc}

\begin{cfuncdesc}{PyObject*}{PyComplex_FromCComplex}{Py_complex v}
Create a new Python complex number object from a C
\ctype{Py_complex} value.
\end{cfuncdesc}

\begin{cfuncdesc}{PyObject*}{PyComplex_FromDoubles}{double real, double imag}
Returns a new \ctype{PyComplexObject} object from \var{real} and \var{imag}.
\end{cfuncdesc}

\begin{cfuncdesc}{double}{PyComplex_RealAsDouble}{PyObject *op}
Returns the real part of \var{op} as a C \ctype{double}.
\end{cfuncdesc}

\begin{cfuncdesc}{double}{PyComplex_ImagAsDouble}{PyObject *op}
Returns the imaginary part of \var{op} as a C \ctype{double}.
\end{cfuncdesc}

\begin{cfuncdesc}{Py_complex}{PyComplex_AsCComplex}{PyObject *op}
Returns the \ctype{Py_complex} value of the complex number \var{op}.
\end{cfuncdesc}



\section{Other Objects \label{otherObjects}}

\subsection{File Objects \label{fileObjects}}

\obindex{file}
Python's built-in file objects are implemented entirely on the
\ctype{FILE*} support from the C standard library.  This is an
implementation detail and may change in future releases of Python.

\begin{ctypedesc}{PyFileObject}
This subtype of \ctype{PyObject} represents a Python file object.
\end{ctypedesc}

\begin{cvardesc}{PyTypeObject}{PyFile_Type}
This instance of \ctype{PyTypeObject} represents the Python file
type.  This is exposed to Python programs as \code{types.FileType}.
\withsubitem{(in module types)}{\ttindex{FileType}}
\end{cvardesc}

\begin{cfuncdesc}{int}{PyFile_Check}{PyObject *p}
Returns true if its argument is a \ctype{PyFileObject}.
\end{cfuncdesc}

\begin{cfuncdesc}{PyObject*}{PyFile_FromString}{char *filename, char *mode}
On success, returns a new file object that is opened on the
file given by \var{filename}, with a file mode given by \var{mode},
where \var{mode} has the same semantics as the standard C routine
\cfunction{fopen()}\ttindex{fopen()}.  On failure, returns \NULL.
\end{cfuncdesc}

\begin{cfuncdesc}{PyObject*}{PyFile_FromFile}{FILE *fp,
                                              char *name, char *mode,
                                              int (*close)(FILE*)}
Creates a new \ctype{PyFileObject} from the already-open standard C
file pointer, \var{fp}.  The function \var{close} will be called when
the file should be closed.  Returns \NULL{} on failure.
\end{cfuncdesc}

\begin{cfuncdesc}{FILE*}{PyFile_AsFile}{PyFileObject *p}
Returns the file object associated with \var{p} as a \ctype{FILE*}.
\end{cfuncdesc}

\begin{cfuncdesc}{PyObject*}{PyFile_GetLine}{PyObject *p, int n}
Equivalent to \code{\var{p}.readline(\optional{\var{n}})}, this
function reads one line from the object \var{p}.  \var{p} may be a
file object or any object with a \method{readline()} method.  If
\var{n} is \code{0}, exactly one line is read, regardless of the
length of the line.  If \var{n} is greater than \code{0}, no more than 
\var{n} bytes will be read from the file; a partial line can be
returned.  In both cases, an empty string is returned if the end of
the file is reached immediately.  If \var{n} is less than \code{0},
however, one line is read regardless of length, but
\exception{EOFError} is raised if the end of the file is reached
immediately.
\withsubitem{(built-in exception)}{\ttindex{EOFError}}
\end{cfuncdesc}

\begin{cfuncdesc}{PyObject*}{PyFile_Name}{PyObject *p}
Returns the name of the file specified by \var{p} as a string object.
\end{cfuncdesc}

\begin{cfuncdesc}{void}{PyFile_SetBufSize}{PyFileObject *p, int n}
Available on systems with \cfunction{setvbuf()}\ttindex{setvbuf()}
only.  This should only be called immediately after file object
creation.
\end{cfuncdesc}

\begin{cfuncdesc}{int}{PyFile_SoftSpace}{PyObject *p, int newflag}
This function exists for internal use by the interpreter.
Sets the \member{softspace} attribute of \var{p} to \var{newflag} and
\withsubitem{(file attribute)}{\ttindex{softspace}}returns the
previous value.  \var{p} does not have to be a file object
for this function to work properly; any object is supported (thought
its only interesting if the \member{softspace} attribute can be set).
This function clears any errors, and will return \code{0} as the
previous value if the attribute either does not exist or if there were
errors in retrieving it.  There is no way to detect errors from this
function, but doing so should not be needed.
\end{cfuncdesc}

\begin{cfuncdesc}{int}{PyFile_WriteObject}{PyObject *obj, PyFileObject *p,
                                           int flags}
Writes object \var{obj} to file object \var{p}.  The only supported
flag for \var{flags} is \constant{Py_PRINT_RAW}\ttindex{Py_PRINT_RAW};
if given, the \function{str()} of the object is written instead of the 
\function{repr()}.  Returns \code{0} on success or \code{-1} on
failure; the appropriate exception will be set.
\end{cfuncdesc}

\begin{cfuncdesc}{int}{PyFile_WriteString}{char *s, PyFileObject *p,
                                           int flags}
Writes string \var{s} to file object \var{p}.  Returns \code{0} on
success or \code{-1} on failure; the appropriate exception will be
set.
\end{cfuncdesc}


\subsection{Instance Objects \label{instanceObjects}}

\obindex{instance}
There are very few functions specific to instance objects.

\begin{cvardesc}{PyTypeObject}{PyInstance_Type}
  Type object for class instances.
\end{cvardesc}

\begin{cfuncdesc}{int}{PyInstance_Check}{PyObject *obj}
  Returns true if \var{obj} is an instance.
\end{cfuncdesc}

\begin{cfuncdesc}{PyObject*}{PyInstance_New}{PyObject *class,
                                             PyObject *arg,
                                             PyObject *kw}
  Create a new instance of a specific class.  The parameters \var{arg}
  and \var{kw} are used as the positional and keyword parameters to
  the object's constructor.
\end{cfuncdesc}

\begin{cfuncdesc}{PyObject*}{PyInstance_NewRaw}{PyObject *class,
                                                PyObject *dict}
  Create a new instance of a specific class without calling it's
  constructor.  \var{class} is the class of new object.  The
  \var{dict} parameter will be used as the object's \member{__dict__};
  if \NULL, a new dictionary will be created for the instance.
\end{cfuncdesc}


\subsection{Module Objects \label{moduleObjects}}

\obindex{module}
There are only a few functions special to module objects.

\begin{cvardesc}{PyTypeObject}{PyModule_Type}
This instance of \ctype{PyTypeObject} represents the Python module
type.  This is exposed to Python programs as \code{types.ModuleType}.
\withsubitem{(in module types)}{\ttindex{ModuleType}}
\end{cvardesc}

\begin{cfuncdesc}{int}{PyModule_Check}{PyObject *p}
Returns true if its argument is a module object.
\end{cfuncdesc}

\begin{cfuncdesc}{PyObject*}{PyModule_New}{char *name}
Return a new module object with the \member{__name__} attribute set to
\var{name}.  Only the module's \member{__doc__} and
\member{__name__} attributes are filled in; the caller is responsible
for providing a \member{__file__} attribute.
\withsubitem{(module attribute)}{
  \ttindex{__name__}\ttindex{__doc__}\ttindex{__file__}}
\end{cfuncdesc}

\begin{cfuncdesc}{PyObject*}{PyModule_GetDict}{PyObject *module}
Return the dictionary object that implements \var{module}'s namespace; 
this object is the same as the \member{__dict__} attribute of the
module object.  This function never fails.
\withsubitem{(module attribute)}{\ttindex{__dict__}}
\end{cfuncdesc}

\begin{cfuncdesc}{char*}{PyModule_GetName}{PyObject *module}
Return \var{module}'s \member{__name__} value.  If the module does not 
provide one, or if it is not a string, \exception{SystemError} is
raised and \NULL{} is returned.
\withsubitem{(module attribute)}{\ttindex{__name__}}
\withsubitem{(built-in exception)}{\ttindex{SystemError}}
\end{cfuncdesc}

\begin{cfuncdesc}{char*}{PyModule_GetFilename}{PyObject *module}
Return the name of the file from which \var{module} was loaded using
\var{module}'s \member{__file__} attribute.  If this is not defined,
or if it is not a string, raise \exception{SystemError} and return
\NULL.
\withsubitem{(module attribute)}{\ttindex{__file__}}
\withsubitem{(built-in exception)}{\ttindex{SystemError}}
\end{cfuncdesc}

\begin{cfuncdesc}{int}{PyModule_AddObject}{PyObject *module,
                                           char *name, PyObject *value}
Add an object to \var{module} as \var{name}.  This is a convenience
function which can be used from the module's initialization function.
This steals a reference to \var{value}.  Returns \code{-1} on error,
\code{0} on success.
\versionadded{2.0}
\end{cfuncdesc}

\begin{cfuncdesc}{int}{PyModule_AddIntConstant}{PyObject *module,
                                                char *name, int value}
Add an integer constant to \var{module} as \var{name}.  This convenience
function can be used from the module's initialization function.
Returns \code{-1} on error, \code{0} on success.
\versionadded{2.0}
\end{cfuncdesc}

\begin{cfuncdesc}{int}{PyModule_AddStringConstant}{PyObject *module,
                                                   char *name, char *value}
Add a string constant to \var{module} as \var{name}.  This convenience
function can be used from the module's initialization function.  The
string \var{value} must be null-terminated.  Returns \code{-1} on
error, \code{0} on success.
\versionadded{2.0}
\end{cfuncdesc}


\subsection{CObjects \label{cObjects}}

\obindex{CObject}
Refer to \emph{Extending and Embedding the Python Interpreter},
section 1.12 (``Providing a C API for an Extension Module''), for more 
information on using these objects.


\begin{ctypedesc}{PyCObject}
This subtype of \ctype{PyObject} represents an opaque value, useful for
C extension modules who need to pass an opaque value (as a
\ctype{void*} pointer) through Python code to other C code.  It is
often used to make a C function pointer defined in one module
available to other modules, so the regular import mechanism can be
used to access C APIs defined in dynamically loaded modules.
\end{ctypedesc}

\begin{cfuncdesc}{int}{PyCObject_Check}{PyObject *p}
Returns true if its argument is a \ctype{PyCObject}.
\end{cfuncdesc}

\begin{cfuncdesc}{PyObject*}{PyCObject_FromVoidPtr}{void* cobj, 
	void (*destr)(void *)}
Creates a \ctype{PyCObject} from the \code{void *}\var{cobj}.  The
\var{destr} function will be called when the object is reclaimed, unless
it is \NULL.
\end{cfuncdesc}

\begin{cfuncdesc}{PyObject*}{PyCObject_FromVoidPtrAndDesc}{void* cobj,
	void* desc, void (*destr)(void *, void *) }
Creates a \ctype{PyCObject} from the \ctype{void *}\var{cobj}.  The
\var{destr} function will be called when the object is reclaimed.  The
\var{desc} argument can be used to pass extra callback data for the
destructor function.
\end{cfuncdesc}

\begin{cfuncdesc}{void*}{PyCObject_AsVoidPtr}{PyObject* self}
Returns the object \ctype{void *} that the
\ctype{PyCObject} \var{self} was created with.
\end{cfuncdesc}

\begin{cfuncdesc}{void*}{PyCObject_GetDesc}{PyObject* self}
Returns the description \ctype{void *} that the
\ctype{PyCObject} \var{self} was created with.
\end{cfuncdesc}


\chapter{Initialization, Finalization, and Threads
         \label{initialization}}

\begin{cfuncdesc}{void}{Py_Initialize}{}
Initialize the Python interpreter.  In an application embedding 
Python, this should be called before using any other Python/C API 
functions; with the exception of
\cfunction{Py_SetProgramName()}\ttindex{Py_SetProgramName()},
\cfunction{PyEval_InitThreads()}\ttindex{PyEval_InitThreads()},
\cfunction{PyEval_ReleaseLock()}\ttindex{PyEval_ReleaseLock()},
and \cfunction{PyEval_AcquireLock()}\ttindex{PyEval_AcquireLock()}.
This initializes the table of loaded modules (\code{sys.modules}), and
\withsubitem{(in module sys)}{\ttindex{modules}\ttindex{path}}creates the
fundamental modules \module{__builtin__}\refbimodindex{__builtin__},
\module{__main__}\refbimodindex{__main__} and
\module{sys}\refbimodindex{sys}.  It also initializes the module
search\indexiii{module}{search}{path} path (\code{sys.path}).
It does not set \code{sys.argv}; use
\cfunction{PySys_SetArgv()}\ttindex{PySys_SetArgv()} for that.  This
is a no-op when called for a second time (without calling
\cfunction{Py_Finalize()}\ttindex{Py_Finalize()} first).  There is no
return value; it is a fatal error if the initialization fails.
\end{cfuncdesc}

\begin{cfuncdesc}{int}{Py_IsInitialized}{}
Return true (nonzero) when the Python interpreter has been
initialized, false (zero) if not.  After \cfunction{Py_Finalize()} is
called, this returns false until \cfunction{Py_Initialize()} is called
again.
\end{cfuncdesc}

\begin{cfuncdesc}{void}{Py_Finalize}{}
Undo all initializations made by \cfunction{Py_Initialize()} and
subsequent use of Python/C API functions, and destroy all
sub-interpreters (see \cfunction{Py_NewInterpreter()} below) that were
created and not yet destroyed since the last call to
\cfunction{Py_Initialize()}.  Ideally, this frees all memory allocated
by the Python interpreter.  This is a no-op when called for a second
time (without calling \cfunction{Py_Initialize()} again first).  There
is no return value; errors during finalization are ignored.

This function is provided for a number of reasons.  An embedding 
application might want to restart Python without having to restart the 
application itself.  An application that has loaded the Python 
interpreter from a dynamically loadable library (or DLL) might want to 
free all memory allocated by Python before unloading the DLL. During a 
hunt for memory leaks in an application a developer might want to free 
all memory allocated by Python before exiting from the application.

\strong{Bugs and caveats:} The destruction of modules and objects in 
modules is done in random order; this may cause destructors 
(\method{__del__()} methods) to fail when they depend on other objects 
(even functions) or modules.  Dynamically loaded extension modules 
loaded by Python are not unloaded.  Small amounts of memory allocated 
by the Python interpreter may not be freed (if you find a leak, please 
report it).  Memory tied up in circular references between objects is 
not freed.  Some memory allocated by extension modules may not be 
freed.  Some extension may not work properly if their initialization 
routine is called more than once; this can happen if an applcation 
calls \cfunction{Py_Initialize()} and \cfunction{Py_Finalize()} more
than once.
\end{cfuncdesc}

\begin{cfuncdesc}{PyThreadState*}{Py_NewInterpreter}{}
Create a new sub-interpreter.  This is an (almost) totally separate
environment for the execution of Python code.  In particular, the new
interpreter has separate, independent versions of all imported
modules, including the fundamental modules
\module{__builtin__}\refbimodindex{__builtin__},
\module{__main__}\refbimodindex{__main__} and
\module{sys}\refbimodindex{sys}.  The table of loaded modules
(\code{sys.modules}) and the module search path (\code{sys.path}) are
also separate.  The new environment has no \code{sys.argv} variable.
It has new standard I/O stream file objects \code{sys.stdin},
\code{sys.stdout} and \code{sys.stderr} (however these refer to the
same underlying \ctype{FILE} structures in the C library).
\withsubitem{(in module sys)}{
  \ttindex{stdout}\ttindex{stderr}\ttindex{stdin}}

The return value points to the first thread state created in the new 
sub-interpreter.  This thread state is made the current thread state.  
Note that no actual thread is created; see the discussion of thread 
states below.  If creation of the new interpreter is unsuccessful, 
\NULL{} is returned; no exception is set since the exception state 
is stored in the current thread state and there may not be a current 
thread state.  (Like all other Python/C API functions, the global 
interpreter lock must be held before calling this function and is 
still held when it returns; however, unlike most other Python/C API 
functions, there needn't be a current thread state on entry.)

Extension modules are shared between (sub-)interpreters as follows: 
the first time a particular extension is imported, it is initialized 
normally, and a (shallow) copy of its module's dictionary is 
squirreled away.  When the same extension is imported by another 
(sub-)interpreter, a new module is initialized and filled with the 
contents of this copy; the extension's \code{init} function is not
called.  Note that this is different from what happens when an
extension is imported after the interpreter has been completely
re-initialized by calling
\cfunction{Py_Finalize()}\ttindex{Py_Finalize()} and
\cfunction{Py_Initialize()}\ttindex{Py_Initialize()}; in that case,
the extension's \code{init\var{module}} function \emph{is} called
again.

\strong{Bugs and caveats:} Because sub-interpreters (and the main 
interpreter) are part of the same process, the insulation between them 
isn't perfect --- for example, using low-level file operations like 
\withsubitem{(in module os)}{\ttindex{close()}}
\function{os.close()} they can (accidentally or maliciously) affect each 
other's open files.  Because of the way extensions are shared between 
(sub-)interpreters, some extensions may not work properly; this is 
especially likely when the extension makes use of (static) global 
variables, or when the extension manipulates its module's dictionary 
after its initialization.  It is possible to insert objects created in 
one sub-interpreter into a namespace of another sub-interpreter; this 
should be done with great care to avoid sharing user-defined 
functions, methods, instances or classes between sub-interpreters, 
since import operations executed by such objects may affect the 
wrong (sub-)interpreter's dictionary of loaded modules.  (XXX This is 
a hard-to-fix bug that will be addressed in a future release.)
\end{cfuncdesc}

\begin{cfuncdesc}{void}{Py_EndInterpreter}{PyThreadState *tstate}
Destroy the (sub-)interpreter represented by the given thread state.  
The given thread state must be the current thread state.  See the 
discussion of thread states below.  When the call returns, the current 
thread state is \NULL{}.  All thread states associated with this 
interpreted are destroyed.  (The global interpreter lock must be held 
before calling this function and is still held when it returns.)  
\cfunction{Py_Finalize()}\ttindex{Py_Finalize()} will destroy all
sub-interpreters that haven't been explicitly destroyed at that point.
\end{cfuncdesc}

\begin{cfuncdesc}{void}{Py_SetProgramName}{char *name}
This function should be called before
\cfunction{Py_Initialize()}\ttindex{Py_Initialize()} is called
for the first time, if it is called at all.  It tells the interpreter 
the value of the \code{argv[0]} argument to the
\cfunction{main()}\ttindex{main()} function of the program.  This is
used by \cfunction{Py_GetPath()}\ttindex{Py_GetPath()} and some other  
functions below to find the Python run-time libraries relative to the 
interpreter executable.  The default value is \code{'python'}.  The 
argument should point to a zero-terminated character string in static 
storage whose contents will not change for the duration of the 
program's execution.  No code in the Python interpreter will change 
the contents of this storage.
\end{cfuncdesc}

\begin{cfuncdesc}{char*}{Py_GetProgramName}{}
Return the program name set with
\cfunction{Py_SetProgramName()}\ttindex{Py_SetProgramName()}, or the
default.  The returned string points into static storage; the caller 
should not modify its value.
\end{cfuncdesc}

\begin{cfuncdesc}{char*}{Py_GetPrefix}{}
Return the \emph{prefix} for installed platform-independent files.  This 
is derived through a number of complicated rules from the program name 
set with \cfunction{Py_SetProgramName()} and some environment variables; 
for example, if the program name is \code{'/usr/local/bin/python'}, 
the prefix is \code{'/usr/local'}.  The returned string points into 
static storage; the caller should not modify its value.  This 
corresponds to the \makevar{prefix} variable in the top-level 
\file{Makefile} and the \longprogramopt{prefix} argument to the 
\program{configure} script at build time.  The value is available to 
Python code as \code{sys.prefix}.  It is only useful on \UNIX{}.  See 
also the next function.
\end{cfuncdesc}

\begin{cfuncdesc}{char*}{Py_GetExecPrefix}{}
Return the \emph{exec-prefix} for installed platform-\emph{de}pendent 
files.  This is derived through a number of complicated rules from the 
program name set with \cfunction{Py_SetProgramName()} and some environment 
variables; for example, if the program name is 
\code{'/usr/local/bin/python'}, the exec-prefix is 
\code{'/usr/local'}.  The returned string points into static storage; 
the caller should not modify its value.  This corresponds to the 
\makevar{exec_prefix} variable in the top-level \file{Makefile} and the 
\longprogramopt{exec-prefix} argument to the
\program{configure} script at build  time.  The value is available to
Python code as \code{sys.exec_prefix}.  It is only useful on \UNIX{}.

Background: The exec-prefix differs from the prefix when platform 
dependent files (such as executables and shared libraries) are 
installed in a different directory tree.  In a typical installation, 
platform dependent files may be installed in the 
\file{/usr/local/plat} subtree while platform independent may be 
installed in \file{/usr/local}.

Generally speaking, a platform is a combination of hardware and 
software families, e.g.  Sparc machines running the Solaris 2.x 
operating system are considered the same platform, but Intel machines 
running Solaris 2.x are another platform, and Intel machines running 
Linux are yet another platform.  Different major revisions of the same 
operating system generally also form different platforms.  Non-\UNIX{} 
operating systems are a different story; the installation strategies 
on those systems are so different that the prefix and exec-prefix are 
meaningless, and set to the empty string.  Note that compiled Python 
bytecode files are platform independent (but not independent from the 
Python version by which they were compiled!).

System administrators will know how to configure the \program{mount} or 
\program{automount} programs to share \file{/usr/local} between platforms 
while having \file{/usr/local/plat} be a different filesystem for each 
platform.
\end{cfuncdesc}

\begin{cfuncdesc}{char*}{Py_GetProgramFullPath}{}
Return the full program name of the Python executable; this is 
computed as a side-effect of deriving the default module search path 
from the program name (set by
\cfunction{Py_SetProgramName()}\ttindex{Py_SetProgramName()} above).
The returned string points into static storage; the caller should not 
modify its value.  The value is available to Python code as 
\code{sys.executable}.
\withsubitem{(in module sys)}{\ttindex{executable}}
\end{cfuncdesc}

\begin{cfuncdesc}{char*}{Py_GetPath}{}
\indexiii{module}{search}{path}
Return the default module search path; this is computed from the 
program name (set by \cfunction{Py_SetProgramName()} above) and some 
environment variables.  The returned string consists of a series of 
directory names separated by a platform dependent delimiter character.  
The delimiter character is \character{:} on \UNIX{}, \character{;} on
DOS/Windows, and \character{\e n} (the \ASCII{} newline character) on
Macintosh.  The returned string points into static storage; the caller
should not modify its value.  The value is available to Python code 
as the list \code{sys.path}\withsubitem{(in module sys)}{\ttindex{path}},
which may be modified to change the future search path for loaded
modules.

% XXX should give the exact rules
\end{cfuncdesc}

\begin{cfuncdesc}{const char*}{Py_GetVersion}{}
Return the version of this Python interpreter.  This is a string that 
looks something like

\begin{verbatim}
"1.5 (#67, Dec 31 1997, 22:34:28) [GCC 2.7.2.2]"
\end{verbatim}

The first word (up to the first space character) is the current Python 
version; the first three characters are the major and minor version 
separated by a period.  The returned string points into static storage; 
the caller should not modify its value.  The value is available to 
Python code as the list \code{sys.version}.
\withsubitem{(in module sys)}{\ttindex{version}}
\end{cfuncdesc}

\begin{cfuncdesc}{const char*}{Py_GetPlatform}{}
Return the platform identifier for the current platform.  On \UNIX{}, 
this is formed from the ``official'' name of the operating system, 
converted to lower case, followed by the major revision number; e.g., 
for Solaris 2.x, which is also known as SunOS 5.x, the value is 
\code{'sunos5'}.  On Macintosh, it is \code{'mac'}.  On Windows, it 
is \code{'win'}.  The returned string points into static storage; 
the caller should not modify its value.  The value is available to 
Python code as \code{sys.platform}.
\withsubitem{(in module sys)}{\ttindex{platform}}
\end{cfuncdesc}

\begin{cfuncdesc}{const char*}{Py_GetCopyright}{}
Return the official copyright string for the current Python version, 
for example

\code{'Copyright 1991-1995 Stichting Mathematisch Centrum, Amsterdam'}

The returned string points into static storage; the caller should not 
modify its value.  The value is available to Python code as the list 
\code{sys.copyright}.
\withsubitem{(in module sys)}{\ttindex{copyright}}
\end{cfuncdesc}

\begin{cfuncdesc}{const char*}{Py_GetCompiler}{}
Return an indication of the compiler used to build the current Python 
version, in square brackets, for example:

\begin{verbatim}
"[GCC 2.7.2.2]"
\end{verbatim}

The returned string points into static storage; the caller should not 
modify its value.  The value is available to Python code as part of 
the variable \code{sys.version}.
\withsubitem{(in module sys)}{\ttindex{version}}
\end{cfuncdesc}

\begin{cfuncdesc}{const char*}{Py_GetBuildInfo}{}
Return information about the sequence number and build date and time 
of the current Python interpreter instance, for example

\begin{verbatim}
"#67, Aug  1 1997, 22:34:28"
\end{verbatim}

The returned string points into static storage; the caller should not 
modify its value.  The value is available to Python code as part of 
the variable \code{sys.version}.
\withsubitem{(in module sys)}{\ttindex{version}}
\end{cfuncdesc}

\begin{cfuncdesc}{int}{PySys_SetArgv}{int argc, char **argv}
Set \code{sys.argv} based on \var{argc} and \var{argv}.  These
parameters are similar to those passed to the program's
\cfunction{main()}\ttindex{main()} function with the difference that
the first entry should refer to the script file to be executed rather
than the executable hosting the Python interpreter.  If there isn't a
script that will be run, the first entry in \var{argv} can be an empty
string.  If this function fails to initialize \code{sys.argv}, a fatal 
condition is signalled using
\cfunction{Py_FatalError()}\ttindex{Py_FatalError()}.
\withsubitem{(in module sys)}{\ttindex{argv}}
% XXX impl. doesn't seem consistent in allowing 0/NULL for the params; 
% check w/ Guido.
\end{cfuncdesc}

% XXX Other PySys thingies (doesn't really belong in this chapter)

\section{Thread State and the Global Interpreter Lock
         \label{threads}}

\index{global interpreter lock}
\index{interpreter lock}
\index{lock, interpreter}

The Python interpreter is not fully thread safe.  In order to support
multi-threaded Python programs, there's a global lock that must be
held by the current thread before it can safely access Python objects.
Without the lock, even the simplest operations could cause problems in
a multi-threaded program: for example, when two threads simultaneously
increment the reference count of the same object, the reference count
could end up being incremented only once instead of twice.

Therefore, the rule exists that only the thread that has acquired the
global interpreter lock may operate on Python objects or call Python/C
API functions.  In order to support multi-threaded Python programs,
the interpreter regularly releases and reacquires the lock --- by
default, every ten bytecode instructions (this can be changed with
\withsubitem{(in module sys)}{\ttindex{setcheckinterval()}}
\function{sys.setcheckinterval()}).  The lock is also released and
reacquired around potentially blocking I/O operations like reading or
writing a file, so that other threads can run while the thread that
requests the I/O is waiting for the I/O operation to complete.

The Python interpreter needs to keep some bookkeeping information
separate per thread --- for this it uses a data structure called
\ctype{PyThreadState}\ttindex{PyThreadState}.  This is new in Python
1.5; in earlier versions, such state was stored in global variables,
and switching threads could cause problems.  In particular, exception
handling is now thread safe, when the application uses
\withsubitem{(in module sys)}{\ttindex{exc_info()}}
\function{sys.exc_info()} to access the exception last raised in the
current thread.

There's one global variable left, however: the pointer to the current
\ctype{PyThreadState}\ttindex{PyThreadState} structure.  While most
thread packages have a way to store ``per-thread global data,''
Python's internal platform independent thread abstraction doesn't
support this yet.  Therefore, the current thread state must be
manipulated explicitly.

This is easy enough in most cases.  Most code manipulating the global
interpreter lock has the following simple structure:

\begin{verbatim}
Save the thread state in a local variable.
Release the interpreter lock.
...Do some blocking I/O operation...
Reacquire the interpreter lock.
Restore the thread state from the local variable.
\end{verbatim}

This is so common that a pair of macros exists to simplify it:

\begin{verbatim}
Py_BEGIN_ALLOW_THREADS
...Do some blocking I/O operation...
Py_END_ALLOW_THREADS
\end{verbatim}

The \code{Py_BEGIN_ALLOW_THREADS}\ttindex{Py_BEGIN_ALLOW_THREADS} macro
opens a new block and declares a hidden local variable; the
\code{Py_END_ALLOW_THREADS}\ttindex{Py_END_ALLOW_THREADS} macro closes 
the block.  Another advantage of using these two macros is that when
Python is compiled without thread support, they are defined empty,
thus saving the thread state and lock manipulations.

When thread support is enabled, the block above expands to the
following code:

\begin{verbatim}
    PyThreadState *_save;

    _save = PyEval_SaveThread();
    ...Do some blocking I/O operation...
    PyEval_RestoreThread(_save);
\end{verbatim}

Using even lower level primitives, we can get roughly the same effect
as follows:

\begin{verbatim}
    PyThreadState *_save;

    _save = PyThreadState_Swap(NULL);
    PyEval_ReleaseLock();
    ...Do some blocking I/O operation...
    PyEval_AcquireLock();
    PyThreadState_Swap(_save);
\end{verbatim}

There are some subtle differences; in particular,
\cfunction{PyEval_RestoreThread()}\ttindex{PyEval_RestoreThread()} saves
and restores the value of the  global variable
\cdata{errno}\ttindex{errno}, since the lock manipulation does not
guarantee that \cdata{errno} is left alone.  Also, when thread support
is disabled,
\cfunction{PyEval_SaveThread()}\ttindex{PyEval_SaveThread()} and
\cfunction{PyEval_RestoreThread()} don't manipulate the lock; in this
case, \cfunction{PyEval_ReleaseLock()}\ttindex{PyEval_ReleaseLock()} and
\cfunction{PyEval_AcquireLock()}\ttindex{PyEval_AcquireLock()} are not
available.  This is done so that dynamically loaded extensions
compiled with thread support enabled can be loaded by an interpreter
that was compiled with disabled thread support.

The global interpreter lock is used to protect the pointer to the
current thread state.  When releasing the lock and saving the thread
state, the current thread state pointer must be retrieved before the
lock is released (since another thread could immediately acquire the
lock and store its own thread state in the global variable).
Conversely, when acquiring the lock and restoring the thread state,
the lock must be acquired before storing the thread state pointer.

Why am I going on with so much detail about this?  Because when
threads are created from C, they don't have the global interpreter
lock, nor is there a thread state data structure for them.  Such
threads must bootstrap themselves into existence, by first creating a
thread state data structure, then acquiring the lock, and finally
storing their thread state pointer, before they can start using the
Python/C API.  When they are done, they should reset the thread state
pointer, release the lock, and finally free their thread state data
structure.

When creating a thread data structure, you need to provide an
interpreter state data structure.  The interpreter state data
structure hold global data that is shared by all threads in an
interpreter, for example the module administration
(\code{sys.modules}).  Depending on your needs, you can either create
a new interpreter state data structure, or share the interpreter state
data structure used by the Python main thread (to access the latter,
you must obtain the thread state and access its \member{interp} member;
this must be done by a thread that is created by Python or by the main
thread after Python is initialized).


\begin{ctypedesc}{PyInterpreterState}
This data structure represents the state shared by a number of
cooperating threads.  Threads belonging to the same interpreter
share their module administration and a few other internal items.
There are no public members in this structure.

Threads belonging to different interpreters initially share nothing,
except process state like available memory, open file descriptors and
such.  The global interpreter lock is also shared by all threads,
regardless of to which interpreter they belong.
\end{ctypedesc}

\begin{ctypedesc}{PyThreadState}
This data structure represents the state of a single thread.  The only
public data member is \ctype{PyInterpreterState *}\member{interp},
which points to this thread's interpreter state.
\end{ctypedesc}

\begin{cfuncdesc}{void}{PyEval_InitThreads}{}
Initialize and acquire the global interpreter lock.  It should be
called in the main thread before creating a second thread or engaging
in any other thread operations such as
\cfunction{PyEval_ReleaseLock()}\ttindex{PyEval_ReleaseLock()} or
\code{PyEval_ReleaseThread(\var{tstate})}\ttindex{PyEval_ReleaseThread()}.
It is not needed before calling
\cfunction{PyEval_SaveThread()}\ttindex{PyEval_SaveThread()} or
\cfunction{PyEval_RestoreThread()}\ttindex{PyEval_RestoreThread()}.

This is a no-op when called for a second time.  It is safe to call
this function before calling
\cfunction{Py_Initialize()}\ttindex{Py_Initialize()}.

When only the main thread exists, no lock operations are needed.  This
is a common situation (most Python programs do not use threads), and
the lock operations slow the interpreter down a bit.  Therefore, the
lock is not created initially.  This situation is equivalent to having
acquired the lock: when there is only a single thread, all object
accesses are safe.  Therefore, when this function initializes the
lock, it also acquires it.  Before the Python
\module{thread}\refbimodindex{thread} module creates a new thread,
knowing that either it has the lock or the lock hasn't been created
yet, it calls \cfunction{PyEval_InitThreads()}.  When this call
returns, it is guaranteed that the lock has been created and that it
has acquired it.

It is \strong{not} safe to call this function when it is unknown which
thread (if any) currently has the global interpreter lock.

This function is not available when thread support is disabled at
compile time.
\end{cfuncdesc}

\begin{cfuncdesc}{void}{PyEval_AcquireLock}{}
Acquire the global interpreter lock.  The lock must have been created
earlier.  If this thread already has the lock, a deadlock ensues.
This function is not available when thread support is disabled at
compile time.
\end{cfuncdesc}

\begin{cfuncdesc}{void}{PyEval_ReleaseLock}{}
Release the global interpreter lock.  The lock must have been created
earlier.  This function is not available when thread support is
disabled at compile time.
\end{cfuncdesc}

\begin{cfuncdesc}{void}{PyEval_AcquireThread}{PyThreadState *tstate}
Acquire the global interpreter lock and then set the current thread
state to \var{tstate}, which should not be \NULL{}.  The lock must
have been created earlier.  If this thread already has the lock,
deadlock ensues.  This function is not available when thread support
is disabled at compile time.
\end{cfuncdesc}

\begin{cfuncdesc}{void}{PyEval_ReleaseThread}{PyThreadState *tstate}
Reset the current thread state to \NULL{} and release the global
interpreter lock.  The lock must have been created earlier and must be
held by the current thread.  The \var{tstate} argument, which must not
be \NULL{}, is only used to check that it represents the current
thread state --- if it isn't, a fatal error is reported.  This
function is not available when thread support is disabled at compile
time.
\end{cfuncdesc}

\begin{cfuncdesc}{PyThreadState*}{PyEval_SaveThread}{}
Release the interpreter lock (if it has been created and thread
support is enabled) and reset the thread state to \NULL{},
returning the previous thread state (which is not \NULL{}).  If
the lock has been created, the current thread must have acquired it.
(This function is available even when thread support is disabled at
compile time.)
\end{cfuncdesc}

\begin{cfuncdesc}{void}{PyEval_RestoreThread}{PyThreadState *tstate}
Acquire the interpreter lock (if it has been created and thread
support is enabled) and set the thread state to \var{tstate}, which
must not be \NULL{}.  If the lock has been created, the current
thread must not have acquired it, otherwise deadlock ensues.  (This
function is available even when thread support is disabled at compile
time.)
\end{cfuncdesc}

The following macros are normally used without a trailing semicolon;
look for example usage in the Python source distribution.

\begin{csimplemacrodesc}{Py_BEGIN_ALLOW_THREADS}
This macro expands to
\samp{\{ PyThreadState *_save; _save = PyEval_SaveThread();}.
Note that it contains an opening brace; it must be matched with a
following \code{Py_END_ALLOW_THREADS} macro.  See above for further
discussion of this macro.  It is a no-op when thread support is
disabled at compile time.
\end{csimplemacrodesc}

\begin{csimplemacrodesc}{Py_END_ALLOW_THREADS}
This macro expands to
\samp{PyEval_RestoreThread(_save); \}}.
Note that it contains a closing brace; it must be matched with an
earlier \code{Py_BEGIN_ALLOW_THREADS} macro.  See above for further
discussion of this macro.  It is a no-op when thread support is
disabled at compile time.
\end{csimplemacrodesc}

\begin{csimplemacrodesc}{Py_BEGIN_BLOCK_THREADS}
This macro expands to \samp{PyEval_RestoreThread(_save);} i.e. it
is equivalent to \code{Py_END_ALLOW_THREADS} without the closing
brace.  It is a no-op when thread support is disabled at compile
time.
\end{csimplemacrodesc}

\begin{csimplemacrodesc}{Py_BEGIN_UNBLOCK_THREADS}
This macro expands to \samp{_save = PyEval_SaveThread();} i.e. it is
equivalent to \code{Py_BEGIN_ALLOW_THREADS} without the opening brace
and variable declaration.  It is a no-op when thread support is
disabled at compile time.
\end{csimplemacrodesc}

All of the following functions are only available when thread support
is enabled at compile time, and must be called only when the
interpreter lock has been created.

\begin{cfuncdesc}{PyInterpreterState*}{PyInterpreterState_New}{}
Create a new interpreter state object.  The interpreter lock need not
be held, but may be held if it is necessary to serialize calls to this
function.
\end{cfuncdesc}

\begin{cfuncdesc}{void}{PyInterpreterState_Clear}{PyInterpreterState *interp}
Reset all information in an interpreter state object.  The interpreter
lock must be held.
\end{cfuncdesc}

\begin{cfuncdesc}{void}{PyInterpreterState_Delete}{PyInterpreterState *interp}
Destroy an interpreter state object.  The interpreter lock need not be
held.  The interpreter state must have been reset with a previous
call to \cfunction{PyInterpreterState_Clear()}.
\end{cfuncdesc}

\begin{cfuncdesc}{PyThreadState*}{PyThreadState_New}{PyInterpreterState *interp}
Create a new thread state object belonging to the given interpreter
object.  The interpreter lock need not be held, but may be held if it
is necessary to serialize calls to this function.
\end{cfuncdesc}

\begin{cfuncdesc}{void}{PyThreadState_Clear}{PyThreadState *tstate}
Reset all information in a thread state object.  The interpreter lock
must be held.
\end{cfuncdesc}

\begin{cfuncdesc}{void}{PyThreadState_Delete}{PyThreadState *tstate}
Destroy a thread state object.  The interpreter lock need not be
held.  The thread state must have been reset with a previous
call to \cfunction{PyThreadState_Clear()}.
\end{cfuncdesc}

\begin{cfuncdesc}{PyThreadState*}{PyThreadState_Get}{}
Return the current thread state.  The interpreter lock must be held.
When the current thread state is \NULL{}, this issues a fatal
error (so that the caller needn't check for \NULL{}).
\end{cfuncdesc}

\begin{cfuncdesc}{PyThreadState*}{PyThreadState_Swap}{PyThreadState *tstate}
Swap the current thread state with the thread state given by the
argument \var{tstate}, which may be \NULL{}.  The interpreter lock
must be held.
\end{cfuncdesc}


\chapter{Memory Management \label{memory}}
\sectionauthor{Vladimir Marangozov}{Vladimir.Marangozov@inrialpes.fr}


\section{Overview \label{memoryOverview}}

Memory management in Python involves a private heap containing all
Python objects and data structures. The management of this private
heap is ensured internally by the \emph{Python memory manager}.  The
Python memory manager has different components which deal with various
dynamic storage management aspects, like sharing, segmentation,
preallocation or caching.

At the lowest level, a raw memory allocator ensures that there is
enough room in the private heap for storing all Python-related data
by interacting with the memory manager of the operating system. On top
of the raw memory allocator, several object-specific allocators
operate on the same heap and implement distinct memory management
policies adapted to the peculiarities of every object type. For
example, integer objects are managed differently within the heap than
strings, tuples or dictionaries because integers imply different
storage requirements and speed/space tradeoffs. The Python memory
manager thus delegates some of the work to the object-specific
allocators, but ensures that the latter operate within the bounds of
the private heap.

It is important to understand that the management of the Python heap
is performed by the interpreter itself and that the user has no
control on it, even if she regularly manipulates object pointers to
memory blocks inside that heap.  The allocation of heap space for
Python objects and other internal buffers is performed on demand by
the Python memory manager through the Python/C API functions listed in
this document.

To avoid memory corruption, extension writers should never try to
operate on Python objects with the functions exported by the C
library: \cfunction{malloc()}\ttindex{malloc()},
\cfunction{calloc()}\ttindex{calloc()},
\cfunction{realloc()}\ttindex{realloc()} and
\cfunction{free()}\ttindex{free()}.  This will result in 
mixed calls between the C allocator and the Python memory manager
with fatal consequences, because they implement different algorithms
and operate on different heaps.  However, one may safely allocate and
release memory blocks with the C library allocator for individual
purposes, as shown in the following example:

\begin{verbatim}
    PyObject *res;
    char *buf = (char *) malloc(BUFSIZ); /* for I/O */

    if (buf == NULL)
        return PyErr_NoMemory();
    ...Do some I/O operation involving buf...
    res = PyString_FromString(buf);
    free(buf); /* malloc'ed */
    return res;
\end{verbatim}

In this example, the memory request for the I/O buffer is handled by
the C library allocator. The Python memory manager is involved only
in the allocation of the string object returned as a result.

In most situations, however, it is recommended to allocate memory from
the Python heap specifically because the latter is under control of
the Python memory manager. For example, this is required when the
interpreter is extended with new object types written in C. Another
reason for using the Python heap is the desire to \emph{inform} the
Python memory manager about the memory needs of the extension module.
Even when the requested memory is used exclusively for internal,
highly-specific purposes, delegating all memory requests to the Python
memory manager causes the interpreter to have a more accurate image of
its memory footprint as a whole. Consequently, under certain
circumstances, the Python memory manager may or may not trigger
appropriate actions, like garbage collection, memory compaction or
other preventive procedures. Note that by using the C library
allocator as shown in the previous example, the allocated memory for
the I/O buffer escapes completely the Python memory manager.


\section{Memory Interface \label{memoryInterface}}

The following function sets, modeled after the ANSI C standard, are
available for allocating and releasing memory from the Python heap:


\begin{cfuncdesc}{void*}{PyMem_Malloc}{size_t n}
Allocates \var{n} bytes and returns a pointer of type \ctype{void*} to
the allocated memory, or \NULL{} if the request fails. Requesting zero
bytes returns a non-\NULL{} pointer.
\end{cfuncdesc}

\begin{cfuncdesc}{void*}{PyMem_Realloc}{void *p, size_t n}
Resizes the memory block pointed to by \var{p} to \var{n} bytes. The
contents will be unchanged to the minimum of the old and the new
sizes. If \var{p} is \NULL{}, the call is equivalent to
\cfunction{PyMem_Malloc(\var{n})}; if \var{n} is equal to zero, the memory block
is resized but is not freed, and the returned pointer is non-\NULL{}.
Unless \var{p} is \NULL{}, it must have been returned by a previous
call to \cfunction{PyMem_Malloc()} or \cfunction{PyMem_Realloc()}.
\end{cfuncdesc}

\begin{cfuncdesc}{void}{PyMem_Free}{void *p}
Frees the memory block pointed to by \var{p}, which must have been
returned by a previous call to \cfunction{PyMem_Malloc()} or
\cfunction{PyMem_Realloc()}.  Otherwise, or if
\cfunction{PyMem_Free(p)} has been called before, undefined behaviour
occurs. If \var{p} is \NULL{}, no operation is performed.
\end{cfuncdesc}

The following type-oriented macros are provided for convenience.  Note 
that \var{TYPE} refers to any C type.

\begin{cfuncdesc}{\var{TYPE}*}{PyMem_New}{TYPE, size_t n}
Same as \cfunction{PyMem_Malloc()}, but allocates \code{(\var{n} *
sizeof(\var{TYPE}))} bytes of memory.  Returns a pointer cast to
\ctype{\var{TYPE}*}.
\end{cfuncdesc}

\begin{cfuncdesc}{\var{TYPE}*}{PyMem_Resize}{void *p, TYPE, size_t n}
Same as \cfunction{PyMem_Realloc()}, but the memory block is resized
to \code{(\var{n} * sizeof(\var{TYPE}))} bytes.  Returns a pointer
cast to \ctype{\var{TYPE}*}.
\end{cfuncdesc}

\begin{cfuncdesc}{void}{PyMem_Del}{void *p}
Same as \cfunction{PyMem_Free()}.
\end{cfuncdesc}

In addition, the following macro sets are provided for calling the
Python memory allocator directly, without involving the C API functions
listed above. However, note that their use does not preserve binary
compatibility accross Python versions and is therefore deprecated in
extension modules.

\cfunction{PyMem_MALLOC()}, \cfunction{PyMem_REALLOC()}, \cfunction{PyMem_FREE()}.

\cfunction{PyMem_NEW()}, \cfunction{PyMem_RESIZE()}, \cfunction{PyMem_DEL()}.


\section{Examples \label{memoryExamples}}

Here is the example from section \ref{memoryOverview}, rewritten so
that the I/O buffer is allocated from the Python heap by using the
first function set:

\begin{verbatim}
    PyObject *res;
    char *buf = (char *) PyMem_Malloc(BUFSIZ); /* for I/O */

    if (buf == NULL)
        return PyErr_NoMemory();
    /* ...Do some I/O operation involving buf... */
    res = PyString_FromString(buf);
    PyMem_Free(buf); /* allocated with PyMem_Malloc */
    return res;
\end{verbatim}

The same code using the type-oriented function set:

\begin{verbatim}
    PyObject *res;
    char *buf = PyMem_New(char, BUFSIZ); /* for I/O */

    if (buf == NULL)
        return PyErr_NoMemory();
    /* ...Do some I/O operation involving buf... */
    res = PyString_FromString(buf);
    PyMem_Del(buf); /* allocated with PyMem_New */
    return res;
\end{verbatim}

Note that in the two examples above, the buffer is always
manipulated via functions belonging to the same set. Indeed, it
is required to use the same memory API family for a given
memory block, so that the risk of mixing different allocators is
reduced to a minimum. The following code sequence contains two errors,
one of which is labeled as \emph{fatal} because it mixes two different
allocators operating on different heaps.

\begin{verbatim}
char *buf1 = PyMem_New(char, BUFSIZ);
char *buf2 = (char *) malloc(BUFSIZ);
char *buf3 = (char *) PyMem_Malloc(BUFSIZ);
...
PyMem_Del(buf3);  /* Wrong -- should be PyMem_Free() */
free(buf2);       /* Right -- allocated via malloc() */
free(buf1);       /* Fatal -- should be PyMem_Del()  */
\end{verbatim}

In addition to the functions aimed at handling raw memory blocks from
the Python heap, objects in Python are allocated and released with
\cfunction{PyObject_New()}, \cfunction{PyObject_NewVar()} and
\cfunction{PyObject_Del()}, or with their corresponding macros
\cfunction{PyObject_NEW()}, \cfunction{PyObject_NEW_VAR()} and
\cfunction{PyObject_DEL()}.

These will be explained in the next chapter on defining and
implementing new object types in C.


\chapter{Defining New Object Types \label{newTypes}}

\begin{cfuncdesc}{PyObject*}{_PyObject_New}{PyTypeObject *type}
\end{cfuncdesc}

\begin{cfuncdesc}{PyVarObject*}{_PyObject_NewVar}{PyTypeObject *type, int size}
\end{cfuncdesc}

\begin{cfuncdesc}{void}{_PyObject_Del}{PyObject *op}
\end{cfuncdesc}

\begin{cfuncdesc}{PyObject*}{PyObject_Init}{PyObject *op,
						PyTypeObject *type}
\end{cfuncdesc}

\begin{cfuncdesc}{PyVarObject*}{PyObject_InitVar}{PyVarObject *op,
						PyTypeObject *type, int size}
\end{cfuncdesc}

\begin{cfuncdesc}{\var{TYPE}*}{PyObject_New}{TYPE, PyTypeObject *type}
\end{cfuncdesc}

\begin{cfuncdesc}{\var{TYPE}*}{PyObject_NewVar}{TYPE, PyTypeObject *type,
                                                int size}
\end{cfuncdesc}

\begin{cfuncdesc}{void}{PyObject_Del}{PyObject *op}
\end{cfuncdesc}

\begin{cfuncdesc}{\var{TYPE}*}{PyObject_NEW}{TYPE, PyTypeObject *type}
\end{cfuncdesc}

\begin{cfuncdesc}{\var{TYPE}*}{PyObject_NEW_VAR}{TYPE, PyTypeObject *type,
                                                int size}
\end{cfuncdesc}

\begin{cfuncdesc}{void}{PyObject_DEL}{PyObject *op}
\end{cfuncdesc}

\begin{cfuncdesc}{PyObject*}{Py_InitModule}{char *name,
                                            PyMethodDef *methods}
  Create a new module object based on a name and table of functions,
  returning the new module object.
\end{cfuncdesc}

\begin{cfuncdesc}{PyObject*}{Py_InitModule3}{char *name,
                                             PyMethodDef *methods,
                                             char *doc}
  Create a new module object based on a name and table of functions,
  returning the new module object.  If \var{doc} is non-\NULL, it will
  be used to define the docstring for the module.
\end{cfuncdesc}

\begin{cfuncdesc}{PyObject*}{Py_InitModule4}{char *name,
                                             PyMethodDef *methods,
                                             char *doc, PyObject *self,
                                             int apiver}
  Create a new module object based on a name and table of functions,
  returning the new module object.  If \var{doc} is non-\NULL, it will
  be used to define the docstring for the module.  If \var{self} is
  non-\NULL, it will passed to the functions of the module as their
  (otherwise \NULL) first parameter.  (This was added as an
  experimental feature, and there are no known uses in the current
  version of Python.)  For \var{apiver}, the only value which should
  be passed is defined by the constant \constant{PYTHON_API_VERSION}.

  \strong{Note:}  Most uses of this function should probably be using
  the \cfunction{Py_InitModule3()} instead; only use this if you are
  sure you need it.
\end{cfuncdesc}

PyArg_ParseTupleAndKeywords, PyArg_ParseTuple, PyArg_Parse

Py_BuildValue

DL_IMPORT

_Py_NoneStruct


\section{Common Object Structures \label{common-structs}}

PyObject, PyVarObject

PyObject_HEAD, PyObject_HEAD_INIT, PyObject_VAR_HEAD

Typedefs:
unaryfunc, binaryfunc, ternaryfunc, inquiry, coercion, intargfunc,
intintargfunc, intobjargproc, intintobjargproc, objobjargproc,
destructor, printfunc, getattrfunc, getattrofunc, setattrfunc,
setattrofunc, cmpfunc, reprfunc, hashfunc

\begin{ctypedesc}{PyCFunction}
Type of the functions used to implement most Python callables in C.
\end{ctypedesc}

\begin{ctypedesc}{PyMethodDef}
Structure used to describe a method of an extension type.  This
structure has four fields:

\begin{tableiii}{l|l|l}{member}{Field}{C Type}{Meaning}
  \lineiii{ml_name}{char *}{name of the method}
  \lineiii{ml_meth}{PyCFunction}{pointer to the C implementation}
  \lineiii{ml_flags}{int}{flag bits indicating how the call should be
                          constructed}
  \lineiii{ml_doc}{char *}{points to the contents of the docstring}
\end{tableiii}
\end{ctypedesc}

\begin{cfuncdesc}{PyObject*}{Py_FindMethod}{PyMethodDef[] table,
                                            PyObject *ob, char *name}
Return a bound method object for an extension type implemented in C.
This function also handles the special attribute \member{__methods__},
returning a list of all the method names defined in \var{table}.
\end{cfuncdesc}


\section{Mapping Object Structures \label{mapping-structs}}

\begin{ctypedesc}{PyMappingMethods}
Structure used to hold pointers to the functions used to implement the 
mapping protocol for an extension type.
\end{ctypedesc}


\section{Number Object Structures \label{number-structs}}

\begin{ctypedesc}{PyNumberMethods}
Structure used to hold pointers to the functions an extension type
uses to implement the number protocol.
\end{ctypedesc}


\section{Sequence Object Structures \label{sequence-structs}}

\begin{ctypedesc}{PySequenceMethods}
Structure used to hold pointers to the functions which an object uses
to implement the sequence protocol.
\end{ctypedesc}


\section{Buffer Object Structures \label{buffer-structs}}
\sectionauthor{Greg J. Stein}{greg@lyra.org}

The buffer interface exports a model where an object can expose its
internal data as a set of chunks of data, where each chunk is
specified as a pointer/length pair.  These chunks are called
\dfn{segments} and are presumed to be non-contiguous in memory.

If an object does not export the buffer interface, then its
\member{tp_as_buffer} member in the \ctype{PyTypeObject} structure
should be \NULL{}.  Otherwise, the \member{tp_as_buffer} will point to
a \ctype{PyBufferProcs} structure.

\strong{Note:} It is very important that your
\ctype{PyTypeObject} structure uses \constant{Py_TPFLAGS_DEFAULT} for
the value of the \member{tp_flags} member rather than \code{0}.  This
tells the Python runtime that your \ctype{PyBufferProcs} structure
contains the \member{bf_getcharbuffer} slot. Older versions of Python
did not have this member, so a new Python interpreter using an old
extension needs to be able to test for its presence before using it.

\begin{ctypedesc}{PyBufferProcs}
Structure used to hold the function pointers which define an
implementation of the buffer protocol.

The first slot is \member{bf_getreadbuffer}, of type
\ctype{getreadbufferproc}.  If this slot is \NULL{}, then the object
does not support reading from the internal data.  This is
non-sensical, so implementors should fill this in, but callers should
test that the slot contains a non-\NULL{} value.

The next slot is \member{bf_getwritebuffer} having type
\ctype{getwritebufferproc}. This slot may be \NULL{} if the object
does not allow writing into its returned buffers.

The third slot is \member{bf_getsegcount}, with type
\ctype{getsegcountproc}.  This slot must not be \NULL{} and is used to 
inform the caller how many segments the object contains.  Simple
objects such as \ctype{PyString_Type} and
\ctype{PyBuffer_Type} objects contain a single segment.

The last slot is \member{bf_getcharbuffer}, of type
\ctype{getcharbufferproc}.  This slot will only be present if the
\constant{Py_TPFLAGS_HAVE_GETCHARBUFFER} flag is present in the
\member{tp_flags} field of the object's \ctype{PyTypeObject}.  Before using
this slot, the caller should test whether it is present by using the
\cfunction{PyType_HasFeature()}\ttindex{PyType_HasFeature()} function.
If present, it may be \NULL, indicating that the object's contents
cannot be used as \emph{8-bit characters}.
The slot function may also raise an error if the object's contents
cannot be interpreted as 8-bit characters.  For example, if the object
is an array which is configured to hold floating point values, an
exception may be raised if a caller attempts to use
\member{bf_getcharbuffer} to fetch a sequence of 8-bit characters.
This notion of exporting the internal buffers as ``text'' is used to
distinguish between objects that are binary in nature, and those which
have character-based content.

\strong{Note:} The current policy seems to state that these characters
may be multi-byte characters. This implies that a buffer size of
\var{N} does not mean there are \var{N} characters present.
\end{ctypedesc}

\begin{datadesc}{Py_TPFLAGS_HAVE_GETCHARBUFFER}
Flag bit set in the type structure to indicate that the
\member{bf_getcharbuffer} slot is known.  This being set does not
indicate that the object supports the buffer interface or that the
\member{bf_getcharbuffer} slot is non-\NULL.
\end{datadesc}

\begin{ctypedesc}[getreadbufferproc]{int (*getreadbufferproc)
                            (PyObject *self, int segment, void **ptrptr)}
Return a pointer to a readable segment of the buffer.  This function
is allowed to raise an exception, in which case it must return
\code{-1}.  The \var{segment} which is passed must be zero or
positive, and strictly less than the number of segments returned by
the \member{bf_getsegcount} slot function.  On success, it returns the
length of the buffer memory, and sets \code{*\var{ptrptr}} to a
pointer to that memory.
\end{ctypedesc}

\begin{ctypedesc}[getwritebufferproc]{int (*getwritebufferproc)
                            (PyObject *self, int segment, void **ptrptr)}
Return a pointer to a writable memory buffer in \code{*\var{ptrptr}},
and the length of that segment as the function return value.
The memory buffer must correspond to buffer segment \var{segment}.
Must return \code{-1} and set an exception on error.
\exception{TypeError} should be raised if the object only supports
read-only buffers, and \exception{SystemError} should be raised when
\var{segment} specifies a segment that doesn't exist.
% Why doesn't it raise ValueError for this one?
% GJS: because you shouldn't be calling it with an invalid
%      segment. That indicates a blatant programming error in the C
%      code.
\end{ctypedesc}

\begin{ctypedesc}[getsegcountproc]{int (*getsegcountproc)
                            (PyObject *self, int *lenp)}
Return the number of memory segments which comprise the buffer.  If
\var{lenp} is not \NULL, the implementation must report the sum of the 
sizes (in bytes) of all segments in \code{*\var{lenp}}.
The function cannot fail.
\end{ctypedesc}

\begin{ctypedesc}[getcharbufferproc]{int (*getcharbufferproc)
                            (PyObject *self, int segment, const char **ptrptr)}
\end{ctypedesc}


\section{Supporting Cyclic Garbarge Collection
         \label{supporting-cycle-detection}}

Python's support for detecting and collecting garbage which involves
circular references requires support from object types which are
``containers'' for other objects which may also be containers.  Types
which do not store references to other objects, or which only store
references to atomic types (such as numbers or strings), do not need
to provide any explicit support for garbage collection.

To create a container type, the \member{tp_flags} field of the type
object must include the \constant{Py_TPFLAGS_GC} and provide an
implementation of the \member{tp_traverse} handler.  The computed
value of the \member{tp_basicsize} field must include
\constant{PyGC_HEAD_SIZE} as well.  If instances of the type are
mutable, a \member{tp_clear} implementation must also be provided.

\begin{datadesc}{Py_TPFLAGS_GC}
  Objects with a type with this flag set must conform with the rules
  documented here.  For convenience these objects will be referred to
  as container objects.
\end{datadesc}

\begin{datadesc}{PyGC_HEAD_SIZE}
  Extra memory needed for the garbage collector.  Container objects
  must include this in the calculation of their tp_basicsize.  If the
  collector is disabled at compile time then this is \code{0}.
\end{datadesc}

Constructors for container types must conform to two rules:

\begin{enumerate}
\item  The memory for the object must be allocated using
       \cfunction{PyObject_New()} or \cfunction{PyObject_VarNew()}.

\item  Once all the fields which may contain references to other
       containers are initialized, it must call
       \cfunction{PyObject_GC_Init()}.
\end{enumerate}

\begin{cfuncdesc}{void}{PyObject_GC_Init}{PyObject *op}
  Adds the object \var{op} to the set of container objects tracked by
  the collector.  The collector can run at unexpected times so objects
  must be valid while being tracked.  This should be called once all
  the fields followed by the \member{tp_traverse} handler become valid,
  usually near the end of the constructor.
\end{cfuncdesc}

Similarly, the deallocator for the object must conform to a similar
pair of rules:

\begin{enumerate}
\item  Before fields which refer to other containers are invalidated,
       \cfunction{PyObject_GC_Fini()} must be called.

\item  The object's memory must be deallocated using
       \cfunction{PyObject_Del()}.
\end{enumerate}

\begin{cfuncdesc}{void}{PyObject_GC_Fini}{PyObject *op}
  Remove the object \var{op} from the set of container objects tracked
  by the collector.  Note that \cfunction{PyObject_GC_Init()} can be
  called again on this object to add it back to the set of tracked
  objects.  The deallocator (\member{tp_dealloc} handler) should call
  this for the object before any of the fields used by the
  \member{tp_traverse} handler become invalid.

  \strong{Note:}  Any container which may be referenced from another
  object reachable by the collector must itself be tracked by the
  collector, so it is generally not safe to call this function
  anywhere but in the object's deallocator.
\end{cfuncdesc}

The \member{tp_traverse} handler accepts a function parameter of this
type:

\begin{ctypedesc}[visitproc]{int (*visitproc)(PyObject *object, void *arg)}
  Type of the visitor function passed to the \member{tp_traverse}
  handler.  The function should be called with an object to traverse
  as \var{object} and the third parameter to the \member{tp_traverse}
  handler as \var{arg}.
\end{ctypedesc}

The \member{tp_traverse} handler must have the following type:

\begin{ctypedesc}[traverseproc]{int (*traverseproc)(PyObject *self,
                                visitproc visit, void *arg)}
  Traversal function for a container object.  Implementations must
  call the \var{visit} function for each object directly contained by
  \var{self}, with the parameters to \var{visit} being the contained
  object and the \var{arg} value passed to the handler.  If
  \var{visit} returns a non-zero value then an error has occurred and
  that value should be returned immediately.
\end{ctypedesc}

The \member{tp_clear} handler must be of the \ctype{inquiry} type, or
\NULL{} if the object is immutable.

\begin{ctypedesc}[inquiry]{int (*inquiry)(PyObject *self)}
  Drop references that may have created reference cycles.  Immutable
  objects do not have to define this method since they can never
  directly create reference cycles.  Note that the object must still
  be valid after calling this method (i.e., don't just call
  \cfunction{Py_DECREF()} on a reference).  The collector will call
  this method if it detects that this object is involved in a
  reference cycle.
\end{ctypedesc}


\subsection{Example Cycle Collector Support
            \label{example-cycle-support}}

This example shows only enough of the implementation of an extension
type to show how the garbage collector support needs to be added.  It
shows the definition of the object structure, the
\member{tp_traverse}, \member{tp_clear} and \member{tp_dealloc}
implementations, the type structure, and a constructor --- the module
initialization needed to export the constructor to Python is not shown
as there are no special considerations there for the collector.  To
make this interesting, assume that the module exposes ways for the
\member{container} field of the object to be modified.  Note that
since no checks are made on the type of the object used to initialize
\member{container}, we have to assume that it may be a container.

\begin{verbatim}
#include "Python.h"

typedef struct {
    PyObject_HEAD
    PyObject *container;
} MyObject;

static int
my_traverse(MyObject *self, visitproc visit, void *arg)
{
    if (self->container != NULL)
        return visit(self->container, arg);
    else
        return 0;
}

static int
my_clear(MyObject *self)
{
    Py_XDECREF(self->container);
    self->container = NULL;

    return 0;
}

static void
my_dealloc(MyObject *self)
{
    PyObject_GC_Fini((PyObject *) self);
    Py_XDECREF(self->container);
    PyObject_Del(self);
}
\end{verbatim}

\begin{verbatim}
statichere PyTypeObject
MyObject_Type = {
    PyObject_HEAD_INIT(NULL)
    0,
    "MyObject",
    sizeof(MyObject) + PyGC_HEAD_SIZE,
    0,
    (destructor)my_dealloc,     /* tp_dealloc */
    0,                          /* tp_print */
    0,                          /* tp_getattr */
    0,                          /* tp_setattr */
    0,                          /* tp_compare */
    0,                          /* tp_repr */
    0,                          /* tp_as_number */
    0,                          /* tp_as_sequence */
    0,                          /* tp_as_mapping */
    0,                          /* tp_hash */
    0,                          /* tp_call */
    0,                          /* tp_str */
    0,                          /* tp_getattro */
    0,                          /* tp_setattro */
    0,                          /* tp_as_buffer */
    Py_TPFLAGS_DEFAULT | Py_TPFLAGS_GC,
    0,                          /* tp_doc */
    (traverseproc)my_traverse,  /* tp_traverse */
    (inquiry)my_clear,          /* tp_clear */
    0,                          /* tp_richcompare */
    0,                          /* tp_weaklistoffset */
};

/* This constructor should be made accessible from Python. */
static PyObject *
new_object(PyObject *unused, PyObject *args)
{
    PyObject *container = NULL;
    MyObject *result = NULL;

    if (PyArg_ParseTuple(args, "|O:new_object", &container)) {
        result = PyObject_New(MyObject, &MyObject_Type);
        if (result != NULL) {
            result->container = container;
            PyObject_GC_Init();
        }
    }
    return (PyObject *) result;
}
\end{verbatim}


% \chapter{Debugging \label{debugging}}
%
% XXX Explain Py_DEBUG, Py_TRACE_REFS, Py_REF_DEBUG.


\appendix
\chapter{Reporting Bugs}
\input{reportingbugs}

\input{api.ind}			% Index -- must be last

\end{document}
			% Index -- must be last

\end{document}
			% Index -- must be last

\end{document}
			% Index -- must be last

\end{document}
