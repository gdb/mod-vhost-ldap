\chapter{Exception Handling \label{exceptionHandling}}

The functions described in this chapter will let you handle and raise Python
exceptions.  It is important to understand some of the basics of
Python exception handling.  It works somewhat like the
\UNIX{} \cdata{errno} variable: there is a global indicator (per
thread) of the last error that occurred.  Most functions don't clear
this on success, but will set it to indicate the cause of the error on
failure.  Most functions also return an error indicator, usually
\NULL{} if they are supposed to return a pointer, or \code{-1} if they
return an integer (exception: the \cfunction{PyArg_*()} functions
return \code{1} for success and \code{0} for failure).

When a function must fail because some function it called failed, it
generally doesn't set the error indicator; the function it called
already set it.  It is responsible for either handling the error and
clearing the exception or returning after cleaning up any resources it
holds (such as object references or memory allocations); it should
\emph{not} continue normally if it is not prepared to handle the
error.  If returning due to an error, it is important to indicate to
the caller that an error has been set.  If the error is not handled or
carefully propogated, additional calls into the Python/C API may not
behave as intended and may fail in mysterious ways.

The error indicator consists of three Python objects corresponding to
\withsubitem{(in module sys)}{
  \ttindex{exc_type}\ttindex{exc_value}\ttindex{exc_traceback}}
the Python variables \code{sys.exc_type}, \code{sys.exc_value} and
\code{sys.exc_traceback}.  API functions exist to interact with the
error indicator in various ways.  There is a separate error indicator
for each thread.

% XXX Order of these should be more thoughtful.
% Either alphabetical or some kind of structure.

\begin{cfuncdesc}{void}{PyErr_Print}{}
  Print a standard traceback to \code{sys.stderr} and clear the error
  indicator.  Call this function only when the error indicator is
  set.  (Otherwise it will cause a fatal error!)
\end{cfuncdesc}

\begin{cfuncdesc}{PyObject*}{PyErr_Occurred}{}
  Test whether the error indicator is set.  If set, return the
  exception \emph{type} (the first argument to the last call to one of
  the \cfunction{PyErr_Set*()} functions or to
  \cfunction{PyErr_Restore()}).  If not set, return \NULL.  You do
  not own a reference to the return value, so you do not need to
  \cfunction{Py_DECREF()} it.  \note{Do not compare the return value
    to a specific exception; use \cfunction{PyErr_ExceptionMatches()}
    instead, shown below.  (The comparison could easily fail since the
    exception may be an instance instead of a class, in the case of a
    class exception, or it may the a subclass of the expected
    exception.)}
\end{cfuncdesc}

\begin{cfuncdesc}{int}{PyErr_ExceptionMatches}{PyObject *exc}
  Equivalent to \samp{PyErr_GivenExceptionMatches(PyErr_Occurred(),
  \var{exc})}.  This should only be called when an exception is
  actually set; a memory access violation will occur if no exception
  has been raised.
\end{cfuncdesc}

\begin{cfuncdesc}{int}{PyErr_GivenExceptionMatches}{PyObject *given, PyObject *exc}
  Return true if the \var{given} exception matches the exception in
  \var{exc}.  If \var{exc} is a class object, this also returns true
  when \var{given} is an instance of a subclass.  If \var{exc} is a
  tuple, all exceptions in the tuple (and recursively in subtuples)
  are searched for a match.  If \var{given} is \NULL, a memory access
  violation will occur.
\end{cfuncdesc}

\begin{cfuncdesc}{void}{PyErr_NormalizeException}{PyObject**exc, PyObject**val, PyObject**tb}
  Under certain circumstances, the values returned by
  \cfunction{PyErr_Fetch()} below can be ``unnormalized'', meaning
  that \code{*\var{exc}} is a class object but \code{*\var{val}} is
  not an instance of the  same class.  This function can be used to
  instantiate the class in that case.  If the values are already
  normalized, nothing happens.  The delayed normalization is
  implemented to improve performance.
\end{cfuncdesc}

\begin{cfuncdesc}{void}{PyErr_Clear}{}
  Clear the error indicator.  If the error indicator is not set, there
  is no effect.
\end{cfuncdesc}

\begin{cfuncdesc}{void}{PyErr_Fetch}{PyObject **ptype, PyObject **pvalue,
                                     PyObject **ptraceback}
  Retrieve the error indicator into three variables whose addresses
  are passed.  If the error indicator is not set, set all three
  variables to \NULL.  If it is set, it will be cleared and you own a
  reference to each object retrieved.  The value and traceback object
  may be \NULL{} even when the type object is not.  \note{This
  function is normally only used by code that needs to handle
  exceptions or by code that needs to save and restore the error
  indicator temporarily.}
\end{cfuncdesc}

\begin{cfuncdesc}{void}{PyErr_Restore}{PyObject *type, PyObject *value,
                                       PyObject *traceback}
  Set  the error indicator from the three objects.  If the error
  indicator is already set, it is cleared first.  If the objects are
  \NULL, the error indicator is cleared.  Do not pass a \NULL{} type
  and non-\NULL{} value or traceback.  The exception type should be a
  string or class; if it is a class, the value should be an instance
  of that class.  Do not pass an invalid exception type or value.
  (Violating these rules will cause subtle problems later.)  This call
  takes away a reference to each object: you must own a reference to
  each object before the call and after the call you no longer own
  these references.  (If you don't understand this, don't use this
  function.  I warned you.)  \note{This function is normally only used
  by code that needs to save and restore the error indicator
  temporarily.}
\end{cfuncdesc}

\begin{cfuncdesc}{void}{PyErr_SetString}{PyObject *type, char *message}
  This is the most common way to set the error indicator.  The first
  argument specifies the exception type; it is normally one of the
  standard exceptions, e.g. \cdata{PyExc_RuntimeError}.  You need not
  increment its reference count.  The second argument is an error
  message; it is converted to a string object.
\end{cfuncdesc}

\begin{cfuncdesc}{void}{PyErr_SetObject}{PyObject *type, PyObject *value}
  This function is similar to \cfunction{PyErr_SetString()} but lets
  you specify an arbitrary Python object for the ``value'' of the
  exception.  You need not increment its reference count.
\end{cfuncdesc}

\begin{cfuncdesc}{PyObject*}{PyErr_Format}{PyObject *exception,
                                           const char *format, \moreargs}
  This function sets the error indicator and returns \NULL..
  \var{exception} should be a Python exception (string or class, not
  an instance).  \var{format} should be a string, containing format
  codes, similar to \cfunction{printf()}. The \code{width.precision}
  before a format code is parsed, but the width part is ignored.

  \begin{tableii}{c|l}{character}{Character}{Meaning}
    \lineii{c}{Character, as an \ctype{int} parameter}
    \lineii{d}{Number in decimal, as an \ctype{int} parameter}
    \lineii{x}{Number in hexadecimal, as an \ctype{int} parameter}
    \lineii{s}{A string, as a \ctype{char *} parameter}
    \lineii{p}{A hex pointer, as a \ctype{void *} parameter}
  \end{tableii}

  An unrecognized format character causes all the rest of the format
  string to be copied as-is to the result string, and any extra
  arguments discarded.
\end{cfuncdesc}

\begin{cfuncdesc}{void}{PyErr_SetNone}{PyObject *type}
  This is a shorthand for \samp{PyErr_SetObject(\var{type},
  Py_None)}.
\end{cfuncdesc}

\begin{cfuncdesc}{int}{PyErr_BadArgument}{}
  This is a shorthand for \samp{PyErr_SetString(PyExc_TypeError,
  \var{message})}, where \var{message} indicates that a built-in
  operation was invoked with an illegal argument.  It is mostly for
  internal use.
\end{cfuncdesc}

\begin{cfuncdesc}{PyObject*}{PyErr_NoMemory}{}
  This is a shorthand for \samp{PyErr_SetNone(PyExc_MemoryError)}; it
  returns \NULL{} so an object allocation function can write
  \samp{return PyErr_NoMemory();} when it runs out of memory.
\end{cfuncdesc}

\begin{cfuncdesc}{PyObject*}{PyErr_SetFromErrno}{PyObject *type}
  This is a convenience function to raise an exception when a C
  library function has returned an error and set the C variable
  \cdata{errno}.  It constructs a tuple object whose first item is the
  integer \cdata{errno} value and whose second item is the
  corresponding error message (gotten from
  \cfunction{strerror()}\ttindex{strerror()}), and then calls
  \samp{PyErr_SetObject(\var{type}, \var{object})}.  On \UNIX, when
  the \cdata{errno} value is \constant{EINTR}, indicating an
  interrupted system call, this calls
  \cfunction{PyErr_CheckSignals()}, and if that set the error
  indicator, leaves it set to that.  The function always returns
  \NULL, so a wrapper function around a system call can write
  \samp{return PyErr_SetFromErrno(\var{type});} when the system call
  returns an error.
\end{cfuncdesc}

\begin{cfuncdesc}{PyObject*}{PyErr_SetFromErrnoWithFilename}{PyObject *type,
                                                             char *filename}
  Similar to \cfunction{PyErr_SetFromErrno()}, with the additional
  behavior that if \var{filename} is not \NULL, it is passed to the
  constructor of \var{type} as a third parameter.  In the case of
  exceptions such as \exception{IOError} and \exception{OSError}, this
  is used to define the \member{filename} attribute of the exception
  instance.
\end{cfuncdesc}

\begin{cfuncdesc}{PyObject*}{PyErr_SetFromWindowsErr}{int ierr}
  This is a convenience function to raise \exception{WindowsError}.
  If called with \var{ierr} of \cdata{0}, the error code returned by a
  call to \cfunction{GetLastError()} is used instead.  It calls the
  Win32 function \cfunction{FormatMessage()} to retrieve the Windows
  description of error code given by \var{ierr} or
  \cfunction{GetLastError()}, then it constructs a tuple object whose
  first item is the \var{ierr} value and whose second item is the
  corresponding error message (gotten from
  \cfunction{FormatMessage()}), and then calls
  \samp{PyErr_SetObject(\var{PyExc_WindowsError}, \var{object})}.
  This function always returns \NULL.
  Availability: Windows.
\end{cfuncdesc}

\begin{cfuncdesc}{PyObject*}{PyErr_SetFromWindowsErrWithFilename}{int ierr,
                                                                char *filename}
  Similar to \cfunction{PyErr_SetFromWindowsErr()}, with the
  additional behavior that if \var{filename} is not \NULL, it is
  passed to the constructor of \exception{WindowsError} as a third
  parameter.
  Availability: Windows.
\end{cfuncdesc}

\begin{cfuncdesc}{void}{PyErr_BadInternalCall}{}
  This is a shorthand for \samp{PyErr_SetString(PyExc_TypeError,
  \var{message})}, where \var{message} indicates that an internal
  operation (e.g. a Python/C API function) was invoked with an illegal
  argument.  It is mostly for internal use.
\end{cfuncdesc}

\begin{cfuncdesc}{int}{PyErr_Warn}{PyObject *category, char *message}
  Issue a warning message.  The \var{category} argument is a warning
  category (see below) or \NULL; the \var{message} argument is a
  message string.

  This function normally prints a warning message to \var{sys.stderr};
  however, it is also possible that the user has specified that
  warnings are to be turned into errors, and in that case this will
  raise an exception.  It is also possible that the function raises an
  exception because of a problem with the warning machinery (the
  implementation imports the \module{warnings} module to do the heavy
  lifting).  The return value is \code{0} if no exception is raised,
  or \code{-1} if an exception is raised.  (It is not possible to
  determine whether a warning message is actually printed, nor what
  the reason is for the exception; this is intentional.)  If an
  exception is raised, the caller should do its normal exception
  handling (for example, \cfunction{Py_DECREF()} owned references and
  return an error value).

  Warning categories must be subclasses of \cdata{Warning}; the
  default warning category is \cdata{RuntimeWarning}.  The standard
  Python warning categories are available as global variables whose
  names are \samp{PyExc_} followed by the Python exception name.
  These have the type \ctype{PyObject*}; they are all class objects.
  Their names are \cdata{PyExc_Warning}, \cdata{PyExc_UserWarning},
  \cdata{PyExc_DeprecationWarning}, \cdata{PyExc_SyntaxWarning}, and
  \cdata{PyExc_RuntimeWarning}.  \cdata{PyExc_Warning} is a subclass
  of \cdata{PyExc_Exception}; the other warning categories are
  subclasses of \cdata{PyExc_Warning}.

  For information about warning control, see the documentation for the
  \module{warnings} module and the \programopt{-W} option in the
  command line documentation.  There is no C API for warning control.
\end{cfuncdesc}

\begin{cfuncdesc}{int}{PyErr_WarnExplicit}{PyObject *category, char *message,
                char *filename, int lineno, char *module, PyObject *registry}
  Issue a warning message with explicit control over all warning
  attributes.  This is a straightforward wrapper around the Python
  function \function{warnings.warn_explicit()}, see there for more
  information.  The \var{module} and \var{registry} arguments may be
  set to \NULL{} to get the default effect described there.
\end{cfuncdesc}

\begin{cfuncdesc}{int}{PyErr_CheckSignals}{}
  This function interacts with Python's signal handling.  It checks
  whether a signal has been sent to the processes and if so, invokes
  the corresponding signal handler.  If the
  \module{signal}\refbimodindex{signal} module is supported, this can
  invoke a signal handler written in Python.  In all cases, the
  default effect for \constant{SIGINT}\ttindex{SIGINT} is to raise the
  \withsubitem{(built-in exception)}{\ttindex{KeyboardInterrupt}}
  \exception{KeyboardInterrupt} exception.  If an exception is raised
  the error indicator is set and the function returns \code{1};
  otherwise the function returns \code{0}.  The error indicator may or
  may not be cleared if it was previously set.
\end{cfuncdesc}

\begin{cfuncdesc}{void}{PyErr_SetInterrupt}{}
  This function is obsolete.  It simulates the effect of a
  \constant{SIGINT}\ttindex{SIGINT} signal arriving --- the next time
  \cfunction{PyErr_CheckSignals()} is called,
  \withsubitem{(built-in exception)}{\ttindex{KeyboardInterrupt}}
  \exception{KeyboardInterrupt} will be raised.  It may be called
  without holding the interpreter lock.
\end{cfuncdesc}

\begin{cfuncdesc}{PyObject*}{PyErr_NewException}{char *name,
                                                 PyObject *base,
                                                 PyObject *dict}
  This utility function creates and returns a new exception object.
  The \var{name} argument must be the name of the new exception, a C
  string of the form \code{module.class}.  The \var{base} and
  \var{dict} arguments are normally \NULL.  This creates a class
  object derived from the root for all exceptions, the built-in name
  \exception{Exception} (accessible in C as \cdata{PyExc_Exception}).
  The \member{__module__} attribute of the new class is set to the
  first part (up to the last dot) of the \var{name} argument, and the
  class name is set to the last part (after the last dot).  The
  \var{base} argument can be used to specify an alternate base class.
  The \var{dict} argument can be used to specify a dictionary of class
  variables and methods.
\end{cfuncdesc}

\begin{cfuncdesc}{void}{PyErr_WriteUnraisable}{PyObject *obj}
  This utility function prints a warning message to \code{sys.stderr}
  when an exception has been set but it is impossible for the
  interpreter to actually raise the exception.  It is used, for
  example, when an exception occurs in an \method{__del__()} method.

  The function is called with a single argument \var{obj} that
  identifies where the context in which the unraisable exception
  occurred.  The repr of \var{obj} will be printed in the warning
  message.
\end{cfuncdesc}

\section{Standard Exceptions \label{standardExceptions}}

All standard Python exceptions are available as global variables whose
names are \samp{PyExc_} followed by the Python exception name.  These
have the type \ctype{PyObject*}; they are all class objects.  For
completeness, here are all the variables:

\begin{tableiii}{l|l|c}{cdata}{C Name}{Python Name}{Notes}
  \lineiii{PyExc_Exception}{\exception{Exception}}{(1)}
  \lineiii{PyExc_StandardError}{\exception{StandardError}}{(1)}
  \lineiii{PyExc_ArithmeticError}{\exception{ArithmeticError}}{(1)}
  \lineiii{PyExc_LookupError}{\exception{LookupError}}{(1)}
  \lineiii{PyExc_AssertionError}{\exception{AssertionError}}{}
  \lineiii{PyExc_AttributeError}{\exception{AttributeError}}{}
  \lineiii{PyExc_EOFError}{\exception{EOFError}}{}
  \lineiii{PyExc_EnvironmentError}{\exception{EnvironmentError}}{(1)}
  \lineiii{PyExc_FloatingPointError}{\exception{FloatingPointError}}{}
  \lineiii{PyExc_IOError}{\exception{IOError}}{}
  \lineiii{PyExc_ImportError}{\exception{ImportError}}{}
  \lineiii{PyExc_IndexError}{\exception{IndexError}}{}
  \lineiii{PyExc_KeyError}{\exception{KeyError}}{}
  \lineiii{PyExc_KeyboardInterrupt}{\exception{KeyboardInterrupt}}{}
  \lineiii{PyExc_MemoryError}{\exception{MemoryError}}{}
  \lineiii{PyExc_NameError}{\exception{NameError}}{}
  \lineiii{PyExc_NotImplementedError}{\exception{NotImplementedError}}{}
  \lineiii{PyExc_OSError}{\exception{OSError}}{}
  \lineiii{PyExc_OverflowError}{\exception{OverflowError}}{}
  \lineiii{PyExc_ReferenceError}{\exception{ReferenceError}}{(2)}
  \lineiii{PyExc_RuntimeError}{\exception{RuntimeError}}{}
  \lineiii{PyExc_SyntaxError}{\exception{SyntaxError}}{}
  \lineiii{PyExc_SystemError}{\exception{SystemError}}{}
  \lineiii{PyExc_SystemExit}{\exception{SystemExit}}{}
  \lineiii{PyExc_TypeError}{\exception{TypeError}}{}
  \lineiii{PyExc_ValueError}{\exception{ValueError}}{}
  \lineiii{PyExc_WindowsError}{\exception{WindowsError}}{(3)}
  \lineiii{PyExc_ZeroDivisionError}{\exception{ZeroDivisionError}}{}
\end{tableiii}

\noindent
Notes:
\begin{description}
\item[(1)]
  This is a base class for other standard exceptions.

\item[(2)]
  This is the same as \exception{weakref.ReferenceError}.

\item[(3)]
  Only defined on Windows; protect code that uses this by testing that
  the preprocessor macro \code{MS_WINDOWS} is defined.
\end{description}


\section{Deprecation of String Exceptions}

All exceptions built into Python or provided in the standard library
are derived from \exception{Exception}.
\withsubitem{(built-in exception)}{\ttindex{Exception}}

String exceptions are still supported in the interpreter to allow
existing code to run unmodified, but this will also change in a future 
release.
