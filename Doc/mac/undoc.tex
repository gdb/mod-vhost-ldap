\chapter{Undocumented Modules \label{undocumented-modules}}


The modules in this chapter are poorly documented (if at all).  If you
wish to contribute documentation of any of these modules, please get in
touch with
\ulink{\email{python-docs@python.org}}{mailto:python-docs@python.org}.

\localmoduletable


\section{\module{buildtools} --- Helper module for BuildApplet and Friends}
\declaremodule{standard}{buildtools}
  \platform{Mac}
\modulesynopsis{Helper module for BuildApplet, BuildApplication and
                macfreeze}


\section{\module{py_resource} --- Resources from Python code}
\declaremodule[pyresource]{standard}{py_resource}
  \platform{Mac}
\modulesynopsis{Helper to create \texttt{'PYC '} resources for compiled
                applications}

This module is primarily used as a help module for BuildApplet and
BuildApplication.  It is able to store compiled Python code as
\texttt{'PYC '} resources in a file.


\section{\module{cfmfile} --- Code Fragment Resource module}
\declaremodule{standard}{cfmfile}
  \platform{Mac}
\modulesynopsis{Code Fragment Resource module}

\module{cfmfile} is a module that understands Code Fragments and the
accompanying ``cfrg'' resources. It can parse them and merge them, and is
used by BuildApplication to combine all plugin modules to a single
executable.


\section{\module{macerrors} --- MacOS Errors}
\declaremodule{standard}{macerrors}
  \platform{Mac}
\modulesynopsis{Constant definitions for many MacOS error codes}

\module{macerrors} cotains constant definitions for many MacOS error codes.


\section{\module{macfsn} --- NavServices calls}
\declaremodule{standard}{macfsn}
  \platform{Mac}
\modulesynopsis{NavServices versions of StandardFile calls}


\module{macfsn} contains wrapper functions that have the same API as the macfs 
StandardFile calls, but are implemented with Navigation Services. Importing it 
will replace the methods in macfs with these, if Navigation Services is 
available on your machine.


\section{\module{icopen} --- Internet Config replacement for \method{open()}}
\declaremodule{standard}{icopen}
  \platform{Mac}
\modulesynopsis{Internet Config replacement for \method{open()}}

Importing \module{icopen} will replace the builtin \method{open()}
with a version that uses Internet Config to set file type and creator
for new files.


\section{\module{mactty} --- Serial line connections}
\declaremodule{standard}{mactty}
  \platform{Mac}
\modulesynopsis{Easy access serial to line connections}


\section{\module{nsremote} --- Wrapper around Netscape OSA modules}
\declaremodule{standard}{nsremote}
  \platform{Mac}
\modulesynopsis{Wrapper around Netscape OSA modules}

\module{nsremote} is a wrapper around the Netscape OSA modules that
allows you to easily send your browser to a given URL.  A related
module that may be of interest is the \module{webbrowser} module,
documented in the \citetitle[../lib/lib.html]{Python Library
Reference}.


\section{\module{PixMapWrapper} --- Wrapper for PixMap objects}
\declaremodule{standard}{PixMapWrapper}
  \platform{Mac}
\modulesynopsis{Wrapper for PixMap objects}

\module{PixMapWrapper} wraps a PixMap object with a Python object that
allows access to the fields by name. It also has methods to convert
to and from \module{PIL} images.


\section{\module{preferences} --- Application preferences manager}
\declaremodule{standard}{preferences}
  \platform{Mac}
\modulesynopsis{Nice application preferences manager with support for
                defaults}

The \module{preferences} module allows storage of user preferences in
the system-wide preferences folder, with defaults coming from the
application itself and the possibility to override preferences for
specific situations.


\section{\module{pythonprefs} --- Preferences manager for Python}
\declaremodule{standard}{pythonprefs}
  \platform{Mac}
\modulesynopsis{Specialized preferences manager for the Python
                interpreter}

This module is a specialization of the \refmodule{preferences} module
that allows reading and writing of the preferences for the Python
interpreter.


\section{\module{quietconsole} --- non-visible stdout output}
\declaremodule{standard}{quietconsole}
  \platform{Mac}
\modulesynopsis{buffered, non-visible stdout output}

\module{quietconsole} allows you to keep stdio output in a buffer
without displaying it (or without displaying the stdout window
altogether, if set with \program{EditPythonPrefs}) until you try to read from
stdin or disable the buffering, at which point all the saved output is
sent to the window.  Good for programs with graphilcal user interfaces
that do want to display their output at a crash.


\section{\module{W} --- Widgets built on \module{FrameWork}}
\declaremodule{standard}{W}
  \platform{Mac}
\modulesynopsis{Widgets for the Mac, built on top of \module{FrameWork}}

The \module{W} widgets are used extensively in the \program{IDE}.
