\section{Built-in Module \sectcode{macfs}}
\bimodindex{macfs}

\renewcommand{\indexsubitem}{(in module macfs)}

This module provides access to macintosh FSSpec handling, the Alias
Manager, finder aliases and the Standard File package.

Whenever a function or method expects a \var{file} argument, this
argument can be one of three things:\ (1) a full or partial Macintosh
pathname, (2) an FSSpec object or (3) a 3-tuple \code{(wdRefNum,
parID, name)} as described in Inside Mac VI\@. A description of aliases
and the standard file package can also be found there.

\begin{funcdesc}{FSSpec}{file}
Create an FSSpec object for the specified file.
\end{funcdesc}

\begin{funcdesc}{RawFSSpec}{data}
Create an FSSpec object given the raw data for the C structure for the
FSSpec as a string.  This is mainly useful if you have obtained an
FSSpec structure over a network.
\end{funcdesc}

\begin{funcdesc}{RawAlias}{data}
Create an Alias object given the raw data for the C structure for the
alias as a string.  This is mainly useful if you have obtained an
FSSpec structure over a network.
\end{funcdesc}

\begin{funcdesc}{FInfo}{}
Create a zero-filled FInfo object.
\end{funcdesc}

\begin{funcdesc}{ResolveAliasFile}{file}
Resolve an alias file. Returns a 3-tuple \code{(\var{fsspec}, \var{isfolder},
\var{aliased})} where \var{fsspec} is the resulting FSSpec object,
\var{isfolder} is true if \var{fsspec} points to a folder and
\var{aliased} is true if the file was an alias in the first place
(otherwise the FSSpec object for the file itself is returned).
\end{funcdesc}

\begin{funcdesc}{StandardGetFile}{\optional{type\, ...}}
Present the user with a standard ``open input file''
dialog. Optionally, you can pass up to four 4-char file types to limit
the files the user can choose from. The function returns an FSSpec
object and a flag indicating that the user completed the dialog
without cancelling.
\end{funcdesc}

\begin{funcdesc}{StandardPutFile}{prompt\, \optional{default}}
Present the user with a standard ``open output file''
dialog. \var{prompt} is the prompt string, and the optional
\var{default} argument initializes the output file name. The function
returns an FSSpec object and a flag indicating that the user completed
the dialog without cancelling.
\end{funcdesc}

\begin{funcdesc}{GetDirectory}{}
Present the user with a non-standard ``select a directory''
dialog. Return an FSSpec object and a success-indicator.
\end{funcdesc}

\begin{funcdesc}{FindFolder}{where\, which\, create}
Locates one of the ``special'' folders that MacOS knows about, such as
the trash or the Preferences folder. \var{Where} is the disk to search
(\code{0x8000} for the boot disk), \var{which} is the 4-char string
specifying which folder to locate. Setting \var{create} causes the
folder to be created if it does not exist. Returns a \code{(vrefnum,
dirid)} tuple. See Inside Mac VI for a complete description, including
4-char names.
\end{funcdesc}

\subsection{FSSpec objects}

\renewcommand{\indexsubitem}{(FSSpec object attribute)}
\begin{datadesc}{data}
The raw data from the FSSpec object, suitable for passing
to other applications, for instance.
\end{datadesc}

\renewcommand{\indexsubitem}{(FSSpec object method)}
\begin{funcdesc}{as_pathname}{}
Return the full pathname of the file described by the FSSpec object.
\end{funcdesc}

\begin{funcdesc}{as_tuple}{}
Return the \code{(\var{wdRefNum}, \var{parID}, \var{name})} tuple of the file described
by the FSSpec object.
\end{funcdesc}

\begin{funcdesc}{NewAlias}{\optional{file}}
Create an Alias object pointing to the file described by this
FSSpec. If the optional \var{file} parameter is present the alias
will be relative to that file, otherwise it will be absolute.
\end{funcdesc}

\begin{funcdesc}{NewAliasMinimal}{}
Create a minimal alias pointing to this file.
\end{funcdesc}

\begin{funcdesc}{GetCreatorType}{}
Return the 4-char creator and type of the file.
\end{funcdesc}

\begin{funcdesc}{SetCreatorType}{creator\, type}
Set the 4-char creator and type of the file.
\end{funcdesc}

\begin{funcdesc}{GetFInfo}{}
Return a FInfo object describing the finder info for the file.
\end{funcdesc}

\begin{funcdesc}{SetFInfo}{finfo}
Set the finder info for the file to the values specified in the
\var{finfo} object.
\end{funcdesc}

\subsection{alias objects}

\renewcommand{\indexsubitem}{(alias object attribute)}
\begin{datadesc}{data}
The raw data for the Alias record, suitable for storing in a resource
or transmitting to other programs.
\end{datadesc}

\renewcommand{\indexsubitem}{(alias object method)}
\begin{funcdesc}{Resolve}{\optional{file}}
Resolve the alias. If the alias was created as a relative alias you
should pass the file relative to which it is. Return the FSSpec for
the file pointed to and a flag indicating whether the alias object
itself was modified during the search process. 
\end{funcdesc}

\begin{funcdesc}{GetInfo}{num}
An interface to the C routine \code{GetAliasInfo()}.
\end{funcdesc}

\begin{funcdesc}{Update}{file\, \optional{file2}}
Update the alias to point to the \var{file} given. If \var{file2} is
present a relative alias will be created.
\end{funcdesc}

Note that it is currently not possible to directly manipulate a resource
as an alias object. Hence, after calling \var{Update} or after
\var{Resolve} indicates that the alias has changed the Python program
is responsible for getting the \var{data} from the alias object and
modifying the resource.


\subsection{FInfo objects}

See Inside Mac for a complete description of what the various fields
mean.

\renewcommand{\indexsubitem}{(FInfo object attribute)}
\begin{datadesc}{Creator}
The 4-char creator code of the file.
\end{datadesc}

\begin{datadesc}{Type}
The 4-char type code of the file.
\end{datadesc}

\begin{datadesc}{Flags}
The finder flags for the file as 16-bit integer.
\end{datadesc}

\begin{datadesc}{Location}
A Point giving the position of the file's icon in its folder.
\end{datadesc}

\begin{datadesc}{Fldr}
The folder the file is in (as an integer).
\end{datadesc}
