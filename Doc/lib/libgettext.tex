\section{\module{gettext} ---
         Multilingual internationalization services}

\declaremodule{standard}{gettext}
\modulesynopsis{Multilingual internationalization services.}
\moduleauthor{Barry A. Warsaw}{barry@zope.com}
\sectionauthor{Barry A. Warsaw}{barry@zope.com}


The \module{gettext} module provides internationalization (I18N) and
localization (L10N) services for your Python modules and applications.
It supports both the GNU \code{gettext} message catalog API and a
higher level, class-based API that may be more appropriate for Python
files.  The interface described below allows you to write your
module and application messages in one natural language, and provide a
catalog of translated messages for running under different natural
languages.

Some hints on localizing your Python modules and applications are also
given.

\subsection{GNU \program{gettext} API}

The \module{gettext} module defines the following API, which is very
similar to the GNU \program{gettext} API.  If you use this API you
will affect the translation of your entire application globally.  Often
this is what you want if your application is monolingual, with the choice
of language dependent on the locale of your user.  If you are
localizing a Python module, or if your application needs to switch
languages on the fly, you probably want to use the class-based API
instead.

\begin{funcdesc}{bindtextdomain}{domain\optional{, localedir}}
Bind the \var{domain} to the locale directory
\var{localedir}.  More concretely, \module{gettext} will look for
binary \file{.mo} files for the given domain using the path (on \UNIX):
\file{\var{localedir}/\var{language}/LC_MESSAGES/\var{domain}.mo},
where \var{languages} is searched for in the environment variables
\envvar{LANGUAGE}, \envvar{LC_ALL}, \envvar{LC_MESSAGES}, and
\envvar{LANG} respectively.

If \var{localedir} is omitted or \code{None}, then the current binding
for \var{domain} is returned.\footnote{
        The default locale directory is system dependent; for example,
        on RedHat Linux it is \file{/usr/share/locale}, but on Solaris
        it is \file{/usr/lib/locale}.  The \module{gettext} module
        does not try to support these system dependent defaults;
        instead its default is \file{\code{sys.prefix}/share/locale}.
        For this reason, it is always best to call
        \function{bindtextdomain()} with an explicit absolute path at
        the start of your application.}
\end{funcdesc}

\begin{funcdesc}{bind_textdomain_codeset}{domain\optional{, codeset}}
Bind the \var{domain} to \var{codeset}, changing the encoding of
strings returned by the \function{gettext()} family of functions.
If \var{codeset} is omitted, then the current binding is returned.

\versionadded{2.4}
\end{funcdesc}

\begin{funcdesc}{textdomain}{\optional{domain}}
Change or query the current global domain.  If \var{domain} is
\code{None}, then the current global domain is returned, otherwise the
global domain is set to \var{domain}, which is returned.
\end{funcdesc}

\begin{funcdesc}{gettext}{message}
Return the localized translation of \var{message}, based on the
current global domain, language, and locale directory.  This function
is usually aliased as \function{_} in the local namespace (see
examples below).
\end{funcdesc}

\begin{funcdesc}{lgettext}{message}
Equivalent to \function{gettext()}, but the translation is returned
in the preferred system encoding, if no other encoding was explicitly
set with \function{bind_textdomain_codeset()}.

\versionadded{2.4}
\end{funcdesc}

\begin{funcdesc}{dgettext}{domain, message}
Like \function{gettext()}, but look the message up in the specified
\var{domain}.
\end{funcdesc}

\begin{funcdesc}{ldgettext}{domain, message}
Equivalent to \function{dgettext()}, but the translation is returned
in the preferred system encoding, if no other encoding was explicitly
set with \function{bind_textdomain_codeset()}.

\versionadded{2.4}
\end{funcdesc}

\begin{funcdesc}{ngettext}{singular, plural, n}

Like \function{gettext()}, but consider plural forms. If a translation
is found, apply the plural formula to \var{n}, and return the
resulting message (some languages have more than two plural forms).
If no translation is found, return \var{singular} if \var{n} is 1;
return \var{plural} otherwise.

The Plural formula is taken from the catalog header. It is a C or
Python expression that has a free variable \var{n}; the expression evaluates
to the index of the plural in the catalog. See the GNU gettext
documentation for the precise syntax to be used in \file{.po} files and the
formulas for a variety of languages.

\versionadded{2.3}

\end{funcdesc}

\begin{funcdesc}{lngettext}{singular, plural, n}
Equivalent to \function{ngettext()}, but the translation is returned
in the preferred system encoding, if no other encoding was explicitly
set with \function{bind_textdomain_codeset()}.

\versionadded{2.4}
\end{funcdesc}

\begin{funcdesc}{dngettext}{domain, singular, plural, n}
Like \function{ngettext()}, but look the message up in the specified
\var{domain}.

\versionadded{2.3}
\end{funcdesc}

\begin{funcdesc}{ldngettext}{domain, singular, plural, n}
Equivalent to \function{dngettext()}, but the translation is returned
in the preferred system encoding, if no other encoding was explicitly
set with \function{bind_textdomain_codeset()}.

\versionadded{2.4}
\end{funcdesc}



Note that GNU \program{gettext} also defines a \function{dcgettext()}
method, but this was deemed not useful and so it is currently
unimplemented.

Here's an example of typical usage for this API:

\begin{verbatim}
import gettext
gettext.bindtextdomain('myapplication', '/path/to/my/language/directory')
gettext.textdomain('myapplication')
_ = gettext.gettext
# ...
print _('This is a translatable string.')
\end{verbatim}

\subsection{Class-based API}

The class-based API of the \module{gettext} module gives you more
flexibility and greater convenience than the GNU \program{gettext}
API.  It is the recommended way of localizing your Python applications and
modules.  \module{gettext} defines a ``translations'' class which
implements the parsing of GNU \file{.mo} format files, and has methods
for returning either standard 8-bit strings or Unicode strings.
Instances of this ``translations'' class can also install themselves 
in the built-in namespace as the function \function{_()}.

\begin{funcdesc}{find}{domain\optional{, localedir\optional{, 
                        languages\optional{, all}}}}
This function implements the standard \file{.mo} file search
algorithm.  It takes a \var{domain}, identical to what
\function{textdomain()} takes.  Optional \var{localedir} is as in
\function{bindtextdomain()}  Optional \var{languages} is a list of
strings, where each string is a language code.

If \var{localedir} is not given, then the default system locale
directory is used.\footnote{See the footnote for
\function{bindtextdomain()} above.}  If \var{languages} is not given,
then the following environment variables are searched: \envvar{LANGUAGE},
\envvar{LC_ALL}, \envvar{LC_MESSAGES}, and \envvar{LANG}.  The first one
returning a non-empty value is used for the \var{languages} variable.
The environment variables should contain a colon separated list of
languages, which will be split on the colon to produce the expected
list of language code strings.

\function{find()} then expands and normalizes the languages, and then
iterates through them, searching for an existing file built of these
components:

\file{\var{localedir}/\var{language}/LC_MESSAGES/\var{domain}.mo}

The first such file name that exists is returned by \function{find()}.
If no such file is found, then \code{None} is returned. If \var{all}
is given, it returns a list of all file names, in the order in which
they appear in the languages list or the environment variables.
\end{funcdesc}

\begin{funcdesc}{translation}{domain\optional{, localedir\optional{,
                              languages\optional{, class_\optional{,
			      fallback\optional{, codeset}}}}}}
Return a \class{Translations} instance based on the \var{domain},
\var{localedir}, and \var{languages}, which are first passed to
\function{find()} to get a list of the
associated \file{.mo} file paths.  Instances with
identical \file{.mo} file names are cached.  The actual class instantiated
is either \var{class_} if provided, otherwise
\class{GNUTranslations}.  The class's constructor must take a single
file object argument. If provided, \var{codeset} will change the
charset used to encode translated strings.

If multiple files are found, later files are used as fallbacks for
earlier ones. To allow setting the fallback, \function{copy.copy}
is used to clone each translation object from the cache; the actual
instance data is still shared with the cache.

If no \file{.mo} file is found, this function raises
\exception{IOError} if \var{fallback} is false (which is the default),
and returns a \class{NullTranslations} instance if \var{fallback} is
true.

\versionchanged[Added the \var{codeset} parameter]{2.4}
\end{funcdesc}

\begin{funcdesc}{install}{domain\optional{, localedir\optional{, unicode
			  \optional{, codeset\optional{, names}}}}}
This installs the function \function{_} in Python's builtin namespace,
based on \var{domain}, \var{localedir}, and \var{codeset} which are
passed to the function \function{translation()}.  The \var{unicode}
flag is passed to the resulting translation object's \method{install}
method.

For the \var{names} parameter, please see the description of the
translation object's \method{install} method.

As seen below, you usually mark the strings in your application that are
candidates for translation, by wrapping them in a call to the
\function{_()} function, like this:

\begin{verbatim}
print _('This string will be translated.')
\end{verbatim}

For convenience, you want the \function{_()} function to be installed in
Python's builtin namespace, so it is easily accessible in all modules
of your application.  

\versionchanged[Added the \var{codeset} parameter]{2.4}
\versionchanged[Added the \var{names} parameter]{2.5}
\end{funcdesc}

\subsubsection{The \class{NullTranslations} class}
Translation classes are what actually implement the translation of
original source file message strings to translated message strings.
The base class used by all translation classes is
\class{NullTranslations}; this provides the basic interface you can use
to write your own specialized translation classes.  Here are the
methods of \class{NullTranslations}:

\begin{methoddesc}[NullTranslations]{__init__}{\optional{fp}}
Takes an optional file object \var{fp}, which is ignored by the base
class.  Initializes ``protected'' instance variables \var{_info} and
\var{_charset} which are set by derived classes, as well as \var{_fallback},
which is set through \method{add_fallback}.  It then calls
\code{self._parse(fp)} if \var{fp} is not \code{None}.
\end{methoddesc}

\begin{methoddesc}[NullTranslations]{_parse}{fp}
No-op'd in the base class, this method takes file object \var{fp}, and
reads the data from the file, initializing its message catalog.  If
you have an unsupported message catalog file format, you should
override this method to parse your format.
\end{methoddesc}

\begin{methoddesc}[NullTranslations]{add_fallback}{fallback}
Add \var{fallback} as the fallback object for the current translation
object. A translation object should consult the fallback if it cannot
provide a translation for a given message.
\end{methoddesc}

\begin{methoddesc}[NullTranslations]{gettext}{message}
If a fallback has been set, forward \method{gettext()} to the fallback.
Otherwise, return the translated message.  Overridden in derived classes.
\end{methoddesc}

\begin{methoddesc}[NullTranslations]{lgettext}{message}
If a fallback has been set, forward \method{lgettext()} to the fallback.
Otherwise, return the translated message.  Overridden in derived classes.

\versionadded{2.4}
\end{methoddesc}

\begin{methoddesc}[NullTranslations]{ugettext}{message}
If a fallback has been set, forward \method{ugettext()} to the fallback.
Otherwise, return the translated message as a Unicode string.
Overridden in derived classes.
\end{methoddesc}

\begin{methoddesc}[NullTranslations]{ngettext}{singular, plural, n}
If a fallback has been set, forward \method{ngettext()} to the fallback.
Otherwise, return the translated message.  Overridden in derived classes.

\versionadded{2.3}
\end{methoddesc}

\begin{methoddesc}[NullTranslations]{lngettext}{singular, plural, n}
If a fallback has been set, forward \method{ngettext()} to the fallback.
Otherwise, return the translated message.  Overridden in derived classes.

\versionadded{2.4}
\end{methoddesc}

\begin{methoddesc}[NullTranslations]{ungettext}{singular, plural, n}
If a fallback has been set, forward \method{ungettext()} to the fallback.
Otherwise, return the translated message as a Unicode string.
Overridden in derived classes.

\versionadded{2.3}
\end{methoddesc}

\begin{methoddesc}[NullTranslations]{info}{}
Return the ``protected'' \member{_info} variable.
\end{methoddesc}

\begin{methoddesc}[NullTranslations]{charset}{}
Return the ``protected'' \member{_charset} variable.
\end{methoddesc}

\begin{methoddesc}[NullTranslations]{output_charset}{}
Return the ``protected'' \member{_output_charset} variable, which
defines the encoding used to return translated messages.

\versionadded{2.4}
\end{methoddesc}

\begin{methoddesc}[NullTranslations]{set_output_charset}{charset}
Change the ``protected'' \member{_output_charset} variable, which
defines the encoding used to return translated messages.

\versionadded{2.4}
\end{methoddesc}

\begin{methoddesc}[NullTranslations]{install}{\optional{unicode
                                              \optional{, names}}}
If the \var{unicode} flag is false, this method installs
\method{self.gettext()} into the built-in namespace, binding it to
\samp{_}.  If \var{unicode} is true, it binds \method{self.ugettext()}
instead.  By default, \var{unicode} is false.

If the \var{names} parameter is given, it must be a sequence containing
the names of functions you want to install in the builtin namespace in
addition to \function{_()}. Supported names are \code{'gettext'} (bound
to \method{self.gettext()} or \method{self.ugettext()} according to the
\var{unicode} flag), \code{'ngettext'} (bound to \method{self.ngettext()}
or \method{self.ungettext()} according to the \var{unicode} flag),
\code{'lgettext'} and \code{'lngettext'}.

Note that this is only one way, albeit the most convenient way, to
make the \function{_} function available to your application.  Because it
affects the entire application globally, and specifically the built-in
namespace, localized modules should never install \function{_}.
Instead, they should use this code to make \function{_} available to
their module:

\begin{verbatim}
import gettext
t = gettext.translation('mymodule', ...)
_ = t.gettext
\end{verbatim}

This puts \function{_} only in the module's global namespace and so
only affects calls within this module.

\versionchanged[Added the \var{names} parameter]{2.5}
\end{methoddesc}

\subsubsection{The \class{GNUTranslations} class}

The \module{gettext} module provides one additional class derived from
\class{NullTranslations}: \class{GNUTranslations}.  This class
overrides \method{_parse()} to enable reading GNU \program{gettext}
format \file{.mo} files in both big-endian and little-endian format.
It also coerces both message ids and message strings to Unicode.

\class{GNUTranslations} parses optional meta-data out of the
translation catalog.  It is convention with GNU \program{gettext} to
include meta-data as the translation for the empty string.  This
meta-data is in \rfc{822}-style \code{key: value} pairs, and should
contain the \code{Project-Id-Version} key.  If the key
\code{Content-Type} is found, then the \code{charset} property is used
to initialize the ``protected'' \member{_charset} instance variable,
defaulting to \code{None} if not found.  If the charset encoding is
specified, then all message ids and message strings read from the
catalog are converted to Unicode using this encoding.  The
\method{ugettext()} method always returns a Unicode, while the
\method{gettext()} returns an encoded 8-bit string.  For the message
id arguments of both methods, either Unicode strings or 8-bit strings
containing only US-ASCII characters are acceptable.  Note that the
Unicode version of the methods (i.e. \method{ugettext()} and
\method{ungettext()}) are the recommended interface to use for
internationalized Python programs.

The entire set of key/value pairs are placed into a dictionary and set
as the ``protected'' \member{_info} instance variable.

If the \file{.mo} file's magic number is invalid, or if other problems
occur while reading the file, instantiating a \class{GNUTranslations} class
can raise \exception{IOError}.

The following methods are overridden from the base class implementation:

\begin{methoddesc}[GNUTranslations]{gettext}{message}
Look up the \var{message} id in the catalog and return the
corresponding message string, as an 8-bit string encoded with the
catalog's charset encoding, if known.  If there is no entry in the
catalog for the \var{message} id, and a fallback has been set, the
look up is forwarded to the fallback's \method{gettext()} method.
Otherwise, the \var{message} id is returned.
\end{methoddesc}

\begin{methoddesc}[GNUTranslations]{lgettext}{message}
Equivalent to \method{gettext()}, but the translation is returned
in the preferred system encoding, if no other encoding was explicitly
set with \method{set_output_charset()}.

\versionadded{2.4}
\end{methoddesc}

\begin{methoddesc}[GNUTranslations]{ugettext}{message}
Look up the \var{message} id in the catalog and return the
corresponding message string, as a Unicode string.  If there is no
entry in the catalog for the \var{message} id, and a fallback has been
set, the look up is forwarded to the fallback's \method{ugettext()}
method.  Otherwise, the \var{message} id is returned.
\end{methoddesc}

\begin{methoddesc}[GNUTranslations]{ngettext}{singular, plural, n}
Do a plural-forms lookup of a message id.  \var{singular} is used as
the message id for purposes of lookup in the catalog, while \var{n} is
used to determine which plural form to use.  The returned message
string is an 8-bit string encoded with the catalog's charset encoding,
if known.

If the message id is not found in the catalog, and a fallback is
specified, the request is forwarded to the fallback's
\method{ngettext()} method.  Otherwise, when \var{n} is 1 \var{singular} is
returned, and \var{plural} is returned in all other cases.

\versionadded{2.3}
\end{methoddesc}

\begin{methoddesc}[GNUTranslations]{lngettext}{singular, plural, n}
Equivalent to \method{gettext()}, but the translation is returned
in the preferred system encoding, if no other encoding was explicitly
set with \method{set_output_charset()}.

\versionadded{2.4}
\end{methoddesc}

\begin{methoddesc}[GNUTranslations]{ungettext}{singular, plural, n}
Do a plural-forms lookup of a message id.  \var{singular} is used as
the message id for purposes of lookup in the catalog, while \var{n} is
used to determine which plural form to use.  The returned message
string is a Unicode string.

If the message id is not found in the catalog, and a fallback is
specified, the request is forwarded to the fallback's
\method{ungettext()} method.  Otherwise, when \var{n} is 1 \var{singular} is
returned, and \var{plural} is returned in all other cases.

Here is an example:

\begin{verbatim}
n = len(os.listdir('.'))
cat = GNUTranslations(somefile)
message = cat.ungettext(
    'There is %(num)d file in this directory',
    'There are %(num)d files in this directory',
    n) % {'num': n}
\end{verbatim}

\versionadded{2.3}
\end{methoddesc}

\subsubsection{Solaris message catalog support}

The Solaris operating system defines its own binary
\file{.mo} file format, but since no documentation can be found on
this format, it is not supported at this time.

\subsubsection{The Catalog constructor}

GNOME\index{GNOME} uses a version of the \module{gettext} module by
James Henstridge, but this version has a slightly different API.  Its
documented usage was:

\begin{verbatim}
import gettext
cat = gettext.Catalog(domain, localedir)
_ = cat.gettext
print _('hello world')
\end{verbatim}

For compatibility with this older module, the function
\function{Catalog()} is an alias for the \function{translation()}
function described above.

One difference between this module and Henstridge's: his catalog
objects supported access through a mapping API, but this appears to be
unused and so is not currently supported.

\subsection{Internationalizing your programs and modules}
Internationalization (I18N) refers to the operation by which a program
is made aware of multiple languages.  Localization (L10N) refers to
the adaptation of your program, once internationalized, to the local
language and cultural habits.  In order to provide multilingual
messages for your Python programs, you need to take the following
steps:

\begin{enumerate}
    \item prepare your program or module by specially marking
          translatable strings
    \item run a suite of tools over your marked files to generate raw
          messages catalogs
    \item create language specific translations of the message catalogs
    \item use the \module{gettext} module so that message strings are
          properly translated
\end{enumerate}

In order to prepare your code for I18N, you need to look at all the
strings in your files.  Any string that needs to be translated
should be marked by wrapping it in \code{_('...')} --- that is, a call
to the function \function{_()}.  For example:

\begin{verbatim}
filename = 'mylog.txt'
message = _('writing a log message')
fp = open(filename, 'w')
fp.write(message)
fp.close()
\end{verbatim}

In this example, the string \code{'writing a log message'} is marked as
a candidate for translation, while the strings \code{'mylog.txt'} and
\code{'w'} are not.

The Python distribution comes with two tools which help you generate
the message catalogs once you've prepared your source code.  These may
or may not be available from a binary distribution, but they can be
found in a source distribution, in the \file{Tools/i18n} directory.

The \program{pygettext}\footnote{Fran\c cois Pinard has
written a program called
\program{xpot} which does a similar job.  It is available as part of
his \program{po-utils} package at
\url{http://po-utils.progiciels-bpi.ca/}.} program
scans all your Python source code looking for the strings you
previously marked as translatable.  It is similar to the GNU
\program{gettext} program except that it understands all the
intricacies of Python source code, but knows nothing about C or \Cpp
source code.  You don't need GNU \code{gettext} unless you're also
going to be translating C code (such as C extension modules).

\program{pygettext} generates textual Uniforum-style human readable
message catalog \file{.pot} files, essentially structured human
readable files which contain every marked string in the source code,
along with a placeholder for the translation strings.
\program{pygettext} is a command line script that supports a similar
command line interface as \program{xgettext}; for details on its use,
run:

\begin{verbatim}
pygettext.py --help
\end{verbatim}

Copies of these \file{.pot} files are then handed over to the
individual human translators who write language-specific versions for
every supported natural language.  They send you back the filled in
language-specific versions as a \file{.po} file.  Using the
\program{msgfmt.py}\footnote{\program{msgfmt.py} is binary
compatible with GNU \program{msgfmt} except that it provides a
simpler, all-Python implementation.  With this and
\program{pygettext.py}, you generally won't need to install the GNU
\program{gettext} package to internationalize your Python
applications.} program (in the \file{Tools/i18n} directory), you take the
\file{.po} files from your translators and generate the
machine-readable \file{.mo} binary catalog files.  The \file{.mo}
files are what the \module{gettext} module uses for the actual
translation processing during run-time.

How you use the \module{gettext} module in your code depends on
whether you are internationalizing a single module or your entire application.
The next two sections will discuss each case.

\subsubsection{Localizing your module}

If you are localizing your module, you must take care not to make
global changes, e.g. to the built-in namespace.  You should not use
the GNU \code{gettext} API but instead the class-based API.  

Let's say your module is called ``spam'' and the module's various
natural language translation \file{.mo} files reside in
\file{/usr/share/locale} in GNU \program{gettext} format.  Here's what
you would put at the top of your module:

\begin{verbatim}
import gettext
t = gettext.translation('spam', '/usr/share/locale')
_ = t.lgettext
\end{verbatim}

If your translators were providing you with Unicode strings in their
\file{.po} files, you'd instead do:

\begin{verbatim}
import gettext
t = gettext.translation('spam', '/usr/share/locale')
_ = t.ugettext
\end{verbatim}

\subsubsection{Localizing your application}

If you are localizing your application, you can install the \function{_()}
function globally into the built-in namespace, usually in the main driver file
of your application.  This will let all your application-specific
files just use \code{_('...')} without having to explicitly install it in
each file.

In the simple case then, you need only add the following bit of code
to the main driver file of your application:

\begin{verbatim}
import gettext
gettext.install('myapplication')
\end{verbatim}

If you need to set the locale directory or the \var{unicode} flag,
you can pass these into the \function{install()} function:

\begin{verbatim}
import gettext
gettext.install('myapplication', '/usr/share/locale', unicode=1)
\end{verbatim}

\subsubsection{Changing languages on the fly}

If your program needs to support many languages at the same time, you
may want to create multiple translation instances and then switch
between them explicitly, like so:

\begin{verbatim}
import gettext

lang1 = gettext.translation('myapplication', languages=['en'])
lang2 = gettext.translation('myapplication', languages=['fr'])
lang3 = gettext.translation('myapplication', languages=['de'])

# start by using language1
lang1.install()

# ... time goes by, user selects language 2
lang2.install()

# ... more time goes by, user selects language 3
lang3.install()
\end{verbatim}

\subsubsection{Deferred translations}

In most coding situations, strings are translated where they are coded.
Occasionally however, you need to mark strings for translation, but
defer actual translation until later.  A classic example is:

\begin{verbatim}
animals = ['mollusk',
           'albatross',
	   'rat',
	   'penguin',
	   'python',
	   ]
# ...
for a in animals:
    print a
\end{verbatim}

Here, you want to mark the strings in the \code{animals} list as being
translatable, but you don't actually want to translate them until they
are printed.

Here is one way you can handle this situation:

\begin{verbatim}
def _(message): return message

animals = [_('mollusk'),
           _('albatross'),
	   _('rat'),
	   _('penguin'),
	   _('python'),
	   ]

del _

# ...
for a in animals:
    print _(a)
\end{verbatim}

This works because the dummy definition of \function{_()} simply returns
the string unchanged.  And this dummy definition will temporarily
override any definition of \function{_()} in the built-in namespace
(until the \keyword{del} command).
Take care, though if you have a previous definition of \function{_} in
the local namespace.

Note that the second use of \function{_()} will not identify ``a'' as
being translatable to the \program{pygettext} program, since it is not
a string.

Another way to handle this is with the following example:

\begin{verbatim}
def N_(message): return message

animals = [N_('mollusk'),
           N_('albatross'),
	   N_('rat'),
	   N_('penguin'),
	   N_('python'),
	   ]

# ...
for a in animals:
    print _(a)
\end{verbatim}

In this case, you are marking translatable strings with the function
\function{N_()},\footnote{The choice of \function{N_()} here is totally
arbitrary; it could have just as easily been
\function{MarkThisStringForTranslation()}.
} which won't conflict with any definition of
\function{_()}.  However, you will need to teach your message extraction
program to look for translatable strings marked with \function{N_()}.
\program{pygettext} and \program{xpot} both support this through the
use of command line switches.

\subsubsection{\function{gettext()} vs. \function{lgettext()}}
In Python 2.4 the \function{lgettext()} family of functions were
introduced. The intention of these functions is to provide an
alternative which is more compliant with the current
implementation of GNU gettext. Unlike \function{gettext()}, which
returns strings encoded with the same codeset used in the
translation file, \function{lgettext()} will return strings
encoded with the preferred system encoding, as returned by
\function{locale.getpreferredencoding()}. Also notice that
Python 2.4 introduces new functions to explicitly choose
the codeset used in translated strings. If a codeset is explicitly
set, even \function{lgettext()} will return translated strings in
the requested codeset, as would be expected in the GNU gettext
implementation.

\subsection{Acknowledgements}

The following people contributed code, feedback, design suggestions,
previous implementations, and valuable experience to the creation of
this module:

\begin{itemize}
    \item Peter Funk
    \item James Henstridge
    \item Juan David Ib\'a\~nez Palomar
    \item Marc-Andr\'e Lemburg
    \item Martin von L\"owis
    \item Fran\c cois Pinard
    \item Barry Warsaw
    \item Gustavo Niemeyer
\end{itemize}
