\section{\module{doctest} ---
         Test interactive Python examples}

\declaremodule{standard}{doctest}
\moduleauthor{Tim Peters}{tim@python.org}
\sectionauthor{Tim Peters}{tim@python.org}
\sectionauthor{Moshe Zadka}{moshez@debian.org}
\sectionauthor{Edward Loper}{edloper@users.sourceforge.net}

\modulesynopsis{A framework for verifying interactive Python examples.}

The \refmodule{doctest} module searches for pieces of text that look like
interactive Python sessions, and then executes those sessions to
verify that they work exactly as shown.  There are several common ways to
use doctest:

\begin{itemize}
\item To check that a module's docstrings are up-to-date by verifying
      that all interactive examples still work as documented.
\item To perform regression testing by verifying that interactive
      examples from a test file or a test object work as expected.
\item To write tutorial documentation for a package, liberally
      illustrated with input-output examples.  Depending on whether
      the examples or the expository text are emphasized, this has
      the flavor of "literate testing" or "executable documentation".
\end{itemize}

Here's a complete but small example module:

\begin{verbatim}
"""
This is the "example" module.

The example module supplies one function, factorial().  For example,

>>> factorial(5)
120
"""

def factorial(n):
    """Return the factorial of n, an exact integer >= 0.

    If the result is small enough to fit in an int, return an int.
    Else return a long.

    >>> [factorial(n) for n in range(6)]
    [1, 1, 2, 6, 24, 120]
    >>> [factorial(long(n)) for n in range(6)]
    [1, 1, 2, 6, 24, 120]
    >>> factorial(30)
    265252859812191058636308480000000L
    >>> factorial(30L)
    265252859812191058636308480000000L
    >>> factorial(-1)
    Traceback (most recent call last):
        ...
    ValueError: n must be >= 0

    Factorials of floats are OK, but the float must be an exact integer:
    >>> factorial(30.1)
    Traceback (most recent call last):
        ...
    ValueError: n must be exact integer
    >>> factorial(30.0)
    265252859812191058636308480000000L

    It must also not be ridiculously large:
    >>> factorial(1e100)
    Traceback (most recent call last):
        ...
    OverflowError: n too large
    """

\end{verbatim}
% allow LaTeX to break here.
\begin{verbatim}

    import math
    if not n >= 0:
        raise ValueError("n must be >= 0")
    if math.floor(n) != n:
        raise ValueError("n must be exact integer")
    if n+1 == n:  # catch a value like 1e300
        raise OverflowError("n too large")
    result = 1
    factor = 2
    while factor <= n:
        result *= factor
        factor += 1
    return result

def _test():
    import doctest
    doctest.testmod()

if __name__ == "__main__":
    _test()
\end{verbatim}

If you run \file{example.py} directly from the command line,
\refmodule{doctest} works its magic:

\begin{verbatim}
$ python example.py
$
\end{verbatim}

There's no output!  That's normal, and it means all the examples
worked.  Pass \programopt{-v} to the script, and \refmodule{doctest}
prints a detailed log of what it's trying, and prints a summary at the
end:

\begin{verbatim}
$ python example.py -v
Trying:
    factorial(5)
Expecting:
    120
ok
Trying:
    [factorial(n) for n in range(6)]
Expecting:
    [1, 1, 2, 6, 24, 120]
ok
Trying:
    [factorial(long(n)) for n in range(6)]
Expecting:
    [1, 1, 2, 6, 24, 120]
ok
\end{verbatim}

And so on, eventually ending with:

\begin{verbatim}
Trying:
    factorial(1e100)
Expecting:
    Traceback (most recent call last):
        ...
    OverflowError: n too large
ok
1 items had no tests:
    __main__._test
2 items passed all tests:
   1 tests in __main__
   8 tests in __main__.factorial
9 tests in 3 items.
9 passed and 0 failed.
Test passed.
$
\end{verbatim}

That's all you need to know to start making productive use of
\refmodule{doctest}!  Jump in.  The following sections provide full
details.  Note that there are many examples of doctests in
the standard Python test suite and libraries.  Especially useful examples
can be found in the standard test file \file{Lib/test/test_doctest.py}.

\subsection{Simple Usage: Checking Examples in
            Docstrings\label{doctest-simple-testmod}}

The simplest way to start using doctest (but not necessarily the way
you'll continue to do it) is to end each module \module{M} with:

\begin{verbatim}
def _test():
    import doctest
    doctest.testmod()

if __name__ == "__main__":
    _test()
\end{verbatim}

\refmodule{doctest} then examines docstrings in module \module{M}.

Running the module as a script causes the examples in the docstrings
to get executed and verified:

\begin{verbatim}
python M.py
\end{verbatim}

This won't display anything unless an example fails, in which case the
failing example(s) and the cause(s) of the failure(s) are printed to stdout,
and the final line of output is
\samp{***Test Failed*** \var{N} failures.}, where \var{N} is the
number of examples that failed.

Run it with the \programopt{-v} switch instead:

\begin{verbatim}
python M.py -v
\end{verbatim}

and a detailed report of all examples tried is printed to standard
output, along with assorted summaries at the end.

You can force verbose mode by passing \code{verbose=True} to
\function{testmod()}, or
prohibit it by passing \code{verbose=False}.  In either of those cases,
\code{sys.argv} is not examined by \function{testmod()} (so passing
\programopt{-v} or not has no effect).

For more information on \function{testmod()}, see
section~\ref{doctest-basic-api}.

\subsection{Simple Usage: Checking Examples in a Text
            File\label{doctest-simple-testfile}}

Another simple application of doctest is testing interactive examples
in a text file.  This can be done with the \function{testfile()}
function:

\begin{verbatim}
import doctest
doctest.testfile("example.txt")
\end{verbatim}

That short script executes and verifies any interactive Python
examples contained in the file \file{example.txt}.  The file content
is treated as if it were a single giant docstring; the file doesn't
need to contain a Python program!   For example, perhaps \file{example.txt}
contains this:

\begin{verbatim}
The ``example`` module
======================

Using ``factorial``
-------------------

This is an example text file in reStructuredText format.  First import
``factorial`` from the ``example`` module:

    >>> from example import factorial

Now use it:

    >>> factorial(6)
    120
\end{verbatim}

Running \code{doctest.testfile("example.txt")} then finds the error
in this documentation:

\begin{verbatim}
File "./example.txt", line 14, in example.txt
Failed example:
    factorial(6)
Expected:
    120
Got:
    720
\end{verbatim}

As with \function{testmod()}, \function{testfile()} won't display anything
unless an example fails.  If an example does fail, then the failing
example(s) and the cause(s) of the failure(s) are printed to stdout, using
the same format as \function{testmod()}.

By default, \function{testfile()} looks for files in the calling
module's directory.  See section~\ref{doctest-basic-api} for a
description of the optional arguments that can be used to tell it to
look for files in other locations.

Like \function{testmod()}, \function{testfile()}'s verbosity can be
set with the \programopt{-v} command-line switch or with the optional
keyword argument \var{verbose}.

For more information on \function{testfile()}, see
section~\ref{doctest-basic-api}.

\subsection{How It Works\label{doctest-how-it-works}}

This section examines in detail how doctest works: which docstrings it
looks at, how it finds interactive examples, what execution context it
uses, how it handles exceptions, and how option flags can be used to
control its behavior.  This is the information that you need to know
to write doctest examples; for information about actually running
doctest on these examples, see the following sections.

\subsubsection{Which Docstrings Are Examined?\label{doctest-which-docstrings}}

The module docstring, and all function, class and method docstrings are
searched.  Objects imported into the module are not searched.

In addition, if \code{M.__test__} exists and "is true", it must be a
dict, and each entry maps a (string) name to a function object, class
object, or string.  Function and class object docstrings found from
\code{M.__test__} are searched, and strings are treated as if they
were docstrings.  In output, a key \code{K} in \code{M.__test__} appears
with name

\begin{verbatim}
<name of M>.__test__.K
\end{verbatim}

Any classes found are recursively searched similarly, to test docstrings in
their contained methods and nested classes.

\versionchanged[A "private name" concept is deprecated and no longer
                documented]{2.4}

\subsubsection{How are Docstring Examples
               Recognized?\label{doctest-finding-examples}}

In most cases a copy-and-paste of an interactive console session works
fine, but doctest isn't trying to do an exact emulation of any specific
Python shell.  All hard tab characters are expanded to spaces, using
8-column tab stops.  If you don't believe tabs should mean that, too
bad:  don't use hard tabs, or write your own \class{DocTestParser}
class.

\versionchanged[Expanding tabs to spaces is new; previous versions
                tried to preserve hard tabs, with confusing results]{2.4}

\begin{verbatim}
>>> # comments are ignored
>>> x = 12
>>> x
12
>>> if x == 13:
...     print "yes"
... else:
...     print "no"
...     print "NO"
...     print "NO!!!"
...
no
NO
NO!!!
>>>
\end{verbatim}

Any expected output must immediately follow the final
\code{'>\code{>}>~'} or \code{'...~'} line containing the code, and
the expected output (if any) extends to the next \code{'>\code{>}>~'}
or all-whitespace line.

The fine print:

\begin{itemize}

\item Expected output cannot contain an all-whitespace line, since such a
  line is taken to signal the end of expected output.  If expected
  output does contain a blank line, put \code{<BLANKLINE>} in your
  doctest example each place a blank line is expected.
  \versionchanged[\code{<BLANKLINE>} was added; there was no way to
                  use expected output containing empty lines in
                  previous versions]{2.4}

\item Output to stdout is captured, but not output to stderr (exception
  tracebacks are captured via a different means).

\item If you continue a line via backslashing in an interactive session,
  or for any other reason use a backslash, you should use a raw
  docstring, which will preserve your backslashes exactly as you type
  them:

\begin{verbatim}
>>> def f(x):
...     r'''Backslashes in a raw docstring: m\n'''
>>> print f.__doc__
Backslashes in a raw docstring: m\n
\end{verbatim}

  Otherwise, the backslash will be interpreted as part of the string.
  For example, the "{\textbackslash}" above would be interpreted as a
  newline character.  Alternatively, you can double each backslash in the
  doctest version (and not use a raw string):

\begin{verbatim}
>>> def f(x):
...     '''Backslashes in a raw docstring: m\\n'''
>>> print f.__doc__
Backslashes in a raw docstring: m\n
\end{verbatim}

\item The starting column doesn't matter:

\begin{verbatim}
  >>> assert "Easy!"
        >>> import math
            >>> math.floor(1.9)
            1.0
\end{verbatim}

and as many leading whitespace characters are stripped from the
expected output as appeared in the initial \code{'>\code{>}>~'} line
that started the example.
\end{itemize}

\subsubsection{What's the Execution Context?\label{doctest-execution-context}}

By default, each time \refmodule{doctest} finds a docstring to test, it
uses a \emph{shallow copy} of \module{M}'s globals, so that running tests
doesn't change the module's real globals, and so that one test in
\module{M} can't leave behind crumbs that accidentally allow another test
to work.  This means examples can freely use any names defined at top-level
in \module{M}, and names defined earlier in the docstring being run.
Examples cannot see names defined in other docstrings.

You can force use of your own dict as the execution context by passing
\code{globs=your_dict} to \function{testmod()} or
\function{testfile()} instead.

\subsubsection{What About Exceptions?\label{doctest-exceptions}}

No problem, provided that the traceback is the only output produced by
the example:  just paste in the traceback.  Since tracebacks contain
details that are likely to change rapidly (for example, exact file paths
and line numbers), this is one case where doctest works hard to be
flexible in what it accepts.

Simple example:

\begin{verbatim}
>>> [1, 2, 3].remove(42)
Traceback (most recent call last):
  File "<stdin>", line 1, in ?
ValueError: list.remove(x): x not in list
\end{verbatim}

That doctest succeeds if \exception{ValueError} is raised, with the
\samp{list.remove(x): x not in list} detail as shown.

The expected output for an exception must start with a traceback
header, which may be either of the following two lines, indented the
same as the first line of the example:

\begin{verbatim}
Traceback (most recent call last):
Traceback (innermost last):
\end{verbatim}

The traceback header is followed by an optional traceback stack, whose
contents are ignored by doctest.  The traceback stack is typically
omitted, or copied verbatim from an interactive session.

The traceback stack is followed by the most interesting part:  the
line(s) containing the exception type and detail.  This is usually the
last line of a traceback, but can extend across multiple lines if the
exception has a multi-line detail:

\begin{verbatim}
>>> raise ValueError('multi\n    line\ndetail')
Traceback (most recent call last):
  File "<stdin>", line 1, in ?
ValueError: multi
    line
detail
\end{verbatim}

The last three lines (starting with \exception{ValueError}) are
compared against the exception's type and detail, and the rest are
ignored.

Best practice is to omit the traceback stack, unless it adds
significant documentation value to the example.  So the last example
is probably better as:

\begin{verbatim}
>>> raise ValueError('multi\n    line\ndetail')
Traceback (most recent call last):
    ...
ValueError: multi
    line
detail
\end{verbatim}

Note that tracebacks are treated very specially.  In particular, in the
rewritten example, the use of \samp{...} is independent of doctest's
\constant{ELLIPSIS} option.  The ellipsis in that example could be left
out, or could just as well be three (or three hundred) commas or digits,
or an indented transcript of a Monty Python skit.

Some details you should read once, but won't need to remember:

\begin{itemize}

\item Doctest can't guess whether your expected output came from an
  exception traceback or from ordinary printing.  So, e.g., an example
  that expects \samp{ValueError: 42 is prime} will pass whether
  \exception{ValueError} is actually raised or if the example merely
  prints that traceback text.  In practice, ordinary output rarely begins
  with a traceback header line, so this doesn't create real problems.

\item Each line of the traceback stack (if present) must be indented
  further than the first line of the example, \emph{or} start with a
  non-alphanumeric character.  The first line following the traceback
  header indented the same and starting with an alphanumeric is taken
  to be the start of the exception detail.  Of course this does the
  right thing for genuine tracebacks.

\item When the \constant{IGNORE_EXCEPTION_DETAIL} doctest option is
  is specified, everything following the leftmost colon is ignored.

\item The interactive shell omits the traceback header line for some
  \exception{SyntaxError}s.  But doctest uses the traceback header
  line to distinguish exceptions from non-exceptions.  So in the rare
  case where you need to test a \exception{SyntaxError} that omits the
  traceback header, you will need to manually add the traceback header
  line to your test example.
  
\item For some \exception{SyntaxError}s, Python displays the character
  position of the syntax error, using a \code{\^} marker:

\begin{verbatim}
>>> 1 1
  File "<stdin>", line 1
    1 1
      ^
SyntaxError: invalid syntax
\end{verbatim}

  Since the lines showing the position of the error come before the
  exception type and detail, they are not checked by doctest.  For
  example, the following test would pass, even though it puts the
  \code{\^} marker in the wrong location:

\begin{verbatim}
>>> 1 1
  File "<stdin>", line 1
    1 1
    ^
SyntaxError: invalid syntax
\end{verbatim}

\end{itemize}

\versionchanged[The ability to handle a multi-line exception detail,
                and the \constant{IGNORE_EXCEPTION_DETAIL} doctest option,
                were added]{2.4}

\subsubsection{Option Flags and Directives\label{doctest-options}}

A number of option flags control various aspects of doctest's
behavior.  Symbolic names for the flags are supplied as module constants,
which can be or'ed together and passed to various functions.  The names
can also be used in doctest directives (see below).

The first group of options define test semantics, controlling
aspects of how doctest decides whether actual output matches an
example's expected output:

\begin{datadesc}{DONT_ACCEPT_TRUE_FOR_1}
    By default, if an expected output block contains just \code{1},
    an actual output block containing just \code{1} or just
    \code{True} is considered to be a match, and similarly for \code{0}
    versus \code{False}.  When \constant{DONT_ACCEPT_TRUE_FOR_1} is
    specified, neither substitution is allowed.  The default behavior
    caters to that Python changed the return type of many functions
    from integer to boolean; doctests expecting "little integer"
    output still work in these cases.  This option will probably go
    away, but not for several years.
\end{datadesc}

\begin{datadesc}{DONT_ACCEPT_BLANKLINE}
    By default, if an expected output block contains a line
    containing only the string \code{<BLANKLINE>}, then that line
    will match a blank line in the actual output.  Because a
    genuinely blank line delimits the expected output, this is
    the only way to communicate that a blank line is expected.  When
    \constant{DONT_ACCEPT_BLANKLINE} is specified, this substitution
    is not allowed.
\end{datadesc}

\begin{datadesc}{NORMALIZE_WHITESPACE}
    When specified, all sequences of whitespace (blanks and newlines) are
    treated as equal.  Any sequence of whitespace within the expected
    output will match any sequence of whitespace within the actual output.
    By default, whitespace must match exactly.
    \constant{NORMALIZE_WHITESPACE} is especially useful when a line
    of expected output is very long, and you want to wrap it across
    multiple lines in your source.
\end{datadesc}

\begin{datadesc}{ELLIPSIS}
    When specified, an ellipsis marker (\code{...}) in the expected output
    can match any substring in the actual output.  This includes
    substrings that span line boundaries, and empty substrings, so it's
    best to keep usage of this simple.  Complicated uses can lead to the
    same kinds of "oops, it matched too much!" surprises that \regexp{.*}
    is prone to in regular expressions.
\end{datadesc}

\begin{datadesc}{IGNORE_EXCEPTION_DETAIL}
    When specified, an example that expects an exception passes if
    an exception of the expected type is raised, even if the exception
    detail does not match.  For example, an example expecting
    \samp{ValueError: 42} will pass if the actual exception raised is
    \samp{ValueError: 3*14}, but will fail, e.g., if
    \exception{TypeError} is raised.

    Note that a similar effect can be obtained using \constant{ELLIPSIS},
    and \constant{IGNORE_EXCEPTION_DETAIL} may go away when Python releases
    prior to 2.4 become uninteresting.  Until then,
    \constant{IGNORE_EXCEPTION_DETAIL} is the only clear way to write a
    doctest that doesn't care about the exception detail yet continues
    to pass under Python releases prior to 2.4 (doctest directives
    appear to be comments to them).  For example,

\begin{verbatim}
>>> (1, 2)[3] = 'moo' #doctest: +IGNORE_EXCEPTION_DETAIL
Traceback (most recent call last):
  File "<stdin>", line 1, in ?
TypeError: object doesn't support item assignment
\end{verbatim}

    passes under Python 2.4 and Python 2.3.  The detail changed in 2.4,
    to say "does not" instead of "doesn't".

\end{datadesc}

\begin{datadesc}{COMPARISON_FLAGS}
    A bitmask or'ing together all the comparison flags above.
\end{datadesc}

The second group of options controls how test failures are reported:

\begin{datadesc}{REPORT_UDIFF}
    When specified, failures that involve multi-line expected and
    actual outputs are displayed using a unified diff.
\end{datadesc}

\begin{datadesc}{REPORT_CDIFF}
    When specified, failures that involve multi-line expected and
    actual outputs will be displayed using a context diff.
\end{datadesc}

\begin{datadesc}{REPORT_NDIFF}
    When specified, differences are computed by \code{difflib.Differ},
    using the same algorithm as the popular \file{ndiff.py} utility.
    This is the only method that marks differences within lines as
    well as across lines.  For example, if a line of expected output
    contains digit \code{1} where actual output contains letter \code{l},
    a line is inserted with a caret marking the mismatching column
    positions.
\end{datadesc}

\begin{datadesc}{REPORT_ONLY_FIRST_FAILURE}
  When specified, display the first failing example in each doctest,
  but suppress output for all remaining examples.  This will prevent
  doctest from reporting correct examples that break because of
  earlier failures; but it might also hide incorrect examples that
  fail independently of the first failure.  When
  \constant{REPORT_ONLY_FIRST_FAILURE} is specified, the remaining
  examples are still run, and still count towards the total number of
  failures reported; only the output is suppressed.
\end{datadesc}

\begin{datadesc}{REPORTING_FLAGS}
    A bitmask or'ing together all the reporting flags above.
\end{datadesc}

"Doctest directives" may be used to modify the option flags for
individual examples.  Doctest directives are expressed as a special
Python comment following an example's source code:

\begin{productionlist}[doctest]
    \production{directive}
               {"\#" "doctest:" \token{directive_options}}
    \production{directive_options}
               {\token{directive_option} ("," \token{directive_option})*}
    \production{directive_option}
               {\token{on_or_off} \token{directive_option_name}}
    \production{on_or_off}
               {"+" | "-"}
    \production{directive_option_name}
               {"DONT_ACCEPT_BLANKLINE" | "NORMALIZE_WHITESPACE" | ...}
\end{productionlist}

Whitespace is not allowed between the \code{+} or \code{-} and the
directive option name.  The directive option name can be any of the
option flag names explained above.

An example's doctest directives modify doctest's behavior for that
single example.  Use \code{+} to enable the named behavior, or
\code{-} to disable it.

For example, this test passes:

\begin{verbatim}
>>> print range(20) #doctest: +NORMALIZE_WHITESPACE
[0,   1,  2,  3,  4,  5,  6,  7,  8,  9,
10,  11, 12, 13, 14, 15, 16, 17, 18, 19]
\end{verbatim}

Without the directive it would fail, both because the actual output
doesn't have two blanks before the single-digit list elements, and
because the actual output is on a single line.  This test also passes,
and also requires a directive to do so:

\begin{verbatim}
>>> print range(20) # doctest:+ELLIPSIS
[0, 1, ..., 18, 19]
\end{verbatim}

Multiple directives can be used on a single physical line, separated
by commas:

\begin{verbatim}
>>> print range(20) # doctest: +ELLIPSIS, +NORMALIZE_WHITESPACE
[0,    1, ...,   18,    19]
\end{verbatim}

If multiple directive comments are used for a single example, then
they are combined:

\begin{verbatim}
>>> print range(20) # doctest: +ELLIPSIS
...                 # doctest: +NORMALIZE_WHITESPACE
[0,    1, ...,   18,    19]
\end{verbatim}

As the previous example shows, you can add \samp{...} lines to your
example containing only directives.  This can be useful when an
example is too long for a directive to comfortably fit on the same
line:

\begin{verbatim}
>>> print range(5) + range(10,20) + range(30,40) + range(50,60)
... # doctest: +ELLIPSIS
[0, ..., 4, 10, ..., 19, 30, ..., 39, 50, ..., 59]
\end{verbatim}

Note that since all options are disabled by default, and directives apply
only to the example they appear in, enabling options (via \code{+} in a
directive) is usually the only meaningful choice.  However, option flags
can also be passed to functions that run doctests, establishing different
defaults.  In such cases, disabling an option via \code{-} in a directive
can be useful.

\versionchanged[Constants \constant{DONT_ACCEPT_BLANKLINE},
    \constant{NORMALIZE_WHITESPACE}, \constant{ELLIPSIS},
    \constant{IGNORE_EXCEPTION_DETAIL},
    \constant{REPORT_UDIFF}, \constant{REPORT_CDIFF},
    \constant{REPORT_NDIFF}, \constant{REPORT_ONLY_FIRST_FAILURE},
    \constant{COMPARISON_FLAGS} and \constant{REPORTING_FLAGS}
    were added; by default \code{<BLANKLINE>} in expected output
    matches an empty line in actual output; and doctest directives
    were added]{2.4}

There's also a way to register new option flag names, although this
isn't useful unless you intend to extend \refmodule{doctest} internals
via subclassing:

\begin{funcdesc}{register_optionflag}{name}
  Create a new option flag with a given name, and return the new
  flag's integer value.  \function{register_optionflag()} can be
  used when subclassing \class{OutputChecker} or
  \class{DocTestRunner} to create new options that are supported by
  your subclasses.  \function{register_optionflag} should always be
  called using the following idiom:

\begin{verbatim}
  MY_FLAG = register_optionflag('MY_FLAG')
\end{verbatim}

  \versionadded{2.4}
\end{funcdesc}

\subsubsection{Warnings\label{doctest-warnings}}

\refmodule{doctest} is serious about requiring exact matches in expected
output.  If even a single character doesn't match, the test fails.  This
will probably surprise you a few times, as you learn exactly what Python
does and doesn't guarantee about output.  For example, when printing a
dict, Python doesn't guarantee that the key-value pairs will be printed
in any particular order, so a test like

% Hey! What happened to Monty Python examples?
% Tim: ask Guido -- it's his example!
\begin{verbatim}
>>> foo()
{"Hermione": "hippogryph", "Harry": "broomstick"}
\end{verbatim}

is vulnerable!  One workaround is to do

\begin{verbatim}
>>> foo() == {"Hermione": "hippogryph", "Harry": "broomstick"}
True
\end{verbatim}

instead.  Another is to do

\begin{verbatim}
>>> d = foo().items()
>>> d.sort()
>>> d
[('Harry', 'broomstick'), ('Hermione', 'hippogryph')]
\end{verbatim}

There are others, but you get the idea.

Another bad idea is to print things that embed an object address, like

\begin{verbatim}
>>> id(1.0) # certain to fail some of the time
7948648
>>> class C: pass
>>> C()   # the default repr() for instances embeds an address
<__main__.C instance at 0x00AC18F0>
\end{verbatim}

The \constant{ELLIPSIS} directive gives a nice approach for the last
example:

\begin{verbatim}
>>> C() #doctest: +ELLIPSIS
<__main__.C instance at 0x...>
\end{verbatim}

Floating-point numbers are also subject to small output variations across
platforms, because Python defers to the platform C library for float
formatting, and C libraries vary widely in quality here.

\begin{verbatim}
>>> 1./7  # risky
0.14285714285714285
>>> print 1./7 # safer
0.142857142857
>>> print round(1./7, 6) # much safer
0.142857
\end{verbatim}

Numbers of the form \code{I/2.**J} are safe across all platforms, and I
often contrive doctest examples to produce numbers of that form:

\begin{verbatim}
>>> 3./4  # utterly safe
0.75
\end{verbatim}

Simple fractions are also easier for people to understand, and that makes
for better documentation.

\subsection{Basic API\label{doctest-basic-api}}

The functions \function{testmod()} and \function{testfile()} provide a
simple interface to doctest that should be sufficient for most basic
uses.  For a less formal introduction to these two functions, see
sections \ref{doctest-simple-testmod} and
\ref{doctest-simple-testfile}.

\begin{funcdesc}{testfile}{filename\optional{, module_relative}\optional{,
                          name}\optional{, package}\optional{,
                          globs}\optional{, verbose}\optional{,
                          report}\optional{, optionflags}\optional{,
                          extraglobs}\optional{, raise_on_error}\optional{,
                          parser}}

  All arguments except \var{filename} are optional, and should be
  specified in keyword form.

  Test examples in the file named \var{filename}.  Return
  \samp{(\var{failure_count}, \var{test_count})}.

  Optional argument \var{module_relative} specifies how the filename
  should be interpreted:

  \begin{itemize}
  \item If \var{module_relative} is \code{True} (the default), then
        \var{filename} specifies an OS-independent module-relative
        path.  By default, this path is relative to the calling
        module's directory; but if the \var{package} argument is
        specified, then it is relative to that package.  To ensure
        OS-independence, \var{filename} should use \code{/} characters
        to separate path segments, and may not be an absolute path
        (i.e., it may not begin with \code{/}).
  \item If \var{module_relative} is \code{False}, then \var{filename}
        specifies an OS-specific path.  The path may be absolute or
        relative; relative paths are resolved with respect to the
        current working directory.
  \end{itemize}

  Optional argument \var{name} gives the name of the test; by default,
  or if \code{None}, \code{os.path.basename(\var{filename})} is used.

  Optional argument \var{package} is a Python package or the name of a
  Python package whose directory should be used as the base directory
  for a module-relative filename.  If no package is specified, then
  the calling module's directory is used as the base directory for
  module-relative filenames.  It is an error to specify \var{package}
  if \var{module_relative} is \code{False}.

  Optional argument \var{globs} gives a dict to be used as the globals
  when executing examples.  A new shallow copy of this dict is
  created for the doctest, so its examples start with a clean slate.
  By default, or if \code{None}, a new empty dict is used.

  Optional argument \var{extraglobs} gives a dict merged into the
  globals used to execute examples.  This works like
  \method{dict.update()}:  if \var{globs} and \var{extraglobs} have a
  common key, the associated value in \var{extraglobs} appears in the
  combined dict.  By default, or if \code{None}, no extra globals are
  used.  This is an advanced feature that allows parameterization of
  doctests.  For example, a doctest can be written for a base class, using
  a generic name for the class, then reused to test any number of
  subclasses by passing an \var{extraglobs} dict mapping the generic
  name to the subclass to be tested.

  Optional argument \var{verbose} prints lots of stuff if true, and prints
  only failures if false; by default, or if \code{None}, it's true
  if and only if \code{'-v'} is in \code{sys.argv}.

  Optional argument \var{report} prints a summary at the end when true,
  else prints nothing at the end.  In verbose mode, the summary is
  detailed, else the summary is very brief (in fact, empty if all tests
  passed).

  Optional argument \var{optionflags} or's together option flags.  See
  section~\ref{doctest-options}.

  Optional argument \var{raise_on_error} defaults to false.  If true,
  an exception is raised upon the first failure or unexpected exception
  in an example.  This allows failures to be post-mortem debugged.
  Default behavior is to continue running examples.

  Optional argument \var{parser} specifies a \class{DocTestParser} (or
  subclass) that should be used to extract tests from the files.  It
  defaults to a normal parser (i.e., \code{\class{DocTestParser}()}).

  \versionadded{2.4}
\end{funcdesc}

\begin{funcdesc}{testmod}{\optional{m}\optional{, name}\optional{,
                          globs}\optional{, verbose}\optional{,
                          isprivate}\optional{, report}\optional{,
                          optionflags}\optional{, extraglobs}\optional{,
                          raise_on_error}\optional{, exclude_empty}}

  All arguments are optional, and all except for \var{m} should be
  specified in keyword form.

  Test examples in docstrings in functions and classes reachable
  from module \var{m} (or module \module{__main__} if \var{m} is not
  supplied or is \code{None}), starting with \code{\var{m}.__doc__}.

  Also test examples reachable from dict \code{\var{m}.__test__}, if it
  exists and is not \code{None}.  \code{\var{m}.__test__} maps
  names (strings) to functions, classes and strings; function and class
  docstrings are searched for examples; strings are searched directly,
  as if they were docstrings.

  Only docstrings attached to objects belonging to module \var{m} are
  searched.

  Return \samp{(\var{failure_count}, \var{test_count})}.

  Optional argument \var{name} gives the name of the module; by default,
  or if \code{None}, \code{\var{m}.__name__} is used.

  Optional argument \var{exclude_empty} defaults to false.  If true,
  objects for which no doctests are found are excluded from consideration.
  The default is a backward compatibility hack, so that code still
  using \method{doctest.master.summarize()} in conjunction with
  \function{testmod()} continues to get output for objects with no tests.
  The \var{exclude_empty} argument to the newer \class{DocTestFinder}
  constructor defaults to true.

  Optional arguments \var{extraglobs}, \var{verbose}, \var{report},
  \var{optionflags}, \var{raise_on_error}, and \var{globs} are the same as
  for function \function{testfile()} above, except that \var{globs}
  defaults to \code{\var{m}.__dict__}.

  Optional argument \var{isprivate} specifies a function used to
  determine whether a name is private.  The default function treats
  all names as public.  \var{isprivate} can be set to
  \code{doctest.is_private} to skip over names that are
  private according to Python's underscore naming convention.
  \deprecated{2.4}{\var{isprivate} was a stupid idea -- don't use it.
  If you need to skip tests based on name, filter the list returned by
  \code{DocTestFinder.find()} instead.}

  \versionchanged[The parameter \var{optionflags} was added]{2.3}

  \versionchanged[The parameters \var{extraglobs}, \var{raise_on_error}
                  and \var{exclude_empty} were added]{2.4}
\end{funcdesc}

There's also a function to run the doctests associated with a single object.
This function is provided for backward compatibility.  There are no plans
to deprecate it, but it's rarely useful:

\begin{funcdesc}{run_docstring_examples}{f, globs\optional{,
                            verbose}\optional{, name}\optional{,
                            compileflags}\optional{, optionflags}}

  Test examples associated with object \var{f}; for example, \var{f} may
  be a module, function, or class object.

  A shallow copy of dictionary argument \var{globs} is used for the
  execution context.

  Optional argument \var{name} is used in failure messages, and defaults
  to \code{"NoName"}.

  If optional argument \var{verbose} is true, output is generated even
  if there are no failures.  By default, output is generated only in case
  of an example failure.

  Optional argument \var{compileflags} gives the set of flags that should
  be used by the Python compiler when running the examples.  By default, or
  if \code{None}, flags are deduced corresponding to the set of future
  features found in \var{globs}.

  Optional argument \var{optionflags} works as for function
  \function{testfile()} above.
\end{funcdesc}

\subsection{Unittest API\label{doctest-unittest-api}}

As your collection of doctest'ed modules grows, you'll want a way to run
all their doctests systematically.  Prior to Python 2.4, \refmodule{doctest}
had a barely documented \class{Tester} class that supplied a rudimentary
way to combine doctests from multiple modules. \class{Tester} was feeble,
and in practice most serious Python testing frameworks build on the
\refmodule{unittest} module, which supplies many flexible ways to combine
tests from multiple sources.  So, in Python 2.4, \refmodule{doctest}'s
\class{Tester} class is deprecated, and \refmodule{doctest} provides two
functions that can be used to create \refmodule{unittest} test suites from
modules and text files containing doctests.  These test suites can then be
run using \refmodule{unittest} test runners:

\begin{verbatim}
import unittest
import doctest
import my_module_with_doctests, and_another

suite = unittest.TestSuite()
for mod in my_module_with_doctests, and_another:
    suite.addTest(doctest.DocTestSuite(mod))
runner = unittest.TextTestRunner()
runner.run(suite)
\end{verbatim}

There are two main functions for creating \class{\refmodule{unittest}.TestSuite}
instances from text files and modules with doctests:

\begin{funcdesc}{DocFileSuite}{*paths, **kw}
  Convert doctest tests from one or more text files to a
  \class{\refmodule{unittest}.TestSuite}.

  The returned \class{\refmodule{unittest}.TestSuite} is to be run by the
  unittest framework and runs the interactive examples in each file.  If an
  example in any file fails, then the synthesized unit test fails, and a
  \exception{failureException} exception is raised showing the name of the
  file containing the test and a (sometimes approximate) line number.

  Pass one or more paths (as strings) to text files to be examined.

  Options may be provided as keyword arguments:

  Optional argument \var{module_relative} specifies how
  the filenames in \var{paths} should be interpreted:

  \begin{itemize}
  \item If \var{module_relative} is \code{True} (the default), then
        each filename specifies an OS-independent module-relative
        path.  By default, this path is relative to the calling
        module's directory; but if the \var{package} argument is
        specified, then it is relative to that package.  To ensure
        OS-independence, each filename should use \code{/} characters
        to separate path segments, and may not be an absolute path
        (i.e., it may not begin with \code{/}).
  \item If \var{module_relative} is \code{False}, then each filename
        specifies an OS-specific path.  The path may be absolute or
        relative; relative paths are resolved with respect to the
        current working directory.
  \end{itemize}

  Optional argument \var{package} is a Python package or the name
  of a Python package whose directory should be used as the base
  directory for module-relative filenames.  If no package is
  specified, then the calling module's directory is used as the base
  directory for module-relative filenames.  It is an error to specify
  \var{package} if \var{module_relative} is \code{False}.

  Optional argument \var{setUp} specifies a set-up function for
  the test suite.  This is called before running the tests in each
  file.  The \var{setUp} function will be passed a \class{DocTest}
  object.  The setUp function can access the test globals as the
  \var{globs} attribute of the test passed.

  Optional argument \var{tearDown} specifies a tear-down function
  for the test suite.  This is called after running the tests in each
  file.  The \var{tearDown} function will be passed a \class{DocTest}
  object.  The setUp function can access the test globals as the
  \var{globs} attribute of the test passed.

  Optional argument \var{globs} is a dictionary containing the
  initial global variables for the tests.  A new copy of this
  dictionary is created for each test.  By default, \var{globs} is
  a new empty dictionary.

  Optional argument \var{optionflags} specifies the default
  doctest options for the tests, created by or-ing together
  individual option flags.  See section~\ref{doctest-options}.
  See function \function{set_unittest_reportflags()} below for
  a better way to set reporting options.

  Optional argument \var{parser} specifies a \class{DocTestParser} (or
  subclass) that should be used to extract tests from the files.  It
  defaults to a normal parser (i.e., \code{\class{DocTestParser}()}).

  \versionadded{2.4}
\end{funcdesc}

\begin{funcdesc}{DocTestSuite}{\optional{module}\optional{,
                              globs}\optional{, extraglobs}\optional{,
                              test_finder}\optional{, setUp}\optional{,
                              tearDown}\optional{, checker}}
  Convert doctest tests for a module to a
  \class{\refmodule{unittest}.TestSuite}.

  The returned \class{\refmodule{unittest}.TestSuite} is to be run by the
  unittest framework and runs each doctest in the module.  If any of the
  doctests fail, then the synthesized unit test fails, and a
  \exception{failureException} exception is raised showing the name of the
  file containing the test and a (sometimes approximate) line number.

  Optional argument \var{module} provides the module to be tested.  It
  can be a module object or a (possibly dotted) module name.  If not
  specified, the module calling this function is used.

  Optional argument \var{globs} is a dictionary containing the
  initial global variables for the tests.  A new copy of this
  dictionary is created for each test.  By default, \var{globs} is
  a new empty dictionary.

  Optional argument \var{extraglobs} specifies an extra set of
  global variables, which is merged into \var{globs}.  By default, no
  extra globals are used.

  Optional argument \var{test_finder} is the \class{DocTestFinder}
  object (or a drop-in replacement) that is used to extract doctests
  from the module.

  Optional arguments \var{setUp}, \var{tearDown}, and \var{optionflags}
  are the same as for function \function{DocFileSuite()} above.

  \versionadded{2.3}

  \versionchanged[The parameters \var{globs}, \var{extraglobs},
    \var{test_finder}, \var{setUp}, \var{tearDown}, and
    \var{optionflags} were added; this function now uses the same search
    technique as \function{testmod()}]{2.4}
\end{funcdesc}

Under the covers, \function{DocTestSuite()} creates a
\class{\refmodule{unittest}.TestSuite} out of \class{doctest.DocTestCase}
instances, and \class{DocTestCase} is a subclass of
\class{\refmodule{unittest}.TestCase}. \class{DocTestCase} isn't documented
here (it's an internal detail), but studying its code can answer questions
about the exact details of \refmodule{unittest} integration.

Similarly, \function{DocFileSuite()} creates a
\class{\refmodule{unittest}.TestSuite} out of \class{doctest.DocFileCase}
instances, and \class{DocFileCase} is a subclass of \class{DocTestCase}.

So both ways of creating a \class{\refmodule{unittest}.TestSuite} run
instances of \class{DocTestCase}.  This is important for a subtle reason:
when you run \refmodule{doctest} functions yourself, you can control the
\refmodule{doctest} options in use directly, by passing option flags to
\refmodule{doctest} functions.  However, if you're writing a
\refmodule{unittest} framework, \refmodule{unittest} ultimately controls
when and how tests get run.  The framework author typically wants to
control \refmodule{doctest} reporting options (perhaps, e.g., specified by
command line options), but there's no way to pass options through
\refmodule{unittest} to \refmodule{doctest} test runners.

For this reason, \refmodule{doctest} also supports a notion of
\refmodule{doctest} reporting flags specific to \refmodule{unittest}
support, via this function:

\begin{funcdesc}{set_unittest_reportflags}{flags}
  Set the \refmodule{doctest} reporting flags to use.

  Argument \var{flags} or's together option flags.  See
  section~\ref{doctest-options}.  Only "reporting flags" can be used.

  This is a module-global setting, and affects all future doctests run by
  module \refmodule{unittest}:  the \method{runTest()} method of
  \class{DocTestCase} looks at the option flags specified for the test case
  when the \class{DocTestCase} instance was constructed.  If no reporting
  flags were specified (which is the typical and expected case),
  \refmodule{doctest}'s \refmodule{unittest} reporting flags are or'ed into
  the option flags, and the option flags so augmented are passed to the
  \class{DocTestRunner} instance created to run the doctest.  If any
  reporting flags were specified when the \class{DocTestCase} instance was
  constructed, \refmodule{doctest}'s \refmodule{unittest} reporting flags
  are ignored.

  The value of the \refmodule{unittest} reporting flags in effect before the
  function was called is returned by the function.

  \versionadded{2.4}
\end{funcdesc}


\subsection{Advanced API\label{doctest-advanced-api}}

The basic API is a simple wrapper that's intended to make doctest easy
to use.  It is fairly flexible, and should meet most users' needs;
however, if you require more fine-grained control over testing, or
wish to extend doctest's capabilities, then you should use the
advanced API.

The advanced API revolves around two container classes, which are used
to store the interactive examples extracted from doctest cases:

\begin{itemize}
\item \class{Example}: A single python statement, paired with its
      expected output.
\item \class{DocTest}: A collection of \class{Example}s, typically
      extracted from a single docstring or text file.
\end{itemize}

Additional processing classes are defined to find, parse, and run, and
check doctest examples:

\begin{itemize}
\item \class{DocTestFinder}: Finds all docstrings in a given module,
      and uses a \class{DocTestParser} to create a \class{DocTest}
      from every docstring that contains interactive examples.
\item \class{DocTestParser}: Creates a \class{DocTest} object from
      a string (such as an object's docstring).
\item \class{DocTestRunner}: Executes the examples in a
      \class{DocTest}, and uses an \class{OutputChecker} to verify
      their output.
\item \class{OutputChecker}: Compares the actual output from a
      doctest example with the expected output, and decides whether
      they match.
\end{itemize}

The relationships among these processing classes are summarized in the
following diagram:

\begin{verbatim}
                            list of:
+------+                   +---------+
|module| --DocTestFinder-> | DocTest | --DocTestRunner-> results
+------+    |        ^     +---------+     |       ^    (printed)
            |        |     | Example |     |       |
            v        |     |   ...   |     v       |
           DocTestParser   | Example |   OutputChecker
                           +---------+
\end{verbatim}

\subsubsection{DocTest Objects\label{doctest-DocTest}}
\begin{classdesc}{DocTest}{examples, globs, name, filename, lineno,
                           docstring}
    A collection of doctest examples that should be run in a single
    namespace.  The constructor arguments are used to initialize the
    member variables of the same names.
    \versionadded{2.4}
\end{classdesc}

\class{DocTest} defines the following member variables.  They are
initialized by the constructor, and should not be modified directly.

\begin{memberdesc}{examples}
    A list of \class{Example} objects encoding the individual
    interactive Python examples that should be run by this test.
\end{memberdesc}

\begin{memberdesc}{globs}
    The namespace (aka globals) that the examples should be run in.
    This is a dictionary mapping names to values.  Any changes to the
    namespace made by the examples (such as binding new variables)
    will be reflected in \member{globs} after the test is run.
\end{memberdesc}

\begin{memberdesc}{name}
    A string name identifying the \class{DocTest}.  Typically, this is
    the name of the object or file that the test was extracted from.
\end{memberdesc}

\begin{memberdesc}{filename}
    The name of the file that this \class{DocTest} was extracted from;
    or \code{None} if the filename is unknown, or if the
    \class{DocTest} was not extracted from a file.
\end{memberdesc}

\begin{memberdesc}{lineno}
    The line number within \member{filename} where this
    \class{DocTest} begins, or \code{None} if the line number is
    unavailable.  This line number is zero-based with respect to the
    beginning of the file.
\end{memberdesc}

\begin{memberdesc}{docstring}
    The string that the test was extracted from, or `None` if the
    string is unavailable, or if the test was not extracted from a
    string.
\end{memberdesc}

\subsubsection{Example Objects\label{doctest-Example}}
\begin{classdesc}{Example}{source, want\optional{,
                           exc_msg}\optional{, lineno}\optional{,
                           indent}\optional{, options}}
    A single interactive example, consisting of a Python statement and
    its expected output.  The constructor arguments are used to
    initialize the member variables of the same names.
    \versionadded{2.4}
\end{classdesc}

\class{Example} defines the following member variables.  They are
initialized by the constructor, and should not be modified directly.

\begin{memberdesc}{source}
    A string containing the example's source code.  This source code
    consists of a single Python statement, and always ends with a
    newline; the constructor adds a newline when necessary.
\end{memberdesc}

\begin{memberdesc}{want}
    The expected output from running the example's source code (either
    from stdout, or a traceback in case of exception).  \member{want}
    ends with a newline unless no output is expected, in which case
    it's an empty string.  The constructor adds a newline when
    necessary.
\end{memberdesc}

\begin{memberdesc}{exc_msg}
    The exception message generated by the example, if the example is
    expected to generate an exception; or \code{None} if it is not
    expected to generate an exception.  This exception message is
    compared against the return value of
    \function{traceback.format_exception_only()}.  \member{exc_msg}
    ends with a newline unless it's \code{None}.  The constructor adds
    a newline if needed.
\end{memberdesc}

\begin{memberdesc}{lineno}
    The line number within the string containing this example where
    the example begins.  This line number is zero-based with respect
    to the beginning of the containing string.
\end{memberdesc}

\begin{memberdesc}{indent}
    The example's indentation in the containing string, i.e., the
    number of space characters that preceed the example's first
    prompt.
\end{memberdesc}

\begin{memberdesc}{options}
    A dictionary mapping from option flags to \code{True} or
    \code{False}, which is used to override default options for this
    example.  Any option flags not contained in this dictionary are
    left at their default value (as specified by the
    \class{DocTestRunner}'s \member{optionflags}).
    By default, no options are set.
\end{memberdesc}

\subsubsection{DocTestFinder objects\label{doctest-DocTestFinder}}
\begin{classdesc}{DocTestFinder}{\optional{verbose}\optional{,
                                parser}\optional{, recurse}\optional{,
                                exclude_empty}}
    A processing class used to extract the \class{DocTest}s that are
    relevant to a given object, from its docstring and the docstrings
    of its contained objects.  \class{DocTest}s can currently be
    extracted from the following object types: modules, functions,
    classes, methods, staticmethods, classmethods, and properties.

    The optional argument \var{verbose} can be used to display the
    objects searched by the finder.  It defaults to \code{False} (no
    output).

    The optional argument \var{parser} specifies the
    \class{DocTestParser} object (or a drop-in replacement) that is
    used to extract doctests from docstrings.

    If the optional argument \var{recurse} is false, then
    \method{DocTestFinder.find()} will only examine the given object,
    and not any contained objects.

    If the optional argument \var{exclude_empty} is false, then
    \method{DocTestFinder.find()} will include tests for objects with
    empty docstrings.

    \versionadded{2.4}
\end{classdesc}

\class{DocTestFinder} defines the following method:

\begin{methoddesc}{find}{obj\optional{, name}\optional{,
                   module}\optional{, globs}\optional{, extraglobs}}
    Return a list of the \class{DocTest}s that are defined by
    \var{obj}'s docstring, or by any of its contained objects'
    docstrings.

    The optional argument \var{name} specifies the object's name; this
    name will be used to construct names for the returned
    \class{DocTest}s.  If \var{name} is not specified, then
    \code{\var{obj}.__name__} is used.

    The optional parameter \var{module} is the module that contains
    the given object.  If the module is not specified or is None, then
    the test finder will attempt to automatically determine the
    correct module.  The object's module is used:

    \begin{itemize}
    \item As a default namespace, if \var{globs} is not specified.
    \item To prevent the DocTestFinder from extracting DocTests
          from objects that are imported from other modules.  (Contained
          objects with modules other than \var{module} are ignored.)
    \item To find the name of the file containing the object.
    \item To help find the line number of the object within its file.
    \end{itemize}

    If \var{module} is \code{False}, no attempt to find the module
    will be made.  This is obscure, of use mostly in testing doctest
    itself: if \var{module} is \code{False}, or is \code{None} but
    cannot be found automatically, then all objects are considered to
    belong to the (non-existent) module, so all contained objects will
    (recursively) be searched for doctests.

    The globals for each \class{DocTest} is formed by combining
    \var{globs} and \var{extraglobs} (bindings in \var{extraglobs}
    override bindings in \var{globs}).  A new shallow copy of the globals
    dictionary is created for each \class{DocTest}.  If \var{globs} is
    not specified, then it defaults to the module's \var{__dict__}, if
    specified, or \code{\{\}} otherwise.  If \var{extraglobs} is not
    specified, then it defaults to \code{\{\}}.
\end{methoddesc}

\subsubsection{DocTestParser objects\label{doctest-DocTestParser}}
\begin{classdesc}{DocTestParser}{}
    A processing class used to extract interactive examples from a
    string, and use them to create a \class{DocTest} object.
    \versionadded{2.4}
\end{classdesc}

\class{DocTestParser} defines the following methods:

\begin{methoddesc}{get_doctest}{string, globs, name, filename, lineno}
    Extract all doctest examples from the given string, and collect
    them into a \class{DocTest} object.

    \var{globs}, \var{name}, \var{filename}, and \var{lineno} are
    attributes for the new \class{DocTest} object.  See the
    documentation for \class{DocTest} for more information.
\end{methoddesc}

\begin{methoddesc}{get_examples}{string\optional{, name}}
    Extract all doctest examples from the given string, and return
    them as a list of \class{Example} objects.  Line numbers are
    0-based.  The optional argument \var{name} is a name identifying
    this string, and is only used for error messages.
\end{methoddesc}

\begin{methoddesc}{parse}{string\optional{, name}}
    Divide the given string into examples and intervening text, and
    return them as a list of alternating \class{Example}s and strings.
    Line numbers for the \class{Example}s are 0-based.  The optional
    argument \var{name} is a name identifying this string, and is only
    used for error messages.
\end{methoddesc}

\subsubsection{DocTestRunner objects\label{doctest-DocTestRunner}}
\begin{classdesc}{DocTestRunner}{\optional{checker}\optional{,
                                 verbose}\optional{, optionflags}}
    A processing class used to execute and verify the interactive
    examples in a \class{DocTest}.

    The comparison between expected outputs and actual outputs is done
    by an \class{OutputChecker}.  This comparison may be customized
    with a number of option flags; see section~\ref{doctest-options}
    for more information.  If the option flags are insufficient, then
    the comparison may also be customized by passing a subclass of
    \class{OutputChecker} to the constructor.

    The test runner's display output can be controlled in two ways.
    First, an output function can be passed to
    \method{TestRunner.run()}; this function will be called with
    strings that should be displayed.  It defaults to
    \code{sys.stdout.write}.  If capturing the output is not
    sufficient, then the display output can be also customized by
    subclassing DocTestRunner, and overriding the methods
    \method{report_start}, \method{report_success},
    \method{report_unexpected_exception}, and \method{report_failure}.

    The optional keyword argument \var{checker} specifies the
    \class{OutputChecker} object (or drop-in replacement) that should
    be used to compare the expected outputs to the actual outputs of
    doctest examples.

    The optional keyword argument \var{verbose} controls the
    \class{DocTestRunner}'s verbosity.  If \var{verbose} is
    \code{True}, then information is printed about each example, as it
    is run.  If \var{verbose} is \code{False}, then only failures are
    printed.  If \var{verbose} is unspecified, or \code{None}, then
    verbose output is used iff the command-line switch \programopt{-v}
    is used.

    The optional keyword argument \var{optionflags} can be used to
    control how the test runner compares expected output to actual
    output, and how it displays failures.  For more information, see
    section~\ref{doctest-options}.

    \versionadded{2.4}
\end{classdesc}

\class{DocTestParser} defines the following methods:

\begin{methoddesc}{report_start}{out, test, example}
    Report that the test runner is about to process the given example.
    This method is provided to allow subclasses of
    \class{DocTestRunner} to customize their output; it should not be
    called directly.

    \var{example} is the example about to be processed.  \var{test} is
    the test containing \var{example}.  \var{out} is the output
    function that was passed to \method{DocTestRunner.run()}.
\end{methoddesc}

\begin{methoddesc}{report_success}{out, test, example, got}
    Report that the given example ran successfully.  This method is
    provided to allow subclasses of \class{DocTestRunner} to customize
    their output; it should not be called directly.

    \var{example} is the example about to be processed.  \var{got} is
    the actual output from the example.  \var{test} is the test
    containing \var{example}.  \var{out} is the output function that
    was passed to \method{DocTestRunner.run()}.
\end{methoddesc}

\begin{methoddesc}{report_failure}{out, test, example, got}
    Report that the given example failed.  This method is provided to
    allow subclasses of \class{DocTestRunner} to customize their
    output; it should not be called directly.

    \var{example} is the example about to be processed.  \var{got} is
    the actual output from the example.  \var{test} is the test
    containing \var{example}.  \var{out} is the output function that
    was passed to \method{DocTestRunner.run()}.
\end{methoddesc}

\begin{methoddesc}{report_unexpected_exception}{out, test, example, exc_info}
    Report that the given example raised an unexpected exception.
    This method is provided to allow subclasses of
    \class{DocTestRunner} to customize their output; it should not be
    called directly.

    \var{example} is the example about to be processed.
    \var{exc_info} is a tuple containing information about the
    unexpected exception (as returned by \function{sys.exc_info()}).
    \var{test} is the test containing \var{example}.  \var{out} is the
    output function that was passed to \method{DocTestRunner.run()}.
\end{methoddesc}

\begin{methoddesc}{run}{test\optional{, compileflags}\optional{,
                        out}\optional{, clear_globs}}
    Run the examples in \var{test} (a \class{DocTest} object), and
    display the results using the writer function \var{out}.

    The examples are run in the namespace \code{test.globs}.  If
    \var{clear_globs} is true (the default), then this namespace will
    be cleared after the test runs, to help with garbage collection.
    If you would like to examine the namespace after the test
    completes, then use \var{clear_globs=False}.

    \var{compileflags} gives the set of flags that should be used by
    the Python compiler when running the examples.  If not specified,
    then it will default to the set of future-import flags that apply
    to \var{globs}.

    The output of each example is checked using the
    \class{DocTestRunner}'s output checker, and the results are
    formatted by the \method{DocTestRunner.report_*} methods.
\end{methoddesc}

\begin{methoddesc}{summarize}{\optional{verbose}}
    Print a summary of all the test cases that have been run by this
    DocTestRunner, and return a tuple \samp{(\var{failure_count},
    \var{test_count})}.

    The optional \var{verbose} argument controls how detailed the
    summary is.  If the verbosity is not specified, then the
    \class{DocTestRunner}'s verbosity is used.
\end{methoddesc}

\subsubsection{OutputChecker objects\label{doctest-OutputChecker}}

\begin{classdesc}{OutputChecker}{}
    A class used to check the whether the actual output from a doctest
    example matches the expected output.  \class{OutputChecker}
    defines two methods: \method{check_output}, which compares a given
    pair of outputs, and returns true if they match; and
    \method{output_difference}, which returns a string describing the
    differences between two outputs.
    \versionadded{2.4}
\end{classdesc}

\class{OutputChecker} defines the following methods:

\begin{methoddesc}{check_output}{want, got, optionflags}
    Return \code{True} iff the actual output from an example
    (\var{got}) matches the expected output (\var{want}).  These
    strings are always considered to match if they are identical; but
    depending on what option flags the test runner is using, several
    non-exact match types are also possible.  See
    section~\ref{doctest-options} for more information about option
    flags.
\end{methoddesc}

\begin{methoddesc}{output_difference}{example, got, optionflags}
    Return a string describing the differences between the expected
    output for a given example (\var{example}) and the actual output
    (\var{got}).  \var{optionflags} is the set of option flags used to
    compare \var{want} and \var{got}.
\end{methoddesc}

\subsection{Debugging\label{doctest-debugging}}

Doctest provides several mechanisms for debugging doctest examples:

\begin{itemize}
\item Several functions convert doctests to executable Python
      programs, which can be run under the Python debugger, \refmodule{pdb}.
\item The \class{DebugRunner} class is a subclass of
      \class{DocTestRunner} that raises an exception for the first
      failing example, containing information about that example.
      This information can be used to perform post-mortem debugging on
      the example.
\item The \refmodule{unittest} cases generated by \function{DocTestSuite()}
      support the \method{debug()} method defined by
      \class{\refmodule{unittest}.TestCase}.
\item You can add a call to \function{\refmodule{pdb}.set_trace()} in a
      doctest example, and you'll drop into the Python debugger when that
      line is executed.  Then you can inspect current values of variables,
      and so on.  For example, suppose \file{a.py} contains just this
      module docstring:

\begin{verbatim}
"""
>>> def f(x):
...     g(x*2)
>>> def g(x):
...     print x+3
...     import pdb; pdb.set_trace()
>>> f(3)
9
"""
\end{verbatim}

      Then an interactive Python session may look like this:

\begin{verbatim}
>>> import a, doctest
>>> doctest.testmod(a)
--Return--
> <doctest a[1]>(3)g()->None
-> import pdb; pdb.set_trace()
(Pdb) list
  1     def g(x):
  2         print x+3
  3  ->     import pdb; pdb.set_trace()
[EOF]
(Pdb) print x
6
(Pdb) step
--Return--
> <doctest a[0]>(2)f()->None
-> g(x*2)
(Pdb) list
  1     def f(x):
  2  ->     g(x*2)
[EOF]
(Pdb) print x
3
(Pdb) step
--Return--
> <doctest a[2]>(1)?()->None
-> f(3)
(Pdb) cont
(0, 3)
>>>
\end{verbatim}

    \versionchanged[The ability to use \code{\refmodule{pdb}.set_trace()}
                    usefully inside doctests was added]{2.4}
\end{itemize}

Functions that convert doctests to Python code, and possibly run
the synthesized code under the debugger:

\begin{funcdesc}{script_from_examples}{s}
  Convert text with examples to a script.

  Argument \var{s} is a string containing doctest examples.  The string
  is converted to a Python script, where doctest examples in \var{s}
  are converted to regular code, and everything else is converted to
  Python comments.  The generated script is returned as a string.
  For example,

    \begin{verbatim}
    import doctest
    print doctest.script_from_examples(r"""
        Set x and y to 1 and 2.
        >>> x, y = 1, 2

        Print their sum:
        >>> print x+y
        3
    """)
    \end{verbatim}

  displays:

    \begin{verbatim}
    # Set x and y to 1 and 2.
    x, y = 1, 2
    #
    # Print their sum:
    print x+y
    # Expected:
    ## 3
    \end{verbatim}

  This function is used internally by other functions (see below), but
  can also be useful when you want to transform an interactive Python
  session into a Python script.

  \versionadded{2.4}
\end{funcdesc}

\begin{funcdesc}{testsource}{module, name}
   Convert the doctest for an object to a script.

   Argument \var{module} is a module object, or dotted name of a module,
   containing the object whose doctests are of interest.  Argument
   \var{name} is the name (within the module) of the object with the
   doctests of interest.  The result is a string, containing the
   object's docstring converted to a Python script, as described for
   \function{script_from_examples()} above.  For example, if module
   \file{a.py} contains a top-level function \function{f()}, then

\begin{verbatim}
import a, doctest
print doctest.testsource(a, "a.f")
\end{verbatim}

  prints a script version of function \function{f()}'s docstring,
  with doctests converted to code, and the rest placed in comments.

  \versionadded{2.3}
\end{funcdesc}

\begin{funcdesc}{debug}{module, name\optional{, pm}}
  Debug the doctests for an object.

  The \var{module} and \var{name} arguments are the same as for function
  \function{testsource()} above.  The synthesized Python script for the
  named object's docstring is written to a temporary file, and then that
  file is run under the control of the Python debugger, \refmodule{pdb}.

  A shallow copy of \code{\var{module}.__dict__} is used for both local
  and global execution context.

  Optional argument \var{pm} controls whether post-mortem debugging is
  used.  If \var{pm} has a true value, the script file is run directly, and
  the debugger gets involved only if the script terminates via raising an
  unhandled exception.  If it does, then post-mortem debugging is invoked,
  via \code{\refmodule{pdb}.post_mortem()}, passing the traceback object
  from the unhandled exception.  If \var{pm} is not specified, or is false,
  the script is run under the debugger from the start, via passing an
  appropriate \function{execfile()} call to \code{\refmodule{pdb}.run()}.

  \versionadded{2.3}

  \versionchanged[The \var{pm} argument was added]{2.4}
\end{funcdesc}

\begin{funcdesc}{debug_src}{src\optional{, pm}\optional{, globs}}
  Debug the doctests in a string.

  This is like function \function{debug()} above, except that
  a string containing doctest examples is specified directly, via
  the \var{src} argument.

  Optional argument \var{pm} has the same meaning as in function
  \function{debug()} above.

  Optional argument \var{globs} gives a dictionary to use as both
  local and global execution context.  If not specified, or \code{None},
  an empty dictionary is used.  If specified, a shallow copy of the
  dictionary is used.

  \versionadded{2.4}
\end{funcdesc}

The \class{DebugRunner} class, and the special exceptions it may raise,
are of most interest to testing framework authors, and will only be
sketched here.  See the source code, and especially \class{DebugRunner}'s
docstring (which is a doctest!) for more details:

\begin{classdesc}{DebugRunner}{\optional{checker}\optional{,
                                 verbose}\optional{, optionflags}}

    A subclass of \class{DocTestRunner} that raises an exception as
    soon as a failure is encountered.  If an unexpected exception
    occurs, an \exception{UnexpectedException} exception is raised,
    containing the test, the example, and the original exception.  If
    the output doesn't match, then a \exception{DocTestFailure}
    exception is raised, containing the test, the example, and the
    actual output.

    For information about the constructor parameters and methods, see
    the documentation for \class{DocTestRunner} in
    section~\ref{doctest-advanced-api}.
\end{classdesc}

There are two exceptions that may be raised by \class{DebugRunner}
instances:

\begin{excclassdesc}{DocTestFailure}{test, example, got}
    An exception thrown by \class{DocTestRunner} to signal that a
    doctest example's actual output did not match its expected output.
    The constructor arguments are used to initialize the member
    variables of the same names.
\end{excclassdesc}
\exception{DocTestFailure} defines the following member variables:
\begin{memberdesc}{test}
    The \class{DocTest} object that was being run when the example failed.
\end{memberdesc}
\begin{memberdesc}{example}
    The \class{Example} that failed.
\end{memberdesc}
\begin{memberdesc}{got}
    The example's actual output.
\end{memberdesc}

\begin{excclassdesc}{UnexpectedException}{test, example, exc_info}
    An exception thrown by \class{DocTestRunner} to signal that a
    doctest example raised an unexpected exception.  The constructor
    arguments are used to initialize the member variables of the same
    names.
\end{excclassdesc}
\exception{UnexpectedException} defines the following member variables:
\begin{memberdesc}{test}
    The \class{DocTest} object that was being run when the example failed.
\end{memberdesc}
\begin{memberdesc}{example}
    The \class{Example} that failed.
\end{memberdesc}
\begin{memberdesc}{exc_info}
    A tuple containing information about the unexpected exception, as
    returned by \function{sys.exc_info()}.
\end{memberdesc}

\subsection{Soapbox\label{doctest-soapbox}}

As mentioned in the introduction, \refmodule{doctest} has grown to have
three primary uses:

\begin{enumerate}
\item Checking examples in docstrings.
\item Regression testing.
\item Executable documentation / literate testing.
\end{enumerate}

These uses have different requirements, and it is important to
distinguish them.  In particular, filling your docstrings with obscure
test cases makes for bad documentation.

When writing a docstring, choose docstring examples with care.
There's an art to this that needs to be learned---it may not be
natural at first.  Examples should add genuine value to the
documentation.  A good example can often be worth many words.
If done with care, the examples will be invaluable for your users, and
will pay back the time it takes to collect them many times over as the
years go by and things change.  I'm still amazed at how often one of
my \refmodule{doctest} examples stops working after a "harmless"
change.

Doctest also makes an excellent tool for regression testing, especially if
you don't skimp on explanatory text.  By interleaving prose and examples,
it becomes much easier to keep track of what's actually being tested, and
why.  When a test fails, good prose can make it much easier to figure out
what the problem is, and how it should be fixed.  It's true that you could
write extensive comments in code-based testing, but few programmers do.
Many have found that using doctest approaches instead leads to much clearer
tests.  Perhaps this is simply because doctest makes writing prose a little
easier than writing code, while writing comments in code is a little
harder.  I think it goes deeper than just that:  the natural attitude
when writing a doctest-based test is that you want to explain the fine
points of your software, and illustrate them with examples.  This in
turn naturally leads to test files that start with the simplest features,
and logically progress to complications and edge cases.  A coherent
narrative is the result, instead of a collection of isolated functions
that test isolated bits of functionality seemingly at random.  It's
a different attitude, and produces different results, blurring the
distinction between testing and explaining.

Regression testing is best confined to dedicated objects or files.  There
are several options for organizing tests:

\begin{itemize}
\item Write text files containing test cases as interactive examples,
      and test the files using \function{testfile()} or
      \function{DocFileSuite()}.  This is recommended, although is
      easiest to do for new projects, designed from the start to use
      doctest.
\item Define functions named \code{_regrtest_\textit{topic}} that
      consist of single docstrings, containing test cases for the
      named topics.  These functions can be included in the same file
      as the module, or separated out into a separate test file.
\item Define a \code{__test__} dictionary mapping from regression test
      topics to docstrings containing test cases.
\end{itemize}
