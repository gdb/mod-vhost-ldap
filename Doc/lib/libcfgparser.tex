\section{\module{ConfigParser} ---
         Configuration file parser}

\declaremodule{standard}{ConfigParser}
\modulesynopsis{Configuration file parser.}
\moduleauthor{Ken Manheimer}{klm@digicool.com}
\moduleauthor{Barry Warsaw}{bwarsaw@python.org}
\moduleauthor{Eric S. Raymond}{esr@thyrsus.com}
\sectionauthor{Christopher G. Petrilli}{petrilli@amber.org}

This module defines the class \class{ConfigParser}.
\indexii{.ini}{file}\indexii{configuration}{file}\index{ini file}
\index{Windows ini file}
The \class{ConfigParser} class implements a basic configuration file
parser language which provides a structure similar to what you would
find on Microsoft Windows INI files.  (Beware: this library does \emph{not}
interpret or write the value-type prefixes used in the Windows
Registry extended version of INI syntax.)  You can use this to write Python
programs which can be customized by end users easily.

The configuration file consists of sections, led by a
\samp{[section]} header and followed by \samp{name: value} entries,
with continuations in the style of \rfc{822}; \samp{name=value} is
also accepted.  Note that leading whitespace is removed from values.
The optional values can contain format strings which refer to other
values in the same section, or values in a special
\code{DEFAULT} section.  Additional defaults can be provided on
initialization and retrieval.  Lines beginning with \character{\#} or
\character{;} are ignored and may be used to provide comments.

For example:

\begin{verbatim}
[My Section]
foodir: %(dir)s/whatever
dir=frob
\end{verbatim}

would resolve the \samp{\%(dir)s} to the value of
\samp{dir} (\samp{frob} in this case).  All reference expansions are
done on demand.

Default values can be specified by passing them into the
\class{ConfigParser} constructor as a dictionary.  Additional defaults 
may be passed into the \method{get()} method which will override all
others.

\begin{classdesc}{RawConfigParser}{\optional{defaults}}
The basic configuration object.  When \var{defaults} is given, it is
initialized into the dictionary of intrinsic defaults.  This class
does not support the magical interpolation behavior.
\versionadded{2.3}
\end{classdesc}

\begin{classdesc}{ConfigParser}{\optional{defaults}}
Derived class of \class{RawConfigParser} that implements the magical
interpolation feature and adds optional arguments the \method{get()}
and \method{items()} methods.  The values in \var{defaults} must be
appropriate for the \samp{\%()s} string interpolation.  Note that
\var{__name__} is an intrinsic default; its value is the section name,
and will override any value provided in \var{defaults}.
\end{classdesc}

\begin{classdesc}{SafeConfigParser}{\optional{defaults}}
Derived class of \class{ConfigParser} that implements a more-sane
variant of the magical interpolation feature.  This implementation is
more predictable as well.
% XXX Need to explain what's safer/more predictable about it.
New applications should prefer this version if they don't need to be
compatible with older versions of Python.
\versionadded{2.3}
\end{classdesc}

\begin{excdesc}{NoSectionError}
Exception raised when a specified section is not found.
\end{excdesc}

\begin{excdesc}{DuplicateSectionError}
Exception raised when multiple sections with the same name are found,
or if \method{add_section()} is called with the name of a section that 
is already present.
\end{excdesc}

\begin{excdesc}{NoOptionError}
Exception raised when a specified option is not found in the specified 
section.
\end{excdesc}

\begin{excdesc}{InterpolationError}
Exception raised when problems occur performing string interpolation.
\end{excdesc}

\begin{excdesc}{InterpolationDepthError}
Exception raised when string interpolation cannot be completed because
the number of iterations exceeds \constant{MAX_INTERPOLATION_DEPTH}.
\end{excdesc}

\begin{excdesc}{MissingSectionHeaderError}
Exception raised when attempting to parse a file which has no section
headers.
\end{excdesc}

\begin{excdesc}{ParsingError}
Exception raised when errors occur attempting to parse a file.
\end{excdesc}

\begin{datadesc}{MAX_INTERPOLATION_DEPTH}
The maximum depth for recursive interpolation for \method{get()} when
the \var{raw} parameter is false.  This is relevant only for the
\class{ConfigParser} class.
\end{datadesc}


\begin{seealso}
  \seemodule{shlex}{Support for a creating \UNIX{} shell-like
                    minilanguages which can be used as an alternate format
                    for application configuration files.}
\end{seealso}


\subsection{RawConfigParser Objects \label{RawConfigParser-objects}}

\class{RawConfigParser} instances have the following methods:

\begin{methoddesc}{defaults}{}
Return a dictionary containing the instance-wide defaults.
\end{methoddesc}

\begin{methoddesc}{sections}{}
Return a list of the sections available; \code{DEFAULT} is not
included in the list.
\end{methoddesc}

\begin{methoddesc}{add_section}{section}
Add a section named \var{section} to the instance.  If a section by
the given name already exists, \exception{DuplicateSectionError} is
raised.
\end{methoddesc}

\begin{methoddesc}{has_section}{section}
Indicates whether the named section is present in the
configuration. The \code{DEFAULT} section is not acknowledged.
\end{methoddesc}

\begin{methoddesc}{options}{section}
Returns a list of options available in the specified \var{section}.
\end{methoddesc}

\begin{methoddesc}{has_option}{section, option}
If the given section exists, and contains the given option. return 1;
otherwise return 0.
\versionadded{1.6}
\end{methoddesc}

\begin{methoddesc}{read}{filenames}
Read and parse a list of filenames.  If \var{filenames} is a string or
Unicode string, it is treated as a single filename.
If a file named in \var{filenames} cannot be opened, that file will be
ignored.  This is designed so that you can specify a list of potential
configuration file locations (for example, the current directory, the
user's home directory, and some system-wide directory), and all
existing configuration files in the list will be read.  If none of the
named files exist, the \class{ConfigParser} instance will contain an
empty dataset.  An application which requires initial values to be
loaded from a file should load the required file or files using
\method{readfp()} before calling \method{read()} for any optional
files:

\begin{verbatim}
import ConfigParser, os

config = ConfigParser.ConfigParser()
config.readfp(open('defaults.cfg'))
config.read(['site.cfg', os.path.expanduser('~/.myapp.cfg')])
\end{verbatim}
\end{methoddesc}

\begin{methoddesc}{readfp}{fp\optional{, filename}}
Read and parse configuration data from the file or file-like object in
\var{fp} (only the \method{readline()} method is used).  If
\var{filename} is omitted and \var{fp} has a \member{name} attribute,
that is used for \var{filename}; the default is \samp{<???>}.
\end{methoddesc}

\begin{methoddesc}{get}{section, option}
Get an \var{option} value for the named \var{section}.
\end{methoddesc}

\begin{methoddesc}{getint}{section, option}
A convenience method which coerces the \var{option} in the specified
\var{section} to an integer.
\end{methoddesc}

\begin{methoddesc}{getfloat}{section, option}
A convenience method which coerces the \var{option} in the specified
\var{section} to a floating point number.
\end{methoddesc}

\begin{methoddesc}{getboolean}{section, option}
A convenience method which coerces the \var{option} in the specified
\var{section} to a Boolean value.  Note that the accepted values
for the option are \code{1}, \code{yes}, \code{true}, and \code{on},
which cause this method to return true, and \code{0}, \code{no},
\code{false}, and \code{off}, which cause it to return false.  These
values are checked in a case-insensitive manner.  Any other value will
cause it to raise \exception{ValueError}.
\end{methoddesc}

\begin{methoddesc}{items}{section}
Return a list of \code{(\var{name}, \var{value})} pairs for each
option in the given \var{section}.
\end{methoddesc}

\begin{methoddesc}{set}{section, option, value}
If the given section exists, set the given option to the specified value;
otherwise raise \exception{NoSectionError}.
\versionadded{1.6}
\end{methoddesc}

\begin{methoddesc}{write}{fileobject}
Write a representation of the configuration to the specified file
object.  This representation can be parsed by a future \method{read()}
call.
\versionadded{1.6}
\end{methoddesc}

\begin{methoddesc}{remove_option}{section, option}
Remove the specified \var{option} from the specified \var{section}.
If the section does not exist, raise \exception{NoSectionError}. 
If the option existed to be removed, return 1; otherwise return 0.
\versionadded{1.6}
\end{methoddesc}

\begin{methoddesc}{remove_section}{section}
Remove the specified \var{section} from the configuration.
If the section in fact existed, return \code{True}.
Otherwise return \code{False}.
\end{methoddesc}

\begin{methoddesc}{optionxform}{option}
Transforms the option name \var{option} as found in an input file or
as passed in by  client code to the form that should be used in the
internal structures.  The default implementation returns a lower-case
version of \var{option}; subclasses may override this or client code
can set an attribute of this name on instances to affect this
behavior.  Setting this to \function{str()}, for example, would make
option names case sensitive.
\end{methoddesc}


\subsection{ConfigParser Objects \label{ConfigParser-objects}}

The \class{ConfigParser} class extends some methods of the
\class{RawConfigParser} interface, adding some optional arguments.

\begin{methoddesc}{get}{section, option\optional{, raw\optional{, vars}}}
Get an \var{option} value for the named \var{section}.  All the
\character{\%} interpolations are expanded in the return values, based
on the defaults passed into the constructor, as well as the options
\var{vars} provided, unless the \var{raw} argument is true.
\end{methoddesc}

\begin{methoddesc}{items}{section\optional{, raw\optional{, vars}}}
Create a generator which will return a tuple \code{(name, value)} for
each option in the given \var{section}. Optional arguments have the
same meaning as for the \code{get()} method.
\versionadded{2.3}
\end{methoddesc}
