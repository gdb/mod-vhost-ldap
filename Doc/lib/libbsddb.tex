\section{\module{bsddb} ---
         Interface to Berkeley DB library}

\declaremodule{extension}{bsddb}
  \platform{Unix, Windows}
\modulesynopsis{Interface to Berkeley DB database library}
\sectionauthor{Skip Montanaro}{skip@mojam.com}


The \module{bsddb} module provides an interface to the Berkeley DB library.
Users can create hash, btree or record based library files using the
appropriate open call. Bsddb objects behave generally like dictionaries.
Keys and values must be strings, however, so to use other objects as keys or 
to store other kinds of objects the user must serialize them somehow,
typically using marshal.dumps or pickle.dumps.

The \module{bsddb} module is only available on \UNIX{} systems, so it is not
built by default in the standard Python distribution.  Also, there are two
incompatible versions of the underlying library.  Version 1.85 is widely
available, but has some known bugs.  Version 2 is not quite as widely used,
but does offer some improvements.  The \module{bsddb} module uses the 1.85
interface.  Users wishing to use version 2 of the Berkeley DB library will
have to modify the source for the module to include db_185.h instead of
db.h.

The \module{bsddb} module defines the following functions that create
objects that access the appropriate type of Berkeley DB file.  The first two
arguments of each function are the same.  For ease of portability, only the
first two arguments should be used in most instances.

\begin{funcdesc}{hashopen}{filename\optional{, flag\optional{,
mode\optional{, bsize\optional{, ffactor\optional{, nelem\optional{,
cachesize\optional{, hash\optional{, lorder}}}}}}}}}
Open the hash format file named \var{filename}.  The optional \var{flag}
identifies the mode used to open the file.  It may be ``r'' (read only),
``w'' (read-write), ``c'' (read-write - create if necessary) or ``n''
(read-write - truncate to zero length).  The other arguments are rarely used
and are just passed to the low-level dbopen function.  Consult the
Berkeley DB documentation for their use and interpretation.
\end{funcdesc}


\begin{funcdesc}{btopen}{filename\optional{, flag\optional{,
mode\optional{, btflags\optional{, cachesize\optional{, maxkeypage\optional{,
minkeypage\optional{, psize\optional{, lorder}}}}}}}}}
Open the btree format file named \var{filename}.  The optional \var{flag}
identifies the mode used to open the file.  It may be ``r'' (read only),
``w'' (read-write), ``c'' (read-write - create if necessary) or ``n''
(read-write - truncate to zero length).  The other arguments are rarely used
and are just passed to the low-level dbopen function.  Consult the
Berkeley DB documentation for their use and interpretation.
\end{funcdesc}

\begin{funcdesc}{rnopen}{filename\optional{, flag\optional{, mode\optional{,
rnflags\optional{, cachesize\optional{, psize\optional{, lorder\optional{,
reclen\optional{, bval\optional{, bfname}}}}}}}}}}
Open a DB record format file named \var{filename}.  The optional \var{flag}
identifies the mode used to open the file.  It may be ``r'' (read only),
``w'' (read-write), ``c'' (read-write - create if necessary) or ``n''
(read-write - truncate to zero length).  The other arguments are rarely used
and are just passed to the low-level dbopen function.  Consult the
Berkeley DB documentation for their use and interpretation.
\end{funcdesc}


\begin{seealso}
  \seemodule{dbhash}{DBM-style interface to the \module{bsddb}}
\end{seealso}


\subsection{Hash, BTree and Record Objects \label{bsddb-objects}}

Once instantiated, hash, btree and record objects support the following
methods:

\begin{methoddesc}{close}{}
Close the underlying file.  The object can no longer be accessed.  Since
there is no open \method{open} method for these objects, to open the file
again a new \module{bsddb} module open function must be called.
\end{methoddesc}

\begin{methoddesc}{keys}{}
Return the list of keys contained in the DB file.  The order of the list is
unspecified and should not be relied on.  In particular, the order of the
list returned is different for different file formats.
\end{methoddesc}

\begin{methoddesc}{has_key}{key}
Return 1 if the DB file contains the argument as a key.
\end{methoddesc}

\begin{methoddesc}{set_location}{key}
Set the cursor to the item indicated by the key and return it.
\end{methoddesc}

\begin{methoddesc}{first}{}
Set the cursor to the first item in the DB file and return it.  The order of 
keys in the file is unspecified.
\end{methoddesc}

\begin{methoddesc}{next}{}
Set the cursor to the next item in the DB file and return it.  The order of 
keys in the file is unspecified.
\end{methoddesc}

\begin{methoddesc}{previous}{}
Set the cursor to the first item in the DB file and return it.  The order of 
keys in the file is unspecified.
\end{methoddesc}

\begin{methoddesc}{last}{}
Set the cursor to the last item in the DB file and return it.  The order of 
keys in the file is unspecified.
\end{methoddesc}

\begin{methoddesc}{sync}{}
Synchronize the database on disk.
\end{methoddesc}

Example:

\begin{verbatim}
>>> import bsddb
>>> db = bsddb.btopen('/tmp/spam.db', 'c')
>>> for i in range(10): db['%d'%i] = '%d'% (i*i)
... 
>>> db['3']
'9'
>>> db.keys()
['0', '1', '2', '3', '4', '5', '6', '7', '8', '9']
>>> db.first()
('0', '0')
>>> db.next()
('1', '1')
>>> db.last()
('9', '81')
>>> db.set_location('2')
('2', '4')
>>> db.previous() 
('1', '1')
>>> db.sync()
0
\end{verbatim}
