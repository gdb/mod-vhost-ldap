\section{\module{bsddb} ---
         Interface to Berkeley DB library}

\declaremodule{extension}{bsddb}
\modulesynopsis{Interface to Berkeley DB database library}
\sectionauthor{Skip Montanaro}{skip@mojam.com}


The \module{bsddb} module provides an interface to the Berkeley DB
library.  Users can create hash, btree or record based library files
using the appropriate open call. Bsddb objects behave generally like
dictionaries.  Keys and values must be strings, however, so to use
other objects as keys or to store other kinds of objects the user must
serialize them somehow, typically using \function{marshal.dumps()} or 
\function{pickle.dumps()}.

The \module{bsddb} module requires a Berkeley DB library version from
3.3 thru 4.5.

\begin{seealso}
  \seeurl{http://pybsddb.sourceforge.net/}
         {The website with documentation for the \module{bsddb.db}
          Python Berkeley DB interface that closely mirrors the object
          oriented interface provided in Berkeley DB 3 and 4.}

  \seeurl{http://www.oracle.com/database/berkeley-db/}
         {The Berkeley DB library.}
\end{seealso}

A more modern DB, DBEnv and DBSequence object interface is available in the
\module{bsddb.db} module which closely matches the Berkeley DB C API
documented at the above URLs.  Additional features provided by the
\module{bsddb.db} API include fine tuning, transactions, logging, and
multiprocess concurrent database access.

The following is a description of the legacy \module{bsddb} interface
compatible with the old Python bsddb module.  Starting in Python 2.5 this
interface should be safe for multithreaded access.  The \module{bsddb.db}
API is recommended for threading users as it provides better control.

The \module{bsddb} module defines the following functions that create
objects that access the appropriate type of Berkeley DB file.  The
first two arguments of each function are the same.  For ease of
portability, only the first two arguments should be used in most
instances.

\begin{funcdesc}{hashopen}{filename\optional{, flag\optional{,
                           mode\optional{, pgsize\optional{,
                           ffactor\optional{, nelem\optional{,
                           cachesize\optional{, lorder\optional{,
                           hflags}}}}}}}}}
Open the hash format file named \var{filename}.  Files never intended
to be preserved on disk may be created by passing \code{None} as the 
\var{filename}.  The optional
\var{flag} identifies the mode used to open the file.  It may be
\character{r} (read only), \character{w} (read-write) ,
\character{c} (read-write - create if necessary; the default) or
\character{n} (read-write - truncate to zero length).  The other
arguments are rarely used and are just passed to the low-level
\cfunction{dbopen()} function.  Consult the Berkeley DB documentation
for their use and interpretation.
\end{funcdesc}

\begin{funcdesc}{btopen}{filename\optional{, flag\optional{,
mode\optional{, btflags\optional{, cachesize\optional{, maxkeypage\optional{,
minkeypage\optional{, pgsize\optional{, lorder}}}}}}}}}

Open the btree format file named \var{filename}.  Files never intended 
to be preserved on disk may be created by passing \code{None} as the 
\var{filename}.  The optional
\var{flag} identifies the mode used to open the file.  It may be
\character{r} (read only), \character{w} (read-write),
\character{c} (read-write - create if necessary; the default) or
\character{n} (read-write - truncate to zero length).  The other
arguments are rarely used and are just passed to the low-level dbopen
function.  Consult the Berkeley DB documentation for their use and
interpretation.
\end{funcdesc}

\begin{funcdesc}{rnopen}{filename\optional{, flag\optional{, mode\optional{,
rnflags\optional{, cachesize\optional{, pgsize\optional{, lorder\optional{,
rlen\optional{, delim\optional{, source\optional{, pad}}}}}}}}}}}

Open a DB record format file named \var{filename}.  Files never intended 
to be preserved on disk may be created by passing \code{None} as the 
\var{filename}.  The optional
\var{flag} identifies the mode used to open the file.  It may be
\character{r} (read only), \character{w} (read-write),
\character{c} (read-write - create if necessary; the default) or
\character{n} (read-write - truncate to zero length).  The other
arguments are rarely used and are just passed to the low-level dbopen
function.  Consult the Berkeley DB documentation for their use and
interpretation.
\end{funcdesc}


\begin{notice}
Beginning in 2.3 some \UNIX{} versions of Python may have a \module{bsddb185}
module.  This is present \emph{only} to allow backwards compatibility with
systems which ship with the old Berkeley DB 1.85 database library.  The
\module{bsddb185} module should never be used directly in new code.
\end{notice}


\begin{seealso}
  \seemodule{dbhash}{DBM-style interface to the \module{bsddb}}
\end{seealso}

\subsection{Hash, BTree and Record Objects \label{bsddb-objects}}

Once instantiated, hash, btree and record objects support
the same methods as dictionaries.  In addition, they support
the methods listed below.
\versionchanged[Added dictionary methods]{2.3.1}

\begin{methoddesc}[bsddbobject]{close}{}
Close the underlying file.  The object can no longer be accessed.  Since
there is no open \method{open} method for these objects, to open the file
again a new \module{bsddb} module open function must be called.
\end{methoddesc}

\begin{methoddesc}[bsddbobject]{keys}{}
Return the list of keys contained in the DB file.  The order of the list is
unspecified and should not be relied on.  In particular, the order of the
list returned is different for different file formats.
\end{methoddesc}

\begin{methoddesc}[bsddbobject]{has_key}{key}
Return \code{1} if the DB file contains the argument as a key.
\end{methoddesc}

\begin{methoddesc}[bsddbobject]{set_location}{key}
Set the cursor to the item indicated by \var{key} and return a tuple
containing the key and its value.  For binary tree databases (opened
using \function{btopen()}), if \var{key} does not actually exist in
the database, the cursor will point to the next item in sorted order
and return that key and value.  For other databases,
\exception{KeyError} will be raised if \var{key} is not found in the
database.
\end{methoddesc}

\begin{methoddesc}[bsddbobject]{first}{}
Set the cursor to the first item in the DB file and return it.  The order of 
keys in the file is unspecified, except in the case of B-Tree databases.
This method raises \exception{bsddb.error} if the database is empty.
\end{methoddesc}

\begin{methoddesc}[bsddbobject]{next}{}
Set the cursor to the next item in the DB file and return it.  The order of 
keys in the file is unspecified, except in the case of B-Tree databases.
\end{methoddesc}

\begin{methoddesc}[bsddbobject]{previous}{}
Set the cursor to the previous item in the DB file and return it.  The
order of keys in the file is unspecified, except in the case of B-Tree
databases.  This is not supported on hashtable databases (those opened
with \function{hashopen()}).
\end{methoddesc}

\begin{methoddesc}[bsddbobject]{last}{}
Set the cursor to the last item in the DB file and return it.  The
order of keys in the file is unspecified.  This is not supported on
hashtable databases (those opened with \function{hashopen()}).
This method raises \exception{bsddb.error} if the database is empty.
\end{methoddesc}

\begin{methoddesc}[bsddbobject]{sync}{}
Synchronize the database on disk.
\end{methoddesc}

Example:

\begin{verbatim}
>>> import bsddb
>>> db = bsddb.btopen('/tmp/spam.db', 'c')
>>> for i in range(10): db['%d'%i] = '%d'% (i*i)
... 
>>> db['3']
'9'
>>> db.keys()
['0', '1', '2', '3', '4', '5', '6', '7', '8', '9']
>>> db.first()
('0', '0')
>>> db.next()
('1', '1')
>>> db.last()
('9', '81')
>>> db.set_location('2')
('2', '4')
>>> db.previous() 
('1', '1')
>>> for k, v in db.iteritems():
...     print k, v
0 0
1 1
2 4
3 9
4 16
5 25
6 36
7 49
8 64
9 81
>>> '8' in db
True
>>> db.sync()
0
\end{verbatim}
