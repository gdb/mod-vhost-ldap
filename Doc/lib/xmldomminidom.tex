\section{\module{xml.dom.minidom} ---
         Lightweight DOM implementation}

\declaremodule{standard}{xml.dom.minidom}
\modulesynopsis{Lightweight Document Object Model (DOM) implementation.}
\moduleauthor{Paul Prescod}{paul@prescod.net}
\sectionauthor{Paul Prescod}{paul@prescod.net}
\sectionauthor{Martin v. L\"owis}{loewis@informatik.hu-berlin.de}

\versionadded{2.0}

\module{xml.dom.minidom} is a light-weight implementation of the
Document Object Model interface.  It is intended to be
simpler than the full DOM and also significantly smaller.

DOM applications typically start by parsing some XML into a DOM.  With
\module{xml.dom.minidom}, this is done through the parse functions:

\begin{verbatim}
from xml.dom.minidom import parse, parseString

dom1 = parse('c:\\temp\\mydata.xml') # parse an XML file by name

datasource = open('c:\\temp\\mydata.xml')
dom2 = parse(datasource)   # parse an open file

dom3 = parseString('<myxml>Some data<empty/> some more data</myxml>')
\end{verbatim}

The parse function can take either a filename or an open file object. 

\begin{funcdesc}{parse}{filename_or_file{, parser}}
  Return a \class{Document} from the given input. \var{filename_or_file}
  may be either a file name, or a file-like object. \var{parser}, if
  given, must be a SAX2 parser object. This function will change the
  document handler of the parser and activate namespace support; other
  parser configuration (like setting an entity resolver) must have been
  done in advance.
\end{funcdesc}

If you have XML in a string, you can use the
\function{parseString()} function instead:

\begin{funcdesc}{parseString}{string\optional{, parser}}
  Return a \class{Document} that represents the \var{string}. This
  method creates a \class{StringIO} object for the string and passes
  that on to \function{parse}.
\end{funcdesc}

Both functions return a \class{Document} object representing the
content of the document.

You can also create a \class{Document} node merely by instantiating a 
document object.  Then you could add child nodes to it to populate
the DOM:

\begin{verbatim}
from xml.dom.minidom import Document

newdoc = Document()
newel = newdoc.createElement("some_tag")
newdoc.appendChild(newel)
\end{verbatim}

Once you have a DOM document object, you can access the parts of your
XML document through its properties and methods.  These properties are
defined in the DOM specification.  The main property of the document
object is the \member{documentElement} property.  It gives you the
main element in the XML document: the one that holds all others.  Here
is an example program:

\begin{verbatim}
dom3 = parseString("<myxml>Some data</myxml>")
assert dom3.documentElement.tagName == "myxml"
\end{verbatim}

When you are finished with a DOM, you should clean it up.  This is
necessary because some versions of Python do not support garbage
collection of objects that refer to each other in a cycle.  Until this
restriction is removed from all versions of Python, it is safest to
write your code as if cycles would not be cleaned up.

The way to clean up a DOM is to call its \method{unlink()} method:

\begin{verbatim}
dom1.unlink()
dom2.unlink()
dom3.unlink()
\end{verbatim}

\method{unlink()} is a \module{xml.dom.minidom}-specific extension to
the DOM API.  After calling \method{unlink()} on a node, the node and
its descendents are essentially useless.

\begin{seealso}
  \seetitle[http://www.w3.org/TR/REC-DOM-Level-1/]{Document Object
            Model (DOM) Level 1 Specification}
           {The W3C recommendation for the
            DOM supported by \module{xml.dom.minidom}.}
\end{seealso}


\subsection{DOM objects \label{dom-objects}}

The definition of the DOM API for Python is given as part of the
\refmodule{xml.dom} module documentation.  This section lists the
differences between the API and \refmodule{xml.dom.minidom}.


\begin{methoddesc}{unlink}{}
Break internal references within the DOM so that it will be garbage
collected on versions of Python without cyclic GC.  Even when cyclic
GC is available, using this can make large amounts of memory available
sooner, so calling this on DOM objects as soon as they are no longer
needed is good practice.  This only needs to be called on the
\class{Document} object, but may be called on child nodes to discard
children of that node.
\end{methoddesc}

\begin{methoddesc}{writexml}{writer}
Write XML to the writer object.  The writer should have a
\method{write()} method which matches that of the file object
interface.
\end{methoddesc}

\begin{methoddesc}{toxml}{}
Return the XML that the DOM represents as a string.
\end{methoddesc}

The following standard DOM methods have special considerations with
\refmodule{xml.dom.minidom}:

\begin{methoddesc}{cloneNode}{deep}
Although this method was present in the version of
\refmodule{xml.dom.minidom} packaged with Python 2.0, it was seriously
broken.  This has been corrected for subsequent releases.
\end{methoddesc}


\subsection{DOM Example \label{dom-example}}

This example program is a fairly realistic example of a simple
program. In this particular case, we do not take much advantage
of the flexibility of the DOM.

\begin{verbatim}
import xml.dom.minidom

document = """\
<slideshow>
<title>Demo slideshow</title>
<slide><title>Slide title</title>
<point>This is a demo</point>
<point>Of a program for processing slides</point>
</slide>

<slide><title>Another demo slide</title>
<point>It is important</point>
<point>To have more than</point>
<point>one slide</point>
</slide>
</slideshow>
"""

dom = xml.dom.minidom.parseString(document)

space = " "
def getText(nodelist):
    rc = ""
    for node in nodelist:
        if node.nodeType == node.TEXT_NODE:
            rc = rc + node.data
    return rc

def handleSlideshow(slideshow):
    print "<html>"
    handleSlideshowTitle(slideshow.getElementsByTagName("title")[0])
    slides = slideshow.getElementsByTagName("slide")
    handleToc(slides)
    handleSlides(slides)
    print "</html>"

def handleSlides(slides):
    for slide in slides:
       handleSlide(slide)

def handleSlide(slide):
    handleSlideTitle(slide.getElementsByTagName("title")[0])
    handlePoints(slide.getElementsByTagName("point"))

def handleSlideshowTitle(title):
    print "<title>%s</title>" % getText(title.childNodes)

def handleSlideTitle(title):
    print "<h2>%s</h2>" % getText(title.childNodes)

def handlePoints(points):
    print "<ul>"
    for point in points:
        handlePoint(point)
    print "</ul>"

def handlePoint(point):
    print "<li>%s</li>" % getText(point.childNodes)

def handleToc(slides):
    for slide in slides:
        title = slide.getElementsByTagName("title")[0]
        print "<p>%s</p>" % getText(title.childNodes)

handleSlideshow(dom)
\end{verbatim}


\subsection{minidom and the DOM standard \label{minidom-and-dom}}

The \refmodule{xml.dom.minidom} module is essentially a DOM
1.0-compatible DOM with some DOM 2 features (primarily namespace
features).

Usage of the DOM interface in Python is straight-forward.  The
following mapping rules apply:

\begin{itemize}
\item Interfaces are accessed through instance objects. Applications
      should not instantiate the classes themselves; they should use
      the creator functions available on the \class{Document} object.
      Derived interfaces support all operations (and attributes) from
      the base interfaces, plus any new operations.

\item Operations are used as methods. Since the DOM uses only
      \keyword{in} parameters, the arguments are passed in normal
      order (from left to right).   There are no optional
      arguments. \keyword{void} operations return \code{None}.

\item IDL attributes map to instance attributes. For compatibility
      with the OMG IDL language mapping for Python, an attribute
      \code{foo} can also be accessed through accessor methods
      \method{_get_foo()} and \method{_set_foo()}.  \keyword{readonly}
      attributes must not be changed; this is not enforced at
      runtime.

\item The types \code{short int}, \code{unsigned int}, \code{unsigned
      long long}, and \code{boolean} all map to Python integer
      objects.

\item The type \code{DOMString} maps to Python strings.
      \refmodule{xml.dom.minidom} supports either byte or Unicode
      strings, but will normally produce Unicode strings.  Attributes
      of type \code{DOMString} may also be \code{None}.

\item \keyword{const} declarations map to variables in their
      respective scope
      (e.g. \code{xml.dom.minidom.Node.PROCESSING_INSTRUCTION_NODE});
      they must not be changed.

\item \code{DOMException} is currently not supported in
      \refmodule{xml.dom.minidom}.  Instead,
      \refmodule{xml.dom.minidom} uses standard Python exceptions such
      as \exception{TypeError} and \exception{AttributeError}.

\item \class{NodeList} objects are implemented as Python's built-in
      list type, so don't support the official API, but are much more
      ``Pythonic.''
\end{itemize}


The following interfaces have no implementation in
\refmodule{xml.dom.minidom}:

\begin{itemize}
\item DOMTimeStamp

\item DocumentType (added in Python 2.1)

\item DOMImplementation (added in Python 2.1)

\item CharacterData

\item CDATASection

\item Notation

\item Entity

\item EntityReference

\item DocumentFragment
\end{itemize}

Most of these reflect information in the XML document that is not of
general utility to most DOM users.
