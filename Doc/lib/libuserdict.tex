\section{\module{UserDict} ---
         Class wrapper for dictionary objects}

\declaremodule{standard}{UserDict}
\modulesynopsis{Class wrapper for dictionary objects.}

This module defines a class that acts as a wrapper around
dictionary objects.  It is a useful base class for
your own dictionary-like classes, which can inherit from
them and override existing methods or add new ones.  In this way one
can add new behaviours to dictionaries.

The \module{UserDict} module defines the \class{UserDict} class:

\begin{classdesc}{UserDict}{\optional{intialdata}}
Return a class instance that simulates a dictionary.  The instance's
contents are kept in a regular dictionary, which is accessible via the
\member{data} attribute of \class{UserDict} instances.  If
\var{initialdata} is provided, \member{data} is initialized with its
contents; note that a reference to \var{initialdata} will not be kept, 
allowing it be used used for other purposes.
\end{classdesc}

In addition to supporting the methods and operations of mappings (see
section \ref{typesmapping}), \class{UserDict} instances provide the
following attribute:

\begin{memberdesc}{data}
A real dictionary used to store the contents of the \class{UserDict}
class.
\end{memberdesc}


\section{\module{UserList} ---
         Class wrapper for list objects}

\declaremodule{standard}{UserList}
\modulesynopsis{Class wrapper for list objects.}


This module defines a class that acts as a wrapper around
list objects.  It is a useful base class for
your own list-like classes, which can inherit from
them and override existing methods or add new ones.  In this way one
can add new behaviours to lists.

The \module{UserList} module defines the \class{UserList} class:

\begin{classdesc}{UserList}{\optional{list}}
Return a class instance that simulates a list.  The instance's
contents are kept in a regular list, which is accessible via the
\member{data} attribute of \class{UserList} instances.  The instance's
contents are initially set to a copy of \var{list}, defaulting to the
empty list \code{[]}.  \var{list} can be either a regular Python list,
or an instance of \class{UserList} (or a subclass).
\end{classdesc}

In addition to supporting the methods and operations of mutable
sequences (see section \ref{typesseq}), \class{UserList} instances
provide the following attribute:

\begin{memberdesc}{data}
A real Python list object used to store the contents of the
\class{UserList} class.
\end{memberdesc}
