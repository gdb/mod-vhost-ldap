\section{\module{grp} ---
         The group database.}
\declaremodule{builtin}{grp}


\modulesynopsis{The group database (\function{getgrnam()} and friends).}

This module provides access to the \UNIX{} group database.
It is available on all \UNIX{} versions.

Group database entries are reported as 4-tuples containing the
following items from the group database (see \code{<grp.h>}), in order:
\code{gr_name},
\code{gr_passwd},
\code{gr_gid},
\code{gr_mem}.
The gid is an integer, name and password are strings, and the member
list is a list of strings.
(Note that most users are not explicitly listed as members of the
group they are in according to the password database.)
A \code{KeyError} exception is raised if the entry asked for cannot be found.

It defines the following items:

\begin{funcdesc}{getgrgid}{gid}
Return the group database entry for the given numeric group ID.
\end{funcdesc}

\begin{funcdesc}{getgrnam}{name}
Return the group database entry for the given group name.
\end{funcdesc}

\begin{funcdesc}{getgrall}{}
Return a list of all available group entries, in arbitrary order.
\end{funcdesc}
