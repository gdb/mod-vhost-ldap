\section{Built-in Module \module{jpeg}}
\label{module-jpeg}
\bimodindex{jpeg}

The module \module{jpeg} provides access to the jpeg compressor and
decompressor written by the Independent JPEG Group%
\index{Independent JPEG Group}%
. JPEG is a (draft?)
standard for compressing pictures.  For details on JPEG or the
Independent JPEG Group software refer to the JPEG standard or the
documentation provided with the software.

The \module{jpeg} module defines an exception and some functions.

\begin{excdesc}{error}
Exception raised by \function{compress()} and \function{decompress()}
in case of errors.
\end{excdesc}

\begin{funcdesc}{compress}{data, w, h, b}
Treat data as a pixmap of width \var{w} and height \var{h}, with
\var{b} bytes per pixel.  The data is in SGI GL order, so the first
pixel is in the lower-left corner. This means that \function{gl.lrectread()}
return data can immediately be passed to \function{compress()}.
Currently only 1 byte and 4 byte pixels are allowed, the former being
treated as greyscale and the latter as RGB color.
\function{compress()} returns a string that contains the compressed
picture, in JFIF\index{JFIF} format.
\end{funcdesc}

\begin{funcdesc}{decompress}{data}
Data is a string containing a picture in JFIF\index{JFIF} format. It
returns a tuple \code{(\var{data}, \var{width}, \var{height},
\var{bytesperpixel})}.  Again, the data is suitable to pass to
\function{gl.lrectwrite()}.
\end{funcdesc}

\begin{funcdesc}{setoption}{name, value}
Set various options.  Subsequent \function{compress()} and
\function{decompress()} calls will use these options.  The following
options are available:

\begin{tableii}{l|p{3in}}{code}{Option}{Effect}
  \lineii{'forcegray'}{%
    Force output to be grayscale, even if input is RGB.}
  \lineii{'quality'}{%
    Set the quality of the compressed image to a value between
    \code{0} and \code{100} (default is \code{75}).  This only affects
    compression.}
  \lineii{'optimize'}{%
    Perform Huffman table optimization.  Takes longer, but results in
    smaller compressed image.  This only affects compression.}
  \lineii{'smooth'}{%
    Perform inter-block smoothing on uncompressed image.  Only useful
    for low-quality images.  This only affects decompression.}
\end{tableii}
\end{funcdesc}
