\section{\module{smtplib} ---
         SMTP protocol client.}
\declaremodule{standard}{smtplib}
\sectionauthor{Eric S. Raymond}{esr@snark.thyrsus.com}

\modulesynopsis{SMTP protocol client (requires sockets).}

\indexii{SMTP}{protocol}
\index{Simple Mail Transfer Protocol}

The \module{smtplib} module defines an SMTP client session object that
can be used to send mail to any Internet machine with an SMTP or ESMTP
listener daemon.  For details of SMTP and ESMTP operation, consult
\rfc{821} (\emph{Simple Mail Transfer Protocol}) and \rfc{1869}
(\emph{SMTP Service Extensions}).

\begin{classdesc}{SMTP}{\optional{host\optional{, port}}}
A \class{SMTP} instance encapsulates an SMTP connection.  It has
methods that support a full repertoire of SMTP and ESMTP
operations. If the optional host and port parameters are given, the
SMTP \method{connect()} method is called with those parameters during
initialization.

For normal use, you should only require the initialization/connect,
\method{sendmail()}, and \method{quit()} methods  An example is
included below.
\end{classdesc}


\subsection{SMTP Objects}
\label{SMTP-objects}

An \class{SMTP} instance has the following methods:

\begin{methoddesc}{set_debuglevel}{level}
Set the debug output level.  A true value for \var{level} results in
debug messages for connection and for all messages sent to and
received from the server.
\end{methoddesc}

\begin{methoddesc}{connect}{\optional{host\optional{, port}}}
Connect to a host on a given port.  The defaults are to connect to the 
local host at the standard SMTP port (25).

If the hostname ends with a colon (\character{:}) followed by a
number, that suffix will be stripped off and the number interpreted as 
the port number to use.

Note:  This method is automatically invoked by the constructor if a
host is specified during instantiation.
\end{methoddesc}

\begin{methoddesc}{docmd}{cmd, \optional{, argstring}}
Send a command \var{cmd} to the server.  The optional argument
\var{argstring} is simply concatenated to the command, separated by a
space.

This returns a 2-tuple composed of a numeric response code and the
actual response line (multiline responses are joined into one long
line.)

In normal operation it should not be necessary to call this method
explicitly.  It is used to implement other methods and may be useful
for testing private extensions.
\end{methoddesc}

\begin{methoddesc}{helo}{\optional{hostname}}
Identify yourself to the SMTP server using \samp{HELO}.  The hostname
argument defaults to the fully qualified domain name of the local
host.

In normal operation it should not be necessary to call this method
explicitly.  It will be implicitly called by the \method{sendmail()}
when necessary.
\end{methoddesc}

\begin{methoddesc}{ehlo}{\optional{hostname}}
Identify yourself to an ESMTP server using \samp{HELO}.  The hostname
argument defaults to the fully qualified domain name of the local
host.  Examine the response for ESMTP option and store them for use by
\method{has_option()}.

Unless you wish to use \method{has_option()} before sending
mail, it should not be necessary to call this method explicitly.  It
will be implicitly called by \method{sendmail()} when necessary.
\end{methoddesc}

\begin{methoddesc}{has_option}{name}
Return \code{1} if \var{name} is in the set of ESMTP options returned
by the server, \code{0} otherwise.  Case is ignored.
\end{methoddesc}

\begin{methoddesc}{verify}{address}
Check the validity of an address on this server using SMTP \samp{VRFY}.
Returns a tuple consisting of code 250 and a full \rfc{822} address
(including human name) if the user address is valid. Otherwise returns
an SMTP error code of 400 or greater and an error string.

Note: many sites disable SMTP \samp{VRFY} in order to foil spammers.
\end{methoddesc}

\begin{methoddesc}{sendmail}{from_addr, to_addrs, msg\optional{, options}}
Send mail.  The required arguments are an \rfc{822} from-address
string, a list of \rfc{822} to-address strings, and a message string.
The caller may pass a list of ESMTP options to be used in \samp{MAIL
FROM} commands as \var{options}.

If there has been no previous \samp{EHLO} or \samp{HELO} command this
session, this method tries ESMTP \samp{EHLO} first. If the server does
ESMTP, message size and each of the specified options will be passed
to it (if the option is in the feature set the server advertises).  If
\samp{EHLO} fails, \samp{HELO} will be tried and ESMTP options
suppressed.

This method will return normally if the mail is accepted for at least 
one recipient. Otherwise it will throw an exception (either
\exception{SMTPSenderRefused}, \exception{SMTPRecipientsRefused}, or
\exception{SMTPDataError}).  That is, if this method does not throw an
exception, then someone should get your mail.  If this method does not
throw an exception, it returns a dictionary, with one entry for each
recipient that was refused.
\end{methoddesc}

\begin{methoddesc}{quit}{}
Terminate the SMTP session and close the connection.
\end{methoddesc}

Low-level methods corresponding to the standard SMTP/ESMTP commands
\samp{HELP}, \samp{RSET}, \samp{NOOP}, \samp{MAIL}, \samp{RCPT}, and
\samp{DATA} are also supported.  Normally these do not need to be
called directly, so they are not documented here.  For details,
consult the module code.


\subsection{SMTP Example \label{SMTP-example}}

% really need a little description here...

\begin{verbatim}
import rfc822, string, sys

def prompt(prompt):
    sys.stdout.write(prompt + ": ")
    return string.strip(sys.stdin.readline())

fromaddr = prompt("From")
toaddrs  = string.splitfields(prompt("To"), ',')
print "Enter message, end with ^D:"
msg = ''
while 1:
    line = sys.stdin.readline()
    if not line:
        break
    msg = msg + line
print "Message length is " + `len(msg)`

server = SMTP('localhost')
server.set_debuglevel(1)
server.sendmail(fromaddr, toaddrs, msg)
server.quit()
\end{verbatim}


\begin{seealso}
\seetext{\rfc{821}, \emph{Simple Mail Transfer Protocol}.  Available
online at \url{http://info.internet.isi.edu/in-notes/rfc/files/rfc821.txt}.}

\seetext{\rfc{1869}, \emph{SMTP Service Extensions}.  Available online 
at \url{http://info.internet.isi.edu/in-notes/rfc/files/rfc1869.txt}.}
\end{seealso}
