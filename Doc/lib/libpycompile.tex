% Documentation based on module docstrings, by Fred L. Drake, Jr.
% <fdrake@acm.org>

\section{\module{py_compile} ---
         Compile Python source files.}

\declaremodule[pycompile]{standard}{py_compile}

\modulesynopsis{Compile Python source files to byte-code files.}


\indexii{file}{byte-code}
The \module{py_compile} module provides a single function to generate
a byte-code file from a source file.

Though not often needed, this function can be useful when installing
modules for shared use, especially if some of the users may not have
permission to write the byte-code cache files in the directory
containing the source code.


\begin{funcdesc}{compile}{file\optional{, cfile\optional{, dfile}}}
  Compile a source file to byte-code and write out the byte-code cache 
  file.  The source code is loaded from the file name \var{file}.  The 
  byte-code is written to \var{cfile}, which defaults to \var{file}
  \code{+} \code{'c'} (\code{'o'} if optimization is enabled in the
  current interpreter).  If \var{dfile} is specified, it is used as
  the name of the source file in error messages instead of \var{file}. 
\end{funcdesc}


\begin{seealso}
  \seemodule{compileall}{Utilities to compile all Python source files
                         in a directory tree.}
\end{seealso}
