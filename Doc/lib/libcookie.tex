\section{\module{Cookie} ---
         RFC2109 HTTP State Management (AKA Cookies) Support}
\declaremodule{standard}{Cookie}

\moduleauthor{Timothy O'Malley}{timo@alum.mit.edu}
\sectionauthor{Moshe Zadka}{moshez@zadka.site.co.il}

\modulesynopsis{Support HTTP State Management (Cookies)}

The \module{Cookie} module defines classes for abstracting the concept of 
Cookies, an HTTP state management mechanism. It supports both simplistic
string-only cookies, and provides an abstraction for having any serializable
data-type as cookie value.

\subsection{Example \label{cookie-example}}

The following example demonstrates how to open a can of spam using the
\module{spam} module.

\begin{verbatim}
   >>> import Cookie
   >>> C = Cookie.SimpleCookie()
   >>> C = Cookie.SerialCookie()
   >>> C = Cookie.SmartCookie()
   >>> C = Cookie.Cookie() # backwards compatible alias for SmartCookie
   >>> C = Cookie.SmartCookie()
   >>> C["fig"] = "newton"
   >>> C["sugar"] = "wafer"
   >>> C # generate HTTP headers
   Set-Cookie: sugar=wafer;
   Set-Cookie: fig=newton;
   >>> C = Cookie.SmartCookie()
   >>> C["rocky"] = "road"
   >>> C["rocky"]["path"] = "/cookie"
   >>> print C.output(header="Cookie:")
   Cookie: rocky=road; Path=/cookie;
   >>> print C.output(attrs=[], header="Cookie:")
   Cookie: rocky=road;
   >>> C = Cookie.SmartCookie()
   >>> C.load("chips=ahoy; vienna=finger") # load from a string (HTTP header)
   >>> C
   Set-Cookie: vienna=finger;
   Set-Cookie: chips=ahoy;
   >>> C = Cookie.SmartCookie()
   >>> C.load('keebler="E=everybody; L=\\"Loves\\"; fudge=\\012;";')
   >>> C
   Set-Cookie: keebler="E=everybody; L=\"Loves\"; fudge=\012;";
   >>> C = Cookie.SmartCookie()
   >>> C["oreo"] = "doublestuff"
   >>> C["oreo"]["path"] = "/"
   >>> C
   Set-Cookie: oreo="doublestuff"; Path=/;
   >>> C = Cookie.SmartCookie()
   >>> C["twix"] = "none for you"
   >>> C["twix"].value
   'none for you'
   >>> C = Cookie.SimpleCookie()
   >>> C["number"] = 7 # equivalent to C["number"] = str(7)
   >>> C["string"] = "seven"
   >>> C["number"].value
   '7'
   >>> C["string"].value
   'seven'
   >>> C
   Set-Cookie: number=7;
   Set-Cookie: string=seven;
   >>> C = Cookie.SerialCookie()
   >>> C["number"] = 7
   >>> C["string"] = "seven"
   >>> C["number"].value
   7
   >>> C["string"].value
   'seven'
   >>> C
   Set-Cookie: number="I7\012.";
   Set-Cookie: string="S'seven'\012p1\012.";
   >>> C = Cookie.SmartCookie()
   >>> C["number"] = 7
   >>> C["string"] = "seven"
   >>> C["number"].value
   7
   >>> C["string"].value
   'seven'
   >>> C
   Set-Cookie: number="I7\012.";
   Set-Cookie: string=seven;
\end{verbatim}

\begin{excdesc}{CookieError}
Exception failing because of RFC2109 invalidity: incorrect attributes,
incorrect \code{Set-Cookie} header, etc.
\end{excdesc}

%\subsection{Morsel Objects}
%\label{morsel-objects}

\begin{classdesc}{Morsel}{}
Abstract a key/value pair, which has some RFC2109 attributes.

Morsels are dictionary-like objects, whose set of keys is constant ---
the valid RFC2109 attributes, which are

\begin{itemize}
	\item \code{expires}
	\item \code{path}
	\item \code{comment}
	\item \code{domain}
	\item \code{max-age}
	\item \code{secure}
	\item \code{version}
	\end{itemize}
\end{itemize}

The keys are case-insensitive.
\end{classdesc}

\begin{memberdesc}[Morsel]{value}
The value of the cookie.
\end{methoddesc}

\begin{memberdesc}[Morsel]{coded_value}
The encoded value of the cookie --- this is what should be sent.
\end{methoddesc}


\begin{memberdesc}[Morsel]{key}
The name of the cookie.
\end{methoddesc}

\begin{methodesc}[Morsel]{set}{key, value, coded_value}
Set the \var{key}, \var{value} and \var{coded_value} members.
\end{methoddesc}

\begin{methoddesc}[Morsel]{isReservedKey}{K}
Whether \var{K} is a member of the set of keys of a \class{Morsel}.
\end{methoddesc}

\begin{methoddesc}[Morsel]{output}{\opt{attrs, \opt{header}}
Return a string representation of the Morsel, suitable
to be sent as an HTTP header. By default, all the attributes are included,
unless \var{attrs} is given, in which case it should be a list of attributes
to use. \var{header} is by default \code{"Set-Cookie:"}.
\end{methoddesc}

\begin{methoddesc}[Morsel]{js_output}{\opt{attrs}}
Return an embeddable JavaScript snippet, which, if run on a browser which
supports JavaScript, will act the same as if the HTTP header was sent.

The meaning for \var{attrs} is the same as in \method{output()}.
\end{methoddesc}.

\begin{methoddesc}[Morsel]{OutputString}{\opt{attrs}}
Return a string representing the Morsel, without any surrounding HTTP
or JavaScript.

The meaning for \var{attrs} is the same as in \method{output()}.
\end{methoddesc}
                
# This used to be strict parsing based on the RFC2109 and RFC2068
# specifications.  I have since discovered that MSIE 3.0x doesn't
# follow the character rules outlined in those specs.  As a
# result, the parsing rules here are less strict.

\begin{classdesc}{BaseCookie}{\opt{input}}
This class is a dictionary-like object whose keys are strings and
whose values are \class{Morsel}s. Note that upon setting a key to
a value, the value is first converted to a \class{Morsel} containing
the key and the value.

If \var{input} is given, it is passed to the \method{load} method.
\end{classdesc}

\begin{methoddesc}[BaseCookie]{value_decode}{val}
Return a decoded value from a string representation. Return value can
be any type. This method does nothing in \class{BaseCookie} --- it exists
so it can be overridden.
\end{methoddesc}

\begin{methoddesc}[BaseCookie]{value_encode}{val}
Return an encoded value. \var{val} can be any type, but return value
must be a string. This method does nothing in \class{BaseCookie} --- it exists
so it can be overridden

In general, it should be the case that \method{value_encode} and 
\method{value_decode} are inverses on the range of \var{value_decode}.
\end{methoddesc}.

\begin{methoddesc}[BaseCookie]{output}{\opt{attrs\opt{, header\opt{, sep}}}}
Return a string representation suitable to be sent as HTTP headers.
\var{attrs} and \var{header} are sent to each \class{Morsel}'s \method{output}
method. \var{sep} is used to join the headers together, and is by default
a newline.
\end{methoddesc}

\begin{methoddesc}[BaseCookie]{js_output}{\opt{attrs}}
Return an embeddable JavaScript snippet, which, if run on a browser which
supports JavaScript, will act the same as if the HTTP headers was sent.

The meaning for \var{attrs} is the same as in \method{output()}.
\end{methoddesc}

\begin{methoddesc}[BaseCookie]{load}{rawdata}
If \var{rawdata} is a string, parse it as an \code{HTTP_COOKIE} and add
the values found there as \class{Morsel}s. If it is a dictionary, it
is equivalent to calling

\begin{verbatim}
map(BaseCookie.__setitem__, rawdata.keys(), rawdata.values())
\end{varbatim}
\end{methoddesc}

\begin{classdesc}{SimpleCookie}{\opt{input}}
This class derives from \class{BaseCookie} and overrides \method{value_decode}
and \method{value_encode} to be the identity and \function{str()} respectively.
\end{classdesc}

\begin{classdesc}{SerialCookie}{\opt{input}}
This class derives from \class{BaseCookie} and overrides \method{value_decode}
and \method{value_encode} to be the \function{pickle.loads()} and 
\function{pickle.dumps}. Note that using this class is a security hole,
as arbitrary client-code can be run on \function{pickle.loads()}.
\end{classdesc}

\begin{classdesc}{SmartCookie}{\opt{input}}
This class derives from \class{BaseCookie}. It overrides \method{value_decode}
to be \function{pickle.loads()} if it is a valid pickle, and otherwise
the value itself. It overrides \method{value_encode} to be 
\function{pickle.dumps()} unless it is a string, in which case it returns
the value itself.

The same security warning from \class{SerialCookie} applies here.
\end{classdesc}
