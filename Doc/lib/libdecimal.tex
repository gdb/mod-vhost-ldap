\section{\module{decimal} ---
         Decimal floating point arithmetic}

\declaremodule{standard}{decimal}
\modulesynopsis{Implementation of the General Decimal Arithmetic 
Specification.}

\moduleauthor{Eric Price}{eprice at tjhsst.edu}
\moduleauthor{Facundo Batista}{facundo at taniquetil.com.ar}
\moduleauthor{Raymond Hettinger}{python at rcn.com}
\moduleauthor{Aahz}{aahz at pobox.com}
\moduleauthor{Tim Peters}{tim.one at comcast.net}

\sectionauthor{Raymond D. Hettinger}{python at rcn.com}

\versionadded{2.4}

The \module{decimal} module provides support for decimal floating point
arithmetic.  It offers several advantages over the \class{float()} datatype:

\begin{itemize}

\item Decimal numbers can be represented exactly.  In contrast, numbers like
\constant{1.1} do not have an exact representation in binary floating point.
End users typically would not expect \constant{1.1} to display as
\constant{1.1000000000000001} as it does with binary floating point.

\item The exactness carries over into arithmetic.  In decimal floating point,
\samp{0.1 + 0.1 + 0.1 - 0.3} is exactly equal to zero.  In binary floating
point, result is \constant{5.5511151231257827e-017}.  While near to zero, the
differences prevent reliable equality testing and differences can accumulate.
For this reason, decimal would be preferred in accounting applications which
have strict equality invariants.

\item The decimal module incorporates a notion of significant places so that
\samp{1.30 + 1.20} is \constant{2.50}.  The trailing zero is kept to indicate
significance.  This is the customary presentation for monetary applications. For
multiplication, the ``schoolbook'' approach uses all the figures in the
multiplicands.  For instance, \samp{1.3 * 1.2} gives \constant{1.56} while
\samp{1.30 * 1.20} gives \constant{1.5600}.

\item Unlike hardware based binary floating point, the decimal module has a user
settable precision (defaulting to 28 places) which can be as large as needed for
a given problem:

\begin{verbatim}
>>> getcontext().prec = 6
>>> Decimal(1) / Decimal(7)
Decimal("0.142857")
>>> getcontext().prec = 28
>>> Decimal(1) / Decimal(7)
Decimal("0.1428571428571428571428571429")
\end{verbatim}

\item Both binary and decimal floating point are implemented in terms of published
standards.  While the built-in float type exposes only a modest portion of its
capabilities, the decimal module exposes all required parts of the standard.
When needed, the programmer has full control over rounding and signal handling.

\end{itemize}


The module design is centered around three concepts:  the decimal number, the
context for arithmetic, and signals.

A decimal number is immutable.  It has a sign, coefficient digits, and an
exponent.  To preserve significance, the coefficient digits do not truncate
trailing zeroes.  Decimals also include special values such as
\constant{Infinity}, \constant{-Infinity}, and \constant{NaN}.  The standard
also differentiates \constant{-0} from \constant{+0}.
                                                   
The context for arithmetic is an environment specifying precision, rounding
rules, limits on exponents, flags indicating the results of operations,
and trap enablers which determine whether signals are treated as
exceptions.  Rounding options include \constant{ROUND_CEILING},
\constant{ROUND_DOWN}, \constant{ROUND_FLOOR}, \constant{ROUND_HALF_DOWN},
\constant{ROUND_HALF_EVEN}, \constant{ROUND_HALF_UP}, and \constant{ROUND_UP}.

Signals are groups of exceptional conditions arising during the course of
computation.  Depending on the needs of the application, signals may be
ignored, considered as informational, or treated as exceptions. The signals in
the decimal module are: \constant{Clamped}, \constant{InvalidOperation},
\constant{DivisionByZero}, \constant{Inexact}, \constant{Rounded},
\constant{Subnormal}, \constant{Overflow}, and \constant{Underflow}.

For each signal there is a flag and a trap enabler.  When a signal is
encountered, its flag is incremented from zero and, then, if the trap enabler
is set to one, an exception is raised.  Flags are sticky, so the user
needs to reset them before monitoring a calculation.


\begin{seealso}
  \seetext{IBM's General Decimal Arithmetic Specification,
           \citetitle[http://www2.hursley.ibm.com/decimal/decarith.html]
           {The General Decimal Arithmetic Specification}.}

  \seetext{IEEE standard 854-1987,
           \citetitle[http://www.cs.berkeley.edu/\textasciitilde ejr/projects/754/private/drafts/854-1987/dir.html]
           {Unofficial IEEE 854 Text}.} 
\end{seealso}



%%%%%%%%%%%%%%%%%%%%%%%%%%%%%%%%%%%%%%%%%%%%%%%%%%%%%%%%%%%%%%%
\subsection{Quick-start Tutorial \label{decimal-tutorial}}

The usual start to using decimals is importing the module, viewing the current
context with \function{getcontext()} and, if necessary, setting new values
for precision, rounding, or enabled traps:

\begin{verbatim}
>>> from decimal import *
>>> getcontext()
Context(prec=28, rounding=ROUND_HALF_EVEN, Emin=-999999999, Emax=999999999,
        capitals=1, flags=[], traps=[Overflow, InvalidOperation,
        DivisionByZero])

>>> getcontext().prec = 7       # Set a new precision
\end{verbatim}


Decimal instances can be constructed from integers, strings, or tuples.  To
create a Decimal from a \class{float}, first convert it to a string.  This
serves as an explicit reminder of the details of the conversion (including
representation error).  Decimal numbers include special values such as
\constant{NaN} which stands for ``Not a number'', positive and negative
\constant{Infinity}, and \constant{-0}.        

\begin{verbatim}
>>> Decimal(10)
Decimal("10")
>>> Decimal("3.14")
Decimal("3.14")
>>> Decimal((0, (3, 1, 4), -2))
Decimal("3.14")
>>> Decimal(str(2.0 ** 0.5))
Decimal("1.41421356237")
>>> Decimal("NaN")
Decimal("NaN")
>>> Decimal("-Infinity")
Decimal("-Infinity")
\end{verbatim}


The significance of a new Decimal is determined solely by the number
of digits input.  Context precision and rounding only come into play during
arithmetic operations.

\begin{verbatim}
>>> getcontext().prec = 6
>>> Decimal('3.0')
Decimal("3.0")
>>> Decimal('3.1415926535')
Decimal("3.1415926535")
>>> Decimal('3.1415926535') + Decimal('2.7182818285')
Decimal("5.85987")
>>> getcontext().rounding = ROUND_UP
>>> Decimal('3.1415926535') + Decimal('2.7182818285')
Decimal("5.85988")
\end{verbatim}


Decimals interact well with much of the rest of Python.  Here is a small
decimal floating point flying circus:
    
\begin{verbatim}    
>>> data = map(Decimal, '1.34 1.87 3.45 2.35 1.00 0.03 9.25'.split())
>>> max(data)
Decimal("9.25")
>>> min(data)
Decimal("0.03")
>>> sorted(data)
[Decimal("0.03"), Decimal("1.00"), Decimal("1.34"), Decimal("1.87"),
 Decimal("2.35"), Decimal("3.45"), Decimal("9.25")]
>>> sum(data)
Decimal("19.29")
>>> a,b,c = data[:3]
>>> str(a)
'1.34'
>>> float(a)
1.3400000000000001
>>> round(a, 1)     # round() first converts to binary floating point
1.3
>>> int(a)
1
>>> a * 5
Decimal("6.70")
>>> a * b
Decimal("2.5058")
>>> c % a
Decimal("0.77")
\end{verbatim}

The \method{quantize()} method rounds a number to a fixed exponent.  This
method is useful for monetary applications that often round results to a fixed
number of places:

\begin{verbatim} 
>>> Decimal('7.325').quantize(Decimal('.01'), rounding=ROUND_DOWN)
Decimal("7.32")
>>> Decimal('7.325').quantize(Decimal('1.'), rounding=ROUND_UP)
Decimal("8")
\end{verbatim}

As shown above, the \function{getcontext()} function accesses the current
context and allows the settings to be changed.  This approach meets the
needs of most applications.

For more advanced work, it may be useful to create alternate contexts using
the Context() constructor.  To make an alternate active, use the
\function{setcontext()} function.

In accordance with the standard, the \module{Decimal} module provides two
ready to use standard contexts, \constant{BasicContext} and
\constant{ExtendedContext}. The former is especially useful for debugging
because many of the traps are enabled:

\begin{verbatim}
>>> myothercontext = Context(prec=60, rounding=ROUND_HALF_DOWN)
>>> setcontext(myothercontext)
>>> Decimal(1) / Decimal(7)
Decimal("0.142857142857142857142857142857142857142857142857142857142857")

>>> ExtendedContext
Context(prec=9, rounding=ROUND_HALF_EVEN, Emin=-999999999, Emax=999999999,
        capitals=1, flags=[], traps=[])
>>> setcontext(ExtendedContext)
>>> Decimal(1) / Decimal(7)
Decimal("0.142857143")
>>> Decimal(42) / Decimal(0)
Decimal("Infinity")

>>> setcontext(BasicContext)
>>> Decimal(42) / Decimal(0)
Traceback (most recent call last):
  File "<pyshell#143>", line 1, in -toplevel-
    Decimal(42) / Decimal(0)
DivisionByZero: x / 0
\end{verbatim}


Contexts also have signal flags for monitoring exceptional conditions
encountered during computations.  The flags remain set until explicitly
cleared, so it is best to clear the flags before each set of monitored
computations by using the \method{clear_flags()} method.

\begin{verbatim}
>>> setcontext(ExtendedContext)
>>> getcontext().clear_flags()
>>> Decimal(355) / Decimal(113)
Decimal("3.14159292")
>>> getcontext()
Context(prec=9, rounding=ROUND_HALF_EVEN, Emin=-999999999, Emax=999999999,
        capitals=1, flags=[Inexact, Rounded], traps=[])
\end{verbatim}

The \var{flags} entry shows that the rational approximation to \constant{Pi}
was rounded (digits beyond the context precision were thrown away) and that
the result is inexact (some of the discarded digits were non-zero).

Individual traps are set using the dictionary in the \member{traps}
field of a context:

\begin{verbatim}
>>> Decimal(1) / Decimal(0)
Decimal("Infinity")
>>> getcontext().traps[DivisionByZero] = 1
>>> Decimal(1) / Decimal(0)
Traceback (most recent call last):
  File "<pyshell#112>", line 1, in -toplevel-
    Decimal(1) / Decimal(0)
DivisionByZero: x / 0
\end{verbatim}

Most programs adjust the current context only once, at the beginning of the
program.  And, in many applications, data is converted to \class{Decimal} with
a single cast inside a loop.  With context set and decimals created, the bulk
of the program manipulates the data no differently than with other Python
numeric types.



%%%%%%%%%%%%%%%%%%%%%%%%%%%%%%%%%%%%%%%%%%%%%%%%%%%%%%%%%%%%%%%
\subsection{Decimal objects \label{decimal-decimal}}

\begin{classdesc}{Decimal}{\optional{value \optional{, context}}}
  Constructs a new \class{Decimal} object based from \var{value}.

  \var{value} can be an integer, string, tuple, or another \class{Decimal}
  object. If no \var{value} is given, returns \code{Decimal("0")}.  If
  \var{value} is a string, it should conform to the decimal numeric string
  syntax:
    
  \begin{verbatim}
    sign           ::=  '+' | '-'
    digit          ::=  '0' | '1' | '2' | '3' | '4' | '5' | '6' | '7' | '8' | '9'
    indicator      ::=  'e' | 'E'
    digits         ::=  digit [digit]...
    decimal-part   ::=  digits '.' [digits] | ['.'] digits
    exponent-part  ::=  indicator [sign] digits
    infinity       ::=  'Infinity' | 'Inf'
    nan            ::=  'NaN' [digits] | 'sNaN' [digits]
    numeric-value  ::=  decimal-part [exponent-part] | infinity
    numeric-string ::=  [sign] numeric-value | [sign] nan  
  \end{verbatim}

  If \var{value} is a \class{tuple}, it should have three components,
  a sign (\constant{0} for positive or \constant{1} for negative),
  a \class{tuple} of digits, and an integer exponent. For example,
  \samp{Decimal((0, (1, 4, 1, 4), -3))} returns \code{Decimal("1.414")}.

  The \var{context} precision does not affect how many digits are stored.
  That is determined exclusively by the number of digits in \var{value}. For
  example, \samp{Decimal("3.00000")} records all five zeroes even if the
  context precision is only three.

  The purpose of the \var{context} argument is determining what to do if
  \var{value} is a malformed string.  If the context traps
  \constant{InvalidOperation}, an exception is raised; otherwise, the
  constructor returns a new Decimal with the value of \constant{NaN}.

  Once constructed, \class{Decimal} objects are immutable.
\end{classdesc}

Decimal floating point objects share many properties with the other builtin
numeric types such as \class{float} and \class{int}.  All of the usual
math operations and special methods apply.  Likewise, decimal objects can
be copied, pickled, printed, used as dictionary keys, used as set elements,
compared, sorted, and coerced to another type (such as \class{float}
or \class{long}).

In addition to the standard numeric properties, decimal floating point objects
also have a number of specialized methods:

\begin{methoddesc}{adjusted}{}
  Return the adjusted exponent after shifting out the coefficient's rightmost
  digits until only the lead digit remains: \code{Decimal("321e+5").adjusted()}
  returns seven.  Used for determining the position of the most significant
  digit with respect to the decimal point.
\end{methoddesc}

\begin{methoddesc}{as_tuple}{}
  Returns a tuple representation of the number:
  \samp{(sign, digittuple, exponent)}.
\end{methoddesc}

\begin{methoddesc}{compare}{other\optional{, context}}
  Compares like \method{__cmp__()} but returns a decimal instance:
  \begin{verbatim}
        a or b is a NaN ==> Decimal("NaN")
        a < b           ==> Decimal("-1")
        a == b          ==> Decimal("0")
        a > b           ==> Decimal("1")
  \end{verbatim}
\end{methoddesc}

\begin{methoddesc}{max}{other\optional{, context}}
  Like \samp{max(self, other)} except that the context rounding rule
  is applied before returning and that \constant{NaN} values are
  either signalled or ignored (depending on the context and whether
  they are signaling or quiet).
\end{methoddesc}

\begin{methoddesc}{min}{other\optional{, context}}
  Like \samp{min(self, other)} except that the context rounding rule
  is applied before returning and that \constant{NaN} values are
  either signalled or ignored (depending on the context and whether
  they are signaling or quiet).
\end{methoddesc}

\begin{methoddesc}{normalize}{\optional{context}}
  Normalize the number by stripping the rightmost trailing zeroes and
  converting any result equal to \constant{Decimal("0")} to
  \constant{Decimal("0e0")}. Used for producing canonical values for members
  of an equivalence class. For example, \code{Decimal("32.100")} and
  \code{Decimal("0.321000e+2")} both normalize to the equivalent value
  \code{Decimal("32.1")}.
\end{methoddesc}                                              

\begin{methoddesc}{quantize}
  {exp \optional{, rounding\optional{, context\optional{, watchexp}}}}
  Quantize makes the exponent the same as \var{exp}.  Searches for a
  rounding method in \var{rounding}, then in \var{context}, and then
  in the current context.

  If \var{watchexp} is set (default), then an error is returned whenever
  the resulting exponent is greater than \member{Emax} or less than
  \member{Etiny}.
\end{methoddesc} 

\begin{methoddesc}{remainder_near}{other\optional{, context}}
  Computes the modulo as either a positive or negative value depending
  on which is closest to zero.  For instance,
  \samp{Decimal(10).remainder_near(6)} returns \code{Decimal("-2")}
  which is closer to zero than \code{Decimal("4")}.

  If both are equally close, the one chosen will have the same sign
  as \var{self}.
\end{methoddesc}  

\begin{methoddesc}{same_quantum}{other\optional{, context}}
  Test whether self and other have the same exponent or whether both
  are \constant{NaN}.
\end{methoddesc}

\begin{methoddesc}{sqrt}{\optional{context}}
  Return the square root to full precision.
\end{methoddesc}                    
 
\begin{methoddesc}{to_eng_string}{\optional{context}}
  Convert to an engineering-type string.

  Engineering notation has an exponent which is a multiple of 3, so there
  are up to 3 digits left of the decimal place.  For example, converts
  \code{Decimal('123E+1')} to \code{Decimal("1.23E+3")}
\end{methoddesc}  

\begin{methoddesc}{to_integral}{\optional{rounding\optional{, context}}}                   
  Rounds to the nearest integer without signaling \constant{Inexact}
  or \constant{Rounded}.  If given, applies \var{rounding}; otherwise,
  uses the rounding method in either the supplied \var{context} or the
  current context.
\end{methoddesc} 



%%%%%%%%%%%%%%%%%%%%%%%%%%%%%%%%%%%%%%%%%%%%%%%%%%%%%%%%%%%%%%%            
\subsection{Context objects \label{decimal-context}}

Contexts are environments for arithmetic operations.  They govern precision,
set rules for rounding, determine which signals are treated as exceptions, and
limit the range for exponents.

Each thread has its own current context which is accessed or changed using
the \function{getcontext()} and \function{setcontext()} functions:

\begin{funcdesc}{getcontext}{}
  Return the current context for the active thread.
\end{funcdesc}            

\begin{funcdesc}{setcontext}{c}
  Set the current context for the active thread to \var{c}.
\end{funcdesc}  

Beginning with Python 2.5, you can also use the \keyword{with} statement
and the \function{localcontext()} function to temporarily change the
active context.

\begin{funcdesc}{localcontext}{\optional{c}}
  Return a context manager that will set the current context for
  the active thread to a copy of \var{c} on entry to the with-statement
  and restore the previous context when exiting the with-statement. If
  no context is specified, a copy of the current context is used.
  \versionadded{2.5}

  For example, the following code sets the current decimal precision
  to 42 places, performs a calculation, and then automatically restores
  the previous context:
\begin{verbatim}
    from __future__ import with_statement
    from decimal import localcontext

    with localcontext() as ctx:
        ctx.prec = 42   # Perform a high precision calculation
        s = calculate_something()
    s = +s  # Round the final result back to the default precision
\end{verbatim}
\end{funcdesc}

New contexts can also be created using the \class{Context} constructor
described below. In addition, the module provides three pre-made
contexts:

\begin{classdesc*}{BasicContext}
  This is a standard context defined by the General Decimal Arithmetic
  Specification.  Precision is set to nine.  Rounding is set to
  \constant{ROUND_HALF_UP}.  All flags are cleared.  All traps are enabled
  (treated as exceptions) except \constant{Inexact}, \constant{Rounded}, and
  \constant{Subnormal}.

  Because many of the traps are enabled, this context is useful for debugging.
\end{classdesc*}

\begin{classdesc*}{ExtendedContext}
  This is a standard context defined by the General Decimal Arithmetic
  Specification.  Precision is set to nine.  Rounding is set to
  \constant{ROUND_HALF_EVEN}.  All flags are cleared.  No traps are enabled
  (so that exceptions are not raised during computations).

  Because the trapped are disabled, this context is useful for applications
  that prefer to have result value of \constant{NaN} or \constant{Infinity}
  instead of raising exceptions.  This allows an application to complete a
  run in the presence of conditions that would otherwise halt the program.
\end{classdesc*}

\begin{classdesc*}{DefaultContext}
  This context is used by the \class{Context} constructor as a prototype for
  new contexts.  Changing a field (such a precision) has the effect of
  changing the default for new contexts creating by the \class{Context}
  constructor.

  This context is most useful in multi-threaded environments.  Changing one of
  the fields before threads are started has the effect of setting system-wide
  defaults.  Changing the fields after threads have started is not recommended
  as it would require thread synchronization to prevent race conditions.

  In single threaded environments, it is preferable to not use this context
  at all.  Instead, simply create contexts explicitly as described below.

  The default values are precision=28, rounding=ROUND_HALF_EVEN, and enabled
  traps for Overflow, InvalidOperation, and DivisionByZero.
\end{classdesc*}


In addition to the three supplied contexts, new contexts can be created
with the \class{Context} constructor.

\begin{classdesc}{Context}{prec=None, rounding=None, traps=None,
        flags=None, Emin=None, Emax=None, capitals=1}
  Creates a new context.  If a field is not specified or is \constant{None},
  the default values are copied from the \constant{DefaultContext}.  If the
  \var{flags} field is not specified or is \constant{None}, all flags are
  cleared.

  The \var{prec} field is a positive integer that sets the precision for
  arithmetic operations in the context.

  The \var{rounding} option is one of:
  \begin{itemize}
  \item \constant{ROUND_CEILING} (towards \constant{Infinity}),
  \item \constant{ROUND_DOWN} (towards zero),
  \item \constant{ROUND_FLOOR} (towards \constant{-Infinity}),
  \item \constant{ROUND_HALF_DOWN} (to nearest with ties going towards zero),
  \item \constant{ROUND_HALF_EVEN} (to nearest with ties going to nearest even integer),
  \item \constant{ROUND_HALF_UP} (to nearest with ties going away from zero), or
  \item \constant{ROUND_UP} (away from zero).
  \end{itemize}

  The \var{traps} and \var{flags} fields list any signals to be set.
  Generally, new contexts should only set traps and leave the flags clear.

  The \var{Emin} and \var{Emax} fields are integers specifying the outer
  limits allowable for exponents.

  The \var{capitals} field is either \constant{0} or \constant{1} (the
  default). If set to \constant{1}, exponents are printed with a capital
  \constant{E}; otherwise, a lowercase \constant{e} is used:
  \constant{Decimal('6.02e+23')}.
\end{classdesc}

The \class{Context} class defines several general purpose methods as well as a
large number of methods for doing arithmetic directly in a given context.

\begin{methoddesc}{clear_flags}{}
  Resets all of the flags to \constant{0}.
\end{methoddesc}  

\begin{methoddesc}{copy}{}
  Return a duplicate of the context.
\end{methoddesc}  

\begin{methoddesc}{create_decimal}{num}
  Creates a new Decimal instance from \var{num} but using \var{self} as
  context. Unlike the \class{Decimal} constructor, the context precision,
  rounding method, flags, and traps are applied to the conversion.

  This is useful because constants are often given to a greater precision than
  is needed by the application.  Another benefit is that rounding immediately
  eliminates unintended effects from digits beyond the current precision.
  In the following example, using unrounded inputs means that adding zero
  to a sum can change the result:

  \begin{verbatim}
    >>> getcontext().prec = 3
    >>> Decimal("3.4445") + Decimal("1.0023")
    Decimal("4.45")
    >>> Decimal("3.4445") + Decimal(0) + Decimal("1.0023")
    Decimal("4.44")
  \end{verbatim}
      
\end{methoddesc} 

\begin{methoddesc}{Etiny}{}
  Returns a value equal to \samp{Emin - prec + 1} which is the minimum
  exponent value for subnormal results.  When underflow occurs, the
  exponent is set to \constant{Etiny}.
\end{methoddesc} 

\begin{methoddesc}{Etop}{}
  Returns a value equal to \samp{Emax - prec + 1}.
\end{methoddesc} 


The usual approach to working with decimals is to create \class{Decimal}
instances and then apply arithmetic operations which take place within the
current context for the active thread.  An alternate approach is to use
context methods for calculating within a specific context.  The methods are
similar to those for the \class{Decimal} class and are only briefly recounted
here.

\begin{methoddesc}{abs}{x}
  Returns the absolute value of \var{x}.
\end{methoddesc}

\begin{methoddesc}{add}{x, y}
  Return the sum of \var{x} and \var{y}.
\end{methoddesc}
   
\begin{methoddesc}{compare}{x, y}
  Compares values numerically.
  
  Like \method{__cmp__()} but returns a decimal instance:
  \begin{verbatim}
        a or b is a NaN ==> Decimal("NaN")
        a < b           ==> Decimal("-1")
        a == b          ==> Decimal("0")
        a > b           ==> Decimal("1")
  \end{verbatim}                                          
\end{methoddesc}

\begin{methoddesc}{divide}{x, y}
  Return \var{x} divided by \var{y}.
\end{methoddesc}   
  
\begin{methoddesc}{divmod}{x, y}
  Divides two numbers and returns the integer part of the result.
\end{methoddesc} 

\begin{methoddesc}{max}{x, y}
  Compare two values numerically and return the maximum.

  If they are numerically equal then the left-hand operand is chosen as the
  result.
\end{methoddesc} 
 
\begin{methoddesc}{min}{x, y}
  Compare two values numerically and return the minimum.

  If they are numerically equal then the left-hand operand is chosen as the
  result.
\end{methoddesc}

\begin{methoddesc}{minus}{x}
  Minus corresponds to the unary prefix minus operator in Python.
\end{methoddesc}

\begin{methoddesc}{multiply}{x, y}
  Return the product of \var{x} and \var{y}.
\end{methoddesc}

\begin{methoddesc}{normalize}{x}
  Normalize reduces an operand to its simplest form.

  Essentially a \method{plus} operation with all trailing zeros removed from
  the result.
\end{methoddesc}
  
\begin{methoddesc}{plus}{x}
  Plus corresponds to the unary prefix plus operator in Python.  This
  operation applies the context precision and rounding, so it is
  \emph{not} an identity operation.
\end{methoddesc}

\begin{methoddesc}{power}{x, y\optional{, modulo}}
  Return \samp{x ** y} to the \var{modulo} if given.

  The right-hand operand must be a whole number whose integer part (after any
  exponent has been applied) has no more than 9 digits and whose fractional
  part (if any) is all zeros before any rounding. The operand may be positive,
  negative, or zero; if negative, the absolute value of the power is used, and
  the left-hand operand is inverted (divided into 1) before use.

  If the increased precision needed for the intermediate calculations exceeds
  the capabilities of the implementation then an \constant{InvalidOperation}
  condition is signaled.

  If, when raising to a negative power, an underflow occurs during the
  division into 1, the operation is not halted at that point but continues. 
\end{methoddesc}

\begin{methoddesc}{quantize}{x, y}
  Returns a value equal to \var{x} after rounding and having the exponent of
  \var{y}.

  Unlike other operations, if the length of the coefficient after the quantize
  operation would be greater than precision, then an
  \constant{InvalidOperation} is signaled. This guarantees that, unless there
  is an error condition, the quantized exponent is always equal to that of the
  right-hand operand.

  Also unlike other operations, quantize never signals Underflow, even
  if the result is subnormal and inexact.  
\end{methoddesc} 

\begin{methoddesc}{remainder}{x, y}
  Returns the remainder from integer division.

  The sign of the result, if non-zero, is the same as that of the original
  dividend. 
\end{methoddesc}
 
\begin{methoddesc}{remainder_near}{x, y}
  Computed the modulo as either a positive or negative value depending
  on which is closest to zero.  For instance,
  \samp{Decimal(10).remainder_near(6)} returns \code{Decimal("-2")}
  which is closer to zero than \code{Decimal("4")}.

  If both are equally close, the one chosen will have the same sign
  as \var{self}.
\end{methoddesc}

\begin{methoddesc}{same_quantum}{x, y}
  Test whether \var{x} and \var{y} have the same exponent or whether both are
  \constant{NaN}.
\end{methoddesc}

\begin{methoddesc}{sqrt}{x}
  Return the square root of \var{x} to full precision.
\end{methoddesc}                    

\begin{methoddesc}{subtract}{x, y}
  Return the difference between \var{x} and \var{y}.
\end{methoddesc}
 
\begin{methoddesc}{to_eng_string}{}
  Convert to engineering-type string.

  Engineering notation has an exponent which is a multiple of 3, so there
  are up to 3 digits left of the decimal place.  For example, converts
  \code{Decimal('123E+1')} to \code{Decimal("1.23E+3")}
\end{methoddesc}  

\begin{methoddesc}{to_integral}{x}                  
  Rounds to the nearest integer without signaling \constant{Inexact}
  or \constant{Rounded}.                                        
\end{methoddesc} 

\begin{methoddesc}{to_sci_string}{x}
  Converts a number to a string using scientific notation.
\end{methoddesc} 



%%%%%%%%%%%%%%%%%%%%%%%%%%%%%%%%%%%%%%%%%%%%%%%%%%%%%%%%%%%%%%%            
\subsection{Signals \label{decimal-signals}}

Signals represent conditions that arise during computation.
Each corresponds to one context flag and one context trap enabler.

The context flag is incremented whenever the condition is encountered.
After the computation, flags may be checked for informational
purposes (for instance, to determine whether a computation was exact).
After checking the flags, be sure to clear all flags before starting
the next computation.

If the context's trap enabler is set for the signal, then the condition
causes a Python exception to be raised.  For example, if the
\class{DivisionByZero} trap is set, then a \exception{DivisionByZero}
exception is raised upon encountering the condition.


\begin{classdesc*}{Clamped}
    Altered an exponent to fit representation constraints.

    Typically, clamping occurs when an exponent falls outside the context's
    \member{Emin} and \member{Emax} limits.  If possible, the exponent is
    reduced to fit by adding zeroes to the coefficient.
\end{classdesc*}

\begin{classdesc*}{DecimalException}
    Base class for other signals and a subclass of
    \exception{ArithmeticError}.
\end{classdesc*}

\begin{classdesc*}{DivisionByZero}
    Signals the division of a non-infinite number by zero.

    Can occur with division, modulo division, or when raising a number to a
    negative power.  If this signal is not trapped, returns
    \constant{Infinity} or \constant{-Infinity} with the sign determined by
    the inputs to the calculation.
\end{classdesc*}

\begin{classdesc*}{Inexact}
    Indicates that rounding occurred and the result is not exact.

    Signals when non-zero digits were discarded during rounding. The rounded
    result is returned.  The signal flag or trap is used to detect when
    results are inexact.
\end{classdesc*}

\begin{classdesc*}{InvalidOperation}
    An invalid operation was performed.

    Indicates that an operation was requested that does not make sense.
    If not trapped, returns \constant{NaN}.  Possible causes include:

    \begin{verbatim}
        Infinity - Infinity
        0 * Infinity
        Infinity / Infinity
        x % 0
        Infinity % x
        x._rescale( non-integer )
        sqrt(-x) and x > 0
        0 ** 0
        x ** (non-integer)
        x ** Infinity      
    \end{verbatim}    
\end{classdesc*}

\begin{classdesc*}{Overflow}
    Numerical overflow.

    Indicates the exponent is larger than \member{Emax} after rounding has
    occurred.  If not trapped, the result depends on the rounding mode, either
    pulling inward to the largest representable finite number or rounding
    outward to \constant{Infinity}.  In either case, \class{Inexact} and
    \class{Rounded} are also signaled.   
\end{classdesc*}

\begin{classdesc*}{Rounded}
    Rounding occurred though possibly no information was lost.

    Signaled whenever rounding discards digits; even if those digits are
    zero (such as rounding \constant{5.00} to \constant{5.0}).   If not
    trapped, returns the result unchanged.  This signal is used to detect
    loss of significant digits.
\end{classdesc*}

\begin{classdesc*}{Subnormal}
    Exponent was lower than \member{Emin} prior to rounding.
          
    Occurs when an operation result is subnormal (the exponent is too small).
    If not trapped, returns the result unchanged.
\end{classdesc*}

\begin{classdesc*}{Underflow}
    Numerical underflow with result rounded to zero.

    Occurs when a subnormal result is pushed to zero by rounding.
    \class{Inexact} and \class{Subnormal} are also signaled.
\end{classdesc*}

The following table summarizes the hierarchy of signals:

\begin{verbatim}    
    exceptions.ArithmeticError(exceptions.StandardError)
        DecimalException
            Clamped
            DivisionByZero(DecimalException, exceptions.ZeroDivisionError)
            Inexact
                Overflow(Inexact, Rounded)
                Underflow(Inexact, Rounded, Subnormal)
            InvalidOperation
            Rounded
            Subnormal
\end{verbatim}            


%%%%%%%%%%%%%%%%%%%%%%%%%%%%%%%%%%%%%%%%%%%%%%%%%%%%%%%%%%%%%%%
\subsection{Floating Point Notes \label{decimal-notes}}

\subsubsection{Mitigating round-off error with increased precision}

The use of decimal floating point eliminates decimal representation error
(making it possible to represent \constant{0.1} exactly); however, some
operations can still incur round-off error when non-zero digits exceed the
fixed precision.

The effects of round-off error can be amplified by the addition or subtraction
of nearly offsetting quantities resulting in loss of significance.  Knuth
provides two instructive examples where rounded floating point arithmetic with
insufficient precision causes the breakdown of the associative and
distributive properties of addition:

\begin{verbatim}
# Examples from Seminumerical Algorithms, Section 4.2.2.
>>> from decimal import Decimal, getcontext
>>> getcontext().prec = 8

>>> u, v, w = Decimal(11111113), Decimal(-11111111), Decimal('7.51111111')
>>> (u + v) + w
Decimal("9.5111111")
>>> u + (v + w)
Decimal("10")

>>> u, v, w = Decimal(20000), Decimal(-6), Decimal('6.0000003')
>>> (u*v) + (u*w)
Decimal("0.01")
>>> u * (v+w)
Decimal("0.0060000")
\end{verbatim}

The \module{decimal} module makes it possible to restore the identities
by expanding the precision sufficiently to avoid loss of significance:

\begin{verbatim}
>>> getcontext().prec = 20
>>> u, v, w = Decimal(11111113), Decimal(-11111111), Decimal('7.51111111')
>>> (u + v) + w
Decimal("9.51111111")
>>> u + (v + w)
Decimal("9.51111111")
>>> 
>>> u, v, w = Decimal(20000), Decimal(-6), Decimal('6.0000003')
>>> (u*v) + (u*w)
Decimal("0.0060000")
>>> u * (v+w)
Decimal("0.0060000")
\end{verbatim}

\subsubsection{Special values}

The number system for the \module{decimal} module provides special
values including \constant{NaN}, \constant{sNaN}, \constant{-Infinity},
\constant{Infinity}, and two zeroes, \constant{+0} and \constant{-0}.

Infinities can be constructed directly with:  \code{Decimal('Infinity')}. Also,
they can arise from dividing by zero when the \exception{DivisionByZero}
signal is not trapped.  Likewise, when the \exception{Overflow} signal is not
trapped, infinity can result from rounding beyond the limits of the largest
representable number.

The infinities are signed (affine) and can be used in arithmetic operations
where they get treated as very large, indeterminate numbers.  For instance,
adding a constant to infinity gives another infinite result.

Some operations are indeterminate and return \constant{NaN}, or if the
\exception{InvalidOperation} signal is trapped, raise an exception.  For
example, \code{0/0} returns \constant{NaN} which means ``not a number''.  This
variety of \constant{NaN} is quiet and, once created, will flow through other
computations always resulting in another \constant{NaN}.  This behavior can be
useful for a series of computations that occasionally have missing inputs ---
it allows the calculation to proceed while flagging specific results as
invalid.     

A variant is \constant{sNaN} which signals rather than remaining quiet
after every operation.  This is a useful return value when an invalid
result needs to interrupt a calculation for special handling.

The signed zeros can result from calculations that underflow.
They keep the sign that would have resulted if the calculation had
been carried out to greater precision.  Since their magnitude is
zero, both positive and negative zeros are treated as equal and their
sign is informational.

In addition to the two signed zeros which are distinct yet equal,
there are various representations of zero with differing precisions
yet equivalent in value.  This takes a bit of getting used to.  For
an eye accustomed to normalized floating point representations, it
is not immediately obvious that the following calculation returns
a value equal to zero:          

\begin{verbatim}
>>> 1 / Decimal('Infinity')
Decimal("0E-1000000026")
\end{verbatim}

%%%%%%%%%%%%%%%%%%%%%%%%%%%%%%%%%%%%%%%%%%%%%%%%%%%%%%%%%%%%%%%
\subsection{Working with threads \label{decimal-threads}}

The \function{getcontext()} function accesses a different \class{Context}
object for each thread.  Having separate thread contexts means that threads
may make changes (such as \code{getcontext.prec=10}) without interfering with
other threads.

Likewise, the \function{setcontext()} function automatically assigns its target
to the current thread.

If \function{setcontext()} has not been called before \function{getcontext()},
then \function{getcontext()} will automatically create a new context for use
in the current thread.

The new context is copied from a prototype context called
\var{DefaultContext}. To control the defaults so that each thread will use the
same values throughout the application, directly modify the
\var{DefaultContext} object. This should be done \emph{before} any threads are
started so that there won't be a race condition between threads calling
\function{getcontext()}. For example:

\begin{verbatim}
# Set applicationwide defaults for all threads about to be launched
DefaultContext.prec = 12
DefaultContext.rounding = ROUND_DOWN
DefaultContext.traps = ExtendedContext.traps.copy()
DefaultContext.traps[InvalidOperation] = 1
setcontext(DefaultContext)

# Afterwards, the threads can be started
t1.start()
t2.start()
t3.start()
 . . .
\end{verbatim}



%%%%%%%%%%%%%%%%%%%%%%%%%%%%%%%%%%%%%%%%%%%%%%%%%%%%%%%%%%%%%%%
\subsection{Recipes \label{decimal-recipes}}

Here are a few recipes that serve as utility functions and that demonstrate
ways to work with the \class{Decimal} class:

\begin{verbatim}
def moneyfmt(value, places=2, curr='', sep=',', dp='.',
             pos='', neg='-', trailneg=''):
    """Convert Decimal to a money formatted string.

    places:  required number of places after the decimal point
    curr:    optional currency symbol before the sign (may be blank)
    sep:     optional grouping separator (comma, period, space, or blank)
    dp:      decimal point indicator (comma or period)
             only specify as blank when places is zero
    pos:     optional sign for positive numbers: '+', space or blank
    neg:     optional sign for negative numbers: '-', '(', space or blank
    trailneg:optional trailing minus indicator:  '-', ')', space or blank

    >>> d = Decimal('-1234567.8901')
    >>> moneyfmt(d, curr='$')
    '-$1,234,567.89'
    >>> moneyfmt(d, places=0, sep='.', dp='', neg='', trailneg='-')
    '1.234.568-'
    >>> moneyfmt(d, curr='$', neg='(', trailneg=')')
    '($1,234,567.89)'
    >>> moneyfmt(Decimal(123456789), sep=' ')
    '123 456 789.00'
    >>> moneyfmt(Decimal('-0.02'), neg='<', trailneg='>')
    '<.02>'

    """
    q = Decimal((0, (1,), -places))    # 2 places --> '0.01'
    sign, digits, exp = value.quantize(q).as_tuple()
    assert exp == -places    
    result = []
    digits = map(str, digits)
    build, next = result.append, digits.pop
    if sign:
        build(trailneg)
    for i in range(places):
        if digits:
            build(next())
        else:
            build('0')
    build(dp)
    i = 0
    while digits:
        build(next())
        i += 1
        if i == 3 and digits:
            i = 0
            build(sep)
    build(curr)
    if sign:
        build(neg)
    else:
        build(pos)
    result.reverse()
    return ''.join(result)

def pi():
    """Compute Pi to the current precision.

    >>> print pi()
    3.141592653589793238462643383
    
    """
    getcontext().prec += 2  # extra digits for intermediate steps
    three = Decimal(3)      # substitute "three=3.0" for regular floats
    lasts, t, s, n, na, d, da = 0, three, 3, 1, 0, 0, 24
    while s != lasts:
        lasts = s
        n, na = n+na, na+8
        d, da = d+da, da+32
        t = (t * n) / d
        s += t
    getcontext().prec -= 2
    return +s               # unary plus applies the new precision

def exp(x):
    """Return e raised to the power of x.  Result type matches input type.

    >>> print exp(Decimal(1))
    2.718281828459045235360287471
    >>> print exp(Decimal(2))
    7.389056098930650227230427461
    >>> print exp(2.0)
    7.38905609893
    >>> print exp(2+0j)
    (7.38905609893+0j)
    
    """
    getcontext().prec += 2
    i, lasts, s, fact, num = 0, 0, 1, 1, 1
    while s != lasts:
        lasts = s    
        i += 1
        fact *= i
        num *= x     
        s += num / fact   
    getcontext().prec -= 2        
    return +s

def cos(x):
    """Return the cosine of x as measured in radians.

    >>> print cos(Decimal('0.5'))
    0.8775825618903727161162815826
    >>> print cos(0.5)
    0.87758256189
    >>> print cos(0.5+0j)
    (0.87758256189+0j)
    
    """
    getcontext().prec += 2
    i, lasts, s, fact, num, sign = 0, 0, 1, 1, 1, 1
    while s != lasts:
        lasts = s    
        i += 2
        fact *= i * (i-1)
        num *= x * x
        sign *= -1
        s += num / fact * sign 
    getcontext().prec -= 2        
    return +s

def sin(x):
    """Return the sine of x as measured in radians.

    >>> print sin(Decimal('0.5'))
    0.4794255386042030002732879352
    >>> print sin(0.5)
    0.479425538604
    >>> print sin(0.5+0j)
    (0.479425538604+0j)
    
    """
    getcontext().prec += 2
    i, lasts, s, fact, num, sign = 1, 0, x, 1, x, 1
    while s != lasts:
        lasts = s    
        i += 2
        fact *= i * (i-1)
        num *= x * x
        sign *= -1
        s += num / fact * sign 
    getcontext().prec -= 2        
    return +s

\end{verbatim}                                             



%%%%%%%%%%%%%%%%%%%%%%%%%%%%%%%%%%%%%%%%%%%%%%%%%%%%%%%%%%%%%%%
\subsection{Decimal FAQ \label{decimal-faq}}

Q.  It is cumbersome to type \code{decimal.Decimal('1234.5')}.  Is there a way
to minimize typing when using the interactive interpreter?

A.  Some users abbreviate the constructor to just a single letter:

\begin{verbatim}
>>> D = decimal.Decimal
>>> D('1.23') + D('3.45')
Decimal("4.68")
\end{verbatim}


Q.  In a fixed-point application with two decimal places, some inputs
have many places and need to be rounded.  Others are not supposed to have
excess digits and need to be validated.  What methods should be used?

A.  The \method{quantize()} method rounds to a fixed number of decimal places.
If the \constant{Inexact} trap is set, it is also useful for validation:

\begin{verbatim}
>>> TWOPLACES = Decimal(10) ** -2       # same as Decimal('0.01')

>>> # Round to two places
>>> Decimal("3.214").quantize(TWOPLACES)
Decimal("3.21")

>>> # Validate that a number does not exceed two places 
>>> Decimal("3.21").quantize(TWOPLACES, context=Context(traps=[Inexact]))
Decimal("3.21")

>>> Decimal("3.214").quantize(TWOPLACES, context=Context(traps=[Inexact]))
Traceback (most recent call last):
   ...
Inexact: Changed in rounding
\end{verbatim}


Q.  Once I have valid two place inputs, how do I maintain that invariant
throughout an application?

A.  Some operations like addition and subtraction automatically preserve fixed
point.  Others, like multiplication and division, change the number of decimal
places and need to be followed-up with a \method{quantize()} step.


Q.  There are many ways to express the same value.  The numbers
\constant{200}, \constant{200.000}, \constant{2E2}, and \constant{.02E+4} all
have the same value at various precisions. Is there a way to transform them to
a single recognizable canonical value?

A.  The \method{normalize()} method maps all equivalent values to a single
representative:

\begin{verbatim}
>>> values = map(Decimal, '200 200.000 2E2 .02E+4'.split())
>>> [v.normalize() for v in values]
[Decimal("2E+2"), Decimal("2E+2"), Decimal("2E+2"), Decimal("2E+2")]
\end{verbatim}


Q.  Some decimal values always print with exponential notation.  Is there
a way to get a non-exponential representation?

A.  For some values, exponential notation is the only way to express
the number of significant places in the coefficient.  For example,
expressing \constant{5.0E+3} as \constant{5000} keeps the value
constant but cannot show the original's two-place significance.


Q.  Is there a way to convert a regular float to a \class{Decimal}?

A.  Yes, all binary floating point numbers can be exactly expressed as a
Decimal.  An exact conversion may take more precision than intuition would
suggest, so trapping \constant{Inexact} will signal a need for more precision:

\begin{verbatim}
def floatToDecimal(f):
    "Convert a floating point number to a Decimal with no loss of information"
    # Transform (exactly) a float to a mantissa (0.5 <= abs(m) < 1.0) and an
    # exponent.  Double the mantissa until it is an integer.  Use the integer
    # mantissa and exponent to compute an equivalent Decimal.  If this cannot
    # be done exactly, then retry with more precision.

    mantissa, exponent = math.frexp(f)
    while mantissa != int(mantissa):
        mantissa *= 2.0
        exponent -= 1
    mantissa = int(mantissa)

    oldcontext = getcontext()
    setcontext(Context(traps=[Inexact]))
    try:
        while True:
            try:
               return mantissa * Decimal(2) ** exponent
            except Inexact:
                getcontext().prec += 1
    finally:
        setcontext(oldcontext)
\end{verbatim}


Q.  Why isn't the \function{floatToDecimal()} routine included in the module?

A.  There is some question about whether it is advisable to mix binary and
decimal floating point.  Also, its use requires some care to avoid the
representation issues associated with binary floating point:

\begin{verbatim}
>>> floatToDecimal(1.1)
Decimal("1.100000000000000088817841970012523233890533447265625")
\end{verbatim}


Q.  Within a complex calculation, how can I make sure that I haven't gotten a
spurious result because of insufficient precision or rounding anomalies.

A.  The decimal module makes it easy to test results.  A best practice is to
re-run calculations using greater precision and with various rounding modes.
Widely differing results indicate insufficient precision, rounding mode
issues, ill-conditioned inputs, or a numerically unstable algorithm.


Q.  I noticed that context precision is applied to the results of operations
but not to the inputs.  Is there anything to watch out for when mixing
values of different precisions?

A.  Yes.  The principle is that all values are considered to be exact and so
is the arithmetic on those values.  Only the results are rounded.  The
advantage for inputs is that ``what you type is what you get''.  A
disadvantage is that the results can look odd if you forget that the inputs
haven't been rounded:

\begin{verbatim}
>>> getcontext().prec = 3
>>> Decimal('3.104') + D('2.104')
Decimal("5.21")
>>> Decimal('3.104') + D('0.000') + D('2.104')
Decimal("5.20")
\end{verbatim}

The solution is either to increase precision or to force rounding of inputs
using the unary plus operation:

\begin{verbatim}
>>> getcontext().prec = 3
>>> +Decimal('1.23456789')      # unary plus triggers rounding
Decimal("1.23")
\end{verbatim}

Alternatively, inputs can be rounded upon creation using the
\method{Context.create_decimal()} method:

\begin{verbatim}
>>> Context(prec=5, rounding=ROUND_DOWN).create_decimal('1.2345678')
Decimal("1.2345")
\end{verbatim}
