\section{\module{DocXMLRPCServer} ---
         Self-documenting XML-RPC server}

\declaremodule{standard}{DocXMLRPCServer}
\modulesynopsis{Self-documenting XML-RPC server implementation.}
\moduleauthor{Brian Quinlan}{brianq@activestate.com}
\sectionauthor{Brian Quinlan}{brianq@activestate.com}

\versionadded{2.3}

The \module{DocXMLRPCServer} module extends the classes found in
\module{SimpleXMLRPCServer} to serve HTML documentation in response to
HTTP GET requests. Servers can either be free standing, using
\class{DocXMLRPCServer}, or embedded in a CGI environment, using
\class{DocCGIXMLRPCRequestHandler}.

\begin{classdesc}{DocXMLRPCServer}{addr\optional{, 
                                   requestHandler\optional{, logRequests}}}

Create a new server instance. All parameters have the same meaning as
for \class{SimpleXMLRPCServer.SimpleXMLRPCServer};
\var{requestHandler} defaults to \class{DocXMLRPCRequestHandler}.

\end{classdesc}

\begin{classdesc}{DocCGIXMLRPCRequestHandler}{}

Create a new instance to handle XML-RPC requests in a CGI environment.

\end{classdesc}

\begin{classdesc}{DocXMLRPCRequestHandler}{}

Create a new request handler instance. This request handler supports
XML-RPC POST requests, documentation GET requests, and modifies
logging so that the \var{logRequests} parameter to the
\class{DocXMLRPCServer} constructor parameter is honored.

\end{classdesc}

\subsection{DocXMLRPCServer Objects \label{doc-xmlrpc-servers}}

The \class{DocXMLRPCServer} class is derived from
\class{SimpleXMLRPCServer.SimpleXMLRPCServer} and provides a means of
creating self-documenting, stand alone XML-RPC servers. HTTP POST
requests are handled as XML-RPC method calls. HTTP GET requests are
handled by generating pydoc-style HTML documentation. This allows a
server to provide its own web-based documentation.

\begin{methoddesc}{set_server_title}{server_title}

Set the title used in the generated HTML documentation. This title
will be used inside the HTML "title" element.

\end{methoddesc}

\begin{set_server_name}{server_name}

Set the name used in the generated HTML documentation. This name will
appear at the top of the generated documentation inside a "h1"
element.

\end{methoddesc}


\begin{set_server_documentation}{server_documentation}

Set the description used in the generated HTML documentation. This
description will appear as a paragraph, below the server name, in the
documentation.

\end{methoddesc}

\subsection{DocCGIXMLRPCRequestHandler}

The \class{DocCGIXMLRPCRequestHandler} class is derived from
\class{SimpleXMLRPCServer.CGIXMLRPCRequestHandler} and provides a means
of creating self-documenting, XML-RPC CGI scripts. HTTP POST requests
are handled as XML-RPC method calls. HTTP GET requests are handled by
generating pydoc-style HTML documentation. This allows a server to
provide its own web-based documentation.

\begin{methoddesc}{set_server_title}{server_title}

Set the title used in the generated HTML documentation. This title
will be used inside the HTML "title" element.

\end{methoddesc}

\begin{methoddesc}{set_server_name}{server_name}

Set the name used in the generated HTML documentation. This name will
appear at the top of the generated documentation inside a "h1"
element.

\end{methoddesc}


\begin{methoddesc}{set_server_documentation}{server_documentation}

Set the description used in the generated HTML documentation. This
description will appear as a paragraph, below the server name, in the
documentation.

\end{methoddesc}
