\section{\module{asyncore} ---
         Asyncronous socket handler}

\declaremodule{builtin}{asyncore}
\modulesynopsis{A base class for developing asyncronous socket 
                handling services.}
\moduleauthor{Sam Rushing}{rushing@nightmare.com}
\sectionauthor{Christopher Petrilli}{petrilli@amber.org}
% Heavily adapted from original documentation by Sam Rushing.

This module provides the basic infrastructure for writing asyncronous 
socket service clients and servers.

%\subsection{Why Asyncronous?}

There are only two ways to have a program on a single processor do 
``more than one thing at a time.'' Multi-threaded programming is the 
simplest and most popular way to do it, but there is another very 
different technique, that lets youhave nearly all the advantages of 
multi-threading, without actually using multiple threads.  it's really 
only practical if your program is largely I/O bound.  If your program 
is CPU bound, then pre-emtpive scheduled threads are probably what 
you really need. Network servers are rarely CPU-bound, however.

If your operating system supports the \cfunction{select()} system call 
in its I/O library (and nearly all do), then you can use it to juggle 
multiple communication channels at once; doing other work while your 
I/O is taking place in the ``background.''  Although this strategy can 
seem strange and complex, especially at first, it is in many ways 
easier to understand and control than multi-threaded programming.  
The module documented here solves manyof the difficult problems for 
you, making the task of building sophisticated high-performance 
network servers and clients a snap.

\begin{classdesc}{dispatcher}{}
  The first class we will introduce is the \class{dispatcher} class. 
  This is a thin wrapper around a low-level socket object.  To make 
  it more useful, it has a few methods for event-handling on it.  
  Otherwise, it can be treated as a normal non-blocking socket object.

  The direct interface between the select loop and the socket object
  are the \method{handle_read_event()} and 
  \method{handle_write_event()} methods. These are called whenever an 
  object `fires' that event.

  The firing of these low-level events can tell us whether certain 
  higher-level events have taken place, depending on the timing and 
  the state of the connection.  For example, if we have asked for a 
  socket to connect to another host, we know that the connection has 
  been made when the socket fires a write event (at this point you 
  know that you may write to it with the expectation of success).  
  The implied higher-level events are:

  \begin{tableii}{l|l}{code}{Event}{Description}
    \lineii{handle_connect()}{Implied by a write event}
    \lineii{handle_close()}{Implied by a read event with no data available}
    \lineii{handle_accept()}{Implied by a read event on a listening socket}
  \end{tableii}
\end{classdesc}

This set of user-level events is larger than the basics.  The 
full set of methods that can be overridden in your subclass are:

\begin{methoddesc}{handle_read}{}
  Called when there is new data to be read from a socket.
\end{methoddesc}

\begin{methoddesc}{handle_write}{}
  Called when there is an attempt to write data to the object.  
  Often this method will implement the necessary buffering for 
  performance.  For example:

\begin{verbatim}
def handle_write(self):
    sent = self.send(self.buffer)
    self.buffer = self.buffer[sent:]
\end{verbatim}
\end{methoddesc}

\begin{methoddesc}{handle_expt}{}
  Called when there is out of band (OOB) data for a socket 
  connection.  This will almost never happen, as OOB is 
  tenuously supported and rarely used.
\end{methoddesc}

\begin{methoddesc}{handle_connect}{}
  Called when the socket actually makes a connection.  This 
  might be used to send a ``welcome'' banner, or something 
  similar.
\end{methoddesc}

\begin{methoddesc}{handle_close}{}
  Called when the socket is closed.
\end{methoddesc}

\begin{methoddesc}{handle_accept}{}
  Called on listening sockets when they actually accept a new 
  connection.
\end{methoddesc}

\begin{methoddesc}{readable}{}
  Each time through the \method{select()} loop, the set of sockets 
  is scanned, and this method is called to see if there is any 
  interest in reading.  The default method simply returns \code{1}, 
  indicating that by default, all channels will be interested.
\end{methoddesc}

\begin{methoddesc}{writeable}{}
  Each time through the \method{select()} loop, the set of sockets 
  is scanned, and this method is called to see if there is any 
  interest in writing.  The default method simply returns \code{1}, 
  indiciating that by default, all channels will be interested.
\end{methoddesc}

In addition, there are the basic methods needed to construct and
manipulate ``channels,'' which are what we will call the socket
connections in this context. Note that most of these are nearly 
identical to their \class{socket} partners.

\begin{methoddesc}{create_socket}{family, type}
  This is identical to the creation of a normal socket, and 
  will use the same options for creation.  This means you will 
  need to reference the \refmodule{socket} module.
\end{methoddesc}

\begin{methoddesc}{connect}{address}
  As with the normal \class{socket} object, \var{address} is a 
  tuple with the first element the host to connect to, and the 
  second the port.
\end{methoddesc}

\begin{methoddesc}{send}{data}
  Send \var{data} out the socket.
\end{methoddesc}

\begin{methoddesc}{recv}{buffer_size}
  Read at most \var{buffer_size} bytes from the socket.
\end{methoddesc}

\begin{methoddesc}{listen}{\optional{backlog}}
  Listen for connections made to the socket.  The \var{backlog}
  argument specifies the maximum number of queued connections
  and should be at least 1; the maximum value is
  system-dependent (usually 5).
\end{methoddesc}

\begin{methoddesc}{bind}{address}
  Bind the socket to \var{address}.  The socket must not already
  be bound.  (The format of \var{address} depends on the address
  family --- see above.)
\end{methoddesc}

\begin{methoddesc}{accept}{}
  Accept a connection.  The socket must be bound to an address
  and listening for connections.  The return value is a pair
  \code{(\var{conn}, \var{address})} where \var{conn} is a
  \emph{new} socket object usable to send and receive data on
  the connection, and \var{address} is the address bound to the
  socket on the other end of the connection.
\end{methoddesc}

\begin{methoddesc}{close}{}
  Close the socket.  All future operations on the socket object
  will fail.  The remote end will receive no more data (after
  queued data is flushed).  Sockets are automatically closed
  when they are garbage-collected.
\end{methoddesc}


\subsection{Example basic HTTP client \label{asyncore-example}}

As a basic example, below is a very basic HTTP client that uses the 
\class{dispatcher} class to implement its socket handling:

\begin{verbatim}
class http_client(asyncore.dispatcher):
    def __init__(self, host,path):
        asyncore.dispatcher.__init__(self)
        self.path = path
        self.create_socket(socket.AF_INET, socket.SOCK_STREAM)
        self.connect( (host, 80) )
        self.buffer = 'GET %s HTTP/1.0\r\b\r\n' % self.path
        
    def handle_connect(self):
        pass
        
    def handle_read(self):
        data = self.recv(8192)
        print data
        
    def writeable(self):
        return (len(self.buffer) > 0)
    
    def handle_write(self):
        sent = self.send(self.buffer)
        self.buffer = self.buffer[sent:]
\end{verbatim}
