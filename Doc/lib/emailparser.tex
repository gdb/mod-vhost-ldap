\declaremodule{standard}{email.Parser}
\modulesynopsis{Parse flat text email messages to produce a message
	        object tree.}

Message object trees can be created in one of two ways: they can be
created from whole cloth by instantiating \class{Message} objects and
stringing them together via \method{add_payload()} and
\method{set_payload()} calls, or they can be created by parsing a flat text
representation of the email message.

The \module{email} package provides a standard parser that understands
most email document structures, including MIME documents.  You can
pass the parser a string or a file object, and the parser will return
to you the root \class{Message} instance of the object tree.  For
simple, non-MIME messages the payload of this root object will likely
be a string containing the text of the message.  For MIME
messages, the root object will return true from its
\method{is_multipart()} method, and the subparts can be accessed via
the \method{get_payload()} and \method{walk()} methods.

Note that the parser can be extended in limited ways, and of course
you can implement your own parser completely from scratch.  There is
no magical connection between the \module{email} package's bundled
parser and the \class{Message} class, so your custom parser can create
message object trees any way it finds necessary.

The primary parser class is \class{Parser} which parses both the
headers and the payload of the message.  In the case of
\mimetype{multipart} messages, it will recursively parse the body of
the container message.  The \module{email.Parser} module also provides
a second class, called \class{HeaderParser} which can be used if
you're only interested in the headers of the message.
\class{HeaderParser} can be much faster in this situations, since it
does not attempt to parse the message body, instead setting the
payload to the raw body as a string.  \class{HeaderParser} has the
same API as the \class{Parser} class.

\subsubsection{Parser class API}

\begin{classdesc}{Parser}{\optional{_class}}
The constructor for the \class{Parser} class takes a single optional
argument \var{_class}.  This must be a callable factory (such as a
function or a class), and it is used whenever a sub-message object
needs to be created.  It defaults to \class{Message} (see
\refmodule{email.Message}).  The factory will be called without
arguments.
\end{classdesc}

The other public \class{Parser} methods are:

\begin{methoddesc}[Parser]{parse}{fp}
Read all the data from the file-like object \var{fp}, parse the
resulting text, and return the root message object.  \var{fp} must
support both the \method{readline()} and the \method{read()} methods
on file-like objects.

The text contained in \var{fp} must be formatted as a block of \rfc{2822}
style headers and header continuation lines, optionally preceeded by a
\emph{Unix-From} header.  The header block is terminated either by the
end of the data or by a blank line.  Following the header block is the
body of the message (which may contain MIME-encoded subparts).
\end{methoddesc}

\begin{methoddesc}[Parser]{parsestr}{text}
Similar to the \method{parse()} method, except it takes a string
object instead of a file-like object.  Calling this method on a string
is exactly equivalent to wrapping \var{text} in a \class{StringIO}
instance first and calling \method{parse()}.
\end{methoddesc}

Since creating a message object tree from a string or a file object is
such a common task, two functions are provided as a convenience.  They
are available in the top-level \module{email} package namespace.

\begin{funcdesc}{message_from_string}{s\optional{, _class}}
Return a message object tree from a string.  This is exactly
equivalent to \code{Parser().parsestr(s)}.  Optional \var{_class} is
interpreted as with the \class{Parser} class constructor.
\end{funcdesc}

\begin{funcdesc}{message_from_file}{fp\optional{, _class}}
Return a message object tree from an open file object.  This is exactly
equivalent to \code{Parser().parse(fp)}.  Optional \var{_class} is
interpreted as with the \class{Parser} class constructor.
\end{funcdesc}

Here's an example of how you might use this at an interactive Python
prompt:

\begin{verbatim}
>>> import email
>>> msg = email.message_from_string(myString)
\end{verbatim}

\subsubsection{Additional notes}

Here are some notes on the parsing semantics:

\begin{itemize}
\item Most non-\mimetype{multipart} type messages are parsed as a single
      message object with a string payload.  These objects will return
      0 for \method{is_multipart()}.
\item One exception is for \mimetype{message/delivery-status} type
      messages.  Because the body of such messages consist of
      blocks of headers, \class{Parser} will create a non-multipart
      object containing non-multipart subobjects for each header
      block.
\item Another exception is for \mimetype{message/*} types (more
      general than \mimetype{message/delivery-status}).  These are
      typically \mimetype{message/rfc822} messages, represented as a
      non-multipart object containing a singleton payload which is
      another non-multipart \class{Message} instance.
\end{itemize}
