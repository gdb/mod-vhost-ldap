\section{\module{mailbox} ---
         Read various mailbox formats}

\declaremodule{standard}{mailbox}
\modulesynopsis{Read various mailbox formats.}


This module defines a number of classes that allow easy and uniform
access to mail messages in a (\UNIX{}) mailbox.

\begin{classdesc}{UnixMailbox}{fp\optional{, factory}}
Access to a classic \UNIX{}-style mailbox, where all messages are
contained in a single file and separate by ``From '' (a.k.a ``From_'')
lines.  The file object \var{fp} points to the mailbox file.  Optional
\var{factory} is a callable that should create new message objects.
It is called with one argument, \var{fp} by the \method{next()}
method.  The default is the \class{rfc822.Message} class (see the
\refmodule{rfc822} module).

For maximum portability, messages in a \UNIX{}-style mailbox are
separated by any line that begins exactly with the letters \emph{F},
\emph{r}, \emph{o}, \emph{m}, \emph{[space]} if preceded by exactly two
newlines.  Because of the wide-range of variations in practice,
nothing else on the From_ line should be considered.  However, the
current implementation doesn't check for the leading two newlines.
This is usually fine for most applications.

The \class{UnixMailbox} class implements a more strict version of
From_ line checking, using a regular expression that usually correctly
matched From_ delimiters.  It considers delimiter line to be separated
by ``From name time'' lines.  For maximum portability, use the
\class{PortableUnixMailbox} class instead.  This
class is completely identical to \class{UnixMailbox} except that
individual messages are separated by only ``From '' lines.

For more
information see
\url{http://home.netscape.com/eng/mozilla/2.0/relnotes/demo/content-length.html}.
\end{classdesc}

\begin{classdesc}{MmdfMailbox}{fp\optional{, factory}}
Access an MMDF-style mailbox, where all messages are contained
in a single file and separated by lines consisting of 4 control-A
characters.  The file object \var{fp} points to the mailbox file.
Optional \var{factory} is as with the \class{UnixMailbox} class.
\end{classdesc}

\begin{classdesc}{MHMailbox}{dirname\optional{, factory}}
Access an MH mailbox, a directory with each message in a separate
file with a numeric name.
The name of the mailbox directory is passed in \var{dirname}.
\var{factory} is as with the \class{UnixMailbox} class.
\end{classdesc}

\begin{classdesc}{Maildir}{dirname\optional{, factory}}
Access a Qmail mail directory.  All new and current mail for the
mailbox specified by \var{dirname} is made available.
\var{factory} is as with the \class{UnixMailbox} class.
\end{classdesc}

\begin{classdesc}{BabylMailbox}{fp\optional{, factory}}
Access a Babyl mailbox, which is similar to an MMDF mailbox.  Mail
messages start with a line containing only \code{'*** EOOH ***'} and
end with a line containing only \code{'\e{}037\e{}014'}.
\var{factory} is as with the \class{UnixMailbox} class.
\end{classdesc}


\subsection{Mailbox Objects \label{mailbox-objects}}

All implementations of Mailbox objects have one externally visible
method:

\begin{methoddesc}[mailbox]{next}{}
Return the next message in the mailbox, created with the optional
\var{factory} argument passed into the mailbox object's constructor.
By defaul this is an \class{rfc822.Message}
object (see the \refmodule{rfc822} module).  Depending on the mailbox
implementation the \var{fp} attribute of this object may be a true
file object or a class instance simulating a file object, taking care
of things like message boundaries if multiple mail messages are
contained in a single file, etc.  If no more messages are available,
this method returns \code{None}.
\end{methoddesc}
