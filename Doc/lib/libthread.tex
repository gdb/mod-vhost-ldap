\section{Built-in Module \sectcode{thread}}
\label{module-thread}
\bimodindex{thread}

This module provides low-level primitives for working with multiple
threads (a.k.a.\ \dfn{light-weight processes} or \dfn{tasks}) --- multiple
threads of control sharing their global data space.  For
synchronization, simple locks (a.k.a.\ \dfn{mutexes} or \dfn{binary
semaphores}) are provided.

The module is optional and supported on SGI IRIX 4.x and 5.x and Sun
Solaris 2.x systems, as well as on systems that have a PTHREAD
implementation (e.g.\ KSR).

It defines the following constant and functions:

\renewcommand{\indexsubitem}{(in module thread)}
\begin{excdesc}{error}
Raised on thread-specific errors.
\end{excdesc}

\begin{funcdesc}{start_new_thread}{func\, arg}
Start a new thread.  The thread executes the function \var{func}
with the argument list \var{arg} (which must be a tuple).  When the
function returns, the thread silently exits.  When the function
terminates with an unhandled exception, a stack trace is printed and
then the thread exits (but other threads continue to run).
\end{funcdesc}

\begin{funcdesc}{exit}{}
This is a shorthand for \code{thread.exit_thread()}.
\end{funcdesc}

\begin{funcdesc}{exit_thread}{}
Raise the \code{SystemExit} exception.  When not caught, this will
cause the thread to exit silently.
\end{funcdesc}

%\begin{funcdesc}{exit_prog}{status}
%Exit all threads and report the value of the integer argument
%\var{status} as the exit status of the entire program.
%\strong{Caveat:} code in pending \code{finally} clauses, in this thread
%or in other threads, is not executed.
%\end{funcdesc}

\begin{funcdesc}{allocate_lock}{}
Return a new lock object.  Methods of locks are described below.  The
lock is initially unlocked.
\end{funcdesc}

\begin{funcdesc}{get_ident}{}
Return the `thread identifier' of the current thread.  This is a
nonzero integer.  Its value has no direct meaning; it is intended as a
magic cookie to be used e.g. to index a dictionary of thread-specific
data.  Thread identifiers may be recycled when a thread exits and
another thread is created.
\end{funcdesc}

Lock objects have the following methods:

\renewcommand{\indexsubitem}{(lock method)}
\begin{funcdesc}{acquire}{\optional{waitflag}}
Without the optional argument, this method acquires the lock
unconditionally, if necessary waiting until it is released by another
thread (only one thread at a time can acquire a lock --- that's their
reason for existence), and returns \code{None}.  If the integer
\var{waitflag} argument is present, the action depends on its value:\
if it is zero, the lock is only acquired if it can be acquired
immediately without waiting, while if it is nonzero, the lock is
acquired unconditionally as before.  If an argument is present, the
return value is 1 if the lock is acquired successfully, 0 if not.
\end{funcdesc}

\begin{funcdesc}{release}{}
Releases the lock.  The lock must have been acquired earlier, but not
necessarily by the same thread.
\end{funcdesc}

\begin{funcdesc}{locked}{}
Return the status of the lock:\ 1 if it has been acquired by some
thread, 0 if not.
\end{funcdesc}

{\bf Caveats:}

\begin{itemize}
\item
Threads interact strangely with interrupts: the
\code{KeyboardInterrupt} exception will be received by an arbitrary
thread.  (When the \code{signal} module is available, interrupts
always go to the main thread.)

\item
Calling \code{sys.exit()} or raising the \code{SystemExit} is
equivalent to calling \code{thread.exit_thread()}.

\item
Not all built-in functions that may block waiting for I/O allow other
threads to run.  (The most popular ones (\code{sleep}, \code{read},
\code{select}) work as expected.)

\end{itemize}
