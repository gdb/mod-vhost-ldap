\section{\module{string} ---
         Common string operations}

\declaremodule{standard}{string}
\modulesynopsis{Common string operations.}


This module defines some constants useful for checking character
classes and some useful string functions.  See the module
\refmodule{re}\refstmodindex{re} for string functions based on regular
expressions.

The constants defined in this module are:

\begin{datadesc}{ascii_letters}
  The concatenation of the \constant{ascii_lowercase} and
  \constant{ascii_uppercase} constants described below.  This value is
  not locale-dependent.
\end{datadesc}

\begin{datadesc}{ascii_lowercase}
  The lowercase letters \code{'abcdefghijklmnopqrstuvwxyz'}.  This
  value is not locale-dependent and will not change.
\end{datadesc}

\begin{datadesc}{ascii_uppercase}
  The uppercase letters \code{'ABCDEFGHIJKLMNOPQRSTUVWXYZ'}.  This
  value is not locale-dependent and will not change.
\end{datadesc}

\begin{datadesc}{digits}
  The string \code{'0123456789'}.
\end{datadesc}

\begin{datadesc}{hexdigits}
  The string \code{'0123456789abcdefABCDEF'}.
\end{datadesc}

\begin{datadesc}{letters}
  The concatenation of the strings \constant{lowercase} and
  \constant{uppercase} described below.  The specific value is
  locale-dependent, and will be updated when
  \function{locale.setlocale()} is called.
\end{datadesc}

\begin{datadesc}{lowercase}
  A string containing all the characters that are considered lowercase
  letters.  On most systems this is the string
  \code{'abcdefghijklmnopqrstuvwxyz'}.  Do not change its definition ---
  the effect on the routines \function{upper()} and
  \function{swapcase()} is undefined.  The specific value is
  locale-dependent, and will be updated when
  \function{locale.setlocale()} is called.
\end{datadesc}

\begin{datadesc}{octdigits}
  The string \code{'01234567'}.
\end{datadesc}

\begin{datadesc}{punctuation}
  String of \ASCII{} characters which are considered punctuation
  characters in the \samp{C} locale.
\end{datadesc}

\begin{datadesc}{printable}
  String of characters which are considered printable.  This is a
  combination of \constant{digits}, \constant{letters},
  \constant{punctuation}, and \constant{whitespace}.
\end{datadesc}

\begin{datadesc}{uppercase}
  A string containing all the characters that are considered uppercase
  letters.  On most systems this is the string
  \code{'ABCDEFGHIJKLMNOPQRSTUVWXYZ'}.  Do not change its definition ---
  the effect on the routines \function{lower()} and
  \function{swapcase()} is undefined.  The specific value is
  locale-dependent, and will be updated when
  \function{locale.setlocale()} is called.
\end{datadesc}

\begin{datadesc}{whitespace}
  A string containing all characters that are considered whitespace.
  On most systems this includes the characters space, tab, linefeed,
  return, formfeed, and vertical tab.  Do not change its definition ---
  the effect on the routines \function{strip()} and \function{split()}
  is undefined.
\end{datadesc}


Many of the functions provided by this module are also defined as
methods of string and Unicode objects; see ``String Methods'' (section
\ref{string-methods}) for more information on those.
The functions defined in this module are:

\begin{funcdesc}{atof}{s}
  \deprecated{2.0}{Use the \function{float()} built-in function.}
  Convert a string to a floating point number.  The string must have
  the standard syntax for a floating point literal in Python,
  optionally preceded by a sign (\samp{+} or \samp{-}).  Note that
  this behaves identical to the built-in function
  \function{float()}\bifuncindex{float} when passed a string.

  \note{When passing in a string, values for NaN\index{NaN}
  and Infinity\index{Infinity} may be returned, depending on the
  underlying C library.  The specific set of strings accepted which
  cause these values to be returned depends entirely on the C library
  and is known to vary.}
\end{funcdesc}

\begin{funcdesc}{atoi}{s\optional{, base}}
  \deprecated{2.0}{Use the \function{int()} built-in function.}
  Convert string \var{s} to an integer in the given \var{base}.  The
  string must consist of one or more digits, optionally preceded by a
  sign (\samp{+} or \samp{-}).  The \var{base} defaults to 10.  If it
  is 0, a default base is chosen depending on the leading characters
  of the string (after stripping the sign): \samp{0x} or \samp{0X}
  means 16, \samp{0} means 8, anything else means 10.  If \var{base}
  is 16, a leading \samp{0x} or \samp{0X} is always accepted, though
  not required.  This behaves identically to the built-in function
  \function{int()} when passed a string.  (Also note: for a more
  flexible interpretation of numeric literals, use the built-in
  function \function{eval()}\bifuncindex{eval}.)
\end{funcdesc}

\begin{funcdesc}{atol}{s\optional{, base}}
  \deprecated{2.0}{Use the \function{long()} built-in function.}
  Convert string \var{s} to a long integer in the given \var{base}.
  The string must consist of one or more digits, optionally preceded
  by a sign (\samp{+} or \samp{-}).  The \var{base} argument has the
  same meaning as for \function{atoi()}.  A trailing \samp{l} or
  \samp{L} is not allowed, except if the base is 0.  Note that when
  invoked without \var{base} or with \var{base} set to 10, this
  behaves identical to the built-in function
  \function{long()}\bifuncindex{long} when passed a string.
\end{funcdesc}

\begin{funcdesc}{capitalize}{word}
  Return a copy of \var{word} with only its first character capitalized.
\end{funcdesc}

\begin{funcdesc}{capwords}{s}
  Split the argument into words using \function{split()}, capitalize
  each word using \function{capitalize()}, and join the capitalized
  words using \function{join()}.  Note that this replaces runs of
  whitespace characters by a single space, and removes leading and
  trailing whitespace.
\end{funcdesc}

\begin{funcdesc}{expandtabs}{s\optional{, tabsize}}
  Expand tabs in a string, i.e.\ replace them by one or more spaces,
  depending on the current column and the given tab size.  The column
  number is reset to zero after each newline occurring in the string.
  This doesn't understand other non-printing characters or escape
  sequences.  The tab size defaults to 8.
\end{funcdesc}

\begin{funcdesc}{find}{s, sub\optional{, start\optional{,end}}}
  Return the lowest index in \var{s} where the substring \var{sub} is
  found such that \var{sub} is wholly contained in
  \code{\var{s}[\var{start}:\var{end}]}.  Return \code{-1} on failure.
  Defaults for \var{start} and \var{end} and interpretation of
  negative values is the same as for slices.
\end{funcdesc}

\begin{funcdesc}{rfind}{s, sub\optional{, start\optional{, end}}}
  Like \function{find()} but find the highest index.
\end{funcdesc}

\begin{funcdesc}{index}{s, sub\optional{, start\optional{, end}}}
  Like \function{find()} but raise \exception{ValueError} when the
  substring is not found.
\end{funcdesc}

\begin{funcdesc}{rindex}{s, sub\optional{, start\optional{, end}}}
  Like \function{rfind()} but raise \exception{ValueError} when the
  substring is not found.
\end{funcdesc}

\begin{funcdesc}{count}{s, sub\optional{, start\optional{, end}}}
  Return the number of (non-overlapping) occurrences of substring
  \var{sub} in string \code{\var{s}[\var{start}:\var{end}]}.
  Defaults for \var{start} and \var{end} and interpretation of
  negative values are the same as for slices.
\end{funcdesc}

\begin{funcdesc}{lower}{s}
  Return a copy of \var{s}, but with upper case letters converted to
  lower case.
\end{funcdesc}

\begin{funcdesc}{maketrans}{from, to}
  Return a translation table suitable for passing to
  \function{translate()} or \function{regex.compile()}, that will map
  each character in \var{from} into the character at the same position
  in \var{to}; \var{from} and \var{to} must have the same length.

  \warning{Don't use strings derived from \constant{lowercase}
  and \constant{uppercase} as arguments; in some locales, these don't have
  the same length.  For case conversions, always use
  \function{lower()} and \function{upper()}.}
\end{funcdesc}

\begin{funcdesc}{split}{s\optional{, sep\optional{, maxsplit}}}
  Return a list of the words of the string \var{s}.  If the optional
  second argument \var{sep} is absent or \code{None}, the words are
  separated by arbitrary strings of whitespace characters (space, tab, 
  newline, return, formfeed).  If the second argument \var{sep} is
  present and not \code{None}, it specifies a string to be used as the 
  word separator.  The returned list will then have one more item
  than the number of non-overlapping occurrences of the separator in
  the string.  The optional third argument \var{maxsplit} defaults to
  0.  If it is nonzero, at most \var{maxsplit} number of splits occur,
  and the remainder of the string is returned as the final element of
  the list (thus, the list will have at most \code{\var{maxsplit}+1}
  elements).

  The behavior of split on an empty string depends on the value of \var{sep}.
  If \var{sep} is not specified, or specified as \code{None}, the result will
  be an empty list.  If \var{sep} is specified as any string, the result will
  be a list containing one element which is an empty string.
\end{funcdesc}

\begin{funcdesc}{rsplit}{s\optional{, sep\optional{, maxsplit}}}
  Return a list of the words of the string \var{s}, scanning \var{s}
  from the end.  To all intents and purposes, the resulting list of
  words is the same as returned by \function{split()}, except when the
  optional third argument \var{maxsplit} is explicitly specified and
  nonzero.  When \var{maxsplit} is nonzero, at most \var{maxsplit}
  number of splits -- the \emph{rightmost} ones -- occur, and the remainder
  of the string is returned as the first element of the list (thus, the
  list will have at most \code{\var{maxsplit}+1} elements).
  \versionadded{2.4}
\end{funcdesc}

\begin{funcdesc}{splitfields}{s\optional{, sep\optional{, maxsplit}}}
  This function behaves identically to \function{split()}.  (In the
  past, \function{split()} was only used with one argument, while
  \function{splitfields()} was only used with two arguments.)
\end{funcdesc}

\begin{funcdesc}{join}{words\optional{, sep}}
  Concatenate a list or tuple of words with intervening occurrences of 
  \var{sep}.  The default value for \var{sep} is a single space
  character.  It is always true that
  \samp{string.join(string.split(\var{s}, \var{sep}), \var{sep})}
  equals \var{s}.
\end{funcdesc}

\begin{funcdesc}{joinfields}{words\optional{, sep}}
  This function behaves identically to \function{join()}.  (In the past, 
  \function{join()} was only used with one argument, while
  \function{joinfields()} was only used with two arguments.)
  Note that there is no \method{joinfields()} method on string
  objects; use the \method{join()} method instead.
\end{funcdesc}

\begin{funcdesc}{lstrip}{s\optional{, chars}}
Return a copy of the string with leading characters removed.  If
\var{chars} is omitted or \code{None}, whitespace characters are
removed.  If given and not \code{None}, \var{chars} must be a string;
the characters in the string will be stripped from the beginning of
the string this method is called on.
\versionchanged[The \var{chars} parameter was added.  The \var{chars}
parameter cannot be passed in earlier 2.2 versions]{2.2.3}
\end{funcdesc}

\begin{funcdesc}{rstrip}{s\optional{, chars}}
Return a copy of the string with trailing characters removed.  If
\var{chars} is omitted or \code{None}, whitespace characters are
removed.  If given and not \code{None}, \var{chars} must be a string;
the characters in the string will be stripped from the end of the
string this method is called on.
\versionchanged[The \var{chars} parameter was added.  The \var{chars}
parameter cannot be passed in earlier 2.2 versions]{2.2.3}
\end{funcdesc}

\begin{funcdesc}{strip}{s\optional{, chars}}
Return a copy of the string with leading and trailing characters
removed.  If \var{chars} is omitted or \code{None}, whitespace
characters are removed.  If given and not \code{None}, \var{chars}
must be a string; the characters in the string will be stripped from
the both ends of the string this method is called on.
\versionchanged[The \var{chars} parameter was added.  The \var{chars}
parameter cannot be passed in earlier 2.2 versions]{2.2.3}
\end{funcdesc}

\begin{funcdesc}{swapcase}{s}
  Return a copy of \var{s}, but with lower case letters
  converted to upper case and vice versa.
\end{funcdesc}

\begin{funcdesc}{translate}{s, table\optional{, deletechars}}
  Delete all characters from \var{s} that are in \var{deletechars} (if 
  present), and then translate the characters using \var{table}, which 
  must be a 256-character string giving the translation for each
  character value, indexed by its ordinal.
\end{funcdesc}

\begin{funcdesc}{upper}{s}
  Return a copy of \var{s}, but with lower case letters converted to
  upper case.
\end{funcdesc}

\begin{funcdesc}{ljust}{s, width}
\funcline{rjust}{s, width}
\funcline{center}{s, width}
  These functions respectively left-justify, right-justify and center
  a string in a field of given width.  They return a string that is at
  least \var{width} characters wide, created by padding the string
  \var{s} with spaces until the given width on the right, left or both
  sides.  The string is never truncated.
\end{funcdesc}

\begin{funcdesc}{zfill}{s, width}
  Pad a numeric string on the left with zero digits until the given
  width is reached.  Strings starting with a sign are handled
  correctly.
\end{funcdesc}

\begin{funcdesc}{replace}{str, old, new\optional{, maxreplace}}
  Return a copy of string \var{str} with all occurrences of substring
  \var{old} replaced by \var{new}.  If the optional argument
  \var{maxreplace} is given, the first \var{maxreplace} occurrences are
  replaced.
\end{funcdesc}
