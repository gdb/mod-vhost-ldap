\section{\module{shutil} ---
         High-level file operations}

\declaremodule{standard}{shutil}
\modulesynopsis{High-level file operations, including copying.}
\sectionauthor{Fred L. Drake, Jr.}{fdrake@acm.org}
% partly based on the docstrings


The \module{shutil} module offers a number of high-level operations on
files and collections of files.  In particular, functions are provided 
which support file copying and removal.
\index{file!copying}
\index{copying files}

\strong{Caveat:}  On MacOS, the resource fork and other metadata are
not used.  For file copies, this means that resources will be lost and 
file type and creator codes will not be correct.


\begin{funcdesc}{copyfile}{src, dst}
  Copy the contents of \var{src} to \var{dst}.  If \var{dst} exists,
  it will be replaced, otherwise it will be created.
\end{funcdesc}

\begin{funcdesc}{copymode}{src, dst}
  Copy the permission bits from \var{src} to \var{dst}.  The file
  contents, owner, and group are unaffected.
\end{funcdesc}

\begin{funcdesc}{copystat}{src, dst}
  Copy the permission bits, last access time, and last modification
  time from \var{src} to \var{dst}.  The file contents, owner, and
  group are unaffected.
\end{funcdesc}

\begin{funcdesc}{copy}{src, dst}
  Copy the file \var{src} to the file or directory \var{dst}.  If
  \var{dst} is a directory, a file with the same basename as \var{src} 
  is created (or overwritten) in the directory specified.  Permission
  bits are copied.
\end{funcdesc}

\begin{funcdesc}{copy2}{src, dst}
  Similar to \function{copy()}, but last access time and last
  modification time are copied as well.  This is similar to the
  \UNIX{} command \program{cp} \programopt{-p}.
\end{funcdesc}

\begin{funcdesc}{copytree}{src, dst\optional{, symlinks}}
  Recursively copy an entire directory tree rooted at \var{src}.  The
  destination directory, named by \var{dst}, must not already exist;
  it will be created.  Individual files are copied using
  \function{copy2()}.  If \var{symlinks} is true, symbolic links in
  the source tree are represented as symbolic links in the new tree;
  if false or omitted, the contents of the linked files are copied to
  the new tree.  Errors are reported to standard output.

  The source code for this should be considered an example rather than 
  a tool.
\end{funcdesc}

\begin{funcdesc}{rmtree}{path\optional{, ignore_errors\optional{, onerror}}}
\index{directory!deleting}
  Delete an entire directory tree.  If \var{ignore_errors} is true,
  errors will be ignored; if false or omitted, errors are handled by
  calling a handler specified by \var{onerror} or raise an exception.

  If \var{onerror} is provided, it must be a callable that accepts
  three parameters: \var{function}, \var{path}, and \var{excinfo}.
  The first parameter, \var{function}, is the function which raised
  the exception; it will be \function{os.remove()} or
  \function{os.rmdir()}.  The second parameter, \var{path}, will be
  the path name passed to \var{function}.  The third parameter,
  \var{excinfo}, will be the exception information return by
  \function{sys.exc_info()}.  Exceptions raised by \var{onerror} will
  not be caught.
\end{funcdesc}


\subsection{Example \label{shutil-example}}

This example is the implementation of the \function{copytree()}
function, described above, with the docstring omitted.  It
demonstrates many of the other functions provided by this module.

\begin{verbatim}
def copytree(src, dst, symlinks=0):
    names = os.listdir(src)
    os.mkdir(dst)
    for name in names:
        srcname = os.path.join(src, name)
        dstname = os.path.join(dst, name)
        try:
            if symlinks and os.path.islink(srcname):
                linkto = os.readlink(srcname)
                os.symlink(linkto, dstname)
            elif os.path.isdir(srcname):
                copytree(srcname, dstname)
            else:
                copy2(srcname, dstname)
            # XXX What about devices, sockets etc.?
        except (IOError, os.error), why:
            print "Can't copy %s to %s: %s" % (`srcname`, `dstname`, str(why))
\end{verbatim}
