\section{Built-in Types \label{types}}

The following sections describe the standard types that are built into
the interpreter.  Historically, Python's built-in types have differed
from user-defined types because it was not possible to use the built-in
types as the basis for object-oriented inheritance. With the 2.2
release this situation has started to change, although the intended
unification of user-defined and built-in types is as yet far from
complete.

The principal built-in types are numerics, sequences, mappings, files
classes, instances and exceptions.
\indexii{built-in}{types}

Some operations are supported by several object types; in particular,
practically all objects can be compared, tested for truth value,
and converted to a string (with the \code{`\textrm{\ldots}`} notation,
the equivalent \function{repr()} function, or the slightly different
\function{str()} function).  The latter
function is implicitly used when an object is written by the
\keyword{print}\stindex{print} statement.
(Information on \ulink{\keyword{print} statement}{../ref/print.html}
and other language statements can be found in the
\citetitle[../ref/ref.html]{Python Reference Manual} and the
\citetitle[../tut/tut.html]{Python Tutorial}.)


\subsection{Truth Value Testing\label{truth}}

Any object can be tested for truth value, for use in an \keyword{if} or
\keyword{while} condition or as operand of the Boolean operations below.
The following values are considered false:
\stindex{if}
\stindex{while}
\indexii{truth}{value}
\indexii{Boolean}{operations}
\index{false}

\begin{itemize}

\item	\code{None}
        \withsubitem{(Built-in object)}{\ttindex{None}}

\item	\code{False}
        \withsubitem{(Built-in object)}{\ttindex{False}}

\item	zero of any numeric type, for example, \code{0}, \code{0L},
        \code{0.0}, \code{0j}.

\item	any empty sequence, for example, \code{''}, \code{()}, \code{[]}.

\item	any empty mapping, for example, \code{\{\}}.

\item	instances of user-defined classes, if the class defines a
        \method{__nonzero__()} or \method{__len__()} method, when that
        method returns the integer zero or \class{bool} value
        \code{False}.\footnote{Additional 
information on these special methods may be found in the
\citetitle[../ref/ref.html]{Python Reference Manual}.}

\end{itemize}

All other values are considered true --- so objects of many types are
always true.
\index{true}

Operations and built-in functions that have a Boolean result always
return \code{0} or \code{False} for false and \code{1} or \code{True}
for true, unless otherwise stated.  (Important exception: the Boolean
operations \samp{or}\opindex{or} and \samp{and}\opindex{and} always
return one of their operands.)
\index{False}
\index{True}

\subsection{Boolean Operations ---
	    \keyword{and}, \keyword{or}, \keyword{not}
	    \label{boolean}}

These are the Boolean operations, ordered by ascending priority:
\indexii{Boolean}{operations}

\begin{tableiii}{c|l|c}{code}{Operation}{Result}{Notes}
  \lineiii{\var{x} or \var{y}}
          {if \var{x} is false, then \var{y}, else \var{x}}{(1)}
  \lineiii{\var{x} and \var{y}}
          {if \var{x} is false, then \var{x}, else \var{y}}{(1)}
  \hline
  \lineiii{not \var{x}}
          {if \var{x} is false, then \code{True}, else \code{False}}{(2)}
\end{tableiii}
\opindex{and}
\opindex{or}
\opindex{not}

\noindent
Notes:

\begin{description}

\item[(1)]
These only evaluate their second argument if needed for their outcome.

\item[(2)]
\samp{not} has a lower priority than non-Boolean operators, so
\code{not \var{a} == \var{b}} is interpreted as \code{not (\var{a} ==
\var{b})}, and \code{\var{a} == not \var{b}} is a syntax error.

\end{description}


\subsection{Comparisons \label{comparisons}}

Comparison operations are supported by all objects.  They all have the
same priority (which is higher than that of the Boolean operations).
Comparisons can be chained arbitrarily; for example, \code{\var{x} <
\var{y} <= \var{z}} is equivalent to \code{\var{x} < \var{y} and
\var{y} <= \var{z}}, except that \var{y} is evaluated only once (but
in both cases \var{z} is not evaluated at all when \code{\var{x} <
\var{y}} is found to be false).
\indexii{chaining}{comparisons}

This table summarizes the comparison operations:

\begin{tableiii}{c|l|c}{code}{Operation}{Meaning}{Notes}
  \lineiii{<}{strictly less than}{}
  \lineiii{<=}{less than or equal}{}
  \lineiii{>}{strictly greater than}{}
  \lineiii{>=}{greater than or equal}{}
  \lineiii{==}{equal}{}
  \lineiii{!=}{not equal}{(1)}
  \lineiii{<>}{not equal}{(1)}
  \lineiii{is}{object identity}{}
  \lineiii{is not}{negated object identity}{}
\end{tableiii}
\indexii{operator}{comparison}
\opindex{==} % XXX *All* others have funny characters < ! >
\opindex{is}
\opindex{is not}

\noindent
Notes:

\begin{description}

\item[(1)]
\code{<>} and \code{!=} are alternate spellings for the same operator.
\code{!=} is the preferred spelling; \code{<>} is obsolescent.

\end{description}

Objects of different types, except different numeric types and different string types, never
compare equal; such objects are ordered consistently but arbitrarily
(so that sorting a heterogeneous array yields a consistent result).
Furthermore, some types (for example, file objects) support only a
degenerate notion of comparison where any two objects of that type are
unequal.  Again, such objects are ordered arbitrarily but
consistently. The \code{<}, \code{<=}, \code{>} and \code{>=}
operators will raise a \exception{TypeError} exception when any operand
is a complex number. 
\indexii{object}{numeric}
\indexii{objects}{comparing}

Instances of a class normally compare as non-equal unless the class
\withsubitem{(instance method)}{\ttindex{__cmp__()}}
defines the \method{__cmp__()} method.  Refer to the
\citetitle[../ref/customization.html]{Python Reference Manual} for
information on the use of this method to effect object comparisons.

\strong{Implementation note:} Objects of different types except
numbers are ordered by their type names; objects of the same types
that don't support proper comparison are ordered by their address.

Two more operations with the same syntactic priority,
\samp{in}\opindex{in} and \samp{not in}\opindex{not in}, are supported
only by sequence types (below).


\subsection{Numeric Types ---
	    \class{int}, \class{float}, \class{long}, \class{complex}
	    \label{typesnumeric}}

There are four distinct numeric types: \dfn{plain integers},
\dfn{long integers}, 
\dfn{floating point numbers}, and \dfn{complex numbers}.
In addition, Booleans are a subtype of plain integers.
Plain integers (also just called \dfn{integers})
are implemented using \ctype{long} in C, which gives them at least 32
bits of precision (\code{sys.maxint} is always set to the maximum
plain integer value for the current platform, the minimum value is 
\code{-sys.maxint - 1}).  Long integers have unlimited precision.
Floating point numbers are implemented using \ctype{double} in C.
All bets on their precision are off unless you happen to know the
machine you are working with.
\obindex{numeric}
\obindex{Boolean}
\obindex{integer}
\obindex{long integer}
\obindex{floating point}
\obindex{complex number}
\indexii{C}{language}

Complex numbers have a real and imaginary part, which are each
implemented using \ctype{double} in C.  To extract these parts from
a complex number \var{z}, use \code{\var{z}.real} and \code{\var{z}.imag}.

Numbers are created by numeric literals or as the result of built-in
functions and operators.  Unadorned integer literals (including hex
and octal numbers) yield plain integers unless the value they denote
is too large to be represented as a plain integer, in which case
they yield a long integer.  Integer literals with an
\character{L} or \character{l} suffix yield long integers
(\character{L} is preferred because \samp{1l} looks too much like
eleven!).  Numeric literals containing a decimal point or an exponent
sign yield floating point numbers.  Appending \character{j} or
\character{J} to a numeric literal yields a complex number with a
zero real part. A complex numeric literal is the sum of a real and
an imaginary part.
\indexii{numeric}{literals}
\indexii{integer}{literals}
\indexiii{long}{integer}{literals}
\indexii{floating point}{literals}
\indexii{complex number}{literals}
\indexii{hexadecimal}{literals}
\indexii{octal}{literals}

Python fully supports mixed arithmetic: when a binary arithmetic
operator has operands of different numeric types, the operand with the
``narrower'' type is widened to that of the other, where plain
integer is narrower than long integer is narrower than floating point is
narrower than complex.
Comparisons between numbers of mixed type use the same rule.\footnote{
	As a consequence, the list \code{[1, 2]} is considered equal
        to \code{[1.0, 2.0]}, and similarly for tuples.
} The constructors \function{int()}, \function{long()}, \function{float()},
and \function{complex()} can be used
to produce numbers of a specific type.
\index{arithmetic}
\bifuncindex{int}
\bifuncindex{long}
\bifuncindex{float}
\bifuncindex{complex}

All numeric types (except complex) support the following operations,
sorted by ascending priority (operations in the same box have the same
priority; all numeric operations have a higher priority than
comparison operations):

\begin{tableiii}{c|l|c}{code}{Operation}{Result}{Notes}
  \lineiii{\var{x} + \var{y}}{sum of \var{x} and \var{y}}{}
  \lineiii{\var{x} - \var{y}}{difference of \var{x} and \var{y}}{}
  \hline
  \lineiii{\var{x} * \var{y}}{product of \var{x} and \var{y}}{}
  \lineiii{\var{x} / \var{y}}{quotient of \var{x} and \var{y}}{(1)}
  \lineiii{\var{x} // \var{y}}{(floored) quotient of \var{x} and \var{y}}{(5)}
  \lineiii{\var{x} \%{} \var{y}}{remainder of \code{\var{x} / \var{y}}}{(4)}
  \hline
  \lineiii{-\var{x}}{\var{x} negated}{}
  \lineiii{+\var{x}}{\var{x} unchanged}{}
  \hline
  \lineiii{abs(\var{x})}{absolute value or magnitude of \var{x}}{}
  \lineiii{int(\var{x})}{\var{x} converted to integer}{(2)}
  \lineiii{long(\var{x})}{\var{x} converted to long integer}{(2)}
  \lineiii{float(\var{x})}{\var{x} converted to floating point}{}
  \lineiii{complex(\var{re},\var{im})}{a complex number with real part \var{re}, imaginary part \var{im}.  \var{im} defaults to zero.}{}
  \lineiii{\var{c}.conjugate()}{conjugate of the complex number \var{c}}{}
  \lineiii{divmod(\var{x}, \var{y})}{the pair \code{(\var{x} // \var{y}, \var{x} \%{} \var{y})}}{(3)(4)}
  \lineiii{pow(\var{x}, \var{y})}{\var{x} to the power \var{y}}{}
  \lineiii{\var{x} ** \var{y}}{\var{x} to the power \var{y}}{}
\end{tableiii}
\indexiii{operations on}{numeric}{types}
\withsubitem{(complex number method)}{\ttindex{conjugate()}}

\noindent
Notes:
\begin{description}

\item[(1)]
For (plain or long) integer division, the result is an integer.
The result is always rounded towards minus infinity: 1/2 is 0,
(-1)/2 is -1, 1/(-2) is -1, and (-1)/(-2) is 0.  Note that the result
is a long integer if either operand is a long integer, regardless of
the numeric value.
\indexii{integer}{division}
\indexiii{long}{integer}{division}

\item[(2)]
Conversion from floating point to (long or plain) integer may round or
truncate as in C; see functions \function{floor()} and
\function{ceil()} in the \refmodule{math}\refbimodindex{math} module
for well-defined conversions.
\withsubitem{(in module math)}{\ttindex{floor()}\ttindex{ceil()}}
\indexii{numeric}{conversions}
\indexii{C}{language}

\item[(3)]
See section \ref{built-in-funcs}, ``Built-in Functions,'' for a full
description.

\item[(4)]
Complex floor division operator, modulo operator, and \function{divmod()}.

\deprecated{2.3}{Instead convert to float using \function{abs()}
if appropriate.}

\item[(5)]
Also referred to as integer division.  The resultant value is a whole integer,
though the result's type is not necessarily int.
\end{description}
% XXXJH exceptions: overflow (when? what operations?) zerodivision

\subsubsection{Bit-string Operations on Integer Types \label{bitstring-ops}}
\nodename{Bit-string Operations}

Plain and long integer types support additional operations that make
sense only for bit-strings.  Negative numbers are treated as their 2's
complement value (for long integers, this assumes a sufficiently large
number of bits that no overflow occurs during the operation).

The priorities of the binary bit-wise operations are all lower than
the numeric operations and higher than the comparisons; the unary
operation \samp{\~} has the same priority as the other unary numeric
operations (\samp{+} and \samp{-}).

This table lists the bit-string operations sorted in ascending
priority (operations in the same box have the same priority):

\begin{tableiii}{c|l|c}{code}{Operation}{Result}{Notes}
  \lineiii{\var{x} | \var{y}}{bitwise \dfn{or} of \var{x} and \var{y}}{}
  \lineiii{\var{x} \^{} \var{y}}{bitwise \dfn{exclusive or} of \var{x} and \var{y}}{}
  \lineiii{\var{x} \&{} \var{y}}{bitwise \dfn{and} of \var{x} and \var{y}}{}
  % The empty groups below prevent conversion to guillemets.
  \lineiii{\var{x} <{}< \var{n}}{\var{x} shifted left by \var{n} bits}{(1), (2)}
  \lineiii{\var{x} >{}> \var{n}}{\var{x} shifted right by \var{n} bits}{(1), (3)}
  \hline
  \lineiii{\~\var{x}}{the bits of \var{x} inverted}{}
\end{tableiii}
\indexiii{operations on}{integer}{types}
\indexii{bit-string}{operations}
\indexii{shifting}{operations}
\indexii{masking}{operations}

\noindent
Notes:
\begin{description}
\item[(1)] Negative shift counts are illegal and cause a
\exception{ValueError} to be raised.
\item[(2)] A left shift by \var{n} bits is equivalent to
multiplication by \code{pow(2, \var{n})} without overflow check.
\item[(3)] A right shift by \var{n} bits is equivalent to
division by \code{pow(2, \var{n})} without overflow check.
\end{description}


\subsection{Iterator Types \label{typeiter}}

\versionadded{2.2}
\index{iterator protocol}
\index{protocol!iterator}
\index{sequence!iteration}
\index{container!iteration over}

Python supports a concept of iteration over containers.  This is
implemented using two distinct methods; these are used to allow
user-defined classes to support iteration.  Sequences, described below
in more detail, always support the iteration methods.

One method needs to be defined for container objects to provide
iteration support:

\begin{methoddesc}[container]{__iter__}{}
  Return an iterator object.  The object is required to support the
  iterator protocol described below.  If a container supports
  different types of iteration, additional methods can be provided to
  specifically request iterators for those iteration types.  (An
  example of an object supporting multiple forms of iteration would be
  a tree structure which supports both breadth-first and depth-first
  traversal.)  This method corresponds to the \member{tp_iter} slot of
  the type structure for Python objects in the Python/C API.
\end{methoddesc}

The iterator objects themselves are required to support the following
two methods, which together form the \dfn{iterator protocol}:

\begin{methoddesc}[iterator]{__iter__}{}
  Return the iterator object itself.  This is required to allow both
  containers and iterators to be used with the \keyword{for} and
  \keyword{in} statements.  This method corresponds to the
  \member{tp_iter} slot of the type structure for Python objects in
  the Python/C API.
\end{methoddesc}

\begin{methoddesc}[iterator]{next}{}
  Return the next item from the container.  If there are no further
  items, raise the \exception{StopIteration} exception.  This method
  corresponds to the \member{tp_iternext} slot of the type structure
  for Python objects in the Python/C API.
\end{methoddesc}

Python defines several iterator objects to support iteration over
general and specific sequence types, dictionaries, and other more
specialized forms.  The specific types are not important beyond their
implementation of the iterator protocol.

The intention of the protocol is that once an iterator's
\method{next()} method raises \exception{StopIteration}, it will
continue to do so on subsequent calls.  Implementations that
do not obey this property are deemed broken.  (This constraint
was added in Python 2.3; in Python 2.2, various iterators are
broken according to this rule.)

Python's generators provide a convenient way to implement the
iterator protocol.  If a container object's \method{__iter__()}
method is implemented as a generator, it will automatically
return an iterator object (technically, a generator object)
supplying the \method{__iter__()} and \method{next()} methods.


\subsection{Sequence Types ---
	    \class{str}, \class{unicode}, \class{list},
	    \class{tuple}, \class{buffer}, \class{xrange}
	    \label{typesseq}}

There are six sequence types: strings, Unicode strings, lists,
tuples, buffers, and xrange objects.

String literals are written in single or double quotes:
\code{'xyzzy'}, \code{"frobozz"}.  See chapter 2 of the
\citetitle[../ref/strings.html]{Python Reference Manual} for more about
string literals.  Unicode strings are much like strings, but are
specified in the syntax using a preceding \character{u} character:
\code{u'abc'}, \code{u"def"}.  Lists are constructed with square brackets,
separating items with commas: \code{[a, b, c]}.  Tuples are
constructed by the comma operator (not within square brackets), with
or without enclosing parentheses, but an empty tuple must have the
enclosing parentheses, such as \code{a, b, c} or \code{()}.  A single
item tuple must have a trailing comma, such as \code{(d,)}.
\obindex{sequence}
\obindex{string}
\obindex{Unicode}
\obindex{tuple}
\obindex{list}

Buffer objects are not directly supported by Python syntax, but can be
created by calling the builtin function
\function{buffer()}.\bifuncindex{buffer}  They don't support
concatenation or repetition.
\obindex{buffer}

Xrange objects are similar to buffers in that there is no specific
syntax to create them, but they are created using the \function{xrange()}
function.\bifuncindex{xrange}  They don't support slicing,
concatenation or repetition, and using \code{in}, \code{not in},
\function{min()} or \function{max()} on them is inefficient.
\obindex{xrange}

Most sequence types support the following operations.  The \samp{in} and
\samp{not in} operations have the same priorities as the comparison
operations.  The \samp{+} and \samp{*} operations have the same
priority as the corresponding numeric operations.\footnote{They must
have since the parser can't tell the type of the operands.}

This table lists the sequence operations sorted in ascending priority
(operations in the same box have the same priority).  In the table,
\var{s} and \var{t} are sequences of the same type; \var{n}, \var{i}
and \var{j} are integers:

\begin{tableiii}{c|l|c}{code}{Operation}{Result}{Notes}
  \lineiii{\var{x} in \var{s}}{\code{True} if an item of \var{s} is equal to \var{x}, else \code{False}}{(1)}
  \lineiii{\var{x} not in \var{s}}{\code{False} if an item of \var{s} is
equal to \var{x}, else \code{True}}{(1)}
  \hline
  \lineiii{\var{s} + \var{t}}{the concatenation of \var{s} and \var{t}}{(6)}
  \lineiii{\var{s} * \var{n}\textrm{,} \var{n} * \var{s}}{\var{n} shallow copies of \var{s} concatenated}{(2)}
  \hline
  \lineiii{\var{s}[\var{i}]}{\var{i}'th item of \var{s}, origin 0}{(3)}
  \lineiii{\var{s}[\var{i}:\var{j}]}{slice of \var{s} from \var{i} to \var{j}}{(3), (4)}
  \lineiii{\var{s}[\var{i}:\var{j}:\var{k}]}{slice of \var{s} from \var{i} to \var{j} with step \var{k}}{(3), (5)}
  \hline
  \lineiii{len(\var{s})}{length of \var{s}}{}
  \lineiii{min(\var{s})}{smallest item of \var{s}}{}
  \lineiii{max(\var{s})}{largest item of \var{s}}{}
\end{tableiii}
\indexiii{operations on}{sequence}{types}
\bifuncindex{len}
\bifuncindex{min}
\bifuncindex{max}
\indexii{concatenation}{operation}
\indexii{repetition}{operation}
\indexii{subscript}{operation}
\indexii{slice}{operation}
\indexii{extended slice}{operation}
\opindex{in}
\opindex{not in}

\noindent
Notes:

\begin{description}
\item[(1)] When \var{s} is a string or Unicode string object the
\code{in} and \code{not in} operations act like a substring test.  In
Python versions before 2.3, \var{x} had to be a string of length 1.
In Python 2.3 and beyond, \var{x} may be a string of any length.

\item[(2)] Values of \var{n} less than \code{0} are treated as
  \code{0} (which yields an empty sequence of the same type as
  \var{s}).  Note also that the copies are shallow; nested structures
  are not copied.  This often haunts new Python programmers; consider:

\begin{verbatim}
>>> lists = [[]] * 3
>>> lists
[[], [], []]
>>> lists[0].append(3)
>>> lists
[[3], [3], [3]]
\end{verbatim}

  What has happened is that \code{[[]]} is a one-element list containing
  an empty list, so all three elements of \code{[[]] * 3} are (pointers to)
  this single empty list.  Modifying any of the elements of \code{lists}
  modifies this single list.  You can create a list of different lists this
  way:

\begin{verbatim}
>>> lists = [[] for i in range(3)]
>>> lists[0].append(3)
>>> lists[1].append(5)
>>> lists[2].append(7)
>>> lists
[[3], [5], [7]]
\end{verbatim}

\item[(3)] If \var{i} or \var{j} is negative, the index is relative to
  the end of the string: \code{len(\var{s}) + \var{i}} or
  \code{len(\var{s}) + \var{j}} is substituted.  But note that \code{-0} is
  still \code{0}.

\item[(4)] The slice of \var{s} from \var{i} to \var{j} is defined as
  the sequence of items with index \var{k} such that \code{\var{i} <=
  \var{k} < \var{j}}.  If \var{i} or \var{j} is greater than
  \code{len(\var{s})}, use \code{len(\var{s})}.  If \var{i} is omitted
  or \code{None}, use \code{0}.  If \var{j} is omitted or \code{None},
  use \code{len(\var{s})}.  If \var{i} is greater than or equal to \var{j},
  the slice is empty.

\item[(5)] The slice of \var{s} from \var{i} to \var{j} with step
  \var{k} is defined as the sequence of items with index 
  \code{\var{x} = \var{i} + \var{n}*\var{k}} such that
  $0 \leq n < \frac{j-i}{k}$.  In other words, the indices
  are \code{i}, \code{i+k}, \code{i+2*k}, \code{i+3*k} and so on, stopping when
  \var{j} is reached (but never including \var{j}).  If \var{i} or \var{j}
  is greater than \code{len(\var{s})}, use \code{len(\var{s})}.  If
  \var{i} or \var{j} are omitted or \code{None}, they become ``end'' values
  (which end depends on the sign of \var{k}).  Note, \var{k} cannot
  be zero. If \var{k} is \code{None}, it is treated like \code{1}.

\item[(6)] If \var{s} and \var{t} are both strings, some Python
implementations such as CPython can usually perform an in-place optimization
for assignments of the form \code{\var{s}=\var{s}+\var{t}} or
\code{\var{s}+=\var{t}}.  When applicable, this optimization makes
quadratic run-time much less likely.  This optimization is both version
and implementation dependent.  For performance sensitive code, it is
preferable to use the \method{str.join()} method which assures consistent
linear concatenation performance across versions and implementations.
\versionchanged[Formerly, string concatenation never occurred in-place]{2.4}

\end{description}


\subsubsection{String Methods \label{string-methods}}

These are the string methods which both 8-bit strings and Unicode
objects support:

\begin{methoddesc}[string]{capitalize}{}
Return a copy of the string with only its first character capitalized.

For 8-bit strings, this method is locale-dependent.
\end{methoddesc}

\begin{methoddesc}[string]{center}{width\optional{, fillchar}}
Return centered in a string of length \var{width}. Padding is done
using the specified \var{fillchar} (default is a space).
\versionchanged[Support for the \var{fillchar} argument]{2.4}
\end{methoddesc}

\begin{methoddesc}[string]{count}{sub\optional{, start\optional{, end}}}
Return the number of occurrences of substring \var{sub} in string
S\code{[\var{start}:\var{end}]}.  Optional arguments \var{start} and
\var{end} are interpreted as in slice notation.
\end{methoddesc}

\begin{methoddesc}[string]{decode}{\optional{encoding\optional{, errors}}}
Decodes the string using the codec registered for \var{encoding}.
\var{encoding} defaults to the default string encoding.  \var{errors}
may be given to set a different error handling scheme.  The default is
\code{'strict'}, meaning that encoding errors raise
\exception{UnicodeError}.  Other possible values are \code{'ignore'},
\code{'replace'} and any other name registered via
\function{codecs.register_error}, see section~\ref{codec-base-classes}.
\versionadded{2.2}
\versionchanged[Support for other error handling schemes added]{2.3}
\end{methoddesc}

\begin{methoddesc}[string]{encode}{\optional{encoding\optional{,errors}}}
Return an encoded version of the string.  Default encoding is the current
default string encoding.  \var{errors} may be given to set a different
error handling scheme.  The default for \var{errors} is
\code{'strict'}, meaning that encoding errors raise a
\exception{UnicodeError}.  Other possible values are \code{'ignore'},
\code{'replace'}, \code{'xmlcharrefreplace'}, \code{'backslashreplace'}
and any other name registered via \function{codecs.register_error},
see section~\ref{codec-base-classes}.
For a list of possible encodings, see section~\ref{standard-encodings}.
\versionadded{2.0}
\versionchanged[Support for \code{'xmlcharrefreplace'} and
\code{'backslashreplace'} and other error handling schemes added]{2.3}
\end{methoddesc}

\begin{methoddesc}[string]{endswith}{suffix\optional{, start\optional{, end}}}
Return \code{True} if the string ends with the specified \var{suffix},
otherwise return \code{False}.  With optional \var{start}, test beginning at
that position.  With optional \var{end}, stop comparing at that position.
\end{methoddesc}

\begin{methoddesc}[string]{expandtabs}{\optional{tabsize}}
Return a copy of the string where all tab characters are expanded
using spaces.  If \var{tabsize} is not given, a tab size of \code{8}
characters is assumed.
\end{methoddesc}

\begin{methoddesc}[string]{find}{sub\optional{, start\optional{, end}}}
Return the lowest index in the string where substring \var{sub} is
found, such that \var{sub} is contained in the range [\var{start},
\var{end}].  Optional arguments \var{start} and \var{end} are
interpreted as in slice notation.  Return \code{-1} if \var{sub} is
not found.
\end{methoddesc}

\begin{methoddesc}[string]{index}{sub\optional{, start\optional{, end}}}
Like \method{find()}, but raise \exception{ValueError} when the
substring is not found.
\end{methoddesc}

\begin{methoddesc}[string]{isalnum}{}
Return true if all characters in the string are alphanumeric and there
is at least one character, false otherwise.

For 8-bit strings, this method is locale-dependent.
\end{methoddesc}

\begin{methoddesc}[string]{isalpha}{}
Return true if all characters in the string are alphabetic and there
is at least one character, false otherwise.

For 8-bit strings, this method is locale-dependent.
\end{methoddesc}

\begin{methoddesc}[string]{isdigit}{}
Return true if all characters in the string are digits and there
is at least one character, false otherwise.

For 8-bit strings, this method is locale-dependent.
\end{methoddesc}

\begin{methoddesc}[string]{islower}{}
Return true if all cased characters in the string are lowercase and
there is at least one cased character, false otherwise.

For 8-bit strings, this method is locale-dependent.
\end{methoddesc}

\begin{methoddesc}[string]{isspace}{}
Return true if there are only whitespace characters in the string and
there is at least one character, false otherwise.

For 8-bit strings, this method is locale-dependent.
\end{methoddesc}

\begin{methoddesc}[string]{istitle}{}
Return true if the string is a titlecased string and there is at least one
character, for example uppercase characters may only follow uncased
characters and lowercase characters only cased ones.  Return false
otherwise.

For 8-bit strings, this method is locale-dependent.
\end{methoddesc}

\begin{methoddesc}[string]{isupper}{}
Return true if all cased characters in the string are uppercase and
there is at least one cased character, false otherwise.

For 8-bit strings, this method is locale-dependent.
\end{methoddesc}

\begin{methoddesc}[string]{join}{seq}
Return a string which is the concatenation of the strings in the
sequence \var{seq}.  The separator between elements is the string
providing this method.
\end{methoddesc}

\begin{methoddesc}[string]{ljust}{width\optional{, fillchar}}
Return the string left justified in a string of length \var{width}.
Padding is done using the specified \var{fillchar} (default is a
space).  The original string is returned if
\var{width} is less than \code{len(\var{s})}.
\versionchanged[Support for the \var{fillchar} argument]{2.4}
\end{methoddesc}

\begin{methoddesc}[string]{lower}{}
Return a copy of the string converted to lowercase.

For 8-bit strings, this method is locale-dependent.
\end{methoddesc}

\begin{methoddesc}[string]{lstrip}{\optional{chars}}
Return a copy of the string with leading characters removed.  The
\var{chars} argument is a string specifying the set of characters
to be removed.  If omitted or \code{None}, the \var{chars} argument
defaults to removing whitespace.  The \var{chars} argument is not
a prefix; rather, all combinations of its values are stripped:
\begin{verbatim}
    >>> '   spacious   '.lstrip()
    'spacious   '
    >>> 'www.example.com'.lstrip('cmowz.')
    'example.com'
\end{verbatim}
\versionchanged[Support for the \var{chars} argument]{2.2.2}
\end{methoddesc}

\begin{methoddesc}[string]{replace}{old, new\optional{, count}}
Return a copy of the string with all occurrences of substring
\var{old} replaced by \var{new}.  If the optional argument
\var{count} is given, only the first \var{count} occurrences are
replaced.
\end{methoddesc}

\begin{methoddesc}[string]{rfind}{sub \optional{,start \optional{,end}}}
Return the highest index in the string where substring \var{sub} is
found, such that \var{sub} is contained within s[start,end].  Optional
arguments \var{start} and \var{end} are interpreted as in slice
notation.  Return \code{-1} on failure.
\end{methoddesc}

\begin{methoddesc}[string]{rindex}{sub\optional{, start\optional{, end}}}
Like \method{rfind()} but raises \exception{ValueError} when the
substring \var{sub} is not found.
\end{methoddesc}

\begin{methoddesc}[string]{rjust}{width\optional{, fillchar}}
Return the string right justified in a string of length \var{width}.
Padding is done using the specified \var{fillchar} (default is a space).
The original string is returned if
\var{width} is less than \code{len(\var{s})}.
\versionchanged[Support for the \var{fillchar} argument]{2.4}
\end{methoddesc}

\begin{methoddesc}[string]{rsplit}{\optional{sep \optional{,maxsplit}}}
Return a list of the words in the string, using \var{sep} as the
delimiter string.  If \var{maxsplit} is given, at most \var{maxsplit}
splits are done, the \emph{rightmost} ones.  If \var{sep} is not specified
or \code{None}, any whitespace string is a separator.  Except for splitting
from the right, \method{rsplit()} behaves like \method{split()} which
is described in detail below.
\versionadded{2.4}
\end{methoddesc}

\begin{methoddesc}[string]{rstrip}{\optional{chars}}
Return a copy of the string with trailing characters removed.  The
\var{chars} argument is a string specifying the set of characters
to be removed.  If omitted or \code{None}, the \var{chars} argument
defaults to removing whitespace.  The \var{chars} argument is not
a suffix; rather, all combinations of its values are stripped:
\begin{verbatim}
    >>> '   spacious   '.rstrip()
    '   spacious'
    >>> 'mississippi'.rstrip('ipz')
    'mississ'
\end{verbatim}
\versionchanged[Support for the \var{chars} argument]{2.2.2}
\end{methoddesc}

\begin{methoddesc}[string]{split}{\optional{sep \optional{,maxsplit}}}
Return a list of the words in the string, using \var{sep} as the
delimiter string.  If \var{maxsplit} is given, at most \var{maxsplit}
splits are done. (thus, the list will have at most \code{\var{maxsplit}+1}
elements).  If \var{maxsplit} is not specified, then there
is no limit on the number of splits (all possible splits are made).
Consecutive delimiters are not grouped together and are
deemed to delimit empty strings (for example, \samp{'1,,2'.split(',')}
returns \samp{['1', '', '2']}).  The \var{sep} argument may consist of
multiple characters (for example, \samp{'1, 2, 3'.split(', ')} returns
\samp{['1', '2', '3']}).  Splitting an empty string with a specified
separator returns \samp{['']}.

If \var{sep} is not specified or is \code{None}, a different splitting
algorithm is applied.  First, whitespace characters (spaces, tabs,
newlines, returns, and formfeeds) are stripped from both ends.  Then,
words are separated by arbitrary length strings of whitespace
characters. Consecutive whitespace delimiters are treated as a single
delimiter (\samp{'1  2  3'.split()} returns \samp{['1', '2', '3']}).
Splitting an empty string or a string consisting of just whitespace
returns an empty list.
\end{methoddesc}

\begin{methoddesc}[string]{splitlines}{\optional{keepends}}
Return a list of the lines in the string, breaking at line
boundaries.  Line breaks are not included in the resulting list unless
\var{keepends} is given and true.
\end{methoddesc}

\begin{methoddesc}[string]{startswith}{prefix\optional{,
                                       start\optional{, end}}}
Return \code{True} if string starts with the \var{prefix}, otherwise
return \code{False}.  With optional \var{start}, test string beginning at
that position.  With optional \var{end}, stop comparing string at that
position.
\end{methoddesc}

\begin{methoddesc}[string]{strip}{\optional{chars}}
Return a copy of the string with the leading and trailing characters
removed.  The \var{chars} argument is a string specifying the set of
characters to be removed.  If omitted or \code{None}, the \var{chars}
argument defaults to removing whitespace.  The \var{chars} argument is not
a prefix or suffix; rather, all combinations of its values are stripped:
\begin{verbatim}
    >>> '   spacious   '.strip()
    'spacious'
    >>> 'www.example.com'.strip('cmowz.')
    'example'
\end{verbatim}
\versionchanged[Support for the \var{chars} argument]{2.2.2}
\end{methoddesc}

\begin{methoddesc}[string]{swapcase}{}
Return a copy of the string with uppercase characters converted to
lowercase and vice versa.

For 8-bit strings, this method is locale-dependent.
\end{methoddesc}

\begin{methoddesc}[string]{title}{}
Return a titlecased version of the string: words start with uppercase
characters, all remaining cased characters are lowercase.

For 8-bit strings, this method is locale-dependent.
\end{methoddesc}

\begin{methoddesc}[string]{translate}{table\optional{, deletechars}}
Return a copy of the string where all characters occurring in the
optional argument \var{deletechars} are removed, and the remaining
characters have been mapped through the given translation table, which
must be a string of length 256.

For Unicode objects, the \method{translate()} method does not
accept the optional \var{deletechars} argument.  Instead, it
returns a copy of the \var{s} where all characters have been mapped
through the given translation table which must be a mapping of
Unicode ordinals to Unicode ordinals, Unicode strings or \code{None}.
Unmapped characters are left untouched. Characters mapped to \code{None}
are deleted.  Note, a more flexible approach is to create a custom
character mapping codec using the \refmodule{codecs} module (see
\module{encodings.cp1251} for an example).      
\end{methoddesc}

\begin{methoddesc}[string]{upper}{}
Return a copy of the string converted to uppercase.

For 8-bit strings, this method is locale-dependent.
\end{methoddesc}

\begin{methoddesc}[string]{zfill}{width}
Return the numeric string left filled with zeros in a string
of length \var{width}. The original string is returned if
\var{width} is less than \code{len(\var{s})}.
\versionadded{2.2.2}
\end{methoddesc}


\subsubsection{String Formatting Operations \label{typesseq-strings}}

\index{formatting, string (\%{})}
\index{interpolation, string (\%{})}
\index{string!formatting}
\index{string!interpolation}
\index{printf-style formatting}
\index{sprintf-style formatting}
\index{\protect\%{} formatting}
\index{\protect\%{} interpolation}

String and Unicode objects have one unique built-in operation: the
\code{\%} operator (modulo).  This is also known as the string
\emph{formatting} or \emph{interpolation} operator.  Given
\code{\var{format} \% \var{values}} (where \var{format} is a string or
Unicode object), \code{\%} conversion specifications in \var{format}
are replaced with zero or more elements of \var{values}.  The effect
is similar to the using \cfunction{sprintf()} in the C language.  If
\var{format} is a Unicode object, or if any of the objects being
converted using the \code{\%s} conversion are Unicode objects, the
result will also be a Unicode object.

If \var{format} requires a single argument, \var{values} may be a
single non-tuple object.\footnote{To format only a tuple you
should therefore provide a singleton tuple whose only element
is the tuple to be formatted.}  Otherwise, \var{values} must be a tuple with
exactly the number of items specified by the format string, or a
single mapping object (for example, a dictionary).

A conversion specifier contains two or more characters and has the
following components, which must occur in this order:

\begin{enumerate}
  \item  The \character{\%} character, which marks the start of the
         specifier.
  \item  Mapping key (optional), consisting of a parenthesised sequence
         of characters (for example, \code{(somename)}).
  \item  Conversion flags (optional), which affect the result of some
         conversion types.
  \item  Minimum field width (optional).  If specified as an
         \character{*} (asterisk), the actual width is read from the
         next element of the tuple in \var{values}, and the object to
         convert comes after the minimum field width and optional
         precision.
  \item  Precision (optional), given as a \character{.} (dot) followed
         by the precision.  If specified as \character{*} (an
         asterisk), the actual width is read from the next element of
         the tuple in \var{values}, and the value to convert comes after
         the precision.
  \item  Length modifier (optional).
  \item  Conversion type.
\end{enumerate}

When the right argument is a dictionary (or other mapping type), then
the formats in the string \emph{must} include a parenthesised mapping key into
that dictionary inserted immediately after the \character{\%}
character. The mapping key selects the value to be formatted from the
mapping.  For example:

\begin{verbatim}
>>> print '%(language)s has %(#)03d quote types.' % \
          {'language': "Python", "#": 2}
Python has 002 quote types.
\end{verbatim}

In this case no \code{*} specifiers may occur in a format (since they
require a sequential parameter list).

The conversion flag characters are:

\begin{tableii}{c|l}{character}{Flag}{Meaning}
  \lineii{\#}{The value conversion will use the ``alternate form''
              (where defined below).}
  \lineii{0}{The conversion will be zero padded for numeric values.}
  \lineii{-}{The converted value is left adjusted (overrides
             the \character{0} conversion if both are given).}
  \lineii{{~}}{(a space) A blank should be left before a positive number
             (or empty string) produced by a signed conversion.}
  \lineii{+}{A sign character (\character{+} or \character{-}) will
             precede the conversion (overrides a "space" flag).}
\end{tableii}

A length modifier (\code{h}, \code{l}, or \code{L}) may be
present, but is ignored as it is not necessary for Python.

The conversion types are:

\begin{tableiii}{c|l|c}{character}{Conversion}{Meaning}{Notes}
  \lineiii{d}{Signed integer decimal.}{}
  \lineiii{i}{Signed integer decimal.}{}
  \lineiii{o}{Unsigned octal.}{(1)}
  \lineiii{u}{Unsigned decimal.}{}
  \lineiii{x}{Unsigned hexadecimal (lowercase).}{(2)}
  \lineiii{X}{Unsigned hexadecimal (uppercase).}{(2)}
  \lineiii{e}{Floating point exponential format (lowercase).}{}
  \lineiii{E}{Floating point exponential format (uppercase).}{}
  \lineiii{f}{Floating point decimal format.}{}
  \lineiii{F}{Floating point decimal format.}{}
  \lineiii{g}{Same as \character{e} if exponent is greater than -4 or
              less than precision, \character{f} otherwise.}{}
  \lineiii{G}{Same as \character{E} if exponent is greater than -4 or
              less than precision, \character{F} otherwise.}{}
  \lineiii{c}{Single character (accepts integer or single character
              string).}{}
  \lineiii{r}{String (converts any python object using
              \function{repr()}).}{(3)}
  \lineiii{s}{String (converts any python object using
              \function{str()}).}{(4)}
  \lineiii{\%}{No argument is converted, results in a \character{\%}
               character in the result.}{}
\end{tableiii}

\noindent
Notes:
\begin{description}
  \item[(1)]
    The alternate form causes a leading zero (\character{0}) to be
    inserted between left-hand padding and the formatting of the
    number if the leading character of the result is not already a
    zero.
  \item[(2)]
    The alternate form causes a leading \code{'0x'} or \code{'0X'}
    (depending on whether the \character{x} or \character{X} format
    was used) to be inserted between left-hand padding and the
    formatting of the number if the leading character of the result is
    not already a zero.
  \item[(3)]
    The \code{\%r} conversion was added in Python 2.0.
  \item[(4)]
    If the object or format provided is a \class{unicode} string,
    the resulting string will also be \class{unicode}.
\end{description}

% XXX Examples?

Since Python strings have an explicit length, \code{\%s} conversions
do not assume that \code{'\e0'} is the end of the string.

For safety reasons, floating point precisions are clipped to 50;
\code{\%f} conversions for numbers whose absolute value is over 1e25
are replaced by \code{\%g} conversions.\footnote{
  These numbers are fairly arbitrary.  They are intended to
  avoid printing endless strings of meaningless digits without hampering
  correct use and without having to know the exact precision of floating
  point values on a particular machine.
}  All other errors raise exceptions.

Additional string operations are defined in standard modules
\refmodule{string}\refstmodindex{string}\ and
\refmodule{re}.\refstmodindex{re}


\subsubsection{XRange Type \label{typesseq-xrange}}

The \class{xrange}\obindex{xrange} type is an immutable sequence which
is commonly used for looping.  The advantage of the \class{xrange}
type is that an \class{xrange} object will always take the same amount
of memory, no matter the size of the range it represents.  There are
no consistent performance advantages.

XRange objects have very little behavior: they only support indexing,
iteration, and the \function{len()} function.


\subsubsection{Mutable Sequence Types \label{typesseq-mutable}}

List objects support additional operations that allow in-place
modification of the object.
Other mutable sequence types (when added to the language) should
also support these operations.
Strings and tuples are immutable sequence types: such objects cannot
be modified once created.
The following operations are defined on mutable sequence types (where
\var{x} is an arbitrary object):
\indexiii{mutable}{sequence}{types}
\obindex{list}

\begin{tableiii}{c|l|c}{code}{Operation}{Result}{Notes}
  \lineiii{\var{s}[\var{i}] = \var{x}}
	{item \var{i} of \var{s} is replaced by \var{x}}{}
  \lineiii{\var{s}[\var{i}:\var{j}] = \var{t}}
  	{slice of \var{s} from \var{i} to \var{j} is replaced by \var{t}}{}
  \lineiii{del \var{s}[\var{i}:\var{j}]}
	{same as \code{\var{s}[\var{i}:\var{j}] = []}}{}
  \lineiii{\var{s}[\var{i}:\var{j}:\var{k}] = \var{t}}
  	{the elements of \code{\var{s}[\var{i}:\var{j}:\var{k}]} are replaced by those of \var{t}}{(1)}
  \lineiii{del \var{s}[\var{i}:\var{j}:\var{k}]}
	{removes the elements of \code{\var{s}[\var{i}:\var{j}:\var{k}]} from the list}{}
  \lineiii{\var{s}.append(\var{x})}
	{same as \code{\var{s}[len(\var{s}):len(\var{s})] = [\var{x}]}}{(2)}
  \lineiii{\var{s}.extend(\var{x})}
        {same as \code{\var{s}[len(\var{s}):len(\var{s})] = \var{x}}}{(3)}
  \lineiii{\var{s}.count(\var{x})}
    {return number of \var{i}'s for which \code{\var{s}[\var{i}] == \var{x}}}{}
  \lineiii{\var{s}.index(\var{x}\optional{, \var{i}\optional{, \var{j}}})}
    {return smallest \var{k} such that \code{\var{s}[\var{k}] == \var{x}} and
    \code{\var{i} <= \var{k} < \var{j}}}{(4)}
  \lineiii{\var{s}.insert(\var{i}, \var{x})}
	{same as \code{\var{s}[\var{i}:\var{i}] = [\var{x}]}}{(5)}
  \lineiii{\var{s}.pop(\optional{\var{i}})}
    {same as \code{\var{x} = \var{s}[\var{i}]; del \var{s}[\var{i}]; return \var{x}}}{(6)}
  \lineiii{\var{s}.remove(\var{x})}
	{same as \code{del \var{s}[\var{s}.index(\var{x})]}}{(4)}
  \lineiii{\var{s}.reverse()}
	{reverses the items of \var{s} in place}{(7)}
  \lineiii{\var{s}.sort(\optional{\var{cmp}\optional{,
                        \var{key}\optional{, \var{reverse}}}})}
	{sort the items of \var{s} in place}{(7), (8), (9), (10)}
\end{tableiii}
\indexiv{operations on}{mutable}{sequence}{types}
\indexiii{operations on}{sequence}{types}
\indexiii{operations on}{list}{type}
\indexii{subscript}{assignment}
\indexii{slice}{assignment}
\indexii{extended slice}{assignment}
\stindex{del}
\withsubitem{(list method)}{
  \ttindex{append()}\ttindex{extend()}\ttindex{count()}\ttindex{index()}
  \ttindex{insert()}\ttindex{pop()}\ttindex{remove()}\ttindex{reverse()}
  \ttindex{sort()}}
\noindent
Notes:
\begin{description}
\item[(1)] \var{t} must have the same length as the slice it is 
  replacing.

\item[(2)] The C implementation of Python has historically accepted
  multiple parameters and implicitly joined them into a tuple; this
  no longer works in Python 2.0.  Use of this misfeature has been
  deprecated since Python 1.4.

\item[(3)] \var{x} can be any iterable object.

\item[(4)] Raises \exception{ValueError} when \var{x} is not found in
  \var{s}. When a negative index is passed as the second or third parameter
  to the \method{index()} method, the list length is added, as for slice
  indices.  If it is still negative, it is truncated to zero, as for
  slice indices.  \versionchanged[Previously, \method{index()} didn't
  have arguments for specifying start and stop positions]{2.3}

\item[(5)] When a negative index is passed as the first parameter to
  the \method{insert()} method, the list length is added, as for slice
  indices.  If it is still negative, it is truncated to zero, as for
  slice indices.  \versionchanged[Previously, all negative indices
  were truncated to zero]{2.3}

\item[(6)] The \method{pop()} method is only supported by the list and
  array types.  The optional argument \var{i} defaults to \code{-1},
  so that by default the last item is removed and returned.

\item[(7)] The \method{sort()} and \method{reverse()} methods modify the
  list in place for economy of space when sorting or reversing a large
  list.  To remind you that they operate by side effect, they don't return
  the sorted or reversed list.

\item[(8)] The \method{sort()} method takes optional arguments for
  controlling the comparisons.

  \var{cmp} specifies a custom comparison function of two arguments
     (list items) which should return a negative, zero or positive number
     depending on whether the first argument is considered smaller than,
     equal to, or larger than the second argument:
     \samp{\var{cmp}=\keyword{lambda} \var{x},\var{y}:
     \function{cmp}(x.lower(), y.lower())}
     
  \var{key} specifies a function of one argument that is used to
     extract a comparison key from each list element:
     \samp{\var{key}=\function{str.lower}}

  \var{reverse} is a boolean value.  If set to \code{True}, then the
     list elements are sorted as if each comparison were reversed.

  In general, the \var{key} and \var{reverse} conversion processes are
  much faster than specifying an equivalent \var{cmp} function.  This is
  because \var{cmp} is called multiple times for each list element while
  \var{key} and \var{reverse} touch each element only once.

  \versionchanged[Support for \code{None} as an equivalent to omitting
  \var{cmp} was added]{2.3}

  \versionchanged[Support for \var{key} and \var{reverse} was added]{2.4}

\item[(9)] Starting with Python 2.3, the \method{sort()} method is
  guaranteed to be stable.  A sort is stable if it guarantees not to
  change the relative order of elements that compare equal --- this is
  helpful for sorting in multiple passes (for example, sort by
  department, then by salary grade).

\item[(10)] While a list is being sorted, the effect of attempting to
  mutate, or even inspect, the list is undefined.  The C
  implementation of Python 2.3 and newer makes the list appear empty
  for the duration, and raises \exception{ValueError} if it can detect
  that the list has been mutated during a sort.
\end{description}

\subsection{Set Types ---
	    \class{set}, \class{frozenset}
	    \label{types-set}}
\obindex{set}

A \dfn{set} object is an unordered collection of immutable values.
Common uses include membership testing, removing duplicates from a sequence,
and computing mathematical operations such as intersection, union, difference,
and symmetric difference.
\versionadded{2.4}     

Like other collections, sets support \code{\var{x} in \var{set}},
\code{len(\var{set})}, and \code{for \var{x} in \var{set}}.  Being an
unordered collection, sets do not record element position or order of
insertion.  Accordingly, sets do not support indexing, slicing, or
other sequence-like behavior.     

There are currently two builtin set types, \class{set} and \class{frozenset}.
The \class{set} type is mutable --- the contents can be changed using methods
like \method{add()} and \method{remove()}.  Since it is mutable, it has no
hash value and cannot be used as either a dictionary key or as an element of
another set.  The \class{frozenset} type is immutable and hashable --- its
contents cannot be altered after is created; however, it can be used as
a dictionary key or as an element of another set.

Instances of \class{set} and \class{frozenset} provide the following operations:

\begin{tableiii}{c|c|l}{code}{Operation}{Equivalent}{Result}
  \lineiii{len(\var{s})}{}{cardinality of set \var{s}}

  \hline
  \lineiii{\var{x} in \var{s}}{}
         {test \var{x} for membership in \var{s}}
  \lineiii{\var{x} not in \var{s}}{}
         {test \var{x} for non-membership in \var{s}}
  \lineiii{\var{s}.issubset(\var{t})}{\code{\var{s} <= \var{t}}}
         {test whether every element in \var{s} is in \var{t}}
  \lineiii{\var{s}.issuperset(\var{t})}{\code{\var{s} >= \var{t}}}
         {test whether every element in \var{t} is in \var{s}}

  \hline
  \lineiii{\var{s}.union(\var{t})}{\var{s} | \var{t}}
         {new set with elements from both \var{s} and \var{t}}
  \lineiii{\var{s}.intersection(\var{t})}{\var{s} \&\ \var{t}}
         {new set with elements common to \var{s} and \var{t}}
  \lineiii{\var{s}.difference(\var{t})}{\var{s} - \var{t}}
         {new set with elements in \var{s} but not in \var{t}}
  \lineiii{\var{s}.symmetric_difference(\var{t})}{\var{s} \^\ \var{t}}
         {new set with elements in either \var{s} or \var{t} but not both}
  \lineiii{\var{s}.copy()}{}
         {new set with a shallow copy of \var{s}}
\end{tableiii}

Note, the non-operator versions of \method{union()}, \method{intersection()},
\method{difference()}, and \method{symmetric_difference()},
\method{issubset()}, and \method{issuperset()} methods will accept any
iterable as an argument.  In contrast, their operator based counterparts
require their arguments to be sets.  This precludes error-prone constructions
like \code{set('abc') \&\ 'cbs'} in favor of the more readable
\code{set('abc').intersection('cbs')}.

Both \class{set} and \class{frozenset} support set to set comparisons.
Two sets are equal if and only if every element of each set is contained in
the other (each is a subset of the other).
A set is less than another set if and only if the first set is a proper
subset of the second set (is a subset, but is not equal).
A set is greater than another set if and only if the first set is a proper
superset of the second set (is a superset, but is not equal).

Instances of \class{set} are compared to instances of \class{frozenset} based
on their members.  For example, \samp{set('abc') == frozenset('abc')} returns
\code{True}.     

The subset and equality comparisons do not generalize to a complete
ordering function.  For example, any two disjoint sets are not equal and
are not subsets of each other, so \emph{all} of the following return
\code{False}:  \code{\var{a}<\var{b}}, \code{\var{a}==\var{b}}, or
\code{\var{a}>\var{b}}.
Accordingly, sets do not implement the \method{__cmp__} method.

Since sets only define partial ordering (subset relationships), the output
of the \method{list.sort()} method is undefined for lists of sets.

Set elements are like dictionary keys; they need to define both
\method{__hash__} and \method{__eq__} methods.

Binary operations that mix \class{set} instances with \class{frozenset}
return the type of the first operand.  For example:
\samp{frozenset('ab') | set('bc')} returns an instance of \class{frozenset}.

The following table lists operations available for \class{set}
that do not apply to immutable instances of \class{frozenset}:

\begin{tableiii}{c|c|l}{code}{Operation}{Equivalent}{Result}
  \lineiii{\var{s}.update(\var{t})}
         {\var{s} |= \var{t}}
         {update set \var{s}, adding elements from \var{t}}
  \lineiii{\var{s}.intersection_update(\var{t})}
         {\var{s} \&= \var{t}}
         {update set \var{s}, keeping only elements found in both \var{s} and \var{t}}
  \lineiii{\var{s}.difference_update(\var{t})}
         {\var{s} -= \var{t}}
         {update set \var{s}, removing elements found in \var{t}}
  \lineiii{\var{s}.symmetric_difference_update(\var{t})}
         {\var{s} \textasciicircum= \var{t}}
         {update set \var{s}, keeping only elements found in either \var{s} or \var{t}
          but not in both}

  \hline
  \lineiii{\var{s}.add(\var{x})}{}
         {add element \var{x} to set \var{s}}
  \lineiii{\var{s}.remove(\var{x})}{}
         {remove \var{x} from set \var{s}; raises \exception{KeyError}
	  if not present}
  \lineiii{\var{s}.discard(\var{x})}{}
         {removes \var{x} from set \var{s} if present}
  \lineiii{\var{s}.pop()}{}
         {remove and return an arbitrary element from \var{s}; raises
	  \exception{KeyError} if empty}
  \lineiii{\var{s}.clear()}{}
         {remove all elements from set \var{s}}
\end{tableiii}

Note, the non-operator versions of the \method{update()},
\method{intersection_update()}, \method{difference_update()}, and
\method{symmetric_difference_update()} methods will accept any iterable
as an argument.

The design of the set types was based on lessons learned from the
\module{sets} module.
     
\begin{seealso}     
  \seelink{comparison-to-builtin-set.html}
          {Comparison to the built-in set types}
          {Differences between the \module{sets} module and the
           built-in set types.}					      
\end{seealso}
     

\subsection{Mapping Types --- \class{dict} \label{typesmapping}}
\obindex{mapping}
\obindex{dictionary}

A \dfn{mapping} object maps  immutable values to
arbitrary objects.  Mappings are mutable objects.  There is currently
only one standard mapping type, the \dfn{dictionary}.  A dictionary's keys are
almost arbitrary values.  Only values containing lists, dictionaries
or other mutable types (that are compared by value rather than by
object identity) may not be used as keys.
Numeric types used for keys obey the normal rules for numeric
comparison: if two numbers compare equal (such as \code{1} and
\code{1.0}) then they can be used interchangeably to index the same
dictionary entry.

Dictionaries are created by placing a comma-separated list of
\code{\var{key}: \var{value}} pairs within braces, for example:
\code{\{'jack': 4098, 'sjoerd': 4127\}} or
\code{\{4098: 'jack', 4127: 'sjoerd'\}}.

The following operations are defined on mappings (where \var{a} and
\var{b} are mappings, \var{k} is a key, and \var{v} and \var{x} are
arbitrary objects):
\indexiii{operations on}{mapping}{types}
\indexiii{operations on}{dictionary}{type}
\stindex{del}
\bifuncindex{len}
\withsubitem{(dictionary method)}{
  \ttindex{clear()}
  \ttindex{copy()}
  \ttindex{has_key()}
  \ttindex{fromkeys()}    
  \ttindex{items()}
  \ttindex{keys()}
  \ttindex{update()}
  \ttindex{values()}
  \ttindex{get()}
  \ttindex{setdefault()}
  \ttindex{pop()}
  \ttindex{popitem()}
  \ttindex{iteritems()}
  \ttindex{iterkeys()}
  \ttindex{itervalues()}}

\begin{tableiii}{c|l|c}{code}{Operation}{Result}{Notes}
  \lineiii{len(\var{a})}{the number of items in \var{a}}{}
  \lineiii{\var{a}[\var{k}]}{the item of \var{a} with key \var{k}}{(1), (10)}
  \lineiii{\var{a}[\var{k}] = \var{v}}
          {set \code{\var{a}[\var{k}]} to \var{v}}
          {}
  \lineiii{del \var{a}[\var{k}]}
          {remove \code{\var{a}[\var{k}]} from \var{a}}
          {(1)}
  \lineiii{\var{a}.clear()}{remove all items from \code{a}}{}
  \lineiii{\var{a}.copy()}{a (shallow) copy of \code{a}}{}
  \lineiii{\var{a}.has_key(\var{k})}
          {\code{True} if \var{a} has a key \var{k}, else \code{False}}
          {}
  \lineiii{\var{k} \code{in} \var{a}}
          {Equivalent to \var{a}.has_key(\var{k})}
          {(2)}
  \lineiii{\var{k} not in \var{a}}
          {Equivalent to \code{not} \var{a}.has_key(\var{k})}
          {(2)}
  \lineiii{\var{a}.items()}
          {a copy of \var{a}'s list of (\var{key}, \var{value}) pairs}
          {(3)}
  \lineiii{\var{a}.keys()}{a copy of \var{a}'s list of keys}{(3)}
  \lineiii{\var{a}.update(\optional{\var{b}})}
          {updates (and overwrites) key/value pairs from \var{b}}
          {(9)}
  \lineiii{\var{a}.fromkeys(\var{seq}\optional{, \var{value}})}
          {Creates a new dictionary with keys from \var{seq} and values set to \var{value}}
          {(7)}			   
  \lineiii{\var{a}.values()}{a copy of \var{a}'s list of values}{(3)}
  \lineiii{\var{a}.get(\var{k}\optional{, \var{x}})}
          {\code{\var{a}[\var{k}]} if \code{\var{k} in \var{a}},
           else \var{x}}
          {(4)}
  \lineiii{\var{a}.setdefault(\var{k}\optional{, \var{x}})}
          {\code{\var{a}[\var{k}]} if \code{\var{k} in \var{a}},
           else \var{x} (also setting it)}
          {(5)}
  \lineiii{\var{a}.pop(\var{k}\optional{, \var{x}})}
          {\code{\var{a}[\var{k}]} if \code{\var{k} in \var{a}},
           else \var{x} (and remove k)}
          {(8)}
  \lineiii{\var{a}.popitem()}
          {remove and return an arbitrary (\var{key}, \var{value}) pair}
          {(6)}
  \lineiii{\var{a}.iteritems()}
          {return an iterator over (\var{key}, \var{value}) pairs}
          {(2), (3)}
  \lineiii{\var{a}.iterkeys()}
          {return an iterator over the mapping's keys}
          {(2), (3)}
  \lineiii{\var{a}.itervalues()}
          {return an iterator over the mapping's values}
          {(2), (3)}
\end{tableiii}

\noindent
Notes:
\begin{description}
\item[(1)] Raises a \exception{KeyError} exception if \var{k} is not
in the map.

\item[(2)] \versionadded{2.2}

\item[(3)] Keys and values are listed in an arbitrary order which is
non-random, varies across Python implementations, and depends on the
dictionary's history of insertions and deletions.
If \method{items()}, \method{keys()}, \method{values()},
\method{iteritems()}, \method{iterkeys()}, and \method{itervalues()}
are called with no intervening modifications to the dictionary, the
lists will directly correspond.  This allows the creation of
\code{(\var{value}, \var{key})} pairs using \function{zip()}:
\samp{pairs = zip(\var{a}.values(), \var{a}.keys())}.  The same
relationship holds for the \method{iterkeys()} and
\method{itervalues()} methods: \samp{pairs = zip(\var{a}.itervalues(),
\var{a}.iterkeys())} provides the same value for \code{pairs}.
Another way to create the same list is \samp{pairs = [(v, k) for (k,
v) in \var{a}.iteritems()]}.

\item[(4)] Never raises an exception if \var{k} is not in the map,
instead it returns \var{x}.  \var{x} is optional; when \var{x} is not
provided and \var{k} is not in the map, \code{None} is returned.

\item[(5)] \function{setdefault()} is like \function{get()}, except
that if \var{k} is missing, \var{x} is both returned and inserted into
the dictionary as the value of \var{k}. \var{x} defaults to \var{None}.

\item[(6)] \function{popitem()} is useful to destructively iterate
over a dictionary, as often used in set algorithms.  If the dictionary
is empty, calling \function{popitem()} raises a \exception{KeyError}.

\item[(7)] \function{fromkeys()} is a class method that returns a
new dictionary. \var{value} defaults to \code{None}.  \versionadded{2.3}

\item[(8)] \function{pop()} raises a \exception{KeyError} when no default
value is given and the key is not found.  \versionadded{2.3}

\item[(9)] \function{update()} accepts either another mapping object
or an iterable of key/value pairs (as a tuple or other iterable of
length two).  If keyword arguments are specified, the mapping is
then is updated with those key/value pairs:
\samp{d.update(red=1, blue=2)}.
\versionchanged[Allowed the argument to be an iterable of key/value
                pairs and allowed keyword arguments]{2.4}

\item[(10)] If a subclass of dict defines a method \method{__missing__},
if the key \var{k} is not present, the \var{a}[\var{k}] operation calls
that method with the key \var{k} as argument.  The \var{a}[\var{k}]
operation then returns or raises whatever is returned or raised by the
\function{__missing__}(\var{k}) call if the key is not present.
No other operations or methods invoke \method{__missing__}().
If \method{__missing__} is not defined, \exception{KeyError} is raised.
\method{__missing__} must be a method; it cannot be an instance variable.
For an example, see \module{collections}.\class{defaultdict}.
\versionadded{2.5}

\end{description}

\subsection{File Objects
            \label{bltin-file-objects}}

File objects\obindex{file} are implemented using C's \code{stdio}
package and can be created with the built-in constructor
\function{file()}\bifuncindex{file} described in section
\ref{built-in-funcs}, ``Built-in Functions.''\footnote{\function{file()}
is new in Python 2.2.  The older built-in \function{open()} is an
alias for \function{file()}.}  File objects are also returned
by some other built-in functions and methods, such as
\function{os.popen()} and \function{os.fdopen()} and the
\method{makefile()} method of socket objects.
\refstmodindex{os}
\refbimodindex{socket}

When a file operation fails for an I/O-related reason, the exception
\exception{IOError} is raised.  This includes situations where the
operation is not defined for some reason, like \method{seek()} on a tty
device or writing a file opened for reading.

Files have the following methods:


\begin{methoddesc}[file]{close}{}
  Close the file.  A closed file cannot be read or written any more.
  Any operation which requires that the file be open will raise a
  \exception{ValueError} after the file has been closed.  Calling
  \method{close()} more than once is allowed.

  As of Python 2.5, you can avoid having to call this method explicitly
  if you use the \keyword{with} statement.  For example, the following
  code will automatically close \code{f} when the \keyword{with} block
  is exited:

\begin{verbatim}
from __future__ import with_statement

with open("hello.txt") as f:
    for line in f:
        print line
\end{verbatim}

  In older versions of Python, you would have needed to do this to get
  the same effect:

\begin{verbatim}
f = open("hello.txt")
try:
    for line in f:
        print line
finally:
    f.close()
\end{verbatim}

  \note{Not all ``file-like'' types in Python support use as a context
  manager for the \keyword{with} statement.  If your code is intended to
  work with any file-like object, you can use the \function{closing()}
  function in the \module{contextlib} module instead of using the object
  directly.  See section~\ref{context-closing} for details.}
  
\end{methoddesc}

\begin{methoddesc}[file]{flush}{}
  Flush the internal buffer, like \code{stdio}'s
  \cfunction{fflush()}.  This may be a no-op on some file-like
  objects.
\end{methoddesc}

\begin{methoddesc}[file]{fileno}{}
  \index{file descriptor}
  \index{descriptor, file}
  Return the integer ``file descriptor'' that is used by the
  underlying implementation to request I/O operations from the
  operating system.  This can be useful for other, lower level
  interfaces that use file descriptors, such as the
  \refmodule{fcntl}\refbimodindex{fcntl} module or
  \function{os.read()} and friends.  \note{File-like objects
  which do not have a real file descriptor should \emph{not} provide
  this method!}
\end{methoddesc}

\begin{methoddesc}[file]{isatty}{}
  Return \code{True} if the file is connected to a tty(-like) device, else
  \code{False}.  \note{If a file-like object is not associated
  with a real file, this method should \emph{not} be implemented.}
\end{methoddesc}

\begin{methoddesc}[file]{next}{}
A file object is its own iterator, for example \code{iter(\var{f})} returns
\var{f} (unless \var{f} is closed).  When a file is used as an
iterator, typically in a \keyword{for} loop (for example,
\code{for line in f: print line}), the \method{next()} method is
called repeatedly.  This method returns the next input line, or raises
\exception{StopIteration} when \EOF{} is hit.  In order to make a
\keyword{for} loop the most efficient way of looping over the lines of
a file (a very common operation), the \method{next()} method uses a
hidden read-ahead buffer.  As a consequence of using a read-ahead
buffer, combining \method{next()} with other file methods (like
\method{readline()}) does not work right.  However, using
\method{seek()} to reposition the file to an absolute position will
flush the read-ahead buffer.
\versionadded{2.3}
\end{methoddesc}

\begin{methoddesc}[file]{read}{\optional{size}}
  Read at most \var{size} bytes from the file (less if the read hits
  \EOF{} before obtaining \var{size} bytes).  If the \var{size}
  argument is negative or omitted, read all data until \EOF{} is
  reached.  The bytes are returned as a string object.  An empty
  string is returned when \EOF{} is encountered immediately.  (For
  certain files, like ttys, it makes sense to continue reading after
  an \EOF{} is hit.)  Note that this method may call the underlying
  C function \cfunction{fread()} more than once in an effort to
  acquire as close to \var{size} bytes as possible. Also note that
  when in non-blocking mode, less data than what was requested may
  be returned, even if no \var{size} parameter was given.
\end{methoddesc}

\begin{methoddesc}[file]{readline}{\optional{size}}
  Read one entire line from the file.  A trailing newline character is
  kept in the string (but may be absent when a file ends with an
  incomplete line).\footnote{
	The advantage of leaving the newline on is that
	returning an empty string is then an unambiguous \EOF{}
	indication.  It is also possible (in cases where it might
	matter, for example, if you
	want to make an exact copy of a file while scanning its lines)
	to tell whether the last line of a file ended in a newline
	or not (yes this happens!).
  }  If the \var{size} argument is present and
  non-negative, it is a maximum byte count (including the trailing
  newline) and an incomplete line may be returned.
  An empty string is returned \emph{only} when \EOF{} is encountered
  immediately.  \note{Unlike \code{stdio}'s \cfunction{fgets()}, the
  returned string contains null characters (\code{'\e 0'}) if they
  occurred in the input.}
\end{methoddesc}

\begin{methoddesc}[file]{readlines}{\optional{sizehint}}
  Read until \EOF{} using \method{readline()} and return a list containing
  the lines thus read.  If the optional \var{sizehint} argument is
  present, instead of reading up to \EOF, whole lines totalling
  approximately \var{sizehint} bytes (possibly after rounding up to an
  internal buffer size) are read.  Objects implementing a file-like
  interface may choose to ignore \var{sizehint} if it cannot be
  implemented, or cannot be implemented efficiently.
\end{methoddesc}

\begin{methoddesc}[file]{xreadlines}{}
  This method returns the same thing as \code{iter(f)}.
  \versionadded{2.1}
  \deprecated{2.3}{Use \samp{for \var{line} in \var{file}} instead.}
\end{methoddesc}

\begin{methoddesc}[file]{seek}{offset\optional{, whence}}
  Set the file's current position, like \code{stdio}'s \cfunction{fseek()}.
  The \var{whence} argument is optional and defaults to \code{0}
  (absolute file positioning); other values are \code{1} (seek
  relative to the current position) and \code{2} (seek relative to the
  file's end).  There is no return value.  Note that if the file is
  opened for appending (mode \code{'a'} or \code{'a+'}), any
  \method{seek()} operations will be undone at the next write.  If the
  file is only opened for writing in append mode (mode \code{'a'}),
  this method is essentially a no-op, but it remains useful for files
  opened in append mode with reading enabled (mode \code{'a+'}).  If the
  file is opened in text mode (without \code{'b'}), only offsets returned
  by \method{tell()} are legal.  Use of other offsets causes undefined
  behavior.

  Note that not all file objects are seekable.
\end{methoddesc}

\begin{methoddesc}[file]{tell}{}
  Return the file's current position, like \code{stdio}'s
  \cfunction{ftell()}.

  \note{On Windows, \method{tell()} can return illegal values (after an
  \cfunction{fgets()}) when reading files with \UNIX{}-style line-endings.
  Use binary mode (\code{'rb'}) to circumvent this problem.}
\end{methoddesc}

\begin{methoddesc}[file]{truncate}{\optional{size}}
  Truncate the file's size.  If the optional \var{size} argument is
  present, the file is truncated to (at most) that size.  The size
  defaults to the current position.  The current file position is
  not changed.  Note that if a specified size exceeds the file's
  current size, the result is platform-dependent:  possibilities
  include that the file may remain unchanged, increase to the specified
  size as if zero-filled, or increase to the specified size with
  undefined new content.
  Availability:  Windows, many \UNIX{} variants.
\end{methoddesc}

\begin{methoddesc}[file]{write}{str}
  Write a string to the file.  There is no return value.  Due to
  buffering, the string may not actually show up in the file until
  the \method{flush()} or \method{close()} method is called.
\end{methoddesc}

\begin{methoddesc}[file]{writelines}{sequence}
  Write a sequence of strings to the file.  The sequence can be any
  iterable object producing strings, typically a list of strings.
  There is no return value.
  (The name is intended to match \method{readlines()};
  \method{writelines()} does not add line separators.)
\end{methoddesc}


Files support the iterator protocol.  Each iteration returns the same
result as \code{\var{file}.readline()}, and iteration ends when the
\method{readline()} method returns an empty string.


File objects also offer a number of other interesting attributes.
These are not required for file-like objects, but should be
implemented if they make sense for the particular object.

\begin{memberdesc}[file]{closed}
bool indicating the current state of the file object.  This is a
read-only attribute; the \method{close()} method changes the value.
It may not be available on all file-like objects.
\end{memberdesc}

\begin{memberdesc}[file]{encoding}
The encoding that this file uses. When Unicode strings are written
to a file, they will be converted to byte strings using this encoding.
In addition, when the file is connected to a terminal, the attribute
gives the encoding that the terminal is likely to use (that 
information might be incorrect if the user has misconfigured the 
terminal). The attribute is read-only and may not be present on
all file-like objects. It may also be \code{None}, in which case
the file uses the system default encoding for converting Unicode
strings.

\versionadded{2.3}
\end{memberdesc}

\begin{memberdesc}[file]{mode}
The I/O mode for the file.  If the file was created using the
\function{open()} built-in function, this will be the value of the
\var{mode} parameter.  This is a read-only attribute and may not be
present on all file-like objects.
\end{memberdesc}

\begin{memberdesc}[file]{name}
If the file object was created using \function{open()}, the name of
the file.  Otherwise, some string that indicates the source of the
file object, of the form \samp{<\mbox{\ldots}>}.  This is a read-only
attribute and may not be present on all file-like objects.
\end{memberdesc}

\begin{memberdesc}[file]{newlines}
If Python was built with the \longprogramopt{with-universal-newlines}
option to \program{configure} (the default) this read-only attribute
exists, and for files opened in
universal newline read mode it keeps track of the types of newlines
encountered while reading the file. The values it can take are
\code{'\e r'}, \code{'\e n'}, \code{'\e r\e n'}, \code{None} (unknown,
no newlines read yet) or a tuple containing all the newline
types seen, to indicate that multiple
newline conventions were encountered. For files not opened in universal
newline read mode the value of this attribute will be \code{None}.
\end{memberdesc}

\begin{memberdesc}[file]{softspace}
Boolean that indicates whether a space character needs to be printed
before another value when using the \keyword{print} statement.
Classes that are trying to simulate a file object should also have a
writable \member{softspace} attribute, which should be initialized to
zero.  This will be automatic for most classes implemented in Python
(care may be needed for objects that override attribute access); types
implemented in C will have to provide a writable
\member{softspace} attribute.
\note{This attribute is not used to control the
\keyword{print} statement, but to allow the implementation of
\keyword{print} to keep track of its internal state.}
\end{memberdesc}


\subsection{Other Built-in Types \label{typesother}}

The interpreter supports several other kinds of objects.
Most of these support only one or two operations.


\subsubsection{Modules \label{typesmodules}}

The only special operation on a module is attribute access:
\code{\var{m}.\var{name}}, where \var{m} is a module and \var{name}
accesses a name defined in \var{m}'s symbol table.  Module attributes
can be assigned to.  (Note that the \keyword{import} statement is not,
strictly speaking, an operation on a module object; \code{import
\var{foo}} does not require a module object named \var{foo} to exist,
rather it requires an (external) \emph{definition} for a module named
\var{foo} somewhere.)

A special member of every module is \member{__dict__}.
This is the dictionary containing the module's symbol table.
Modifying this dictionary will actually change the module's symbol
table, but direct assignment to the \member{__dict__} attribute is not
possible (you can write \code{\var{m}.__dict__['a'] = 1}, which
defines \code{\var{m}.a} to be \code{1}, but you can't write
\code{\var{m}.__dict__ = \{\}}).  Modifying \member{__dict__} directly
is not recommended.

Modules built into the interpreter are written like this:
\code{<module 'sys' (built-in)>}.  If loaded from a file, they are
written as \code{<module 'os' from
'/usr/local/lib/python\shortversion/os.pyc'>}.


\subsubsection{Classes and Class Instances \label{typesobjects}}
\nodename{Classes and Instances}

See chapters 3 and 7 of the \citetitle[../ref/ref.html]{Python
Reference Manual} for these.


\subsubsection{Functions \label{typesfunctions}}

Function objects are created by function definitions.  The only
operation on a function object is to call it:
\code{\var{func}(\var{argument-list})}.

There are really two flavors of function objects: built-in functions
and user-defined functions.  Both support the same operation (to call
the function), but the implementation is different, hence the
different object types.

See the \citetitle[../ref/ref.html]{Python Reference Manual} for more
information.

\subsubsection{Methods \label{typesmethods}}
\obindex{method}

Methods are functions that are called using the attribute notation.
There are two flavors: built-in methods (such as \method{append()} on
lists) and class instance methods.  Built-in methods are described
with the types that support them.

The implementation adds two special read-only attributes to class
instance methods: \code{\var{m}.im_self} is the object on which the
method operates, and \code{\var{m}.im_func} is the function
implementing the method.  Calling \code{\var{m}(\var{arg-1},
\var{arg-2}, \textrm{\ldots}, \var{arg-n})} is completely equivalent to
calling \code{\var{m}.im_func(\var{m}.im_self, \var{arg-1},
\var{arg-2}, \textrm{\ldots}, \var{arg-n})}.

Class instance methods are either \emph{bound} or \emph{unbound},
referring to whether the method was accessed through an instance or a
class, respectively.  When a method is unbound, its \code{im_self}
attribute will be \code{None} and if called, an explicit \code{self}
object must be passed as the first argument.  In this case,
\code{self} must be an instance of the unbound method's class (or a
subclass of that class), otherwise a \exception{TypeError} is raised.

Like function objects, methods objects support getting
arbitrary attributes.  However, since method attributes are actually
stored on the underlying function object (\code{meth.im_func}),
setting method attributes on either bound or unbound methods is
disallowed.  Attempting to set a method attribute results in a
\exception{TypeError} being raised.  In order to set a method attribute,
you need to explicitly set it on the underlying function object:

\begin{verbatim}
class C:
    def method(self):
        pass

c = C()
c.method.im_func.whoami = 'my name is c'
\end{verbatim}

See the \citetitle[../ref/ref.html]{Python Reference Manual} for more
information.


\subsubsection{Code Objects \label{bltin-code-objects}}
\obindex{code}

Code objects are used by the implementation to represent
``pseudo-compiled'' executable Python code such as a function body.
They differ from function objects because they don't contain a
reference to their global execution environment.  Code objects are
returned by the built-in \function{compile()} function and can be
extracted from function objects through their \member{func_code}
attribute.
\bifuncindex{compile}
\withsubitem{(function object attribute)}{\ttindex{func_code}}

A code object can be executed or evaluated by passing it (instead of a
source string) to the \keyword{exec} statement or the built-in
\function{eval()} function.
\stindex{exec}
\bifuncindex{eval}

See the \citetitle[../ref/ref.html]{Python Reference Manual} for more
information.


\subsubsection{Type Objects \label{bltin-type-objects}}

Type objects represent the various object types.  An object's type is
accessed by the built-in function \function{type()}.  There are no special
operations on types.  The standard module \refmodule{types} defines names
for all standard built-in types.
\bifuncindex{type}
\refstmodindex{types}

Types are written like this: \code{<type 'int'>}.


\subsubsection{The Null Object \label{bltin-null-object}}

This object is returned by functions that don't explicitly return a
value.  It supports no special operations.  There is exactly one null
object, named \code{None} (a built-in name).

It is written as \code{None}.


\subsubsection{The Ellipsis Object \label{bltin-ellipsis-object}}

This object is used by extended slice notation (see the
\citetitle[../ref/ref.html]{Python Reference Manual}).  It supports no
special operations.  There is exactly one ellipsis object, named
\constant{Ellipsis} (a built-in name).

It is written as \code{Ellipsis}.

\subsubsection{Boolean Values}

Boolean values are the two constant objects \code{False} and
\code{True}.  They are used to represent truth values (although other
values can also be considered false or true).  In numeric contexts
(for example when used as the argument to an arithmetic operator),
they behave like the integers 0 and 1, respectively.  The built-in
function \function{bool()} can be used to cast any value to a Boolean,
if the value can be interpreted as a truth value (see section Truth
Value Testing above).

They are written as \code{False} and \code{True}, respectively.
\index{False}
\index{True}
\indexii{Boolean}{values}


\subsubsection{Internal Objects \label{typesinternal}}

See the \citetitle[../ref/ref.html]{Python Reference Manual} for this
information.  It describes stack frame objects, traceback objects, and
slice objects.


\subsection{Special Attributes \label{specialattrs}}

The implementation adds a few special read-only attributes to several
object types, where they are relevant.  Some of these are not reported
by the \function{dir()} built-in function.

\begin{memberdesc}[object]{__dict__}
A dictionary or other mapping object used to store an
object's (writable) attributes.
\end{memberdesc}

\begin{memberdesc}[object]{__methods__}
\deprecated{2.2}{Use the built-in function \function{dir()} to get a
list of an object's attributes.  This attribute is no longer available.}
\end{memberdesc}

\begin{memberdesc}[object]{__members__}
\deprecated{2.2}{Use the built-in function \function{dir()} to get a
list of an object's attributes.  This attribute is no longer available.}
\end{memberdesc}

\begin{memberdesc}[instance]{__class__}
The class to which a class instance belongs.
\end{memberdesc}

\begin{memberdesc}[class]{__bases__}
The tuple of base classes of a class object.  If there are no base
classes, this will be an empty tuple.
\end{memberdesc}

\begin{memberdesc}[class]{__name__}
The name of the class or type.
\end{memberdesc}
