\section{Standard Module \sectcode{urllib}}
\stmodindex{urllib}
\index{WWW}
\index{World-Wide Web}
\index{URL}

\renewcommand{\indexsubitem}{(in module urllib)}

This module provides a high-level interface for fetching data across
the World-Wide Web.  In particular, the \code{urlopen} function is
similar to the built-in function \code{open}, but accepts URLs
(Universal Resource Locators) instead of filenames.  Some restrictions
apply --- it can only open URLs for reading, and no seek operations
are available.

it defines the following public functions:

\begin{funcdesc}{urlopen}{url}
Open a network object denoted by a URL for reading.  If the URL does
not have a scheme identifier, or if it has \samp{file:} as its scheme
identifier, this opens a local file; otherwise it opens a socket to a
server somewhere on the network.  If the connection cannot be made, or
if the server returns an error code, the \code{IOError} exception is
raised.  If all went well, a file-like object is returned.  This
supports the following methods: \code{read()}, \code{readline()},
\code{readlines()}, \code{fileno()}, \code{close()} and \code{info()}.
Except for the last one, these methods have the same interface as for
file objects --- see the section on File Objects earlier in this
manual.  (It's not a built-in file object, however, so it can't be
used at those few places where a true built-in file object is
required.)

The \code{info()} method returns an instance of the class
\code{rfc822.Message} containing the headers received from the server,
if the protocol uses such headers (currently the only supported
protocol that uses this is HTTP).  See the description of the
\code{rfc822} module.
\end{funcdesc}

\begin{funcdesc}{urlretrieve}{url}
Copy a network object denoted by a URL to a local file, if necessary.
If the URL points to a local file, or a valid cached copy of the
object exists, the object is not copied.  Return a tuple (\var{filename},
\var{headers}) where \var{filename} is the local file name under which
the object can be found, and \var{headers} is either \code{None} (for
a local object) or whatever the \code{info()} method of the object
returned by \code{urlopen()} returned (for a remote object, possibly
cached).  Exceptions are the same as for \code{urlopen()}.
\end{funcdesc}

\begin{funcdesc}{urlcleanup}{}
Clear the cache that may have been built up by previous calls to
\code{urlretrieve()}.
\end{funcdesc}

\begin{funcdesc}{quote}{string\optional{\, addsafe}}
Replace special characters in \var{string} using the \code{\%xx} escape.
Letters, digits, and the characters ``\code{_,.-}'' are never quoted.
The optional \var{addsafe} parameter specifies additional characters
that should not be quoted --- its default value is \code{'/'}.

Example: \code{quote('/\~connolly/')} yields \code{'/\%7econnolly/'}.
\end{funcdesc}

\begin{funcdesc}{quote_plus}{string\optional{\, addsafe}}
Like \code{quote()}, but also replaces spaces by plus signs, as
required for quoting HTML form values.
\end{funcdesc}

\begin{funcdesc}{unquote}{string}
Replace \samp{\%xx} escapes by their single-character equivalent.

Example: \code{unquote('/\%7Econnolly/')} yields \code{'/\~connolly/'}.
\end{funcdesc}

\begin{funcdesc}{unquote_plus}{string}
Like \code{unquote()}, but also replaces plus signs by spaces, as
required for unquoting HTML form values.
\end{funcdesc}

Restrictions:

\begin{itemize}

\item
Currently, only the following protocols are supported: HTTP, (versions
0.9 and 1.0), Gopher (but not Gopher-+), FTP, and local files.
\index{HTTP}
\index{Gopher}
\index{FTP}

\item
The caching feature of \code{urlretrieve()} has been disabled until I
find the time to hack proper processing of Expiration time headers.

\item
There should be a function to query whether a particular URL is in
the cache.

\item
For backward compatibility, if a URL appears to point to a local file
but the file can't be opened, the URL is re-interpreted using the FTP
protocol.  This can sometimes cause confusing error messages.

\item
The \code{urlopen()} and \code{urlretrieve()} functions can cause
arbitrarily long delays while waiting for a network connection to be
set up.  This means that it is difficult to build an interactive
web client using these functions without using threads.

\item
The data returned by \code{urlopen()} or \code{urlretrieve()} is the
raw data returned by the server.  This may be binary data (e.g. an
image), plain text or (for example) HTML.  The HTTP protocol provides
type information in the reply header, which can be inspected by
looking at the \code{Content-type} header.  For the Gopher protocol,
type information is encoded in the URL; there is currently no easy way
to extract it.  If the returned data is HTML, you can use the module
\code{htmllib} to parse it.
\index{HTML}
\index{HTTP}
\index{Gopher}
\stmodindex{htmllib}

\item
Although the \code{urllib} module contains (undocumented) routines to
parse and unparse URL strings, the recommended interface for URL
manipulation is in module \code{urlparse}.
\stmodindex{urlparse}

\end{itemize}
