\section{Built-in Module \sectcode{regex}}
\label{module-regex}

\bimodindex{regex}
This module provides regular expression matching operations similar to
those found in Emacs.

\strong{Obsolescence note:}
This module is obsolete as of Python version 1.5; it is still being
maintained because much existing code still uses it.  All new code in
need of regular expressions should use the new \code{re} module, which
supports the more powerful and regular Perl-style regular expressions.
Existing code should be converted.  The standard library module
\code{reconvert} helps in converting \code{regex} style regular
expressions to \code{re} style regular expressions.  (For more
conversion help, see the URL
\file{http://starship.skyport.net/crew/amk/regex/regex-to-re.html}.)

By default the patterns are Emacs-style regular expressions
(with one exception).  There is
a way to change the syntax to match that of several well-known
\UNIX{} utilities.  The exception is that Emacs' \samp{\e s}
pattern is not supported, since the original implementation references
the Emacs syntax tables.

This module is 8-bit clean: both patterns and strings may contain null
bytes and characters whose high bit is set.

\strong{Please note:} There is a little-known fact about Python string
literals which means that you don't usually have to worry about
doubling backslashes, even though they are used to escape special
characters in string literals as well as in regular expressions.  This
is because Python doesn't remove backslashes from string literals if
they are followed by an unrecognized escape character.
\emph{However}, if you want to include a literal \dfn{backslash} in a
regular expression represented as a string literal, you have to
\emph{quadruple} it or enclose it in a singleton character class.
E.g.\  to extract \LaTeX\ \samp{\e section\{{\rm
\ldots}\}} headers from a document, you can use this pattern:
\code{'[\e ]section\{\e (.*\e )\}'}.  \emph{Another exception:}
the escape sequece \samp{\e b} is significant in string literals
(where it means the ASCII bell character) as well as in Emacs regular
expressions (where it stands for a word boundary), so in order to
search for a word boundary, you should use the pattern \code{'\e \e b'}.
Similarly, a backslash followed by a digit 0-7 should be doubled to
avoid interpretation as an octal escape.

\subsection{Regular Expressions}

A regular expression (or RE) specifies a set of strings that matches
it; the functions in this module let you check if a particular string
matches a given regular expression (or if a given regular expression
matches a particular string, which comes down to the same thing).

Regular expressions can be concatenated to form new regular
expressions; if \emph{A} and \emph{B} are both regular expressions,
then \emph{AB} is also an regular expression.  If a string \emph{p}
matches A and another string \emph{q} matches B, the string \emph{pq}
will match AB.  Thus, complex expressions can easily be constructed
from simpler ones like the primitives described here.  For details of
the theory and implementation of regular expressions, consult almost
any textbook about compiler construction.

% XXX The reference could be made more specific, say to 
% "Compilers: Principles, Techniques and Tools", by Alfred V. Aho, 
% Ravi Sethi, and Jeffrey D. Ullman, or some FA text.   

A brief explanation of the format of regular expressions follows.

Regular expressions can contain both special and ordinary characters.
Ordinary characters, like '\code{A}', '\code{a}', or '\code{0}', are
the simplest regular expressions; they simply match themselves.  You
can concatenate ordinary characters, so '\code{last}' matches the
characters 'last'.  (In the rest of this section, we'll write RE's in
\code{this special font}, usually without quotes, and strings to be
matched 'in single quotes'.)

Special characters either stand for classes of ordinary characters, or
affect how the regular expressions around them are interpreted.

The special characters are:
\begin{itemize}
\item[\code{.}] (Dot.)  Matches any character except a newline.
\item[\code{\^}] (Caret.)  Matches the start of the string.
\item[\code{\$}] Matches the end of the string.  
\code{foo} matches both 'foo' and 'foobar', while the regular
expression '\code{foo\$}' matches only 'foo'.
\item[\code{*}] Causes the resulting RE to
match 0 or more repetitions of the preceding RE.  \code{ab*} will
match 'a', 'ab', or 'a' followed by any number of 'b's.
\item[\code{+}] Causes the
resulting RE to match 1 or more repetitions of the preceding RE.
\code{ab+} will match 'a' followed by any non-zero number of 'b's; it
will not match just 'a'.
\item[\code{?}] Causes the resulting RE to
match 0 or 1 repetitions of the preceding RE.  \code{ab?} will
match either 'a' or 'ab'.

\item[\code{\e}] Either escapes special characters (permitting you to match
characters like '*?+\&\$'), or signals a special sequence; special
sequences are discussed below.  Remember that Python also uses the
backslash as an escape sequence in string literals; if the escape
sequence isn't recognized by Python's parser, the backslash and
subsequent character are included in the resulting string.  However,
if Python would recognize the resulting sequence, the backslash should
be repeated twice.  

\item[\code{[]}] Used to indicate a set of characters.  Characters can
be listed individually, or a range is indicated by giving two
characters and separating them by a '-'.  Special characters are
not active inside sets.  For example, \code{[akm\$]}
will match any of the characters 'a', 'k', 'm', or '\$'; \code{[a-z]} will
match any lowercase letter.  

If you want to include a \code{]} inside a
set, it must be the first character of the set; to include a \code{-},
place it as the first or last character. 

Characters \emph{not} within a range can be matched by including a
\code{\^} as the first character of the set; \code{\^} elsewhere will
simply match the '\code{\^}' character.  
\end{itemize}

The special sequences consist of '\code{\e}' and a character
from the list below.  If the ordinary character is not on the list,
then the resulting RE will match the second character.  For example,
\code{\e\$} matches the character '\$'.  Ones where the backslash
should be doubled in string literals are indicated.

\begin{itemize}
\item[\code{\e|}]\code{A\e|B}, where A and B can be arbitrary REs,
creates a regular expression that will match either A or B.  This can
be used inside groups (see below) as well.
%
\item[\code{\e( \e)}] Indicates the start and end of a group; the
contents of a group can be matched later in the string with the
\code{\e [1-9]} special sequence, described next.
%
{\fulllineitems\item[\code{\e \e 1, ... \e \e 7, \e 8, \e 9}]
Matches the contents of the group of the same
number.  For example, \code{\e (.+\e ) \e \e 1} matches 'the the' or
'55 55', but not 'the end' (note the space after the group).  This
special sequence can only be used to match one of the first 9 groups;
groups with higher numbers can be matched using the \code{\e v}
sequence.  (\code{\e 8} and \code{\e 9} don't need a double backslash
because they are not octal digits.)}
%
\item[\code{\e \e b}] Matches the empty string, but only at the
beginning or end of a word.  A word is defined as a sequence of
alphanumeric characters, so the end of a word is indicated by
whitespace or a non-alphanumeric character.
%
\item[\code{\e B}] Matches the empty string, but when it is \emph{not} at the
beginning or end of a word.
%
\item[\code{\e v}] Must be followed by a two digit decimal number, and
matches the contents of the group of the same number.  The group number must be between 1 and 99, inclusive.
%
\item[\code{\e w}]Matches any alphanumeric character; this is
equivalent to the set \code{[a-zA-Z0-9]}.
%
\item[\code{\e W}] Matches any non-alphanumeric character; this is
equivalent to the set \code{[\^a-zA-Z0-9]}.
\item[\code{\e <}] Matches the empty string, but only at the beginning of a
word.  A word is defined as a sequence of alphanumeric characters, so
the end of a word is indicated by whitespace or a non-alphanumeric 
character.
\item[\code{\e >}] Matches the empty string, but only at the end of a
word.

\item[\code{\e \e \e \e}] Matches a literal backslash.

% In Emacs, the following two are start of buffer/end of buffer.  In
% Python they seem to be synonyms for ^$.
\item[\code{\e `}] Like \code{\^}, this only matches at the start of the
string.
\item[\code{\e \e '}] Like \code{\$}, this only matches at the end of the
string.
% end of buffer
\end{itemize}

\subsection{Module Contents}

The module defines these functions, and an exception:

\renewcommand{\indexsubitem}{(in module regex)}

\begin{funcdesc}{match}{pattern\, string}
  Return how many characters at the beginning of \var{string} match
  the regular expression \var{pattern}.  Return \code{-1} if the
  string does not match the pattern (this is different from a
  zero-length match!).
\end{funcdesc}

\begin{funcdesc}{search}{pattern\, string}
  Return the first position in \var{string} that matches the regular
  expression \var{pattern}.  Return \code{-1} if no position in the string
  matches the pattern (this is different from a zero-length match
  anywhere!).
\end{funcdesc}

\begin{funcdesc}{compile}{pattern\optional{\, translate}}
  Compile a regular expression pattern into a regular expression
  object, which can be used for matching using its \code{match} and
  \code{search} methods, described below.  The optional argument
  \var{translate}, if present, must be a 256-character string
  indicating how characters (both of the pattern and of the strings to
  be matched) are translated before comparing them; the \code{i}-th
  element of the string gives the translation for the character with
  \ASCII{} code \code{i}.  This can be used to implement
  case-insensitive matching; see the \code{casefold} data item below.

  The sequence

\bcode\begin{verbatim}
prog = regex.compile(pat)
result = prog.match(str)
\end{verbatim}\ecode
%
is equivalent to

\bcode\begin{verbatim}
result = regex.match(pat, str)
\end{verbatim}\ecode
%
but the version using \code{compile()} is more efficient when multiple
regular expressions are used concurrently in a single program.  (The
compiled version of the last pattern passed to \code{regex.match()} or
\code{regex.search()} is cached, so programs that use only a single
regular expression at a time needn't worry about compiling regular
expressions.)
\end{funcdesc}

\begin{funcdesc}{set_syntax}{flags}
  Set the syntax to be used by future calls to \code{compile},
  \code{match} and \code{search}.  (Already compiled expression objects
  are not affected.)  The argument is an integer which is the OR of
  several flag bits.  The return value is the previous value of
  the syntax flags.  Names for the flags are defined in the standard
  module \code{regex_syntax}; read the file \file{regex_syntax.py} for
  more information.
\end{funcdesc}

\begin{funcdesc}{get_syntax}{}
  Returns the current value of the syntax flags as an integer.
\end{funcdesc}

\begin{funcdesc}{symcomp}{pattern\optional{\, translate}}
This is like \code{compile}, but supports symbolic group names: if a
parenthesis-enclosed group begins with a group name in angular
brackets, e.g. \code{'\e(<id>[a-z][a-z0-9]*\e)'}, the group can
be referenced by its name in arguments to the \code{group} method of
the resulting compiled regular expression object, like this:
\code{p.group('id')}.  Group names may contain alphanumeric characters
and \code{'_'} only.
\end{funcdesc}

\begin{excdesc}{error}
  Exception raised when a string passed to one of the functions here
  is not a valid regular expression (e.g., unmatched parentheses) or
  when some other error occurs during compilation or matching.  (It is
  never an error if a string contains no match for a pattern.)
\end{excdesc}

\begin{datadesc}{casefold}
A string suitable to pass as \var{translate} argument to
\code{compile} to map all upper case characters to their lowercase
equivalents.
\end{datadesc}

\noindent
Compiled regular expression objects support these methods:

\renewcommand{\indexsubitem}{(regex method)}
\begin{funcdesc}{match}{string\optional{\, pos}}
  Return how many characters at the beginning of \var{string} match
  the compiled regular expression.  Return \code{-1} if the string
  does not match the pattern (this is different from a zero-length
  match!).
  
  The optional second parameter \var{pos} gives an index in the string
  where the search is to start; it defaults to \code{0}.  This is not
  completely equivalent to slicing the string; the \code{'\^'} pattern
  character matches at the real begin of the string and at positions
  just after a newline, not necessarily at the index where the search
  is to start.
\end{funcdesc}

\begin{funcdesc}{search}{string\optional{\, pos}}
  Return the first position in \var{string} that matches the regular
  expression \code{pattern}.  Return \code{-1} if no position in the
  string matches the pattern (this is different from a zero-length
  match anywhere!).
  
  The optional second parameter has the same meaning as for the
  \code{match} method.
\end{funcdesc}

\begin{funcdesc}{group}{index\, index\, ...}
This method is only valid when the last call to the \code{match}
or \code{search} method found a match.  It returns one or more
groups of the match.  If there is a single \var{index} argument,
the result is a single string; if there are multiple arguments, the
result is a tuple with one item per argument.  If the \var{index} is
zero, the corresponding return value is the entire matching string; if
it is in the inclusive range [1..99], it is the string matching the
the corresponding parenthesized group (using the default syntax,
groups are parenthesized using \code{{\e}(} and \code{{\e})}).  If no
such group exists, the corresponding result is \code{None}.

If the regular expression was compiled by \code{symcomp} instead of
\code{compile}, the \var{index} arguments may also be strings
identifying groups by their group name.
\end{funcdesc}

\noindent
Compiled regular expressions support these data attributes:

\renewcommand{\indexsubitem}{(regex attribute)}

\begin{datadesc}{regs}
When the last call to the \code{match} or \code{search} method found a
match, this is a tuple of pairs of indices corresponding to the
beginning and end of all parenthesized groups in the pattern.  Indices
are relative to the string argument passed to \code{match} or
\code{search}.  The 0-th tuple gives the beginning and end or the
whole pattern.  When the last match or search failed, this is
\code{None}.
\end{datadesc}

\begin{datadesc}{last}
When the last call to the \code{match} or \code{search} method found a
match, this is the string argument passed to that method.  When the
last match or search failed, this is \code{None}.
\end{datadesc}

\begin{datadesc}{translate}
This is the value of the \var{translate} argument to
\code{regex.compile} that created this regular expression object.  If
the \var{translate} argument was omitted in the \code{regex.compile}
call, this is \code{None}.
\end{datadesc}

\begin{datadesc}{givenpat}
The regular expression pattern as passed to \code{compile} or
\code{symcomp}.
\end{datadesc}

\begin{datadesc}{realpat}
The regular expression after stripping the group names for regular
expressions compiled with \code{symcomp}.  Same as \code{givenpat}
otherwise.
\end{datadesc}

\begin{datadesc}{groupindex}
A dictionary giving the mapping from symbolic group names to numerical
group indices for regular expressions compiled with \code{symcomp}.
\code{None} otherwise.
\end{datadesc}
