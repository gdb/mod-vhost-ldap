\section{Built-in Module {\tt crypt}}
\label{module-crypt}
\bimodindex{crypt}

This module implements an interface to the crypt({\bf 3}) routine,
which is a one-way hash function based upon a modified DES algorithm;
see the Unix man page for further details.  Possible uses include
allowing Python scripts to accept typed passwords from the user, or
attempting to crack Unix passwords with a dictionary.
\index{crypt(3)}

\renewcommand{\indexsubitem}{(in module crypt)}
\begin{funcdesc}{crypt}{word\, salt} 
\var{word} will usually be a user's password.  \var{salt} is a
2-character string which will be used to select one of 4096 variations
of DES.  The characters in \var{salt} must be either \code{.},
\code{/}, or an alphanumeric character.  Returns the hashed password
as a string, which will be composed of characters from the same
alphabet as the salt.
\end{funcdesc}

The module and documentation were written by Steve Majewski.
\index{Majewski, Steve}
