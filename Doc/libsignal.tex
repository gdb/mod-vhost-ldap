\section{Built-in Module \sectcode{signal}}

\bimodindex{signal}
This module provides mechanisms to write signal handlers in Python.

{\bf Warning:} Some care must be taken if both signals and threads
will be used in the same program.  The fundamental thing to remember
in using signals and threads simultaneously is: always perform
\code{signal()} operations in the main thread of execution.  Any
thread can perform a \code{alarm()}, \code{getsignal()}, or
\code{pause()}; only the main thread can set a new signal handler, and
the main thread will be the only one to receive signals.  This means
that signals can't be used as a means of interthread communication.
Use locks instead.

The variables defined in the signal module are:

\renewcommand{\indexsubitem}{(in module signal)}
\begin{datadesc}{SIG_DFL}
  This is one of two standard signal handling options; it will simply
  perform the default function for the signal.  For example, on most
  systems the default action for SIGQUIT is to dump core and exit,
  while the default action for SIGCLD is to simply ignore it.
\end{datadesc}

\begin{datadesc}{SIG_IGN}
  This is another standard signal handler, which will simply ignore
  the given signal.
\end{datadesc}

\begin{datadesc}{SIG*}
  All the signal numbers are defined symbolically.  For example, the
  hangup signal is defined as \code{signal.SIGHUP}; the variable names
  are identical to the names used in C programs, as found in
  \file{signal.h}.
  The UNIX man page for \file{signal} lists the existing signals (on
  some systems this is \file{signal(2)}, on others the list is in
  \file{signal(7)}).
  Note that not all systems define the same set of signal names; only
  those names defined by the system are defined by this module.
\end{datadesc}

The signal module defines the following functions:

\begin{funcdesc}{alarm}{time}
  If \var{time} is non-zero, this function requests that a
  \code{SIGALRM} signal be sent to the process in \var{time} seconds.
  Any previously scheduled alarm is canceled (i.e. only one alarm can
  be scheduled at any time).  The returned value is then the number of
  seconds before any previously set alarm was to have been delivered.
  If \var{time} is zero, no alarm id scheduled, and any scheduled
  alarm is canceled.  The return value is the number of seconds
  remaining before a previously scheduled alarm.  If the return value
  is zero, no alarm is currently scheduled.  (See the UNIX man page
  \code{alarm(2)}.)
\end{funcdesc}

\begin{funcdesc}{getsignal}{signalnum}
  Returns the current signal handler for the signal \var{signalnum}.
  The returned value may be a callable Python object, or one of the
  special values \code{signal.SIG_IGN} or \code{signal.SIG_DFL}.
\end{funcdesc}

\begin{funcdesc}{pause}{}
  Causes the process to sleep until a signal is received; the
  appropriate handler will then be called.  Returns nothing.  (See the
  UNIX man page \code{signal(2)}.)
\end{funcdesc}

\begin{funcdesc}{signal}{signalnum\, handler}
  Sets the handler for signal \var{signalnum} to the function
  \var{handler}.  \var{handler} can be any callable Python object, or
  one of the special values \code{signal.SIG_IGN} or
  \code{signal.SIG_DFL}.  The previous signal handler will be
  returned.  (See the UNIX man page \code{signal(2)}.)

  If threads are enabled, this function can only be called from the
  main thread; attempting to call it from other threads will cause a
  \code{ValueError} exception will be raised.
\end{funcdesc}
