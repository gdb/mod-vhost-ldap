\section{Built-in Module \sectcode{imp}}
\bimodindex{imp}
\index{import}

This module provides an interface to the mechanisms used to implement
the \code{import} statement.  It defines the following constants and
functions:

\renewcommand{\indexsubitem}{(in module struct)}

\begin{funcdesc}{get_magic}{}
Return the magic string value used to recognize byte-compiled code
files (``\code{.pyc} files'').
\end{funcdesc}

\begin{funcdesc}{get_suffixes}{}
Return a list of triples, each describing a particular type of file.
Each triple has the form \code{(\var{suffix}, \var{mode},
\var{type})}, where \var{suffix} is a string to be appended to the
module name to form the filename to search for, \var{mode} is the mode
string to pass to the built-in \code{open} function to open the file
(this can be \code{'r'} for text files or \code{'rb'} for binary
files), and \var{type} is the file type, which has one of the values
\code{PY_SOURCE}, \code{PY_COMPILED} or \code{C_EXTENSION}, defined
below.  (System-dependent values may also be returned.)
\end{funcdesc}

\begin{funcdesc}{find_module}{name\, \optional{path}}
Try to find the module \var{name} on the search path \var{path}.  The
default \var{path} is \code{sys.path}.  The return value is a triple
\code{(\var{file}, \var{pathname}, \var{description})} where
\var{file} is an open file object positioned at the beginning,
\var{pathname} is the pathname of the
file found, and \var{description} is a triple as contained in the list
returned by \code{get_suffixes} describing the kind of file found.
\end{funcdesc}

\begin{funcdesc}{init_builtin}{name}
Initialize the built-in module called \var{name} and return its module
object.  If the module was already initialized, it will be initialized
{\em again}.  A few modules cannot be initialized twice --- attempting
to initialize these again will raise an \code{ImportError} exception.
If there is no
built-in module called \var{name}, \code{None} is returned.
\end{funcdesc}

\begin{funcdesc}{init_frozen}{name}
Initialize the frozen module called \var{name} and return its module
object.  If the module was already initialized, it will be initialized
{\em again}.  If there is no frozen module called \var{name},
\code{None} is returned.  (Frozen modules are modules written in
Python whose compiled byte-code object is incorporated into a
custom-built Python interpreter by Python's \code{freeze} utility.
See \code{Tools/freeze} for now.)
\end{funcdesc}

\begin{funcdesc}{is_builtin}{name}
Return \code{1} if there is a built-in module called \var{name} which can be
initialized again.  Return \code{-1} if there is a built-in module
called \var{name} which cannot be initialized again (see
\code{init_builtin}).  Return \code{0} if there is no built-in module
called \var{name}.
\end{funcdesc}

\begin{funcdesc}{is_frozen}{name}
Return \code{1} if there is a frozen module (see \code{init_frozen})
called \var{name}, \code{0} if there is no such module.
\end{funcdesc}

\begin{funcdesc}{load_compiled}{name\, pathname\, \optional{file}}
Load and initialize a module implemented as a byte-compiled code file
and return its module object.  If the module was already initialized,
it will be initialized {\em again}.  The \var{name} argument is used
to create or access a module object.  The \var{pathname} argument
points to the byte-compiled code file.  The optional \var{file}
argument is the byte-compiled code file, open for reading in binary
mode, from the beginning --- if not given, the function opens
\var{pathname}.  It must currently be a real file object, not a
user-defined class emulating a file.
\end{funcdesc}

\begin{funcdesc}{load_dynamic}{name\, pathname\, \optional{file}}
Load and initialize a module implemented as a dynamically loadable
shared library and return its module object.  If the module was
already initialized, it will be initialized {\em again}.  Some modules
don't like that and may raise an exception.  The \var{pathname}
argument must point to the shared library.  The \var{name} argument is
used to construct the name of the initialization function: an external
C function called \code{init\var{name}()} in the shared library is
called.  The optional \var{file} argment is ignored.  (Note: using
shared libraries is highly system dependent, and not all systems
support it.)
\end{funcdesc}

\begin{funcdesc}{load_source}{name\, pathname\, \optional{file}}
Load and initialize a module implemented as a Python source file and
return its module object.  If the module was already initialized, it
will be initialized {\em again}.  The \var{name} argument is used to
create or access a module object.  The \var{pathname} argument points
to the source file.  The optional \var{file} argument is the source
file, open for reading as text, from the beginning --- if not given,
the function opens \var{pathname}.  It must currently be a real file
object, not a user-defined class emulating a file.  Note that if a
properly matching byte-compiled file (with suffix \code{.pyc}) exists,
it will be used instead of parsing the given source file.
\end{funcdesc}

\begin{funcdesc}{new_module}{name}
Return a new empty module object called \var{name}.  This object is
{\em not} inserted in \code{sys.modules}.
\end{funcdesc}

The following constants with integer values, defined in the module,
are used to indicate the search result of \code{imp.find_module}.

\begin{datadesc}{SEARCH_ERROR}
The module was not found.
\end{datadesc}

\begin{datadesc}{PY_SOURCE}
The module was found as a source file.
\end{datadesc}

\begin{datadesc}{PY_COMPILED}
The module was found as a compiled code object file.
\end{datadesc}

\begin{datadesc}{C_EXTENSION}
The module was found as dynamically loadable shared library.
\end{datadesc}

\subsection{Examples}
The following function emulates the default import statement:

\begin{verbatim}
import imp
import sys

def __import__(name, globals=None, locals=None, fromlist=None):
    # Fast path: see if the module has already been imported.
    if sys.modules.has_key(name):
        return sys.modules[name]

    # If any of the following calls raises an exception,
    # there's a problem we can't handle -- let the caller handle it.

    # See if it's a built-in module.
    m = imp.init_builtin(name)
    if m:
        return m

    # See if it's a frozen module.
    m = imp.init_frozen(name)
    if m:
        return m

    # Search the default path (i.e. sys.path).
    fp, pathname, (suffix, mode, type) = imp.find_module(name)

    # See what we got.
    try:
        if type == imp.C_EXTENSION:
            return imp.load_dynamic(name, pathname)
        if type == imp.PY_SOURCE:
            return imp.load_source(name, pathname, fp)
        if type == imp.PY_COMPILED:
            return imp.load_compiled(name, pathname, fp)

        # Shouldn't get here at all.
        raise ImportError, '%s: unknown module type (%d)' % (name, type)
    finally:
        # Since we may exit via an exception, close fp explicitly.
        fp.close()
\end{verbatim}
