% Format this file with latex.

\documentstyle[11pt,myformat]{report}

\title{\bf
	Python Reference Manual \\
	{\em Incomplete Draft}
}
	
\author{
	Guido van Rossum \\
	Dept. CST, CWI, Kruislaan 413 \\
	1098 SJ Amsterdam, The Netherlands \\
	E-mail: {\tt guido@cwi.nl}
}

\begin{document}

\pagenumbering{roman}

\maketitle

\begin{abstract}

\noindent
Python is a simple, yet powerful, interpreted programming language
that bridges the gap between C and shell programming, and is thus
ideally suited for ``throw-away programming'' and rapid prototyping.
Its syntax is put together from constructs borrowed from a variety of
other languages; most prominent are influences from ABC, C, Modula-3
and Icon.

The Python interpreter is easily extended with new functions and data
types implemented in C.  Python is also suitable as an extension
language for highly customizable C applications such as editors or
window managers.

Python is available for various operating systems, amongst which
several flavors of {\UNIX}, Amoeba, the Apple Macintosh O.S.,
and MS-DOS.

This reference manual describes the syntax and ``core semantics'' of
the language.  It is terse, but attempts to be exact and complete.
The semantics of non-essential built-in object types and of the
built-in functions and modules are described in the {\em Python
Library Reference}.  For an informal introduction to the language, see
the {\em Python Tutorial}.

\end{abstract}

\pagebreak

{
\parskip = 0mm
\tableofcontents
}

\pagebreak

\pagenumbering{arabic}

\chapter{Introduction}

This reference manual describes the Python programming language.
It is not intended as a tutorial.

While I am trying to be as precise as possible, I chose to use English
rather than formal specifications for everything except syntax and
lexical analysis.  This should make the document better understandable
to the average reader, but will leave room for ambiguities.
Consequently, if you were coming from Mars and tried to re-implement
Python from this document alone, you might have to guess things and in
fact you would be implementing quite a different language.
On the other hand, if you are using
Python and wonder what the precise rules about a particular area of
the language are, you should definitely be able to find it here.

It is dangerous to add too many implementation details to a language
reference document -- the implementation may change, and other
implementations of the same language may work differently.  On the
other hand, there is currently only one Python implementation, and
its particular quirks are sometimes worth being mentioned, especially
where the implementation imposes additional limitations.

Every Python implementation comes with a number of built-in and
standard modules.  These are not documented here, but in the separate
{\em Python Library Reference} document.  A few built-in modules are
mentioned when they interact in a significant way with the language
definition.

\section{Warning}

This version of the manual is incomplete.  Sections that still need to
be written or need considerable work are marked with ``XXX''.

\section{Notation}

The descriptions of lexical analysis and syntax use a modified BNF
grammar notation.  This uses the following style of definition:

\begin{verbatim}
name:           lc_letter (lc_letter | "_")*
lc_letter:      "a"..."z"
\end{verbatim}

The first line says that a \verb\name\ is an \verb\lc_letter\ followed by
a sequence of zero or more \verb\lc_letter\s and underscores.  An
\verb\lc_letter\ in turn is any of the single characters `a' through `z'.
(This rule is actually adhered to for the names defined in syntax and
grammar rules in this document.)

Each rule begins with a name (which is the name defined by the rule)
and a colon.  A vertical bar
(\verb\|\) is used to separate alternatives; it is the least binding
operator in this notation.  A star (\verb\*\) means zero or more
repetitions of the preceding item; likewise, a plus (\verb\+\) means
one or more repetitions, and a question mark (\verb\?\) zero or one
(in other words, the preceding item is optional).  These three
operators bind as tightly as possible; parentheses are used for
grouping.  Literal strings are enclosed in double quotes.  White space
is only meaningful to separate tokens.  Rules are normally contained
on a single line; rules with many alternatives may be formatted
alternatively with each line after the first beginning with a
vertical bar.

In lexical definitions (as the example above), two more conventions
are used: Two literal characters separated by three dots mean a choice
of any single character in the given (inclusive) range of ASCII
characters.  A phrase between angular brackets (\verb\<...>\) gives an
informal description of the symbol defined; e.g., this could be used
to describe the notion of `control character' if needed.

Even though the notation used is almost the same, there is a big
difference between the meaning of lexical and syntactic definitions:
a lexical definition operates on the individual characters of the
input source, while a syntax definition operates on the stream of
tokens generated by the lexical analysis.

\chapter{Lexical analysis}

A Python program is read by a {\em parser}.  Input to the parser is a
stream of {\em tokens}, generated by the {\em lexical analyzer}.  This
chapter describes how the lexical analyzer breaks a file into tokens.

\section{Line structure}

A Python program is divided in a number of logical lines.  The end of
a logical line is represented by the token NEWLINE.  Statements cannot
cross logical line boundaries except where NEWLINE is allowed by the
syntax (e.g., between statements in compound statements).

\subsection{Comments}

A comment starts with a hash character (\verb\#\) that is not part of
a string literal, and ends at the end of the physical line.  A comment
always signifies the end of the logical line.  Comments are ignored by
the syntax.

\subsection{Line joining}

Two or more physical lines may be joined into logical lines using
backslash characters (\verb/\/), as follows: when a physical line ends
in a backslash that is not part of a string literal or comment, it is
joined with the following forming a single logical line, deleting the
backslash and the following end-of-line character.  For example:
%
\begin{verbatim}
moth_names = ['Januari', 'Februari', 'Maart',     \
              'April',   'Mei',      'Juni',      \
              'Juli',    'Augustus', 'September', \
              'Oktober', 'November', 'December']
\end{verbatim}

\subsection{Blank lines}

A logical line that contains only spaces, tabs, and possibly a
comment, is ignored (i.e., no NEWLINE token is generated), except that
during interactive input of statements, an entirely blank logical line
terminates a multi-line statement.

\subsection{Indentation}

Leading whitespace (spaces and tabs) at the beginning of a logical
line is used to compute the indentation level of the line, which in
turn is used to determine the grouping of statements.

First, tabs are replaced (from left to right) by one to eight spaces
such that the total number of characters up to there is a multiple of
eight (this is intended to be the same rule as used by UNIX).  The
total number of spaces preceding the first non-blank character then
determines the line's indentation.  Indentation cannot be split over
multiple physical lines using backslashes.

The indentation levels of consecutive lines are used to generate
INDENT and DEDENT tokens, using a stack, as follows.

Before the first line of the file is read, a single zero is pushed on
the stack; this will never be popped off again.  The numbers pushed on
the stack will always be strictly increasing from bottom to top.  At
the beginning of each logical line, the line's indentation level is
compared to the top of the stack.  If it is equal, nothing happens.
If it larger, it is pushed on the stack, and one INDENT token is
generated.  If it is smaller, it {\em must} be one of the numbers
occurring on the stack; all numbers on the stack that are larger are
popped off, and for each number popped off a DEDENT token is
generated.  At the end of the file, a DEDENT token is generated for
each number remaining on the stack that is larger than zero.

Here is an example of a correctly (though confusingly) indented piece
of Python code:

\begin{verbatim}
def perm(l):
        # Compute the list of all permutations of l

    if len(l) <= 1:
                  return [l]
    r = []
    for i in range(len(l)):
             s = l[:i] + l[i+1:]
             p = perm(s)
             for x in p:
              r.append(l[i:i+1] + x)
    return r
\end{verbatim}

The following example shows various indentation errors:

\begin{verbatim}
    def perm(l):                        # error: first line indented
    for i in range(len(l)):             # error: not indented
        s = l[:i] + l[i+1:]
            p = perm(l[:i] + l[i+1:])   # error: unexpected indent
            for x in p:
                    r.append(l[i:i+1] + x)
                return r                # error: inconsistent indent
\end{verbatim}

(Actually, the first three errors are detected by the parser; only the
last error is found by the lexical analyzer -- the indentation of
\verb\return r\ does not match a level popped off the stack.)

\section{Other tokens}

Besides NEWLINE, INDENT and DEDENT, the following categories of tokens
exist: identifiers, keywords, literals, operators, and delimiters.
Spaces and tabs are not tokens, but serve to delimit tokens.  Where
ambiguity exists, a token comprises the longest possible string that
forms a legal token, when read from left to right.

\section{Identifiers}

Identifiers are described by the following regular expressions:

\begin{verbatim}
identifier:     (letter|"_") (letter|digit|"_")*
letter:         lowercase | uppercase
lowercase:      "a"..."z"
uppercase:      "A"..."Z"
digit:          "0"..."9"
\end{verbatim}

Identifiers are unlimited in length.  Case is significant.

\subsection{Keywords}

The following identifiers are used as reserved words, or {\em
keywords} of the language, and cannot be used as ordinary
identifiers.  They must be spelled exactly as written here:

\begin{verbatim}
and        del        for        in         print
break      elif       from       is         raise
class      else       global     not        return
continue   except     if         or         try
def        finally    import     pass       while
\end{verbatim}

%	# This Python program sorts and formats the above table
%	import string
%	l = []
%	try:
%		while 1:
%			l = l + string.split(raw_input())
%	except EOFError:
%		pass
%	l.sort()
%	for i in range((len(l)+4)/5):
%		for j in range(i, len(l), 5):
%			print string.ljust(l[j], 10),
%		print

\section{Literals}

\subsection{String literals}

String literals are described by the following regular expressions:

\begin{verbatim}
stringliteral:  "'" stringitem* "'"
stringitem:     stringchar | escapeseq
stringchar:     <any ASCII character except newline or "\" or "'">
escapeseq:      "'" <any ASCII character except newline>
\end{verbatim}

String literals cannot span physical line boundaries.  Escape
sequences in strings are actually interpreted according to rules
simular to those used by Standard C.  The recognized escape sequences
are:

\begin{center}
\begin{tabular}{|l|l|}
\hline
\verb/\\/	& Backslash (\verb/\/) \\
\verb/\'/	& Single quote (\verb/'/) \\
\verb/\a/	& ASCII Bell (BEL) \\
\verb/\b/	& ASCII Backspace (BS) \\
%\verb/\E/	& ASCII Escape (ESC) \\
\verb/\f/	& ASCII Formfeed (FF) \\
\verb/\n/	& ASCII Linefeed (LF) \\
\verb/\r/	& ASCII Carriage Return (CR) \\
\verb/\t/	& ASCII Horizontal Tab (TAB) \\
\verb/\v/	& ASCII Vertical Tab (VT) \\
\verb/\/{\em ooo}	& ASCII character with octal value {\em ooo} \\
\verb/\x/{em xx...}	& ASCII character with hex value {\em xx...} \\
\hline
\end{tabular}
\end{center}

In strict compatibility with in Standard C, up to three octal digits are
accepted, but an unlimited number of hex digits is taken to be part of
the hex escape (and then the lower 8 bits of the resulting hex number
are used in all current implementations...).

All unrecognized escape sequences are left in the string unchanged,
i.e., {\em the backslash is left in the string.}  (This rule is
useful when debugging: if an escape sequence is mistyped, the
resulting output is more easily recognized as broken.  It also helps a
great deal for string literals used as regular expressions or
otherwise passed to other modules that do their own escape handling --
but you may end up quadrupling backslashes that must appear literally.)

\subsection{Numeric literals}

There are three types of numeric literals: plain integers, long
integers, and floating point numbers.

Integers and long integers are described by the following regular expressions:

\begin{verbatim}
longinteger:    integer ("l"|"L")
integer:        decimalinteger | octinteger | hexinteger
decimalinteger: nonzerodigit digit* | "0"
octinteger:     "0" octdigit+
hexinteger:     "0" ("x"|"X") hexdigit+

nonzerodigit:   "1"..."9"
octdigit:       "0"..."7"
hexdigit:        digit|"a"..."f"|"A"..."F"
\end{verbatim}

Although both lower case `l'and upper case `L' are allowed as suffix
for long integers, it is strongly recommended to always use `L', since
the letter `l' looks too much like the digit `1'.

Plain integer decimal literals must be at most $2^{31} - 1$ (i.e., the
largest positive integer, assuming 32-bit arithmetic); octal and
hexadecimal literals may be as large as $2^{32} - 1$.  There is no limit
for long integer literals.

Some examples of plain and long integer literals:

\begin{verbatim}
7     2147483647                        0177    0x80000000
3L    79228162514264337593543950336L    0377L   0100000000L
\end{verbatim}

Floating point numbers are described by the following regular expressions:

\begin{verbatim}
floatnumber:    pointfloat | exponentfloat
pointfloat:     [intpart] fraction | intpart "."
exponentfloat:  (intpart | pointfloat) exponent
intpart:        digit+
fraction:       "." digit+
exponent:       ("e"|"E") ["+"|"-"] digit+
\end{verbatim}

The allowed range of floating point literals is
implementation-dependent.

Some examples of floating point literals:

\begin{verbatim}
3.14    10.    .001    1e100    3.14e-10
\end{verbatim}

Note that numeric literals do not include a sign; a phrase like
\verb\-1\ is actually an expression composed of the operator
\verb\-\ and the literal \verb\1\.

\section{Operators}

The following tokens are operators:

\begin{verbatim}
+       -       *       /       %
<<      >>      &       |       ^       ~
<       ==      >       <=      <>      !=      >=
\end{verbatim}

The comparison operators \verb\<>\ and \verb\!=\ are alternate
spellings of the same operator.

\section{Delimiters}

The following tokens serve as delimiters or otherwise have a special
meaning:

\begin{verbatim}
(       )       [       ]       {       }
;       ,       :       .       `       =
\end{verbatim}

The following printing ASCII characters are not used in Python (except
in string literals and in comments).  Their occurrence is an
unconditional error:

\begin{verbatim}
!       @       $       "       ?
\end{verbatim}

They may be used by future versions of the language though!

\chapter{Execution model}

\section{Objects, values and types}

I won't try to define rigorously here what an object is, but I'll give
some properties of objects that are important to know about.

Every object has an identity, a type and a value.  An object's {\em
identity} never changes once it has been created; think of it as the
object's (permanent) address.  An object's {\em type} determines the
operations that an object supports (e.g., does it have a length?)  and
also defines the ``meaning'' of the object's value.  The type also
never changes.  The {\em value} of some objects can change; whether
this is possible is a property of its type.

Objects are never explicitly destroyed; however, when they become
unreachable they may be garbage-collected.  An implementation is
allowed to delay garbage collection or omit it altogether -- it is a
matter of implementation quality how garbage collection is
implemented, as long as no objects are collected that are still
reachable.  (Implementation note: the current implementation uses a
reference-counting scheme which collects most objects as soon as they
become unreachable, but never collects garbage containing circular
references.)

Note that the use of the implementation's tracing or debugging
facilities may keep objects alive that would normally be collectable.

(Some objects contain references to ``external'' resources such as
open files.  It is understood that these resources are freed when the
object is garbage-collected, but since garbage collection is not
guaranteed, such objects also provide an explicit way to release the
external resource (e.g., a \verb\close\ method).  Programs are strongly
recommended to use this.)

Some objects contain references to other objects.  These references
are part of the object's value; in most cases, when such a
``container'' object is compared to another (of the same type), the
comparison applies to the {\em values} of the referenced objects (not
their identities).

Types affect almost all aspects of objects.
Even object identity is affected in some sense: for immutable
types, operations that compute new values may actually return a
reference to any existing object with the same type and value, while
for mutable objects this is not allowed.  E.g., after

\begin{verbatim}
a = 1; b = 1; c = []; d = []
\end{verbatim}

\verb\a\ and \verb\b\ may or may not refer to the same object, but
\verb\c\ and \verb\d\ are guaranteed to refer to two different, unique,
newly created lists.

\section{Execution frames, name spaces, and scopes}

XXX code blocks, scopes, name spaces, name binding, exceptions

\chapter{The standard type hierarchy}

The following types are built into Python.  Extension modules
written in C can define additional types.  Future versions of Python
may also add types to the type hierarchy (e.g., rational or complex
numbers, lists of efficiently stored integers, etc.).

\begin{description}

\item[None]
This type has a single value.  There is a single object with this value.
This object is accessed through the built-in name \verb\None\.
It is returned from functions that don't explicitly return an object.

\item[Numbers]
These are created by numeric literals and returned as results
by arithmetic operators and arithmetic built-in functions.
Numeric objects are immutable; once created their value never changes.
Python numbers are of course strongly related to mathematical numbers,
but subject to the limitations of numerical representation in computers.

Python distinguishes between integers and floating point numbers:

\begin{description}
\item[Integers]
These represent elements from the mathematical set of whole numbers.

There are two types of integers:

\begin{description}

\item[Plain integers]
These represent numbers in the range $-2^{31}$ through $2^{31}-1$.
(The range may be larger on machines with a larger natural word
size, but not smaller.)
When the result of an operation falls outside this range, the
exception \verb\OverflowError\ is raised.
For the purpose of shift and mask operations, integers are assumed to
have a binary, 2's complement notation using 32 or more bits, and
hiding no bits from the user (i.e., all $2^{32}$ different bit
patterns correspond to different values).

\item[Long integers]
These represent numbers in an unlimited range, subject to avaiable
(virtual) memory only.  For the purpose of shift and mask operations,
a binary representation is assumed, and negative numbers are
represented in a variant of 2's complement which gives the illusion of
an infinite string of sign bits extending to the left.

\end{description} % Integers

The rules for integer representation are intended to give the most
meaningful interpretation of shift and mask operations involving
negative integers and the least surprises when switching between the
plain and long integer domains.  For any operation except left shift,
if it yields a result in the plain integer domain without causing
overflow, it will yield the same result in the long integer domain or
when using mixed operands.

\item[Floating point numbers]
These represent machine-level double precision floating point numbers.  
You are at the mercy of the underlying machine architecture and
C implementation for the accepted range and handling of overflow.

\end{description} % Numbers

\item[Sequences]
These represent finite ordered sets indexed by natural numbers.
The built-in function \verb\len()\ returns the number of elements
of a sequence.  When this number is $n$, the index set contains
the numbers $0, 1, \ldots, n-1$.  Element \verb\i\ of sequence
\verb\a\ is selected by \verb\a[i]\.

Sequences also support slicing: \verb\a[i:j]\ selects all elements
with index $k$ such that $i < k < j$.  When used as an expression,
a slice is a sequence of the same type -- this implies that the
index set is renumbered so that it starts at 0 again.

Sequences are distinguished according to their mutability:

\begin{description}
%
\item[Immutable sequences]
An object of an immutable sequence type cannot change once it is
created.  (If the object contains references to other objects,
these other objects may be mutable and may be changed; however
the collection of objects directly referenced by an immutable object
cannot change.)

The following types are immutable sequences:

\begin{description}

\item[Strings]
The elements of a string are characters.  There is no separate
character type; a character is represented by a string of one element.
Characters represent (at least) 8-bit bytes.  The built-in
functions \verb\chr()\ and \verb\ord()\ convert between characters
and nonnegative integers representing the byte values.
Bytes with the values 0-127 represent the corresponding ASCII values.

(On systems whose native character set is not ASCII, strings may use
EBCDIC in their internal representation, provided the functions
\verb\chr()\ and \verb\ord()\ implement a mapping between ASCII and
EBCDIC, and string comparisons preserve the ASCII order.
Or perhaps someone can propose a better rule?)

\item[Tuples]
The elements of a tuple are arbitrary Python objects.
Tuples of two or more elements are formed by comma-separated lists
of expressions.  A tuple of one element can be formed by affixing
a comma to an expression (an expression by itself of course does
not create a tuple).  An empty tuple can be formed by enclosing
`nothing' in parentheses.

\end{description} % Immutable sequences

\item[Mutable sequences]
Mutable sequences can be changed after they are created.
The subscript and slice notations can be used as the target
of assignment and \verb\del\ (delete) statements.

There is currently a single mutable sequence type:

\begin{description}

\item[Lists]
The elements of a list are arbitrary Python objects.
Lists are formed by placing a comma-separated list of expressions
in square brackets.  (Note that there are no special cases for lists
of length 0 or 1.)

\end{description} % Mutable sequences

\end{description} % Sequences

\item[Mapping types]
These represent finite sets of objects indexed by arbitrary index sets.
The subscript notation \verb\a[k]\ selects the element indexed
by \verb\k\ from the mapping \verb\a\; this can be used in
expressions and as the target of assignments or \verb\del\ statements.
The built-in function \verb\len()\ returns the number of elements
in a mapping.

There is currently a single mapping type:

\begin{description}

\item[Dictionaries]
These represent finite sets of objects indexed by strings.
Dictionaries are created by the \verb\{...}\ notation (see section
\ref{dict}).  (Implementation note: the strings used for indexing must
not contain null bytes.)

\end{description} % Mapping types

\item[Callable types]
These are the types to which the function call operation can be applied:

\begin{description}
\item[User-defined functions]
XXX
\item[Built-in functions]
XXX
\item[User-defined methods]
XXX
\item[Built-in methods]
XXX
\item[User-defined classes]
XXX
\end{description}

\item[Modules]
XXX

\item[Class instances]
XXX

\item[Files]
XXX

\item[Internal types]
A few types used internally by the interpreter are exposed to the user.
Their definition may change with future versions of the interpreter,
but they are mentioned here for completeness.

\begin{description}
\item[Code objects]
XXX
\item[Traceback objects]
XXX
\end{description} % Internal types

\end{description} % Types

\chapter{Expressions and conditions}

From now on, extended BNF notation will be used to describe syntax,
not lexical analysis.

This chapter explains the meaning of the elements of expressions and
conditions.  Conditions are a superset of expressions, and a condition
may be used wherever an expression is required by enclosing it in
parentheses.  The only places where expressions are used in the syntax
instead of conditions is in expression statements and on the
right-hand side of assignments; this catches some nasty bugs like
accedentally writing \verb\x == 1\ instead of \verb\x = 1\.

The comma has several roles in Python's syntax.  It is usually an
operator with a lower precedence than all others, but occasionally
serves other purposes as well; e.g., it separates function arguments,
is used in list and dictionary constructors, and has special semantics
in \verb\print\ statements.

When (one alternative of) a syntax rule has the form

\begin{verbatim}
name:           othername
\end{verbatim}

and no semantics are given, the semantics of this form of \verb\name\
are the same as for \verb\othername\.

\section{Arithmetic conversions}

When a description of an arithmetic operator below uses the phrase
``the numeric arguments are converted to a common type'',
this both means that if either argument is not a number, a
\verb\TypeError\ exception is raised, and that otherwise
the following conversions are applied:

\begin{itemize}
\item	first, if either argument is a floating point number,
	the other is converted to floating point;
\item	else, if either argument is a long integer,
	the other is converted to long integer;
\item	otherwise, both must be plain integers and no conversion
	is necessary.
\end{itemize}

\section{Atoms}

Atoms are the most basic elements of expressions.  Forms enclosed in
reverse quotes or in parentheses, brackets or braces are also
categorized syntactically as atoms.  The syntax for atoms is:

\begin{verbatim}
atom:           identifier | literal | enclosure
enclosure:      parenth_form | list_display | dict_display | string_conversion
\end{verbatim}

\subsection{Identifiers (Names)}

An identifier occurring as an atom is a reference to a local, global
or built-in name binding.  If a name can be assigned to anywhere in a
code block, and is not mentioned in a \verb\global\ statement in that
code block, it refers to a local name throughout that code block.
Otherwise, it refers to a global name if one exists, else to a
built-in name.

When the name is bound to an object, evaluation of the atom yields
that object.  When a name is not bound, an attempt to evaluate it
raises a \verb\NameError\ exception.

\subsection{Literals}

Python knows string and numeric literals:

\begin{verbatim}
literal:        stringliteral | integer | longinteger | floatnumber
\end{verbatim}

Evaluation of a literal yields an object of the given type
(string, integer, long integer, floating point number)
with the given value.
The value may be approximated in the case of floating point literals.

All literals correspond to immutable data types, and hence the
object's identity is less important than its value.  Multiple
evaluations of literals with the same value (either the same
occurrence in the program text or a different occurrence) may obtain
the same object or a different object with the same value.

(In the original implementation, all literals in the same code block
with the same type and value yield the same object.)

\subsection{Parenthesized forms}

A parenthesized form is an optional condition list enclosed in
parentheses:

\begin{verbatim}
parenth_form:      "(" [condition_list] ")"
\end{verbatim}

A parenthesized condition list yields whatever that condition list
yields.

An empty pair of parentheses yields an empty tuple object.  Since
tuples are immutable, the rules for literals apply here.

(Note that tuples are not formed by the parentheses, but rather by use
of the comma operator.  The exception is the empty tuple, for which
parentheses {\em are} required -- allowing unparenthesized ``nothing''
in expressions would causes ambiguities and allow common typos to
pass uncaught.)

\subsection{List displays}

A list display is a possibly empty series of conditions enclosed in
square brackets:

\begin{verbatim}
list_display:   "[" [condition_list] "]"
\end{verbatim}

A list display yields a new list object.

If it has no condition list, the list object has no items.
Otherwise, the elements of the condition list are evaluated
from left to right and inserted in the list object in that order.

\subsection{Dictionary displays} \label{dict}

A dictionary display is a possibly empty series of key/datum pairs
enclosed in curly braces:

\begin{verbatim}
dict_display:   "{" [key_datum_list] "}"
key_datum_list: [key_datum ("," key_datum)* [","]
key_datum:      condition ":" condition
\end{verbatim}

A dictionary display yields a new dictionary object.

The key/datum pairs are evaluated from left to right to define the
entries of the dictionary: each key object is used as a key into the
dictionary to store the corresponding datum.

Keys must be strings, otherwise a \verb\TypeError\ exception is raised.
Clashes between duplicate keys are not detected; the last datum
(textually rightmost in the display) stored for a given key value
prevails.

\subsection{String conversions}

A string conversion is a condition list enclosed in reverse (or
backward) quotes:

\begin{verbatim}
string_conversion: "`" condition_list "`"
\end{verbatim}

A string conversion evaluates the contained condition list and converts the
resulting object into a string according to rules specific to its type.

If the object is a string, a number, \verb\None\, or a tuple, list or
dictionary containing only objects whose type is one of these, the
resulting string is a valid Python expression which can be passed to
the built-in function \verb\eval()\ to yield an expression with the
same value (or an approximation, if floating point numbers are
involved).

(In particular, converting a string adds quotes around it and converts
``funny'' characters to escape sequences that are safe to print.)

It is illegal to attempt to convert recursive objects (e.g., lists or
dictionaries that contain a reference to themselves, directly or
indirectly.)

\section{Primaries}

Primaries represent the most tightly bound operations of the language.
Their syntax is:

\begin{verbatim}
primary:        atom | attributeref | subscription | slicing | call
\end{verbatim}

\subsection{Attribute references}

An attribute reference is a primary followed by a period and a name:

\begin{verbatim}
attributeref:   primary "." identifier
\end{verbatim}

The primary must evaluate to an object of a type that supports
attribute references, e.g., a module or a list.  This object is then
asked to produce the attribute whose name is the identifier.  If this
attribute is not available, the exception \verb\AttributeError\ is
raised.  Otherwise, the type and value of the object produced is
determined by the object.  Multiple evaluations of the same attribute
reference may yield different objects.

\subsection{Subscriptions}

A subscription selects an item of a sequence or mapping object:

\begin{verbatim}
subscription:   primary "[" condition "]"
\end{verbatim}

The primary must evaluate to an object of a sequence or mapping type.

If it is a mapping, the condition must evaluate to an object whose
value is one of the keys of the mapping, and the subscription selects
the value in the mapping that corresponds to that key.

If it is a sequence, the condition must evaluate to a plain integer.
If this value is negative, the length of the sequence is added to it
(so that, e.g., \verb\x[-1]\ selects the last item of \verb\x\.)
The resulting value must be a nonnegative integer smaller than the
number of items in the sequence, and the subscription selects the item
whose index is that value (counting from zero).

A string's items are characters.  A character is not a separate data
type but a string of exactly one character.

\subsection{Slicings}

A slicing selects a range of items in a sequence object:

\begin{verbatim}
slicing:        primary "[" [condition] ":" [condition] "]"
\end{verbatim}

The primary must evaluate to a sequence object.  The lower and upper
bound expressions, if present, must evaluate to plain integers;
defaults are zero and the sequence's length, respectively.  If either
bound is negative, the sequence's length is added to it.  The slicing
now selects all items with index $k$ such that $i <= k < j$ where $i$
and $j$ are the specified lower and upper bounds.  This may be an
empty sequence.  It is not an error if $i$ or $j$ lie outside the
range of valid indexes (such items don't exist so they aren't
selected).

\subsection{Calls}

A call calls a function with a possibly empty series of arguments:

\begin{verbatim}
call:           primary "(" [condition_list] ")"
\end{verbatim}

The primary must evaluate to a callable object (user-defined
functions, built-in functions, methods of built-in objects, class
objects, and methods of class instances are callable).  If it is a
class, the argument list must be empty.

XXX explain what happens on function call

\section{Factors}

Factors represent the unary numeric operators.
Their syntax is:

\begin{verbatim}
factor:         primary | "-" factor | "+" factor | "~" factor
\end{verbatim}

The unary \verb\"-"\ operator yields the negative of its
numeric argument.

The unary \verb\"+"\ operator yields its numeric argument unchanged.

The unary \verb\"~"\ operator yields the bit-wise negation of its
plain or long integer argument.  The bit-wise negation negation of
\verb\x\ is defined as \verb\-(x+1)\.

In all three cases, if the argument does not have the proper type,
a \verb\TypeError\ exception is raised.

\section{Terms}

Terms represent the most tightly binding binary operators:
%
\begin{verbatim}
term:           factor | term "*" factor | term "/" factor | term "%" factor
\end{verbatim}
%
The \verb\"*"\ (multiplication) operator yields the product of its
arguments.  The arguments must either both be numbers, or one argument
must be a plain integer and the other must be a sequence.  In the
former case, the numbers are converted to a common type and then
multiplied together.  In the latter case, sequence repetition is
performed; a negative repetition factor yields an empty sequence.

The \verb\"/"\ (division) operator yields the quotient of its
arguments.  The numeric arguments are first converted to a common
type.  Plain or long integer division yields an integer of the same
type; the result is that of mathematical division with the `floor'
function applied to the result.  Division by zero raises the
\verb\ZeroDivisionError\ exception.

The \verb\"%"\ (modulo) operator yields the remainder from the
division of the first argument by the second.  The numeric arguments
are first converted to a common type.  A zero right argument raises the
\verb\ZeroDivisionError\ exception.  The arguments may be floating point
numbers, e.g., \verb\3.14 % 0.7\ equals \verb\0.34\.  The modulo operator
always yields a result with the same sign as its second operand (or
zero); the absolute value of the result is strictly smaller than the
second operand.

The integer division and modulo operators are connected by the
following identity: \verb\x == (x/y)*y + (x%y)\.
Integer division and modulo are also connected with the built-in
function \verb\divmod()\: \verb\divmod(x, y) == (x/y, x%y)\.
These identities don't hold for floating point numbers; there a
similar identity holds where \verb\x/y\ is replaced by
\verb\floor(x/y)\).

\section{Arithmetic expressions}

\begin{verbatim}
arith_expr:     term | arith_expr "+" term | arith_expr "-" term
\end{verbatim}

The \verb|"+"| operator yields the sum of its arguments.  The
arguments must either both be numbers, or both sequences of the same
type.  In the former case, the numbers are converted to a common type
and then added together.  In the latter case, the sequences are
concatenated.

The \verb|"-"| operator yields the difference of its arguments.
The numeric arguments are first converted to a common type.

\section{Shift expressions}

\begin{verbatim}
shift_expr:     arith_expr | shift_expr ( "<<" | ">>" ) arith_expr
\end{verbatim}

These operators accept plain or long integers as arguments.  The
arguments are converted to a common type.  They shift the first
argument to the left or right by the number of bits given by the
second argument.

A right shift by $n$ bits is defined as division by $2^n$.  A left
shift by $n$ bits is defined as multiplication with $2^n$ without
overflow check; for plain integers this drops bits if the result is
not less than $2^{31} - 1$ in absolute value.

Negative shift counts raise a \verb\ValueError\ exception.

\section{Bitwise AND expressions}

\begin{verbatim}
and_expr:       shift_expr | and_expr "&" shift_expr
\end{verbatim}

This operator yields the bitwise AND of its arguments, which must be
plain or long integers.  The arguments are converted to a common type.

\section{Bitwise XOR expressions}

\begin{verbatim}
xor_expr:       and_expr | xor_expr "^" and_expr
\end{verbatim}

This operator yields the bitwise exclusive OR of its arguments, which
must be plain or long integers.  The arguments are converted to a
common type.

\section{Bitwise OR expressions}

\begin{verbatim}
or_expr:       xor_expr | or_expr "|" xor_expr
\end{verbatim}

This operator yields the bitwise OR of its arguments, which must be
plain or long integers.  The arguments are converted to a common type.

\section{Comparisons}

\begin{verbatim}
comparison:     or_expr (comp_operator or_expr)*
comp_operator:  "<"|">"|"=="|">="|"<="|"<>"|"!="|"is" ["not"]|["not"] "in"
\end{verbatim}

Comparisons yield integer value: 1 for true, 0 for false.

Comparisons can be chained arbitrarily,
e.g., $x < y <= z$ is equivalent to
$x < y$ \verb\and\ $y <= z$, except that $y$ is evaluated only once
(but in both cases $z$ is not evaluated at all when $x < y$ is
found to be false).

Formally, $e_0 op_1 e_1 op_2 e_2 ...e_{n-1} op_n e_n$ is equivalent to
$e_0 op_1 e_1$ \verb\and\ $e_1 op_2 e_2$ \verb\and\ ... \verb\and\
$e_{n-1} op_n e_n$, except that each expression is evaluated at most once.

Note that $e_0 op_1 e_1 op_2 e_2$ does not imply any kind of comparison
between $e_0$ and $e_2$, e.g., $x < y > z$ is perfectly legal.

The forms \verb\<>\ and \verb\!=\ are equivalent; for consistency with
C, \verb\!=\ is preferred; where \verb\!=\ is mentioned below
\verb\<>\ is also implied.

The operators {\tt "<", ">", "==", ">=", "<="}, and {\tt "!="} compare
the values of two objects.  The objects needn't have the same type.
If both are numbers, they are coverted to a common type.  Otherwise,
objects of different types {\em always} compare unequal, and are
ordered consistently but arbitrarily.

(This unusual
definition of comparison is done to simplify the definition of
operations like sorting and the \verb\in\ and \verb\not in\ operators.)

Comparison of objects of the same type depends on the type:

\begin{itemize}

\item
Numbers are compared arithmetically.

\item
Strings are compared lexicographically using the numeric equivalents
(the result of the built-in function \verb\ord\) of their characters.

\item
Tuples and lists are compared lexicographically using comparison of
corresponding items.

\item
Mappings (dictionaries) are compared through lexicographic
comparison of their sorted (key, value) lists.%
\footnote{This is expensive since it requires sorting the keys first,
but about the only sensible definition.  It was tried to compare
dictionaries using the following rules, but this gave surprises in
cases like \verb|if d == {}: ...|.}

\item
Most other types compare unequal unless they are the same object;
the choice whether one object is considered smaller or larger than
another one is made arbitrarily but consistently within one
execution of a program.

\end{itemize}

The operators \verb\in\ and \verb\not in\ test for sequence
membership: if $y$ is a sequence, $x ~\verb\in\~ y$ is true if and
only if there exists an index $i$ such that $x = y[i]$.
$x ~\verb\not in\~ y$ yields the inverse truth value.  The exception
\verb\TypeError\ is raised when $y$ is not a sequence, or when $y$ is
a string and $x$ is not a string of length one.%
\footnote{The latter restriction is sometimes a nuisance.}

The operators \verb\is\ and \verb\is not\ compare object identity:
$x ~\verb\is\~ y$ is true if and only if $x$ and $y$ are the same
object.  $x ~\verb\is not\~ y$ yields the inverse truth value.

\section{Boolean operators}

\begin{verbatim}
condition:      or_test
or_test:        and_test | or_test "or" and_test
and_test:       not_test | and_test "and" not_test
not_test:       comparison | "not" not_test
\end{verbatim}

In the context of Boolean operators, and also when conditions are used
by control flow statements, the following values are interpreted as
false: \verb\None\, numeric zero of all types, empty sequences
(strings, tuples and lists), and empty mappings (dictionaries).  All
other values are interpreted as true.

The operator \verb\not\ yields 1 if its argument is false, 0 otherwise.

The condition $x ~\verb\and\~ y$ first evaluates $x$; if $x$ is false,
$x$ is returned; otherwise, $y$ is evaluated and returned.

The condition $x ~\verb\or\~ y$ first evaluates $x$; if $x$ is true,
$x$ is returned; otherwise, $y$ is evaluated and returned.

(Note that \verb\and\ and \verb\or\ do not restrict the value and type
they return to 0 and 1, but rather return the last evaluated argument.
This is sometimes useful, e.g., if \verb\s\ is a string, which should be
replaced by a default value if it is empty, \verb\s or 'foo'\
returns the desired value.  Because \verb\not\ has to invent a value
anyway, it does not bother to return a value of the same type as its
argument, so \verb\not 'foo'\ yields \verb\0\, not \verb\''\.)

\section{Expression lists and condition lists}

\begin{verbatim}
expr_list:      or_expr ("," or_expr)* [","]
cond_list:      condition ("," condition)* [","]
\end{verbatim}

The only difference between expression lists and condition lists is
the lowest priority of operators that can be used in them without
being enclosed in parentheses; condition lists allow all operators,
while expression lists don't allow comparisons and Boolean operators
(they do allow bitwise and shift operators though).

Expression lists are used in expression statements and assignments;
condition lists are used everywhere else.

An expression (condition) list containing at least one comma yields a
tuple.  The length of the tuple is the number of expressions
(conditions) in the list.  The expressions (conditions) are evaluated
from left to right.

The trailing comma is required only to create a single tuple (a.k.a. a
{\em singleton}); it is optional in all other cases.  A single
expression (condition) without a trailing comma doesn't create a
tuple, but rather yields the value of that expression (condition).

To create an empty tuple, use an empty pair of parentheses: \verb\()\.

\chapter{Simple statements}

Simple statements are comprised within a single logical line.
Several simple statements may occur on a single line separated
by semicolons.  The syntax for simple statements is:

\begin{verbatim}
stmt_list:      simple_stmt (";" simple_stmt)* [";"]
simple_stmt:    expression_stmt
              | assignment
              | pass_stmt
              | del_stmt
              | print_stmt
              | return_stmt
              | raise_stmt
              | break_stmt
              | continue_stmt
              | import_stmt
              | global_stmt
\end{verbatim}

\section{Expression statements}

\begin{verbatim}
expression_stmt: expression_list
\end{verbatim}

An expression statement evaluates the expression list (which may
be a single expression).
If the value is not \verb\None\, it is converted to a string
using the rules for string conversions, and the resulting string
is written to standard output on a line by itself.

(The exception for \verb\None\ is made so that procedure calls, which
are syntactically equivalent to expressions, do not cause any output.
A tuple with only \verb\None\ items is written normally.)

\section{Assignments}

\begin{verbatim}
assignment:     (target_list "=")+ expression_list
target_list:    target ("," target)* [","]
target:         identifier | "(" target_list ")" | "[" target_list "]"
              | attributeref | subscription | slicing
\end{verbatim}

(See the section on primaries for the syntax definition of the last
three symbols.)

An assignment evaluates the expression list (remember that this can
be a single expression or a comma-separated list,
the latter yielding a tuple)
and assigns the single resulting object to each of the target lists,
from left to right.

Assignment is defined recursively depending on the form of the target.
When a target is part of a mutable object (an attribute reference,
subscription or slicing), the mutable object must ultimately perform
the assignment and decide about its validity, and may raise an
exception if the assignment is unacceptable.  The rules observed by
various types and the exceptions raised are given with the definition
of the object types (some of which are defined in the library
reference).

Assignment of an object to a target list is recursively
defined as follows.

\begin{itemize}
\item
If the target list contains no commas (except in nested constructs):
the object is assigned to the single target contained in the list.

\item
If the target list contains commas (that are not in nested constructs):
the object must be a tuple with the same number of items
as the list contains targets, and the items are assigned, from left
to right, to the corresponding targets.

\end{itemize}

Assignment of an object to a (non-list)
target is recursively defined as follows.

\begin{itemize}

\item
If the target is an identifier (name):
\begin{itemize}
\item
If the name does not occur in a \verb\global\ statement in the current
code block: the object is bound to that name in the current local
name space.
\item
Otherwise: the object is bound to that name in the current global name
space.
\end{itemize}
A previous binding of the same name in the same name space is undone.

\item
If the target is a target list enclosed in parentheses:
the object is assigned to that target list.

\item
If the target is a target list enclosed in square brackets:
the object must be a list with the same number of items
as the target list contains targets,
and the list's items are assigned, from left to right,
to the corresponding targets.

\item
If the target is an attribute reference:
The primary expression in the reference is evaluated.
It should yield an object with assignable attributes;
if this is not the case, \verb\TypeError\ is raised.
That object is then asked to assign the assigned object
to the given attribute; if it cannot perform the assignment,
it raises an exception.

\item
If the target is a subscription: The primary expression in the
reference is evaluated.  It should yield either a mutable sequence
(list) object or a mapping (dictionary) object.  Next, the subscript
expression is evaluated.

If the primary is a sequence object, the subscript must yield a plain
integer.  If it is negative, the sequence's length is added to it.
The resulting value must be a nonnegative integer less than the
sequence's length, and the sequence is asked to assign the assigned
object to its item with that index.  If the index is out of range,
\verb\IndexError\ is raised (assignment to a subscripted sequence
cannot add new items to a list).

If the primary is a mapping object, the subscript must have a type
compatible with the mapping's key type, and the mapping is then asked
to to create a key/datum pair which maps the subscript to the assigned
object.  This can either replace an existing key/value pair with the
same key value, or insert a new key/value pair (if no key with the
same value existed).

\item
If the target is a slicing: The primary expression in the reference is
evaluated.  It should yield a mutable sequence (list) object.  The
assigned object should be a sequence object of the same type.  Next,
the lower and upper bound expressions are evaluated, insofar they are
present; defaults are zero and the sequence's length.  The bounds
should evaluate to (small) integers.  If either bound is negative, the
sequence's length is added to it.  The resulting bounds are clipped to
lie between zero and the sequence's length, inclusive.  Finally, the
sequence object is asked to replace the items indicated by the slice
with the items of the assigned sequence.  This may change the
sequence's length, if it allows it.

\end{itemize}
	
(In the original implementation, the syntax for targets is taken
to be the same as for expressions, and invalid syntax is rejected
during the code generation phase, causing less detailed error
messages.)

\section{The \verb\pass\ statement}

\begin{verbatim}
pass_stmt:      "pass"
\end{verbatim}

\verb\pass\ is a null operation -- when it is executed, nothing
happens.  It is useful as a placeholder when a statement is
required syntactically, but no code needs to be executed, for example:

\begin{verbatim}
def f(arg): pass    # a no-op function

class C: pass       # an empty class
\end{verbatim}

\section{The \verb\del\ statement}

\begin{verbatim}
del_stmt:       "del" target_list
\end{verbatim}

Deletion is recursively defined very similar to the way assignment is
defined. Rather that spelling it out in full details, here are some
hints.

Deletion of a target list recursively deletes each target,
from left to right.

Deletion of a name removes the binding of that name (which must exist)
from the local or global name space, depending on whether the name
occurs in a \verb\global\ statement in the same code block.

Deletion of attribute references, subscriptions and slicings
is passed to the primary object involved; deletion of a slicing
is in general equivalent to assignment of an empty slice of the
right type (but even this is determined by the sliced object).

\section{The \verb\print\ statement}

\begin{verbatim}
print_stmt:     "print" [ condition ("," condition)* [","] ]
\end{verbatim}

\verb\print\ evaluates each condition in turn and writes the resulting
object to standard output (see below).  If an object is not a string,
it is first converted to a string using the rules for string
conversions.  The (resulting or original) string is then written.  A
space is written before each object is (converted and) written, unless
the output system believes it is positioned at the beginning of a
line.  This is the case: (1) when no characters have yet been written
to standard output; or (2) when the last character written to standard
output is \verb/\n/; or (3) when the last write operation on standard
output was not a \verb\print\ statement.  (In some cases it may be
functional to write an empty string to standard output for this
reason.)

A \verb/"\n"/ character is written at the end, unless the \verb\print\
statement ends with a comma.  This is the only action if the statement
contains just the keyword \verb\print\.

Standard output is defined as the file object named \verb\stdout\
in the built-in module \verb\sys\.  If no such object exists,
or if it is not a writable file, a \verb\RuntimeError\ exception is raised.
(The original implementation attempts to write to the system's original
standard output instead, but this is not safe, and should be fixed.)

\section{The \verb\return\ statement}

\begin{verbatim}
return_stmt:    "return" [condition_list]
\end{verbatim}

\verb\return\ may only occur syntactically nested in a function
definition, not within a nested class definition.

If a condition list is present, it is evaluated, else \verb\None\
is substituted.

\verb\return\ leaves the current function call with the condition
list (or \verb\None\) as return value.

When \verb\return\ passes control out of a \verb\try\ statement
with a \verb\finally\ clause, that finally clause is executed
before really leaving the function.

\section{The \verb\raise\ statement}

\begin{verbatim}
raise_stmt:     "raise" condition ["," condition]
\end{verbatim}

\verb\raise\ evaluates its first condition, which must yield
a string object.  If there is a second condition, this is evaluated,
else \verb\None\ is substituted.

It then raises the exception identified by the first object,
with the second one (or \verb\None\) as its parameter.

\section{The \verb\break\ statement}

\begin{verbatim}
break_stmt:     "break"
\end{verbatim}

\verb\break\ may only occur syntactically nested in a \verb\for\
or \verb\while\ loop, not nested in a function or class definition.

It terminates the neares enclosing loop, skipping the optional
\verb\else\ clause if the loop has one.

If a \verb\for\ loop is terminated by \verb\break\, the loop control
target keeps its current value.

When \verb\break\ passes control out of a \verb\try\ statement
with a \verb\finally\ clause, that finally clause is executed
before really leaving the loop.

\section{The \verb\continue\ statement}

\begin{verbatim}
continue_stmt:  "continue"
\end{verbatim}

\verb\continue\ may only occur syntactically nested in a \verb\for\ or
\verb\while\ loop, not nested in a function or class definition, and
not nested in the \verb\try\ clause of a \verb\try\ statement with a
\verb\finally\ clause (it may occur nested in a \verb\except\ or
\verb\finally\ clause of a \verb\try\ statement though).

It continues with the next cycle of the nearest enclosing loop.

\section{The \verb\import\ statement}

\begin{verbatim}
import_stmt:    "import" identifier ("," identifier)*
              | "from" identifier "import" identifier ("," identifier)*
              | "from" identifier "import" "*"
\end{verbatim}

Import statements are executed in two steps: (1) find a module, and
initialize it if necessary; (2) define a name or names in the local
name space.  The first form (without \verb\from\) repeats these steps
for each identifier in the list.

The system maintains a table of modules that have been initialized,
indexed by module name.  (The current implementation makes this table
accessible as \verb\sys.modules\.)  When a module name is found in
this table, step (1) is finished.  If not, a search for a module
definition is started.  This first looks for a built-in module
definition, and if no built-in module if the given name is found, it
searches a user-specified list of directories for a file whose name is
the module name with extension \verb\".py"\.  (The current
implementation uses the list of strings \verb\sys.path\ as the search
path; it is initialized from the shell environment variable
\verb\$PYTHONPATH\, with an installation-dependent default.)

If a built-in module is found, its built-in initialization code is
executed and step (1) is finished.  If no matching file is found,
\ImportError\ is raised (and step (2) is never started).  If a file is
found, it is parsed.  If a syntax error occurs, HIRO

\section{The \verb\global\ statement}

\begin{verbatim}
global_stmt:    "global" identifier ("," identifier)*
\end{verbatim}

(XXX To be done.)

\chapter{Compound statements}

(XXX The semantic definitions of this chapter are still to be done.)

\begin{verbatim}
statement:      stmt_list NEWLINE | compound_stmt
compound_stmt:  if_stmt | while_stmt | for_stmt | try_stmt | funcdef | classdef
suite:          statement | NEWLINE INDENT statement+ DEDENT
\end{verbatim}

\section{The \verb\if\ statement}

\begin{verbatim}
if_stmt:        "if" condition ":" suite
               ("elif" condition ":" suite)*
               ["else" ":" suite]
\end{verbatim}

\section{The \verb\while\ statement}

\begin{verbatim}
while_stmt:     "while" condition ":" suite ["else" ":" suite]
\end{verbatim}

\section{The \verb\for\ statement}

\begin{verbatim}
for_stmt:       "for" target_list "in" condition_list ":" suite
               ["else" ":" suite]
\end{verbatim}

\section{The \verb\try\ statement}

\begin{verbatim}
try_stmt:       "try" ":" suite
               ("except" condition ["," condition] ":" suite)*
               ["finally" ":" suite]
\end{verbatim}

\section{Function definitions}

\begin{verbatim}
funcdef:        "def" identifier "(" [parameter_list] ")" ":" suite
parameter_list: parameter ("," parameter)*
parameter:      identifier | "(" parameter_list ")"
\end{verbatim}

\section{Class definitions}

\begin{verbatim}
classdef:       "class" identifier [inheritance] ":" suite
inheritance:    "(" expression ("," expression)* ")"
\end{verbatim}

XXX Syntax for scripts, modules
XXX Syntax for interactive input, eval, exec, input
XXX New definition of expressions (as conditions)

\end{document}
