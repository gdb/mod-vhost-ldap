% cfuncdesc should be called as an \begin{cfuncdesc} ... \end{cfuncdesc}
\newcommand{\cfuncline}[3]{\item[\code{#1 #2(\varvars{#3})}]\ttindex{#2}}
\newcommand{\cfuncdesc}[3]{\fulllineitems\cfuncline{#1}{#2}{#3}}
\let\endcfuncdesc\endfulllineitems

\newcommand{\NULL}{\code{NULL}}

\chapter{Extension Reference}

From the viewpoint of of C access to Python services, we have:

\begin{enumerate}
  \item "Very high level layer": two or three functions that let you exec or
    eval arbitrary Python code given as a string in a module whose name is
    given, passing C values in and getting C values out using
    mkvalue/getargs style format strings.  This does not require the user
    to declare any variables of type "PyObject *".  This should be enough
    to write a simple application that gets Python code from the user,
    execs it, and returns the output or errors.

  \item "Abstract objects layer": which is the subject of this proposal.
    It has many functions operating on objects, and lest you do many
    things from C that you can also write in Python, without going
    through the Python parser.

  \item "Concrete objects layer": This is the public type-dependent
    interface provided by the standard built-in types, such as floats,
    strings, and lists.  This interface exists and is currently
    documented by the collection of include files provides with the
    Python distributions.

  From the point of view of Python accessing services provided by C
  modules: 

  \item "Python module interface": this interface consist of the basic
    routines used to define modules and their members.  Most of the
    current extensions-writing guide deals with this interface.

  \item "Built-in object interface": this is the interface that a new
    built-in type must provide and the mechanisms and rules that a
    developer of a new built-in type must use and follow.
\end{enumerate}

  The Python C object interface provides four protocols: object,
  numeric, sequence, and mapping.  Each protocol consists of a
  collection of related operations.  If an operation that is not
  provided by a particular type is invoked, then a standard exception,
  NotImplementedError is raised with a operation name as an argument.
  In addition, for convenience this interface defines a set of
  constructors for building objects of built-in types.  This is needed
  so new objects can be returned from C functions that otherwise treat
  objects generically.

\section{Object Protocol}

     \begin{cfuncdesc}{int}{PyObject_Print}{PyObject *o, FILE *fp, int flags}
         Print an object \code{o}, on file \code{fp}.  Returns -1 on error
	 The flags argument is used to enable certain printing
	 options. The only option currently supported is \code{Py_Print_RAW}. 
     \end{cfuncdesc}

     \begin{cfuncdesc}{int}{PyObject_HasAttrString}{PyObject *o, char *attr_name}
         Returns 1 if o has the attribute attr_name, and 0 otherwise.
	 This is equivalent to the Python expression:
	 \code{hasattr(o,attr_name)}.
	 This function always succeeds.
     \end{cfuncdesc}

     \begin{cfuncdesc}{PyObject*}{PyObject_AttrString}{PyObject *o, char *attr_name}
	 Retrieve an attributed named attr_name form object o.
	 Returns the attribute value on success, or {\NULL} on failure.
	 This is the equivalent of the Python expression: \code{o.attr_name}.
     \end{cfuncdesc}


     \begin{cfuncdesc}{int}{PyObject_HasAttr}{PyObject *o, PyObject *attr_name}
         Returns 1 if o has the attribute attr_name, and 0 otherwise.
	 This is equivalent to the Python expression:
	 \code{hasattr(o,attr_name)}. 
	 This function always succeeds.
     \end{cfuncdesc}


     \begin{cfuncdesc}{PyObject*}{PyObject_GetAttr}{PyObject *o, PyObject *attr_name}
	 Retrieve an attributed named attr_name form object o.
	 Returns the attribute value on success, or {\NULL} on failure.
	 This is the equivalent of the Python expression: o.attr_name.
     \end{cfuncdesc}


     \begin{cfuncdesc}{int}{PyObject_SetAttrString}{PyObject *o, char *attr_name, PyObject *v}
	 Set the value of the attribute named \code{attr_name}, for object \code{o},
	 to the value \code{v}. Returns -1 on failure.  This is
	 the equivalent of the Python statement: \code{o.attr_name=v}.
     \end{cfuncdesc}


     \begin{cfuncdesc}{int}{PyObject_SetAttr}{PyObject *o, PyObject *attr_name, PyObject *v}
	 Set the value of the attribute named \code{attr_name}, for
	 object \code{o},
	 to the value \code{v}. Returns -1 on failure.  This is
	 the equivalent of the Python statement: \code{o.attr_name=v}.
     \end{cfuncdesc}


     \begin{cfuncdesc}{int}{PyObject_DelAttrString}{PyObject *o, char *attr_name}
	 Delete attribute named \code{attr_name}, for object \code{o}. Returns -1 on
	 failure.  This is the equivalent of the Python
	 statement: \code{del o.attr_name}.
     \end{cfuncdesc}


     \begin{cfuncdesc}{int}{PyObject_DelAttr}{PyObject *o, PyObject *attr_name}
	 Delete attribute named \code{attr_name}, for object \code{o}. Returns -1 on
	 failure.  This is the equivalent of the Python
	 statement: \code{del o.attr_name}.
     \end{cfuncdesc}


     \begin{cfuncdesc}{int}{PyObject_Cmp}{PyObject *o1, PyObject *o2, int *result}
	 Compare the values of \code{o1} and \code{o2} using a routine provided by
	 \code{o1}, if one exists, otherwise with a routine provided by \code{o2}.
	 The result of the comparison is returned in \code{result}.  Returns
	 -1 on failure.  This is the equivalent of the Python
	 statement: \code{result=cmp(o1,o2)}.
     \end{cfuncdesc}


     \begin{cfuncdesc}{int}{PyObject_Compare}{PyObject *o1, PyObject *o2}
	 Compare the values of \code{o1} and \code{o2} using a routine provided by
	 \code{o1}, if one exists, otherwise with a routine provided by \code{o2}.
	 Returns the result of the comparison on success.  On error,
	 the value returned is undefined. This is equivalent to the
	 Python expression: \code{cmp(o1,o2)}.
     \end{cfuncdesc}


     \begin{cfuncdesc}{PyObject*}{PyObject_Repr}{PyObject *o}
	 Compute the string representation of object, \code{o}.  Returns the
	 string representation on success, {\NULL} on failure.  This is
	 the equivalent of the Python expression: \code{repr(o)}.
	 Called by the \code{repr()} built-in function and by reverse quotes.
     \end{cfuncdesc}


     \begin{cfuncdesc}{PyObject*}{PyObject_Str}{PyObject *o}
	 Compute the string representation of object, \code{o}.  Returns the
	 string representation on success, {\NULL} on failure.  This is
	 the equivalent of the Python expression: \code{str(o)}.
	 Called by the \code{str()} built-in function and by the \code{print}
	 statement.
     \end{cfuncdesc}


     \begin{cfuncdesc}{int}{PyCallable_Check}{PyObject *o}
	 Determine if the object \code{o}, is callable.  Return 1 if the
	 object is callable and 0 otherwise.
	 This function always succeeds.
     \end{cfuncdesc}


     \begin{cfuncdesc}{PyObject*}{PyObject_CallObject}{PyObject *callable_object, PyObject *args}
	 Call a callable Python object \code{callable_object}, with
	 arguments given by the tuple \code{args}.  If no arguments are
	 needed, then args may be {\NULL}.  Returns the result of the
	 call on success, or {\NULL} on failure.  This is the equivalent
	 of the Python expression: \code{apply(o, args)}.
     \end{cfuncdesc}

     \begin{cfuncdesc}{PyObject*}{PyObject_CallFunction}{PyObject *callable_object, char *format, ...}
         Call a callable Python object \code{callable_object}, with a
         variable number of C arguments. The C arguments are described
         using a mkvalue-style format string. The format may be {\NULL},
         indicating that no arguments are provided.  Returns the
         result of the call on success, or {\NULL} on failure.  This is
         the equivalent of the Python expression: \code{apply(o,args)}.
     \end{cfuncdesc}


     \begin{cfuncdesc}{PyObject*}{PyObject_CallMethod}{PyObject *o, char *m, char *format, ...}
         Call the method named \code{m} of object \code{o} with a variable number of
         C arguments.  The C arguments are described by a mkvalue
         format string.  The format may be {\NULL}, indicating that no
         arguments are provided. Returns the result of the call on
         success, or {\NULL} on failure.  This is the equivalent of the
         Python expression: \code{o.method(args)}.
         Note that Special method names, such as "\code{__add__}",
         "\code{__getitem__}", and so on are not supported. The specific
         abstract-object routines for these must be used.
     \end{cfuncdesc}


     \begin{cfuncdesc}{int}{PyObject_Hash}{PyObject *o}
         Compute and return the hash value of an object \code{o}.  On
         failure, return -1.  This is the equivalent of the Python
         expression: \code{hash(o)}.
     \end{cfuncdesc}


     \begin{cfuncdesc}{int}{PyObject_IsTrue}{PyObject *o}
	 Returns 1 if the object \code{o} is considered to be true, and
	 0 otherwise. This is equivalent to the Python expression:
	 \code{not not o}.
	 This function always succeeds.
     \end{cfuncdesc}
	 

     \begin{cfuncdesc}{PyObject*}{PyObject_Type}{PyObject *o}
	 On success, returns a type object corresponding to the object
	 type of object \code{o}. On failure, returns {\NULL}.  This is
	 equivalent to the Python expression: \code{type(o)}.
     \end{cfuncdesc}

     \begin{cfuncdesc}{int}{PyObject_Length}{PyObject *o}
         Return the length of object \code{o}.  If the object \code{o} provides
	 both sequence and mapping protocols, the sequence length is
	 returned. On error, -1 is returned.  This is the equivalent
	 to the Python expression: \code{len(o)}.
     \end{cfuncdesc}


     \begin{cfuncdesc}{PyObject*}{PyObject_GetItem}{PyObject *o, PyObject *key}
	 Return element of \code{o} corresponding to the object \code{key} or {\NULL}
	 on failure. This is the equivalent of the Python expression:
	 \code{o[key]}.
     \end{cfuncdesc}


     \begin{cfuncdesc}{int}{PyObject_SetItem}{PyObject *o, PyObject *key, PyObject *v}
	 Map the object \code{key} to the value \code{v}.
	 Returns -1 on failure.  This is the equivalent
	 of the Python statement: \code{o[key]=v}.
     \end{cfuncdesc}


     \begin{cfuncdesc}{int}{PyObject_DelItem}{PyObject *o, PyObject *key, PyObject *v}
	 Delete the mapping for \code{key} from \code{*o}.  Returns -1
	 on failure.
	 This is the equivalent of the Python statement: del o[key].
     \end{cfuncdesc}


\section{Number Protocol}

     \begin{cfuncdesc}{int}{PyNumber_Check}{PyObject *o}
         Returns 1 if the object \code{o} provides numeric protocols, and
	 false otherwise. 
	 This function always succeeds.
     \end{cfuncdesc}


     \begin{cfuncdesc}{PyObject*}{PyNumber_Add}{PyObject *o1, PyObject *o2}
	 Returns the result of adding \code{o1} and \code{o2}, or null on failure.
	 This is the equivalent of the Python expression: \code{o1+o2}.
     \end{cfuncdesc}


     \begin{cfuncdesc}{PyObject*}{PyNumber_Subtract}{PyObject *o1, PyObject *o2}
	 Returns the result of subtracting \code{o2} from \code{o1}, or null on
	 failure.  This is the equivalent of the Python expression:
	 \code{o1-o2}.
     \end{cfuncdesc}


     \begin{cfuncdesc}{PyObject*}{PyNumber_Multiply}{PyObject *o1, PyObject *o2}
	 Returns the result of multiplying \code{o1} and \code{o2}, or null on
	 failure.  This is the equivalent of the Python expression:
	 \code{o1*o2}.
     \end{cfuncdesc}


     \begin{cfuncdesc}{PyObject*}{PyNumber_Divide}{PyObject *o1, PyObject *o2}
	 Returns the result of dividing \code{o1} by \code{o2}, or null on failure.
	 This is the equivalent of the Python expression: \code{o1/o2}.
     \end{cfuncdesc}


     \begin{cfuncdesc}{PyObject*}{PyNumber_Remainder}{PyObject *o1, PyObject *o2}
	 Returns the remainder of dividing \code{o1} by \code{o2}, or null on
	 failure.  This is the equivalent of the Python expression:
	 \code{o1\%o2}.
     \end{cfuncdesc}


     \begin{cfuncdesc}{PyObject*}{PyNumber_Divmod}{PyObject *o1, PyObject *o2}
	 See the built-in function divmod.  Returns {\NULL} on failure.
	 This is the equivalent of the Python expression:
	 \code{divmod(o1,o2)}.
     \end{cfuncdesc}


     \begin{cfuncdesc}{PyObject*}{PyNumber_Power}{PyObject *o1, PyObject *o2, PyObject *o3}
	 See the built-in function pow.  Returns {\NULL} on failure.
	 This is the equivalent of the Python expression:
	 \code{pow(o1,o2,o3)}, where \code{o3} is optional.
     \end{cfuncdesc}


     \begin{cfuncdesc}{PyObject*}{PyNumber_Negative}{PyObject *o}
	 Returns the negation of \code{o} on success, or null on failure.
	 This is the equivalent of the Python expression: \code{-o}.
     \end{cfuncdesc}


     \begin{cfuncdesc}{PyObject*}{PyNumber_Positive}{PyObject *o}
         Returns \code{o} on success, or {\NULL} on failure.
	 This is the equivalent of the Python expression: \code{+o}.
     \end{cfuncdesc}


     \begin{cfuncdesc}{PyObject*}{PyNumber_Absolute}{PyObject *o}
	 Returns the absolute value of \code{o}, or null on failure.  This is
	 the equivalent of the Python expression: \code{abs(o)}.
     \end{cfuncdesc}


     \begin{cfuncdesc}{PyObject*}{PyNumber_Invert}{PyObject *o}
	 Returns the bitwise negation of \code{o} on success, or {\NULL} on
	 failure.  This is the equivalent of the Python expression:
	 \code{~o}.
     \end{cfuncdesc}


     \begin{cfuncdesc}{PyObject*}{PyNumber_Lshift}{PyObject *o1, PyObject *o2}
	 Returns the result of left shifting \code{o1} by \code{o2} on success, or
	 {\NULL} on failure.  This is the equivalent of the Python
	 expression: \code{o1 << o2}.
     \end{cfuncdesc}


     \begin{cfuncdesc}{PyObject*}{PyNumber_Rshift}{PyObject *o1, PyObject *o2}
	 Returns the result of right shifting \code{o1} by \code{o2} on success, or
	 {\NULL} on failure.  This is the equivalent of the Python
	 expression: \code{o1 >> o2}.
     \end{cfuncdesc}


     \begin{cfuncdesc}{PyObject*}{PyNumber_And}{PyObject *o1, PyObject *o2}
	 Returns the result of "anding" \code{o2} and \code{o2} on success and {\NULL}
	 on failure. This is the equivalent of the Python
	 expression: \code{o1 and o2}.
     \end{cfuncdesc}


     \begin{cfuncdesc}{PyObject*}{PyNumber_Xor}{PyObject *o1, PyObject *o2}
	 Returns the bitwise exclusive or of \code{o1} by \code{o2} on success, or
	 {\NULL} on failure.  This is the equivalent of the Python
	 expression: \code{o1\^{ }o2}.
     \end{cfuncdesc}

     \begin{cfuncdesc}{PyObject*}{PyNumber_Or}{PyObject *o1, PyObject *o2}
	 Returns the result or \code{o1} and \code{o2} on success, or {\NULL} on
	 failure.  This is the equivalent of the Python expression: 
	 \code{o1 or o2}.
     \end{cfuncdesc}


     \begin{cfuncdesc}{PyObject*}{PyNumber_Coerce}{PyObject *o1, PyObject *o2}
         On success, returns a tuple containing \code{o1} and \code{o2} converted to
	 a common numeric type, or None if no conversion is possible.
	 Returns -1 on failure. This is equivalent to the Python
	 expression: \code{coerce(o1,o2)}.
     \end{cfuncdesc}


     \begin{cfuncdesc}{PyObject*}{PyNumber_Int}{PyObject *o}
	 Returns the \code{o} converted to an integer object on success, or
	 {\NULL} on failure.  This is the equivalent of the Python
	 expression: \code{int(o)}.
     \end{cfuncdesc}


     \begin{cfuncdesc}{PyObject*}{PyNumber_Long}{PyObject *o}
	 Returns the \code{o} converted to a long integer object on success,
	 or {\NULL} on failure.  This is the equivalent of the Python
	 expression: \code{long(o)}.
     \end{cfuncdesc}


     \begin{cfuncdesc}{PyObject*}{PyNumber_Float}{PyObject *o}
	 Returns the \code{o} converted to a float object on success, or {\NULL}
	 on failure.  This is the equivalent of the Python expression:
	 \code{float(o)}.
     \end{cfuncdesc}


\section{Sequence protocol}

     \begin{cfuncdesc}{int}{PySequence_Check}{PyObject *o}
         Return 1 if the object provides sequence protocol, and 0
	 otherwise.  
	 This function always succeeds.
     \end{cfuncdesc}


     \begin{cfuncdesc}{PyObject*}{PySequence_Concat}{PyObject *o1, PyObject *o2}
	 Return the concatination of \code{o1} and \code{o2} on success, and {\NULL} on
	 failure.   This is the equivalent of the Python
	 expression: \code{o1+o2}.
     \end{cfuncdesc}


     \begin{cfuncdesc}{PyObject*}{PySequence_Repeat}{PyObject *o, int count}
	 Return the result of repeating sequence object \code{o} count times,
	 or {\NULL} on failure.  This is the equivalent of the Python
	 expression: \code{o*count}.
     \end{cfuncdesc}


     \begin{cfuncdesc}{PyObject*}{PySequence_GetItem}{PyObject *o, int i}
	 Return the ith element of \code{o}, or {\NULL} on failure. This is the
	 equivalent of the Python expression: \code{o[i]}.
     \end{cfuncdesc}


     \begin{cfuncdesc}{PyObject*}{PySequence_GetSlice}{PyObject *o, int i1, int i2}
	 Return the slice of sequence object \code{o} between \code{i1} and \code{i2}, or
	 {\NULL} on failure. This is the equivalent of the Python
	 expression, \code{o[i1:i2]}.
     \end{cfuncdesc}


     \begin{cfuncdesc}{int}{PySequence_SetItem}{PyObject *o, int i, PyObject *v}
	 Assign object \code{v} to the \code{i}th element of \code{o}.
Returns -1 on failure.  This is the equivalent of the Python
	 statement, \code{o[i]=v}.
     \end{cfuncdesc}

     \begin{cfuncdesc}{int}{PySequence_DelItem}{PyObject *o, int i}
	 Delete the \code{i}th element of object \code{v}.  Returns
	 -1 on failure.  This is the equivalent of the Python
	 statement: \code{del o[i]}.
     \end{cfuncdesc}

     \begin{cfuncdesc}{int}{PySequence_SetSlice}{PyObject *o, int i1, int i2, PyObject *v}
         Assign the sequence object \code{v} to the slice in sequence
	 object \code{o} from \code{i1} to \code{i2}.  This is the equivalent of the Python
	 statement, \code{o[i1:i2]=v}.
     \end{cfuncdesc}

     \begin{cfuncdesc}{int}{PySequence_DelSlice}{PyObject *o, int i1, int i2}
	 Delete the slice in sequence object, \code{o}, from \code{i1} to \code{i2}.
	 Returns -1 on failure. This is the equivalent of the Python
	 statement: \code{del o[i1:i2]}.
     \end{cfuncdesc}

     \begin{cfuncdesc}{PyObject*}{PySequence_Tuple}{PyObject *o}
	 Returns the \code{o} as a tuple on success, and {\NULL} on failure.
	 This is equivalent to the Python expression: \code{tuple(o)}.
     \end{cfuncdesc}

     \begin{cfuncdesc}{int}{PySequence_Count}{PyObject *o, PyObject *value}
         Return the number of occurrences of \code{value} on \code{o}, that is,
	 return the number of keys for which \code{o[key]==value}.  On
	 failure, return -1.  This is equivalent to the Python
	 expression: \code{o.count(value)}.
     \end{cfuncdesc}

     \begin{cfuncdesc}{int}{PySequence_In}{PyObject *o, PyObject *value}
	 Determine if \code{o} contains \code{value}.  If an item in \code{o} is equal to
	 \code{value}, return 1, otherwise return 0.  On error, return -1.  This
	 is equivalent to the Python expression: \code{value in o}.
     \end{cfuncdesc}

     \begin{cfuncdesc}{int}{PySequence_Index}{PyObject *o, PyObject *value}
	 Return the first index for which \code{o[i]=value}.  On error,
	 return -1.    This is equivalent to the Python
	 expression: \code{o.index(value)}.
     \end{cfuncdesc}

\section{Mapping protocol}

     \begin{cfuncdesc}{int}{PyMapping_Check}{PyObject *o}
         Return 1 if the object provides mapping protocol, and 0
	 otherwise.  
	 This function always succeeds.
     \end{cfuncdesc}


     \begin{cfuncdesc}{int}{PyMapping_Length}{PyObject *o}
         Returns the number of keys in object \code{o} on success, and -1 on
	 failure.  For objects that do not provide sequence protocol,
	 this is equivalent to the Python expression: \code{len(o)}.
     \end{cfuncdesc}


     \begin{cfuncdesc}{int}{PyMapping_DelItemString}{PyObject *o, char *key}
	 Remove the mapping for object \code{key} from the object \code{o}.
	 Return -1 on failure.  This is equivalent to
	 the Python statement: \code{del o[key]}.
     \end{cfuncdesc}


     \begin{cfuncdesc}{int}{PyMapping_DelItem}{PyObject *o, PyObject *key}
	 Remove the mapping for object \code{key} from the object \code{o}.
	 Return -1 on failure.  This is equivalent to
	 the Python statement: \code{del o[key]}.
     \end{cfuncdesc}


     \begin{cfuncdesc}{int}{PyMapping_HasKeyString}{PyObject *o, char *key}
	 On success, return 1 if the mapping object has the key \code{key}
	 and 0 otherwise.  This is equivalent to the Python expression:
	 \code{o.has_key(key)}. 
	 This function always succeeds.
     \end{cfuncdesc}


     \begin{cfuncdesc}{int}{PyMapping_HasKey}{PyObject *o, PyObject *key}
	 Return 1 if the mapping object has the key \code{key}
	 and 0 otherwise.  This is equivalent to the Python expression:
	 \code{o.has_key(key)}. 
	 This function always succeeds.
     \end{cfuncdesc}


     \begin{cfuncdesc}{PyObject*}{PyMapping_Keys}{PyObject *o}
         On success, return a list of the keys in object \code{o}.  On
	 failure, return {\NULL}. This is equivalent to the Python
	 expression: \code{o.keys()}.
     \end{cfuncdesc}


     \begin{cfuncdesc}{PyObject*}{PyMapping_Values}{PyObject *o}
         On success, return a list of the values in object \code{o}.  On
	 failure, return {\NULL}. This is equivalent to the Python
	 expression: \code{o.values()}.
     \end{cfuncdesc}


     \begin{cfuncdesc}{PyObject*}{PyMapping_Items}{PyObject *o}
         On success, return a list of the items in object \code{o}, where
	 each item is a tuple containing a key-value pair.  On
	 failure, return {\NULL}. This is equivalent to the Python
	 expression: \code{o.items()}.
     \end{cfuncdesc}

     \begin{cfuncdesc}{int}{PyMapping_Clear}{PyObject *o}
         Make object \code{o} empty.  Returns 1 on success and 0 on failure.
	 This is equivalent to the Python statement:
	 \code{for key in o.keys(): del o[key]}
     \end{cfuncdesc}


     \begin{cfuncdesc}{PyObject*}{PyMapping_GetItemString}{PyObject *o, char *key}
	 Return element of \code{o} corresponding to the object \code{key} or {\NULL}
	 on failure. This is the equivalent of the Python expression:
	 \code{o[key]}.
     \end{cfuncdesc}

     \begin{cfuncdesc}{PyObject*}{PyMapping_SetItemString}{PyObject *o, char *key, PyObject *v}
         Map the object \code{key} to the value \code{v} in object \code{o}.  Returns 
         -1 on failure.  This is the equivalent of the Python
         statement: \code{o[key]=v}.
     \end{cfuncdesc}


\section{Constructors}

     \begin{cfuncdesc}{PyObject*}{PyFile_FromString}{char *file_name, char *mode}
	 On success, returns a new file object that is opened on the
	 file given by \code{file_name}, with a file mode given by \code{mode},
	 where \code{mode} has the same semantics as the standard C routine,
	 fopen.  On failure, return -1.
     \end{cfuncdesc}
     
     \begin{cfuncdesc}{PyObject*}{PyFile_FromFile}{FILE *fp, char *file_name, char *mode, int close_on_del}
	 Return a new file object for an already opened standard C
	 file pointer, \code{fp}.  A file name, \code{file_name}, and open mode,
	 \code{mode}, must be provided as well as a flag, \code{close_on_del}, that
	 indicates whether the file is to be closed when the file
	 object is destroyed.  On failure, return -1.
     \end{cfuncdesc}

     \begin{cfuncdesc}{PyObject*}{PyFloat_FromDouble}{double v}
	 Returns a new float object with the value \code{v} on success, and
	 {\NULL} on failure.
     \end{cfuncdesc}
     
     \begin{cfuncdesc}{PyObject*}{PyInt_FromLong}{long v}
	 Returns a new int object with the value \code{v} on success, and
	 {\NULL} on failure.
     \end{cfuncdesc}

     \begin{cfuncdesc}{PyObject*}{PyList_New}{int l}
	 Returns a new list of length \code{l} on success, and {\NULL} on
	 failure.
     \end{cfuncdesc}

     \begin{cfuncdesc}{PyObject*}{PyLong_FromLong}{long v}
	 Returns a new long object with the value \code{v} on success, and
	 {\NULL} on failure.
     \end{cfuncdesc}

     \begin{cfuncdesc}{PyObject*}{PyLong_FromDouble}{double v}
	 Returns a new long object with the value \code{v} on success, and
	 {\NULL} on failure.
     \end{cfuncdesc}

     \begin{cfuncdesc}{PyObject*}{PyDict_New}{}
	 Returns a new empty dictionary on success, and {\NULL} on
	 failure.
     \end{cfuncdesc}

     \begin{cfuncdesc}{PyObject*}{PyString_FromString}{char *v}
	 Returns a new string object with the value \code{v} on success, and
	 {\NULL} on failure.
     \end{cfuncdesc}

     \begin{cfuncdesc}{PyObject*}{PyString_FromStringAndSize}{char *v, int l}
	 Returns a new string object with the value \code{v} and length \code{l}
	 on success, and {\NULL} on failure.
     \end{cfuncdesc}

     \begin{cfuncdesc}{PyObject*}{PyTuple_New}{int l}
	 Returns a new tuple of length \code{l} on success, and {\NULL} on
	 failure.
     \end{cfuncdesc}

