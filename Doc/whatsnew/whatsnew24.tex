\documentclass{howto}
\usepackage{distutils}
% $Id$

\title{What's New in Python 2.4}
\release{0.0}
\author{A.M.\ Kuchling}
\authoraddress{\email{amk@amk.ca}}

\begin{document}
\maketitle
\tableofcontents

This article explains the new features in Python 2.4.  No release date
for Python 2.4 has been set; expect that this will happen in 2004.

While Python 2.3 was primarily a library development release, Python
2.4 may extend the core language and interpreter in
as-yet-undetermined ways.

This article doesn't attempt to provide a complete specification of
the new features, but instead provides a convenient overview.  For
full details, you should refer to the documentation for Python 2.4.
% add hyperlink when the documentation becomes available online.
If you want to understand the complete implementation and design
rationale, refer to the PEP for a particular new feature.


%======================================================================

% Large, PEP-level features and changes should be described here.


%======================================================================
\section{Other Language Changes}

Here are all of the changes that Python 2.4 makes to the core Python
language.

\begin{itemize}
\item TBD

\end{itemize}


%======================================================================
\subsection{Optimizations}

\begin{itemize}

\item Optimizations should be described here.

\end{itemize}

The net result of the 2.4 optimizations is that Python 2.4 runs the
pystone benchmark around XX\% faster than Python 2.3 and YY\% faster
than Python 2.2.


%======================================================================
\section{New, Improved, and Deprecated Modules}

As usual, Python's standard library received a number of enhancements and
bug fixes.  Here's a partial list of the most notable changes, sorted
alphabetically by module name. Consult the
\file{Misc/NEWS} file in the source tree for a more
complete list of changes, or look through the CVS logs for all the
details.

\begin{itemize}

\item The \module{curses} modules now supports the ncurses extension 
   \function{use_default_colors()}.   On platforms where the terminal 
   supports transparency, this makes it possible to use a transparent background.
   (Contributed by J\"org Lehmann.)
   
\end{itemize}


%======================================================================
% whole new modules get described in \subsections here


% ======================================================================
\section{Build and C API Changes}

Changes to Python's build process and to the C API include:

\begin{itemize}

\item Detailed changes are listed here.

\end{itemize}


%======================================================================
\subsection{Port-Specific Changes}

Platform-specific changes go here.


%======================================================================
\section{Other Changes and Fixes \label{section-other}}

As usual, there were a bunch of other improvements and bugfixes
scattered throughout the source tree.  A search through the CVS change
logs finds there were XXX patches applied and YYY bugs fixed between
Python 2.3 and 2.4.  Both figures are likely to be underestimates.

Some of the more notable changes are:

\begin{itemize}

\item Details go here.

\end{itemize}


%======================================================================
\section{Porting to Python 2.4}

This section lists previously described changes that may require
changes to your code:

\begin{itemize}

\item dircache.listdir now passes exceptions to the caller, 
instead of returning empty lists.

\end{itemize}


%======================================================================
\section{Acknowledgements \label{acks}}

The author would like to thank the following people for offering
suggestions, corrections and assistance with various drafts of this
article: .

\end{document}
