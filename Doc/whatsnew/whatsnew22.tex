\documentclass{howto}

% $Id$

\title{What's New in Python 2.2}
\release{0.02}
\author{A.M. Kuchling}
\authoraddress{\email{akuchlin@mems-exchange.org}}
\begin{document}
\maketitle\tableofcontents

\section{Introduction}

{\large This document is a draft, and is subject to change until the
final version of Python 2.2 is released.  Currently it's not up to
date at all.  Please send any comments, bug reports, or questions, no
matter how minor, to \email{akuchlin@mems-exchange.org}.  }

This article explains the new features in Python 2.2.  Python 2.2
includes some significant changes that go far toward cleaning up the
language's darkest corners, and some exciting new features.

This article doesn't attempt to provide a complete specification for
the new features, but instead provides a convenient overview of the
new features.  For full details, you should refer to 2.2 documentation
such as the
\citetitle[http://python.sourceforge.net/devel-docs/lib/lib.html]{Python
Library Reference} and the
\citetitle[http://python.sourceforge.net/devel-docs/ref/ref.html]{Python
Reference Manual}, or to the PEP for a particular new feature.
% These \citetitle marks should get the python.org URLs for the final
% release, just as soon as the docs are published there.

The final release of Python 2.2 is planned for October 2001.

%======================================================================
% It looks like this set of changes will likely get into 2.2,
% so I need to read and digest the relevant PEPs.
%\section{PEP 252: Type and Class Changes}

%XXX

%\begin{seealso}

%\seepep{252}{Making Types Look More Like Classes}{Written and implemented 
%by GvR.}

%\end{seealso}

%======================================================================
\section{PEP 234: Iterators}

A significant addition to 2.2 is an iteration interface at both the C
and Python levels.  Objects can define how they can be looped over by
callers.

In Python versions up to 2.1, the usual way to make \code{for item in
obj} work is to define a \method{__getitem__()} method that looks
something like this:

\begin{verbatim}
    def __getitem__(self, index):
        return <next item>
\end{verbatim}

\method{__getitem__()} is more properly used to define an indexing
operation on an object so that you can write \code{obj[5]} to retrieve
the fifth element.  It's a bit misleading when you're using this only
to support \keyword{for} loops.  Consider some file-like object that
wants to be looped over; the \var{index} parameter is essentially
meaningless, as the class probably assumes that a series of
\method{__getitem__()} calls will be made, with \var{index}
incrementing by one each time.  In other words, the presence of the
\method{__getitem__()} method doesn't mean that \code{file[5]} will
work, though it really should.

In Python 2.2, iteration can be implemented separately, and
\method{__getitem__()} methods can be limited to classes that really
do support random access.  The basic idea of iterators is quite
simple.  A new built-in function, \function{iter(obj)}, returns an
iterator for the object \var{obj}.  (It can also take two arguments:
\code{iter(\var{C}, \var{sentinel})} will call the callable \var{C},
until it returns \var{sentinel}, which will signal that the iterator
is done.  This form probably won't be used very often.)

Python classes can define an \method{__iter__()} method, which should
create and return a new iterator for the object; if the object is its
own iterator, this method can just return \code{self}.  In particular,
iterators will usually be their own iterators.  Extension types
implemented in C can implement a \code{tp_iter} function in order to
return an iterator, and extension types that want to behave as
iterators can define a \code{tp_iternext} function.

So what do iterators do?  They have one required method,
\method{next()}, which takes no arguments and returns the next value.
When there are no more values to be returned, calling \method{next()}
should raise the \exception{StopIteration} exception.

\begin{verbatim}
>>> L = [1,2,3]
>>> i = iter(L)
>>> print i
<iterator object at 0x8116870>
>>> i.next()
1
>>> i.next()
2
>>> i.next()
3
>>> i.next()
Traceback (most recent call last):
  File "<stdin>", line 1, in ?
StopIteration
>>>      
\end{verbatim}

In 2.2, Python's \keyword{for} statement no longer expects a sequence;
it expects something for which \function{iter()} will return something.
For backward compatibility, and convenience, an iterator is
automatically constructed for sequences that don't implement
\method{__iter__()} or a \code{tp_iter} slot, so \code{for i in
[1,2,3]} will still work.  Wherever the Python interpreter loops over
a sequence, it's been changed to use the iterator protocol.  This
means you can do things like this:

\begin{verbatim}
>>> i = iter(L)
>>> a,b,c = i
>>> a,b,c
(1, 2, 3)
>>>
\end{verbatim}

Iterator support has been added to some of Python's basic types.  The
\keyword{in} operator now works on dictionaries, so \code{\var{key} in
dict} is now equivalent to \code{dict.has_key(\var{key})}.
Calling \function{iter()} on a dictionary will return an iterator
which loops over its keys:

\begin{verbatim}
>>> m = {'Jan': 1, 'Feb': 2, 'Mar': 3, 'Apr': 4, 'May': 5, 'Jun': 6,
...      'Jul': 7, 'Aug': 8, 'Sep': 9, 'Oct': 10, 'Nov': 11, 'Dec': 12}
>>> for key in m: print key, m[key]
...
Mar 3
Feb 2
Aug 8
Sep 9
May 5
Jun 6
Jul 7
Jan 1
Apr 4
Nov 11
Dec 12
Oct 10
>>>
\end{verbatim}          

That's just the default behaviour.  If you want to iterate over keys,
values, or key/value pairs, you can explicitly call the
\method{iterkeys()}, \method{itervalues()}, or \method{iteritems()}
methods to get an appropriate iterator.  

Files also provide an iterator, which calls its \method{readline()}
method until there are no more lines in the file.  This means you can
now read each line of a file using code like this:

\begin{verbatim}
for line in file:
    # do something for each line
\end{verbatim}

Note that you can only go forward in an iterator; there's no way to
get the previous element, reset the iterator, or make a copy of it.
An iterator object could provide such additional capabilities, but the
iterator protocol only requires a \method{next()} method.

\begin{seealso}

\seepep{234}{Iterators}{Written by Ka-Ping Yee and GvR; implemented 
by the Python Labs crew, mostly by GvR and Tim Peters.}

\end{seealso}

%======================================================================
\section{PEP 255: Simple Generators}

Generators are another new feature, one that interacts with the
introduction of iterators.

You're doubtless familiar with how function calls work in Python or
C.  When you call a function, it gets a private area where its local
variables are created.  When the function reaches a \keyword{return}
statement, the local variables are destroyed and the resulting value
is returned to the caller.  A later call to the same function will get
a fresh new set of local variables.  But, what if the local variables
weren't destroyed on exiting a function?  What if you could later
resume the function where it left off?  This is what generators
provide; they can be thought of as resumable functions.

Here's the simplest example of a generator function:

\begin{verbatim}
def generate_ints(N):
    for i in range(N):
        yield i
\end{verbatim}

A new keyword, \keyword{yield}, was introduced for generators.  Any
function containing a \keyword{yield} statement is a generator
function; this is detected by Python's bytecode compiler which
compiles the function specially.  Because a new keyword was
introduced, generators must be explicitly enabled in a module by
including a \code{from __future__ import generators} statement near
the top of the module's source code.  In Python 2.3 this statement
will become unnecessary.

When you call a generator function, it doesn't return a single value;
instead it returns a generator object that supports the iterator
interface.  On executing the \keyword{yield} statement, the generator
outputs the value of \code{i}, similar to a \keyword{return}
statement.  The big difference between \keyword{yield} and a
\keyword{return} statement is that, on reaching a \keyword{yield} the
generator's state of execution is suspended and local variables are
preserved.  On the next call to the generator's \code{.next()} method,
the function will resume executing immediately after the
\keyword{yield} statement.  (For complicated reasons, the
\keyword{yield} statement isn't allowed inside the \keyword{try} block
of a \code{try...finally} statement; read PEP 255 for a full
explanation of the interaction between \keyword{yield} and
exceptions.)

Here's a sample usage of the \function{generate_ints} generator:

\begin{verbatim}
>>> gen = generate_ints(3)
>>> gen
<generator object at 0x8117f90>
>>> gen.next()
0
>>> gen.next()
1
>>> gen.next()
2
>>> gen.next()
Traceback (most recent call last):
  File "<stdin>", line 1, in ?
  File "<stdin>", line 2, in generate_ints
StopIteration
>>>
\end{verbatim}

You could equally write \code{for i in generate_ints(5)}, or
\code{a,b,c = generate_ints(3)}.

Inside a generator function, the \keyword{return} statement can only
be used without a value, and signals the end of the procession of
values; afterwards the generator cannot return any further values.
\keyword{return} with a value, such as \code{return 5}, is a syntax
error inside a generator function.  The end of the generator's results
can also be indicated by raising \exception{StopIteration} manually,
or by just letting the flow of execution fall off the bottom of the
function.

You could achieve the effect of generators manually by writing your
own class and storing all the local variables of the generator as
instance variables.  For example, returning a list of integers could
be done by setting \code{self.count} to 0, and having the
\method{next()} method increment \code{self.count} and return it.
However, for a moderately complicated generator, writing a
corresponding class would be much messier.
\file{Lib/test/test_generators.py} contains a number of more
interesting examples.  The simplest one implements an in-order
traversal of a tree using generators recursively.

\begin{verbatim}
# A recursive generator that generates Tree leaves in in-order.
def inorder(t):
    if t:
        for x in inorder(t.left):
            yield x
        yield t.label
        for x in inorder(t.right):
            yield x
\end{verbatim}

Two other examples in \file{Lib/test/test_generators.py} produce
solutions for the N-Queens problem (placing $N$ queens on an $NxN$
chess board so that no queen threatens another) and the Knight's Tour
(a route that takes a knight to every square of an $NxN$ chessboard
without visiting any square twice). 

The idea of generators comes from other programming languages,
especially Icon (\url{http://www.cs.arizona.edu/icon/}), where the
idea of generators is central to the language.  In Icon, every
expression and function call behaves like a generator.  One example
from ``An Overview of the Icon Programming Language'' at
\url{http://www.cs.arizona.edu/icon/docs/ipd266.htm} gives an idea of
what this looks like:

\begin{verbatim}
sentence := "Store it in the neighboring harbor"
if (i := find("or", sentence)) > 5 then write(i)
\end{verbatim}

The \function{find()} function returns the indexes at which the
substring ``or'' is found: 3, 23, 33.  In the \keyword{if} statement,
\code{i} is first assigned a value of 3, but 3 is less than 5, so the
comparison fails, and Icon retries it with the second value of 23.  23
is greater than 5, so the comparison now succeeds, and the code prints
the value 23 to the screen.

Python doesn't go nearly as far as Icon in adopting generators as a
central concept.  Generators are considered a new part of the core
Python language, but learning or using them isn't compulsory; if they
don't solve any problems that you have, feel free to ignore them.
This is different from Icon where the idea of generators is a basic
concept.  One novel feature of Python's interface as compared to
Icon's is that a generator's state is represented as a concrete object
that can be passed around to other functions or stored in a data
structure.

\begin{seealso}

\seepep{255}{Simple Generators}{Written by Neil Schemenauer, Tim
Peters, Magnus Lie Hetland.  Implemented mostly by Neil Schemenauer
and Tim Peters, with other fixes from the Python Labs crew.}

\end{seealso}

%======================================================================
\section{Unicode Changes}

Python's Unicode support has been enhanced a bit in 2.2.  Unicode
strings are usually stored as UCS-2, as 16-bit unsigned integers.
Python 2.2 can also be compiled to use UCS-4, 32-bit unsigned integers
by supplying \longprogramopt{enable-unicode=ucs4} to the configure script.

XXX explain surrogates?  I have to figure out what the changes mean to users.

Since their introduction, Unicode strings (XXX and regular strings in
2.1?)  have supported an \method{encode()} method to convert the
string to a selected encoding such as UTF-8 or Latin-1.  A symmetric
\method{decode(\optional{\var{encoding}})} method has been added to
both 8-bit and Unicode strings in 2.2, which assumes that the string
is in the specified encoding and decodes it. This means that
\method{encode()} and \method{decode()} can be called on both types of
strings, and can be used for tasks not directly related to Unicode.
For example, codecs have been added for UUencoding, MIME's base-64
encoding, and compression with the \module{zlib} module.

\begin{verbatim}
>>> s = """Here is a lengthy piece of redundant, overly verbose,
... and repetitive text.
... """
>>> data = s.encode('zlib')
>>> data
'x\x9c\r\xc9\xc1\r\x80 \x10\x04\xc0?Ul...'
>>> data.decode('zlib')
'Here is a lengthy piece of redundant, overly verbose,\nand repetitive text.\n'
>>> print s.encode('uu')
begin 666 <data>
M2&5R92!I<R!A(&QE;F=T:'D@<&EE8V4@;V8@<F5D=6YD86YT+"!O=F5R;'D@
>=F5R8F]S92P*86YD(')E<&5T:71I=F4@=&5X="X*

end
>>> "sheesh".encode('rot-13')
'furrfu'
\end{verbatim}

References: http://mail.python.org/pipermail/i18n-sig/2001-June/001107.html  
and following thread.

%======================================================================
\section{PEP 227: Nested Scopes}

In Python 2.1, statically nested scopes were added as an optional
feature, to be enabled by a \code{from __future__ import
nested_scopes} directive.  In 2.2 nested scopes no longer need to be
specially enabled, but are always enabled.  The rest of this section
is a copy of the description of nested scopes from my ``What's New in
Python 2.1'' document; if you read it when 2.1 came out, you can skip
the rest of this section.

The largest change introduced in Python 2.1, and made complete in 2.2,
is to Python's scoping rules.  In Python 2.0, at any given time there
are at most three namespaces used to look up variable names: local,
module-level, and the built-in namespace.  This often surprised people
because it didn't match their intuitive expectations.  For example, a
nested recursive function definition doesn't work:

\begin{verbatim}
def f():
    ...
    def g(value):
        ...
        return g(value-1) + 1
    ...
\end{verbatim}

The function \function{g()} will always raise a \exception{NameError}
exception, because the binding of the name \samp{g} isn't in either
its local namespace or in the module-level namespace.  This isn't much
of a problem in practice (how often do you recursively define interior
functions like this?), but this also made using the \keyword{lambda}
statement clumsier, and this was a problem in practice.  In code which
uses \keyword{lambda} you can often find local variables being copied
by passing them as the default values of arguments.

\begin{verbatim}
def find(self, name):
    "Return list of any entries equal to 'name'"
    L = filter(lambda x, name=name: x == name,
               self.list_attribute)
    return L
\end{verbatim}

The readability of Python code written in a strongly functional style
suffers greatly as a result.

The most significant change to Python 2.2 is that static scoping has
been added to the language to fix this problem.  As a first effect,
the \code{name=name} default argument is now unnecessary in the above
example.  Put simply, when a given variable name is not assigned a
value within a function (by an assignment, or the \keyword{def},
\keyword{class}, or \keyword{import} statements), references to the
variable will be looked up in the local namespace of the enclosing
scope.  A more detailed explanation of the rules, and a dissection of
the implementation, can be found in the PEP.

This change may cause some compatibility problems for code where the
same variable name is used both at the module level and as a local
variable within a function that contains further function definitions.
This seems rather unlikely though, since such code would have been
pretty confusing to read in the first place.  

One side effect of the change is that the \code{from \var{module}
import *} and \keyword{exec} statements have been made illegal inside
a function scope under certain conditions.  The Python reference
manual has said all along that \code{from \var{module} import *} is
only legal at the top level of a module, but the CPython interpreter
has never enforced this before.  As part of the implementation of
nested scopes, the compiler which turns Python source into bytecodes
has to generate different code to access variables in a containing
scope.  \code{from \var{module} import *} and \keyword{exec} make it
impossible for the compiler to figure this out, because they add names
to the local namespace that are unknowable at compile time.
Therefore, if a function contains function definitions or
\keyword{lambda} expressions with free variables, the compiler will
flag this by raising a \exception{SyntaxError} exception.

To make the preceding explanation a bit clearer, here's an example:

\begin{verbatim}
x = 1
def f():
    # The next line is a syntax error
    exec 'x=2'  
    def g():
        return x
\end{verbatim}

Line 4 containing the \keyword{exec} statement is a syntax error,
since \keyword{exec} would define a new local variable named \samp{x}
whose value should be accessed by \function{g()}.  

This shouldn't be much of a limitation, since \keyword{exec} is rarely
used in most Python code (and when it is used, it's often a sign of a
poor design anyway).

\begin{seealso}

\seepep{227}{Statically Nested Scopes}{Written and implemented by
Jeremy Hylton.}

\end{seealso}


%======================================================================
\section{New and Improved Modules}

\begin{itemize}

  \item The \module{xmlrpclib} module was contributed to the standard
library by Fredrik Lundh.  It provides support for writing XML-RPC
clients; XML-RPC is a simple remote procedure call protocol built on
top of HTTP and XML. For example, the following snippet retrieves a
list of RSS channels from the O'Reilly Network, and then retrieves a
list of the recent headlines for one channel:

\begin{verbatim}
import xmlrpclib
s = xmlrpclib.Server(
      'http://www.oreillynet.com/meerkat/xml-rpc/server.php')
channels = s.meerkat.getChannels()
# channels is a list of dictionaries, like this:
# [{'id': 4, 'title': 'Freshmeat Daily News'}
#  {'id': 190, 'title': '32Bits Online'},
#  {'id': 4549, 'title': '3DGamers'}, ... ]

# Get the items for one channel
items = s.meerkat.getItems( {'channel': 4} )

# 'items' is another list of dictionaries, like this:
# [{'link': 'http://freshmeat.net/releases/52719/', 
#   'description': 'A utility which converts HTML to XSL FO.', 
#   'title': 'html2fo 0.3 (Default)'}, ... ]
\end{verbatim}

See \url{http://www.xmlrpc.com/} for more information about XML-RPC.

  \item The \module{socket} module can be compiled to support IPv6;
  specify the \longprogramopt{enable-ipv6} option to Python's configure
  script.  (Contributed by Jun-ichiro ``itojun'' Hagino.)

  \item Two new format characters were added to the \module{struct}
  module for 64-bit integers on platforms that support the C
  \ctype{long long} type.  \samp{q} is for a signed 64-bit integer,
  and \samp{Q} is for an unsigned one.  The value is returned in
  Python's long integer type.  (Contributed by Tim Peters.)

  \item In the interpreter's interactive mode, there's a new built-in
  function \function{help()}, that uses the \module{pydoc} module
  introduced in Python 2.1 to provide interactive.
  \code{help(\var{object})} displays any available help text about
  \var{object}.  \code{help()} with no argument puts you in an online
  help utility, where you can enter the names of functions, classes,
  or modules to read their help text.
  (Contributed by Guido van Rossum, using Ka-Ping Yee's \module{pydoc} module.)

  \item Various bugfixes and performance improvements have been made
  to the SRE engine underlying the \module{re} module.  For example,
  \function{re.sub()} will now use \function{string.replace()}
  automatically when the pattern and its replacement are both just
  literal strings without regex metacharacters.  Another contributed
  patch speeds up certain Unicode character ranges by a factor of
  two. (SRE is maintained by Fredrik Lundh.  The BIGCHARSET patch was
  contributed by Martin von L\"owis.)

  \item The \module{imaplib} module now has support for the IMAP
  NAMESPACE extension defined in \rfc{2342}.  (Contributed by Michel
  Pelletier.)

  \item The \module{rfc822} module's parsing of email addresses is
  now compliant with \rfc{2822}, an update to \rfc{822}.  The module's
  name is \emph{not} going to be changed to \samp{rfc2822}.
  (Contributed by Barry Warsaw.)
  
\end{itemize}


%======================================================================
\section{Other Changes and Fixes}

As usual there were a bunch of other improvements and bugfixes
scattered throughout the source tree.  A search through the CVS change
logs finds there were XXX patches applied, and XXX bugs fixed; both
figures are likely to be underestimates.  Some of the more notable
changes are:

\begin{itemize}

  \item Keyword arguments passed to builtin functions that don't take them
  now cause a \exception{TypeError} exception to be raised, with the
  message "\var{function} takes no keyword arguments".
  
  \item The code for the Mac OS port for Python, maintained by Jack
  Jansen, is now kept in the main Python CVS tree.

  \item The new license introduced with Python 1.6 wasn't
  GPL-compatible.  This is fixed by some minor textual changes to the
  2.2 license, so Python can now be embedded inside a GPLed program
  again.  The license changes were also applied to the Python 2.0.1
  and 2.1.1 releases.

  \item Profiling and tracing functions can now be implemented in C,
  which can operate at much higher speeds than Python-based functions
  and should reduce the overhead of enabling profiling and tracing, so
  it will be of interest to authors of development environments for
  Python.  Two new C functions were added to Python's API,
  \cfunction{PyEval_SetProfile()} and \cfunction{PyEval_SetTrace()}.
  The existing \function{sys.setprofile()} and
  \function{sys.settrace()} functions still exist, and have simply
  been changed to use the new C-level interface.  (Contributed by Fred
  L. Drake, Jr.)

  \item The \file{Tools/scripts/ftpmirror.py} script
  now parses a \file{.netrc} file, if you have one.
  (Contributed by Mike Romberg.) 

  \item Some features of the object returned by the
  \function{xrange()} function are now deprecated, and trigger
  warnings when they're accessed; they'll disappear in Python 2.3.
  \class{xrange} objects tried to pretend they were full sequence
  types by supporting slicing, sequence multiplication, and the
  \keyword{in} operator, but these features were rarely used and
  therefore buggy.  The \method{tolist()} method and the
  \member{start}, \member{stop}, and \member{step} attributes are also
  being deprecated.  At the C level, the fourth argument to the
  \cfunction{PyRange_New()} function, \samp{repeat}, has also been
  deprecated.

  \item On Windows, Python can now be compiled with Borland C thanks 
  to a number of patches contribued by Stephen Hansen.

  \item XXX C API: Reorganization of object calling 

The \cfunction{call_object()} function, originally in \file{ceval.c},
begins a new life as the official API \cfunction{PyObject_Call()}.  It
is also much simplified: all it does is call the \member{tp_call}
slot, or raise an exception if that's \NULL.

%The subsidiary functions (call_eval_code2(), call_cfunction(),
%call_instance(), and call_method()) have all been moved to the file
%implementing their particular object type, renamed according to the
%local convention, and added to the type's tp_call slot.  Note that
%call_eval_code2() became function_call(); the tp_slot for class
%objects now simply points to PyInstance_New(), which already has the
%correct signature.
 
%Because of these moves, there are some more new APIs that expose
%helpers in ceval.c that are now needed outside: PyEval_GetFuncName(),
%PyEval_GetFuncDesc(), PyEval_EvalCodeEx() (formerly get_func_name(),
%get_func_desc(), and eval_code2().

  \item XXX Add support for Windows using "mbcs" as the default
  Unicode encoding when dealing with the file system.  As discussed on
  python-dev and in patch 410465.

  \item XXX Lots of patches to dictionaries; measure performance
  improvement, if any.

\end{itemize}



%======================================================================
\section{Acknowledgements}

The author would like to thank the following people for offering
suggestions and corrections to various drafts of this article: Fred
Bremmer, Fred L. Drake, Jr., Tim Peters, Neil Schemenauer.  

\end{document}
