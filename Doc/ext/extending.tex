\chapter{Extending Python with C or \Cpp \label{intro}}


It is quite easy to add new built-in modules to Python, if you know
how to program in C.  Such \dfn{extension modules} can do two things
that can't be done directly in Python: they can implement new built-in
object types, and they can call C library functions and system calls.

To support extensions, the Python API (Application Programmers
Interface) defines a set of functions, macros and variables that
provide access to most aspects of the Python run-time system.  The
Python API is incorporated in a C source file by including the header
\code{"Python.h"}.

The compilation of an extension module depends on its intended use as
well as on your system setup; details are given in later chapters.


\section{A Simple Example
         \label{simpleExample}}

Let's create an extension module called \samp{spam} (the favorite food
of Monty Python fans...) and let's say we want to create a Python
interface to the C library function \cfunction{system()}.\footnote{An
interface for this function already exists in the standard module
\module{os} --- it was chosen as a simple and straightfoward example.}
This function takes a null-terminated character string as argument and
returns an integer.  We want this function to be callable from Python
as follows:

\begin{verbatim}
>>> import spam
>>> status = spam.system("ls -l")
\end{verbatim}

Begin by creating a file \file{spammodule.c}.  (Historically, if a
module is called \samp{spam}, the C file containing its implementation
is called \file{spammodule.c}; if the module name is very long, like
\samp{spammify}, the module name can be just \file{spammify.c}.)

The first line of our file can be:

\begin{verbatim}
#include <Python.h>
\end{verbatim}

which pulls in the Python API (you can add a comment describing the
purpose of the module and a copyright notice if you like).
Since Python may define some pre-processor definitions which affect
the standard headers on some systems, you must include \file{Python.h}
before any standard headers are included.

All user-visible symbols defined by \file{Python.h} have a prefix of
\samp{Py} or \samp{PY}, except those defined in standard header files.
For convenience, and since they are used extensively by the Python
interpreter, \code{"Python.h"} includes a few standard header files:
\code{<stdio.h>}, \code{<string.h>}, \code{<errno.h>}, and
\code{<stdlib.h>}.  If the latter header file does not exist on your
system, it declares the functions \cfunction{malloc()},
\cfunction{free()} and \cfunction{realloc()} directly.

The next thing we add to our module file is the C function that will
be called when the Python expression \samp{spam.system(\var{string})}
is evaluated (we'll see shortly how it ends up being called):

\begin{verbatim}
static PyObject *
spam_system(PyObject *self, PyObject *args)
{
    char *command;
    int sts;

    if (!PyArg_ParseTuple(args, "s", &command))
        return NULL;
    sts = system(command);
    return Py_BuildValue("i", sts);
}
\end{verbatim}

There is a straightforward translation from the argument list in
Python (for example, the single expression \code{"ls -l"}) to the
arguments passed to the C function.  The C function always has two
arguments, conventionally named \var{self} and \var{args}.

The \var{self} argument is only used when the C function implements a
built-in method, not a function. In the example, \var{self} will
always be a \NULL{} pointer, since we are defining a function, not a
method.  (This is done so that the interpreter doesn't have to
understand two different types of C functions.)

The \var{args} argument will be a pointer to a Python tuple object
containing the arguments.  Each item of the tuple corresponds to an
argument in the call's argument list.  The arguments are Python
objects --- in order to do anything with them in our C function we have
to convert them to C values.  The function \cfunction{PyArg_ParseTuple()}
in the Python API checks the argument types and converts them to C
values.  It uses a template string to determine the required types of
the arguments as well as the types of the C variables into which to
store the converted values.  More about this later.

\cfunction{PyArg_ParseTuple()} returns true (nonzero) if all arguments have
the right type and its components have been stored in the variables
whose addresses are passed.  It returns false (zero) if an invalid
argument list was passed.  In the latter case it also raises an
appropriate exception so the calling function can return
\NULL{} immediately (as we saw in the example).


\section{Intermezzo: Errors and Exceptions
         \label{errors}}

An important convention throughout the Python interpreter is the
following: when a function fails, it should set an exception condition
and return an error value (usually a \NULL{} pointer).  Exceptions
are stored in a static global variable inside the interpreter; if this
variable is \NULL{} no exception has occurred.  A second global
variable stores the ``associated value'' of the exception (the second
argument to \keyword{raise}).  A third variable contains the stack
traceback in case the error originated in Python code.  These three
variables are the C equivalents of the Python variables
\code{sys.exc_type}, \code{sys.exc_value} and \code{sys.exc_traceback} (see
the section on module \module{sys} in the
\citetitle[../lib/lib.html]{Python Library Reference}).  It is
important to know about them to understand how errors are passed
around.

The Python API defines a number of functions to set various types of
exceptions.

The most common one is \cfunction{PyErr_SetString()}.  Its arguments
are an exception object and a C string.  The exception object is
usually a predefined object like \cdata{PyExc_ZeroDivisionError}.  The
C string indicates the cause of the error and is converted to a
Python string object and stored as the ``associated value'' of the
exception.

Another useful function is \cfunction{PyErr_SetFromErrno()}, which only
takes an exception argument and constructs the associated value by
inspection of the global variable \cdata{errno}.  The most
general function is \cfunction{PyErr_SetObject()}, which takes two object
arguments, the exception and its associated value.  You don't need to
\cfunction{Py_INCREF()} the objects passed to any of these functions.

You can test non-destructively whether an exception has been set with
\cfunction{PyErr_Occurred()}.  This returns the current exception object,
or \NULL{} if no exception has occurred.  You normally don't need
to call \cfunction{PyErr_Occurred()} to see whether an error occurred in a
function call, since you should be able to tell from the return value.

When a function \var{f} that calls another function \var{g} detects
that the latter fails, \var{f} should itself return an error value
(usually \NULL{} or \code{-1}).  It should \emph{not} call one of the
\cfunction{PyErr_*()} functions --- one has already been called by \var{g}.
\var{f}'s caller is then supposed to also return an error indication
to \emph{its} caller, again \emph{without} calling \cfunction{PyErr_*()},
and so on --- the most detailed cause of the error was already
reported by the function that first detected it.  Once the error
reaches the Python interpreter's main loop, this aborts the currently
executing Python code and tries to find an exception handler specified
by the Python programmer.

(There are situations where a module can actually give a more detailed
error message by calling another \cfunction{PyErr_*()} function, and in
such cases it is fine to do so.  As a general rule, however, this is
not necessary, and can cause information about the cause of the error
to be lost: most operations can fail for a variety of reasons.)

To ignore an exception set by a function call that failed, the exception
condition must be cleared explicitly by calling \cfunction{PyErr_Clear()}. 
The only time C code should call \cfunction{PyErr_Clear()} is if it doesn't
want to pass the error on to the interpreter but wants to handle it
completely by itself (possibly by trying something else, or pretending
nothing went wrong).

Every failing \cfunction{malloc()} call must be turned into an
exception --- the direct caller of \cfunction{malloc()} (or
\cfunction{realloc()}) must call \cfunction{PyErr_NoMemory()} and
return a failure indicator itself.  All the object-creating functions
(for example, \cfunction{PyInt_FromLong()}) already do this, so this
note is only relevant to those who call \cfunction{malloc()} directly.

Also note that, with the important exception of
\cfunction{PyArg_ParseTuple()} and friends, functions that return an
integer status usually return a positive value or zero for success and
\code{-1} for failure, like \UNIX{} system calls.

Finally, be careful to clean up garbage (by making
\cfunction{Py_XDECREF()} or \cfunction{Py_DECREF()} calls for objects
you have already created) when you return an error indicator!

The choice of which exception to raise is entirely yours.  There are
predeclared C objects corresponding to all built-in Python exceptions,
such as \cdata{PyExc_ZeroDivisionError}, which you can use directly.
Of course, you should choose exceptions wisely --- don't use
\cdata{PyExc_TypeError} to mean that a file couldn't be opened (that
should probably be \cdata{PyExc_IOError}).  If something's wrong with
the argument list, the \cfunction{PyArg_ParseTuple()} function usually
raises \cdata{PyExc_TypeError}.  If you have an argument whose value
must be in a particular range or must satisfy other conditions,
\cdata{PyExc_ValueError} is appropriate.

You can also define a new exception that is unique to your module.
For this, you usually declare a static object variable at the
beginning of your file:

\begin{verbatim}
static PyObject *SpamError;
\end{verbatim}

and initialize it in your module's initialization function
(\cfunction{initspam()}) with an exception object (leaving out
the error checking for now):

\begin{verbatim}
PyMODINIT_FUNC
initspam(void)
{
    PyObject *m;

    m = Py_InitModule("spam", SpamMethods);

    SpamError = PyErr_NewException("spam.error", NULL, NULL);
    Py_INCREF(SpamError);
    PyModule_AddObject(m, "error", SpamError);
}
\end{verbatim}

Note that the Python name for the exception object is
\exception{spam.error}.  The \cfunction{PyErr_NewException()} function
may create a class with the base class being \exception{Exception}
(unless another class is passed in instead of \NULL), described in the
\citetitle[../lib/lib.html]{Python Library Reference} under ``Built-in
Exceptions.''

Note also that the \cdata{SpamError} variable retains a reference to
the newly created exception class; this is intentional!  Since the
exception could be removed from the module by external code, an owned
reference to the class is needed to ensure that it will not be
discarded, causing \cdata{SpamError} to become a dangling pointer.
Should it become a dangling pointer, C code which raises the exception
could cause a core dump or other unintended side effects.

We discuss the use of PyMODINIT_FUNC later in this sample.

\section{Back to the Example
         \label{backToExample}}

Going back to our example function, you should now be able to
understand this statement:

\begin{verbatim}
    if (!PyArg_ParseTuple(args, "s", &command))
        return NULL;
\end{verbatim}

It returns \NULL{} (the error indicator for functions returning
object pointers) if an error is detected in the argument list, relying
on the exception set by \cfunction{PyArg_ParseTuple()}.  Otherwise the
string value of the argument has been copied to the local variable
\cdata{command}.  This is a pointer assignment and you are not supposed
to modify the string to which it points (so in Standard C, the variable
\cdata{command} should properly be declared as \samp{const char
*command}).

The next statement is a call to the \UNIX{} function
\cfunction{system()}, passing it the string we just got from
\cfunction{PyArg_ParseTuple()}:

\begin{verbatim}
    sts = system(command);
\end{verbatim}

Our \function{spam.system()} function must return the value of
\cdata{sts} as a Python object.  This is done using the function
\cfunction{Py_BuildValue()}, which is something like the inverse of
\cfunction{PyArg_ParseTuple()}: it takes a format string and an
arbitrary number of C values, and returns a new Python object.
More info on \cfunction{Py_BuildValue()} is given later.

\begin{verbatim}
    return Py_BuildValue("i", sts);
\end{verbatim}

In this case, it will return an integer object.  (Yes, even integers
are objects on the heap in Python!)

If you have a C function that returns no useful argument (a function
returning \ctype{void}), the corresponding Python function must return
\code{None}.   You need this idiom to do so:

\begin{verbatim}
    Py_INCREF(Py_None);
    return Py_None;
\end{verbatim}

\cdata{Py_None} is the C name for the special Python object
\code{None}.  It is a genuine Python object rather than a \NULL{}
pointer, which means ``error'' in most contexts, as we have seen.


\section{The Module's Method Table and Initialization Function
         \label{methodTable}}

I promised to show how \cfunction{spam_system()} is called from Python
programs.  First, we need to list its name and address in a ``method
table'':

\begin{verbatim}
static PyMethodDef SpamMethods[] = {
    ...
    {"system",  spam_system, METH_VARARGS,
     "Execute a shell command."},
    ...
    {NULL, NULL, 0, NULL}        /* Sentinel */
};
\end{verbatim}

Note the third entry (\samp{METH_VARARGS}).  This is a flag telling
the interpreter the calling convention to be used for the C
function.  It should normally always be \samp{METH_VARARGS} or
\samp{METH_VARARGS | METH_KEYWORDS}; a value of \code{0} means that an
obsolete variant of \cfunction{PyArg_ParseTuple()} is used.

When using only \samp{METH_VARARGS}, the function should expect
the Python-level parameters to be passed in as a tuple acceptable for
parsing via \cfunction{PyArg_ParseTuple()}; more information on this
function is provided below.

The \constant{METH_KEYWORDS} bit may be set in the third field if
keyword arguments should be passed to the function.  In this case, the
C function should accept a third \samp{PyObject *} parameter which
will be a dictionary of keywords.  Use
\cfunction{PyArg_ParseTupleAndKeywords()} to parse the arguments to
such a function.

The method table must be passed to the interpreter in the module's
initialization function.  The initialization function must be named
\cfunction{init\var{name}()}, where \var{name} is the name of the
module, and should be the only non-\keyword{static} item defined in
the module file:

\begin{verbatim}
PyMODINIT_FUNC
initspam(void)
{
    (void) Py_InitModule("spam", SpamMethods);
}
\end{verbatim}

Note that PyMODINIT_FUNC declares the function as \code{void} return type, 
declares any special linkage declarations required by the platform, and for 
\Cpp declares the function as \code{extern "C"}.

When the Python program imports module \module{spam} for the first
time, \cfunction{initspam()} is called. (See below for comments about
embedding Python.)  It calls
\cfunction{Py_InitModule()}, which creates a ``module object'' (which
is inserted in the dictionary \code{sys.modules} under the key
\code{"spam"}), and inserts built-in function objects into the newly
created module based upon the table (an array of \ctype{PyMethodDef}
structures) that was passed as its second argument.
\cfunction{Py_InitModule()} returns a pointer to the module object
that it creates (which is unused here).  It aborts with a fatal error
if the module could not be initialized satisfactorily, so the caller
doesn't need to check for errors.

When embedding Python, the \cfunction{initspam()} function is not
called automatically unless there's an entry in the
\cdata{_PyImport_Inittab} table.  The easiest way to handle this is to 
statically initialize your statically-linked modules by directly
calling \cfunction{initspam()} after the call to
\cfunction{Py_Initialize()} or \cfunction{PyMac_Initialize()}:

\begin{verbatim}
int
main(int argc, char *argv[])
{
    /* Pass argv[0] to the Python interpreter */
    Py_SetProgramName(argv[0]);

    /* Initialize the Python interpreter.  Required. */
    Py_Initialize();

    /* Add a static module */
    initspam();
\end{verbatim}

An example may be found in the file \file{Demo/embed/demo.c} in the
Python source distribution.

\note{Removing entries from \code{sys.modules} or importing
compiled modules into multiple interpreters within a process (or
following a \cfunction{fork()} without an intervening
\cfunction{exec()}) can create problems for some extension modules.
Extension module authors should exercise caution when initializing
internal data structures.
Note also that the \function{reload()} function can be used with
extension modules, and will call the module initialization function
(\cfunction{initspam()} in the example), but will not load the module
again if it was loaded from a dynamically loadable object file
(\file{.so} on \UNIX, \file{.dll} on Windows).}

A more substantial example module is included in the Python source
distribution as \file{Modules/xxmodule.c}.  This file may be used as a 
template or simply read as an example.  The \program{modulator.py}
script included in the source distribution or Windows install provides 
a simple graphical user interface for declaring the functions and
objects which a module should implement, and can generate a template
which can be filled in.  The script lives in the
\file{Tools/modulator/} directory; see the \file{README} file there
for more information.


\section{Compilation and Linkage
         \label{compilation}}

There are two more things to do before you can use your new extension:
compiling and linking it with the Python system.  If you use dynamic
loading, the details may depend on the style of dynamic loading your
system uses; see the chapters about building extension modules
(chapter \ref{building}) and additional information that pertains only
to building on Windows (chapter \ref{building-on-windows}) for more
information about this.
% XXX Add information about Mac OS

If you can't use dynamic loading, or if you want to make your module a
permanent part of the Python interpreter, you will have to change the
configuration setup and rebuild the interpreter.  Luckily, this is
very simple on \UNIX: just place your file (\file{spammodule.c} for
example) in the \file{Modules/} directory of an unpacked source
distribution, add a line to the file \file{Modules/Setup.local}
describing your file:

\begin{verbatim}
spam spammodule.o
\end{verbatim}

and rebuild the interpreter by running \program{make} in the toplevel
directory.  You can also run \program{make} in the \file{Modules/}
subdirectory, but then you must first rebuild \file{Makefile}
there by running `\program{make} Makefile'.  (This is necessary each
time you change the \file{Setup} file.)

If your module requires additional libraries to link with, these can
be listed on the line in the configuration file as well, for instance:

\begin{verbatim}
spam spammodule.o -lX11
\end{verbatim}

\section{Calling Python Functions from C
         \label{callingPython}}

So far we have concentrated on making C functions callable from
Python.  The reverse is also useful: calling Python functions from C.
This is especially the case for libraries that support so-called
``callback'' functions.  If a C interface makes use of callbacks, the
equivalent Python often needs to provide a callback mechanism to the
Python programmer; the implementation will require calling the Python
callback functions from a C callback.  Other uses are also imaginable.

Fortunately, the Python interpreter is easily called recursively, and
there is a standard interface to call a Python function.  (I won't
dwell on how to call the Python parser with a particular string as
input --- if you're interested, have a look at the implementation of
the \programopt{-c} command line option in \file{Python/pythonmain.c}
from the Python source code.)

Calling a Python function is easy.  First, the Python program must
somehow pass you the Python function object.  You should provide a
function (or some other interface) to do this.  When this function is
called, save a pointer to the Python function object (be careful to
\cfunction{Py_INCREF()} it!) in a global variable --- or wherever you
see fit. For example, the following function might be part of a module
definition:

\begin{verbatim}
static PyObject *my_callback = NULL;

static PyObject *
my_set_callback(PyObject *dummy, PyObject *args)
{
    PyObject *result = NULL;
    PyObject *temp;

    if (PyArg_ParseTuple(args, "O:set_callback", &temp)) {
        if (!PyCallable_Check(temp)) {
            PyErr_SetString(PyExc_TypeError, "parameter must be callable");
            return NULL;
        }
        Py_XINCREF(temp);         /* Add a reference to new callback */
        Py_XDECREF(my_callback);  /* Dispose of previous callback */
        my_callback = temp;       /* Remember new callback */
        /* Boilerplate to return "None" */
        Py_INCREF(Py_None);
        result = Py_None;
    }
    return result;
}
\end{verbatim}

This function must be registered with the interpreter using the
\constant{METH_VARARGS} flag; this is described in section
\ref{methodTable}, ``The Module's Method Table and Initialization
Function.''  The \cfunction{PyArg_ParseTuple()} function and its
arguments are documented in section~\ref{parseTuple}, ``Extracting
Parameters in Extension Functions.''

The macros \cfunction{Py_XINCREF()} and \cfunction{Py_XDECREF()}
increment/decrement the reference count of an object and are safe in
the presence of \NULL{} pointers (but note that \var{temp} will not be 
\NULL{} in this context).  More info on them in
section~\ref{refcounts}, ``Reference Counts.''

Later, when it is time to call the function, you call the C function
\cfunction{PyEval_CallObject()}.\ttindex{PyEval_CallObject()}  This
function has two arguments, both pointers to arbitrary Python objects:
the Python function, and the argument list.  The argument list must
always be a tuple object, whose length is the number of arguments.  To
call the Python function with no arguments, pass an empty tuple; to
call it with one argument, pass a singleton tuple.
\cfunction{Py_BuildValue()} returns a tuple when its format string
consists of zero or more format codes between parentheses.  For
example:

\begin{verbatim}
    int arg;
    PyObject *arglist;
    PyObject *result;
    ...
    arg = 123;
    ...
    /* Time to call the callback */
    arglist = Py_BuildValue("(i)", arg);
    result = PyEval_CallObject(my_callback, arglist);
    Py_DECREF(arglist);
\end{verbatim}

\cfunction{PyEval_CallObject()} returns a Python object pointer: this is
the return value of the Python function.  \cfunction{PyEval_CallObject()} is
``reference-count-neutral'' with respect to its arguments.  In the
example a new tuple was created to serve as the argument list, which
is \cfunction{Py_DECREF()}-ed immediately after the call.

The return value of \cfunction{PyEval_CallObject()} is ``new'': either it
is a brand new object, or it is an existing object whose reference
count has been incremented.  So, unless you want to save it in a
global variable, you should somehow \cfunction{Py_DECREF()} the result,
even (especially!) if you are not interested in its value.

Before you do this, however, it is important to check that the return
value isn't \NULL.  If it is, the Python function terminated by
raising an exception.  If the C code that called
\cfunction{PyEval_CallObject()} is called from Python, it should now
return an error indication to its Python caller, so the interpreter
can print a stack trace, or the calling Python code can handle the
exception.  If this is not possible or desirable, the exception should
be cleared by calling \cfunction{PyErr_Clear()}.  For example:

\begin{verbatim}
    if (result == NULL)
        return NULL; /* Pass error back */
    ...use result...
    Py_DECREF(result); 
\end{verbatim}

Depending on the desired interface to the Python callback function,
you may also have to provide an argument list to
\cfunction{PyEval_CallObject()}.  In some cases the argument list is
also provided by the Python program, through the same interface that
specified the callback function.  It can then be saved and used in the
same manner as the function object.  In other cases, you may have to
construct a new tuple to pass as the argument list.  The simplest way
to do this is to call \cfunction{Py_BuildValue()}.  For example, if
you want to pass an integral event code, you might use the following
code:

\begin{verbatim}
    PyObject *arglist;
    ...
    arglist = Py_BuildValue("(l)", eventcode);
    result = PyEval_CallObject(my_callback, arglist);
    Py_DECREF(arglist);
    if (result == NULL)
        return NULL; /* Pass error back */
    /* Here maybe use the result */
    Py_DECREF(result);
\end{verbatim}

Note the placement of \samp{Py_DECREF(arglist)} immediately after the
call, before the error check!  Also note that strictly spoken this
code is not complete: \cfunction{Py_BuildValue()} may run out of
memory, and this should be checked.


\section{Extracting Parameters in Extension Functions
         \label{parseTuple}}

\ttindex{PyArg_ParseTuple()}

The \cfunction{PyArg_ParseTuple()} function is declared as follows:

\begin{verbatim}
int PyArg_ParseTuple(PyObject *arg, char *format, ...);
\end{verbatim}

The \var{arg} argument must be a tuple object containing an argument
list passed from Python to a C function.  The \var{format} argument
must be a format string, whose syntax is explained in
``\ulink{Parsing arguments and building
values}{../api/arg-parsing.html}'' in the
\citetitle[../api/api.html]{Python/C API Reference Manual}.  The
remaining arguments must be addresses of variables whose type is
determined by the format string.

Note that while \cfunction{PyArg_ParseTuple()} checks that the Python
arguments have the required types, it cannot check the validity of the
addresses of C variables passed to the call: if you make mistakes
there, your code will probably crash or at least overwrite random bits
in memory.  So be careful!

Note that any Python object references which are provided to the
caller are \emph{borrowed} references; do not decrement their
reference count!

Some example calls:

\begin{verbatim}
    int ok;
    int i, j;
    long k, l;
    char *s;
    int size;

    ok = PyArg_ParseTuple(args, ""); /* No arguments */
        /* Python call: f() */
\end{verbatim}

\begin{verbatim}
    ok = PyArg_ParseTuple(args, "s", &s); /* A string */
        /* Possible Python call: f('whoops!') */
\end{verbatim}

\begin{verbatim}
    ok = PyArg_ParseTuple(args, "lls", &k, &l, &s); /* Two longs and a string */
        /* Possible Python call: f(1, 2, 'three') */
\end{verbatim}

\begin{verbatim}
    ok = PyArg_ParseTuple(args, "(ii)s#", &i, &j, &s, &size);
        /* A pair of ints and a string, whose size is also returned */
        /* Possible Python call: f((1, 2), 'three') */
\end{verbatim}

\begin{verbatim}
    {
        char *file;
        char *mode = "r";
        int bufsize = 0;
        ok = PyArg_ParseTuple(args, "s|si", &file, &mode, &bufsize);
        /* A string, and optionally another string and an integer */
        /* Possible Python calls:
           f('spam')
           f('spam', 'w')
           f('spam', 'wb', 100000) */
    }
\end{verbatim}

\begin{verbatim}
    {
        int left, top, right, bottom, h, v;
        ok = PyArg_ParseTuple(args, "((ii)(ii))(ii)",
                 &left, &top, &right, &bottom, &h, &v);
        /* A rectangle and a point */
        /* Possible Python call:
           f(((0, 0), (400, 300)), (10, 10)) */
    }
\end{verbatim}

\begin{verbatim}
    {
        Py_complex c;
        ok = PyArg_ParseTuple(args, "D:myfunction", &c);
        /* a complex, also providing a function name for errors */
        /* Possible Python call: myfunction(1+2j) */
    }
\end{verbatim}


\section{Keyword Parameters for Extension Functions
         \label{parseTupleAndKeywords}}

\ttindex{PyArg_ParseTupleAndKeywords()}

The \cfunction{PyArg_ParseTupleAndKeywords()} function is declared as
follows:

\begin{verbatim}
int PyArg_ParseTupleAndKeywords(PyObject *arg, PyObject *kwdict,
                                char *format, char *kwlist[], ...);
\end{verbatim}

The \var{arg} and \var{format} parameters are identical to those of the
\cfunction{PyArg_ParseTuple()} function.  The \var{kwdict} parameter
is the dictionary of keywords received as the third parameter from the
Python runtime.  The \var{kwlist} parameter is a \NULL-terminated
list of strings which identify the parameters; the names are matched
with the type information from \var{format} from left to right.  On
success, \cfunction{PyArg_ParseTupleAndKeywords()} returns true,
otherwise it returns false and raises an appropriate exception.

\note{Nested tuples cannot be parsed when using keyword
arguments!  Keyword parameters passed in which are not present in the
\var{kwlist} will cause \exception{TypeError} to be raised.}

Here is an example module which uses keywords, based on an example by
Geoff Philbrick (\email{philbrick@hks.com}):%
\index{Philbrick, Geoff}

\begin{verbatim}
#include "Python.h"

static PyObject *
keywdarg_parrot(PyObject *self, PyObject *args, PyObject *keywds)
{  
    int voltage;
    char *state = "a stiff";
    char *action = "voom";
    char *type = "Norwegian Blue";

    static char *kwlist[] = {"voltage", "state", "action", "type", NULL};

    if (!PyArg_ParseTupleAndKeywords(args, keywds, "i|sss", kwlist, 
                                     &voltage, &state, &action, &type))
        return NULL; 
  
    printf("-- This parrot wouldn't %s if you put %i Volts through it.\n", 
           action, voltage);
    printf("-- Lovely plumage, the %s -- It's %s!\n", type, state);

    Py_INCREF(Py_None);

    return Py_None;
}

static PyMethodDef keywdarg_methods[] = {
    /* The cast of the function is necessary since PyCFunction values
     * only take two PyObject* parameters, and keywdarg_parrot() takes
     * three.
     */
    {"parrot", (PyCFunction)keywdarg_parrot, METH_VARARGS | METH_KEYWORDS,
     "Print a lovely skit to standard output."},
    {NULL, NULL, 0, NULL}   /* sentinel */
};
\end{verbatim}

\begin{verbatim}
void
initkeywdarg(void)
{
  /* Create the module and add the functions */
  Py_InitModule("keywdarg", keywdarg_methods);
}
\end{verbatim}


\section{Building Arbitrary Values
         \label{buildValue}}

This function is the counterpart to \cfunction{PyArg_ParseTuple()}.  It is
declared as follows:

\begin{verbatim}
PyObject *Py_BuildValue(char *format, ...);
\end{verbatim}

It recognizes a set of format units similar to the ones recognized by
\cfunction{PyArg_ParseTuple()}, but the arguments (which are input to the
function, not output) must not be pointers, just values.  It returns a
new Python object, suitable for returning from a C function called
from Python.

One difference with \cfunction{PyArg_ParseTuple()}: while the latter
requires its first argument to be a tuple (since Python argument lists
are always represented as tuples internally),
\cfunction{Py_BuildValue()} does not always build a tuple.  It builds
a tuple only if its format string contains two or more format units.
If the format string is empty, it returns \code{None}; if it contains
exactly one format unit, it returns whatever object is described by
that format unit.  To force it to return a tuple of size 0 or one,
parenthesize the format string.

Examples (to the left the call, to the right the resulting Python value):

\begin{verbatim}
    Py_BuildValue("")                        None
    Py_BuildValue("i", 123)                  123
    Py_BuildValue("iii", 123, 456, 789)      (123, 456, 789)
    Py_BuildValue("s", "hello")              'hello'
    Py_BuildValue("ss", "hello", "world")    ('hello', 'world')
    Py_BuildValue("s#", "hello", 4)          'hell'
    Py_BuildValue("()")                      ()
    Py_BuildValue("(i)", 123)                (123,)
    Py_BuildValue("(ii)", 123, 456)          (123, 456)
    Py_BuildValue("(i,i)", 123, 456)         (123, 456)
    Py_BuildValue("[i,i]", 123, 456)         [123, 456]
    Py_BuildValue("{s:i,s:i}",
                  "abc", 123, "def", 456)    {'abc': 123, 'def': 456}
    Py_BuildValue("((ii)(ii)) (ii)",
                  1, 2, 3, 4, 5, 6)          (((1, 2), (3, 4)), (5, 6))
\end{verbatim}


\section{Reference Counts
         \label{refcounts}}

In languages like C or \Cpp, the programmer is responsible for
dynamic allocation and deallocation of memory on the heap.  In C,
this is done using the functions \cfunction{malloc()} and
\cfunction{free()}.  In \Cpp, the operators \keyword{new} and
\keyword{delete} are used with essentially the same meaning; they are
actually implemented using \cfunction{malloc()} and
\cfunction{free()}, so we'll restrict the following discussion to the
latter.

Every block of memory allocated with \cfunction{malloc()} should
eventually be returned to the pool of available memory by exactly one
call to \cfunction{free()}.  It is important to call
\cfunction{free()} at the right time.  If a block's address is
forgotten but \cfunction{free()} is not called for it, the memory it
occupies cannot be reused until the program terminates.  This is
called a \dfn{memory leak}.  On the other hand, if a program calls
\cfunction{free()} for a block and then continues to use the block, it
creates a conflict with re-use of the block through another
\cfunction{malloc()} call.  This is called \dfn{using freed memory}.
It has the same bad consequences as referencing uninitialized data ---
core dumps, wrong results, mysterious crashes.

Common causes of memory leaks are unusual paths through the code.  For
instance, a function may allocate a block of memory, do some
calculation, and then free the block again.  Now a change in the
requirements for the function may add a test to the calculation that
detects an error condition and can return prematurely from the
function.  It's easy to forget to free the allocated memory block when
taking this premature exit, especially when it is added later to the
code.  Such leaks, once introduced, often go undetected for a long
time: the error exit is taken only in a small fraction of all calls,
and most modern machines have plenty of virtual memory, so the leak
only becomes apparent in a long-running process that uses the leaking
function frequently.  Therefore, it's important to prevent leaks from
happening by having a coding convention or strategy that minimizes
this kind of errors.

Since Python makes heavy use of \cfunction{malloc()} and
\cfunction{free()}, it needs a strategy to avoid memory leaks as well
as the use of freed memory.  The chosen method is called
\dfn{reference counting}.  The principle is simple: every object
contains a counter, which is incremented when a reference to the
object is stored somewhere, and which is decremented when a reference
to it is deleted.  When the counter reaches zero, the last reference
to the object has been deleted and the object is freed.

An alternative strategy is called \dfn{automatic garbage collection}.
(Sometimes, reference counting is also referred to as a garbage
collection strategy, hence my use of ``automatic'' to distinguish the
two.)  The big advantage of automatic garbage collection is that the
user doesn't need to call \cfunction{free()} explicitly.  (Another claimed
advantage is an improvement in speed or memory usage --- this is no
hard fact however.)  The disadvantage is that for C, there is no
truly portable automatic garbage collector, while reference counting
can be implemented portably (as long as the functions \cfunction{malloc()}
and \cfunction{free()} are available --- which the C Standard guarantees).
Maybe some day a sufficiently portable automatic garbage collector
will be available for C.  Until then, we'll have to live with
reference counts.

While Python uses the traditional reference counting implementation,
it also offers a cycle detector that works to detect reference
cycles.  This allows applications to not worry about creating direct
or indirect circular references; these are the weakness of garbage
collection implemented using only reference counting.  Reference
cycles consist of objects which contain (possibly indirect) references
to themselves, so that each object in the cycle has a reference count
which is non-zero.  Typical reference counting implementations are not
able to reclaim the memory belonging to any objects in a reference
cycle, or referenced from the objects in the cycle, even though there
are no further references to the cycle itself.

The cycle detector is able to detect garbage cycles and can reclaim
them so long as there are no finalizers implemented in Python
(\method{__del__()} methods).  When there are such finalizers, the
detector exposes the cycles through the \ulink{\module{gc}
module}{../lib/module-gc.html} (specifically, the \code{garbage}
variable in that module).  The \module{gc} module also exposes a way
to run the detector (the \function{collect()} function), as well as
configuration interfaces and the ability to disable the detector at
runtime.  The cycle detector is considered an optional component;
though it is included by default, it can be disabled at build time
using the \longprogramopt{without-cycle-gc} option to the
\program{configure} script on \UNIX{} platforms (including Mac OS X)
or by removing the definition of \code{WITH_CYCLE_GC} in the
\file{pyconfig.h} header on other platforms.  If the cycle detector is
disabled in this way, the \module{gc} module will not be available.


\subsection{Reference Counting in Python
            \label{refcountsInPython}}

There are two macros, \code{Py_INCREF(x)} and \code{Py_DECREF(x)},
which handle the incrementing and decrementing of the reference count.
\cfunction{Py_DECREF()} also frees the object when the count reaches zero.
For flexibility, it doesn't call \cfunction{free()} directly --- rather, it
makes a call through a function pointer in the object's \dfn{type
object}.  For this purpose (and others), every object also contains a
pointer to its type object.

The big question now remains: when to use \code{Py_INCREF(x)} and
\code{Py_DECREF(x)}?  Let's first introduce some terms.  Nobody
``owns'' an object; however, you can \dfn{own a reference} to an
object.  An object's reference count is now defined as the number of
owned references to it.  The owner of a reference is responsible for
calling \cfunction{Py_DECREF()} when the reference is no longer
needed.  Ownership of a reference can be transferred.  There are three
ways to dispose of an owned reference: pass it on, store it, or call
\cfunction{Py_DECREF()}.  Forgetting to dispose of an owned reference
creates a memory leak.

It is also possible to \dfn{borrow}\footnote{The metaphor of
``borrowing'' a reference is not completely correct: the owner still
has a copy of the reference.} a reference to an object.  The borrower
of a reference should not call \cfunction{Py_DECREF()}.  The borrower must
not hold on to the object longer than the owner from which it was
borrowed.  Using a borrowed reference after the owner has disposed of
it risks using freed memory and should be avoided
completely.\footnote{Checking that the reference count is at least 1
\strong{does not work} --- the reference count itself could be in
freed memory and may thus be reused for another object!}

The advantage of borrowing over owning a reference is that you don't
need to take care of disposing of the reference on all possible paths
through the code --- in other words, with a borrowed reference you
don't run the risk of leaking when a premature exit is taken.  The
disadvantage of borrowing over leaking is that there are some subtle
situations where in seemingly correct code a borrowed reference can be
used after the owner from which it was borrowed has in fact disposed
of it.

A borrowed reference can be changed into an owned reference by calling
\cfunction{Py_INCREF()}.  This does not affect the status of the owner from
which the reference was borrowed --- it creates a new owned reference,
and gives full owner responsibilities (the new owner must
dispose of the reference properly, as well as the previous owner).


\subsection{Ownership Rules
            \label{ownershipRules}}

Whenever an object reference is passed into or out of a function, it
is part of the function's interface specification whether ownership is
transferred with the reference or not.

Most functions that return a reference to an object pass on ownership
with the reference.  In particular, all functions whose function it is
to create a new object, such as \cfunction{PyInt_FromLong()} and
\cfunction{Py_BuildValue()}, pass ownership to the receiver.  Even if
the object is not actually new, you still receive ownership of a new
reference to that object.  For instance, \cfunction{PyInt_FromLong()}
maintains a cache of popular values and can return a reference to a
cached item.

Many functions that extract objects from other objects also transfer
ownership with the reference, for instance
\cfunction{PyObject_GetAttrString()}.  The picture is less clear, here,
however, since a few common routines are exceptions:
\cfunction{PyTuple_GetItem()}, \cfunction{PyList_GetItem()},
\cfunction{PyDict_GetItem()}, and \cfunction{PyDict_GetItemString()}
all return references that you borrow from the tuple, list or
dictionary.

The function \cfunction{PyImport_AddModule()} also returns a borrowed
reference, even though it may actually create the object it returns:
this is possible because an owned reference to the object is stored in
\code{sys.modules}.

When you pass an object reference into another function, in general,
the function borrows the reference from you --- if it needs to store
it, it will use \cfunction{Py_INCREF()} to become an independent
owner.  There are exactly two important exceptions to this rule:
\cfunction{PyTuple_SetItem()} and \cfunction{PyList_SetItem()}.  These
functions take over ownership of the item passed to them --- even if
they fail!  (Note that \cfunction{PyDict_SetItem()} and friends don't
take over ownership --- they are ``normal.'')

When a C function is called from Python, it borrows references to its
arguments from the caller.  The caller owns a reference to the object,
so the borrowed reference's lifetime is guaranteed until the function
returns.  Only when such a borrowed reference must be stored or passed
on, it must be turned into an owned reference by calling
\cfunction{Py_INCREF()}.

The object reference returned from a C function that is called from
Python must be an owned reference --- ownership is tranferred from the
function to its caller.


\subsection{Thin Ice
            \label{thinIce}}

There are a few situations where seemingly harmless use of a borrowed
reference can lead to problems.  These all have to do with implicit
invocations of the interpreter, which can cause the owner of a
reference to dispose of it.

The first and most important case to know about is using
\cfunction{Py_DECREF()} on an unrelated object while borrowing a
reference to a list item.  For instance:

\begin{verbatim}
void
bug(PyObject *list)
{
    PyObject *item = PyList_GetItem(list, 0);

    PyList_SetItem(list, 1, PyInt_FromLong(0L));
    PyObject_Print(item, stdout, 0); /* BUG! */
}
\end{verbatim}

This function first borrows a reference to \code{list[0]}, then
replaces \code{list[1]} with the value \code{0}, and finally prints
the borrowed reference.  Looks harmless, right?  But it's not!

Let's follow the control flow into \cfunction{PyList_SetItem()}.  The list
owns references to all its items, so when item 1 is replaced, it has
to dispose of the original item 1.  Now let's suppose the original
item 1 was an instance of a user-defined class, and let's further
suppose that the class defined a \method{__del__()} method.  If this
class instance has a reference count of 1, disposing of it will call
its \method{__del__()} method.

Since it is written in Python, the \method{__del__()} method can execute
arbitrary Python code.  Could it perhaps do something to invalidate
the reference to \code{item} in \cfunction{bug()}?  You bet!  Assuming
that the list passed into \cfunction{bug()} is accessible to the
\method{__del__()} method, it could execute a statement to the effect of
\samp{del list[0]}, and assuming this was the last reference to that
object, it would free the memory associated with it, thereby
invalidating \code{item}.

The solution, once you know the source of the problem, is easy:
temporarily increment the reference count.  The correct version of the
function reads:

\begin{verbatim}
void
no_bug(PyObject *list)
{
    PyObject *item = PyList_GetItem(list, 0);

    Py_INCREF(item);
    PyList_SetItem(list, 1, PyInt_FromLong(0L));
    PyObject_Print(item, stdout, 0);
    Py_DECREF(item);
}
\end{verbatim}

This is a true story.  An older version of Python contained variants
of this bug and someone spent a considerable amount of time in a C
debugger to figure out why his \method{__del__()} methods would fail...

The second case of problems with a borrowed reference is a variant
involving threads.  Normally, multiple threads in the Python
interpreter can't get in each other's way, because there is a global
lock protecting Python's entire object space.  However, it is possible
to temporarily release this lock using the macro
\csimplemacro{Py_BEGIN_ALLOW_THREADS}, and to re-acquire it using
\csimplemacro{Py_END_ALLOW_THREADS}.  This is common around blocking
I/O calls, to let other threads use the processor while waiting for
the I/O to complete.  Obviously, the following function has the same
problem as the previous one:

\begin{verbatim}
void
bug(PyObject *list)
{
    PyObject *item = PyList_GetItem(list, 0);
    Py_BEGIN_ALLOW_THREADS
    ...some blocking I/O call...
    Py_END_ALLOW_THREADS
    PyObject_Print(item, stdout, 0); /* BUG! */
}
\end{verbatim}


\subsection{NULL Pointers
            \label{nullPointers}}

In general, functions that take object references as arguments do not
expect you to pass them \NULL{} pointers, and will dump core (or
cause later core dumps) if you do so.  Functions that return object
references generally return \NULL{} only to indicate that an
exception occurred.  The reason for not testing for \NULL{}
arguments is that functions often pass the objects they receive on to
other function --- if each function were to test for \NULL,
there would be a lot of redundant tests and the code would run more
slowly.

It is better to test for \NULL{} only at the ``source:'' when a
pointer that may be \NULL{} is received, for example, from
\cfunction{malloc()} or from a function that may raise an exception.

The macros \cfunction{Py_INCREF()} and \cfunction{Py_DECREF()}
do not check for \NULL{} pointers --- however, their variants
\cfunction{Py_XINCREF()} and \cfunction{Py_XDECREF()} do.

The macros for checking for a particular object type
(\code{Py\var{type}_Check()}) don't check for \NULL{} pointers ---
again, there is much code that calls several of these in a row to test
an object against various different expected types, and this would
generate redundant tests.  There are no variants with \NULL{}
checking.

The C function calling mechanism guarantees that the argument list
passed to C functions (\code{args} in the examples) is never
\NULL{} --- in fact it guarantees that it is always a tuple.\footnote{
These guarantees don't hold when you use the ``old'' style
calling convention --- this is still found in much existing code.}

It is a severe error to ever let a \NULL{} pointer ``escape'' to
the Python user.

% Frank Stajano:
% A pedagogically buggy example, along the lines of the previous listing, 
% would be helpful here -- showing in more concrete terms what sort of 
% actions could cause the problem. I can't very well imagine it from the 
% description.


\section{Writing Extensions in \Cpp
         \label{cplusplus}}

It is possible to write extension modules in \Cpp.  Some restrictions
apply.  If the main program (the Python interpreter) is compiled and
linked by the C compiler, global or static objects with constructors
cannot be used.  This is not a problem if the main program is linked
by the \Cpp{} compiler.  Functions that will be called by the
Python interpreter (in particular, module initalization functions)
have to be declared using \code{extern "C"}.
It is unnecessary to enclose the Python header files in
\code{extern "C" \{...\}} --- they use this form already if the symbol
\samp{__cplusplus} is defined (all recent \Cpp{} compilers define this
symbol).


\section{Providing a C API for an Extension Module
         \label{using-cobjects}}
\sectionauthor{Konrad Hinsen}{hinsen@cnrs-orleans.fr}

Many extension modules just provide new functions and types to be
used from Python, but sometimes the code in an extension module can
be useful for other extension modules. For example, an extension
module could implement a type ``collection'' which works like lists
without order. Just like the standard Python list type has a C API
which permits extension modules to create and manipulate lists, this
new collection type should have a set of C functions for direct
manipulation from other extension modules.

At first sight this seems easy: just write the functions (without
declaring them \keyword{static}, of course), provide an appropriate
header file, and document the C API. And in fact this would work if
all extension modules were always linked statically with the Python
interpreter. When modules are used as shared libraries, however, the
symbols defined in one module may not be visible to another module.
The details of visibility depend on the operating system; some systems
use one global namespace for the Python interpreter and all extension
modules (Windows, for example), whereas others require an explicit
list of imported symbols at module link time (AIX is one example), or
offer a choice of different strategies (most Unices). And even if
symbols are globally visible, the module whose functions one wishes to
call might not have been loaded yet!

Portability therefore requires not to make any assumptions about
symbol visibility. This means that all symbols in extension modules
should be declared \keyword{static}, except for the module's
initialization function, in order to avoid name clashes with other
extension modules (as discussed in section~\ref{methodTable}). And it
means that symbols that \emph{should} be accessible from other
extension modules must be exported in a different way.

Python provides a special mechanism to pass C-level information
(pointers) from one extension module to another one: CObjects.
A CObject is a Python data type which stores a pointer (\ctype{void
*}).  CObjects can only be created and accessed via their C API, but
they can be passed around like any other Python object. In particular, 
they can be assigned to a name in an extension module's namespace.
Other extension modules can then import this module, retrieve the
value of this name, and then retrieve the pointer from the CObject.

There are many ways in which CObjects can be used to export the C API
of an extension module. Each name could get its own CObject, or all C
API pointers could be stored in an array whose address is published in
a CObject. And the various tasks of storing and retrieving the pointers
can be distributed in different ways between the module providing the
code and the client modules.

The following example demonstrates an approach that puts most of the
burden on the writer of the exporting module, which is appropriate
for commonly used library modules. It stores all C API pointers
(just one in the example!) in an array of \ctype{void} pointers which
becomes the value of a CObject. The header file corresponding to
the module provides a macro that takes care of importing the module
and retrieving its C API pointers; client modules only have to call
this macro before accessing the C API.

The exporting module is a modification of the \module{spam} module from
section~\ref{simpleExample}. The function \function{spam.system()}
does not call the C library function \cfunction{system()} directly,
but a function \cfunction{PySpam_System()}, which would of course do
something more complicated in reality (such as adding ``spam'' to
every command). This function \cfunction{PySpam_System()} is also
exported to other extension modules.

The function \cfunction{PySpam_System()} is a plain C function,
declared \keyword{static} like everything else:

\begin{verbatim}
static int
PySpam_System(char *command)
{
    return system(command);
}
\end{verbatim}

The function \cfunction{spam_system()} is modified in a trivial way:

\begin{verbatim}
static PyObject *
spam_system(PyObject *self, PyObject *args)
{
    char *command;
    int sts;

    if (!PyArg_ParseTuple(args, "s", &command))
        return NULL;
    sts = PySpam_System(command);
    return Py_BuildValue("i", sts);
}
\end{verbatim}

In the beginning of the module, right after the line

\begin{verbatim}
#include "Python.h"
\end{verbatim}

two more lines must be added:

\begin{verbatim}
#define SPAM_MODULE
#include "spammodule.h"
\end{verbatim}

The \code{\#define} is used to tell the header file that it is being
included in the exporting module, not a client module. Finally,
the module's initialization function must take care of initializing
the C API pointer array:

\begin{verbatim}
PyMODINIT_FUNC
initspam(void)
{
    PyObject *m;
    static void *PySpam_API[PySpam_API_pointers];
    PyObject *c_api_object;

    m = Py_InitModule("spam", SpamMethods);

    /* Initialize the C API pointer array */
    PySpam_API[PySpam_System_NUM] = (void *)PySpam_System;

    /* Create a CObject containing the API pointer array's address */
    c_api_object = PyCObject_FromVoidPtr((void *)PySpam_API, NULL);

    if (c_api_object != NULL)
        PyModule_AddObject(m, "_C_API", c_api_object);
}
\end{verbatim}

Note that \code{PySpam_API} is declared \keyword{static}; otherwise
the pointer array would disappear when \function{initspam()} terminates!

The bulk of the work is in the header file \file{spammodule.h},
which looks like this:

\begin{verbatim}
#ifndef Py_SPAMMODULE_H
#define Py_SPAMMODULE_H
#ifdef __cplusplus
extern "C" {
#endif

/* Header file for spammodule */

/* C API functions */
#define PySpam_System_NUM 0
#define PySpam_System_RETURN int
#define PySpam_System_PROTO (char *command)

/* Total number of C API pointers */
#define PySpam_API_pointers 1


#ifdef SPAM_MODULE
/* This section is used when compiling spammodule.c */

static PySpam_System_RETURN PySpam_System PySpam_System_PROTO;

#else
/* This section is used in modules that use spammodule's API */

static void **PySpam_API;

#define PySpam_System \
 (*(PySpam_System_RETURN (*)PySpam_System_PROTO) PySpam_API[PySpam_System_NUM])

/* Return -1 and set exception on error, 0 on success. */
static int
import_spam(void)
{
    PyObject *module = PyImport_ImportModule("spam");

    if (module != NULL) {
        PyObject *c_api_object = PyObject_GetAttrString(module, "_C_API");
        if (c_api_object == NULL)
            return -1;
        if (PyCObject_Check(c_api_object))
            PySpam_API = (void **)PyCObject_AsVoidPtr(c_api_object);
        Py_DECREF(c_api_object);
    }
    return 0;
}

#endif

#ifdef __cplusplus
}
#endif

#endif /* !defined(Py_SPAMMODULE_H */
\end{verbatim}

All that a client module must do in order to have access to the
function \cfunction{PySpam_System()} is to call the function (or
rather macro) \cfunction{import_spam()} in its initialization
function:

\begin{verbatim}
PyMODINIT_FUNC
initclient(void)
{
    PyObject *m;

    Py_InitModule("client", ClientMethods);
    if (import_spam() < 0)
        return;
    /* additional initialization can happen here */
}
\end{verbatim}

The main disadvantage of this approach is that the file
\file{spammodule.h} is rather complicated. However, the
basic structure is the same for each function that is
exported, so it has to be learned only once.

Finally it should be mentioned that CObjects offer additional
functionality, which is especially useful for memory allocation and
deallocation of the pointer stored in a CObject. The details
are described in the \citetitle[../api/api.html]{Python/C API
Reference Manual} in the section
``\ulink{CObjects}{../api/cObjects.html}'' and in the implementation
of CObjects (files \file{Include/cobject.h} and
\file{Objects/cobject.c} in the Python source code distribution).
