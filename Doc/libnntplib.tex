\section{Standard Module \sectcode{nntplib}}
\stmodindex{nntplib}

\renewcommand{\indexsubitem}{(in module nntplib)}

This module defines the class \code{NNTP} which implements the client
side of the NNTP protocol.  It can be used to implement a news reader
or poster, or automated news processors.  For more information on NNTP
(Network News Transfer Protocol), see Internet RFC 977.

Due to time constraints, the documentation for this module could not
be completed for this release of the Python documentation.  Here are
two small examples of how it can be used.

To list some statistics about a newsgroup and print the subjects of
the last 10 articles:

\begin{verbatim}
>>> s = NNTP('news.cwi.nl')
>>> resp, count, first, last, name = s.group('comp.lang.python')
>>> print 'Group', name, 'has', count, 'articles, range', first, 'to', last
Group comp.lang.python has 59 articles, range 3742 to 3803
>>> resp, subs = s.xhdr('subject', first + '-' + last)
>>> for id, sub in subs[-10:]: print id, sub
... 
3792 Re: Removing elements from a list while iterating...
3793 Re: Who likes Info files?
3794 Emacs and doc strings
3795 a few questions about the Mac implementation
3796 Re: executable python scripts
3797 Re: executable python scripts
3798 Re: a few questions about the Mac implementation 
3799 Re: PROPOSAL: A Generic Python Object Interface for Python C Modules
3802 Re: executable python scripts 
3803 Re: POSIX wait and SIGCHLD
>>> s.quit()
'205 news.cwi.nl closing connection.  Goodbye.'
>>> 
\end{verbatim}

To post an article from a file (this assumes that the article has
valid headers):

\begin{verbatim}
>>> s = NNTP('news.cwi.nl')
>>> f = open('/tmp/article')
>>> s.post(f)
'240 Article posted successfully.'
>>> s.quit()
'205 news.cwi.nl closing connection.  Goodbye.'
>>> 
\end{verbatim}
