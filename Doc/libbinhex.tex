\section{Standard Module \sectcode{binhex}}
\label{module-binhex}
\stmodindex{binhex}

This module encodes and decodes files in binhex4 format, a format
allowing representation of Macintosh files in ASCII. On the macintosh,
both forks of a file and the finder information are encoded (or
decoded), on other platforms only the data fork is handled.

The \code{binhex} module defines the following functions:

\setindexsubitem{(in module binhex)}

\begin{funcdesc}{binhex}{input\, output}
Convert a binary file with filename \var{input} to binhex file
\var{output}. The \var{output} parameter can either be a filename or a
file-like object (any object supporting a \var{write} and \var{close}
method).
\end{funcdesc}

\begin{funcdesc}{hexbin}{input\optional{\, output}}
Decode a binhex file \var{input}. \var{input} may be a filename or a
file-like object supporting \var{read} and \var{close} methods.
The resulting file is written to a file named \var{output}, unless the
argument is empty in which case the output filename is read from the
binhex file.
\end{funcdesc}

\subsection{Notes}
There is an alternative, more powerful interface to the coder and
decoder, see the source for details.

If you code or decode textfiles on non-Macintosh platforms they will
still use the macintosh newline convention (carriage-return as end of
line).

As of this writing, \var{hexbin} appears to not work in all cases.
